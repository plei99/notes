\documentclass[leqno, openany]{memoir}
\setulmarginsandblock{3.5cm}{3.5cm}{*}
\setlrmarginsandblock{3cm}{3.5cm}{*}
\checkandfixthelayout

\usepackage{amsmath}
\usepackage{amssymb}
\usepackage{amsthm}
%\usepackage{MnSymbol}
\usepackage{bm}
\usepackage{accents}
\usepackage{mathtools}
\usepackage{tikz}
\usetikzlibrary{calc}
\usetikzlibrary{automata,positioning}
\usepackage{tikz-cd}
\usepackage{forest}
\usepackage{braket} 
\usepackage{listings}
\usepackage{mdframed}
\usepackage{verbatim}
\usepackage{physics}
\usepackage{stmaryrd}
\usepackage{stackengine} 

%font
\usepackage[sc]{mathpazo}
\usepackage{eulervm}
\usepackage[scaled=0.86]{berasans} 
\usepackage{inconsolata}
\usepackage{microtype}

%CS packages
\usepackage{algorithmicx}
\usepackage{algpseudocode}
\usepackage{algorithm}

% typeset and bib
\usepackage[english]{babel} 
\usepackage[utf8]{inputenc} 
\usepackage[T1]{fontenc} 
\usepackage[backend=biber, style=alphabetic]{biblatex}
\usepackage[bookmarks, colorlinks, breaklinks]{hyperref} 
\hypersetup{linkcolor=black,citecolor=black,filecolor=black,urlcolor=black}

% other formatting packages
\usepackage{float}
\usepackage{booktabs}
\usepackage{enumitem}
\usepackage{csquotes}
\usepackage{titlesec}
\usepackage{titling}
\usepackage{fancyhdr}
\usepackage{lastpage}
\usepackage{parskip}

\usepackage{lipsum}

% delimiters
\DeclarePairedDelimiter{\gen}{\langle}{\rangle}
\DeclarePairedDelimiter{\floor}{\lfloor}{\rfloor}
\DeclarePairedDelimiter{\ceil}{\lceil}{\rceil}


\newtheorem{thm}{Theorem}[section]
\newtheorem{cor}[thm]{Corollary}
\newtheorem{prop}[thm]{Proposition}
\newtheorem{lem}[thm]{Lemma}
\newtheorem{conj}[thm]{Conjecture}
\newtheorem{quest}[thm]{Question}

\theoremstyle{definition}
\newtheorem{defn}[thm]{Definition}
\newtheorem{defns}[thm]{Definitions}
\newtheorem{con}[thm]{Construction}
\newtheorem{exm}[thm]{Example}
\newtheorem{exms}[thm]{Examples}
\newtheorem{notn}[thm]{Notation}
\newtheorem{notns}[thm]{Notations}
\newtheorem{addm}[thm]{Addendum}
\newtheorem{exer}[thm]{Exercise}

\theoremstyle{remark}
\newtheorem{rmk}[thm]{Remark}
\newtheorem{rmks}[thm]{Remarks}
\newtheorem{warn}[thm]{Warning}
\newtheorem{sch}[thm]{Scholium}


% unnumbered theorems
\theoremstyle{plain}
\newtheorem*{thm*}{Theorem}
\newtheorem*{prop*}{Proposition}
\newtheorem*{lem*}{Lemma}
\newtheorem*{cor*}{Corollary}
\newtheorem*{conj*}{Conjecture}

% unnumbered definitions
\theoremstyle{definition}
\newtheorem*{defn*}{Definition}
\newtheorem*{exer*}{Exercise}
\newtheorem*{defns*}{Definitions}
\newtheorem*{con*}{Construction}
\newtheorem*{exm*}{Example}
\newtheorem*{exms*}{Examples}
\newtheorem*{notn*}{Notation}
\newtheorem*{notns*}{Notations}
\newtheorem*{addm*}{Addendum}


\theoremstyle{remark}
\newtheorem*{rmk*}{Remark}

% shortcuts
\newcommand{\Ima}{\mathrm{Im}}
\newcommand{\A}{\mathbb{A}}
\newcommand{\N}{\mathbb{N}}
\newcommand{\R}{\mathbb{R}}
\newcommand{\C}{\mathbb{C}}
\newcommand{\Z}{\mathbb{Z}}
\newcommand{\Q}{\mathbb{Q}}
\renewcommand{\k}{\Bbbk}
\renewcommand{\P}{\mathbb{P}}
\newcommand{\M}{\overline{M}}
\newcommand{\g}{\mathfrak{g}}
\newcommand{\h}{\mathfrak{h}}
\newcommand{\n}{\mathfrak{n}}
\renewcommand{\b}{\mathfrak{b}}
\newcommand{\ep}{\varepsilon}
\newcommand*{\dt}[1]{%
   \accentset{\mbox{\Huge\bfseries .}}{#1}}
\renewcommand{\abstractname}{Official Description}
\newcommand{\mc}[1]{\mathcal{#1}}
\newcommand{\T}{\mathbb{T}}
\newcommand{\mf}[1]{\mathfrak{#1}}
\newcommand{\mr}[1]{\mathrm{#1}}
\newcommand{\on}[1]{\operatorname{#1}}
\newcommand{\ms}[1]{\mathsf{#1}}
\newcommand{\ol}[1]{\overline{#1}}
\newcommand{\ul}[1]{\underline{#1}}
\newcommand{\wt}[1]{\widetilde{#1}}
\newcommand{\wh}[1]{\widehat{#1}}
\newcommand{\pt}{\mathrm{pt}}
\renewcommand{\op}{\mathrm{op}}

\DeclareMathOperator{\Der}{Der}
\DeclareMathOperator{\Hom}{Hom}
\DeclareMathOperator{\End}{End}
\DeclareMathOperator{\ad}{ad}
\DeclareMathOperator{\Aut}{Aut}
\DeclareMathOperator{\Gal}{Gal}
\DeclareMathOperator{\Rad}{Rad}
\DeclareMathOperator{\supp}{supp}
\DeclareMathOperator{\Supp}{Supp}
\DeclareMathOperator{\sgn}{sgn}
\DeclareMathOperator{\rk}{rk}
\DeclareMathOperator{\spec}{Spec}
\DeclareMathOperator{\Spec}{Spec}
\DeclareMathOperator{\Proj}{Proj}
\DeclareMathOperator{\Ext}{Ext}
\DeclareMathOperator{\Tor}{Tor}
\DeclareMathOperator{\Ann}{Ann}
\DeclareMathOperator{\Ass}{Ass}
\DeclareMathOperator{\dpth}{depth}
\DeclareMathOperator{\pdim}{proj.dim}
\DeclareMathOperator{\idim}{inj.dim}
\DeclareMathOperator{\gdim}{gl.dim}
\DeclareMathOperator{\Pic}{Pic}
\DeclareMathOperator{\NS}{NS}
\DeclareMathOperator{\codim}{codim}
\DeclareMathOperator{\Bl}{Bl}

% Section formatting
\titleformat{\section}
    {\Large\sffamily\scshape\bfseries}{\thesection}{1em}{}
\titleformat{\subsection}[runin]
    {\large\sffamily\bfseries}{\thesubsection}{1em}{}
\titleformat{\subsubsection}[runin]{\normalfont\itshape}{\thesubsubsection}{1em}{}

\title{COURSE TITLE}
\author{Lectures by INSTRUCTOR, Notes by NOTETAKER}
\date{SEMESTER}

\newcommand*{\titleSW}
    {\begingroup% Story of Writing
    \raggedleft
    \vspace*{\baselineskip}
    {\Huge\itshape Moduli Spaces and Hyperk\"ahler Manifolds  \\ Fall 2021}\\[\baselineskip]
    {\large\itshape Notes by Patrick Lei}\\[0.2\textheight]
    {\Large Lectures by Giulia Sacc\`a}\par
    \vfill
    {\Large \sffamily Columbia University}
    \vspace*{\baselineskip}
\endgroup}
\pagestyle{simple}

\chapterstyle{ell}


%\renewcommand{\cftchapterpagefont}{}
\renewcommand\cftchapterfont{\sffamily}
\renewcommand\cftsectionfont{\scshape}
\renewcommand*{\cftchapterleader}{}
\renewcommand*{\cftsectionleader}{}
\renewcommand*{\cftsubsectionleader}{}
\renewcommand*{\cftchapterformatpnum}[1]{~\textbullet~#1}
\renewcommand*{\cftsectionformatpnum}[1]{~\textbullet~#1}
\renewcommand*{\cftsubsectionformatpnum}[1]{~\textbullet~#1}
\renewcommand{\cftchapterafterpnum}{\cftparfillskip}
\renewcommand{\cftsectionafterpnum}{\cftparfillskip}
\renewcommand{\cftsubsectionafterpnum}{\cftparfillskip}
\setrmarg{3.55em plus 1fil}
\setsecnumdepth{subsection}
\maxsecnumdepth{subsection}
\settocdepth{subsection}

\begin{document}
    
\begin{titlingpage}
\titleSW
\end{titlingpage}

\thispagestyle{empty}
\section*{Disclaimer}%
\label{sec:disclaimer}

Unless otherwise noted, these notes were taken during lecture using the
\texttt{vimtex} package of the editor \texttt{neovim}.  Any errors are mine and
not the instructor's.  In addition, my notes are picture-free (but will include
commutative diagrams) and are a mix of my mathematical style and that of the
instructor.  If you find any errors, please contact me at
\texttt{plei@math.columbia.edu}.

\subsection*{Acknowledgements}%
\label{sub:acknowledements}

I would like to acknowledge Kevin Chang, Kuan-Wen Chen, Che Shen, and Nicol\'as Vilches for pointing out mistakes in these notes.


\newpage

\tableofcontents

\chapter{Hyperk\"ahler Manifolds}%
\label{cha:introduction}

Some useful references for K3 surfaces are the book by Huybrechts and the Barth-Peters-van de Ven book \textit{Compact Complex Surfaces} and another book. For Hilbert schemes some references are Chapter 7 of \textit{FGA Explained}, Huybrechts-Lehn, some lectures of notes of Lehn, and Nakajima's \textit{Lectures on Hilbert Schemes}. For Hilbert schemes of K3 surfaces and abelian varieties, there is Beauville's \textit{Variet\'es Kahlerienne dont la premiere classe de Chern est nulle}. 

\section{Motivation}%
\label{sec:motivation}

Giulia believes that hyperk\"ahler manifolds are some of the most interesting objects in algebraic geometry because one can actually prove results about high-dimensional hyperk\"ahler varieties, unlike the usual situation in algebraic geometry. Because these objects are of a differential-geometric nature, through the course we will work over $\C$.

Recall that in order to classify curves, for a given curve $C$, we want to consider the positivity of the canonical bundle. In the first case, we know $\omega_{\P^1} = \mc{O}(-2) < 0$, in the second case of an elliptic curve, we have $\omega_C = \mc{O}_C$, and finally for a higher genus curve the canonical sheaf $\omega_C > 0$ is ample.

In higher dimension, let $X$ be a smooth projective variety. Then there exists an integer $\kappa(X)$, the \textit{Kodaira dimension} of $X$ such that
\[ h^0(\omega_X^{\otimes m}) \sim m^{\kappa(X)} \]
for $m \gg 0$ sufficiently divisible. Of course, this is a birational invariant.

There is a classification of surfaces. Each smooth surface is birational to a \textit{minimal} surface. Here, a surface $S$ is minimal if any birational morphism from $S$ to a smooth surface is a birational curve. By Castelnuovo, we know that $S$ is minimal if and only if it does not contain a $(-1)$-curve. Also, any surface dominates a minimal surface.

\section{Hyperk\"ahler manifolds}%
\label{sec:hyperkahler_manifolds}

\begin{exm}
    If $S$ is a surface, then $\kappa(S) = -\infty$ if and only if 
    \[ P_m(S) \coloneqq h^0(\omega_S^{\otimes m}) = 0 \] 
    for all $m \geq 1$. Some examples of these are $\P^2$ and $\P^1 \times \P^1$.
\end{exm}

\begin{exm}
    If $S$ is a surface, then $\kappa(S) = 0$ if and only if $P_m(S) = 0$ generically and there exists $m$ such that $P_m(S) = 1$. In this case, there are two cases. First, $h^0(\omega_S) = 1$, in which case $\omega_S = \mc{O}_S$, and thus either $h^1(\mc{O}_S) = 0$ (in which case we have a \textit{K3 surface}) or $\h^1(\mc{O}_S) = 2$ (in which case we have an \textit{abelian surface}).

    Alternatively, we may have $h^0(\omega_S) = 0$, in which case there exists $m \geq 2$ (where $m \in \qty{2,3,4,6}$) such that $\omega_S^{\otimes m} = \mc{O}_S$. Either $h^1(\mc{O}_S) = 0$, in which case we have an Enriques surface, or $h^1(\mc{O}_S) = 1$, in which case we have a bi-elliptic surface.

    If $T$ is an Enriques surface, then there exists a K3 surface $S$ with a $2:1$ \'etale cover $S \to T$. On the other hand, any bi-elliptic surface has an $m:1$ \'etale cover from an abelian surface.
\end{exm}

\begin{thm}[Beauville-Bogomolov]
    Let $M$ be a compact K\"ahler manifold with $c_1(\omega_M) = 0$. The there exists a finite \'etale cover of $M$ by a product
    \[ T^n \times \prod Y_i \times \prod X_i \to M, \]
    where $T^n$ is a complex torus, the $Y_i$ are \textit{strict Calabi-Yau}, and the $X_i$ are \textit{irreducible holomorphic symplectic} (or hyperk\"ahler).  
\end{thm}

\begin{defn}
    Let $Y$ be a compact K\"ahler manifold. Then $Y$ is \textit{strict Calabi-Yau} if $\pi_1(Y) = 1$ and $H^0(\Omega^P_Y) = \C$ when $p = 0, \dim Y$ and $H^0(\Omega^P_Y)$ vanishes elsewhere.
\end{defn}

\begin{defn}
    A compact K\"ahler manifold $X$ is \textit{irreducible holomorphic symplectic} if $\pi_1(X) = 1$ and $H^0(\Omega^2_X) = \C \sigma_X$, where $\sigma_X$ is an irreducible symplectic form. In particular, $\sigma^n$ is a nonzero top form and thus trivializes the canonical bundle. In addition, $\sigma$ induces an isomorphism of holomorphic vector bundles $\Omega_X \simeq T_X$.
\end{defn}

\section{Some surfaces}%
\label{sec:some_surfaces}

Returning to the simplest case, we will define K3 surfaces.
\begin{defn}
    A smooth projective surface $S$ is a \textit{K3 surface} if $\omega_S = 0$ and $h^1(\mc{O}_S) = 0$. 
\end{defn}
It follows from the definition that K3 surfaces are simply connected, so they are in fact both strict Calabi-Yau and irreducible holomorphic symplectic. Later in the course, we will see that irreducible holomorphic symplectic varieties are the true higher-dimensional analogues of K3 surfaces.

\begin{lem}
    Let $S$ be a K3 surface and $f \colon S \to C$ be a dominant morphism to a smooth projective curve $C$ with connected fibers. Then $C = \P^1$ and the general fiber of $f$ is an elliptic curve.
\end{lem}

\begin{proof}
    The proof is left as an exercise to the reader.
\end{proof}

Any K3 surface $S$ with a dominant map to a curve is called an \textit{elliptic K3}. As a consequence, any surjective map $f \colon S \to B$ where $B$ is not a point and $f$ has connected fibers has either $B = \P^1$ or $B$ is a singular K3. This is generalized by the following remarkable result:

\begin{thm}[Matsushita]
    Let $X^{2n}$ be an irreducible holomorphic symplectic manifold and $f \colon X \to B$ be a proper surjective morphism with connected fibers with $B$ a normal variety. If $B$ is not a point, then either $\dim B = n$ and $f$ is a Lagrangian fibration where the general fiber is an abelian $n$-fold or $\dim B = 2n$ and $B$ is a singular symplectic variety if $f$ is not an isomorphism. In the second case, $f$ is called a symplectic resolution.
\end{thm}

\begin{rmk}
    There is another extremely difficult result of Hwang, which says that if $B$ is smooth, then $B = \P^n$ (if $\dim B = 3$, apparently $B$ is a $\Q$-factorial Fano threefold with klt singularities).
\end{rmk}

Now we will consider some examples. Beginning in the simplest case, consider a general section $f_4 \in \abs{\mc{O}_{\P^3}(4)}$. By the Bertini theorem, the general $S = (f_4 = 0)$ is smooth, and by the adjunction formula, $\omega_S = \mc{O}_S$. Then we consider the exact sequence
\[ 0 \to \mc{O}_{\P^3}(4) \to \mc{O}_{\P^3} \to \mc{O}_S \to 0, \]
and by the long exact sequence of cohomology and the known values of cohomology for projective space, we have $H^1(\mc{O}_S) = 0$. 

\begin{exm}
    A concrete example of this is the Fermat quartic, which has the equation
    \[ x_0^4 + x_1^4 + x_2^4 + x_3^4 = 0. \]
    We will see that this is an elliptic K3. The first step is to see that $S$ contains a line $\ell \subseteq S \subseteq \P^3$, so we choose a primitive $\zeta_8$ and set $x_0 = \zeta_8 x_1$ and $x_2 = \zeta_8 x_3$. Now we project $S$ from $\ell$, and considering planes that contain $\ell$, we obtain a rational map $S \dashrightarrow \P^1$. This extends over $\ell$. Finally, we know that $S \cap \P^2$ is a quartic curve containing a line $\ell$, so in fact the generic fiber of this map is an elliptic curve.
\end{exm}

Similarly, we may consider other complete intersections, such as the $(2,3)$ complete intersection in $\P^4$ (intersection of a quadric and a cubic) and the $(2,2,2)$ complete intersection in $\P^5$. In higher dimensions, any degree $(n+1)$ hypersurface $Y$ in $\P^n$ has $\omega_Y = \mc{O}_Y$. By the Lefschetz hyperplane theorem, this is a strict Calabi-Yau. 

\begin{exm}
    Let $\Gamma \in \abs{\mc{O}_{\P^2}(6)}$ be a general sextic and $S$ be a $2:1$ cover of $\P^2$ branched along $\Gamma$. We will use the \textit{covering trick}, which holds for any variety $X$, line bundle $L$, and $0 \neq s \in H^0(L^{\otimes m})$ for some $m \geq 1$. Then if we set $D = (s = 0)$, there exists a finite flat morphism $f \colon Y \to X$ that is a $\Z/m$-cover away from $D$ and ramified along $D$. In this case, $f^* L$ has a section $t$ such that $(t=0) \simeq D$. Finally if $X$ and $D$ are smooth, so is $Y$, and $\omega_Y = f^* \omega_X((m-1) (t=0))$. 

    In our example, we have $\omega_S = f^* \omega_{\P^2} \otimes \mc{O}_X(y^2=\Gamma) = \mc{O}_S$, so $S$ is a K3 surface.
\end{exm}

\begin{exm}[Kummer K3 surfaces]
    Let $A$ be an abelian surface. It has an involution $-1$ with fixed locus $A[2]$. Thus $A/\pm 1$ has $16$ singular points that look like $\C^2/\pm 1 = \Spec \C[x^2, xy, y^2] = \Spec \C[a,b,c]/(ab=c^2)$ (the $A_1$ singularity). Now the surface $S = \mr{Bl}_{A[2]} A/\pm 1$ is a K3 surface.

    It is easy to see that the smooth locus of $A/\pm 1$ has a holomorphic symplectic form $\sigma_{A/\pm 1}$. Then we can bull back $f^* \sigma_{A/\pm 1}$ to a holomorphic symplectic form on $f^{-1}(U) \subseteq S$, and this form extends to $S$. The reason for this is that $\Bl_{A[2]} A$ still has the involution $-1$, and $S$ is the quotient of $\Bl_{A[2]} A$ by this involution. If we denote this diagram by
    \begin{equation}
    \begin{tikzcd}
        \Bl_{A[2]} A \arrow{r}{q} \ar{d}{g} & S \ar{d}{f} \\
        A \arrow{r}{p} & A/\pm 1
    \end{tikzcd}
    \end{equation}
    and denote $\wt{A} \coloneqq \Bl_{A[2]} A$, then we obtain
    \begin{align*} 
        \omega_{\wt{A}} &= f^* \omega_A \otimes \mc{O}\qty(\sum E_i) = \mc{O}_{\wt{A}}\qty(\sum E_i) \\
                        &= q^* \omega_S \otimes \mc{O}_{\wt{A}}\qty(\sum E_i) 
    \end{align*}
    and therefore $q^* \omega_S = \mc{O}_{\wt{A}}$, so $\omega_S = \mc{O}_S$. The morphism $f$ is called a \textit{symplectic resolution}. 
\end{exm}

Before we proceed, we will discuss crepant and symplectic resolutions. Let $Y$ be a smooth variety, so $\Omega^1_Y$ is locally free. Then $\omega_Y \coloneqq \bigwedge^{\dim Y} \Omega^1_Y$ is called the \textit{canonical bundle}. Then if $f \colon X \to Y$ is a birational morphism of smooth varieties, we have an exact sequence
\[ 0 \to f^* \Omega^1_Y \to \Omega^1_X \to \Omega^1_{X/Y} \to 0. \]
Then we know $\Omega^1_{X/Y}$ is supported on the exceptional locus of $f$. Because $Y$ is smooth, then the exceptional locus is divisorial, and thus 
\[\omega_X = f^* \omega_Y \otimes \mc{O}_X\qty(\sum a_i E_i). \]
Now suppose that $Y$ is just normal with smooth locus $U$. Also suppose that $Y \setminus U$ has codimension at least $2$, so Weil divisors on $U$ and $Y$ are the same. There are two ways to extend $\omega_U$ to $Y$. The first is to denote the inclusion $j \colon U \subseteq Y$ and consider the sheaf $j_* \omega_U$, which is generally not locally free. On the other hand, we can extend the Weil divisor $K_U$ to $Y$, which determines a Weil divisor $K_Y$ on $Y$, called the \textit{canonical class}.

\begin{rmk}
    In general, the Weil divisor $K_Y$ is not Cartier. In fact, $K_Y$ is Cartier if and only if $j_* \omega_U$ is locally free.
\end{rmk}

Now let $f \colon X \to Y$ be a reolution of $Y$. This means $f$ is proper and an isomorphism over $U$. We want a formula relating of the form $K_X = f^* K_Y + \sum a_i E_i$. Unfortunately, we can only pull back Cartier divisors, so we will assume that $K_Y$ is $\Q$-Cartier, which means that there exists $m \geq 1$ such that $mK_Y$ is Cartier. We know that $f^{-1}(U) \simeq U$, so $\eval{K_X}_{f^{-1}(U)} = \eval{f^* K_Y}_{f^{-1}(U)}$. Thus there exist integers $a_i$ such that
\[ m K_X = f^* m K_Y + \sum a_i E_i, \]
where the $E_i$ are the divisorial components of $X \setminus f^{-1}(U)$. Formally dividing by $m$, we have
\[ K_X = f^* K_Y + \sum a_i E_i. \]
Here, the $a_i$ are known as the \textit{discrepancies} and if $a_i = 0$, then the resolution is called \textit{crepant}.

\begin{exm}
    One example of a crepant resolution is $S \to A / \pm 1$.
\end{exm}

\begin{exm}
    Consider $Y = \C^{2N}/\pm 1$. This is the cone over the degree $2$ Veronese embedding of $\P^{2N-1}$. Now we will write $f \colon X = \Bl_0 Y \to Y$, and the exceptional divisor is a $\P^{2N-1}$. We know that $X$ is the total space of $\mc{O}(-2)$, so there is a projection $X \to \P^{2N-1}$. Now we need to compute $a$ in the formula
    \[ K_X = f^* K_Y + aE. \]
    First note that $K_Y = 0$. This is because the standard holomorphic symplectic form on $\C^{2N}$ descends to the smooth locus $U \subseteq Y$, so we have a symplectic form on $X \setminus E$. Now by the adjunction formula, we have
    \[ K_E = \eval{(K_X+E)}_E, \]
    and thus because $E = \P^{2N-1}$, we have
    \[ \mc{O}_E(-2N) = \eval{(a+1)E}_E. \]
    Finally, we see that $\eval{\mc{O}_X(E)}_E = \mc{O}_{\P^{2N-1}}(-2)$, and thus $a+1 = N$, so $a = N-1$. In particular, $f$ is a crepant resolution if and only if $N = 1$. For $N \geq 2$, $f^* \omega_U$ extends to $X$ with a zero of order $N-1$ along $E$. Therefore the form
    \[ f^* \sigma_U \wedge \cdots \wedge f^* \sigma_U \]
    has a zero of order $N-1$ along $E$, so $f^* \sigma_U$ does not extend over $E$.\footnote{In fact, in this case, no crepant resolution exists. A necessary condition for $f$ to be a symplectic resolution is that it is crepant. In dimension $2$, the two notions are the same.}
\end{exm}

\section{Hilbert Schemes of points on surfaces}%
\label{sec:hilbert_schemes_of_points_on_surfaces}

Let $X$ be a smooth quasiprojective surface. Consider the functor
\[ \mr{Hilb}_X^n \colon \ms{Sch}^{\mr{op}} \to \ms{Set} \]
associating a scheme $T$ to isomorphism classes of flat proper morphisms $T \times X \supseteq Z \to T$ satisfying $p_{Z_t}(t) = n$.

\begin{thm}[Grothendieck]
    The functor $\mr{Hilb}_X^n$ is representable by a quasiprojective scheme $X^{[n]}$. If $X$ is projective, so is $X^{[n]}$.
\end{thm}

Later in the course, we will sketch a construction of the Hilbert scheme, but for now we will simply assume that it exists. A fundamental resolt about Hilbert schemes is
\begin{thm}[Fogarty]
    Let $X$ be a smooth quasiprojective surface. Then $X^{[n]}$ is a smooth connected quasiprojective variety of dimension $2n$ and there exists a morphism $h \colon X^{[n]} \to X^{(n)}$, called the \textit{Hilbert-Chow morphism},\footnote{This may be the most studied morphism in algebraic geometry besides $\P^n \to \Spec k$.} which is a resolution of singularities. Here, if $Z$ is a length $n$ subscheme of $Z$, we have
    \[ h(Z) = \sum_{p \in X} \ell(\mc{O}_{Z,p}) \cdot p. \]
\end{thm}

\begin{exm}
    For $n = 2$, we are looking for ideal sheaves $I \subseteq \mc{O}_X$ with quotient $\mc{O}_Z$ of length $2$. At a point $p$, we know $I/ \mf{m}^2 \subseteq \mf{m}/\mf{m}^2$, and thus subschemes of length $2$ supported on $p$ form a $\P \mf{m}/\mf{m}^2 = \P^1$.
\end{exm}

\begin{proof}[Sketch of smoothness]
    We need to compute the Zariski tangent space at a given point, so we have
    \[ T_{[Z]} X^{[n]} = \Hom_{0 \mapsto Z}(\Spec \C[\ep], X^{[n]}). \]
    By definition, these are flat proper families of length $n$ subschemes $\mc{Z} \to \Spec \C[\ep]$ such that $\eval{\mc{Z}}_{\ep = 0} = Z$, and by a computation (for example in FGA Explained) we have
    \[ T_{[Z]} = \Hom_X(I_Z, \mc{O}_Z). \]
    To compute the dimension, we begin by considering the exact sequence
    \[ 0 \to I_Z \subseteq \mc{O}_X \to \mc{O}_Z \to 0 \]
    and applying the functor $\Hom_X(-, \mc{O}_Z)$, we have an exact sequence
    \[ 0 \to \Hom_X(\mc{O}_Z, \mc{O}_Z) \to \Hom_X(\mc{O}_X, \mc{O}_Z) \to \Hom_X(I_Z, \mc{O}_Z) \to \Ext_X^1(\mc{O}_Z, \mc{O}_Z) \to \Ext_X^1(\mc{O}_Z, \mc{O}_Z), \]
    and thus because $\Hom_X(\mc{O}_X, \mc{O}_Z) \to \Hom_X(I_Z, \mc{O}_Z)$ is the zero morphism, and $\Ext_X^1(\mc{O}_Z, \mc{O}_Z) = H^1(\mc{O}_Z) = 0$, we can simply compute the Ext group. Here, we have
    \[ \chi(\mc{O}_Z, \mc{O}_Z) \coloneqq \sum_{i=0}^2 {(-1)}^i \dim \Ext^i (\mc{O}_Z, \mc{O}_Z). \]
    We simply need to show that the Euler characteristic vanishes because
    \[ \Ext^2(\mc{O}_Z, \mc{O}_Z) = {\Hom(\mc{O}_Z, \omega_X \otimes \mc{O}_Z)}^{\vee} \]
    has dimension $n$, as does $\Hom(\mc{O}_Z, \mc{O}_Z)$. To do this, we use Grothendieck-Riemann-Roch, which says that
    \[ \chi(\mc{F}, \mc{G}) = \mr{ch}(\mc{F}^{\vee}) \cdot \mr{ch}(\mc{G}) \sqrt{\mr{td}(X)}, \]
    and here we see that $\chi(\mc{O}_X, \mc{G}) = \chi(\mc{F})$ where $\mc{F} = \wt{O}_X$. Now because $\supp (\mc{F})$ has dimension $0$, then $\mr{ch}(\mc{F}) = [0, \ldots, \pm \ell (\mc{F})]$.
\end{proof}

\begin{exer}
    Prove that $\chi(\mc{O}_Z, \mc{O}_Z) = 0$ using a locally free resolution in the first variable.
\end{exer}

Now we will review some basic theory of Hilbert schemes for quasiprojective varieties. Here, if $X$ is quasiprojective and $p(t) \in \Q[t]$ is some Hilbert polynomial, consider $[Z] \in \mr{Hilb}_X^{p(t)}$.

\begin{prop}
    If $I \subseteq \mc{O}_X$ is the ideal sheaf of $Z$, then 
    \[ T_{[Z]} \mr{Hilb} = \Hom_X(I, \mc{O}_Z) = \Hom_Z(I/I^2, \mc{O}_Z) = H^0(Z, N_{Z/X}). \]
\end{prop}

\begin{proof}[Sketch of proof]
    We know that $T_{[Z]} \mr{Hilb} = \Hom(\Spec k[\ep], \mr{Hilb}, 0 \mapsto [Z])$. This set of morphisms is the same as the set of $\mc{Z} \subseteq X \times \Spec k[\ep]$ flat over $k[\ep]$. And a module $M$ is flat over $k[\ep]$ if and only if $M \otimes (\ep) \simeq \ep \cdot M$. Now we want an ideal sheaf $\wt{I} \subseteq \mc{O}_X[\ep]$ such that
    \begin{equation*}
    \begin{tikzcd}
        & 0 & 0 & 0 \\
        0 \ar{r} & I \ar{r} \ar{u} & \mc{O}_X \ar{u} \ar{r} & \mc{O}_Z \ar{r} \ar{u} & 0 \\
        0 \ar{r} & \wt{I} \ar{u} \ar[hookrightarrow]{r} & \mc{O}_X[\ep] \ar{r} \ar{u} & \mc{O}_{\wt{\mc{Z}}} \ar{r} \ar{u} & 0 \\
        0 \ar{r} & I \ar{r} \ar[hookrightarrow]{u}{\cdot \ep} & \mc{O}_X \ar{u}{\cdot \ep} \ar{r} & \mc{O}_Z \ar{r} \ar{u} & 0 \\
        & & 0 \ar{u} & 0 \ar{u}
    \end{tikzcd}
    \end{equation*}
    But now we can see that $I = \wt{I}/\ep I \subseteq \mc{O}_X \oplus \ep \mc{O}_Z$, and thus giving $\wt{I}$ is the same as giving an element of $\Hom_X(I, \mc{O}_Z)$.
\end{proof}

\begin{exer}
    Let $X$ be a smooth quasiprojective curve. Show that $X^{[n]} = X^{(n)}$ is smooth of dimension $n$.\footnote{Newton actually proved way back in the day that the symmetric powers of $\A^1$ are smooth (and equal to $\A^n$).}
\end{exer}

\begin{thm}
    Let $X$ be a quasiprojective variety. Then there exists a regular proper morphism 
    \[ h_n \colon X^{[n]} \to X^{(n)} \qquad Z \mapsto \sum_{p \in X} \ell(\mc{O}_Z, p) p \]
    which is surjective and birational. By a result of Fogarty, the fibers of $h_n$ are connected, so if $X$ is connected, so is $X^{[n]}$.
\end{thm}


From now on, we will assume that $X$ is projective. Therefore, for all $Z \subseteq X$ with $\ell(Z) = n$, there exists an open affine neighborhood $U \subseteq X$ containing $Z$. Therefore we have $[Z] \in U^{(n)} \subseteq X^{(n)}$. Now if $Z = \sum \alpha_i p_i$ and $U_i \ni p_i$ are open neighborhoods, then $Z \in \prod U_i^{(\alpha_i)}$.

\begin{rmk}
    If $X$ is a smooth surface, then the local structure of $X^{(n)}$ at $np$ is the same as the local structure of ${(\A^n)}^{(n)}$ at $n \cdot \qty{0}$. In particular, when $n = 2$, we have
    \[ (X^{(2)}, 2p) \simeq (Q, 0) \times \Delta, \]
    where $\Delta$ is smooth of dimension $2$ and $Q$ is the quadric cone.
\end{rmk}

Now for any partition $n = \sum \alpha_i$ of $n$ into positive integers with length $k$, write $\ul{\alpha} = (\alpha_i)$. Then define
\[ X_{\alpha}^{(n)} = \qty{ \sum \alpha_i z_i \mid z_i \neq z_j }. \]
These $X_{\alpha}^{(n)}$ give a stratification of $X^{(n)}$ into locally closed subsets, where the open stratum is $X^{(n)}_{(1,1,\ldots,1)}$ and the closed stratum is $X^{(n)}_{(n)}$. It is easy to see that $\dim X_{\ul{\alpha}}^{(n)} = 2 \ell(\alpha)$. Another important stratum is $X_{(2,1,\ldots,1)}^{(n)}$, where exactly two points come together. Now note that 
\[ h_n^{-1}(\sum \alpha_i z_i) = \prod h_{\alpha_i}^{-1} (\alpha_i z_i), \]
where the $h_{\alpha_i}^{-1}$ are the \textit{punctual Hilbert schemes} $\mr{Hilb}^{\alpha_i}(\mc{O}_X, z_i) \simeq \mr{Hilb}^{\alpha_i} ( k[x_1, x_2], 0 )$. For $\alpha = 2$, the punctual Hilbert scheme is simply $\P \mf{m}/\mf{m}^2$.\footnote{Apparently these are useful in representation theory.}

\begin{thm}[Brian\c{c}on]
    The fiber $h_n^{-1}(nz)$ is irreducible of dimension at most $n-1$.
\end{thm}

In particular, this tells us that $X^{[n]}_{(1,\ldots,1)} \to X^{(n)}_{(1,\ldots,1)}$ has fibers of dimension $0$ and is thus an isomorphism.

\begin{prop}
    The exceptional locus of $h_n$ is an irreducible divisor $E$.
\end{prop}

\begin{proof}
    Because $X^{(n)}$ is normal and $\Q$-factorial,\footnote{Every finite quotient of something smooth is $\Q$-factorial.} then any birational $Y \to X^{(n)}$ from a smooth variety $Y$ has divisorial exceptional divisor.

    Now the exceptional locus $E_{(2,1,\ldots,1)} to X^{(n)}_{(2,1,\ldots,1)}$ has fibers $\P^1$, while for a general $\ul{\alpha}$, we have
    \[ \dim E_{\ul{\alpha}} = \dim X_{\ul{\alpha}}^{(n)} + \sum \dim h_{\alpha_i}^{-1} (\alpha_i z_i) \leq n + \ell(\ul{\alpha}). \]
    Because the strata are irreducible and so are the fibers, we obtain irreducibility for the exceptional divisor.
\end{proof}

\begin{prop}
    Let $X$ be a projective variety. Then there exists a birational surjective morphism $h \colon X^{[n]} \to X^{(n)}$.
\end{prop}

\begin{proof}
    We will show that for all $Z \subseteq T \times X$ proper and flat over $T$ with $\ell(Z_t) = n$ for all $t$, there exists a natural morphism $T \to X^{(n)}$ given by
    \[ t \mapsto \sum_{p \in X} \ell(\mc{O}_{Z_t}, p) \cdot p. \]
    Fix $t_0 \in T$. Because $X$ is projective, there exists $U \subseteq X$ affine with $Z_{t_0} \in U = \Spec A$. Then because $p \colon Z \to T$ is proper, there exists $t_0 \in V \subseteq T$, where $V = \Spec B$ is affine, and for all $t \in V$, $Z_t \in U$. In conclusion, we have a family $Z_V \subseteq V \times U$. At the level of rings, we have a diagram
    \begin{equation*}
    \begin{tikzcd}
        C & B \otimes A \ar[twoheadrightarrow]{l}{\varphi} \\
        B \ar{u}.
    \end{tikzcd}
    \end{equation*}
    Now we need a map ${(A^{\otimes n})}^{S_n} \to B$. Because $Z \to V$ is flat, $C$ is a rank $n$ projective $B$-module. Clearly we have a map $A \to \End_B(C)$ given by $A \mapsto \varphi(1 \otimes a)$, and thus $A^{\otimes n}$ acts on $C^{\otimes n}$. Then we obtain an action of ${(A^{\otimes n})}^{S_n}$ on $\bigwedge^n C$, which is just a map
    \[ {(A^{\otimes n})}^{S_n} \to \End_B\qty({\bigwedge}^n C) \simeq B, \]
    which is the map we want.
\end{proof}

As an example, consider $B = k$. Then $C = \prod C_i$ is Artinian, hence a product of Artinian local rings $C_i$ of length $\alpha_i$. Then the map we defined at the end of the proof factors through $\prod \mr{Sym}^{\alpha_i}(C_i)$.

\begin{thm}[Beauville-Fujiki]
    Let $X$ be a smooth surface. Then the Hilbert-Chow morphism is a crepant resolution and if $X$ has a holomorphic symplectic form, so does $X^{[n]}$.
\end{thm}

\begin{exer}
    If $X$ is a smooth surface, prove that
    \[ X^{[n]} = \Bl_{\Delta} X^{(2)} = (\Bl_{\Delta} X \times X) / S_2. \]
\end{exer}

\begin{proof}
    Consider $X_*^{(n)} = X^{(n)}_{(1,\ldots,1)} \cup X^{(n)}_{(2,1,\ldots,1)}$ and define $X^n_*, X^{[n]}_*$ similarly. In $X^n$ consider $\Delta = \bigcup \Delta_{ij}$, where the $i$-th and $j$-th points coincide. Now consider the diagram
    \begin{equation*}
    \begin{tikzcd}
        \Bl_{\Delta} X_*^n \ar{r}{\eta} \ar{d}{\rho} & X_*^n \ar{d} \\
        X_*^{[n]} \ar{r}{h} & X_*^{(n)},
    \end{tikzcd}
    \end{equation*}
    which quite clearly commutes by the same argument showing that $X^{[n]}$ is the blowup of $X^{(n)}$ along the diagonal. Then the exceptional divisor $\bigcup E_{ij}$ is fixed by $S_2$, so it maps to $E_* \subset X_*^{[n]}$. Now it suffices to prove that $X_*^{[n]} \to X_*^{(n)}$ is crepant because the complement has codimension $2$. To see this, the quotients by $S_n$ have simple ramification, and thus we have
    \begin{align*} 
        K_{\Bl_{\Delta} X_*^n} &= \rho^*(K_{X_*^{[n]}}) + \sum E_{ij} \\
        &= \rho^* h^* K_{X_*^{(n)}} + (a+1) \sum E_{ij} \\
        &= \eta^* K_{X_*^n} + \sum E_{ij},
    \end{align*}
    so $a = 0$ because $\pi \colon X_*^n \to X_*^{(n)}$ is \'etale away from codimension $2$ and thus $K_{X^n_*} = \pi^* K_{X_*^{(n)}}$. Therefore the Hilbert-Chow morphism is a crepant resolution.

    Now suppose that $X$ has a holomorphic symplectic form $\omega_X \in H^0(\Omega^2_X)$. By codimension reasons, it is enough to produce a holomorphic symplectic form on $X_*^{[n]}$. Clearly we have a symplectic form $\omega \coloneqq \sum_i p_i^* (\omega_X)$ on $X_*^n$, which is clearly $S_n$-invariant. Therefore, we obtain a symplectic form $\sigma_{X^{(n)}_{(1,\ldots,1)}}$ on $X^{(n)}_{(1,\ldots,1)}$ and a symplectic form $\eta^* \omega$ on $\Bl_{\Delta} X_*^n$, which is degenerate along $\bigcup E_{ij}$ and $S_n$-invariant. This induces a holomorphic $2$-form $\sigma_{X_*^{[n]}}$ on $X_*^{[n]}$ (as in there exists such a $\sigma$ such that $\eta^* \omega = \rho^* \sigma$). We know that $\sigma_{X_*^{[n]}}$ is generically nondegenerate.

    We now show that $\sigma \coloneqq \sigma_{X_*^{[n]}}$ is symplectic. We know $\sigma^n$ is a section of $\omega_{X_*^{[n]}}$, so the degeneracy locus of $\sigma$ is the zero locus of $\sigma^n$. However, we know $K_{X_*^{(n)}} = 0$ by the existence of $\omega_X$, and because $h$ is crepant, we see that $K_{X_*^[n]} = 0$, and thus $\sigma^n$ must be nonzero everywhere.
\end{proof}

We will now discuss some invariants of $X^{[n]}$.

\begin{prop}
    There is an isomorphicm of Hodge structures
    \[ H^2(X^{[n]}, \Q) = h^* H^2(X^{(n)}) \oplus \Q E.\footnote{Note that $X^{(n)}$ is a finite quotient of something smooth and thus has a pure Hodge structure.} \]
\end{prop}

Now we know that $H^2(X^{(n)}) = {H^2(X^n)}^{S_n}$. By the K\"unneth formula, we have
\[ H^2(X^n) = \bigoplus_{i=1}^n H^2(X) \otimes H^0(X) \oplus \bigoplus_{i,j} H^1(X) \otimes H^1(X), \]
and therefore
\[ H^2(X^{(n)}) = H^2(X) \oplus {\bigwedge}^2 H^1(X). \]
Now
\begin{align*} 
    H^2(X^{[n]}) &= H^2(X_*^{[n]}) \\ 
    &= {H^2(\Bl_{\Delta} X_*^n)}^{S_n}] \\
    &= {(\Im \eta^*)}^{S_n} \oplus {\qty(\bigoplus \Q E_{ij})}^{S_n} \\
    &= \eta^* {(H^2(X_*^n))}^{S_n} \oplus \Q E \\
    &= H^2(X) \oplus {\bigwedge}^2 H^1(X) \oplus \Q E.
\end{align*}
Over $\Z$, we can check in local coordinates that there exists a class $\delta \in H^2(X^[n], \Z)$ such that $2 \delta = E$.\footnote{We had a lengthy discussion checking the computations above, and the moral is that algebraic geometers are bad at basic algebra. Also, to avoid sign problems, work in characteristic $2$.}

\begin{cor}
    If $X$ is a K3 surface, then there is an isomorphism
    \[ H^2(X^{[n]}) = H^2(X) \oplus \Q E \]
    as Hodge structures, and in particular, 
    \[ H^0(\Omega^2_{X^[n]}) = H^{2,0}(X^[n]) \simeq H^2(X) = \C. \]
\end{cor}

We will now sketch the computation of the fundamental group of $X^{[n]}$. One fact is that
\[ h_* \colon \pi_1(X^{[n]}) \to \pi_1(X^{(n)}) = \pi_1(X) / [\pi_1(X), \pi_1(X)] = H_1(X, \Z) \]
is an isomorphism. In particular, if $X$ is a K3 surface, then $\pi_1(X^{[n]}) = 0$, so $X^[n]$ is irreducible holomorphic symplectic.

If $A$ is an abelian surface, we know $A$ has a holomorphic symplectic form $\sigma_A$, and thus $A^{[n]}$ has a holomorphic symplectic form $\sigma_{A^[n]}$, so $\omega_{A^{[n]}} = \mc{O}_{A^[n]}$. However, we know that $H^1(A^[n]) = H^1(A) = \Z^4$, so it is not simply connected. But then we know that 
\[ H^2(A^{[n]}) = H^2(A) \oplus {\bigwedge}^2 H^1(A) = H^2(A) \oplus H^2(A), \] 
so $A^[n]$ has larger $H^{2,0}$. By Beauville-Bogomolov, we know that $A^{[n+1]}$ has an \'etale cover by a product of complex tori, irreducible holomorphic symplectics, and strict Calabi-Yaus. There exists a natural morphism
\[ \mr{alb} \colon A^[n+1] \xrightarrow{h} A^{(n+1)} \to A \]
and an action of $A$ on $A^{[n+1]}$ given by translation. Of course, this is not equivariant because $\sum (z_i + a) \mapsto \sum z_i + (n+1)a$, and by generic smoothness all fibers are isomorphic and smooth. Now if we consider the diagram
\begin{equation*}
\begin{tikzcd}
    A \times_A A^[n+1] \ar{r} \ar{d} & A^{[n+1]} \ar{d}{\mr{alb}} \\
    A \ar{r}{n+1} & A,
\end{tikzcd}
\end{equation*}
we see that $A \times_A A^{[n+1]} = K_n(A) \times A$. Later, we will show that $K_n(A)$ is irreducible holomorphic symplectic.

\begin{exm}
    If $n = 1$, then $K_1(A)$ is the Kummer K3 surface associated to $A$.
\end{exm}

\begin{prop}\leavevmode
    \begin{enumerate}
        \item $\omega_{K^n(A)} = \mc{O}_{K^n(A)}$;
        \item The restriction of teh holomorphic symplectic form $\eval{\sigma_{A^{[n+1]}}}_{K^n(A)}$ is a symplectic form.
        \item $H^2(K^n(A)) = H^2(A) \oplus \Q F$ and $\pi_1(K^n(A)) = 1$, where $F$ is one of the fibers of $E \to A$, where $E \subset A^[n+1]$ is the exceptional divisor of $h$.
    \end{enumerate}
\end{prop}

We will prove a result that 
\begin{prop}
    There exists a (non-effective) line bundle $\mc{L}$ on $X^{[n]}$ such that $\mc{L}^{\otimes 2} = \mc{O}_{X^{[n]}}(E)$.
\end{prop}

\begin{proof}
    Consider $\Bl_{\Delta} X_*^n / A_{n}$. This has simple ramification over $E_* \subseteq X_*^{[n]}$, and thus $f_* \mc{O}_Z = \mc{O}_{X_*^{[n]}} \oplus \mc{L}$. This is the desired line bundle.
\end{proof}

\begin{cor}
    If $X$ is a K3 surface, then $\Pic(X^{[n]}) = \Pic(X) + \Z \delta$, where $\delta = c_1(\mc{L})$.
\end{cor}

\section{Generalized Kummers}%
\label{sec:generalized_kummers}

Recall the construction of the varieties $K^n(A)$ for an abelian surface $A$. Recall the diagram
\begin{equation*}
\begin{tikzcd}
    K^n(A) \ar[hookrightarrow]{r} \ar{dd} & A^{[n+1]} \ar{dd}{a} \ar{dr} & A^{n+1} \ar{d} \\
    & & A^{(n+1)} \ar{dl}{\ep} \\
    0_A \ar[hookrightarrow]{r} & A.
\end{tikzcd}
\end{equation*}
Also recall that $\pi_1(A^{[n+1]}) = \pi_1(A)$. Using the long exact sequence of homotopy groups and the fact that $A$ is a $K(\Z^4,1)$, we see that $\pi_1(K^n(A)) = 0$.

\begin{prop}
    $K^n(A)$ is an irreducible holomorphic symplectic manifold. In particular,
    \begin{enumerate}
        \item If $\sigma_{A^{[n+1]}}$ is the holomorphic symplectic form on $A^[n+1]$, then its restriction to $K^n(A)$ is a holomorphic symplectic form.
        \item $H^2(K^n(A)) = H^2(A) \oplus Q F$, where $F = E \cap K^n(A)$.
    \end{enumerate}
\end{prop}

\begin{proof}
    Consider the Leray filtration on $H^2(A^{[n+1]})$ induced by the map $a$. Here, we have
    \[ H^2(A) = H^2(a_* \Q) \subseteq H^2(A^{[n+1]}) \to H^0(A, R^2 a_* \Q) = {H^2(K^n(A))}^{\mr{inv}}, \]
    where invariants are taken with respect to the monodromy group of $a$, which is $A[n+1]$ because base change by $A \xrightarrow{n+1} A$ trivializes $a$. We will show that the last inclusion is an equality. Also, note that
    \[ H^2(A^{[n+1]}) = {\bigwedge}^2 H^1(A) \oplus H^2(A) \oplus \Q E \]
    and that if $\alpha, \beta \in H^1(A)$, then
    \[ a^* (\alpha \wedge \beta) = a^* \alpha \wedge a^* \beta = \sum p_i^* \alpha \wedge \sum p_i^* \beta. \]
    Next, write $K_*^n(A)$ analogously to $A_*^{[n]}$ and let $N = \ker(A^{n+1} \to A)$. Then we have a diagram
    \begin{equation*}
    \begin{tikzcd}
        & \Bl N_* \ar{dr} \ar{dl} \\
        K_*^n(A) \ar{dr} & & N_* \ar{dl} \\
        & K_*^{(n)}(A).
    \end{tikzcd}
    \end{equation*}
    Note that $N$ has an action of $S_{n+1}$ and an action of $A[n+1]$ given by adding $\ep$ to all elements that preserves $N$ and $\Delta$. Then we know that 
    \[ H^2(N_*) = H^2(N) = H^2(A^n) \]
    has an action of $A[n+1]$ has an action by translation, which is trivial in cohomology. Finally, we conclude that
    \[ H^2(K^n(A)) = H^2(K^n_*(A)) = {H^2(\Bl N_*)}^{S_{n+1}} \]
    and obtain the desired result.
\end{proof}

Now we have two examples of irreducible holomorphic symplectic manifolds. The first is $\mr{K3}^{[n]}$ with $b_2 = b_2(\mr{K3}) + 1 = 23$ and the second is $K^n(A)$ with $b_2 = b_2(A) + 1 = 7$.

\begin{prop}
    Let $f \colon \mc{X} \to B$ be a smooth proper morphism of complex manifolds such that for some $0 \in B$, $\mc{X}_0(B)$ is a K\"ahler irreducible holomorphic symplectic manifold. Then there exists an analytic neighborhood $0 \in V \subseteq B$ such that for all $t \in U$, $\mc{X}_t$ is K\"ahler and holomorphic symplectic.
\end{prop}

\begin{proof}
    By a result of Kodaira, being K\"ahler is an open condition, so there exists an open $U \subseteq B$ such that for all $t \in U$, $\mc{X}_t$ is K\"ahler. Therefore, for all $t \in U$, the map $t \mapsto h^p(\mc{X}_t, \Omega_{\mc{X}_t}^q)$ is constant by Ehresman's theorem that this family is topologically trivial and upper semicontinuity.

    This implies that up to further restricting $U$, $f_* \Omega^2_{\mc{X}/B}|_U$ is free. This implies that $\sigma_0 \in H^0(\Omega^2_{\mc{X}_0})$ extends locally to a section $\wt{\sigma} \in H^0(\mc{X}_U, \Omega^2_{\mc{X}_U/U})$. To check that this is symplectic, we know that $\wt{\sigma}^n \in H^0(K_{\mc{X}/U})$ has closed zero locus which does not intersect the zero fiber, and so we obtain an open set where this form is nondegenerate.
\end{proof}

\begin{prop}
    Let $f \colon \mc{X} \to B$ be a smooth proper family of K\"ahler manifolds. Then if $\mc{X}_0$ is irreducible holomorphic symplectic, so is $\mc{X}_t$ for all $t \in B$.
\end{prop}

\begin{proof}[Sketch of proof]
    First, note that the relative canonical bundle $K_{\mc{X}/B} \cong f^* \mc{L}$, where $\mc{L}$ is a line bundle on $B$. By the same proof as before, there exists $Z \subseteq B$ such that for all $t \in B \setminus Z$, $\mc{X}_t$ is irreducible holomorphic symplectic.

    Now suppose $t_0 \in Z$. Then $K_{\mc{X}_{t_0}}$ is trivial and $\mc{X}_{t_0}$ is simply connected, so $\mc{X}_{t_0}$ is a product of irreducible holomorphic symplectic varieties and strict Calabi-Yau manifolds.

Now we will state without proof the fact that if $X$ is a complex manifold with $K_X = \mc{O}_X$, then $\mr{Def}(X)$ is smooth (as a germ of complex manifold). This is a nontrivial result of Bogomolov-Tian-Todorov. Note that if $X_{t_0} = \prod X_i \times \prod Y_i$, then
    \[ \mr{Def}(X_{t_0}) = \prod \mr{Def}(X_i) \times \prod \mr{Def}(Y_i) \]
    because all $X_i, Y_i$ satisfy $h^{1,0} = 0$. Thus the splitting situation is impossible.
\end{proof}

It is known that if $S$ is a K3 surface, then $\mr{Def}(S)$ has dimension $20$. Also, note that projective K3 surfaces are a $19$-dimensional locus. Here, note that $\mr{Def}(X) = h^1(T_X) = h^1(\Omega^1_X) = h^{1,1}$.

In the next case, if $X = S^{[n]}$, then $\mr{Def}(X)$ has dimension $21$, and there is a $20$-dimensional locus of genuine Hilbert schemes of K3 surfaces. There are also higher-codimension loci parameterizing the spaces $M_v(S, h)$. Note that in both of these situations, the very general object is K\"ahler but not projective.

Now we will discuss some examples of Lagrangian fibrations.

\begin{exm}
    Let $f \colon S \to \P^1$ be an elliptic K3 surface. Then we have a morphism
    \[ S^{[n]} \xrightarrow{f} S^{(n)} \xrightarrow{f^{(n)}} {(\P^1)}^{(n)} = \P^n. \]
    This is clearly a Lagrangian fibration.
\end{exm}

\begin{exm}
    Let $A = E \times F$ be the product of two elliptic curves and let $\varphi \colon A \to F$ be the second projection. Then we have a diagram
    \begin{equation*}
    \begin{tikzcd}
        K^2(A) \ar[hookrightarrow]{r} \ar{d} & A^{[3]} \ar{d} \\
        K^{(2)}(A) \ar[hookrightarrow]{r} & A^{(3)} \ar{d} \ar{r}{\varphi^{(3)}} & F^{(3)} \ar{d}{\ep} & \ep^{-1}(0) \ar{d} \ar[hookrightarrow]{l} \\
        & A & F & 0_F \ar[hookrightarrow]{l}.
    \end{tikzcd}
    \end{equation*}
    Here, we see that $\ep^{-1}(0) = \check{\P}^2$, and so in general there is a Lagrangian fibration $K^n(A) \to \P^n$.
\end{exm}

\section{Some operations}%
\label{sec:some_operations}

Now we will consider some birational transformations. 

\begin{exm}[Atiyah flop]
    Let $f \colon \mc{S} \to \Delta$ be a family of quartic surfaces in $\P^3$. Suppose that $\mc{S}_t$ is smooth and $\mc{S}_0$ has one simple node $p \in \mc{S}_0$. This simple node is given locally by $x^2 + y^2 + z^2 = t$.

    Note that $\Bl_p \mc{S}_0 \eqqcolon \wt{\mc{S}}_0$ is a smooth K3 surface. We would like to modify the family such that we get smooth fibers for all $t \in \Delta$. Now if we take a base change of $\Delta$ by $t \mapsto t^2$, locally at $p \in \ol{\mc{S}}$ we have the equation $x^2 + y^2 + z^2 = t^2$ is a singular point of $\ol{\mc{S}}$. But then $\mc{X} \coloneqq \Bl_p \ol{\mc{S}}$ is smooth, and $\mc{X}_0 = \wt{\mc{S}}_0 \cup Q$, where $Q = \P^1 \times \P^1$.

    Unfortunately, the discrepancy of $\nu \colon \mc{X} \to \ol{\mc{S}}$ is $1$, so $K_{\mc{X}} = \nu^* K_{\ol{\mc{S}}} + Q$, and so by adjunction we see that
    \[ \omega_Q = (K_{\mc{X}} + Q)|_Q = \mc{O}(2Q)|_Q, \]
    and thus $\mc{O}_X(Q) = \mc{O}(-1,-1)$. This tells us that we can contract $Q$ along both of the factors and produce $\mc{S}^+, \mc{S}^-$ with maps to $\ol{\mc{S}}$. Then there is a birational map $\varphi \colon \mc{S}^+ \dashrightarrow \mc{S}^-$ which is an isomorphism away from the central fiber.

    We conclude that $\mc{S}_0^{\pm} = \wt{\mc{S}}_0$ and that $\varphi$ is an isomorphism outside of the copies of $\P^1$ that we contracted $Q$ onto but does not extend over those copies of $\P^1$. Also, note that $\mc{X} = \Gamma_{\varphi}$ and that $\mc{X}_t = \Gamma_{\varphi_t}$ for all $t \neq 0$, and $\mc{X}_0 = \wt{\mc{S}}_0 \cup \P^1 \times \P^1$.

    The next observation is that $H^2(\mc{S}^{\pm}_{t_0}) \simeq H^2(\wt{\mc{S}}_0)$, but passing between the two identifications is actually reflection across the $(-2)$-curve produced from the Atiyah flop.
\end{exm}

Here, we have used the following result of Nakano and Fujiki: Let $\wt{M}$ be a complex manifold and $E \subseteq \wt{M}$ be a smooth divisor that is a $\P^n$-bundle over some $Z$.. Then there exists a complex manifold $M \supset Z$ and $\pi \colon \wt{M} \to M$ such that $\wt{M} = \Bl_Z M$ if and only if $\mc{O}_{\wt{M}}(E)|_E = \mc{O}_X(-1)$.

Another fact that we used to show that $\mc{S}_0^+$ and $\mc{S}_0^-$ are isomorphic is that two birational K3 surfaces are isomorphic.

Now let $X \supseteq \P^n$ be a holomorphic symplectic manifold of dimension $2n$. For example, some K3 surfaces contain $(-2)$-classes, which are isomorphic to $\P^1$.

\begin{lem}
    Any such $\P^n \subseteq X$ is a Lagrangian submanifold of $X$. Moreover, if $Z \subseteq X$ is any Lagrangian, $N_{Z/X} \cong \Omega^1_{Z}$.
\end{lem}

\begin{proof}
    The first part is clear because $H^0(\Omega^2_{\P^n}) = 0$. Next, consider the exact sequencec
    \begin{equation*}
    \begin{tikzcd}
        0 \ar{r} & \mc{J}/\mc{J}^2 \ar{r} & \Omega^1_{X|Z} \ar{r} & \Omega^1_{Z} \ar{r} & 0 \\
        0 \ar{r} & T_{Z} \ar{r} \ar[hookrightarrow]{u} & T_{X|Z} \ar{u}{\sim} \ar{r} & N_{X/Z} \ar{r} \ar[twoheadrightarrow]{u} & 0.
    \end{tikzcd}
    \end{equation*}
    Note that the rightmost vertical morphism is generically injective with torsion kernel, but because $N_{X|\P^n}$ is torsion free, we have an isomorphism.
\end{proof}

Now consider $\Bl_{\P^n} X$ and let $E$ be the exceptional divisor. Denote $\P^n = \P V$ for some vector space $V$.

\begin{lem}
    We have an isomorphism $E \simeq I \subseteq \P V \times \P V^{\vee}$, where $I$ is the incidence subscheme. Moreover, we have $\mc{O}_{\wt{X}}(E) |_E \cong \mc{O}_E(-1, -1)$.
\end{lem}

\begin{proof}
    We know that $E = \P N_{\P^n/X} \simeq \P \Omega^1_{\P^n}$. Now if we consider the Euler sequence
    \[ 0 \to \Omega^1_{\P^n} \to V^{\vee} \otimes \mc{O}_{\P^n}(-1) \xrightarrow{ev} \mc{O}_{\P^n} \to 0, \]
    we obtain an embedding
    \[ \P \Omega^1_{\P^n} \subseteq \P V^{\vee} \times \P V \]
    as the locus $\qty{(s,x) \mid s(x) = 0}$. Next, we use adjunction in $\wt{X}$ and in $\P V \times \P V^{\vee}$ to see that
    \[ \mc{O}_X(-n, -n) = \omega_E = \omega_{\wt{X}}(E)|_E = \mc{O}_{\wt{X}}(nE)|_E. \]
\end{proof}

Now by the Nakano-Fujiki criterion, there exists $\wt{X}' \supseteq \P V$ and $q' \colon \wt{X} \to X'$ such that we have the following diagram:
\begin{equation*}
\begin{tikzcd}
    & \wt{X} \ar{dl} \ar{dr}{q'} \\
    X \ar[dashrightarrow]{rr}{\varphi} & & X'
\end{tikzcd}
\end{equation*}
such that $q'$ takes $E$ to $\P V^{\vee}$.

\begin{defn}
    Such an $X'$ is called the \textit{Mukai flop} of $X$ at $\P^n$.
\end{defn}

\begin{rmk}
    We can perform the Mukai flop whenever we have $Z \subseteq X$ such that there exists some $\P^r$-bundle structure $Z \to B$ and $Z$ has codimension $r$ in $X$. We also require that $N_{Z/X} \simeq \Omega^1_{Z/B}$.
\end{rmk}

\begin{rmk}
    We have a diagram
    \begin{equation*}
    \begin{tikzcd}
        & & \wt{X} \ar{dr} \ar{dl} \\
        \P^n \ar{dr} \ar[hookrightarrow]{r} & X \ar{dr}{\pi} \ar[dashrightarrow]{rr} & & X' \ar{dl} \\
        & p \ar[hookrightarrow]{r} & X_0.
    \end{tikzcd}
    \end{equation*}
\end{rmk}

\begin{rmk}
    If $X'$ and $X$ are isomorphic in codimension $2$, they have isomorphic $H^2$ and $X'$ is holomorphic symplectic.
\end{rmk}

The local structure of $(X_0, p)$ is isomorphic to that of the cone $C^{\bullet}(I)$. In particular, $X_0$ is not $\Q$-factorial because the exceptional locus of $\pi$ is $\P^n$, which is not a divisor. In addition, $\pi$ is a crepant (symplectic resolution).

\begin{prop}
    A birational map $f \colon X \dashrightarrow X'$ of compact complex manifolds (or projective varieties) with trivial canonical bundles is an isomorphism in codimension $2$. In particular, $\pi_1(X) = \pi_1(X')$ and $H^2(X, \Z) = H^2(X', \Z)$.
\end{prop}

\begin{proof}
    Let $\Gamma$ be the graph of $f$ and consider the diagram
    \begin{equation*}
    \begin{tikzcd}
        & \Gamma \ar{dl}{p} \ar{dr}{p'} \\
        X \ar[dashrightarrow]{rr}{f} & & X.
    \end{tikzcd}
    \end{equation*}
    Then if $E, F$ are the exceptional divisors of $p,p'$, we have $K_{\Gamma} = E = F$ up to linear equivalence. However, we know that $H^0(mF) = H^0(mE) = H^0(mK_{\Gamma}) = 1$, but these $h^0(\omega_X^{\otimes m})$ are birational invariants, so $E, F$ do not move in their equivalence class. In particular, we have an isomorphism $X \setminus p(E) \simeq X' \setminus p'(F)$.
\end{proof}

\begin{cor}
    Suppose that $f \colon X \twoheadrightarrow S'$ is a birational map of K3 surfaces. Then $f$ is an isomorphism.
\end{cor}

\begin{proof}
    Consider the graph $\Gamma$ and diagram
    \begin{equation*}
    \begin{tikzcd}
        & \Gamma \ar{dl}{p} \ar{dr}{q} \\
        X \ar[dashrightarrow]{rr}{f} & & S'.
    \end{tikzcd}
    \end{equation*}
    We know that $f$ is an isomorphism away from finitely many points. We know that $f$ is not defined at $x$ if and only if $p^{-1}(x)$ is a curve. But then there exists a curve $C' \subseteq S'$ contracted by $f^{-1}$, which is impossible.
\end{proof}

\begin{exm}[Beauville]
    This example comes from the paper \textit{Some remarks on K\"ahler manifolds with $c_1 = 0$} by Beauville.\footnote{This paper is written in English, but Giulia suggests that we read some math papers in French.} Let $S \subseteq \P^3$ be a quartic K3 surface. Choose a length $2$ point $z \in S^{[2]}$, which has linear span a line. But then $\ell \cap S = z + w$, and so we define a rational map $\varphi \colon S^{[2]} \dashrightarrow S^{[2]}$ given by $z + w$.
\end{exm}

\begin{prop}\leavevmode
    \begin{enumerate}
        \item $\varphi$ is regular at $[Z] \in S^{[2]}$ if and only if $\ell = \ev{Z} \not\subseteq S$.
        \item If $S \supseteq \ell_1, \ldots, \ell_k$ where the $\ell_i$ are disjoint lines, then $\varphi$ is the Mukai flop at $\ell_1^{[2]}, \ldots, \ell_k^{[2]}$.
    \end{enumerate}
\end{prop}

\begin{proof}\leavevmode
    We have a commutative diagram
    \begin{equation*}
    \begin{tikzcd}
        & \Gamma \ar{dl}{q_1} \ar{dr}{q_2} \\
        S^{[2]} \ar[dashrightarrow]{rr} \ar{dr}{p} & & S^{[2]} \ar{dl} \\
        & G = Gr(2,4),
    \end{tikzcd}
    \end{equation*}
    and clearly $p$ is finite over $[\ell] \in G$ if and only if $\ell \not\subseteq S$. In particular, if $\ell \subseteq S$, we have $p^{-1}([\ell]) = \ell^{[2]}$. Now consider the graph $\Gamma$ and note that $\Gamma \subseteq S^{[2]} \times_G S^{[2]} \subseteq S^{[2]} \times S^{[2]}$. Because $S^{[2]}$ is smooth, we know $\varphi$ is regular at $[Z]$ if and only if $q_1^{-1}(Z)$ is finite.

    But now $q_1^{-1}(Z) \subseteq S^{[2]} \times S^{[2]}$ is contained in $[Z] \times p^{-1}(\ell)$. Thus, if $p^{-1}(\ell)$ is finite, so is $q_1^{-1}(Z)$. For dimension reasons, if $\ell \subseteq S$ is a line, then $\ell^{[2]} \times \ell^{[2]}$ is an irreducible component of $S^{[2]} \times_G S^{[2]}$. But then $q_1^{-1}(\ell^{[2]}) = \Gamma \cap \ell^{[2]} \times \ell^{[2]}$, and then $S^{[2]} \times_G S^{[2]} \subseteq S^{[2]} \times S^{[2]}$ is a local complete intersection. But then irreducible components intersect in the correct dimension, so we are done.

    It remains to show that $f$ is the Mukai flop. We may assume that there is a unique line $\ell \subseteq S$. The key technical lemma is that $\varphi$ extends to $\Bl_{\ell^{[2]}} S^{[2]}$, which means we have a map
    \begin{equation*}
    \begin{tikzcd}
        \Bl_{\ell^{[2]}} S^{[2]} \ar{r}{\wt{\varphi}} \ar{d}{\pi} & \Bl_{\ell^{[2]}} S^{[2]} \ar{d}{\pi} \\
        S^{[2]} \ar[dashrightarrow]{r}{\varphi} & S^{[2]}.
    \end{tikzcd}
    \end{equation*}
    But then $\wt{\varphi}$ takes $E$ to itself, which means that it must swap the two rulings on $E$. But this means that the two copies of $\pi$ contract $E$ along the two rulings, as desired.

    To prove the lemma, $S^{[2]} \to G$ factors as $S^{[2]} \to Z \to G$, where $Z$ is normal and $Z \to G$ is finite. But then $\varphi$ descends to an honest morphism $\ol{\varphi} \colon Z \to Z$, and thus if $\ell^{[2]}$ is contracted to $z_0$, $\ol{\varphi}$ lifts to $\Bl_{z_0} Z = \Bl_{\ell^{[2]}} S^{[2]}$.
\end{proof}

\begin{prop}[Huybrechts]\label{prop:biratdefs}
    This proposition comes from the paper \textit{Birational symplectic manifolds and their deformations}. Let $\P^n \subseteq X^{2n}$, where $X$ is K\"ahler and symplectic and $f \colon X \dashrightarrow X'$ be the Mukai flop. Then there exist two birational smooth proper families
    \begin{equation*}
    \begin{tikzcd}
        \mc{X} \ar[dashrightarrow]{rr}{\phi} \ar{dr} & & \mc{X}' \ar{dl} \\
        & \Delta
    \end{tikzcd}
    \end{equation*}
    such that $\phi_t$ is an isomorphism for all $t \neq 0$, $\mc{X}_0 = X$, and $\mc{X}_0' = X'$.
\end{prop}

\begin{cor}
    There exists an isomorphism of Hodge structure $H^*(X) \simeq H^*(X')$.
\end{cor}

\begin{proof}
    Let $\Gamma \subseteq \mc{X} \times_{\Delta} \mc{X}'$ be the fiber product. Then we know $\Gamma_t = \Gamma_{\varphi_t} \subseteq \mc{X}_t \times \mc{X}_t'$, and this implies that
    \[ \gamma_t^* \colon H^*(\mc{X}_t') \to H^*(X_t) \qquad \alpha \mapsto p_1^* [\Gamma] \smile p_2^*(\alpha) \]
    is an isomorphism. But then we know $H^*(\mc{X}_t') \simeq H^*(\mc{X}_0')$ and similarly for $\mc{X}$. We also have a correspondence $\Gamma_0^*$, and this is an isomorphism.
\end{proof}

\begin{exm}
    Let $S \to \P^2$ be a degree $2$ K3 surface. Then we obtain some $\P^2 \subseteq S^{[2]}$. Then the Mukai flop of $S^{[2]}$ is a hyperk\"ahler manifold $M$ with a Lagrangian fibration over $\check{\P}^2$. Here, if $\ell \subset \P^2$ is a line, we consider $C \in \abs{f^* \mc{O}_{\P^2}(1)}$, and the fiber over $[C] \in\abs{C} = \check{\P^2}$ is simply $\Pic^2(C)$. In addition, the Mukai flop takes $z \in S^{[2]}$ to the line bundle $\mc{O}_C(Z)$.
\end{exm}

\begin{prop}
    Let $\P^n \subseteq X^{2n}$ be a K\"ahler holomorphic symplectic manifold and $f \colon X \dashrightarrow X'$ be the Mukai flop. Then there exist $\mc{X}, \mc{X}'$ over a disk $\Delta$ and $\phi_t \colon \mc{X} \to \mc{X}'$ such that $\phi_t$ is an isomorphism for $t \neq 0$, and ${(\Gamma_{\phi})}_0 = \Gamma_f + \P^n \times \check{\P}^n$.
\end{prop}

\begin{cor}
    There exists a universal deformation space for $X$ (as a germ of complex manifold).
\end{cor}

\begin{proof}
    We have an identity
    \[ T_{\mr{Def}(X)} = H^1(X, T_X) = \Ext^1_{\mc{O}_X}(\Omega^1_X, \mc{O}_X). \]
    Here, a deformation $v$ is taken to the exact sequence
    \begin{equation}\label{eqn:normal} 0 \to T_X \to T_{\mc{X}}|_X \to N_{X/\mc{X}} = \mc{O}_X \to 0. \end{equation}
    The first step in the proof is to show that there exists $\mc{X} \to \Delta$ such that $N_{\P^n/\mc{X}} = V^{\vee} \otimes \mc{O}_{\P^n}(-1)$. To do this, note that $H^1(X, T_X) = H^1(X, \Omega^1_X)$ and there is a sequence of maps
    \[ H^1(X, T_X) \to H^1(P, T_X |_P) \to H^1(P, N_{P/X}) = H^1(P, \Omega^1_P). \]
    We also have a map $H^1(X, \Omega^1_X) \to H^1(P, \Omega^1_P)$, and the resulting diagram commutes. We need to find $v \in H^1(X, T_X)$ such that $v|_P \neq 0$. Note that because $X$ is K\"ahler, there exists a K\"ahler form $\omega$ that restricts to a nonzero form on $P$. Next, the exact sequence (\ref{eqn:normal}) remains exact after restricting to $P$, and therefore we have
    \begin{equation*}
    \begin{tikzcd}
        0 \ar{r} & T_X |_P \ar{r} \ar{d} & T_{\mc{X}}|_P \ar{r} \ar{d} & N_{X/\mc{X}} |_P \ar{r} \ar[equal]{d} & 0 \\
        0 \ar{r} & N_{P/X} \ar{r} & N_{P/\mc{X}} \ar{r} & \mc{O}_P \ar{r} & 0.
    \end{tikzcd}
    \end{equation*}
    Because $v_P \neq 0$, the bottom sequence is not split, and because $v_P \in H^1(P, N_{P/X}) = H^1(P, \Omega^1_P)$ is contained in a $1$-dimensional vector space, the sequence is actually the Euler sequence, and thus $N_{P/\mc{X}} = V^{\vee} \otimes \mc{O}_{P}(-1)$.

    Next, we consider the exceptional divisor $\P(V^{\vee} \otimes \mc{O}(-1)) = \P V^{\vee} \times \P V$ of $\Bl_P X$, and we can check that $\mc{O}(E) |_E = \mc{O}(-1, -1)$. By Nakano-Fujiki, there exists a contraction of the exceptional divisor $E$ onto the first factor.
\end{proof}

\section{Deformations}%
\label{sec:deformations}

For this part, we will follow notes by Voisin from a class in 2006-07. Let $X$ be a compact complex manifold (or a reduced variety). We will see that
\[ T_{\mr{Def}_X} = \mr{Def}_X(\C[\ep]) = \Ext^1(\Omega^1_X, \mc{O}_X). \]
Fix $\Delta_n = \Spec \C[t]/t^{n+1}$. By reducedness, for every $\mc{X}_1 \to \Delta_1$, we assign the exact sequence
\[ 0 \to \mc{I}/\mc{I}^2 \to \Omega^1_{\mc{X}}|_X \to \Omega^1_X \to 0. \]
\begin{rmk}
    The sheaf $\Omega^1_{\mc{X}}$ has torsion, but its restriction to the central fiber is locally free if $X$ is smooth. 
\end{rmk}
Conversely, given
\[ 0 \to \mc{O}_X \to \mc{E} \to \Omega^1_X \to 0, \]
we want to define an algebra $\mc{O}_{\mc{X_1}}$ fitting in
\[ 0 \to \mc{O}_X \to \mc{O}_{\mc{X}_1} \to \mc{O}_X \to 0. \]
Equivalently, we want a sheaf $A$ and $A \to \mc{E}$ commuting with the inclusion of $\mc{O}$ and the differentia $d$. We simply set
\[ A = \qty{(\alpha, f) \in \mc{E} \oplus \mc{O}_X} \mid r(\alpha) = \dd{f}, \]
where $r \colon \mc{E} \to \Omega^1_X$ is the map in the exact sequence above. To define the algebra structure, we simply set
\[ (\alpha, f) (\beta, g) = (\alpha g + \beta f, fg). \]
It remains to check that the kernel of $A \to \mc{O}_X$ is a square zero ideal.

Next, we will consider global deformations of $X$ a compact complex structure. We can consider the deformations of $X$ over either germs of complex spaces or local Artinian rings. 

\begin{thm}[Kuranishi]
    If $H^0(X, T_X) = 0$, then there exists a universal family $\mc{X} \to \mr{Def}(X)$ over a germ of (pointed) complex analytic spaces.
\end{thm}

Alternatively, using the point of view of Schlessinger, because $H^0(T_X) = 0$, then $\mr{Def}_X(-)$ satisfies the axiom $H_4$ and is thus pro-representable.

\begin{rmk}
    There is generally no chance of having an algebraic family. Unfortunately, even if the central fiber is algebraic, there are arbitrarily small deformations that are not algebraic. If you want to keep everything algebraic, then we need to mark $X$ with an ample line bundle.
\end{rmk}

\begin{thm}[Bogomolov, Tian, Todorov]
    Let $X$ be a compact K\"ahler Calabi-Yau manifold with $H^0(T_X) = 0$. Then the germ of space $(\mr{Def}(X), 0)$ is smooth (equivalently, the pro-representing ring $R$ is a formal power series ring).
\end{thm}

The proof of this result uses the $T^1$-lifting principle, resting on the fact that by the infinitesimal lifting principle, smoothness of $\mr{Def}(X)$ at $0$ is equivalent to the fact that deformations can be lifted to any order. Before we do this, we need some notation and results.

\begin{lem}
    Given an $n$-th order deformation $f_n \colon \mc{X}_n \to \Delta_n$, the sheaves $\Omega^1_{\mc{X}_n} |_{\mc{X}_{n-1}}$ and $\Omega^1_{\mc{X}_{n-1}/\Delta_{n-1}}$ are both locally free. Moreover, they fit into an exact sequence
    \[ 0 \to \mc{O}_{X_{n-1}} \xrightarrow{\dd{t}} \Omega^1_{\mc{X}_n} |_{\mc{X}_{n-1}} \to \Omega^1_{\mc{X}_{n-1} / \Delta_{n-1}} \to 0. \]
\end{lem}

\begin{proof}
    Note that $\Omega^1_{\Delta_n} = \qty{\dd{t} \mid t^n \dd{t} = 0}$. Then the sheaf $\Omega^1_{\Delta_n} |_{\Delta_{n-1}}$ is locally free of rank $1$ generated by $\dd{t}$. Now because $X$ is smooth, $\wh{\mc{O}}_X \simeq \C[[x_1, \ldots, x_m]]$, and in fact we have isomorphisms
    \[ \C[[x_1, \ldots, x_m, t]] / t^{n+1} \simeq \wh{\mc{O}}_{\mc{X}_n} \]
    for all $n$. This implies that $\Omega^1_{\mc{X}_n}$ is locally generated by $\dd{x_1}, \ldots, \dd{x_m}, \dd{t}$ with $t^n \dd{t} = 0$. In particuar, after killing $t^n$, we see that $\Omega^1_{\mc{X}_n} |_{\mc{X}_{n-1}}$ is locally free and generated by the $\dd{x_i}$. We know the exact sequence
    \[ f^* \Omega^1_{\Delta_n} \to \Omega^1_{X_n} \to \Omega^1_{X_n/\Delta_n} \to 0, \]
    and restricting to $\mc{X}_{n-1}$, we obtain the desired result.
\end{proof}

\begin{defn}
    Given $f_n \colon \mc{X}_n \to \Delta_n$, set
    \[ e_n \coloneqq [0 \to \mc{O}_{\mc{X}_{n-1}} \to \Omega^1_{\mc{X}_n} |_{X_{n-1}} \to \Omega^1_{\mc{X}_{n-1} / \Delta_{n-1}} \to 0] \in \Ext^1_{X_{n-1}}(\Omega^1_{\mc{X}_{n-1}/\Delta_{n-1}}, \mc{X}_{\mc{X}_{n-1}}). \]
    This is called the \textit{Kodaira-Spencer class}.
\end{defn}

\begin{rmk}
    By the lemma, we have
    \[ \Ext^1(\Omega^1_{\mc{X}_{n-1} / \Delta_{n-1}}, \mc{O}_{\mc{X}_{n-1}}) = H^1(T_{X_{n-1}/\Delta_{n-1}}). \]
\end{rmk}

\begin{thm}[Ran]
    Let $X$ be a compact complex manifold. Given $\mc{X}_n \to \Delta_n$, there exists a lift $f_{n+1} \colon \mc{X}_{n+1} \to \Delta_{n+1}$ if and only if $e_n$ lifts to some class $e_{n+1} \in H^1(T_{\mc{X}_n/\Delta_n})$, where the map
    \[ H^1(T_{\mc{X}_n / \Delta_n}) \to H^1(T_{\mc{X}_{n-1}} / \Delta_{n-1}) \]
    is induced as follows: make the identification
    \[ D \in \Der_{\Delta_n}(\mc{O}_{\mc{X}_n}, \mc{O}_{\mc{X}_n}) = T_{\mc{X}_n/\Delta_n} \mapsto D |_{\mc{X}_{n-1}}. \]
    Equivalently, given an extension
    \[ 0 \to \mc{O}_{\mc{X}_n} \to \mc{E} \to \Omega^1_{\mc{X}_n/\Delta_n} \to 0, \]
    everything is locally free, so we can restrict to $\Delta_{n-1}$ and use the identity $\Omega^1_{\mc{X}_n/\Delta_n} |_{\mc{X}_{n-1}} = \Omega^1_{\mc{X}_{n-1}/\Delta_{n-1}}$.
\end{thm}

\begin{lem}
    The algebra $\mc{O}_{\mc{X}_{n+1}}$ is determined by $\mc{O}_{X_n}$ and the short exact sequence
    \[ 0 \to \mc{O}_X \xrightarrow{t^{n} \dd{t}} \Omega^1_{\mc{X}_{n+1}}|_{\mc{X}_n} \xrightarrow{r} \Omega^1_{\mc{X}_n} \to 0. \]
\end{lem}

As we did earlier, we construct $A$ to fit into the exact sequence
\[ 0 \to \mc{O}_X \xrightarrow{t^n} A \to \mc{O}_{\mc{X}_n}. \]

\begin{proof}[Sketch of proof of theorem]
    One direction is clear. If there is a lift, we construct the class $e_{n+1}$ and clearly, it must restrict to $e_n$ by definition.

    In the other direction, suppose there is a class $e_{n+1}$ lifting $e_n$. We want to construct $\mc{O}_{X_{n+1}}$. Given this $e_{n+1}$, we will find $\mc{E}$ fitting into the sequence
    \[ 0 \to \mc{O}_{\mc{X}_n} \to \mc{E} \to \Omega^1_{\mc{X}_n / \Delta_n} \to 0. \]
    Given any $\mc{E}$ as above, there exists an isomorphism $\Omega^1_{\mc{X}_n} |_{\mc{X}_{n-1}} \simeq \mc{E} |_{\mc{X}_{n-1}}$ induced by a surjection $r \colon \mc{E} \to \Omega^1_{\mc{X}_n}$. This will give us the desired algebra.

    To prove that such an isomorphism exists, by the existence of $e_{n+1}$, we already have a surjection $f_1 \colon \mc{E} \to \Omega^1_{\mc{X}_n / \Delta_n}$. Then after restricting to $\mc{X}_{n-1}$, we have a sequence
    \[ \mc{E} \xrightarrow{f_2} \mc{E} \_{\mc{X}_{n-1}} \simeq \Omega^1_{\mc{X}_n} |_{\mc{X}_{n-1}} \xrightarrow{g_2} \Omega^1_{\mc{X}_{n-1} / \Delta_{n-1}}. \]
    We now claim that
    \[ \Omega^1_{\mc{X}_n} \subseteq \Omega^1_{\mc{X}_n / \Delta_n} \oplus \Omega^1_{\mc{X}_n} |_{\mc{X}_{n-1}} \xrightarrow{g_1,g_2} \Omega^1_{\mc{X}_{n-1} / \Delta_{n-1}}, \]
    Because $g_1 \circ f_1 = g_2 \circ f_2$, we have our desired $r \colon \mc{E} \to \Omega^1_{\mc{X}_n}$.

    Next, we show that $\ker r \simeq \mc{O}_X$. We have a short exact seqeunce defining $\mc{E}$, and then we obtain a diagram
    \begin{equation*}
    \begin{tikzcd}
        & \mc{O}_X \ar{r}{\sim} \ar{d}{t^n} & \ker r \ar{d} \\
        0 \ar{r} & \mc{O}_{\mc{X}_n} \ar{r} \ar{d} & \mc{E} \ar{r} \ar{d}{r} & \Omega^1_{\mc{X}_n/\Delta_n} \ar{r} \ar[equal]{d} & 0 \\
        f^* \Omega^1_{\Delta_n} \ar[twoheadrightarrow]{r} & \mc{O}_{\mc{X}_{n-1}} \ar{r} & \Omega^1_{\mc{X}_n} \ar{r} & \Omega^1_{\mc{X}_n} \ar{r} & 0.
    \end{tikzcd}
    \end{equation*}
    Constructing $A \subseteq \mc{E} \oplus \mc{O}_{\mc{X}_n}$ as we did before, we are done.
\end{proof}

To prove that $\mr{Def}(X)$ is smooth when $X$ is a compact Calaby-Yau, we need to check the $T^1$ lifting principle. In order to do this, we need some Hodge theory.

\begin{lem}
    The sequence
    \[ \mc{O}_X \xrightarrow{\partial} \Omega^1_{X} \xrightarrow{\partial} \Omega^2_X \to \cdots \]
    is a resolution of the constant sheaf $\C$ in the analytic topology.
\end{lem}

\begin{cor}
    $H^k(X, \C) = H^k(X, \Omega^{\bullet})$.
\end{cor}

This gives us a filtration b\^ete (for stupid filtration) $F^p \Omega^{\bullet}_X = \Omega_X^{\geq p}$, which by standard techniques leads to a spectral sequence where
\[ E_1^{p,q} = H^q(\Omega_X^p) \Rightarrow \mr{gr}_F H^k(X, \Omega^{\bullet}_X) \]
which comes from the filtration $F^p H^k(\Omega^{\bullet}_X) = \Im (H^k(\Omega_X^{\geq 0}) \to H^k(\Omega_X^{\bullet}))$.

If $X$ is K\"ahler and compact, then the Hodge theorem implies that this spectral seqeunce (called the Frolicher spectral sequence) at the $E_1$-page and that $F^p H^k(X, \C)$ is the Hodge filtration. To see this, note that
\[ b_k(X) = \dim E_{\infty}^{p,q} \leq \sum \dim E_{1}^{p,q} = \sum h^{p,q} = b_k(X). \]
Deligne in the paper \textit{Th\'eor\`eme de Lefschetz et Crit\`eres de D\'eg\'en\'erescence de Suites Spectrales}  shows that if $X$ is any smooth proper scheme over $\C$, the Frolicher spectral sequence degenerates at $E_1$.

Previously, we considered $\mc{X} \to B$ smooth proper morphisms of complex manifolds. then if $X_0$ is K\"ahler, the $h^{p,q}(X_t)$ are locally constant, and in fact the degeneration of the Frolicher spectral sequence at $E_1$ is enough. Thus if $B$ is a complex manifold, the sheaves $R^q f_* \Omega^p_{X/B}$ are locally free and satisfy base change.

\begin{prop}
    Let $f \colon \mc{X} \to B$ be smooth and proper with $B$ a scheme over $\C$ (possibly of finite type). Then the higher direct images $R^k f_* \Omega^{\bullet}_{X/B}$ and $R^q f_* \Omega^p_{X/B}$ are locally free and satisfy base change. Moverover, there exists a filtration $\mc{F}^p R^k f_* \Omega^{\bullet}_{\mc{X}/B}$ whose successive quotients are locally free and whose associated graded components are $R^q f_* \Omega^p_{\mc{X}/B}$.
\end{prop}

\begin{lem}[Deligne]
    Let $A$ be a local Artinian ring over $\C$ and $K^{\bullet}$ be a bounded above complex of free $A$-modules. Then $\ell_A(H^n(K^{\bullet})) \leq \ell(A) \cdot \ell_{\C} H^n(K^{\bullet} \otimes_A \C)$, and if equality holds, then base change holds in degree $n, n+1$, which means that for $j = n, n+1$, we have
    \[ H^j(K^{\bullet}) \otimes_A N \simeq H^j(K^{\bullet} \otimes_A N), \]
    where $N$ is any $B$-module of finite type for some Artinian $A$-algebra $B$. In addition, $H^n(K^{\bullet})$ is a free $A$-module.
\end{lem}

\begin{proof}[Proof of proposition]
    We will reduce to the case of $B = \Spec A$, where $A$ is an artinian ring over $\C$. Recall that we have an exact sequence
    \[ 0 \to f^* \Omega^1_B \to \Omega^1_X \to \Omega^1_{X/B} \to 0. \]
    Then we have the relative de Rham complex which resolves
    \[ f^{-1} \mc{O}_B \to \Omega^{\bullet}_{X/B}. \]
    Now because $B$ is affine, we note that
    \[ R^k f_* \Omega^{\bullet}_{X/B} = H^k(X, f^{-1}\mc{O}_B) = H^k(X, A) = H^k(X_0, \C) \otimes_{\C} A \]
    by $A$-linearity of the differential. Thus $\mc{H}^k R^k f_* \Omega^{\bullet}_{X/B}$ is a free $A$-module (in the general case, we obtain $R^k f_* \Omega^1_{X/B} = (R^k f_* \C) \otimes \mc{O}_B$).

    Now we consider the stupid filtration $\mc{F}^p \Omega^{\bullet}_{X/B} = \Omega^{\geq p}_{X/B}$ and this induces a spectral sequence
    \[ E_1^{p,q} = R^q f_* \Omega^p_{X/B} \Rightarrow \mr{Gr}_F R^k f_* \Omega^{\bullet}_{X/B}, \]
    where $\mc{F}^p R^k f_* \Omega^{\bullet}_{X/B} = \Im (R^k \mc{F}^p \Omega^{\bullet}_{X/B} \to R^k \Omega^{\bullet}_{X/B})$. We now have
    \begin{align*} 
        \ell(A) \cdot b_k(X_0) &= \ell (R^k f_* \Omega^{\bullet}_{X/B}) \\
        &= \sum_{p+q=k} \ell(E_{\infty}^{p,q}) \\
        &\leq \sum_{p+q=k} \ell(E_1^{p,q})  \\
        &= \sum \ell(R^q f_* \Omega^p_{X/B}) \\
        &\leq \ell(A) \sum \ell(H^q(\Omega^p_{X_0})) \\
        &= b_K(X_0) \cdot \ell(A)
    \end{align*}
    by the lemma, so all inequalities are equalities. Using the lemma again, the $R^p f_* \Omega^q_{X/B}$ are free $A$-modules and satisfy base change. The remainder of the result is easy to see.
\end{proof}

\begin{proof}[Proof of Bogomolov-Tian-Todorov]
    Given $\mc{X}_n \to \Delta_n$, we have a class $e_n \in H^1(T_{X_{n-1}/\Delta_{n-1}})$, and we want to lift this to $H^1(T_{X_n} / \Delta_n)$. Suppose that $X_0$ is a compact Calabi-Yau such that the Frolicher spectral sequence degenerates at $E_1$ and $\dim X_0 = m$. Then $\Omega^m_{X_n/\Delta_n} \cong \mc{O}_{X_n}$, so there exists a perfect pairing
    \[ \Omega^1_{X_n/\Delta_n} \otimes \Omega^{m-1}_{X_n / \Delta_n} \to \Omega^m_{X_n / \Delta_n} \simeq \mc{O}_{X_n}. \]
    Thus $T_{X_n/\Delta_n} \simeq \Omega^{m-1}_{X_n/\Delta_n}$. If we consider
    \[ R^q f_{n*} \Omega^{m-1}_{X_n/\Delta_n} = H^1(\Omega^{m-1}_{X_n/\Delta_n}) \to H^1(\Omega^{m-1}_{X_{n-1}/\Delta_{n-1}}) = R^1 f_{n-1,*} \Omega^{m-1}_{X_{n-1}/\Delta_{n-1}} = R^1 f_* \Omega^{m-1}_{X_n/\Delta_n} |_{X_{n-1}}, \]
    we are done.
\end{proof}

\section{Some Hodge theory}%
\label{sec:some_hodge_theory}

Let $f \colon \mc{X} \to B$ be a proper surjective smooth morphism either of schemes over $\C$ or with K\"ahler fibers $X_b$. Let
\[ \mc{H}^k \coloneqq R^k f_* \Omega^{\bullet}_{X/B} = R^k f_* \C \otimes_{\C} \mc{O}_B \]
be the Hodge bundle. Then there is a decreasing filtration $\mc{F}^p \mc{H}^k \subset \mc{H}^k$ in subbundles. Now suppose that $B$ is sufficiently small so that $\mc{H}^k = H^k(X_0, \C) \otimes \mc{O}_B$ is free.

\begin{thm}
    Fix $k, p$. Then there is a holomorphic map, called the \textit{period mapping}, 
    \[ B \xrightarrow{\varphi} \mr{Gr}(f_k^p, H^k(X_0, \C)) \qquad b \mapsto F^p H^k(X_b) \subseteq H^k(X_b, \C) \cong H^k(X_0, \C), \]
    where $f_k^p = \sum_{\ell \geq p} h^{\ell, k-\ell}(X_0)$. Pulling back the tautological seqeunce on the Grassmannian gives $\mc{F}^p \mc{H}^k \subseteq \mc{H}^k$.
\end{thm}

We would like to study $\varphi$, and a first step is to study its differentials. Note that $\varphi$ factors via the universal deformation space $\mr{Def}(X_0)$, so we have a diagram
\begin{equation*}
\begin{tikzcd}
    B \ar{rr}{\varphi} \ar{dr} & & \mr{Gr}(f_k^p, H^k(X_0)) \\
    & \mr{Def}(X_0) \ar{ur}{\wp}.
\end{tikzcd}
\end{equation*}
Now recall that $T_{[W]} \mr{Gr}(j, V) = \Hom(W, V/W)$, and so the differential of the period map is a morphism
\[ H^1(T_X) \xrightarrow{\dd{\wp_0}} \Hom(F^p H^k, H^k / F^p H^k). \]
This factors through $\Hom(F^p, F^{p-1}/F^p)$ by Griffiths transversality, and so we have the diagram
\begin{equation*}
\begin{tikzcd}
    H^1(T_X) \ar{rr} \ar{dr} & & \Hom(F^p, F^{p-1}/F^p) \ar{dl} \\
    & \Hom(F^p/F^{p-1}, F^{p-1}/F^p),
\end{tikzcd}
\end{equation*}
and the bottom term in the diagram is isomorphic to $\Hom(H^{k-p}(\Omega^p_X), H^{k-p+1}(\Omega^{p-1}_X))$.

\begin{prop}
    The morphism $H^1(X, T_X) \to \Hom(H^{k-p}(\Omega^p_X), H^{k-p+1}(\Omega_X^{p-1}))$ is the morphism sending a tangent vector $v$ to the morphism induced by contraction by $v$.
\end{prop}

Returning to irreducible holomorphic symplectic manifolds, let $X$ be irreducible holomorphic symplectic. Then the Hodge filtration here gives
\[ F^2 H^2 = H^0(\Omega^2_X) = C \sigma \qquad F H^2 \twoheadrightarrow H^1(\Omega^1_X), H^2(X) \twoheadrightarrow H^2(\mc{O}_X) = \C \ol{\sigma}. \]
Then the period map is a morphism
\[ \mr{Def}(X) \xrightarrow{\wp} \P H^2(X, \C) \qquad t \mapsto \C \sigma_t, \]
where we have made an identification $\eta_t \colon H^2(X_t, \C) \cong H^2(X_0, \C)$.

\begin{prop}
    The differential $\dd{\wp_0}$ has maximal rank.
\end{prop}

\begin{proof}
    Write the morphism
    \begin{equation*}
    \begin{tikzcd}
        H^1(T_X) \ar{r}{\dd{\wp_0}} \ar{dr} & \Hom(H^{2,0}, H^2/H^{2,0}) \ar{d} \\
        & \Hom(H^{2,0}, F^1 H^2 / H^{2,0}).
    \end{tikzcd}
    \end{equation*}
    It is enough to show that the morphism $H^1(T_X) \to \Hom(H^{2,0}, F^1 H^2 / H^{2,0}) = H^1(\Omega^1)$ is an isomorphism. But this map is given by contraction, and so it is precisely the isomorphism $T_X \simeq \Omega^1_X$ induced by $\sigma$.
\end{proof}
We conclude that $\wp$ is an isomorphism onto its image. There will be a local statement, where we consider small deformations, and a global statement, where we consider the entire moduli space. First, note that $\dim \mr{Def}(X) = b_2 - 2$ and $\dim \P H^2(X) = b_2 - 1$.

\begin{prop}
    There exists a quadric hypersurface $Q \subseteq \P H^2(X, \C)$ such that $\Im \wp \subseteq Q$.
\end{prop}

\begin{proof}
    First note that $\Im \wp$ is contained in a degree $2n$-hypersurface, where $\dim X = 2n$. There is a natural degree $2n$ polynomial on $H^2$ given by the cup product of $H^2$ with itself $2n$ times. By type reasons\footnote{This sounds a lot like computer science.}, because $\sigma_{X_t}$ is a $(2,0)$-form, we know $\sigma_{X_t}^{2n} = 0$, and thus $\sigma_t^{2n} = 0$. Thus if $F$ is the cup product polynomial, we see, $\Im \wp \subseteq \qty{F = 0} \subseteq \P H^2(X, \C)$. (Note that for K3 surfaces, we are done and that this quadric is defined over $\Z$. In fact, the $H^2(S, \Z)$ is a unimodular lattice.)

    Now we will prove that $F = q^n$ up to a constant. To do this, we will prove that $F$ vanishes with order at least $n$ on $\Im \wp$ and then write down a quadric explicitly. Write $\sigma_t = \sigma_0 + t \alpha$, and so we have
    \begin{align*}
        \sigma_t^{2n} &= {(\sigma_0 + t \alpha)}^{2n} \\
        &= \sigma_0^{2n} + t \sigma_0^{2n-1} \alpha + \cdots + t^n \sigma^n \alpha.
    \end{align*}
    Thus $F$ vanishes up to order $n$, as desired.
\end{proof}

\begin{thm}[Beauville, Bogomolov, Fujiki]
    Let $X$ be irreducible holomorphic symplectic manifold of dimension $2n$. There exists an integral, indivisible, quadratic form $q \colon H^2(X, \C) \to \C$ of signature $(3, b_2 - 3)$ and a constant $c_X \in \Q_{>0}$ such that 
    \[ \int_X \alpha^{2n} = c_X {q(\alpha)}^n \] 
    for all $\alpha \in H^2(X)$.
\end{thm}
Note that the relation, called the \textit{Fujiki relation}, identifies $q, c_X$ with no ambiguity except when $n$ is even, in which case we specify $q(\omega) > 0$ for $\omega$ K\"ahler. Moreover:
\begin{itemize}
    \item $\Im \wp \subseteq \Omega = \qty{ q(x) = 0, q(x,\ol{x}) > 0} \subseteq \Q$. In particular, the map $\wp \colon \mr{Def}(X) \to \Omega$ is a local isomorphism.
    \item With respect to $q$, $H^{1,1} \perp H^{2,0} \oplus H^{2,0}$.
\end{itemize}

\begin{proof}
    We will normalize $\sigma$ such that $\int {(\sigma \ol{\sigma})}^n = 1$. For a class $\alpha = a \sigma + \omega + b \ol{\sigma}$, we will define
    \begin{align*} 
        q(\alpha) &= ab + \int {(\sigma \ol{\sigma})}^{n-1} \cdot \omega^2  \\
        &= \frac{n}{2} \int {(\sigma \ol{\sigma})}^{n-1} \alpha^2 + (1-n) \qty(\int \sigma^{n-1} \ol{\sigma}^n \alpha) \cdot \qty(\int \sigma^n \ol{\sigma}^{n-1} \alpha).
    \end{align*}
    First we check that if $\omega$ is a K\"ahler form, then $q(\omega) > 0$. In fact, we have
    \[ q(\alpha) = \int {(\sigma \ol{\sigma})}^{n-1} \cdot \omega^2, \]
    and now we use the Hodge-Riemann bilinear relations. Here, if $M$ is compact and K\"ahler with $\dim M = m$ and $\omega$ a K\"ahler form, define
    \[ (\alpha, \beta) \coloneqq \int \alpha \wedge \ol{\beta} \wedge \omega^{m-k}. \]
    Then the bilinear form
    \[ i^{p-q} {(-1)}^{\frac{k(k-1)}{2}} (-,-) |_{H^{p,q}_{\mr{prim}}} \]
    is positive-definite. In our situation, note that $\sigma^{n-1}$ is a primitive form, and so we obtain the desired result. It is clear that $q(\sigma) = 0, q(\Re(\sigma)) = q(\Im(\sigma)) > 0$, and $q(\Re(\sigma), \Im(\sigma)) = 0$, and so $q$ has rank at least $3$ and is thus irreducible.

    We now prove the Fujiki relation. By type reasons, we have $\sigma_t^{n+1} = 0$, and so if we write $a_t \sigma + \omega_t + b_t {\ol{\sigma}}$, and thus
    \[ {(\sigma_t^{n+1})}_{2n,2} = 0 = a_t^n b_t \sigma^n \ol{\sigma} + a_t^{n-1} \sigma^{n-1} \omega^2, \]
    and when we multiply by $\ol{\sigma}^{n-1}$, we have
    \[ 0 = a_t^{n-1} \qty(a_t b_t + \int {(\sigma \ol{\sigma})}^{n-1} \omega^2). \]
    Because $a_t \neq 0$ for sufficiently small $t$, we see $q(\sigma_t) = 0$, and thus $(q) = I(\Im(\wp)) \ni F$, which gives us the Fujiki relation.

    We will compute the signature of $q$. If we differentiate the equation
    \[ \int \alpha^{2n} = c {q(\alpha)}^{n} \]
    with respect to $t$ under $\alpha + t \beta$, we have
    \[ 2n \cdot \alpha^{2n-1} \beta = 2n c {q(\alpha)}^{n-1} q(\alpha, \beta). \]
    Now if $\omega$ is K\"ahler form, then $\beta$ is primitive if and only if $q(\omega, \beta) = 0$. It is now enough to compute the sign of $q |_{H^{1,1}_{\mr{prim}}}$. Differentiating again, we obtain
    \[ (2n-1) \alpha^{2n-2} \beta \wedge \gamma = 2(n-1) c {q(\alpha)}^{n-2} 2(\alpha, \gamma) q(\alpha, \beta) + c {q(\alpha)}^{n-1} q(\gamma, \beta). \]
    Choosing $\alpha = \omega$ to be K\"ahler and $\beta, \gamma$ primitive, we obtain
    \[ (2n-1) \alpha^{2n-2} \beta \wedge \gamma = c {q(\alpha)}^{n-1} q(\gamma, \beta), \]
    and by the Hodge-Riemann bilinear equations, the left-hand-side is negative.

    Finally, we need to prove integrality. For all $\lambda, \alpha \in H^2(X, \C)$, we have
    \[ {(\lambda^{2n})}^2 q(\alpha) = q(\lambda) [(2n-1) \lambda^{2n}(\lambda^{2n-2} \alpha^2) - (2n-1) {(\lambda^{2n-1} \alpha)}^2]. \]
    This is obtained from previous formulae by multiplying the derivative of the Fujiki repation by $q(\alpha)$, using the Fujiki relation, and taking the derivative again.
\end{proof}

\begin{cor}
    Up to multiplication by a nonzero constant, we can assume that if $\alpha \in H^2(X,\Q)$, then $q(\alpha) \in \Q$.
\end{cor}

\begin{proof}
    We prove that there exists $\lambda \in H^2(X, \Q)$ such that $q(\lambda) \neq 0$. The class $\sigma + \ol{\sigma} \in H^2(X, \R)$ and $q(\sigma + \ol{\sigma}) > 0$, so by density of $H^2(X, \Q) \subset H^2(X, \R)$, we are done.
\end{proof}

Once we have this normalization, we have proven the integrality statement in Beauville-Bogomolov-Fujiki.

\begin{rmk}
    The Fujiki relation implies that both $q(-)$ and $c_X \in \Q_{>0}$ in the relation
    \[ \int \alpha^{2n} = c_X {q(\alpha)}^n \]
    are deformation invariants.
\end{rmk}

Using the Hodge-Riemann bilinear relations, we see that restricting $q$ to $H^{2,0} \oplus H^{0,2}$ (as a real vector space) is positive definite. In addition, a basis of $H^{2,0} \oplus H^{0,2}$ is $\Re \sigma, \Im \sigma$, and in fact $q(\Re \sigma, \Im \sigma) = 0$.

Now the period map $\wp \colon \mr{Def}(X) \to Q$ lands in the set 
\[ \Omega = \qty{x \mid q(x) = 0, q(x+\ol{x}) > 0}. \]
This set $\Omega$ is called the \textit{period domain}.

\begin{rmk}
    The points of $\Omega$ parameterize Hodge structures on $H^2(X, \Z)$ of K3-type ($1, b_2-2, 1$). These structures have $x \in H^2(X, \C)$ with $q(x) = 0, q(x, \ol{x}) > 0$, and then $H^{1,1}$ is the orthogonal complement of $x$.
\end{rmk}

We hope that for every Hodge structure, there is some manifold realizing the Hodge structure, but this is a highly nontrivial result of Huybrechts.

\begin{prop}
    There exists a natural diffeomorphism $\Omega \simeq \mr{Gr}^+(2, H^2(X, \R))$, where the $+$ means that for a subspace $W$, we have both an orientation and positive-definiteness of $q |_W$ given by
    \[ \sigma \mapsto \ev{\Re \sigma, \Im \sigma} \qquad W = \ev{w_1, w_2} \mapsto w_1 + i w_2. \]
\end{prop}

Now we consider the case of K3 surfaces. We know that if $X = S$ is a K3 surface, then $q$ is simply the cup product. To say more about this, first we will prove
\begin{prop}
    We have an identity $c_2(S) = 24 \in H^4(S, \Z) = \Z$, and of course this means $b_2(S) = 22$ and $h^{1,1} = 20$. In addition, we have $H^2(S, \Z) = {E_8(-1)}^{\oplus 2} \oplus U^{\oplus 3}$, where $U = \begin{psmallmatrix} 0 & 1 \\ 1 & 0 \end{psmallmatrix}$.
\end{prop}

\begin{proof}
    Recall Noether's formula, which says that
    \[ \chi(\mc{O}_M) = \frac{{c_1(M)}^2 + c_2(M)}{12}. \]
    In the case of a K3 surface, we know $\chi(\mc{O}_M) = 2$, so we get $c_2(S) = 24$. Because $c_2(S)$ is also the Euler class, we see that $\chi_{\mr{top}}(S) = 24$, which gives us $b_2 = 22$ and $h^{1,1} = 20$. The computation of $H^2(S, \Z)$ as a \textbf{lattice} follows from the abstract classification of lattices once we know that $H^2(S, \Z)$ is unimodular and even of indefinite signature. Recall that a lattice $\Lambda$ is called even if for all $\alpha \in \Lambda$, $\alpha^2 \in 2\Z$ and unimodular if the matrix corresponding to the bilinar form has determinant $\pm 1$.\footnote{For other hyperk\"ahler manifolds, the lattice is not unimodular.}

    To prove that $H^2(S, \Z)$ is even, we use Wu's formula, which says that for all $\alpha \in H^2(M, \Z)$ (for any compact complex surface $M$) we have $c_1 \cdot \alpha \equiv \alpha^2 \pmod 2$. For a K3 surface, this clearly implies that $H^2(S, \Z)$ is even.
\end{proof}

We now consider Hirzebruch-Riemann-Roch on a K3 surface. If $L$ is a line bundle and $X$ is a surface, we have
\[ \chi(X, L) = \frac{L^2 - L \cdot K_X}{2} + \chi(\mc{O}_X). \]
For a K3 surface, we have $\chi(S, L) = \frac{L^2}{2} + 2$.

\begin{cor}\leavevmode
    \begin{itemize}
        \item If $L^2 \geq -2$, then $\pm L$ is effective.
        \item If $L^2 \geq 0$, then either $L = \mc{O}_S$ or $h^0(\pm L) \geq 2$.
        \item If $L = \mc{O}_S(C)$ for an irreducible curve $C$, then $L^2 = 2g-2$.
    \end{itemize}
\end{cor}

\begin{exm}
    If $C = R$ is a smooth rational curve, then $R^2 = -2$. If $C = E$ is an elliptic curve, then $E^2 = 0$.
\end{exm}

Now for a compact complex manifold, define the \textit{Neron-Severi group} $\mr{NS}(X) = \Pic X / \Pic^0 X$. This emerges from the exponential sequence
\[ 0 \to \Z \to \mc{O}_X \to \mc{O}_X^* \to 1 \]
as the image of $\Pic X \xrightarrow{c_1} H^2(X, \Z)$. By the Lefschetz theorem on $(1,1)$-classes, we know $\Im c_1 = H^2(X, \Z) \cap H^{1,1}_{\R}$. In particular, if $X$ is irreducible holomorphic symplectic, $H^1(\mc{O}_X) = 0$, so $\mr{NS} = \Pic$. The reason we care about this is that if we consider $\mr{Def}(X) \to \Omega$ and identify $H^2(X, \R) = H^2(X_t, \R)$, then the rank $\rho(X) = \rk \Pic X$ can vary.

We now return to consider consequences of the local Torelli theorem.
\begin{prop}
    Let $X$ be irreducible holomorphic symplectic. Then some small deformation of $X$ is projective.
\end{prop}

\begin{proof}
    It suffices to show that there exists a small deformation such that $X_t$ contains a K\"ahler class that is rational. By the Kodaira embedding theorem, such a class is the first Chern class of an ample line bundle. To prove this, recall the identification $\Omega \cong \mr{Gr}^+(2, H^2(X, \R))$. But then the set of planes $W \subseteq H^2(X, \R)$ defined over $\Q$ are dense, so for $0 \in \mr{Def}(X)$ and corresponding $\wp(0) \in \Omega$, we can choose a nearby point such that $W$ is defined over $\Q$. But then $W^{\perp} = H^{1,1}$ is also defined over $\Q$. Thus if $t \in \mr{Def}(X)$ satisfies $\wp(t) = W$, then $H^{1,1}(X_t)$ is also defined over $\Q$. In particular, $H^{1,1}_{\R} \cap H^2(X_t, \Q)$ is dense and has maximal rank. But then it must have nonempty intersection with the K\"ahler cone of $X_t$, so we are done.
\end{proof}

Now let $X$ be irreducible holomorphic symplectic, $L$ be a line bundle, and $\ell = c_1(L) \in H^2(X, \Z)$. Set $\Omega_{\ell} \coloneqq \Omega \cap \ell^{\perp}$ and ${\mr{Def}(X)}_{\ell} = \wp^{-1} (\Omega_{\ell}) \subseteq \mr{Def}(X)$. Then if $\mc{X} \to \mr{Def}(X)$ is the universal deformation, we will call $\mc{X}_{\ell}$ the base change to ${\mr{Def}(X)}_{\ell}$.

\begin{prop}
    The space $\mr{Def}(X)_{\ell} \subset \mr{Def}(X)$ is a smooth hypersurface and is the univeral deformation space of $\mr{Def}_{(X, L)}$ (which means there exists a universal line bundle $\mc{L}$ on $\mc{X}_{\ell}$ such that $\mc{L} |_{X_0} = L$, $\mc{L} |_{X_t} = L_t$, and $c_1(L_t) = \ell_t$). More generally, if $L_1, \ldots, L_k \in \Pic X$ have $\ell_1, \ldots, \ell_k$ linearly independent, then $\mr{Def}(X, L_1, \ldots, L_k) \subseteq \mr{Def}(X) = \wp^{-1}(\Omega \cap \ev{\ell_i}^{\perp})$ is smooth of codimension $k$.
\end{prop}

\begin{proof}
    Note that $\ell^{\perp} \subseteq \P H^2(X, \C)$ is smooth if and only if $q(\ell) \neq 0$. On the other hand, if $q(\ell) = 0$, the only singular point is $\ell \in \ell^{\perp}$, but then such an $\ell$ cannot lie in $\Omega$. Thus a necessary condition for $X_t$ to have a line bundle $L_t$ with $c_1(L_t) = \ell_t$ is that $\ell_t$ is a $(1,1)$-class, which is equivalent to $q(\ell_t, \sigma_t) = 0$. Of course, this is equivalent to $\wp(t) \in \Omega_{\ell}$. In particular, we know that if $(\mc{X}, \mc{L}) \to B$ is a deformation of $(X, L)$, the Kodaira-Spencer map $B \to \mr{Def}(X)$ must factor through ${\mr{Def}(X)}_{\ell}$. Of course, by the Lefschetz $(1,1)$-theorem, this is also a sufficient condition. In particular, for all $t \in \mr{Def}(X)_{\ell}$, we have $q(\sigma_t, \ell_t) = 0$, so $\ell_t \in H^{1,1} \cap H^2(X, \Z)$, so there exists a unique line bundle $L_t$ such that $c_1(L_t) = \ell_t$. Therefore on $X_{\ell} \to \mr{Def}(X)_{\ell}$, every fiber has a line bundle, so we prove that there exists a global line bundle and that such a line bundle is universal.

    We show that $\mr{Def}(X, L)$ is unobstructed with tangent space $\ker [H^1(T_X) \xrightarrow{c(L)} H^2(X, \mc{O}_X)] \subseteq H^1(T_X)$. This is induced by the perfect pairing
    \[ H^1(T_X) \otimes H^1(\Omega^1_X) \to H^2(\mc{O}_X) = \C. \]
    Consider $\mc{O}_X \xrightarrow{\dd} \Omega^1_X$ and the corresponding map $\mc{O}_X^* \to \Omega^1_X$ given by $u \mapsto \frac{\dd u}{u}$. This induces a map
    \[ H^1(X, \mc{O}_X) \xrightarrow{c(-)} H^1(X, \Omega_X) = \Ext^1(T_X, \mc{O}_X). \]
    Thus we have $L \mapsto c(L) \in [0 \to \mc{O}_X \to \mc{E}_L \to T_X \to 0]$. This gives us an exact sequence
    \[ 0 = H^1(X, \mc{O}_X) \to H^1(X, \mc{E}_L) \hookrightarrow H^1(X, T_X) \twoheadrightarrow H^2(X, \mc{O}_X) \to H^2(X, \mc{E}_L) \to H^2(X, T_X). \]
    We also have $H^1(X, \mc{E}_L) = T_{\mr{Def}(X, L)}$ and of course $H^1(X, T_X) = T_{\mr{Def}(X)}$. In addition, we have $H^2(X, \mc{E}_L) = \mr{Obs}(\mr{Def}_{(X,L)})$, but deformations of $X$ are unobstructed, so $\mr{Obs}(X, \mc{E}_L) = 0$, and thus deformations of $(X, L)$ are unobstructed.
\end{proof}

\section{Noether-Lefschetz loci}%
\label{sec:noether_lefschetz_loci}

\begin{defn}
    Let $f \colon \mc{X} \to B$ be a non-isotrivial family of irreducible holomorphic symplectic varieties over a connected $B$.\footnote{Note this is equivalent to the Kodaira-Spencer map being nontrivial.} Define
    \[ \rho_0 = \min_{t \in B} \qty{\rho(X_t)}. \]
    Then the \textit{Noether-Lefschetz locus}\footnote{Note this is a union of Hodge loci.} of $f$ is
    \[ \mr{NL}(f) = \qty{t \mid \rho(X_t) > \rho_0}. \]
\end{defn}

\begin{prop}[Green]
    The Noether-Lefschetz locus $\mr{NL}(f) \subseteq B$ is dense in the analytic topology.
\end{prop}

In fact, we will prove a stronger statement.

\begin{prop}[Oguiso]
    Suppose that $B$ is small enough such that there is an identification $\eta_t \colon H^2(X_t, \Z) \simeq H^2(X_0, \Z) \eqqcolon \Lambda$. In this case, there exists a period mapping $\wp \colon B \to \P \Lambda_{\C}$. Then there exists a primitive sublattice $\Lambda_0 \subseteq \Lambda$ of rank $\rho_0$ such that for all $t \in B$, $\NS(X_t) \supseteq \Lambda_0$.\footnote{Giulia did not give us a precise reference for this, but she said that there are two papers with the keywords `Picard rank' and `hyperk\"ahler' and that we would be able to figure out which one it is.}
\end{prop}

\begin{proof}
    Let $I$ be the set of all possible primitive sublattices $\Lambda_{\alpha} \subseteq \Lambda$. Then for all $\alpha \in I$, define $B_{\alpha} = \qty{t \in B \mid \NS(X_t) = \Lambda_{\alpha}}$. But then we see that $B = \bigcup B_{\alpha}$ and that $\wp(B_{\alpha}) \subseteq \P {(\Lambda_{\alpha, \C})}^{\perp}$. But now we know that
    \[ B = \bigcup \wp^{-1} (\Lambda_{\alpha, \C}^{\perp}), \]
    and thus there exists $\alpha_0$ such that $B = \wp^{-1}(\Lambda_{\alpha_0, \C}^{\perp})$ and therefore $\Lambda_{\alpha_0} \subseteq \NS(X_t)$ for all $t$.
\end{proof}

\begin{proof}[Proof of Green]
    We will assume that $B = \Delta$ is a disk and that $\wp \colon \Delta \to \P \Lambda_{\C}$ is injective. Let $\mc{H}^2_{\C} = R^2 f_* \C \otimes_{\C} \mc{C}^{\infty}(\Delta) \simeq H^2(X_0, \C) \times \Delta$ (where the last identification is local). Here, we have a Hodge filtration $\mc{F}^{\bullet} \mc{H}^2_{\C}$, and we let $\mc{H}^2_{\R}$ be the real part of this bundle. If we intersect $\mc{F}^1 \mc{H}^2 \cap \mc{H}^2_{\R}$, we obtain precisely the bundle $\mc{H}^{1,1}_{\R}$.

    The key claim is that the natural map $\phi \colon \mc{H}^{1,1}_{\R} \to H^2(X_0, \R)$ is an open immersion. Assuming this, we know $H^2(X_0, \Q) \subseteq H^2(X, \R)$ is dense and $\Lambda_0 \subseteq H^2(X, \Q)$ has smaller rank, so $H^2(X_0, \Q) \setminus \Lambda_0$ is dense. In particular, $\phi^{-1}(H^2(X_0, \Q) \setminus \Lambda_0) \subseteq \mc{H}^{1,1}_{\R}$ is dense. In particular, for any $\alpha$, the set of $t$ for which $\alpha_t$ has type $(1,1)$ is dense.

    We omit the proof of the key claim because it uses the Gauss-Manin connection. The idea is that the differential of $\phi_{\C} \colon \mc{H}^2_{\C} \to H^2(X_0, \C)$ is surjective. This is done by identifying it in terms of the Gauss-Manin connection and the differential of the period map.
\end{proof}

\begin{defn}
    A pair $(X, H)$ of a complex manifold $X$ and line bundle $H$ is called a \textit{polarized complex manifold} if $H$ is ample (in particular this means $X$ is projective). Here, $H$ is called the polarization. A polarized family $(\mc{X}, \mc{H})$ over $B$ has $\mc{H}$ ample on every fiber.
\end{defn}

\begin{defn}
    A family $\mc{X} \to B$ of irreducible holomorphic symplectic manifolds is \textit{(locally) complete} if $B = \mr{Def}(X_0)_h$.
\end{defn}

\begin{exm}
    The family of quartic K3 surfaces is locally complete.
\end{exm}

For higher-dimensional irreducible holomorphic symplectic manifolds, it is in general very hard to construct such locally complete families. The known such constructions are EPW sextics, which are IHS fourfolds of $K3^{[n]}$ type, fourfolds constructed by Debarre-Voisin, and various examples constructed from cubic fourfolds.

\begin{thm}[Matsushita]
    Let $X$ be irreducible holomorphic symplectic of dimension $2n$ and $f \colon X \to B$ be surjective and proper with connected fibers and $0 < \dim B < 2n$ for $B$ a K\"ahler manifold (alternatively it can be a projective variety). Then $\dim B = n$, $B$ is projective, $b_2(B) = \rho(B) = 1$, and $B$ is Fano (alternatively $B$ is $\Q$-factorial and Fano with log-terminal singularities). Moreover, the general fiber is a complex torus and every component of a fiber is a Lagrangian subvariety.
\end{thm}

\begin{defn}
    Let $X$ be holomorphic symplectic. Then a subvariety $V$ is called \textit{Lagrangian} if $\dim V = \frac{1}{2} \dim X$ and for all resolutions $\nu \colon \wt{V} \to V \subseteq X$, $\nu^* \sigma_X = 0$.
\end{defn}

\begin{proof}
    Recall that if $g \colon Y \to Z$ is a surjective morphism of K\"ahler manifolds, then $g^* \colon H^k(Z, \Q) \to H^k(Y, \Q)$ is injective.\footnote{In general this is not true over $\Z$.} This immediately gives $H^{2,0}(B) = 0$, and thus $H^{1,1}(B) = H^2(B)$, which immediately gives a rational K\"ahler class, so $B$ is projective. Let $H$ be a polarization on $B$. If $m = \dim B$, then ${c_1(H)}^m \neq 0$ but ${c_1(H)}^{m+1} = 0$, and because $m < 2n$, we know ${c_1(H)}^{2n} = 0$, and by Fujiki, we have $q(f^* H) = 0$.

    Now let $\omega$ be a K\"ahler form on $X$. Because ${(f^*H)}^m = [X_b]$, if we write $L = f^* H$, then $L^m \wedge \omega_X^{2n-m} \neq 0$. Similarly, for $k \leq m$, $L^k \wedge \omega_X^{2n-k} \neq 0$ because $L^k = f^{-1}(H_1 \cap \cdots \cap H_k)$. We need to prove that $m = n$. Here, we apply Fujiki to $\omega + t L$. This gives us 
    \[ {(\omega + tL)}^{2n} = c {q(\omega + tL)}^n = c {[q(\omega) + t q(\omega, L)]}^n. \]
    If we expand this, we obtain
    \[ \omega^{2n} + \cdots + t^m \omega^{2n-m} L^m = c [{q(\omega)}^n + \cdots + t^n {q(\omega, L)}^n]. \]
    Comparing coefficients in $t$, we observe that the coefficient of $t^n$ is nonzero, and thus $n = m$.

    Now we prove that if $X_b$ is smooth, then it is Lagrangian. Using the Hodge-Riemann bilinear relations, this is the same as showing that
    \[ \int_{X_b} \sigma |_{X_b} \wedge \ol{\sigma} |_{X_b} \wedge \omega_{X}^{n-2} = 0. \]
    If $\omega$ is the restriction of a K\"ahler form on $X$, then our integral simply becomes
    \[ \int_X \sigma \wedge \ol{\sigma} \wedge \omega^{n-2} \wedge L^n. \]
    Applying Fujiki to $\sigma + \ol{\sigma} + t \omega + s L$ and using the fact that $q(\sigma, L) = q(\ol{\sigma}, L) = 0$, we obtain the desired conclusion. To conclude that all fibers are Lagrangian, we use a major result of Koll\'ar, which says that if $h \colon Y \to Z$ is a proper and surjective morphism of smooth projective varieties, then $R^i h_* \omega_Y$ is torsion free for all $i$. In particular, for $f \colon X \to B$, we see that $R^i f_* \mc{O}_X$ is torsion-free, so for $\ol{\sigma} \in H^2(\mc{O}_X)$, this maps to a torsion section $\ol{\sigma} \in H^0(B, R^2 f_* \mc{O}_X)$, which must vanish. Pulling back to $H^2(\wt{X_b}, \mc{O})$, $X_b$ is Lagrangian. By linear algebra reasons, $\dim X_b \leq n$, so $f$ is equidimensional.

    Next we prove that $B$ is Fano. First, it is clear that $H^1(\mc{O}_B) = 0$, so $\Pic B = \NS B$. But then $f^* \colon H^2(B) \subseteq H^{1,1}(X)$. For all $\alpha \in H^2(B)$, $q(f^* \alpha) = 0$, but because $H^{1,1}(X)$ has signature $(1, -)$, we see that $\dim H^2(B) = 1$. Thus $\NS(B) = \Z H$ for some $H$. This implies that $K_B = m H$, and we want to show that $m < 0$. If we consider the inclusion $f^* \Omega^1_B \hookrightarrow \Omega^1_X$, $\Omega^1_X$ is a slope-semistable bundle because $c_1(X) = 0$. In particular, $\mu(\Omega^1_{B}) \leq \mu(\Omega^1_X) = 0$, so $m \leq 0$. To prove that $m \neq 0$, we see that if $m \neq 0$, then $\mc{O}_B = f^* K_B \hookrightarrow \Omega^n_X$, so $H^0(\mc{O}_B) \subseteq H^0(\Omega^n_X)$, and for type reasons, this is impossible.
\end{proof}

\begin{rmk}
    A similar argument shows that if $\alpha \in H^2(X, \Z)$ satisfies $q(\alpha) = 0$, then $\alpha^n \neq 0$ but $\alpha^{n+1} = 0$.
\end{rmk}

There is a result of Verbitsky that the kernel of $q \colon S^2 H^2(X, \Z) \to H^*(X, \Q)$ is given by $\ev{\alpha^{n+1} \mid q(\alpha) = 0}$. Also, when we proved that $B$ is Fano, we used the following result.

\begin{prop}[Beauville]
    Let $X$ be irreducible holomorphic symplectic of dimension $2n$. then $H^0(\Omega^*_X) = \ev{\sigma_X}$, where $\sigma_X$ is the holomorphic form.
\end{prop}

\begin{proof}[Idea of proof]
    We consider the holonomy representation. We know that $\mr{Hol}(g) = \mr{Sp}(n)$, where $g$ is the hyperk\"ahler metric. This has an action on $\Omega^k_{X, x_0}$. By compactness of $X$, holomorphic tensors are parallel. Coversely, parallel forms are holomorphic. But then we consider representations of $\mr{Sp}(n)$ on $\bigwedge^k \C^{2n}$, but then by the representation-theoretic black box there exists a unique invariant if $k$ is even and no invariants if $k$ is odd.
\end{proof}

\begin{rmk}
    There exists a singular definition of irreducible holomorphic symplectic varieties. This uses the algebra of reflexive holomorphic forms.
\end{rmk}

Now we want to see that the smooth fibers of a Lagrangian fibration $X \to B$ are complex tori. But it is clear that $N_{X_b/X} = \mc{O}_{X_b}^n$ and that $N_{X_b/X} = \Omega^1_{X_b}$. Then we need to prove that $a \colon X \to \mr{Alb}(X)$ is an isomorphism, but we consider the sequence
\[ T \to X \to \mr{Alb}(X), \]
and the map $T \to X$ is \'etale, the map $T \to \mr{Alb}(X)$ is surjective, and finally by considering the effect of $a$ on $H_1$, this is an isomorphism.

Alternatively, we may use the holomorphic version of Arnold-Liouville. This describes smooth compact fibers of a completely integrable system. If $M$ is holomorphic symplectic and $h = (h_1, \ldots, h_n) \colon M \to \C^n$ has compact connected fibers and $\dd{h_1}, \ldots, \dd{h_n}$ linearly independent at every point, $h$ is an integrable system if they Poisson commute. This all implies that the smooth fibers are biholomorphic to complex tori. In our case, each vector field $X_{h_i}$ defined by $\dd{h_i} = \sigma_M(X_{h_i}, -)$ acts infinitesimally on $M$ and preserves the level set of $h_{j}$ for all $j$. Thus $X_{h_i}$ acts on each fiber. Because the fibers are compact, the action lifts to an action of $\C$ on the fibers $M_c$. Of course, all of these actions commute, and we obtain an action of $\C^n$ on each $M_c$. The orbits are open, and the fibers are connected, so there exists a unique orbit and the action is transitive. However, the kernel is discrete and has maximal rank, and thus we obtain $M_c = \C^n / \Lambda$.

In our case, the fact that $[X_{h_i}, X_{h_j}] = 0$ is the same thing as our fibers being Lagrangian, and then we can just work locally.

\section{An explicit computation}%
\label{sec:an_explicit_computation}

We will compute the Beauville-Bogomolov-Fujiki form for irreducible holomorphic symplectic varieties $X$ of $\mr{K3}^{[n]}$ type.

\begin{prop}
    There exists an isomorphism of lattices
    \[ (H^2(X, \Z), q) \cong \Lambda_{\mr{K3}} \oplus \ev{-2(n-1)}. \]
    Moreover, the Fujiki constant is given by 
    \[ c_n \coloneqq \frac{(2n)!}{n!2^n}. \]
\end{prop}

\begin{proof}
    By deformation invariance, it is enough to perform this computation for $X = S^{[n]}$, where $S$ is a K3 surface. Recall that
    \[ H^2(S^{[n]}, \Z) = h^*(H^2(S^{(n)}, \Z)) \oplus \Z \delta \]
    where $h$ is the Hilbert-Chow morphism and we have the isomorphism $H^2(S^{(n)}, \Z) \simeq H^2(S, \Z)$. Also, if $E$ is the exceptional divisor of $h$, then $2 \delta = c_1(E)$. Our strategy is the following:
    \begin{enumerate}
        \item We will prove that $q |_{H^2(S, \Z)} = (-,-)_S$ up to a constant.
        \item We will prove that $\delta \perp H^2(S, \Z)$.
        \item We will compute $q(\delta)$.
        \item We will compute the Fujiki constant.
    \end{enumerate}
    Let $\alpha \in H^2(S, \Z)$. We will compute
    \begin{align*}
        \int_{S^{[n]}} h^* \qty(\sum p_i^* \alpha)^{2n} &= \int_{S^{(n)}} \qty(\sum p_i^* \alpha)^{2n} \\
        &= \frac{1}{n!} \int_{S^n} \qty(\sum p_i^* \alpha)^{2n}. \\
        &= \frac{1}{n!} \int_{S^n} (p_1^* \alpha + \cdots + p_n^* \alpha)^{2n} \\
        &= \frac{1}{n!}\sum \binom{2n}{k_1} \binom{2n-k_1}{k_2} \cdots p_1^* \alpha^{k_1} \cdots p_n^* \alpha^{k_n} \\
        &= \frac{1}{n!} \binom{2n}{2} \binom{2n-2}{2} \cdots \binom{4}{2} \prod_{i=1}^n p_i^*(\alpha \wedge \alpha) \\
        &= \frac{(2n)!}{n!2^n} (\alpha, \alpha)_S^n.
    \end{align*}
    Continuing, we have
    \begin{align*}
        \int_S^{[n]} h^* (\alpha)^{2n-1} \cdot E &= c \cdot q(\alpha)^{n-1} q(\alpha, E) \\
        &= \int_E h^*(i(\alpha))^{2n-1} = 0
    \end{align*}
    because $h(E) = \Delta^{2n-2}$, and so the integral vanishes by dimension reasons. Now we have
    \[ q(\alpha) = \lambda (\alpha, \alpha)_S \]
    for some $\lambda \in \Z$, and because $\int \alpha^{2n} = c_n q(\alpha)^n = c_X (\alpha, \alpha)^n$, we have $c_X = c_n \cdot \lambda^n$. Later, we prove that $\lambda = 1$. To compute $q(\delta)$, we compute
    \begin{align*}
        \int_{S^{[n]}} E^2 \wedge h^* \qty(\sum p_i^* \alpha)^{2n-2} &= \int_E E|_E \wedge h^* \alpha^{2n-2} \\
        &= \int_{\Delta} h_* (E|_E) \wedge \qty(\sum p_i^* \alpha)^{2n-2},
        &= -2 \int_{\Delta} \qty(\sum p_i^* \alpha)^{2n-2}.
    \end{align*}
    Note that $h_* (E|_E) = -2$ because $E \to \Delta$ looks locally like the resolution of a quadric cone. Noting that we have a diagram
    \begin{equation*}
    \begin{tikzcd}
        E \ar{r} & \Delta \\
        S \times S^{n-2} \ar{ur}{\eta} \ar{r}{(n-2)!} & S \times S^{(n-2)} \ar{u}{\mr{bir.}},
    \end{tikzcd}
    \end{equation*}
    our integral becomes
    \begin{align*}
        \frac{-2}{(n-2)!} \int_{S^{n-1}} \eta^* \qty(\sum p_i^* \alpha)^{2n-2} &= \frac{-2}{(n-2)!} \qty(\frac{(2n-2)!}{2^{n-1}} 2^2 (\alpha, \alpha)_S^{n-1}).
    \end{align*}
    On the other hand, we can differentiate the Fujiki relation twice, and we know that
    \begin{align*}
        (2n-1) \int_{S^{[n]}} E^2 \wedge h^* \alpha^{2n-2} &= c_X q(\alpha)^{n-1} q(E) \\
        &= c_X \lambda^{n-1} (\alpha, \alpha)_S^{n-1} q(E). \\
        &= \frac{c_n}{\lambda} (\alpha, \alpha)_S^{n-1} q(E).
    \end{align*}
    Comparing the coefficients, we obtain 
    \[ \frac{-(2n-1)!}{(n-2)! 2^{n-4}} = \frac{4}{\lambda} \frac{(2n)!}{n!2^n} q(\delta), \]
    and after cancelling everything, we have
    \[ \frac{-2}{(n-2)!} = \frac{1}{\lambda} \frac{1}{n-1} q(\delta), \]
    and therefore $q(\delta) = -2(n-1)\lambda$.\footnote{This was exhibit $N \gg 0$ that mathematicians are bad at arithmetic.} Finally, we are forced to take $\lambda = n-1$, so we are done.
\end{proof}

In the case when $n=2$, there is a more explicit argument using intersection theory and Segre classes of the Hilbert scheme of $2$ points.

\begin{rmk}
    It is not a coincidence that $q(E) < 0$.
\end{rmk}

\begin{defn}
    If $X$ is irreducible holomorphic symplectic, then $E \subseteq X$ is called \textit{prime exceptional} if it is integral and $q(E) < \infty$.
\end{defn}

\begin{thm}[Markman,\footnote{Fun fact: Markman was on my undergraduate thesis committee.} Druel]
    $E \subseteq X$ is prime exceptional if and only if there exists a birational $f \colon X \dashrightarrow X'$ such that the strict transform $E'$ of $E$ can be contracted.
\end{thm}

\begin{prop}
    Let $f \colon x \dashrightarrow X'$ be a birational map between irreducible holomorphic symplectic varieties. Then the isomorphism
    \[ f^* \simeq H^2(X', \Z) \simeq H^2(X, \Z) \]
    preserves $q_{X'}$ and $q_X$.
\end{prop}

\begin{proof}
    If we consider the graph $\wt{\Gamma}$ of $f$ with projections $p, q$, we can normalize first so that 
    \[ \int_X (\sigma \ol{\sigma})^n = 1, \]
    and then we compute
    \[ \int_{\Gamma} p^*(\sigma \ol{\sigma})^{n-1} E^2 = 0. \]
\end{proof}

Recall that if $Z$ is a smooth projective variety and $\mc{F}$ is a coherent sheaf, then
\[ \chi(Z, \mc{F}) = \int_Z \mr{ch}(\mc{F}) \mr{td}(Z) \]
by Hirzebruch-Riemann-Roch. If $\mc{F} = L$ is a line bundle, then we have
\[ \chi(Z, L) = \sum \int_Z \frac{c_1(L)^i}{i!} \mr{td}_{\dim Z - i}(Z). \]

\begin{prop}
    Let $X$ be irreducible holomorphic symplectic. Then there exist $q_i \in \Q$ depending only on the deformation class of $X$ such that
    \[ \chi(X, L) = \sum_{i=0}^n q_i q(L)^i. \]
\end{prop}

\begin{cor}
    For all isotropic $L$, we have $\chi(X, L) = n+1$, where $X$ has dimension $2n$.
\end{cor}

To prove the proposition, we need to prove the following Fujiki-like result.
\begin{prop}
    Let $\beta \in H^{4\ell}(X, \R)$ be a class that stays of type $(2\ell, 2\ell)$ for all small deformations of $X$. Then there exists $c_{\beta} \in \R$ such that $\int \beta \wedge \alpha^{2n-2\ell} = c_{\beta} q(\alpha)^{n-\ell}$ for all $\alpha \in H^2(X, \Z)$.
\end{prop}
Proof of this is the same as the proof of Fujiki.

\chapter{Moduli spaces}%
\label{cha:moduli_spaces}

Our goal is to eventually prove the following theorem:
\begin{thm}[Mukai, O'Grady, Yoshioka, Huybrechts]
    Let $S$ be a projective K3 surface. Let $v \in H^*_{\mr{alg}}(S, \Z) = H^0 \oplus \NS(S) \oplus H^4$ be a primitive Mukai vector. Let $H$ be a general polarization and $M_{v, H}$ be the moduli space of $H$-semistable coherent sheaves on $S$ with Mukai vector 
    \[ v(F) \coloneqq \mr{ch}(F) \sqrt{\mr{td}(S)} = \qty(r, c_1, \frac{c_1^2}{2} - c_2 + r) \in H^*_{\mr{alg}}(S, \Z). \]
    If $v^2 \geq -2$, then $M_{v, H}$ is a smooth projective irreducible holomorphic symplectic manifold of dimension $v^2 + 2$ deformation equivalent to $S^{[n]}$, where $n = \frac{v^2}{2} +1$. Moreover, if $v^2 \geq 2$, then there is a canonical isomorphism
    \[ H^2(M_{v, H}, \Z) \simeq v^{\perp} \subseteq H^*(S, \Z) \]
    which is an isomorphism of Hodge structures.
\end{thm}

We'll have to understand what a general polarization is, and here semistability will mean Gieseker stability, and of course we'll need to prove smoothness, projectivity, and symplecticness. OF course just being nonempty and irreducible is a nontrivial result. This result also holds more generally for Bridgeland stability, which we will discuss in this course.

\section{Strategy for deformation equivalence}%
\label{sec:strategy_for_deformation_equivalence}

Here is the strategy for deformation equivalence. Up to deforming $S$ and considering birational maps among moduli spaces with different Mukai vectors, we can relate $M_{v, H}(S)$ to a Hilbert scheme. Because Yoshioka is notoriously hard to read, we apparently will sketch this. The missing piece is the following result:

\begin{thm}[Huybrechts]
    Let $f \colon X \dashrightarrow X'$ be a bimeromorphic map of irreducible holomorphic symplectic manifolds. Then $X$ and $X'$ are deformation equivalent. This means that there exists a diagram
    \begin{equation*}
    \begin{tikzcd}
        \mc{X} \ar[dashrightarrow]{rr}{F} \ar{dr} & & \mc{X}' \ar{dl} \\
        & \Delta
    \end{tikzcd}
    \end{equation*}
    such that $F$ is an isomorphism after restricting to $\Delta^*$ and $(\Gamma_F)_0 = \Gamma_f$.
\end{thm}

This result can be found in a series of papers by Huybrechts titled:
\begin{itemize}
    \item \textit{Birational symplectic manifolds and their deformations};
    \item \textit{Compact hyperk\"ahler manifolds: basic results} and erratum;
    \item \textit{K\"ahler cone...} 
\end{itemize}

Fix a lattice $\Lambda$ of signature $(3, \rho-3)$, where $\rho = \operatorname{rk} \Lambda$. Then the moduli space $\mc{M}_{\Lambda}$ of $\Lambda$-marked hyperk\"ahler varieties parameterizes pairs $(X, \varphi)$ where $X$ is irreducible holomorphic symplectic and $\varphi \colon H^2(X, \Z) \xrightarrow{\sim} \Lambda$ is an isomorphism of lattices. Here, $(X, \varphi) \simeq (X', \varphi')$ if there exists $f \colon X \to X'$ such that $\varphi \circ f^* = \varphi'$.

This is apparently some nonseparated complex manifold because of local Torelli. Here, we obtain a system of charts on $M_{\Lambda}$ that looks like
\begin{equation*}
\begin{tikzcd}
    \mr{Def}(X) \supseteq U \ar{r} \ar{dr}{\wp_X} & \mc{M}_{\Lambda} & V \subseteq \mr{Def}(X') \ar{d}{\wp_{X'}} \ar[hookrightarrow]{l} \\
    & \Omega_X \subseteq \P H^2(X, \Z) \ar{r}{\varphi^{-1} \circ \varphi'} & \Omega_{X'} \subseteq \P H^2(X, \Z).
\end{tikzcd}
\end{equation*}

\begin{lem}[Tautological lemma]
    Two points $(X, \varphi), (X', \varphi') \in \mc{M}_{\Lambda}$ are non-separated if and only if there exist families $\mc{X}, \mc{X}' \to \Delta$ and $V \subseteq \Delta$ such that $0 \in \ol{V}$ and $\mc{X}, \mc{X}'$ are isomorphic over $V$ with central fibers $X, X'$.
\end{lem}

This result tells us that $X, X'$ have the same period map because the diagram
\begin{equation*}
\begin{tikzcd}
    & \Lambda \\
    H^2(X, \Z) \ar{ur}{\varphi} & & H^2(X', \Z) \ar{ul}{\varphi'} \\
    H^2(X_t, \Z) \ar{u}{\sim} \ar{rr}{\sim} & & H^2(X_t', \Z) \ar{u}{\sim}.
\end{tikzcd}
\end{equation*}

\begin{prop}[Huybrechts, Burns-Rapoport]
    The tautological lemma implies that there exist families
    \begin{equation*}
    \begin{tikzcd}
        \mc{X} \ar[dashrightarrow]{rr} \ar{dr} & & \mc{X}' \ar{dl} \\
        & \Delta
    \end{tikzcd}
    \end{equation*}
    that are isomorphic over $\Delta^*$. Now define $\Gamma_{F^*} \subseteq \mc{X} \times_{\Delta^*} \mc{X}'$. The key input is a volume estimate, which says that the volume of $\Gamma_{t}$ is bounded. By a result of Bishop, there exists a limit cycle $\Gamma_0 \subseteq X_0 \times X_0'$.
\end{prop}

Let $\omega_t, \omega_t'$ be K\"ahler forms on $\mc{X}_t, \mc{X}_t'$ varying continuously with $t$ and consider the K\"ahler form $p_t^* \omega_t + p_t'^* \omega_t'$. Then we see that
\begin{align*}
    \mr{vol}(\Gamma_{t_i}) &= \int_{\Gamma_{t_i}} (p_t^* \omega_t + p_t' \omega_t')^{2n} \\ 
    &= \int_{X_t} (\omega_t + f_t^* \omega'_t)^{2n} \\
    &= \int_{X_0} (\eta_t(\omega_t) + (\varphi^{-1} \circ \varphi')(\omega_t'))^{2n}. 
\end{align*}
This implies that
\[ \int_{X_0} (\omega_0 + (\varphi \circ \varphi')\omega_0')^{2n} < \infty, \]
and therefore that the limit cycle $\Gamma_0 \subseteq X \times X'$ exists. We want to show that there exists $\Gamma_f \subseteq \Gamma_0$ inducing a birational map. Because the correspondences $\Gamma_t^*$ are isomorphic, so is $\Gamma_0^*$. This also tells us that $\Gamma_0^* [X'] = [X]$, and so $\Gamma_0$ is dominant of degree $1$ on both factors. We want to prove that there exists a component that is dominant of degree $1$ on both components. We need to exclude the case where
\[ \Gamma_0 = Z + Z' + \sum Y_i, \]
where $Z \to X$ is dominant, $Z' \to X'$ is dominant, and $Y_i$ are not dominant onto either factor. Now we have
\begin{align*}
    \int_X (\sigma \ol{\sigma})^n &= \int_X \Gamma_0^* (\sigma') \sigma^{n-1} \ol{\sigma}^n \\
    &= \int_X p_* (p'^* (\sigma') \smile \Gamma_0) \sigma^{n-1} \ol{\sigma}^n \\
    &= \int_Z p'^* (\sigma') p^*(\sigma^{n-1} \ol{\sigma}^n) + \int_{Z'} p'^*(\sigma') p^*(\sigma^{n-1} \ol{\sigma}^n)+ \sum \int_{Y_i} p'^*(\sigma') p^*(\sigma^{n-1} \ol{\sigma}^n).
\end{align*}
By type reasons, the integrand vanishes on $Z', Y_i$ and thus we only need to study the integral on $Z$. But now $p'^*(\sigma') |_Z = 0$. This is because it comes from a smaller-dimensional subset of $X'$ which must be degenerate, but we know that holomorphic forms are a birational invariant, so this must be degenerate.

\begin{rmk}
    In general, we have $\Gamma_f \subsetneq \Gamma_0$. Sometimes, even their action on $H^2$ is different.
\end{rmk}

\begin{proof}[Proof of Huybrechts for projective case]
    Let $f \colon \dashrightarrow X'$ be birational. We know that $X, X'$ have $L, L'$ such that $q_X(L) = q_{X'}(L')$. Also, we know that $H^0(X, L) = H^0(X', L')$. Now we use Riemann-Roch, and so there exists $a_i, a_i' \in \Q$ such that
    \[ \chi(X, L) = \sum a_i q(L)^i \qquad \chi(X', L') = \sum a_i' q(L')^i. \]
    Without loss of generality, we can assume that for $t \gg 0$, $\sum a_i t^i \geq \sum a_i' t^i$. For $n \gg 0$, this means that 
    \[ \chi(X, L^n) \geq \chi(X', L'^n). \]
    Now choose $L'$ ample. If $f$ is not an isomorphism, then $L$ will not be nef (but it will be big). We know that
    \[ X' = \Proj \qty(\bigoplus H^0(X', L'^n)) = \Proj \qty(\bigoplus H^0(X, L^n)) = X, \]
    and so our map $f \colon X \dashrightarrow X'$ will be induced by $L$ (which is not nef and thus has a base locus, but is big and thus has many global sections).

    Let $(\mc{X}, \mc{L})$ be a deformation of $(X, L)$ such that for a very general $t \in \Delta$, $\rho(X_t) = 1$. Now we use a theorem from the erratum of the second Huybrechts paper, which says that if $X$ is irreducible holomorphic symplectic, then $X$ is projective if and only if there exists $L$ such that $q(L) > 0$. Here, we know that $q(L') = q(L) = q(\mc{L}_t) > 0$, and thus $\mc{L}_t$ is ample on $X_t$ for very general $t$. This is an open condition, so we may assume $\mc{L}_t$ is ample for $t \in \Delta^*$. Thus for all $n > 0$, $H^i(X_t, L_t^n) = 0$.

    The next claim is that $\pi_* \mc{L}^n$ is locally free for $n \gg 0$. Because the base is reduced, it is enough to prove constant fiber dimension. It is enough to show that $t \mapsto h^0(X_t, \mc{L}_t^n)$ is constant for $n \gg 0$. But now we have
    \[ h^0(X_t, \mc{L}_t^n) = \chi(X_t, \mc{L}_t^n) = \chi(X, L^n) \geq \chi(X', L'^n) = h^0(X', L'^n) = h^0(X, L^n). \]
    By upper semicontinuity, everything here must be equal. Taking the relative Proj construction and setting
    \[ \mc{X}' \coloneqq \Proj_{\Delta} \qty(\bigoplus \pi_* \mc{L}^n), \]
    we are done.
\end{proof}

\section{Quot schemes}%
\label{sec:quot_schemes}

Let $k$ be an infinite field with $X \subseteq \P^n_k$. Alternatively, we may consider a projective morphism $X \to B$ with $B$ quasiprojective over $k$. Fix $p(t) \in \Q[t]$. We would like to construct a proper (projective?) separated scheme of finite type over $k$ whose closed points are in bijection with isomorphism classes of sheaves on $X$ whose Hilbert polynomial is $p(t)$. We would also like $M$ to satisfy some universal property, namely representing some functor.

\begin{defn}
    A family $\qty{F_{\alpha}}_{\alpha \in A}$ of (isomorphism classes of) coherent sheaves on $X$ is called \textit{bounded} if there exists a scheme $S$ fo finite type over $k$ and a coherent sheaf $\mc{F}$ on $S \times X$ such that 
    \[ \qty{F_{\alpha}} \subseteq \qty{\mc{F}_s \mid s \in S}. \]
\end{defn}

Unfortunately, fixing the Hilbert polynomial is not in general sufficient to have a bounded family.

\begin{exm}
    On $\P^1_k$ consider the constant Hilbert polynomial $p(t) \equiv t + 2$. We will consider locally free sheaves on $\P^1$ with this Hilbert polynomial. However, we have sheaves $F_a = \mc{O}(a) \oplus \mc{O}(-a)$. But then $h^0(F_a) = a+1$. This family cannot be bounded because for any scheme $S$ and any $m$ the subscheme
    \[ S_m = \qty{F_s \mid h^0(F_s) \geq m} \]
    is closed, and thus $S$ cannot be of finite type.
\end{exm}

\begin{exm}
    Let $f \colon C \to \P^1$ be a hyperelliptic curve with $g(C) \geq 2$. Then $\Pic^{g-1}(C)$ contains a theta divisor $\Theta$, but here we have $h^0(L) = 0$ generically and $h^0(L) > 0$ on $\Theta$. Then there exists a universal line bundle $\mc{L}$ on $\Pic^{g-1}(C) \times C$. Then we have a family of line bundles $F_L = f_* L$ on $\P^1$ which are $\mc{O}(a) \oplus \mc{O}(-a-2)$ for $a \geq -1$. If $L \notin \Theta$, we have $F_L = \mc{O}(-1) \oplus \mc{O}(-1)$. However, if $L \in \Theta$ and $h^0(L) = 1$, we have $F_L = \mc{O} \oplus \mc{O}(-2)$. This gives a nonseparated family of sheaves.
\end{exm}

To achieve boundedness, we need to both fix the Hilbert polynomial and consider sheaves that are quotients of a fixed sheaf $\mc{O}_X^N$. Our main technical tool is the following:

\begin{defn}
    A coherent sheaf $F$ on $\P^n$ is \textit{Castelnuovo-Mumford regular} if for all $i > 0$ we have $H^i(F(-i)) = 0$.
\end{defn}

\begin{thm}[Castelnuovo, Mumford]
    Let $F$ be Castelnuovo-Mumford regular. Then
    \begin{enumerate}
        \item $F$ is globally generated.
        \item $F(\ell)$ is Castelnuovo-Mumford regular for all $\ell \geq 0$.
        \item $H^0(F) \otimes H^0(\mc{O}_X(\ell)) \twoheadrightarrow H^0(F(\ell))$.
    \end{enumerate}
\end{thm}

This can be proved by inducting on the support of $F$ and using the exact sequence
\[ 0 \to F(-1) \to F \to F|_H \to 0. \]
Some references are of course \textit{FGA Explained} and Grothendieck's \textit{Techniques des construction et th\'eor\`eme d'existence en geometrie algebrique IV: les schemas de Hilbert}. Also there is a reference due to Mumford.

\begin{thm}
    Fix $(X, H)$ and some $N \geq 0$. Also fix $p(t) \in \Q[t]$ the Hilbert polynomial. Then there exists an integer $\ell_0$ sich that for all $F$ a quotient of $\mc{O}_X^n \twoheadrightarrow F$, then
    \begin{enumerate}
        \item $H^0(\mc{O}_X^N(\ell_0)) \twoheadrightarrow H^0(F(\ell_0))$ and $H^i(F(\ell_0)) = 0$ for all $i > 0$.
        \item $G(\ell_0)$ is globally generated and
            \[ H^0(G(\ell_0)) \otimes H^0(\mc{O}_X(s)) \twoheadrightarrow H^0(G (\ell_0 + s)) \]
            for all $s \geq 0$, where $G$ is the kernel of $\mc{O}_X^N \to F$.
        \item $H^i(G(\ell_0 + s)) = 0$ for all $s \geq 0$ and $i > 0$.
    \end{enumerate}
\end{thm}

We now consider the functor $\ms{Quot}_X^{N,p}$ that takes a scheme $B$ to the set of isomorphism classes of coherent sheaves $\mc{F}$ on $X \times B$ flat over $B$ such that $p_{\mc{F}_b}(t) = p(t)$ and $\mc{O}_{X \times B}^N \twoheadrightarrow \mc{F}$. We will not need this, but it is possible to identify quotients if their kernels are the same as subsheaves of $\mc{O}_{X \times B}^{N}$.

\begin{thm}[Grothendieck]
    The functor $\ms{Quot}_X^{N, p}$ is represented by a projective scheme $\mr{Quot}_X^{N, p}$. Of course this means there is a universal coherent sheaf $\mc{F}$ on the Quot scheme with a surjection $\mc{O}^N \twoheadrightarrow \mc{F}$.
\end{thm}

\begin{exm}
    Let $r \leq n$ be integers. Then $\mr{Gr}(r, n)$ is the Quot scheme for $X = \mr{pt}$ if we consider quotients $k^n \twoheadrightarrow W$ to vector spaces of dimension $n-r$.
\end{exm}

Of course, we use the Grassmannian to construct the Quot scheme, so we will construct the Grassmannian by hand. If $[V] \in \mr{Gr}(r, n)$, we can choose a splitting $k^n = V \oplus W$. Then the open affines are given by $\Hom(V, W)$ given by associating a map $\varphi$ to its graph $\Gamma_{\varphi}$. There is also the tautological sheaf $\mc{S}$ and exact sequence
\[ 0 \to \mc{S} \subseteq \mc{O}_{\mr{Gr}}^n \to \mc{Q} \to 0, \]
and of course we know that $H^0(\det \mc{S}^{\vee}) = \bigwedge^{\vee} k^n$.

Our strategy is to define the Quot scheme as a set, then slap a scheme structure on it and prove that it is locally closed, and finally we prove that the universal family exists.

Let $\ell_0$ be as in the previous theorem, $n = \dim H^0(\mc{O}_X^N(\ell_0))$, and $r = p(\ell_0)$. We will consider $\mr{Gr}(n-r, r)$. Given $\mc{O}_X^N \twoheadrightarrow F$ with $p_F(t) = p(t)$, let $G = \ker(\mc{O}_X^N \twoheadrightarrow F)$. Then we know
\[ H^0(G(\ell_0)) \subseteq H^0(\mc{O}_X^N(\ell_0)) \to H^0(F(\ell_0)), \]
and because the higher cohomology of $G(\ell_0)$ vanishes, the last arrow is surjective. We know that the middle term has dimension $n$ and the last term has dimension $r$, so $K_{\ell} \coloneqq H^0(G(\ell_0))$ has dimension $n-r$.

First we need to prove that this assignment is injective because $K_{\ell_0}$ determines $G$ and hence $F$. Indeed, by the theorem, we have a series of surjections
\[ K_{\ell_0} \otimes H^0(\mc{O}_X^N(s)) \twoheadrightarrow H^0(G(\ell_0 + s)) \subseteq H^0(\mc{O}_X^N(\ell_0 + s)). \]
Thus it determines $G$ as a graded module, and so in fact $G$ is determined as a coherent sheaf. Alternatively, we can consider $K_{\ell_0} \otimes \mc{O}_X \to \mc{O}_X^N(\ell_0)$, and this surjects onto $G$. We now define the set
\[ Q = \qty{K \mid K \otimes H^0(\mc{O}_X^N(s)) \xrightarrow{\varphi_{s,k}} H^0(\mc{O}_X^N(\ell_0 + s)) \text{ has rank } p_{\mc{O}_X^N}(\ell_0 + s) - p(\ell_0 + s)}. \]
We know that $\varphi_{s, k}$ is the fiber of a map of vector bundles. We know $\mc{S} \subseteq H^0(\mc{O}_X^N(\ell_0)) \otimes \mc{O}_{\mr{Gr}}$ and so we have a map
\[ \mc{S} \otimes H^0(\mc{O}_X(s)) \xrightarrow{\varphi_s} H^0(\mc{O}_X^N(\ell_0 + s)) \otimes \mc{O}_{\mr{gr}}. \]
Now the locus $Q_s$ where the rank of $\varphi_s$ is $p_{\mc{O}_X^N}(\ell_0 + s) - p(\ell_0 + s)$ is localy closed.

\begin{thm}
    Fix $X, H, N, p, \ell_0$. Then there exist finitely many Hilbert polynomials of quotients $\mc{O}_X^N \twoheadrightarrow F$ such that $G(\ell_0)$ is globally generated by $n-r$ global sections.
\end{thm}

\begin{proof}
    Consider $\mr{Gr}(n-r, n) \eqqcolon \mr{Gr}$. We of course have the tautological sequence and the two projections on $\mr{Gr} \times X$. Then we have
    \[p_1^* \mc{S} \subseteq H^0(\mc{O}_X^N(\ell_0)) \otimes \mc{O}_{\mr{Gr} \times X} \to p_2^* \mc{O}_X^N(\ell_0) \twoheadrightarrow \mc{F}(\ell_0). \]
    Call the kernel of the last morphism $\mc{G}(\ell_0)$. By the flattening stratification and Noetherian induction, we have only finitely many possible Hilbert polynomials.
\end{proof}

Now let $\mc{F}$ be as in the proof of the theorem and $Q_1, \ldots, Q_m$ be the possible Hilbert polynomials. Then let $\ell_0'$ be the maximum of the $\ell_0(Q_i)$. For all $s$ such that $\ell_0 + s \geq \ell_0'$, we know that
\[ H^i(\mc{F}_k(\ell_0 + s)) = 0 \]
for all $i > 0$. Thus $p_{\mc{F}_k}(\ell_0 + s) = h^0(\mc{F}_k(\ell_0 + s))$. Also, we know that $H^i(G_k(\ell_0 + s)) = 0$, and thus 
\[ H^0(\mc{O}_X^N(\ell_0 + s)) \twoheadrightarrow H^0(\mc{F}_k(\ell_0 + s)) \]
is surjective. Finally, using the original theorem, the map
\[ H^0(G_k(\ell_0)) \otimes H^0(\mc{O}_X^N(s)) \to H^0(G_k(\ell_0 + s)) \]
is surjective.

Let $\gamma \in \mathbb{N}$ be large enough such that if $Q_i(t) = p(t)$ for $\gamma$ values, then $Q_i(t) \equiv p(t)$. We will consider $K \in \bigcap_{s = 1}^{\gamma} Q_{s + \ell_0} = Q$. Here, we observe that
\[ \varphi_s \colon H^0(G_k(\ell_0)) \otimes H^0(\mc{O}_X^N(s+\ell_0)) \twoheadrightarrow H^0(G_k(s + \ell_0)) \]
has the correct rank for these $\gamma$ values. Therefore $\mc{F}_K$ has the correct Hilbert polynomial, and so we are done.

It remains to prove that $\mr{Quot}$ represents the functor and that it is a closed subscheme of $\mr{Gr}$. First, we will construct a universal family on $\mr{Quot}$. Recall on the Grassmannian that we have the tautological sequence
\[ \mc{S} \hookrightarrow H^0(\mc{O}_X^N(\ell_0)) \otimes \mc{O}_{\mr{Gr} \times X} \to 1. \]
Now restricting to $\mr{Quot}$ and pulling back to $\mr{Quot} \times X$, we have
\begin{equation}\label{eqn:star}
\begin{tikzcd}
    p_1^* \mc{S} \ar[hookrightarrow]{r} \ar{drr} & H^0(\mc{O}_X^N(\ell_0)) \otimes \mc{O}_{\mr{Quot} \times X} \ar{r} & p_2^* \mc{O}_X^N(\ell_0) \ar[twoheadrightarrow]{r} & \mc{F}(\ell_0) \\
    & & \mc{G}(\ell_0) \ar[hookrightarrow]{u}.
\end{tikzcd}
\end{equation}
Now observe that $\mc{F}$ is flat over $\mr{Quot}$ if and only if $p_{1*} \mc{F}(\ell)$ is locally free for $\ell \gg 0$. We will see that
\[ \mc{G}(\ell_0) \hookrightarrow p_2^* \mc{O}_X^N(\ell_0) \twoheadrightarrow \mc{F}(\ell_0) \]
gives the universal quotient on $\mr{Quot} \times X$. But the flatness condition follows from the vanishing result for $\ell \geq \ell_0$. Using the diagram (\ref{eqn:star}), note that
\[ p_{2*} (\mc{G}(\ell_0)) \subseteq  \otimes \mc{O}_{\mr{Quot} \times X}(\ell_0) \to \mc{F}(\ell_0) \]
recovers
\[ \mc{S}|_{\mr{Quot}} \subseteq H^0(\mc{O}_X^N(\ell_0)) \otimes \mc{O}_{\mr{Quot}} \to \mc{Q} |_{\mr{Quot}} \to 0. \]

We will now check the universal property of our family. Consider a quotient $\mc{O}_{X \times B}^N \twoheadrightarrow \mc{F}_B$. We want $\varphi \colon B \to \mr{Quot}$ such that
\[ \varphi^* (\mc{G} \subseteq \mc{O}_X^N \twoheadrightarrow \mc{F}) = (\mc{G}_B \subseteq \mc{O}_{B \times X}^N \to \mc{F}_B). \]
But here we push forward $p_{2*}(\mc{O}_{X \times B}^N \to \mc{F}_B) \otimes \mc{O}(\ell_0)$ and we now have
\[ p_{2*} (\mc{G}_B(\ell_0)) \hookrightarrow H^0(\mc{O}_X^N(\ell_0)) \otimes \mc{O}_B \to p_{2*} \mc{F}_B(\ell_0) \to 0. \]
But now the first factor is locally free of the correct rank, and so we have a map $B \to \mr{Gr}$ factoring through $\mr{Quot}$.

Finally we prove that $\mr{Quot} \subseteq \mr{Gr}$ is a closed immersion. Here, we will use the valuative criterion. Let $R$ be a discrete valuation ring over $k$ with fraction field $K$. Let $U = \Spec K \subseteq \Spec R = C$. We want to fill in the diagram
\begin{equation*}
\begin{tikzcd}
    U \ar{r}{\varphi} & \mr{Quot} \\
    C \ar[hookrightarrow]{u} \ar{r}{\psi} \ar[dashrightarrow]{ur} & \mr{Gr} \ar[hookrightarrow]{u}.
\end{tikzcd}
\end{equation*}
On $U \times X$ we have an exact sequence
\[ \mc{G}_U \coloneqq \varphi^* \mc{G} \hookrightarrow \mc{O}_{U \times X} \to \varphi^* \mc{F}, \]
and on $C$ we have the exact sequence
\begin{equation*}
\begin{tikzcd}
    \psi^* \mc{S} \ar[hookrightarrow]{r} \ar[twoheadrightarrow]{drr} & H^0(\mc{O}_X^N(\ell_0)) \otimes \mc{O}_{C \times X} \ar{r} & p_2^* \mc{O}_X^N(\ell_0) \ar[twoheadrightarrow]{r} & \wt{\mc{F}}_C(\ell_0) \\
    & & \mc{G}_C(\ell_0) \ar[hookrightarrow]{u}.
\end{tikzcd}
\end{equation*}
We know $\wt{\mc{F}}_C$ is flat over $C$ if and only if $t \colon \wt{\mc{F}}_C \to \wt{\mc{F}}_C$ is injective. Set $\mc{F}_C \coloneqq \wt{\mc{F}}_C / [[t^{\infty}]]$. Then we have 
\begin{equation*}
\begin{tikzcd}
    0 \ar{r} & \mc{G}_C \ar{r} \ar{d} & \mc{O}_{C \times X}^N \ar{r} \ar[equals]{d} & \wt{\mc{F}}_C \ar{d} \ar{r} & 0 \\
    0 \ar{r} & \mc{G}_C' \ar{r} & \mc{O}_{C \times X}^N \ar{r} & \mc{F}_C,
\end{tikzcd}
\end{equation*}
But now $\mc{F}_C$ is flat.

\begin{rmk}
    We can replace $\mc{O}_X^N$ with any sheaf $\mc{H}$ on $X$ and consider quotients $\mc{H} \twoheadrightarrow \mc{F}$. Then $\mr{Quot}_{\mc{H}, p}^X \subseteq \mr{Quot}_{\mc{O}_X^N, p}^X$ is a closed subscheme. Here, we may assume that $\mc{H}$ is a quotient of $\mc{O}_X^N$ for some large enough $N$. This also shows that $\mr{Hilb}_X \subseteq \mr{Hilb}_{\P^n}$ is a closed subscheme for $X \subseteq \P^n$.
\end{rmk}

\begin{rmk}
    We may also replace $X/k$ with $X \to S$ a projective morphism with $S$ quasi-projective. Then $\mr{Quot}^{X/S} \to S$ is a projective morphism. Also, we know $(\mr{Quot}^{X/S})_s = \mr{Quot}^{X_s}$.
\end{rmk}

Wew will now compute the tangent space of the Quot scheme. Here we are following chapter 6 of \textit{FGA explained}. We want to compute
\[ T_{[\mc{O}_X^N \twoheadrightarrow F]} \mr{Quot} = \ms{Quot}_{\mc{O}_X^N \twoheadrightarrow F}(\Spec k[\ep]). \]
\begin{thm}
    The deformation functor $\mr{Def}_{\mc{O}_X^N \twoheadrightarrow F}$ has a tangent-obstruction theory given by
    \[ T^1_{\mr{Def}} = \Hom_X(G, F), \qquad T^2 = \Ext_X^1(G, F), \]
    where $G = \ker(\mc{O}_X^N \twoheadrightarrow F)$.
\end{thm}

We will prove the tangent part of the theorem. Consider the exact sequence
\[ 0 \to (\ep) \to k[\ep] \to k \to 0. \]
We have the diagram
\begin{equation*}
\begin{tikzcd}
    & G(\ep) \ar[hookrightarrow]{d} \ar{dr}{\alpha} & ? & G \ar[hookrightarrow]{d} \\
    0 \ar{r} & \mc{O}_X^N(\ep) \ar{r} \ar[twoheadrightarrow]{d} & \mc{O}_{X[\ep]}^N \ar{r} \ar{dr}{\beta} & \mc{O}_X^N \ar{d} \\
    & F(\ep) \ar{d} & ? & F \ar{d} \\
    & 0 & & 0.
\end{tikzcd}
\end{equation*}
We want $G', F'$ flat over $k[\ep]$ filling in the diagram. Remember flatness is equivalent to injectivity of multiplication by $\ep$. First we define $e \in \Ext_X^1(G, F)$. If $e = 0$, then we can fill in the diagram.

\begin{rmk}
    Let $A \to B \to C$ be a split exact sequence. Then the splittings are the same as $\Hom(C, A)$. 
\end{rmk}

Therefore, the possible ways to fill the diagram are the same as splittings of $e$, which are the same as $\Hom(G, F)$. We know that $\Im(\alpha) \subseteq \ker(\beta) \to G$. In particular, we have an exact sequence
\[ 0 \to F(\ep) \to \ker(\beta) / \Im(\alpha) \to G \to 0. \]
We will define $e \in \Ext_X^1(G, F)$ to be the class of this extension. We will omit checking that multiplication by $\ep$ is zero on this exact sequence.

If $e = 0$, then choose some splitting $\xi \colon G \to \ker(\beta) / \Im(\alpha)$. We will construct a filling of the diagram. We will define $G'$ by the diagram
\begin{equation*}
\begin{tikzcd}
    G' \ar[hookrightarrow]{r} \ar[twoheadrightarrow]{d} & \ker(\beta) \ar[hookrightarrow]{r} \ar[twoheadrightarrow]{d} & \mc{O}_{X[\ep]}^N \\
    G \ar[hookrightarrow]{r}{\xi} & \ker(\beta) / \Im(\alpha).
\end{tikzcd}
\end{equation*}
We will define $G'$ to be the preimage of $\xi(G)$. This gives us $G(\ep) \hookrightarrow G' \twoheadrightarrow G$ and of course $F'$ with $F(\ep) \to F' \to F$. Checking flatness is omitted.

Conversely, given $G', F'$ filling the diagram, we know that $G = G' / \Im(\alpha) \subseteq \ker(\beta)/\Im(\alpha)$, and this splits the surjection $\ker(\beta) / \Im(\alpha) \twoheadrightarrow G$. It is not hard to check that these two constructions are inverses of each other.

\section{Semistable sheaves}%
\label{sec:semistable_sheaves}

Given $p(t) \in \Q[t]$, we want to know which sheaves $F$ with $p_F = p$ are quotients of a given sheaf.
\begin{rmk}
    A family $\qty{F_{\alpha}}$ of sheaves is bounded if and only if there exists a sheaf $\mc{H}$ such that $\mc{H} \twoheadrightarrow F_{\alpha}$ and there are only finitely many Hilbert polynomials.
\end{rmk}

Let $X \subseteq \P^n$ be projective and $F$ coherent with $\Supp F \subseteq X$.
\begin{defn}
    $F$ is \textit{pure of dimension $d$} if for all nonzero subsheaves $E \subseteq F$, $\dim \Supp E = d$.
\end{defn}
Note that this is equivalent to saying that all associated points of $F$ have dimension $d$.

\begin{exm}
    If $Z \subseteq X$ is a subscheme of dimension $0$, then $\mc{O}_Z$ has pure dimension $0$. Also, if $j \colon C \to X$ is a curve and $L$ is a line bundle on $C$, then $j_* L$ is pure of dimension $1$.
\end{exm}

\begin{exm}
    Let $X$ be integral and $\dim X = d$. Then $F$ is pure of dimension $d$ if and only if $F$ is torsion-free.
\end{exm}

Now define the \textit{torsion filtration} of $F$ as
\[ 0 \subseteq T_0(F) \subseteq T_1(F) \subseteq \cdots \subseteq T_d(F) = F \]
where $T_i(F)$ is the maximal subsheaf of dimension at most $i$. Note that $F/T_{d-1}(F)$ is pure of dimension $d$.

Let $(X, H)$ be a polarization with $X$ projective over $k = \ol{k}$. If $E$ is a coherent sheaf, recall that 
\[ p_E(t) = \chi(E(tH)) = \sum_{i=0}^{\dim E} \alpha_i(E) \frac{t^i}{2!} = \sum \chi(E|_{\bigcap_{j \leq i} H_j}) \binom{t+i-1}{i}, \]
where $H_i \in \abs{H}$ are generic. We know $\alpha_0 = \chi(E)$. If $d = \dim E$, then $\alpha_d(E) > 0$. 

\begin{rmk}
    If $E = \mc{O}_X$, then $\alpha_d(\mc{O}_X) = H^d$ is the degree of $X$ with respect to $H$. We may now define the generalized rank of $E$ by $\frac{\alpha_d(E)}{\alpha_d(\mc{O}_X)}$. If $X$ is not irreducible, the generalized rank of $E$ may depend on the polarization.
\end{rmk}

\begin{rmk}
    Using Grothendieck-Riemann-Roch, if $X$ is integral, then
    \[ \frac{\alpha_d(E)}{\alpha_d(\mc{O}_X)} = \mr{rk}(E). \]
\end{rmk}

\begin{defn}
    The reduced Hilbert polynomial $p(E)$ of $E$ is defined by
    \[ p(E) \coloneqq \frac{p_E(t)}{\alpha_d(E)}. \]
\end{defn}
If we consider the lexicographic ordering on polynomials, we can make the following definition:
\begin{defn}
    A coherent sheaf $E$ on $X$ of dimension $\dim E = d$ is called \textit{(semi)stable} if $E$ is pure and for all $F \subsetneq E$, $p(F) < (\leq) p(E)$.
\end{defn}

\begin{rmk}
    IT is enough to check semistability for so-called ``saturated'' subsheaves, which are those $F \subseteq E$ with $E/F$ pure of dimension $d$. Indeed, if $F \subseteq E \to G$ and $G$ is not pure, we can consider $T = T_{d-1}(G)$ and then $G/T$ is pure. We can then replace $F$ with $F'$, called the saturation of $F$ in $E$. Because saturating $F$ increases the Hilbert polynomial because $\alpha_d(F) = \alpha_d(F')$ and $\alpha_{d-1}(F) \leq \alpha_{d-1}(F')$, we can check semistability for saturated subsheaves.
\end{rmk}

\begin{lem}
    It is enough to check pure quotients. Namely, for all $E \twoheadrightarrow G$ pure of dimension $d$, $E$ is (semi)-stable if and only if $p(E) < (\leq) p(G)$.
\end{lem}
We can check that
\[ \alpha_d(F)(p(F) - p(E)) = \alpha_d(G)(p(E) - p(G)). \]

\begin{prop}
    Let $F, G$ be pure of dimension $d$ and semistable.
    \begin{enumerate}
        \item If $p(F) > p(G)$, then $\Hom(F, G) = 0$;
        \item If $p(F) = p(G)$ and there exists a nonzero $f \colon F \to G$, then if $F$ is stable, $f$ is injective. If $G$ is stable, then $f$ is surjective.
    \end{enumerate}
\end{prop}

\begin{rmk}
    Even when $X$ is integral, stability depends on $H$ in general.
\end{rmk}

\begin{cor}
    If $E$ is stable and $k = \ol{k}$, then $E$ is simple (which means $\End(E) = k$). Otherwise, $\End(E)$ is a finite-dimensional division algebra over $k$.
\end{cor}

\begin{proof}
    Consider the image $E$ of $f$. By semistability, we have $p(F) \leq p(E) \leq p(G)$, and so a nonzero map cannot exist. For the second part, if $F$ is stable, consider the image $E \subseteq G$ again. If $f$ is not injective, then $p(F) < p(E)$, and this is impossible.
\end{proof}

\begin{exm}
    Let $C$ be a curve of genus $g \geq 1$ and $L$ a line bundle of degree $d$. Consider $e \in H^1(\mc{O}_C) = \Ext^1(L, L)$. Then we have $0 \to L \to E \to L \to 0$, and $E$ is strictly semistable.
\end{exm}

\begin{exm}
    Let $E$ be a vector bundle on $D$ such that $(\deg E, \on{rk} E) = 1$. Then if $E$ is semistable, then $E$ is stable.
\end{exm}

\subsection{Some filtrations}%
\label{sub:some_filtrations}

We will now discuss the Harder-Narasimhan filtration
\begin{lem}
    Let $E$ be a coherent sheaf pure of dimension $d$. Then there exists $F \subseteq E$ such that for all $G \subseteq E$, $p(F) \geq p(G)$ and if equality holds, then $G \subseteq F$. Moreover, $F$ is semistable and uniquely determined.
\end{lem}

\begin{defn}
    The $F$ in the lemma is called the \textit{maximal destabilizing subsheaf} of $E$.
\end{defn}

\begin{proof}
    Consider the ordering on the set of pairs $(F \subseteq E, p(F))$ where $(F, p(F)) \leq (F', p(F'))$ if $F \subseteq F'$ and $p(F) \leq p(F')$. This is clearly a nonempty set and every ascending subsequence of subsheaves has a maximal element. Among all maximal elements, choose one with minimal $\alpha_d > 0$. We will prove that $F$ is the desired maximal destabilizing subsheaf.

    If not, there exists $G \subseteq E$ such that $p(G) \geq p(F)$. If $G \not\subseteq F$, then $F \subsetneq F + G$ and $G \cap F \subsetneq G$. By the maximality of $F$, we can asssume that $p(F) > p(F + G)$. We have the exact sequence
    \[ 0 \to G \cap F \to F \oplus G \to F + G \to 0. \]
    But now we have $p(F + G) < p(F)$ by assumption, but this means that $p(F \cap G) > p(G) \geq p(F)$, and this is a contradiction. Thus every $G' \subseteq E$ with $p(G') > p(F)$ defines a subsheaf $G \subseteq F$ with $p(G) > p(F)$.

    Now let $G \subseteq F$ be a maximal subsheaf with $p(G) > p(F)$. Let $G \subseteq G' \subseteq E$ be maximal with respect to the ordering. By maximality of $G'$, we know that $p(G') \geq p(G) > p(F)$. We will now prove that $G' \not\subseteq F$. By $\alpha_d$-minimality of $F$, if $G' \subseteq F$, then $\alpha_d(G') < \alpha_d(F)$, and thus $G' \not\subseteq F$ by contradiction. This implies that $F \subsetneq F + G'$ and $G' \cap F \subsetneq G'$. By maximality of $F$, $p(F) > p(F + G')$. We assumed that $p(F) < p(G')$, and thus $p(G' \cap F) > p(G') \geq p(G)$, contradicting the maximality of $G$ in $F$.
\end{proof}

\begin{defn}
    Let $E$ be pure of dimension $d$. A \textit{Harder-Narasimhan filtraton} of $E$ is an increasing filtration
    \[ 0 \subseteq HN_1(E) \subseteq \cdots \subseteq HN_{\ell}(E) = E \]
    such that
    \begin{enumerate}
        \item The graded pieces $HN_i / HN_{i-1}$ are semistable with reduced polynomials $p_i$.
        \item We have $p_1 > p_2 > \cdots > p_{\ell}$.
    \end{enumerate}
\end{defn}

\begin{exm}
    If $E$ is semistable, then $0 \subseteq E_1 = E$ is a Harder-Narasimhan filtration.
\end{exm}

\begin{exm}
    Consider $L_1, L_2$ line bundles of degrees $d_1 \geq d_2$. Then $\Ext^1(L_2, L_1) = H^1(L_1 \otimes L_2^{\vee}) \neq 0$. Considering a nontrivial extension
    \[ 0 \to L_1 \to E \to L_2 \to 0, \]
    note that $E$ is unstable and $0 \subseteq L_1 \subseteq E$ is a Harder-Narasimhan filtration.
\end{exm}

\begin{thm}
    Let $E$ be pure of dimension $d$. Then there exists a unique Harder-Narasimhan filtration.
\end{thm}

\begin{proof}
    We induct on $\alpha_d(E)$. We may assume that $E$ is unstable. Otherwise, let $E_1 \subseteq E$ be the maximal destabilizing subsheaf. Then we consider $E/E_1$ and $\alpha_d(E/E_1) < \alpha_d(E)$. By induction, we have a Harder-Narasimhan filtration
    \[ 0 \subseteq G_1 \subseteq \cdots \subseteq G_k = G = E / E_1 \]
    such that $p(G_1) > p(G_2 / G_1) > \cdots p(G_k / p(G_{k-1}))$. Set $E_{i+1} = \pi^{-1}(G_i)$ under $\pi \colon E \to E/E_1$. But now we have $p(E_2 / E_1) = p(G_1), p(E_3/E_2) = p(G_2/G_1), \ldots$, and this gives us
    \[ E_1 \subseteq E_2 \subseteq \cdots. \]
    We only need to check that $p(E_1) > p(E_2 / E_1)$. By the maximality of $E_1$, we know that $p(E_1) > p(E_2)$, and this gives the desired result.

    Now let $E_{\bullet}, E'_{\bullet}$ be two Harder-Narasimhan filtrations. Then we know $E_1, E_1' \subseteq E$, and we may assume that $p(E_1') \geq p(E_1)$. Let $j$ be the minimum integer such that $E_1' \subseteq E_j$. But now the map $E_1' \to E_j / E_{j-1}$ is nonzero, and source and target are both semistable, so $p(E_1') \leq p(E_j / E_{j-1}) < p(E_1)$, and the last inequality holds unless $j = 1$. By assumption, we know $E_1' \subseteq E_1$ and $p(E_1) = p(E_1')$. Reversing the argument, we see that $E_1 \subseteq E_1'$, so $E_1 = E_1'$. This proves uniqueness of the maximal destabilizing subsheaf, and by induction, the two filtrations on $E/E_1$ are the same, so $E_{\bullet} = E'_{\bullet}$.
\end{proof}

We will now move on to another filtration:
\begin{defn}
    Let $E$ be a semistable sheaf pure of dimension $d$. Then a \textit{Jordan-H\"older filtration} for $E$ is an increasing filtration
    \[ 0 = E_0 \subseteq E_1 \subseteq \cdots \subseteq E_{\ell} = E \]
    such that $\mr{gr}_i(E) = E_i / E_{i-1}$ is \textbf{stable} with $p(\mr{gr}_i(E)) = p(E)$.
\end{defn}

\begin{exm}
    If $0 \to L \to E \to L \to 0$ is a nontrivial extension of a line bundle $L$ by itself, this gives a Jordan-H\"older filtration. Also, $E = L^{\oplus n}$ shows that the filtration is not unique in general.
\end{exm}

\begin{prop}
    Jordan-H\"older filtrations exist and
    \[ \mr{gr}_{JH}(E) \coloneqq \bigoplus \mr{gr}_i(E) \]
    is unique up to isomorphism.
\end{prop}

\begin{proof}
    We induct on $\alpha_d(E) > 0$. If $E$ is stable, we are done. Otherwise, there exists $F \subseteq E$ such that $p(F) = p(E)$ and $\alpha_d(F) \leq \alpha_d(E)$. This inequality must be strict because otherwise $p_F(t) = p_E(t)$ and thus $F = E$. Therefore, there exists a subsheaf $F \subseteq E$ with $p(F) = p(E)$ and minimal $\alpha_d(F)$, and this must be stable. By induction, $E/F$ is semistable with the same reduced Hilbert polynomial, and so by induction, it has a Jordan-H\"older filtration. Pulling back to $E$, we obtain the filtration on $E$. 

    The proof of uniqueness is omitted.\footnote{Because there are many things that Giulia wants to do this semester.} The outline is similar to the case of the Harder-Narasimhan filtration, but the actual argument is different.
\end{proof}

\subsection{Boundedness}%
\label{sub:boundedness}

We will prove boundedness of semistable sheaves with a fixed Hilbert polynomial in the case of a smooth projective surface (because we will be using this result for K3 surfaces only). Note that this proof will only work for surfaces.

Recall that for a smooth projective surface $X$ with polarization $H$ and a sheaf $E$ of rank $r$, Riemann-Roch says that
\[ \chi(E) = \frac{1}{2} (c_1(E)^2 - c_1(E) \cdot K_X) - c_2(E) + r \chi(\mc{O}_X). \]
This implies that
\[ p_E(t) = \frac{r}{2} H^2 t^2 + \qty(c_1(E) \cdot H - \frac{r}{2} H \cdot K_X) \cdot t + \chi(E). \]
Thus the coefficients of $p(E)$ are determined by $\frac{c_1(E) \cdot H}{r}$ and $\frac{\chi(E)}{r}$ if $E$ is pure of dimension $2$.

\begin{defn}
    Let $E$ be pure of dimension $2$. Then define the \textit{slope} of $E$ by
    \[ \mu(E) = \frac{c_1(E) \cdot H^{n-1}}{\mr{rk}(E)}. \]
\end{defn}

This gives a notion of slope-stability. It is a fact that slope-stability implies Gieseker stability which implies Gieseker semistability, which implies slope-semistability.

\begin{rmk}
    There exists a maximal destabilizing subsheaf $F \subseteq E$, and we denote $\mu(F) \eqqcolon \mu_{\mr{max}}(E)$.
\end{rmk}

\begin{rmk}[Rudakov]
    We can define stability for any abelian category $\mc{C}$ with a preorder $\leq$. Then we can define stability with respect to $\leq$, and in this case we obtain a Schur Lemma. Under certain finiteness assumptions (where $\mc{C}$ is weakly Artinian or weakly Noetherian with respect to $\leq$), we obtain Harder-Narasimhan filtrations or Jordan-H\"older filtrations.
\end{rmk}

\begin{exm}
    King-stability for quiver representations is an example of such a stability condition.
\end{exm}

\begin{thm}
    Let $(X, H)$ be a smooth projective polarized surface and $p(t) \in \Q[t]$. Then there exist $N,\ell$ only depending on $(X, H)$ and $p(t)$ such that any slope-semistable $F$ is a quotient $\mc{O}_X^N(-\ell) \twoheadrightarrow F$.
\end{thm}

\begin{rmk}
    Note that twisting by $\mc{O}_X(\ell H)$ does not affect (semi)-stability. Therefore the theorem gives boundedness of slope-semistable sheaves with fixed Hilbert polynomial (because they live in the Quot scheme).
\end{rmk}

We will actually prove a slightly stronger result which allows us to use induction.
\begin{thm}
    Let $(X, H)$ and $p(t) \in \Q[t]$ as above and fix $\mu \in \Q$. Then there exists $N, \ell$ only depending on $(X,H), p, \mu$ such that for all $c \geq 0$ and any pure dimension $2$ sheaf $F$ with Hilbert polynomial $p_F = p + c$ and such that $\mu_{\mr{max}}(F) \leq \mu$, $\mc{O}_X^N \twoheadrightarrow F(\ell)$. Moreover, the set of possible $c$ (and hence of possible $p_F$) is finite.
\end{thm}

\begin{proof}
    We will induct on the rank of $F$. Let $F$ be such that $p_F = p + c$ for $c \geq 0$ and $\mu_{\mr{max}}(F) \leq \mu$. First we find a uniform $\ell$ such that $h^0(F(\ell)) \neq 0$. Using Riemann-Roch, we will have
    \[ h^0(F(\ell)) + h^2(F(\ell)) \geq p_F(\ell) \geq p(\ell). \]
    By Serre duality, $h^2(F(\ell)) = \dim \Hom(F, K_X(-\ell))$.

    First, we claim that there exists a uniform $\ell$ such that $\dim \Hom(F, K_X(-\ell)) = 0$. If we consider
    \begin{equation*}
    \begin{tikzcd}
        0 \ar{r} & G \ar{r} & F \ar{r}{\varphi} \ar{dr} & K_X(-\ell) \\
        & & & E \ar[hookrightarrow]{u}
    \end{tikzcd}
    \end{equation*}
    where $E$ has rank $1$, then $c_1(E) \cdot H \leq (K_X - \ell H) \cdot H$, and thus
    \[ \mu(G) = \frac{c_1(G) \cdot H}{r-1} \geq \frac{1}{r-1} (r \mu(F) + \ell H^2 - H \cdot K_X) > \mu. \]
    In particular, $\mu(G) \leq \mu_{\mr{max}}(F) < \mu$, so we can fix $\ell$ such that $h^0(F(\ell)) \neq 0$. If we consider
    \[ \mc{O}_X \xrightarrow{s} F(\ell) \to Q, \]
    we can saturate this and obtain
    \[ \mc{H} \to F(\ell) \to Q/\mr{Tors} \eqqcolon G. \]
    We know that $\mc{H}$ is rank $1$ and torsion free, so $\mc{H} = \mc{L} \otimes \mc{I}_Z$, where $\dim Z = 0$ and $\mc{L}$ is a line bundle. This is because $0 \to \mc{H} \hookrightarrow \mc{H}^{\vee \vee}$, and reflexive sheaves are locally free in codimension $3$ and the cokernel is supported in codimension $2$.

    Now we have $h^0(\mc{H}(\ell)) \neq 0$, and thus $H^0(\mc{L}(\ell)) \neq 0$. Write $\mc{L}(\ell) = \mc{O}(D)$ with $D$ effective. Then 
    \[ -\ell H^2 \leq \mu(H) = D \cdot H - \ell H^2 \leq \mu, \]
    so there are only finitely many possibilities for $c_1(D)$ (because we have bounded $D \cdot \text{ample}$). In particular, this means there are only finitely many possibilities for $c_1(G)$ and thus for $\mu(G)$. In fact, we can uniformly bound $\mu_{\mr{max}}(G)$ because for any $G' \subseteq G$, we can produce
    \[ 0 \to \mc{H} \to F' \to G' \to 0 \]
    and $\mu(F')$ is bounded.

    Now $G$ has rank $r-1$ and slope $\mu_{\mr{max}}(G) \leq \mu'$, where $\mu'$ is uniform. Thus
    \begin{align*}
        p_G &= p_F - p_{\mc{H}} \\
        &= p_F - p_{\ell} + \deg Z \\
        &= p - p_{\ell} + c + \deg Z \\
        &= p' + c'
    \end{align*}
    for some $p', c'$. By induction, there exist uniform $N', \ell'$ such that $\mc{O}_X^{N'} \twoheadrightarrow G(\ell')$. Moreover, there exist finitely many possible $c'$ and thus finitely many $\deg Z$. Applying the theorem to $\mc{H}$ (that there are finitely many $c_1(\mc{H})$ and $\deg Z$), we have a surjection $\mc{O}_X^{N''} \twoheadrightarrow \mc{H}(\ell')$. Because such $\mc{H}$ form a bounded family, we can assume that $H^1(\mc{H}(\ell')) = 0$. Thus we have 
    \[ 0 \to H^0(\mc{H}(\ell')) \to H^0(F(\ell')) \to H^0(G(\ell')) \to 0. \]
    Now we lift the $N'$ sections on the right and consider the $N''$ sections on the right, and we get $\mc{O}_X^{N'''} \twoheadrightarrow F(\ell')$.

    It remains to prove the rank $1$ case. We will prove that give $(X, H), p(t)$ (here $\mu_{\mr{max}} = \mu$ because we are in the rank $1$ case), there exist $N, \ell$ such that if $\mc{H}$ has rank $1$ and $p_{\mc{H}} = p + c$ for $c \geq 0$, then $\mc{O}_X^N \twoheadrightarrow \mc{H}(\ell) \to 0$. We know that $\mc{H} = \mc{L} \otimes \mc{I}_Z$. By Riemann-Roch, there exists $\ell$ such that $H^0(\mc{H}(\ell)) \neq 0$. Fixing such an $\ell$, we know that
    \[ D \cdot H = \mc{L} \cdot H + \ell H^2, \]
    and so there are only finitely many possibilities for $c_1(\mc{H})$. Using Riemann-Roch again, we know
    \[ \frac{1}{2} [c_1(\mc{H})^2 - c_1(\mc{H}) \cdot K_X] - \deg Z + \chi(\mc{O}_X) = \chi(\mc{H}) = p(0) = p(0) + c, \]
    and therefore there are only finitely many possible $\deg Z$.
\end{proof}

\begin{prop}
    The conditions of Gieseker (semi)stability and slope-semistability are open in families.
\end{prop}

\begin{proof}[Sketch]
    Consider $\mc{F}$ on $X \times B$. Then consider a destabilizing quotient $\mc{F}_b \twoheadrightarrow G$. Suppose you prove that there exist only finitely many possible Hilbert polynomials $p_i$. We then consider the relative Quot schemes
    \[ \mr{Quot}_{X \times B / B, p_i}^{\mc{F}} \to B, \]
    which are proper over $B$. Studying this morphism, we obtain the desired result.
\end{proof}

\section{Moduli of sheaves}%
\label{sec:moduli_of_sheaves}

\begin{defn}
    Two pure sheaves $F, F'$ of dimension $d$ on $(X, H)$ are called \textit{$S$-equivalent} if the associated graded pieces $\mr{gr}^{\mr{GH}}(F) \simeq \mr{gr}^{\mr{JH}}(F')$ are the same. Here, the ``S'' apparently stands for Seshadri.
\end{defn}

\begin{defn}
    A \textit{polystable sheaf} is a direct sum $\bigoplus F_i$, where the $F_i$ are stable and $p(F_i) = p(F_j)$ for all $i, j$.
\end{defn}

Note that we cannot separate $S$-equivalent sheaves. More precisely, consider the following example.
\begin{exm}
    Suppose that $F_1, F_2$ are stable with $p(F_1) = p(F_2)$ and suppose that there exists a nontrivial extension
    \[ 0 \to F_1 \to F \to F_2 \to 0. \]
    Clearly $F$ is semistable. We will show that there exists a flat family $\mc{F}$ on $\A^1 \times X$ such that for all $t \neq 0$, $\mc{F}_t = F$ and for $t = 0$, $\mc{F}_0 = F_1 \oplus F_2$. This will tell us that if there is a separated moduli space, then $F_1 \oplus F_2$ and $F$ must be the same closed point.

    Consider the second projection $p_2 \colon \A^1 \times X \to X$. Then if we consider $i_0 \colon \qty{0} \times X \hookrightarrow \A^1 \times X$, the exact sequence
    \[ 0 \to \mc{F} \to p_2^* F \twoheadrightarrow i_{0*} F_2 \]
    will define the desired family $\mc{F}$.
\end{exm}

By boundedness of semistable sheaves with a fixed Hilbert polynomial, there exists $m$ such that all such cheaves are $m$-regular (they become Castelnuovo-Mumford regular after twisting with $\mc{O}(m)$). We will now denote $\alpha_d(F)$ by $r$, and now all sheaves with $0 < r' < r$ and the same reduced Hilbert polynomial are also $m$-regular. We will now fix $m$ and set $N = p(m)$. Then we have a surjection $\mc{O}_X^N(-m) \twoheadrightarrow F$. We may now consider the Quot scheme
\[ \mr{Quot}^{N_m} \subseteq \mr{Grassmannian}. \]
Recall that the map to the Grassmannian was determined by
\[ 0 \to p_{1*} G(\ell) \to p_{1*}(\mc{O}_X^N(\ell - m)) \to p_{2*}(\mc{F}(\ell)) \to 0, \]
which gave $H^0(G(\ell)) \subseteq H^0(\mc{O}_X^N(\ell - m))$. In particular, we have
\[ \det (p_{2*} \mc{F}(\ell)) = i_{\ell}^* (\text{Pl\"ucker}), \]
and so $\mc{L}_{\ell} \coloneqq \det p_{2*} \mc{F}(\ell)$ is very ample. Now write $\mc{H} = \mc{O}_X^N(-m)$ for shorthand. Then $\rho \in \mr{Quot}$ is really a map $\mc{H} \to F$. Now define
\[ R = \qty{F \text{ semistable} \mid h^0(\rho(m)) \text{ is an isomorphism}} \]
and consider $R^s \subseteq R$ the locus of stable $F$. Let $V \coloneqq k^N$. We know that $R$ is preserved by the natural action of $GL(V)$ because $g$ acts by
\[ V \otimes \mc{O}(-m) \xrightarrow{g} V \otimes \mc{O}(-m) \xrightarrow{\rho} F, \]
and so $g \circ \rho = \rho \circ g$.

\begin{lem}
    Let $\rho \in \mr{Quot}$ such that $F(m)$ is globally generated. Then if $h^0(\rho(m))$ is an isomorphim, the stabilizer of $\rho$ is actually $\Aut(F) \subseteq GL(V)$.
\end{lem}

We need to understand this $GL(V)$-action and why there is an natural linearization. On $GL(V)$, there exists a ``universal automorphism'' $\tau \colon V \otimes \mc{O}_{GL(V)} \to V \otimes \mc{O}_{GL(V)}$. On $GL(V) \times \mr{Quot}$, we have
\[ p_2^* \mc{H} \xrightarrow{p_1^* \tau} p_2^* \mc{H} \to p_2^* \mc{F}, \]
and this gives another quotient, and thus defines a map $\sigma \colon GL(V) \times \mr{Quot} \to \mr{Quot}$. By the universal property, we have $p_2^* \mc{F} = \sigma^* \mc{F}$ and thus a linearization of the action. Because all natural constructions of $GL(V)$-linearized sheavs are naturally $GL(V)$-linearized, we know $\det p_{1*} \mc{F}(\ell) = \mc{L}_{\ell}$ is also linearized. Now we have $\ol{R}$ with the linearized line bundle $\mc{L}_{\ell}$ for $G = SL(V)$, so we can form the GIT quotient.

\begin{thm}
    For $\ell$ sufficiently large,
    \begin{enumerate}
        \item $\rho \in \ol{R}$ is GIT semistable if and only if $F$ is semistable;
        \item $\ol{GL(V) \cdot F} \cap \ol{GL(V) \cdot F'} \neq \emptyset$ if and only if $F, F'$ are $S$-equivalent.
        \item $F$ is polystable if and only if $SL(V) \cdot \rho \subseteq R$ is closed.
    \end{enumerate}
\end{thm}

Recall that for $G$ a reductive group acting on projective $X$ and a $G$-linearized ample line bundle $L$ on $X$, the GIT quotient $X \sslash G$ is given by $X^{ss}(L) / G$. Recall that points of $X \sslash G$ correspond to equivalence classes of orbits and that $X \sslash G$ is projective. Also recall that one can explicitly determine GIT semistability using the Hilbert-Mumford criterion. This says that $x \in X$ is semistable if and only if for all $\lambda \colon \mathbb{G}_m \to G$, the number $m(x, \lambda) \geq 0$. This $m(x, \lambda)$ is determined by considering $\lim_{t \to 0} \lambda(t) \cdot x \eqqcolon x_0$ and considering the weight of $\mathbb{G}_m$ on the fiber above $x_0$. Then $m(x, \lambda)$ is defined to be $- \text{weight}$.

Now consider $\rho \colon V \otimes \mc{O}_X(-m) \to F \in \ol{R}$. Let $\lambda \colon \mathbb{G}_m \to SL(V)$ be a one-parameter subgroup. This is determined by its weight decomposition $V = \bigoplus V_n$. Define a filtration on $V$ by $V_{\leq n} = \bigoplus_{m \leq n} V_m$. This determines a filtration 
\[ F_{\leq n} = \rho(V_{\leq n} \otimes \mc{O}_X(-m)) \subseteq F. \]
\begin{lem}\leavevmode
    \begin{enumerate}
        \item We have $\lim_{t \to 0} \lambda(t) \rho = \rho_0$, where 
            \[ \rho_0 \colon \bigoplus V_{\leq n+1} / V_{\leq n} \otimes \mc{O}_X(-m) \to \bigoplus F_{\leq n+1} / F_{\leq n}. \]
        \item The weight of $\mathbb{G}_m$ on $(\mc{L}_{\ell})_{\rho_0}$ is given by $\sum n p_{F_n}(\ell)$.
    \end{enumerate}
\end{lem}

\begin{proof}
    For $\ell \gg 0$, formation of $\mc{L}_{\ell}$ commutes with base change, and so
    \[ (\mc{L}_{\ell})_{\rho_0} = \det \qty( \bigoplus p_{2*} F_n(\ell)) = \bigotimes \det (p_{2*} F_n(\ell)). \]
    Assuming that there is no higher cohomology, this has dimension  $\dim H^0(F_n(\ell)) = p_{F_n}(\ell)$. But now the action of $\mathbb{G}_m$ here is given by $n \cdot h^0(F_n(\ell))$.

    We will now sketch the first part. Consider the map $\mathbb{G}_m \to \mr{Quot}$ given by $\lambda(t) \cdot \rho$. By properness, this extends to a map $\A^1 \to \mr{Quot}$. Now $\rho \colon V \oplus \mc{O}_X(-m) \to F$ defines a sheaf $F \otimes k[t, t^{-1}]$, and we want to extend this to $\A^1$.

    First, define
    \[ \mc{V} \coloneqq \bigoplus V_{\leq n} \cdot t^n \subseteq V \otimes k[t, t^{-1}]. \]
    This is the same as $V \otimes k[t]$, where $v \otimes 1 \mapsto v \otimes t^n$ if $v \in V_n$. Now multiplication by $t$ on each piece is just the natural inclusion $V_{\leq n} \to V_{\leq n+1}$. Thus the fiber of $\mc{V}$ above $0$ is naturally the direct sum $\bigoplus V_{\leq n+1} / V_{\leq n}$. Similarly, we define the sheaf
    \[ \mc{F} \coloneqq \bigoplus F_{\leq n} \otimes t^n \subseteq F \otimes t^{-N} k[t] \]
    on $\A^1 \times X$, and it is easy to see that $\mc{F}_0 = \bigoplus F_n$.
\end{proof}

We now assume that $\sum n \dim V_n = 0$. This tells us that
\begin{align*}
    \mr{wt}(\mc{L}_{\ell}, \rho_0) &= \sum n p_{F_n}(\ell) \\
    &= \frac{1}{\dim V} \qty(\sum n (\dim V p(F_n, \ell) - \dim V_n p(F, \ell))) \\
    &= - \frac{1}{\dim V} \qty(\sum \dim V p(F_{\leq n}, \ell) - \dim V_{\leq n} p(F, \ell)).
\end{align*}
Now for all $V' \subseteq V$, define $F' = \rho(V' \otimes \mc{O}_X(-m)) \subseteq F$, and for all $F' \subseteq F$, define $V' = \rho^{-1} H^0(F'(m))$. We now want to reformulate the Hilbert-Mumford criterion.

\begin{lem}
    $\rho \in \ol{R}$ is GIT semistable if and only if for all $V' \subseteq V$, $\vartheta(V') \geq 0$, where
    \[ \vartheta(V') = \dim V \cdot p(F', \ell) - \dim V' \cdot p(F, \ell). \]
\end{lem}

The next step is to translate this in terms of subsheaves $F' \subseteq F$ and their Hilbert polynomials. Here, we want 
\[ \dim V \cdot p(F') \leq \dim V' \cdot p(F). \]
By a theorem of Le Potier characterizing semistability, we consider $\frac{h^0(F(m))}{r}$.

\begin{thm}[Le Potier]
    Fix $p \in \Q[t]_d$ and a multiplicity $r$. Then there exists $m \gg 0$ such that for all sheaves $F$ pure of dimension $d$ with $\alpha_d(F) = r$ and $p_F = rp$, the following are equivalent:
    \begin{enumerate}
        \item $F$ is semistable.
        \item $rp(m) \leq h^0(F(m))$ and for all $F' \subseteq F$, $\frac{h^0(F'(m))}{r'} \leq p(m)$.
    \end{enumerate}
\end{thm}

Modulo actually proving anything, we now have semistability. It remains to discuss the orbit closures. Consider a semistable $(F, \rho)$. Then we know that $(\mr{JH}(F), \mr{JH}(\rho)) \in \ol{SL(V) \cdot \rho}$. It is enough to show that orbits of polystable sheaves are closed in $R$.

\begin{lem}
    Consider a sheaf $\mc{E}$ on $C \times X$ flat over $C$ and suppose that $F = \bigoplus F_i^{n_i}$ is polystable. Suppose that $\mc{E}_t = F$ for $t \neq 0$ and that $\mc{E}_0$ is semistable. Then $\mc{E}_0 = F$.
\end{lem}

To prove this lemma, consider the upper semicontinuous function $t \mapsto \mr{hom}(F_i, \mc{E}_t)$. But then the map
\[ \bigoplus F_i \otimes \Hom(F_i, \mc{E}_0) \to \mc{E}_0 \]
is injective, and this gives us the desired inequality.

Now consider $R^s \subseteq R$. We know that $R^s \sslash SL(V) \subseteq R \sslash SL(V) \eqqcolon M$ is open, and that $M$ is projective by GIT quotient. We know that $R^s \to R^s \sslash SL(V)$ is a geometric quotient and that $R \to R \sslash SL(V)$ is a good quotient. Also, closed points are the same as $S$-equivalence classes of semistable sheaves while points of $R^s$ are isomorphism classes of stable sheaves. Now recall the functor $\mc{M} \colon \ms{Sch}/k^{\mr{op}} \to \ms{Set}$ sending a scheme $S$ to isomorphism classes of $F$ on $S \times X$ flat over $S$ with the correct Hilbert polynomial and semistable. Here, $F \sim F'$ is there exists $L \in \Pic(S)$ such that $F = p_1^* L \otimes F'$.

\begin{thm}
    The moduli spaces $M$ (resp. $M^s$) universally corepresent $\mc{M}$ (resp. $\mc{M}^s$).\footnote{This means that $M$ is a coarse moduli space for $\mc{M}$.}
\end{thm}

\begin{rmk}
    If there exist strictly semistable sheaves, then $M$ is not a fine moduli space for $\mc{M}$.
\end{rmk}

We now try to salvage existence of a universal family on $M^s$ (which is not always possible).
\begin{defn}
    A sheaf $\mc{E}$ on $M^s \times X$ flat over $M^s$ is a \textit{universal family} if for all $S$-flat families of stable sheaves $F$ with Hilbert polynomial $p$, there exists $L \in \Pic S$ such that $\phi_F^*(\mc{E}) = F \otimes p_1^* L$. The family is \textit{quasi-universal} if there exists a locally free $W$ such that $\phi_F^*(\mc{E}) = F \otimes p_1^* W$.
\end{defn}

Now consider a coherent sheaf $\mc{E}$ on $R^s$. This descends to $R^s \sslash GL(V)$ if and only if the action of $\mathbb{G}_m \subseteq GL(V)$ is trivial on $\mc{E}_x$ for all $x \in R^s$. We know that on $R^s \times X$, we have the universal sheaf $\wt{\mc{F}}$. Clearly this does not descent because of the morphism $\Aut(F) \to GL(V)$, so the action is nontrivial. For $\ell \gg 0$, the sheaf
\[ A_{\ell} \coloneqq p_{1*}(\wt{\mc{F}} \otimes \mc{O}_X(\ell)) \]
is locally free and $\mathbb{G}_m$-linear if weight $0$. Thus the sheaf $\Hom(p_1^* A_{\ell}, \wt{\mc{F}})$ descends to $M^s \times X$. We claim that this is a quasi-universal family.

\subsection{Determinantal line bundles}%
\label{sub:determinantal_line_bundles}

Let $X$ be smooth and projective and let $K(X)$ be the Grothendieck group of coherent sheaves on $X$. Of course, this is a ring, and it has a quadratic form
\[ \chi(a \cdot b) = \int_X \mr{ch}(a) \mr{ch}(b) \mr{td}(X). \]
We may also consider the numerical Grothendieck group $K_{\mr{num}}(X)$ by quotienting out by the kernel of the quadratic form.

Now suppose that $X \to Y$ is a smooth projective morphism of relative dimension $n$ and $F$ is coherent on $X$ and flat over $Y$. Then there exists a locally free resolution $F_{\bullet} \to F$ of length $n$ which computes higher direct images. What we mean by this is that 
\[ R^i f_* F_j = \begin{cases}
    0 & i \leq n \\
    \text{locally free} & i = n.
\end{cases}
\]
Also, we have $\mc{H}^{n-i}(F^n f_* F_{\bullet}) = R^i f_* F$. Then we have
\begin{align*}
    \det (R f_* F) &= \bigotimes \det(R^n f_* F_i)^{(-1)^i} \\
    &= \bigotimes \det (R^i f_* F)^{(-1)^i}.
\end{align*}
Now consider $\mc{E}$ on $S \times X$ flat over $S$ and consider $\lambda_{\mc{E}} \colon \mc{Coh}(X) \to \Pic S$ given by
\[ F \mapsto \det(R p_{1*} (p_2^* F \otimes \mc{E})). \]
$\lambda_{\mc{E}}$ satisfies certain properties. The most important for us right now concerns a rank $r$ locally free sheaf $W$ on $S$. Here, we have
\[ \lambda_{\mc{E} \otimes p_1^* W}(u) = \lambda_{\mc{E}}(u)^r \otimes \det W^{\chi(c \cdot u)}, \]
where $c$ is the class of $\mc{E}_s$, for all $u \in K(X)$. This gives us a map 
\[ K_{\mr{num}}(X) \supseteq c^{\perp} \to \Pic(M^s). \]
For a K3 surface, we can do this on integral cohomology, and then $H^2(M_v, \Z) = v^{\perp}$ as Hodge structures.

\subsection{Mukai's theorem}%
\label{sub:mukai_s_theorem}

\begin{thm}
    Let $X$ be projective over $k$, where $k$ is algebraically closed and of characteristic $0$. If $F$ is a stable sheaf on $X$ and $M$ is the moduli space containing $F$, then $\wh{\mc{O}}_{M, F}$ pro-represents $\mr{Def}_F$. Moreover, 
    \[ T_{[F]} M = \Ext^1(F, F) \xrightarrow{T \det} T_{[\det F]} \Pic(X) = H^1(X, \mc{O}_X) \]
    is the trace map. In adddition, for any small extension
    \[ 0 \to I \to \wt{A} \to A \to 0, \]
    we have $\tr(\mr{ob}(F_A, \wt{A}, A)) = \mr{ob}(\det F_A, \wt{A}, A)$. Also, $\mr{Def}_F$ has a tangent-obstruction theory given by $T^1 = \Ext^1(F, F)$ and $T^2 = \Ext^2(F, F)_0$.
\end{thm}

\begin{cor}
    If $F$ is stable and $X$ is a surface with $\omega_X = \mc{O}_X$, then $M$ is smooth at $[F] \in M$.
\end{cor}

\begin{proof}
    Here, we know that $\Ext^2(F ,F) \simeq H^2(X, \mc{O}_X)$ is $1$-dimensional, ans thus the trace morphism has no kernel, so deformations are unobstructed.
\end{proof}

\begin{proof}[Proof of theorem]
The result about the tangent-obstruction theory of a coherent sheaf is due to Maruyama, Mukai, and Artamkin, and there is a generalization of Lieblich for complexes and for a morphism $X \to S$. 

Next, if $F$ is stable, then $\mr{Def}_F$ is pro-representable. If $F$ is only polystable, then $\mr{Def}_F$ has a hull. Here, we have a map $\mr{Def}_F \to \wh{\mc{O}}_{M, F}$ and $\mr{Quot} \supseteq R^s \to M^2$, and then we use Luna's \'etale slice. First, there there exists $F \in V \subseteq \mr{Quot}$ that is $\Aut(F)$-invariant such that $V \sslash \Aut(F)$ is locally isomorphic to $(M, F)$, and this constructs an inverse when $F$ is stable.

We now want to consider the trace map. If $E$ is locally free, then $\Ext^i(E, E) = H^i(\mr{End}(E))$, and this clearly has a trace map to $H^i(\mc{O}_X)$. We also have the inclusion
\[ \mc{O}_X \xrightarrow{\delta} \End(E) \xrightarrow{\mr{tr}}(E) \mc{O}_X, \]
and this induces multiplication by the rank of $E$ on cohomology. For any $E$, choose a locally free resolution $E^{\bullet} \to E$. Then
\[ \Ext^i(E, E) = \mathbb{H}^i(\Hom(E^{\bullet}, E^{\bullet})), \]
and now we have the maps
\[ \mc{O}_X \xrightarrow{\delta} \Hom(E^{\bullet}, E^{\bullet}) \xrightarrow{\tr} \mc{O}_X. \]
Now we want to compute deformation of $F$ on $k[\ep]$. Here, we have a short exact sequence
\[ 0 \to F \to F_{\ep} \to F \to 0. \]
But now there exists a splitting $k \to k[\ep]$, and so we have an element of $\Ext^1_X(F, F)$. Conversely, given any extension
\[ 0 \to F \to \mc{F} \to F \to 0, \]
we declare the multiplication by $\ep$ to be $\mc{F} \to F \to \mc{F}$.

We now consider obstructions. Up to tensoring by $\mc{O}(m)$, we can assume that $H^i(F) = 0$ for all $i > 0$. Then declare $V = H^0(F)$ and $\mc{H} = V \otimes \mc{O}_X$. Then we have a short exact sequence
\[ 0 \to G \to \mc{H} \to F \to 0. \]
Because $\mc{H}$ is locally free and $X$ is smooth, then $dh(G) = \max \qty{0, dh(F) - 1}$. Applying $\Hom(-, F)$, we now have
\begin{equation*}
\begin{tikzcd}
    0 \ar{r} & \Hom(F, F) \ar{r} & \Hom(\mc{H}, F) \ar{r} & \Hom(G, F) \ar[out=-20, in=160]{dll} \\
    & \Ext^1(F, F) \ar{r} & \Ext^1(\mc{H}, F) \ar{r} & \Ext^1(G, F) \ar{r} & \Ext^2(F, F) \ar{r} & 0.
\end{tikzcd}
\end{equation*}
Becuase $\Ext^1(\mc{H}, F) = 0$ and $\Ext^1(G, F)$ is the obstruction space for the Quot scheme, we have the desired expression for the obstruction. We will in fact give a different proof. For
\[ 0 \to I \to \wt{A} \to A \to 0, \]
we have the following diagram:
\begin{equation*}
\begin{tikzcd}
    & I \ar[equals]{r} \ar{d} & I \ar{d} \\
    0 \ar{r} & \wt{\mf{m}} \ar{r} \ar{d} & \wt{A} \ar{r} \ar{d} & k \ar[equals]{d} \\
    0 \ar{r} & \mf{m} \ar{r} \ar{d} & A \ar{d} \ar{r} & k \ar{r} & 0 \\
    & 0 & 0.
\end{tikzcd}
\end{equation*}
Given $F_A$, we know that $\mr{ob}(F_A, \wt{A}, A) \in \Ext^2(F, F \otimes_k I) = \Ext^2(F, F) \otimes I$. Tensoring by $F_A$, we now have
\begin{equation*}
\begin{tikzcd}
    & F_A \otimes I \ar[equals]{r} \ar{d} & F_A \otimes I \ar{d} \\
    0 \ar{r} & F_A \otimes \wt{\mf{m}} \ar{r} \ar{d} & F_A \otimes \wt{A} \ar{r} \ar{d} & F \ar[equals]{d} \\
    0 \ar{r} & F_A \otimes \mf{m} \ar{r} \ar{d} & F_A \ar{d} \ar{r} & F \ar{r} & 0 \\
    & 0 & 0.
\end{tikzcd}
\end{equation*}
Now we know that
\[ [0 \to F_A \otimes \mf{m} \to F_A \to F \to 0] = f_{F_A} \in \Ext^1(F, F_A \otimes \mf{m}) \]
and
\[[0 \to F_A \otimes I \to F_A \otimes \wt{\mf{m}} \to F_A \otimes \mf{m} \to 0] = e_{F_A} \in \Ext^1(F_A \otimes \mf{m}, F \otimes I). \]
Applying $\Hom(F, -)$, we now obtain
\[ \Ext^1(F, F_A \otimes \wt{\mf{m}}) \to Ext^1(F, F_A \otimes \mf{m}) \xrightarrow{\cup e_{F_A}} \Ext^2(F, F \otimes I). \]
This tells us that the obstruction $\mr{ob}(F_A, \wt{A}, A) = f_{F_A} \cup e_{F_A}$.

\begin{lem}
    If $E$ is locally free, then $\mr{ob}(E_A, \wt{A}, A) \in \ker(\mr{tr} \colon H^2(\End(E)) \to H^2(\mc{O}_X))$.
\end{lem}

To prove this, for some $E_{\wt{A}}$, we get $\qty{G_{\alpha\beta}} \in H^1(\End(E) \otimes \wt{A})$ such that
\[ G_{\alpha\beta} G_{\beta \gamma} G_{\gamma\alpha} = \mr{Id}_{F_A} + \sum o_{\alpha\beta\gamma} \cdot t, \]
and we can construct the obstruction by hand.

Now denote
\[ [0 \to G \to \mc{H} \to F \to 0] = \theta \in \Ext^1(F, G). \]
We claim there exists $\psi \in \Ext^1(G, F)$ such that 
\[ \mr{ob}(G_A, \wt{A}, A) = \psi \cup \theta \in \Ext^2(G, G) \otimes I \]
and 
\[ \mr{ob}(F_A, \wt{A}, A) = \theta \cup \psi \in \Ext^2(F, F) \otimes I. \]
We know that $\tr(\theta \cup \psi) =0$ if and only if $\tr(\psi \cup \theta) = 0$.
\end{proof}

We will now construct a holomorphic form on the moduli space.
\begin{thm}
    Let $X$ be a K3 surface or an abelian surface. Then $M^s$ is smooth and has a holomorphic symplectic form coming from
    \[ \Ext^1(F, F) \times \Ext^1(F, F) \xrightarrow{\cup} \Ext^2(F, F) \xrightarrow{\mr{tr}} H^2(\mc{O}_X) = H^2(\Omega^2_X) = k. \]
    This is in fact induced from Serre duality.
\end{thm}

The first step is to prove that there exists an isomorphism of sheaves
\[ TM^s = \Ext^1_{p_1}(\wt{F}, \wt{F}), \]
where $\Ext^1_f(F, -) = R^i f_* \circ \Hom(F, -)$. We need to consider the Kodaira-Spencer map and the Atiyah class. Given a coherent sheaf $F$ on $Y$, we have
\[ 0 \to \mc{I} / \mc{I}^2 \to \mc{O}_{2 \Delta} \to \mc{O}_{\Delta} \to 0. \]
Considering $p_{1*}(-) \otimes p_2^* F$, the \textit{Atiyah class} of $F$ is
\[ [0 \to \Omega^1_Y \otimes F \to p_{1*}(p_2^*F \otimes \mc{O}_{2 \Delta}) \to F \to 0] = A(F) \in \Ext^1(F, F \otimes \Omega^1_Y). \]
Now given $A^i(F) \in \Ext^i(F, F \otimes \Omega_Y^i)$, we will see that $\tr(A^i(F)) = \mr{ch}_i(F)$.

Given $F$ on $S \times X$ flat over $S$, we consider
\[ \Ext^1(F, F \otimes \Omega^1_{S \times X}) \to H^0(S, \Ext_{p_1}^1(F, F \otimes \Omega^1_{S \times X})) \to H^0(S, \Ext^1_{p_1}(F, F \otimes \Omega^1_S). \]
Note that $\Omega^1_{S \times X} = p_1^* \Omega^1_S \oplus p_2^* \Omega^1_X$. Now we can define $KS_F \colon T_S \to \Ext^1_{p_1}(F, F)$.

The next step is to see that if $F$ is a family of stable sheaves on $S \times X$, then $p_{1*} \Hom(F, F) = H^0(\mc{O}_X) \otimes \mc{O}_S$. Moreover, we have a series of isomorphisms
\[ H^2(\mc{O}_X) \xrightarrow{\delta} \Ext^2_{p_1}(F, F) \xrightarrow{\tr} H^2(\mc{O}_X) \otimes \mc{O}_S. \]
We now apply this on the Quot scheme and descend, and we deduce that
\[ \pi^* T_{M^s} = \Ext^1_{p_1}(\wt{F}, \wt{F}). \]

Before we continue, we review some things about correspondences. Let $X$ be smooth and projective. Suppose $\Gamma \in \mr{CH}^k(S \times X)$, where $S$ is smooth and quasiprojective. Then there exists a lift
\[ \Gamma \in \mr{CH}^k(\ol{S} \times X) \twoheadrightarrow \mr{CH}^k(S \times K) \ni \Gamma. \]
But then there is a cycle map to $H^{2k}(\ol{S} \times X)$, and we obtain a cohomology class $[\ol{\Gamma}]$. Then we obtain a map
\[ H^*(X) \xrightarrow{p_2^*} H^*(\ol{S} \times X) \xrightarrow{\cup [\ol{\Gamma}]} H^{*+2k}(\ol{S} \times X) \xrightarrow{p_{1*}} H^{*+2k-2\dim X}(\ol{S}). \]
Because $[\ol{\Gamma}]$ is a Hodge class, this is a morphism of Hodge structures. Recall that by the K\"unneth formula, we have
\[ H^{2k}(\ol{S} \times X) = \bigoplus H^j(\ol{S}) \otimes H^{2k-j}(X), \]
and by Poincar\'e duality we have $H^{2k-j}(X) = H^{2n-2k+j}(X)^{\vee}$. Thus we have 
\[ \ol{\Gamma}_j \in \Hom(H^{2n-2k+j(X), H^j(\ol{S})}). \]
In general, we will view $\ol{\Gamma}^*$ in $H^k(S \times X, \Omega^k_{S \times X})$. Recall that
\[ \Omega^k_{S \times X} = \bigoplus p_1^* \Omega_S^{k-j} \otimes p_2^* \Omega_X^j, \]
and therefore $\Gamma \in H^k(\Omega^k_{S \times X}) = \bigoplus H^i(\Omega^j_S) \otimes H^{k-i}(\Omega^{k-j}_X)$. By Serre duality, $H^{k-i}(\Omega_X^{k-j}) \simeq H^{n-k+i}(\Omega_X^{n-k+j})^{\vee}$, and so we have $H^{n-k+i}(\Omega_X^{n-k+j}) \to H^i(\Omega_S^j)$. If $\dim X = 2$, we can choose $k=2, i=0$, and so we have
\[ H^0(\Omega_X^j) \to H^0(\Omega_S^j). \]

As a corollary, for all $\alpha \in H^0(\Omega_X^j)$ a holomorphic $j$-form, $\Gamma^*(\alpha)$ is a holomorphic $j$-form on $S$ that extends to a $j$-form on any smooth projective compactification $\ol{S}$ of $S$. In particular, $\Gamma^* (\alpha)$ is closed.

Now let $F$ be a sheaf on $S \times X$ that is a flat family of sheaves on $X$. Then call $\gamma^2(F) = \tr (A^2(F))$, where $A^2$ is the Atiyah class. This was introduced by Atiyah for vector bundle and Illusie for coherent sheaves. Here, $A(F) \in \Ext^1(F, F \otimes \Omega_{S \times X})$. Thus $\gamma^i(F) \in H^i(\Omega^i_{S \times X})$. If $L$ is a line bundle, then $\gamma^i(L) \in H^1(\Omega^1_L)$ is up to a multiple $c_1(L)$. Also, $\frac{\gamma^i(F)}{i!} = \mr{ch}_i(F)$. Now if we consider $\gamma^2(F) \in H^2(\Omega^2_{S \times X})$, considering the correct K\"unneth component gives us a morphism
\[ \tau_F \colon H^0(\Omega^2_X) \to H^0(\Omega^2_S). \]

We will now assume that $F$ is a family of stable sheaves.
\begin{lem}
    For all smooth points $s \in S$ and all $\alpha \in H^0(\Omega^2_X)$, the holomorphic $2$-form $\tau_F(\alpha) \in H^0(\Omega^2_S)$ induces the alternating bilinear pairing given by
    \begin{equation*}
    \begin{tikzcd}
        T_{S,s} \times T_{S,s} \ar{r} \ar{d}{\mr{KS}_{F,s} \times \mr{KS}_{F,s}} & k \\
        \Ext^1(F_s, F_s) \times \Ext^1(F_s, F_s) \ar{d}{\cup} & H^2(\Omega^2_X) \ar{u}{\simeq} \\
        \Ext^2(F_s, F_s) \ar{r} & H^2(\mc{O}_X). \ar{u}{\cup \alpha}
    \end{tikzcd}
    \end{equation*}
\end{lem}

\begin{thm}
    Let $M_0^s$ be the stable locus where the trace map is an isomorphism and $\mc{E}$ be a quasi-universal family on $M_0^S \times X$. Then $\frac{\tau_{\mc{E}}}{\on{rk} \mc{E}} \colon H^0(\Omega^2_X) \to H^0(\Omega^2_{M_0^s})$ is independent of $\mc{E}$. Moreover, for all $\alpha \in H^0(\Omega_X^2)$, $\tau(\alpha)$ is nondegenerate at $[F] \in M_0^s$ if and only if $\Ext^1(F, F) \xrightarrow{\alpha} \Ext^1(F, F \otimes \Omega^2_X)$ is an isomorphism.
\end{thm}

\begin{proof}
    First, let $\mc{E}, \mc{E}'$ be quasi-universal families. Then we know that $\mc{E} \otimes p_1^* W = \mc{E}' \otimes p_1^* W'$ for some $W, W'$. But now we have
    \[ A(\mc{E} \otimes p_1^* W) = A(\mc{E}) \otimes \mr{id}_{p_1^* W} + \mr{id}_{\mc{E}} \otimes p_1^* A(W). \]
    When we consider the component $H^0(\Omega_X^2) \to H^0(\Omega_S^2)$, only the $A(\mc{E})$ term contributes. Finally, if we consider the traces, we obtain the desired result.

    The second part of this follows from Serre duality. Here, we have
    \begin{equation*}
    \begin{tikzcd}
        \Ext^i(F, E \otimes K_X) \otimes \Ext^{n-i}(E, F) \ar{r} & \Ext^n(E, E \otimes K_X) \ar{r}{\tr} & H^n(K_X) \ar{r}{\sim} & k \\
        \Ext^i(F, E) \otimes \Ext^{n-i}(E, F) \ar{r} \ar{u}{\cup \alpha \otimes \mr{id}} & \Ext^n(E, E) \ar{u}{\cup \alpha} \ar{r}{\tr} & H^n(\mc{O}_X) \ar{u}{\cup \alpha}.
    \end{tikzcd}
    \end{equation*}
    This must commute, and thus our pairing must have been nondegenerate.
\end{proof}

\begin{cor}
    Let $X$ be a K3 surface or an abelian surface. Then $M^s$ is smooth and has a holomorphic symplectic form.
\end{cor}

We now want to resolve the following questions:
\begin{enumerate}
    \item What is $\dim M^s$? Is it even non-empty?
    \item When does $M^s = M$? Is it irreducible?
    \item What happens on $M \setminus M^s$? When does there exist a symplectic resolution $\wt{M} \to M$?
\end{enumerate}
We will focus on the case when $X$ is a K3 surface. Then $\dim M^s = \dim Ext^1(F, F)$, where $F$ is a stable sheaf. But now $\dim \Hom(F, F) = \dim \Ext^2(F, F) = 1$, we have $\dim M^s = - \chi(F, F) + 2$. Using Grothendieck-Riemann-Roch, we see that
\[ \chi(F, F) = \on{ch}(F) \on{ch}(F^{\vee}) \on{td}(X). \]

\begin{defn}
    Define $v(F) \coloneqq \on{ch}(F) \cdot \sqrt{\on{td}(X)} \in H^*(X, \Z)$. This is called the \textit{Mukai vector} of $F$ and is equal to $\qty(r, c_1, \frac{c_1^2}{2} - c_2 + r)$.
\end{defn}

For $v = (a,b,c), w = (a',b',c') \in H^*(S, \Z)$, we write $v \cdot w = (v, w^{\vee}) = bb' -ac'-a'c$, where $v^{\vee} = (a,-b,c)$. Now we have $\chi(F, F) = (v(F), v(F)^{\vee}) = -v^2$. Thus $\dim M^s = v^2 + 2$.

We will now discuss $v$-generic polarizations. Let $H$ be a polarization on $X$ and let $F$ be a $\mu_H$-semistable sheaf on $X$. Let $F' \subseteq F$ be such that $\mu_H(F') = \mu_H(F)$. Define $\xi_{F'} = c_1(F) \cdot r' - c_1(F') \cdot r$. Then our condition that $\mu_H(F') = \mu_H(F)$ is equivalent to $\xi_{F'} \cdot H = 0$, and by the Hodge index theorem, $\xi_{F'}^2 \leq 0$. Of course, this means that $\xi_{F'} = 0$, and in fact this is equivalent to $\xi_{F'} \cdot H = 0$.

\begin{defn}
    The locus $\qty{\xi_{F'} \cdot x = 0} \subseteq \on{Amp}(X)_{\R}$ (or in $\mc{H}$ a cross-section) is called the \textit{wall} associated to $F'$.
\end{defn}

\begin{thm}
    The walls of $v(F)$ are \textit{locally finite} in $\mc{H}$.
\end{thm}

\begin{proof}
    We have $\frac{-r^2 \Delta(F)}{4} \leq \xi^2 \leq 0$, where $\Delta(F) = c_2(\End(F))$. This gives local finiteness.
\end{proof}

Now $H$ is called \textit{generic} if $H$-semistability implies $H$-stability. There may not always exist such a polarization, but if $v \in H^*(S, \Z)$ is a primitive element, then a $v$-generic polarization does exist.

\begin{thm}[Yoshioka, O'Grady]
    Let $(X, H)$ be a polarized K3 surface and $v$ a primitive Mukai vector such that $v^2 \geq -2$. Then $M_{v,H}$ is smooth, projective, nonempty, nonempty, irreducible, and connected of dimension $v^2 + 2$ and is deformation to $X^{[v^2/2 + 1]}$.
\end{thm}

This is proved by first deforming the K3 surface to an elliptic K3 and then by using Fourier-Mukai transforms to get birational maps of moduli spaces.

\begin{thm}[O'Grady, KLS, PR]
    Let $(S, H)$ be a polarized K3 surface. Suppose that $v = m v_0$, where $v_0$ is primitive. Suppose that $H$ is $v_0$-generic and $v_0^2 \geq 2$. Then $M_{m v_0}$ is a singular symplectic variety which has a resolution if and only if $m = 2$ and $v_0^2 = 2$. In this case, then the resolution $\wt{M}_{2 v_0}$ is irreducible holomorphic symplectic of OG10 type.
\end{thm}

\begin{rmk}
    If $v_0^2 = 0$, then the moduli space is a symmetric power of a K3 and if $v_0^2 = -2$, the moduli space is a point.
\end{rmk}

Note that $b_2(OG10) = 24 > b_2(S^{[5]})$.

\chapter{Bonus: cubic fourfolds}%
\label{cha:bonus_cubic_fourfolds}

Let $X \subseteq \P^5$ be a smooth cubic hypersurface. By the Lefschetz hyperplane theorem, $H^k(\P^5, \Z) \to H^k(X, \Z)$ is an isomorphism for $k < 4$ and is injective for $k = 4$. The only mystery is $H^4(X, \Z)$. Note that $H^4 = 3$, so if $k > 2$, then $H^{2k}(X, \Z) = \Z \frac{H^k}{3}$. We know that $H^4(X, \Z)$ is torsion free and that the cup product is an odd unimodular lattice. But now we compute $\chi^{\mr{top}}(X) = c_4(X)$, and here we simply use
\[ 0 \to T_X \to T_{\P^5}|_X \to \mc{O}_X(3) \to 0, \]
and therefore $c_4(X) = 27$. In particular, we have $b_4 = 27-4 = 23$. Because $X$ is Fano by the adjunction formula, we have
\[ H^4(X, \C) = H^{3,1} \oplus H^{2,2} \oplus H^{1,2}. \]

\begin{prop}
    $H^1(\Omega^3_X)$ has dimension $1$ and under the natural isomorphism between $\Omega^3_X \simeq T_X(-3)$, the generator corresponds up to a multiple to the extension class of
    \[ 0 \to T_X \to T_{\P^5}|_X \to \mc{O}(3) \to 0 \]
    in $\Ext^1(\mc{O}(3), T_X) = H^1(T_X(-3))$.
\end{prop}

This is proved using Bott vanishing for $H^p(\Omega^q_{\P^n}(k))$ and standard exact sequences. We also obtain $H^3(\Omega_X^1) \simeq H^4(\mc{O}_X(-3))$.

Now let $\Gamma \subseteq S \times X$ be a family of curves on $X$ parameterized by $S$. Then for any smooth projective model $\wt{S}$ of $S$, there exists a holomorphic $2$-form $\wt{\Gamma}^*(\eta) \in H^0(\Omega^2_{\wt{S}})$. Here, we have a map $H^4(X) \to H^2(S)$, and then $H^{3,1} \mapsto H^{2,0}$ because this shifts the Hodge structure down by $(1,1)$.


\section{Fano variety of lines on $X$}%
\label{sec:fano_variety_of_lines_on_x_}

The main theorem is the following
\begin{thm}[Beauville-Donagi]
    Let $X$ be a cubic fourfold. Then the Hilbert scheme $F(X)$ of lines in $X$ is a smooth connected irreducible holomorphic symplectic fourfold deformation equivalent to $K3^{[2]}$.
\end{thm}

Let $X_d \subseteq \P^n$ and write $V = H^0(\mc{O}_{\P^n}(1))^{\vee}$. Then $F(X_d) \subseteq \mr{Gr}(2, n+1)$. Using the tautological exact sequence
\[ 0 \to \mc{S} \to V \otimes \mc{O}_{\mr{Gr}} \to \mc{Q} \to 0, \]
dualizing, and taking symmetric powers, we have
\[ \on{Sym}^d V^{\vee} \otimes \mc{O}_{Gr} \twoheadrightarrow \on{Sym}^d \mc{S}^{\vee}. \]
This takes $X \in \on{Sym}^d V^{\vee}$ to $f_{\mc{S}}$. thus $F(X_d) = \qty{f_{\mc{S}} = 0}$. This has expected dimension $4$.

We now compute the tangent space. If $\ell \subseteq X$, we have the normal bundle
\[ 0 \to N_{\ell/X} \to N_{\ell/\P^5} \to N_{X/\P^5}|_{\ell} \to 0. \]
The terms are $\bigoplus_{i=1}^3 \mc{O}(a_i)$, where $a_i \leq 1$ and $\sum a_i = 1$, $\mc{O}(1)^{\oplus 4}$, and $\mc{O}(3)$. The first term could be either $\mc{O}(1) \oplus \mc{O} \oplus \mc{O}$ or $\mc{O}(1)^{\oplus 2} \oplus \mc{O}(1)$. The first case are lines of the first kind and form an open subset, and the lines of the second kind form a closed nonempty subset. Because $h^0(N_{\ell/X}) = 4$ and $h^1(N_{\ell/X}) = 0$, the deformations are unobstructed, and so $F(X)$ is smooth of the expected dimension. The next step is the following:

\begin{prop}
    For any cubic fourfold $\omega_{F(X)} = \mc{O}_{F(X)}$. In the case when $X$ is a cubic threefold, then $\omega_{F(X)} = h_{F(X)}$, where $h$ is the Pl\"ucker restricted to $F(X)$.
\end{prop}

To see this, we know that $\omega_{F(X)} = \omega_{\mr{Gr}} \otimes \det N_{F/\mr{Gr}}$. This becomes $\det \Hom(\mc{S}, \mc{Q}) \otimes \on{Sym}^3 \mc{S}^{\vee}$, and in the dimension $4$ case, this is $\mc{O}(-6h) \otimes \mc{O}(6h) = \mc{O}_{F(X)}$.

To prove that $F(X)$ is connected, let $\Gamma \subseteq F(X) \times X$ be the universal family of lines. Then $p \colon \Gamma \to F(X)$ is a $\P^1$-bundle if course, and we claim that $q \colon \Gamma \to X$ is a fibration in $(2,3)$ complete intersections in $\P^3$ and in particular, this means that $F(X)$ is connected.

We know that $p^{-1}(x)$ is the locus of all lines containing $x$. Then the locus swept by $\ell \ni x$ is the intersection of $Q_x \cap H_x \cap X$, where $Q_x$ is a quadric cone and $H_x$ is a hyperplane. Choosing local coordinates for $X \cap H_x \subseteq H_x = \P^4$, we may choose $x = [0,0,0,0,1]$. Then $X$ is given by
\[ f_2(x,y,z,u) \cdot v + f_3(x,y,z,u), \]
and then $X \cap H_x = \qty{f_2 = f_3 = 0}$.

\begin{rmk}
    The same shows that if $X_0$ is a cubic fourfold with a node ($A_1$ singularity) $o$, then the set of lines through $o \in X_0$ is a $(2,3)$ complete intersection in $\P^4$ and is therefore a K3 surface.
\end{rmk}

We will finally prove that $F(X)$ is irreducible holomorphic symplectic and deformation equivalent to $S^{[2]}$. Let $\mc{X} \to \Delta$ be a family such that $X_t$ is a smooth cubic fourfold for $t \neq 0$ and $X_0$ is a cubic fourfold with a single $A_1$ singularity. Then we have $F(X_0) \subseteq F(\mc{X}/\Delta)$.

\begin{lem}
    Let $S$ be the K3 surface of lines passing through the node. Then $F(X_0) \simeq \on{Sym}^2(S)$.
\end{lem}

Here, there is a unique plane passing through both a line not containing the node and the node, and the intersection of this $\P^2$ with this line is part of a triangle of lines. Up to a $2:1$ base change of $\Delta$, we can resolve the central fiber. The upshot is that there exists $\wt{F} \to \Delta$ such that $\wt{F}_t = F(X_t)$ for $t \neq 0$ and $\wt{F}_0 = S^{[2]}$.

Now if $\mc{M} \to \Delta$ is a projective morphism and $\mc{M}_0$ is birational to a hyperk\"ahler and the $\mc{M}_t$ is hyperk\"ahler, then up to a base change, we can resolve the central fiber such that $\wt{\mc{M}}_0$ is smooth hyperk\"ahler.

\begin{thm}[Beauville-Donagi]
    The Abel-Jacobi map
    \[ \Gamma^* \colon H^4(X, \Z) \to H^2(F(X), \Z) \]
    is an isomorphism of Hodge structures over $\Z$, and more specifically, 
    \[ (H^4(X, \Z)_0, \cup) \xrightarrow{\sim} (H^2(F(X), \Z), q)(-1) \]
    is an isomorphism of lattices.
\end{thm}

\begin{cor}
    The set of $(F(X), h)$ forms a codimension $1$ locus in the moduli space of irreducible holomorphic symplectic manifolds of their deformation type.
\end{cor}

For very general $X$, the Neron-Severi group of $F(X)$ is $\Z h$.

\begin{prop}
    Let $F \to S \times X$ be a flat family of sheaves. For all $\omega \in H^1(\Omega_X^3)$, $\tr(A^3(F))$ induces a holomorphic $2$-form on $S$ which pointwise is
    \begin{equation*}
    \begin{tikzcd}
        T_s S \times T_s S \ar{r}{KS \times KS} & \Ext^1(F_s, F_s) \times \Ext^1(F_s, F_s) \ar{r}{\cup} & \Ext^2(F, F) \ar[in=160,out=-20,swap]{dll}{\cup A(F)} & \\
        \Ext^3(F, F \otimes \Omega) \ar{r}{\tr} & H^3(\Omega^1) \ar{r}{\cup \omega} & H^4(\Omega^4) = k.
    \end{tikzcd}
    \end{equation*}
\end{prop}

This is proved by considering the K\"unneth component $H^1(\Omega^3_X) \to H^0(\Omega^2_S) \otimes H^3(\Omega^1_X)$.

Now suppose that $M$ is a moduli spaace of stable sheaves. Then a sufficient condition, due to Kuznetsov and Markusevich, for this form to be nondegenerate on a smooth point $[F]$ is that $H^{\bullet}(F) = H^{\bullet}(F(-1)) = H^{\bullet}(F(-2)) = 2$, and this is precisely the same as $F$ being in the Kuznetsov component of $D^b(X)$. This is because
\begin{prop}[Kuznetsov-Markusevich]
    The pairing
    \[ \Ext^i(F, G) \otimes \Ext^{2-i}(G, F) \to \Ext^2(G, G) \xrightarrow{\ep_G} \Ext^5(G, G \otimes \Omega_X^4) \xrightarrow{\tr} H^4(\Omega^4_X) \]
    is nondegenerate. Here, $\ep_G$ is the composition of $\mr{id}_G \otimes \nu \in \Ext^1(G \otimes \Omega^1_X, G \otimes \mc{O}(-3))$ and $A(G) \in \Ext^1(G, G \otimes \Omega^1_X)$.
\end{prop}


\end{document}
