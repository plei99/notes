\documentclass[leqno, openany]{memoir}
\setulmarginsandblock{3.5cm}{3.5cm}{*}
\setlrmarginsandblock{3cm}{3.5cm}{*}
\checkandfixthelayout

\usepackage{amsmath}
\usepackage{amssymb}
\usepackage{amsthm}
%\usepackage{MnSymbol}
\usepackage{bm}
\usepackage{accents}
\usepackage{mathtools}
\usepackage{tikz}
\usetikzlibrary{calc}
\usetikzlibrary{automata,positioning}
\usepackage{tikz-cd}
\usepackage{forest}
\usepackage{braket} 
\usepackage{listings}
\usepackage{mdframed}
\usepackage{verbatim}
\usepackage{physics}
\usepackage{stackengine} 
\usepackage{mathrsfs}
\usepackage{stmaryrd}

%font
\usepackage[osf]{mathpazo}
\usepackage{microtype}

%CS packages
\usepackage{algorithmicx}
\usepackage{algpseudocode}
\usepackage{algorithm}

% typeset and bib
\usepackage[english]{babel} 
\usepackage[utf8]{inputenc} 
\usepackage[backend=biber, style=alphabetic]{biblatex}
\usepackage[bookmarks, colorlinks, breaklinks]{hyperref} 
\hypersetup{linkcolor=black,citecolor=black,filecolor=black,urlcolor=black}

% other formatting packages
\usepackage{float}
\usepackage{booktabs}
\usepackage[shortlabels]{enumitem}
\usepackage{csquotes}
\usepackage{titlesec}
\usepackage{titling}
\usepackage{fancyhdr}
\usepackage{lastpage}
\usepackage{parskip}

\usepackage{lipsum}

% delimiters
\DeclarePairedDelimiter{\gen}{\langle}{\rangle}
\DeclarePairedDelimiter{\floor}{\lfloor}{\rfloor}
\DeclarePairedDelimiter{\ceil}{\lceil}{\rceil}


\newtheorem{thm}{Theorem}[section]
\newtheorem{cor}[thm]{Corollary}
\newtheorem{prop}[thm]{Proposition}
\newtheorem{lem}[thm]{Lemma}
\newtheorem{conj}[thm]{Conjecture}
\newtheorem{quest}[thm]{Question}

\theoremstyle{definition}
\newtheorem{defn}[thm]{Definition}
\newtheorem{defns}[thm]{Definitions}
\newtheorem{con}[thm]{Construction}
\newtheorem{exm}[thm]{Example}
\newtheorem{exms}[thm]{Examples}
\newtheorem{notn}[thm]{Notation}
\newtheorem{notns}[thm]{Notations}
\newtheorem{addm}[thm]{Addendum}
\newtheorem{exer}[thm]{Exercise}

\theoremstyle{remark}
\newtheorem{rmk}[thm]{Remark}
\newtheorem{rmks}[thm]{Remarks}
\newtheorem{warn}[thm]{Warning}
\newtheorem{sch}[thm]{Scholium}


% unnumbered theorems
\theoremstyle{plain}
\newtheorem*{thm*}{Theorem}
\newtheorem*{prop*}{Proposition}
\newtheorem*{lem*}{Lemma}
\newtheorem*{cor*}{Corollary}
\newtheorem*{conj*}{Conjecture}

% unnumbered definitions
\theoremstyle{definition}
\newtheorem*{defn*}{Definition}
\newtheorem*{exer*}{Exercise}
\newtheorem*{defns*}{Definitions}
\newtheorem*{con*}{Construction}
\newtheorem*{exm*}{Example}
\newtheorem*{exms*}{Examples}
\newtheorem*{notn*}{Notation}
\newtheorem*{notns*}{Notations}
\newtheorem*{addm*}{Addendum}


\theoremstyle{remark}
\newtheorem*{rmk*}{Remark}

% shortcuts
\newcommand{\Ima}{\mathrm{Im}}
\newcommand{\A}{\mathbb{A}}
\newcommand{\N}{\mathbb{N}}
\newcommand{\R}{\mathbb{R}}
\newcommand{\F}{\mathbb{F}}
\newcommand{\C}{\mathbb{C}}
\newcommand{\Z}{\mathbb{Z}}
\newcommand{\Q}{\mathbb{Q}}
\newcommand{\I}{\mathbb{I}}
\renewcommand{\k}{\Bbbk}
\renewcommand{\P}{\mathbb{P}}
\newcommand{\M}{\overline{M}}
\newcommand{\g}{\mathfrak{g}}
\newcommand{\h}{\mathfrak{h}}
\newcommand{\n}{\mathfrak{n}}
\renewcommand{\b}{\mathfrak{b}}
\newcommand{\ep}{\varepsilon}
\newcommand*{\dt}[1]{%
   \accentset{\mbox{\Huge\bfseries .}}{#1}}
\renewcommand{\abstractname}{Official Description}
\newcommand{\mc}[1]{\mathcal{#1}}
\newcommand{\T}{\mathbb{T}}
\newcommand{\mf}[1]{\mathfrak{#1}}
\newcommand{\mr}[1]{\mathrm{#1}}
\newcommand{\ms}[1]{\mathsf{#1}}
\newcommand{\mscr}[1]{\mathscr{#1}}
\newcommand{\ol}[1]{\overline{#1}}
\newcommand{\ul}[1]{\underline{#1}}
\newcommand{\wt}[1]{\widetilde{#1}}
\newcommand{\wh}[1]{\widehat{#1}}

\DeclareMathOperator{\Der}{Der}
\DeclareMathOperator{\Hom}{Hom}
\DeclareMathOperator{\End}{End}
\DeclareMathOperator{\ad}{ad}
\DeclareMathOperator{\Aut}{Aut}
\DeclareMathOperator{\Gal}{Gal}
\DeclareMathOperator{\Rad}{Rad}
\DeclareMathOperator{\supp}{supp}
\DeclareMathOperator{\Supp}{Supp}
\DeclareMathOperator{\sgn}{sgn}
\DeclareMathOperator{\spec}{Spec}
\DeclareMathOperator{\Spec}{Spec}
\DeclareMathOperator{\Ext}{Ext}
\DeclareMathOperator{\Tor}{Tor}
\DeclareMathOperator{\Ann}{Ann}
\DeclareMathOperator{\Ass}{Ass}
\DeclareMathOperator{\dpth}{depth}
\DeclareMathOperator{\pdim}{proj.dim}
\DeclareMathOperator{\idim}{inj.dim}
\DeclareMathOperator{\gdim}{gl.dim}
\DeclareMathOperator{\Cl}{Cl}
\DeclareMathOperator{\Pic}{Pic}
\DeclareMathOperator{\Mod}{\mathsf{Mod}}
\DeclareMathOperator{\Ind}{Ind}
\DeclareMathOperator{\Inf}{Inf}
\DeclareMathOperator{\Cor}{Cor}
\DeclareMathOperator{\Nm}{Nm}
\DeclareMathOperator{\coker}{coker}
\DeclareMathOperator{\Frob}{Frob}

% Section formatting
\titleformat{\section}
    {\Large\sffamily\scshape\bfseries}{\thesection}{1em}{}
\titleformat{\subsection}[runin]
    {\large\sffamily\bfseries}{\thesubsection}{1em}{}
\titleformat{\subsubsection}[runin]{\normalfont\itshape}{\thesubsubsection}{1em}{}

\title{COURSE TITLE}
\author{Lectures by INSTRUCTOR, Notes by NOTETAKER}
\date{SEMESTER}

\newcommand*{\titleSW}
    {\begingroup% Story of Writing
    \raggedleft
    \vspace*{\baselineskip}
    {\Huge\itshape Algebraic Number Theory \\ Spring 2021}\\[\baselineskip]
    {\large\itshape Notes by Patrick Lei}\\[0.2\textheight]
    {\Large Lectures by Chao Li}\par
    \vfill
    {\Large \sffamily Columbia University}
    \vspace*{\baselineskip}
\endgroup}
\pagestyle{simple}

\chapterstyle{ell}


%\renewcommand{\cftchapterpagefont}{}
\renewcommand\cftchapterfont{\sffamily}
\renewcommand\cftsectionfont{\scshape}
\renewcommand*{\cftchapterleader}{}
\renewcommand*{\cftsectionleader}{}
\renewcommand*{\cftsubsectionleader}{}
\renewcommand*{\cftchapterformatpnum}[1]{~\textbullet~#1}
\renewcommand*{\cftsectionformatpnum}[1]{~\textbullet~#1}
\renewcommand*{\cftsubsectionformatpnum}[1]{~\textbullet~#1}
\renewcommand{\cftchapterafterpnum}{\cftparfillskip}
\renewcommand{\cftsectionafterpnum}{\cftparfillskip}
\renewcommand{\cftsubsectionafterpnum}{\cftparfillskip}
\setrmarg{3.55em plus 1fil}
\setsecnumdepth{subsection}
\maxsecnumdepth{subsection}
\settocdepth{subsection}

\begin{document}
    
\begin{titlingpage}
\titleSW
\end{titlingpage}

\thispagestyle{empty}
\section*{Disclaimer}%
\label{sec:disclaimer}

These notes were taken during lecture using the \texttt{vimtex} package of the editor \texttt{neovim}. 
Any errors are mine and not the instructor's. 
In addition, my notes are picture-free (but will include commutative diagrams) and are a mix of my mathematical style and that of the instructor.
If you find any errors, please contact me at \texttt{plei@math.columbia.edu}.

\newpage

\tableofcontents


\chapter{Motivation}%
\label{cha:motivation}

Here is a very classical question (that the ancients were interested in):

\begin{quest}
    Which prime numbers $p$ can be written as $p = x^2 + y^2$ for integers $x,y$?
\end{quest}

We can try to answer this by experiment. Clearly, $2=1+1,5=1+4,13=4+9,17=1+16$ and the other primes below $20$ cannot be written as a sum of two squares. Then, because any square is congruent to $0$ or $1$ modulo $4$, we see that if $p$ is an odd prime, then

\begin{thm}[Fermat, Christmas Day, 1640]
    An odd prime $p$ can be written as a sum of two squares if and only if $p \equiv 1 \pmod 4$.\footnote{Of course, Fermat never actually proved anything, and this statement was proved by Euler in the 1740s.} Similarly, we have:
\begin{itemize}
    \item $2 \neq p = x^2 + 2y^2$ if and only if $p \equiv 1,3 \pmod 8$.
    \item $3 \neq p = x^2 + 3y^2$ if and only if $p \equiv 1 \pmod 3$.
\end{itemize}
\end{thm}

In the modern day, we should reinterpret $p = x^2 + y^2$ as a factorization problem in the number field $k = \Q(i)$. Now we write our problem as $p = (x+iy)(x-iy)$. Similarly, the second problem can be written as $p = (x + \sqrt{-2} y)(x - \sqrt{-2} y)$ in $\Q(\sqrt{-2})$ and the third problem can be expressed in the field $\Q(\sqrt{-3})$. More generally, we can consider $k = \Q(\sqrt{d})$ for an arbitrary $d$, called the \textit{discriminant} of $k$. For a general quadratic extension, the ring of integers is not a UFD, but it is Dedekind, so we have unique factorization of prime ideals. Therefore we can write
\[ (p) = \begin{cases}
    \mf{p}_1 \mf{p}_2 & \qty(\frac{d}{p}) = 1 \\
    \mf{p} & \qty(\frac{d}{p}) = -1 \\
    \mf{p}^2 & \qty(\frac{d}{p}) = 0 (\text{ or } p \mid d).
\end{cases} \]
What we want to know is when the ideal $(p)$ splits, and this behavior is governed by the Legendre symbol. This symbol satisfies the miraculous identity (due to Gauss)

\begin{thm}[Quadratic Reciprocity]
    Let $p,q$ be odd primes. Then we have the identity
    \[ \qty(\frac{q}{p}) \qty(\frac{p}{q}) = {(-1)}^{\frac{p-1}{2} \frac{q-1}{2}}. \]
\end{thm}

Recall that if $p$ is odd and $p \nmid d$, then $\qty(\frac{d}{p}) = 1$ if and only if $x^2 \equiv d \pmod p$ has solutions. Quadratic reciprocity tells us that the equation $x^2 \equiv q \pmod p$ is highly related to the equation $x^2 \equiv p \pmod q$. This ability to change the modulus is very helpful in solving these classical problems.

\begin{exm}
    A prime $p$ splits in $\Q(\sqrt{-3})$ if and only if $\qty(\frac{-3}{p}) = 1$ if and only if $\qty(\frac{p}{3}) = 1$ if and only if $p \equiv 1 \pmod 3$.
\end{exm}

Now we can generalize our question about splitting in quadratic fields to more general fields:

\begin{quest}
    Is there a criterion of the form $p$ splits in $k$ if and only if $p \equiv * \pmod N$ for some $N$? If so, we can generalize quadratic reciprocity. This is one of the main questions of class field theory.
\end{quest}

\begin{exm}
    We have some examples of splitting behavior:
    \begin{itemize}
        \item $p$ splits in $k = \Q(\sqrt{-5})$ if and only if $p \equiv 1,3,7,9 \pmod 20$.
        \item $p$ splits in $k = \Q(\sqrt{-5}, i)$ if and only if $p \equiv 1,9 \pmod 20$.
        \item $p$ splits in $k = \Q(\zeta_5)$ if and only if $p \equiv 1 \pmod 5$.
    \end{itemize}
    However, there is no congruence condition for splitting in $k = \Q(\sqrt[3]{2})$ for any modulus $N$. The question is what is different about the last example. First, the fields $\Q(\sqrt{-5}), \Q(\sqrt{-5}, i), \Q(\zeta_5)$ are all Galois extensions of $\Q$ with Galois groups $\Z/2, \Z/2 \times \Z/2, \Z/4$. On the other hand, $k = \Q(\sqrt[3]{2})$ is not Galois and its Galois closure $\Q(\sqrt[3]{2}, \zeta_3)$ has Galois group $S_3$. 
\end{exm}

\begin{defn}
    A field extension $L/K$ is called \textit{abelian} if $L/K$ is Galois and $\Gal(L/K)$ is abelian.
\end{defn}

This gives us the following slogan: given a number field $k$, class field theory
\begin{enumerate}
    \item classifies all abelian extensions $L/K$ in an accessible way;
    \item describes factorization of primes of $K$ in $L$ in terms of groups intrinsic to $K$ (for example the class group of $K$).
\end{enumerate}

\section{A Special Case of CFT}%
\label{sec:a_special_case_of_cft}

Here, we will classify all \textit{unramified} abelian extensions $L/K$. Recall that the \textit{class group} $\Cl(K)$ of a number field $K$ is
\[ \Cl(K) \coloneqq \qty{\text{fractional ideals of $K$}} / \qty{\text{principal ideals of $K$}}. \]
This is always a finite abelian group.

\begin{defn}
    Let $H \subset \Cl(K)$ be a subgroup. Then a finite unramified abelian extension $L/K$ is a \textit{class field} for a subgroup $H \subset \Cl(K)$ if $\mf{p}$ splits in $L/K$ if and only if $[\mf{p}] \in H \subset \Cl(K)$.
\end{defn}

\begin{thm}[Unramified CFT]
    Given $H \subset \Cl(K)$, the class field for $H$ exists and is unique. Moreover, each finite unramified abelian extension arises as a class field. This gives us a bijection
    \[ \qty{\text{finite unramified abelian extensions}} \longleftrightarrow \qty{\text{subgroups of $\Cl(K)$}}. \]
    Moreover, $\Gal(L/K) \cong \Cl(K) / H$.
\end{thm}

\begin{defn}
    Note that the class field for $H = 0$ is the maximal unramified abelian extension $H_K$ of $K$, called the \textit{Hilbert class field}. This gives a canonical isomorphism $\Gal(H_K/K) \cong \Cl(K)$. Also, we see that $\mf{p}$ splits in $H_K$ if and only if $\mf{p}$ is a principal ideal.
\end{defn}

\begin{exm}
    For the fields $K = \Q, \Q(i), \Q(\sqrt{-2}), \Q(\sqrt{-3})$, we have $\Cl(K) = 0$ and thus $H_K = K$.
\end{exm}

\begin{exm}
    The simplest example of a number field with nontrivial class group is $K = \Q(\sqrt{-5})$. Here, $\Cl(K) = \Z/2$ and $H_K = \Q(\sqrt{-5}, i)$.
\end{exm}

\begin{rmk}
    More generally, class field theory will add ramification on both sides of the correspondence to obtain a correspondence
    \[ \qty{\text{finite abelian extensions}} \longleftrightarrow \qty{\text{subgroups of $\mscr{C}_K = \A_K^{\times}/K^{\times}$}}. \]
\end{rmk}

\section{Back to Fermat}%
\label{sec:back_to_fermat}

Consider the equation $p = x^2 + 5y^2$. Recall that we have $2,5 \neq p = x^2 + 5y^2$ if and only if $p \equiv 1,9 \pmod 20$. Note that this is \textbf{different} from the splitting behavior in $K = \Q(\sqrt{-5})$. This happens because $\Cl(\Q(\sqrt{-5})) = \Z/2$ is not trivial, and so we need both the condition that $(p) = \mf{p}_1 \mf{p}_2$ \textbf{and} that the $\mf{p} _i$ are principal. By unramified CFT, we know that $\mf{p}_i$ are principal if and only if they split in $H_K = \Q(\sqrt{-5}, i)$, and therefore $p = x^2 + 5y^2$ if and only if $p$ splits in $\Q(\sqrt{-5}, i)$. In this case, $H_K/\Q$ is abelian, so we have a nice answer.

\begin{exm}
    The primes of the form $p = x^2 + 14 y^2$ \textbf{cannot} be described in terms of a congruence condition. The field $K = \Q(\sqrt{-14})$ has $\Cl(K) = \Z/4$ and $\Gal(H_K/\Q) \cong D_4$ is \textbf{non-abelian}. 
\end{exm}

\begin{rmk}
    We can study non-abelian extensions to get some nice answers that involve modular forms, and this is called the \textit{Langlands program}, which is beyond the scope of this course. 
\end{rmk}

Our outline for the semester is to prove local CFT, then prove global CFT, then do applications if time permits. This will be done using group cohomology.

\chapter{Local Fields}%
\label{cha:local_fields}

Recall that it is very difficult to detect whether a polynomial equation over a global field like $\Q$ has solutions. However, we can embed $\Q$ into the local field $\R$ and then checking whether the polynomial has real solutions is very easy because we can do analysis. To try to recover all information about $\Q$, we can embed $\Q \hookrightarrow \Q_p$ for a prime $p$. We then have the following slogan, known as the local-to-global principle:

\begin{quotation}
\itshape We will study problems in $\Q$ by studying problems in all the local fields $\R$ and $\Q_p$.
\end{quotation}

\section{Absolute Values}%
\label{sec:absolute_values}

\begin{defn}
    Let $K$ be a field. An \textit{absolute value} on $K$ is a function $\abs{-} \colon K \to \R$ such that
    \begin{enumerate}
        \item $\abs{-}$ sends $K^{\times}$ to $\R_{>0}$ and $0$ to $0$.
        \item We have $\abs{xy} = \abs{x} \cdot \abs{y}$ for all $x,y \in K$.
        \item For all $x,y \in K$, we have $\abs{x+y} \leq \abs{x} + \abs{y}$.
    \end{enumerate}
\end{defn}

\begin{exm}
    The usual absolute value on $\R$ defines an absolute value in this semse. This induces an absolute value on $\Q \subseteq \R$ usually denoted by $\abs{-}_{\infty}$. This is known as the archimedean absolute value on $\Q$.

    Similarly, any embedding $K \xhookrightarrow{\sigma} \R$ or $K \xhookrightarrow{\sigma} \C$ induces an absolute value on $K$ defined by $\abs{x}_{\sigma} \coloneqq \abs{\sigma(x)}$.
\end{exm}

There is a different kind of absolute value that is not archimedian. Here, we will strengthen the triangle inequality.

\begin{defn}
    If $\abs{-}$ satisfies the ultrametric inequality
    \[ \abs{x+y} \leq \max \qty{\abs{x}, \abs{y}} \]
    then we say $\abs{-}$ is \textit{nonarchimedean}. 
\end{defn}

\begin{rmk}
    Recall that $\R$ satisfies the archimedean property: If $0 \neq x \in \R$ there exists $n \in \Z$ such that $\abs{nx} > 1$. This property fails for nonarchimedean absolute values because $\abs{nx} \leq \abs{x}$ for all $n \in \Z$. In fact, $\abs{-}$ is nonarchimedean if and only if the set $\qty{\abs{n \cdot 1}}_{n \in \Z}$ is bounded.
\end{rmk}

\begin{exm}
    Let $a \in \Q^{\times}$ and $p$ be a prime. Then define $\mr{ord}_p(a) \in \Z$ such that
    \[ a = \pm \prod_{p} p^{\mr{ord}_p(a)}. \]
    Now for any $c < 1$, we define 
    \[ \abs{a}_p \coloneqq c^{\mr{ord}_p(a)}. \]
    Then we simply need to check that $\abs{-}_p$ is a nonarchimedean absolute value on $\Q$. Here, it is easy to check the ultrametric inequality, and this absolute value is called the \textit{$p$-adic absolute value}. By convention, we will choose $c = p^{-1}$ and this is the \textit{normalized $p$-adic absolute value} on $\Q$.  
\end{exm}

\begin{exm}
    For any number field $K$ and prime ideal $\mf{p} \subseteq \mc{O}_K$. Then we have a normalized $\mf{p}$-adic absolute value
    \[ \abs{a}_{\mf{p}} \coloneqq \qty(\frac{1}{N\mf{p}})^{\mr{ord}_{\mf{p}}(a)} \]
    where $N\mf{p} = \# \mc{O}_K / \mf{p}$.
\end{exm}

\begin{defn}
    An absolute value $\abs{-}$ is \textit{discrete} if $\abs{K^{\times}} \subset \R$ is discrete under the usual topology on $\R$. 
\end{defn}

\begin{exm}
    For a number field $K$ and prime $\mf{p}$, the $p$-adic absolute value $\abs{-}_{\mf{p}}$ is discrete. On the other hand, $\abs{-}_{\infty}$ is not discrete.
\end{exm}

\begin{defn}
    Suppose $\abs{-}$ be nonarchimedean. Then define 
    \begin{enumerate}
        \item $A \coloneqq \qty{a \in K \mid \abs{a} \leq 1}$. This is a subring of $K$.
        \item Now define $A^{\times} = \qty{a \in K \mid \abs{a} = 1}$. This is a subgroup of $A$ of invertible elements.
        \item Set $\mf{m} = \qty{a \in K \mid \abs{a} < 1}$. This is the unique maximal ideal of $A$.
    \end{enumerate}
    Then $\abs{-}$ is discrete if and only if $\mf{m}$ is principal. In this case, a generator $\pi$ of $\mf{m}$ is called a \textit{uniformizer}. Then every $a \in K$ can be uniquely written as $a = \pi^r \cdot u$ for some $r \in \Z, u \in A^{\times}$.
\end{defn}

\begin{exm}(Non-example)
    Consider the field $\Q(\qty{p^{1/n}}, n \in \Z)$ with $p$-adic absolute value. This is not a discrete absolute value.
\end{exm}

\begin{defn}
    An absolute value defines a \textit{metric} on $K$ by $d(a,b) = \abs{a-b}$ for all $a,b \in K$. This induces a topology on $K$ where a basis of open neighborhoods of $a \in K$ is given by open balls 
    \[ B(a,r) \coloneqq \qty{x \in K \mid \abs{x-a} < r}. \]
\end{defn}

\begin{exm}
    In the $p$-adic topology, we see that $a,b \in \Q$ are closer under $\abs{-}_p$ if and only if $\abs{a-b}_{p}$ is smaller, which is equivalent to $\mr{ord}_p(a-b)$ being larger, which is equivalent to $a-b$ being divisible by a large power of $p$. In other words, $a \equiv b \pmod p^N$ for $N$ large.
\end{exm}

\begin{defn}
    We say two absolute values on $K$ are \textit{equivalent}, or $\abs{-} \sin \abs{-}'$ if they induce the same topology on $K$.
\end{defn}

\begin{thm}[Ostrowski, 1916]
    Let $\abs{-}$ be an absolute value on $\Q$.
    \begin{enumerate}
        \item If $\abs{-}$ is archimedean, then $\abs{-} \sim \abs{-}_{\infty}$.
        \item If $\abs{-}$ is nonarchimedean, then $\abs{-} \sim \abs{-}_p$ for a unique $p$.
    \end{enumerate}
\end{thm}

\begin{rmk}
    Similarly, absolute values on a number field $K$ are given by 
    \begin{enumerate}
        \item $\abs{-}_{\mf{p}}$ for a prime ideal $\mf{p} \subseteq \mc{O}_K$ ($\mf{p}$-adic place);
        \item $\abs{-}_{\sigma}$ for some $\sigma \colon K \hookrightarrow \R$ (real place);
        \item $\abs{-}_{\sigma}$ for some complex embeddings $\sigma \colon K \hookrightarrow \C$ (complex place). Here, note that complex embeddings come in conjugate pairs.
    \end{enumerate}
\end{rmk}

\begin{defn}
    An equivalence class of absolute values on $K$ is called a \textit{place (or prime)} of $K$.
\end{defn}

\begin{rmk}
    When $v$ is a complex place it corresponds to a pair of complex embeddings $\sigma, \ol{\sigma}$, so we define 
    \[ \abs{x}_v \coloneqq \abs{\sigma(x)}^2 \]
    and this is the \textit{normalized absolute value for $v$}. 
\end{rmk}

One reason for this normalization is

\begin{thm}[Product Formula]
    Let $K$ be a number field. Then for all $a \in K^{\times}$, we have
    \[ \prod_{v \text{ place of }K} \abs{a}_v = 1. \]
\end{thm}

\begin{rmk}
    When $K = \Q$, let $a = \frac{m}{n}$ for $m,n \in \Z$. Then all but finitely many terms in the product are finite. Now it suffices to check this formula for $a = p$ and $a = -1$.

    When $a = p$, we see that $\abs{a}_p = p^{-1}$ and for primes $\ell \neq p$, we have $\abs{a}_{\ell} = 1$. Finally, we see that $\abs{a}_{\infty} = p$, so the formula holds. When $a = -1$, all absolute values are $1$, so the product of all absolute values is trivial.

    For a general number field, we can simply take the norm map $N_{K/\Q}$ to $\Q$ and check that it behaves well with respect to the places.
\end{rmk}

\begin{thm}[Weak Approximation]
    Let $\abs{-}_1, \ldots, \abs{-}_n$ be inequivalent absolute values on a field $K$. Let $a_1, \ldots, a_n \in K$. Then for all $\ep > 0$, there exists $a \in K$ such that $\abs{a-a_i}_i < \ep$ for all $i = 1, \ldots, n$.
\end{thm}

\begin{rmk}
    This allows us to approximate any finite collection $a_i \in K$ for inequivalent $\abs{-}_i$.
\end{rmk}

\begin{rmk}
    As a sanity check, consider $K = \Q$ and suppose $\abs{-}_i = \abs{-}_{p_i}$. Then given $a_1, \ldots, a_n$, we simply find $a \in \Q$ such that $a_i \equiv a \pmod p_i^N$, which is possible by the Chinese remainder theorem.
\end{rmk}

\begin{rmk}
    More generally, if $\abs{-}_1 \nsim \abs{-}_2$ then one can choose $a \in K$ such that $\abs{a}_1 > 1$ and $\abs{a}_2 < 1$. Then if we consider $\frac{a^r}{1+a^r}$ as $r \to \infty$, the absolute value under $\abs{-}_1$ approaches $1$ and under $\abs{-}_2$ it approaches $0$.
\end{rmk}

\section{Completions}%
\label{sec:completions}

Consider the field $\Q$ equipped with the the absolute value $\abs{-}_{\infty}$. Then we can complete $\Q$ as a metric space to obtain the field $\R$. More generally, if $(K, \abs{-})$ is a field equipped with an absolute value (a \textit{valued field}), then we will produce a general completion $\wh{K}$. Our aim is to produce a field that contains the original field and whose arithmetic is easier to understand.

\begin{defn}
    Let $(K, \abs{-})$ be a valued field. Then a sequence $\qty{a_n}$ of elements in $K$ is called \textit{Cauchy} if for all $\ep > 0$, there exists $N \geq 1$ such that $\abs{a_n - a_m} < \ep$ for all $n,m > N$. We say that $K$ is \textit{complete} if any Cauchy sequence in $K$ has a limit in $K$.
\end{defn}

\begin{exm}
    Consider the sequence of integers $\qty{a_n = 2^n} = 2, 4, 8, 16, 32, \ldots$. Clearly, this is not Cauchy under the usual absolute value on $\Q$, but then if $m > n$, we see that 
    \[ \abs{a_n - a_m}_2 = \qty(\frac{1}{2})^n, \] 
    so $\qty{a_n}$ is Cauchy in $(\Q, \abs{-}_2)$. We then see that $\abs{a_n - 0}_2 = \qty(\frac{1}{2})^n \to 0$ and thus the limit of the sequence is $0$.
\end{exm}

\begin{exm}
    Consider $\qty{a_n} = \qty{4, 34, 334, 3334, \ldots}$. Then if $m > n$, we have
    \[ \abs{a_n - a_m}_5 = {\qty(\frac{1}{5})}^n \]
    and therefore $\qty{a_n}$ is Cauchy in $(\Q,\abs{-}_2)$. We then see that 
    \[ \abs{3 a_n - 2}_5 = \frac{1}{5^n} \xrightarrow{n \to \infty} 0 \]
    and therefore $a_n \to \frac{2}{3}$.
\end{exm}

\begin{rmk}
    In general, the limit of a Cauchy sequence may not exist.
\end{rmk}

\begin{thm}
    Let $(K, \abs{-})$ be a valued field. Then there exists a complete valued field $(\wh{K}, \abs{-})$ and an embedding $K \hookrightarrow \wh{K}$ of valued fields such that any other embedding $K \hookrightarrow L$ of $K$ into a complete valued field factors uniquely through $\wh{K}$. In particular, $\wh{K}$ is unique up to isomorphism and is called the \textit{completion} of $(K, \abs{-})$. 
\end{thm}

\begin{proof}
    Let $\wh{K}$ be the set of all Cauchy sequences in $K$ under the equivalence relation where $\qty{a_n} \sim \qty{b_n}$ where $\lim_{n \to \infty} \abs{a_n - b_n} = 0$. Then $\wh{K}$ is equipped with termwise addition and multiplication and absolute value $\abs{\qty{a_n}} = \lim_{n \to \infty} \abs{a_n}$. Thus $\wh{K}$ is a complete valued field.

    To verify the universal property, we see that $x \mapsto (x, x, \ldots, x)$ embeds $K \hookrightarrow \wh{K}$ and this satisfies the desired universal property.
\end{proof}

\begin{defn}
    Let $K$ be a number field and $v$ a place of $K$. Denote by $K_v \coloneqq (K, \abs{-}_v)$. When $v$ is a finite place, denote by $\mc{O}_{K_v} = \mc{O}_{K,v} = \mc{O}_v$ the valuation ring $\qty{x \in K_v \colon \abs{x} \leq 1} \subseteq K_v$. When $v$ is an infinite place, we see that $K_v \cong \R, \C$.
\end{defn}

\begin{exm}
    Let $K = \Z$. Then we see that $\Q_{\infty} = \R$ and $\Q_p$ has a subring $\Z_p$, which is a discrete valuation ring. Here, elements of $\Z_p$ have a nonnegative lowest power of $p$.
\end{exm}

\begin{exm}
    If $K = \Q(i)$, then $K_{\infty} = \C$ and $K_{\mf{p}}$ for $\mf{p} \subset \mc{O}_K = \Z[i]$ prime ideals are the completions of $K$.
\end{exm}

\begin{rmk}
    Let $K$ be a nonarchimedean discrete valued field. Then $\wh{K}$ is a complete discrete valued field and the valuation ring $\wh{A} \subset \wh{K}$ is the closure of $A \subseteq K$ in $\wh{K}$. Also, the maximal ideal $\wh{\mf{m}} \subseteq \wh{A}$ is the closure of $\mf{m} \subseteq A$ in $\wh{K}$. Finally, a uniformizer $\pi$ of $K$ is also a uniformizer of $\wh{K}$.

    We also see that the natural map $A / \mf{m}^n \to \wh{A} / \wh{\mf{m}}^n$ is an isomorphism. This tells us that we can approximate elements in $\wh{A}$ up to $\pi^n A$ using elements in $A$.
\end{rmk}

\begin{prop}
    Let $K$ be a discrete valued field. Let $S$ be a complete set of representatives of $A/\mf{m}$ and $\pi$ be a uniformizer of $K$. Then any element of $\wh{K}$ can be uniquely written as $a_k \pi^k + a_{k+1} \pi^{k+1} + \cdots$ where $a_i \in S$ and $k \in \Z$.
\end{prop}

\begin{cor}
    Let $x \in \Q_p$. Then $x$ has a \textit{$p$-adic expansion} $x = \sum_{i \geq k} a_i p^i$, where $a_i \in S = \Z/p\Z = \qty{0, 1, \ldots, p-1}$ and $k \in \Z$. 
\end{cor}

\begin{rmk}
    The main term of the $p$-adic expansion is the lowest term $p^k$. This is completely unlike the situation with the decimal digits of $x \in \R$, where the highest power of $10$ is the main term.
\end{rmk}

\begin{rmk}
    $\Q_p$ resembles $\F_p((T))$ but is more complicated arithmetically. When we add two power series, we simply add the coefficients, but addition in $\Q_p$ requires carrying. In addition, we see that $\F_p [[T]] / T^n \hookrightarrow \F_p[[T]]$ but $\Z_p / p^n$ does \textbf{not} embed in $\Z_p$. 
\end{rmk}

\begin{cor}
    We have an isomorphism
    \[ \wh{A} = \lim_{\longleftarrow} \wh{A} / \wh{\mf{m}}^n \cong \lim_{\longleftarrow} A/\mf{m}^n. \]
    For example, we have
    \[ \Z_p = \lim_{\longleftarrow} \Z/p^n \Z. \]
\end{cor}

\begin{proof}[Proof of Proposition 2.2.10]
    Let $x \in \wh{K}^{\times}$. Then $x = \pi^k \cdot y$ for some $k \in \Z$ and $y \in \wh{A}^{\times}$. Then we find the first digit $a_0$ by considering $y \equiv a_0$ in $A/\mf{m} = S$. Then we replace $\pi^{-1} (y-a_0) = a_1 + a_2 \pi^2 + \cdots$ and repeat the process. Repeating this, we simply use the completeness of $\wh{K}$ to obtain the desired expansion.
\end{proof}

The advantage of completeness is that it is much easier to solve equations. Here, we take solutions modulo a high power of $p$ and then take the limit.

\begin{thm}[Hensel's Lemma]
    Let $K$ be a complete discrete valued field and $k = A/\mf{m}$ be the residue field. Now let $f(x) \in A[x]$ be a monic polynomial and let $\ol{f}(x) \coloneqq f(x) \mod \mf{m} \in k[x]$. Assume that $\ol{f}(x) = g_0(x) h_0(x)$ in $k[x]$ where $g_0, h_0$ are monic and coprime. Then there exist unique $g,h \in A[x]$ such that $f(x) = g(x) h(x)$ and $\ol{g} \equiv g_0, \ol{h} = h_0$.
\end{thm}

\begin{cor}
    Suppose $\ol{f}$ has a simple root $\alpha_0 \in k$. Then $f(x)$ has a unique zero $\alpha \in A$ such that $\ol{\alpha} = \alpha_0$.
\end{cor}

\begin{cor}
    If $k = \F_q$ for $q = p^t$, then $f(x) = x^q - x$ has $q$ distinct roots in $k = \F_q$ and hence $q$ distinct roots in $K$. In particular, $K^{\times}$ contains all $(q-1)$-th roots of unity, so we have a map $\F_q^{\times }\hookrightarrow K^{\times}$, called the \textit{Teichmuller lift}. 
\end{cor}

\begin{proof}[Proof of Hensel's lemma]
    Let $g_0, g_0 \in A[x]$ be arbitrary monic lifts. Then $f - g_0 h_0 \in \pi A[x]$. Now inductiely we assume that there exist $g_n, h_n \in A[x]$ monic such that $f - g_n h_n \in \pi^{n+1} A[n]$. We simply write $g_{n+1} = g_n + \pi^{n+1} u$ for some $u \in A[x]$ such that $\deg u < \deg g_n$ and $h_{n+1} = h_n + \pi^{n+1} v$ where $\deg v < \deg h_n$. Then $f - g_{n+1} h_{n+1} \in \pi^{n+2} A$ if and only if 
    \[ u h_n - v g_n \equiv \frac{f - g_n h_n}{\pi^{n+1}} \pmod \pi \]
    but we can find such $u,v$ by Bezout's lemma. Thus the desired $g_{n+1}, h_{n+1}$ exist, and we obtain $g,h$ by taking the limit.
\end{proof}

\section{Extension of Absolute Values and Unramified Extensions}%
\label{sec:extension_of_absolute_values_and_unramified_extensions}

Let $K$ be a complete discrete valued field and $L$ be a finite separable extension of $K$. Suppose $[L:K] = n$. The main result is

\begin{thm}
    Let $K,L$ be as above. Then
    \begin{enumerate}
        \item $\abs{-}_K$ extends uniquely to a discrete absolute value $\abs{-}_L$ on $L$;
        \item $L$ is complete with respect to $\abs{-}_L$;
        \item For all $\beta \in L$, we have 
            \[ \abs{\beta}_L = \abs{N_{L/K}(\beta)}_K^{1/n}. \]
    \end{enumerate}
\end{thm}

\begin{rmk}
    To perform a sanity check, if $\beta \in K$, we have $\abs{\beta}_L = \abs{N_{L/K}(\beta)}_K^{1/n} = \abs{\beta^n}_K^{1/n} = \abs{\beta}_K$.
\end{rmk}

\begin{proof}\leavevmode
    \begin{enumerate}
        \item We first need to prove that a unique extension exists. We know that $\abs{-}_K$ comes from a discrete valuation (hence is nonarchimedean), so let $A \subseteq K$ be the valuation ring. Then $A$ is a Dedekind domain. Then let $B$ be the integral closure of $A$ in $L$. Then $B$ is also a Dedekind domain. But then any absolute value on $L$ extending $\abs{-}_K$ comes from a maximal iddeal of $B$ lying above the unique maximal ideal $\mf{p} \subseteq A$. Therefore, we need to show that $B$ is a local ring.

            To see this, assume not. Suppose there exist two prime ideals $\mf{P}_1, \mf{P}_2 \subseteq B$ lying above $\mf{p}$. Let $\beta \in \mf{P}$ but $\beta \notin \mf{P}_2$. This implies that $A[\beta] \cap \mf{P}_1 \neq A[\beta] \cap \mf{P}_2$. Let $f(x) \in A[x]$ be the minimal polynomial of $\beta$. Then $\ol{f}(x) = f(x) \mod \mf{p} \in A/\mf{p}[x] = k[x]$ must satisfy $\ol{f}(x) = {h(x)}^m$ for an irreducible $h(x) \in k[x]$ (otherwise it has two distinct irreducible factors and Hensel tells us that the factorization can be lifted to $A[x]$). This implies that
            \[ A[\beta] / \mf{p} A[\beta] = A[x]/(\mf{p}, f(x)) = k[x] / (\ol{f}(x)) = k[x]/(h(x))^m \]
            has a unique prime ideal, generated by $h(x)$, which contradicts our original assumption that $A[\beta] \cap \mf{P}_1, A[\beta] \cap \mf{P}_2$ were distinct prime ideals.
        \item Now we show that $L$ is complete. Let $\qty{a_k}$ be a Cauchy sequence in $L$. Choose a $K$-basis $\qty{e_1, \ldots, e_n}$ of $L$ and write
            \[ a_k = a_{1,k} e_1 + \cdots + a_{n,k} e_n \qquad a_{i,k} \in K. \]
            But then each sequence $\qty{a_{i,k}}_k$ forms a Cauchy sequence in $K$. By completeness of $K$, we can take $a_i \coloneqq \lim_{k \to \infty} a_{i.k} \in K$ and so we have
            \[ \lim_{k \to \infty} a_k = a_1 e_1 + \cdots + a_n e_n \in L, \]
            and thus $L$ is complete.
        \item Let $\wt{L}$ be the Galois closure of $L/K$. Then we know that $\abs{-}_K$ also extends uniquely to $\wt{L}$. For any $\sigma \in \Gal(\wt{L}/K)$, the map $L \in \beta \mapsto \abs{\sigma(\beta)}_{\wt{L}}$ is also an absolute value on $L$ extending $\abs{-}_K$. Therefore, by the uniqueness of the extension, we see that $\abs{\beta}_L = \abs{\sigma(\beta)}_{\wt{L}}$. This implies that 
            \[ \abs{N_{L/K}(\beta)}_K = \abs{N_{L/K}}_{\wt{L}} = \prod_{\sigma \colon L \to \wt{L}} \abs{\sigma(\beta)}_{\wt{L}} = \prod_{\sigma \colon L \to \wt{L}} \abs{\beta}_L = \abs{\beta}_L^n, \]
            as desired. \qedhere
    \end{enumerate}
\end{proof}

\begin{cor}
    If $L/K$ is merely an algebraic and separable extension, then $\abs{-}_K$ also extends uniquely to an absolute value on $L$, but $\abs{-}_L$ may fail to be discrete or complete.
\end{cor}

\begin{proof}
    Note that $L$ is the union of all of its finite subextensions.
\end{proof}

\begin{defn}
    Let $K$ be a complete discrete valued field and $L$ be a finite separable extension. Let $\mc{O}_K \subseteq K$ and $\mc{O}_L \subseteq L$ be the valuation rings and $\mf{p} \subseteq \mc{O}_K, \mf{P} \subseteq \mf{O}_L$ be the maximal ideals. Next, let $k = \mc{O}_K/\mf{p}, \ell = \mf{O}_L / \mf{P}$ be the residue fields. 

    Define the \textit{ramification index} $e(L/K)$ to be the $e \geq 1$ such that $\mf{p} \mc{O}_L = \mf{P}^e$. Next, define the \textit{residual degree} $f(L/K)$ to be $f \geq 1$ such that $f = [\ell : k]$. Then $n = ef$.  
\end{defn}

\begin{defn}
    The extension $L/K$ is called
    \begin{enumerate}
        \item \textit{Unramified} if $e(L/K) = 1$ (which implies $f(L/K) = n$ and $\mf{p} = \mf{P}$);
        \item \textit{Totally ramified} if $e(L/K) = n$ (which implies $\ell = k$ and $\mf{p} = \mf{P}^n$).
    \end{enumerate}
\end{defn}

\section{Unramified Extensions}%
\label{sec:unramified_extensions}

We will study unramified extensions. Here, we will try to understand $L/K$ via $\ell / k$.

\begin{prop}
    If $L/K$ is unramified, write $\ell = k(\alpha_0), \alpha_0 \in \ell$. Then for any $\alpha \in \mc{O}_L$ such that $\ol{\alpha} = \alpha_0$ we have $L = K(\alpha)$.
\end{prop}

\begin{proof}
    Let $f(x) \in \mc{O}_K[x]$ be the minimal polynomial of $\alpha$. Then $\deg \ol{f} = \deg f = [K(\alpha) : K] \leq [L:K]$. But then we know that $\deg \ol{f} \geq [k(\alpha_0) : k] = [\ell : k] = [L:K]$. But this implies that $\deg f = [L:K]$, so $K(\alpha) = L$.
\end{proof}

\begin{prop}
    If $L = K(\alpha)$ with minimal polynomial of $\alpha$ given by $f(x)$ such that $\ol{f}(x)$ has no repeated roots over $\ol{k}$, then $L/K$ is unramified.
\end{prop}

\begin{proof}
    If $f(x)$ is irreducible, then by Hensel's lemma, we have $\ol{f}(x) = h(x)^m$ where $h(x) \in k[x]$ is irreducible. But then because $\ol{f}(x)$ has no repeated roots, we see that $m = 1$. But then we see that $[\ell : k] = [L:K]$ and thus $L/K$ is unramified.
\end{proof}

\begin{prop}\leavevmode
    \begin{enumerate}
        \item Let $K \subset L \subset M$ be a tower of field extensions. Then $M/K$ is unramified if and only if $M/L$ and $L/K$ are unramified.
        \item Assume $k$ is perfect. If $L/K$ is unramified and $L'/K$ is finite, then $LL'/L'$ is unramified.
        \item Assume $k$ is perfect. Then if $L/K$ and $L'/K$ are unramified, then $LL'/K$ is unramified.
    \end{enumerate}
\end{prop}

\begin{proof}\leavevmode
    \begin{enumerate}
        \item Note that $M/K$ is unramified if and only if $e(M/K) = 1$, which is equivalent to $e(M/L) = e(L/K) = 1$ by multiplicativity of the ramification index.
        \item Suppose $L/K$ is unramified. Then let $L = K(\alpha)$ and let $f(x) \in \mc{O}_K[x]$ be the minimal polynomial of $\alpha$. Then the reduction $\ol{f}(x) \in k[x]$ is irreducible and $\ell = k(\ol{\alpha})$. Because $k$ is perfect, $\ol{f}(x)$ has no repeated roots.

            Because $LL'/L' = L'(\alpha)/L'$, let $g(x) \in \mc{O}_{L'}[x]$ be the minimal polynomial of $\alpha$. Then $\ol{g}(x) \mid \ol{f}(x)$ and thus $\ol{g}(x)$ has no repeated roots, so $LL'/L'$ is unramified.
        \item Consider the tower $K \subseteq L' \subseteq LL'$. Because $L/K, L'/K$ are unramified, we know $LL'/L$ is unramified. This implies that $LL'/K$ is unramified. \qedhere
    \end{enumerate}
\end{proof}

\begin{thm}
    Assume that $k$ is perfect. Then there is an inclusion-preserving bijection
    \[ \qty{L/K\ \text{finite unramified}} \xleftrightarrow{\simeq} \qty{\ell / k\ \text{finite}} \qquad L \mapsto \ell. \]
    Moreover, $L/K$ is Galois if and only if $\ell/k$ is Galois and $\Gal(L/K) \simeq \Gal(\ell/k)$ in this case.
\end{thm}

\begin{proof}
    We prove surjectivity. Let $\ell / k$ be a finite extension. Write $\ell = k(\alpha_0)$ and let $\ol{f}(x) = k[x]$ be the minimal polynomial of $\alpha_0$. Then any monic lift $f(x) \in \mc{O}_K[x]$ of $\ol{f}(x)$ has a root $\alpha$ such that $\ol{\alpha} = \alpha_0$ by Hensel's Lemma. Then $L = K(\alpha)$ has residue field $\ell = k(\alpha_0)$. Because $\ol{f}$ is irreducible and $k$ is perfect, we know $L/K$ is unramified.

    Now we will prove injectivity. Let $L/K, L'/K$ be unramified with the same residue field $\ell$. Then $LL'/K$ is also unramified with residue field $\ell$. But this implies that
    \[ [LL' : K] = [\ell : k] = [L : K] = [L' : K], \]
    so we must have $L = LL' = L'$.
    
    Now we will show the statements about Galois extensions. If $L/K$ is Galois, then $\Gal(L/K)$ preserves $\mc{O}_L$ and $\mf{p}_L \subseteq \mc{O}_L$ and acts trivially on $\mc{O}_K$ and $\mf{p}_K \subseteq \mc{O}_K$. This implies that any $\sigma \in \Gal(L/K)$ induces $\ol{\sigma} \in \Aut(\ell/k)$. If $L = K(\alpha)$ and $\alpha_0 = \ol{\alpha}$, then $L/K$ is Galois if and only if it contains $\alpha$, but this is equivalent to $\ell$ containing all conjugates of $\alpha_0$, which is equivalent to $\ell/k$ being Galois. Then the natural map $\Gal(L/K) \to \Gal(\ell / k)$ is an isomorphism because the permutation on conjugates of $\alpha$ induces the same permutation on the conjuages of $\alpha_0$.
\end{proof}

\begin{cor}
    If $L/K$ is an algebraic extension (possibly infinite), then there exists a largest unramified subextension $K_0 / K$ of $L/K$. Moreover, $L/K_0$ is totally ramified.
\end{cor}

\begin{proof}
    Let $K_0$ be the compositum of all finite unramified subextensions of $L/K$. Then the residue field of $K_0$ is equal to the residue field of $L$ (otherwise, we can create an even larger unramified extension). This implies $L/K_0$ is totally ramified.
\end{proof}

\begin{cor}
    Assume $k = \F_q$. Then for all $n \geq 1$ there is a unique unramified extension $L/K$ of degree $n$ and $\Gal(L/K) = \Gal(\F_{q^n} / \F_q) \cong \Z/n\Z$.
\end{cor}

\begin{proof}
    There is a unique degree $n$ extension of $\F_q$.
\end{proof}

\begin{defn}
    Define the \textit{Frobenius element} $\sigma \in \Gal(L/K)$ when $k = \F_q$ to be the generator of $\Gal(L/K)$ corresponding to the Frobenius map under $\Gal(L/K) \xrightarrow{\simeq} \Gal(\F_{q^n} / \F_q)$. We will call this element $\mr{Frob}_{L/K}$, and $\mr{Frob}_{L/K}(\alpha) \equiv \alpha^q \pmod \mf{p}_L$ for all $\alpha \in \mc{O}_L$.
\end{defn}

\begin{cor}
    When $k = \F_q$, the maximal unramified extension of $K$ is
    \[ K^{\mr{ur}} = \bigcup_{(n,q) = 0} K(\zeta_n). \]
    In particular, we have
    \[ \Q_p^{\mr{ur}} = \bigcup_{(n,p) = 1} \Q_p(\zeta_n). \]
\end{cor}

\begin{proof}
    We know $\F_{q^n} = \F_q(\zeta_{q^n-1})$, so $\ol{\F}_q$ is given by adjoining all coprime-to-$p$ roots of unity.
\end{proof}

\section{Totally Ramified Extensions}%
\label{sec:totally_ramified_extensions}

\begin{defn}
    A polynomial $f(x) \in K[x]$ is \textit{Eisenstein} if 
    \[ f(x) = a_n x^n + \cdots + a_1 x + a_0 \]
    such that $\abs{a_n} = 1, \abs{a_i} < 1, \abs{a_0} = \abs{\pi}$.
\end{defn}

\begin{exm}
    The polynomial $x^n - \pi$ is Eisenstein. Note that $K[x] / f(x) = K(\sqrt[n]{\pi})$ is totally ramified.
\end{exm}

\begin{prop}
    A finite extension $L/K$ is totally ramified if and only if $L = K(\alpha)$ where $\alpha$ is a root of an Eisenstein polynomial.
\end{prop}

\begin{proof}
    Let $\alpha$ be the root of an Eisenstein polynomial. Then $\abs{\alpha^n} = \prod_{\sigma \colon L \hookrightarrow \wt{L}} \abs{\sigma(\alpha)} = \abs{a_0} = \abs{\pi}$. Therefore $e(L/K) \geq n$ and thus $L/K$ is totally ramified.

    Now suppose $L/K$ is totally ramified. Let $\alpha$ be a uniformizer of $L$. Therefore $(\alpha^n) = (\pi)$ and thus $\abs{\alpha}_L = \abs{\pi}_L^{1/n}$. Then $1, \alpha, \ldots, \alpha^{n-1}$ have absolute values representing different cosets in $\abs{L^{\times}} / \abs{K^{\times}}$. Thus the minimal polynomial of $\alpha$ has degree $n$. Moreover, if we write the minimal polynomial as

    \[ \alpha^n + a_{n-1} \alpha^{n-1} + \cdots + a_1 \alpha + a_0 = 0, \]
    then $\abs{\alpha}^n = \abs{a_0} = \abs{\pi}$ and $\abs{a_i} < 1$ so that the required cancellation happens. But this implies that $\alpha$ is a root of an Eisenstein polynomial.
\end{proof}

\begin{prop}
    Assume $k = \F_q$. Then there are only finitely many totally ramified extensions of $K$.
\end{prop}

\begin{proof}
    Recall \textbf{Krasner's lemma}: Let $f(x) = \sum_{i=0}^n a_i x^i, g(x) = \sum_{i=0}^n b_i x^i \in K[x]$ and assume $\abs{a_i - b_i}$ is sufficiently small for all $i$. If $f(x)$ is irreducible, then so is $g(x)$, and 
    \[ \qty{K(\alpha) \mid f(\alpha) = 0} = \qty{K(\beta) \colon g(\beta) = 0}. \]
    Therefore a totally ramified extension depends only on a small neighborhood of $(a_0, \ldots, a_{n-1})$ in the set
    \[ \qty{\abs{a_0} = \abs{\pi}} \times \qty{\abs{a_1} < 1} \times \cdots \times \qty{\abs{a_{n-1}} < 1}, \]
    which is compact, so it can be covered by finitely many such small neighborhoods.
\end{proof}

Recall that if $K$ is a complete discrete valued field with residue field $k = \F_q$, there exists a unique unramified extension $L/K$ of degree $n$. Together with the proposition, there exist finitely many totally unramified extensions $L/K$ of degree $n$. This is of course false for number fields; for example, $\Q$ has infinitely many quadratic extensions.

\begin{rmk}
    Krasner (1966) gave an explicit formula for the number of extensions of $p$-adic fields of degree $n$ and an algorithm to construct the set of generating polynomials of degree $n$. More desirable is a way to organize all these extensions, and local class field theory achieves this for abelian extensions of all local fields.
\end{rmk}

\begin{defn}
    A \textit{local field} is a valued field $K$ that is locally compact under the topology induced by the absolute value.
\end{defn}

\begin{rmk}
    Recall that
    \begin{enumerate}
        \item A topological space is compact if and only if open cover has a finite subcover.
        \item A topological space is locally compact if every point has compact neighborhood.
        \item A metric space is compact if and only if it is complete and totally bounded.
        \item A metric space is compact if and only if all closed balls are compact.
        \item This tells us that local fields are always complete. To find a limit for a Cauchy sequence, everything is contained in a closed ball, which is complete and thus has a limit.
    \end{enumerate}
\end{rmk}

\begin{exm}\leavevmode
    \begin{enumerate}
        \item The easiest examples of local fields are $\R, \C$.
        \item If $K$ is archimedean and complete, then $K \simeq \R$ or $\C$.
    \end{enumerate}
\end{exm}

\begin{lem}
    Let $K$ be a complete discrete valued field. Then $K$ is locally compact if and only if the residue field $k$ is finite.
\end{lem}

\begin{proof}
    Let $K$ be locally compact. Then $\mc{O}_K = \qty{x \in K \mid \abs{x} \leq 1}$ is a closed ball. This means that $\mc{O}_K$ is compact. If we consider an open cover
    \[ \mc{O}_K = \bigcup_{x \in k} (x + \mf{p}_K), \]
    this has a finite subcover. But all of the $x + \mf{p}_K$ are disjoint, so $k$ is finite.

    Now suppose $k$ is finite. We show that every $x \in K$ has a compact neighborhood. In particular, we will show that $x + \mc{O}_K$ is compact and therefore that $\mc{O}_K$ is compact. To do this, we need to show that $\mc{O}_K$ is totally bounded. Choose $r > 0$ and consider the open balls $B_{a,r}$ give a cover of $\mc{O}_K$ as long as $a \in \mc{O}_K / \mf{p}_K^n$ and $n$ is sufficiently large. By finiteness of $k$, we know that $\mc{O}_K /\mf{p}_K^n$ is finite, as desired.
\end{proof}

\begin{thm}
    Every local field is one of the following:
    \begin{enumerate}
        \item $\R$ or $\C$;
        \item A finite extension of $\Q_p$;
        \item $\F_q((t))$ for a prime power $q$.
    \end{enumerate}
\end{thm}

\begin{proof}
    Suppose $\operatorname{char} K = 0$. Then $\Q \subseteq K$. If $K$ is archimedean, then $K = \R$ or $\C$. Otherwise, $\Q_p \subseteq K$ and by local compactness, $K/\Q_p$ must be finite.

    If $\operatorname{char} K = 0$, then $\F_p \subseteq K$. Let $k = \F_q$ be the residue field. Then 
    \[ K = \qty{\sum_{n \geq k} a_n \pi^n \mid k \in \Z, a_n \in S = \mc{O}_K / \mf{p}_K}. \]
    By Hensel's lemma, we have $\F_q^{\times} \hookrightarrow K^{\times}$ and we thus have $\F_q \subseteq K$. Therefore $K \cong \F_q((\pi))$, as desired.
\end{proof}

\section{Statement of Local Class Field Theory}%
\label{cha:local_class_field_theory}

Recall that a field extension $L/K$ is \textit{abelian} if it is Galois and $\Gal(L/K)$ is abelian.

\begin{exer}
    If $L_1/K, L_2/K$ are abelian, then $L_1 L_2 / K$ is also abelian.
\end{exer}

Define $K^{\mr{ab}}$ to be the maximal abelian extension of $K$. Equivalently, this is the compositum of all finite extensions of $K$. Then $K^{\mr{ab}}/K$ has infinite degree, and classifying abelian extensions of $K$ is equivalent to understanding $\Gal(K^{\mr{ab}}/K)$.

\begin{defn}
    Let $\Omega /K$ be a possible infinite extension. We call $\Omega/K$ \textit{Galois} if it is algebraic, separable, and normal. Equivalently, $\Omega$ is the union of all its finite Galois subextensions. In particular, we have
    \[ \Gal(\Omega/K) = \lim_{\substack{\longleftarrow \\ L/K \text{ finite Galois}}} \Gal(L/K) \]
    is an inverse limit of finite groups, known as a \textit{profinite} group. Then $\Gal(\Omega / K)$ has a \textit{profinite topology} with a basis of open neighborhoods of $1$ given by $\Gal(\Omega / L) \subseteq \Gal(\Omega / K)$ for finite subextensions $L/K$.  
\end{defn}

\begin{exm}
    Consider $\Omega / K = \ol{F}_q / \F_q$. Then 
    \[ \Gal(\ol{F}_q, \F_q) = \lim_{\substack{\longleftarrow \\ n}} \Gal(\F_{q^n} / \F_q) = \lim_{\substack{\longleftarrow \\ n}} \Z/n\Z \eqqcolon \wh{\Z}. \]
    The open neighborhoods of $1$ are given by $\Gal(\ol{F}_q / \F_{q^n}) \cong n \wh{\Z} \subseteq \wh{\Z}$.
\end{exm}

\begin{rmk}
    If $L/K$ is finite, then $\Gal(\Omega / L)$ is an open subgroup. In addition, $\Gal(\Omega / L)$ is also closed, so it is clopen.

    If $L/K$ is any extension, then 
    \[ \Gal(\Omega / L) = \bigcap_{\substack{L_i \subseteq L \\ L_i/K \text{ finite}}} \Gal(\Omega / L_i) \]
    is a closed subgroup.
\end{rmk}

\begin{thm}[Galois correspondence]
    Let $\Omega / K$ be Galois. Then there is a Galois correspondence
    \[ \qty{L/K \text{ subextension of }\Omega /K} \longleftrightarrow \qty{\text{closed subgroups of } \Gal(\Omega/K)}. \]
    Moreover, $L/K$ is Galois if and only if the corresponding closed subgroup $H \subseteq \Gal(\Omega /K)$ is normal.
\end{thm}

\begin{rmk}
    If $H \subseteq \Gal(\Omega /K)$ is not necessarily closed (for example, $\Z \subset \wh{\Z}$ is not closed and its closure is $\wh{\Z}$), then $\Omega^H$ corresponds to $\ol{H}$ under the Galois correspondence. In particular, $\Gal(K^{\mr{ab}}/K) = G/\ol{[G,G]} \eqqcolon G^{\mr{ab}}$, where $G = \Gal(\ol{K} / K)$.

    This gives us the slogan, that when $K$ is a local field, $\Gal(K^{\mr{ab}}/K)$ can be understood in terms of $K^{\times}$.
\end{rmk}

\begin{thm}[Local Artin reciprocity]
    There exists a unique $\phi_K \colon K^{\times} \to \Gal(K^{\mr{ab}}/K)$ (local Artin reciprocity map) such that
    \begin{enumerate}
        \item For any finite abelian $L/K$, there is a commutative diagram 
            \begin{equation*}
            \begin{tikzcd}
                K^{\times} \arrow{r}{\phi_K} \arrow[twoheadrightarrow]{d} & \Gal(K^{\mr{ab}}/K) \arrow[twoheadrightarrow]{d} \\
                K^{\times} / N(L^{\times}) \arrow{r}{\sim}[swap]{\phi_{L/K}} & \Gal(L/K).
            \end{tikzcd}
            \end{equation*}
        \item For any finite unramified $L/K$ and uniformizer $\pi$ of $K$, we have $\phi_{L/K}(\pi) = \mr{Frob}_{L/K} \in \Gal(L/K)$.
    \end{enumerate}
\end{thm}

\section{Norm Subgroups}%
\label{sec:norm_subgroups}

Let $K$ be a nonarchimedean local field.

\begin{defn}
    A subgroup of $K^{\times}$ is a \textit{norm subgroup} if it is of the form $N(L^{\times})$ for some finite abelian extension $L/K$.
\end{defn}

\begin{prop}
    Assume local CFT I. 
    \begin{enumerate}
        \item $N(L_1^{\times}) \cap N(L_2^{\times}) = N({(L_1 L_2)}^{\times})$.
        \item $N(L_1^{\times}) \subseteq N(L_2^{\times})$ if and only if $L_1 \supseteq L_2$.
        \item A subgroup of $K^{\times}$ containing a norm subgroup is also a norm subgroup.
        \item $N(L_1^{\times}) N(L_2^{\times}) = N({(L_1 \cap L_2)}^{\times})$.
    \end{enumerate}
\end{prop}

\begin{proof}
    Recall that there exists a unique local Artin reciprocity map $K^{\times} \xrightarrow{\phi_K} \Gal(K^{\mr{ab}}/K)$.
    \begin{enumerate}
        \item Note that if $K \subseteq L_2 \subseteq L_1$, then $N(L_1^{\times}) \subseteq N(L_2^{\times})$, so clearly for $L_1, L_2$, we see that $N({(L_1L_2)}^{\times}) \subseteq N(L_1^{\times}) \cap N(L_2^{\times})$. Conversely, if $a \in N(L_1^{\times}) \cap N(L_2^{\times})$, then by local Artin reciprocity, we see that $a \in \ker \phi_{L_1/K} \cap \ker \phi_{L_2/K}$. But this means that $\eval{\phi_K(a)}_{L_1} = \eval{\phi_K(a)}_{L_2} = 1$, and thus $\eval{\phi_K(a)}_{L_1L_2} = 1$. But this implies that $a \in \ker \phi_{L_1L_2/K} = N({(L_1L_2)}^{\times})$.
        \item One direction is obvious. Assume that $N(L_1^{\times}) \subseteq N(L_2^{\times})$. Therefore 
            \[ N(L_1^{\times}) = N(L_1^{\times}) \cap N(L_2^{\times}) = N({(L_1L_2)}^{\times}). \]
            However, we know that 
            \[ [L_1L_2:K] = [K^{\times}:N({(L_1L_2)}^{\times})] = [K^{\times}:N(L_1^{\times})] = [L_1:K], \]
            which implies that $L_1 = L_1L_2$, so $L_1 \supseteq L_2$.
        \item Assume $H \supseteq N(L^{\times})$. Let $M = L^{\phi_{L/K}(H) \subseteq \Gal(L/K)}$. Then by local Artin reciprocity, we have a commutative diagram
            \begin{equation*}
            \begin{tikzcd}
                H/N(L^{\times}) \arrow{r}{\simeq} \arrow[hookrightarrow]{d} & \Gal(L/M) \arrow[hookrightarrow]{d} \\
                K^{\times}/N(L^{\times}) \arrow{r}{\substack{\phi_{L/K}\\ \sim}}  \arrow[twoheadrightarrow]{d} & \Gal(L/K) \arrow[twoheadrightarrow]{d} \\
                K^{\times}/H = K^{\times/N(M^{\times})} \arrow{r}{\substack{\phi_{M/K}\\ \sim}} & \Gal(M/K).
            \end{tikzcd}
            \end{equation*}
            This tells us that $H = N(M^{\times})$ is also a norm subgroup.
        \item Note that $L_1 \cap L_2$ is the largest subextension contained in both $L_1, L_2$. On the other hand, $N({(L_1\cap L_2)}^{\times})$ is the smallest subgroup containing both $N(L_1^{\times}), N(L_2^{\times})$, and the desired result follows. \qedhere
    \end{enumerate}
\end{proof}

\begin{cor}
    The map $L \mapsto N(L^{\times})$ defines a bijection
    \[ \qty{L/K\ \text{finite abelian}} \longleftrightarrow \qty{\text{norm subgroups of } K^{\times}}. \]
\end{cor}

The idea of local Artin reciprocity was to understand extrinsic data about extensions using intrinsic data about the group $K^{\times}$. However, the notion of a norm subgroup still still extrinsic, so we want a more intrinsic characterization of norm subgroups.

\begin{lem}
    Let $L/K$ be a finite extension. If $N(L^{\times})$ has finite index in $K^{\times}$, it must be open.
\end{lem}

\begin{proof}
    Note that $N \colon L^{\times} \to K^{\times}$ is continuous and $\mc{O}_L^{\times}$ is compact. Then $N(\mc{O}_L^{\times}) \subseteq K^{\times}$ ic compact and hence closed. But then $\mc{O}_K^{\times} / N(\mc{O}_L^{\times}) \to K^{\times}/N(L^{\times})$, and thus $N(\mc{O}_L^{\times}) \subseteq \mc{O}_K^{\times}$ is open (and closed). But this implies that $N(\mc{O}_L^{\times}) \subseteq K^{\times}$ is open (because $\mc{O}_K^{\times} \subseteq K^{\times}$ is open), and thus $N(L^{\times})$ must be open.
\end{proof}

\begin{cor}
    If $L/K$ is finite abelian, then $N(L^{\times}) \subseteq K^{\times}$ is a finite index open subgroup.
\end{cor}

\begin{thm}[Local CFT II: local existence]
    Every finite index open subgroup of $K^{\times}$ is a norm subgroup.
\end{thm}

\begin{cor}
    We have a bijection
    \[ \qty{L/K\ \text{finite abelian}} \longleftrightarrow \qty{\text{finite index open subgroups of } K^{\times}}. \]
\end{cor}

\begin{rmks}\leavevmode
    \begin{enumerate}
        \item This bijection also holds for archimedean local fields. For $K = \R$, the two extensions $\R, \C$ correspond to $\R^{\times}, \R_{>0}$, while $\C$ is algebraically closed and $\C^{\times}$ is the only finite-index open subgroup of itself.
        \item If $K$ is a finite extension of $\Q_p$, then any finite index subgroup of $K^{\times}$ is automatically open. However, this is not true for $K = \F_q((t))$. In fact, if $H \subseteq K^{\times}$ has finite index $n$, then ${(K^{\times})}^n \subseteq H$. Therefore, it suffices to show that ${(K^{\times})}^n$ is open. It is easy to see that ${(K^{\times})}^n \supseteq 1 + \mf{p}_K^m$ for some $m \gg 0$. Therefore the equation $x^n - a = 0$ has solutions in $K$ for $a \in 1 + \mf{p}_K^m$ by Hensel for $p \nmid n$ and a stronger version for $p \mid n$ (that does not always hold).
    \end{enumerate}
\end{rmks}

Now we want a reformulation of local CFT using the norm topology.

\begin{defn}
    The \textit{norm topology} on $K^{\times}$ is given by declaring a basis of open neighborhoods of $1$ to be the norm subgroups of $K^{\times}$, which are the same as finite index open subgroups in the usual topology.
\end{defn}

\begin{exm}
    Note that $\mc{O}_K^{\times}$ is open under the usual topology, but is not open under the norm topology.
\end{exm}

\begin{rmk}
    The norm topology has fewer open sets and is therefore coarses than the usual topology.
\end{rmk}

\begin{defn}
    Define $\wh{K}^{\times}$ to be the completion of $K^{\times}$ under the norm topology:
    \[ \wh{K}^{\times} \coloneqq \varprojlim_{L/K\ \text{finite abelian}} K^{\times}/N(L^{\times}). \]
    Then the Artin reciprocity map induces an isomorphism $\wh{K}^{\times} \simeq \Gal(K^{\mr{ab}}/K)$.
\end{defn}

\begin{prop}
    We have an isomorphism $\wh{K}^{\times} \cong \mc{O}_K^{\times} \times \wh{Z}$ as topological groups.
\end{prop}

\begin{proof}
    Choose a uniformizer $\pi$ of $K$. Then $K^{\times} \cong \mc{O}_K^{\times} \times \pi^{\Z}$. Then a basis of finite index open subgroups is given by $(q+\mf{p}_k^m) \times \pi^{n\Z}$ for some $m,n \geq 1$. This implies that
    \[ \wh{K}^{\times} \cong \varprojlim_m \mc{O}_K^{\times} / (1+\mf{p}_K^m) \times \varprojlim_n \Z/n\Z = \mc{O}_K^{\times} \times \wh{\Z}. \qedhere \]
\end{proof}

\begin{cor}
    There is an isomorphism $\Gal(K^{\mr{ab}}/K) \cong \mc{O}_K^{\times} \times \wh{\Z}$ as topological groups for any choice of uniformizer $\pi$. Therefore, we have a decomposition $K^{\mr{ab}} = K_{\pi} \cdot K^{\mr{ur}}$, where $K_{\pi} = {(K^{\mr{ab}})}^{\phi_K(\pi)}$ and $K^{\mr{ur}} = {(K^{\mr{ab}})}^{\phi_K(\mc{O}_K^{\times})}$ is the maximal unramified extension. This means that $K_{\pi}$ is the totally ramified part of $K^{\mr{ab}}$.
\end{cor}

\begin{rmk}
    More canonically, consider the short exact sequence
    \[ 0 \to \mc{O}_K^{\times} \to K^{\times} \to \Z \to 0. \]
    If we consider the profinite completion of this, we obtain an exact sequence
    \[ 0 \to \mc{O}_K^{\times} \to \wh{K}^{\times} \to \wh{\Z} \to 0. \]
    Because $\Z \hookrightarrow \wh{\Z}$ is dense, so is $K^{\times} \hookrightarrow \wh{K}^{\times}$.
\end{rmk}


Now, it remains to prove local class field theory. First, we will construct the local Artin reciprocity map $\phi_K$ using Galois cohomology. After we prove that $\phi_K$ has the desired property, we will prove the local existence theorem by constructing enough norm subgroups using cyclic extensions.

\chapter{Group Cohomology}%
\label{sec:group_cohomology}

\begin{defn}
    Let $G$ be a group. Then a \textit{$G$-module} is a (left) module over the ring $\Z[G]$, or in other words, an abelian group with a linear (left) $G$-action.
\end{defn}

\begin{exm}
    Let $L/K$ be a finite Galois extension of fields and $G = \Gal(L/K)$. Then $M = L$ and $M = L^{\times}$ are both $G$-modules.
\end{exm}

\begin{exm}
    Any abelian group $M$ can be regarded as a $G$-module under the trivial action.
\end{exm}

\begin{defn}
    A \textit{homomorphism of $G$-modules} $\alpha \colon M \to N$ is a $G$-equivariant group homomorphism, or equivalently a morphism of $\Z[G]$-modules. The set of such morphisms is denoted $\Hom_G(M,N)$.
\end{defn}

We will denote the category of $G$-modules with $G$-linear maps by $\ms{Mod}_G$. Because $\ms{Mod}_G = \ms{Mod}_{\Z[G]}$, it is an abelian category with enough injectives and projectives. This allows us to develop the full theory of homological algebra in a concrete way.

\begin{defn}
    $M \in \Mod_G$ is \textit{injective} if the functor $\Hom_G(-,M)$ is exact. Dually, $M \in \Mod_G$ is \textit{projective} if $\Hom_G(M,-)$ is exact.
\end{defn}

\begin{defn}
    An abelian category has \textit{enough injectives} if any object can be embedded in an injective object. Similarly, an abelian category has \textit{enough projectives} if any object has a surjection from a projective object.
\end{defn}

\begin{exm}
    The free $\Z[G]$-module $M = \Z[G]$ is projective. In fact, $\Hom_G(\Z[G], M) = M$.
\end{exm}

\section{Definition of Cohomology}%
\label{sec:definition_of_cohomology}

\begin{defn}
    Let $M \in \Mod_G$. Define its \textit{$G$-invariants} by 
    \[ M^G \coloneqq \qty{ x \in M \mid gx = x\ \text{for all}\ g \in G } \subseteq M \]
    to be the largest submodule with trivial $G$-action.
\end{defn}

\begin{exm}
    Let $G = \Gal(L/K)$. Then if $M = L$, Galois theory tells us that $M^G = K$. Similarly, if $M = L^{\times}$, then $M^G = K^{\times}$.
\end{exm}

\begin{rmk}
    In other words, we have $M^G = \Hom_{\Z[G]}(\Z, M)$, where $\Z$ has the trivial action. This implies that the functor $M \mapsto M^G$ is always left-exact. Therefore, for a short exact sequence
    \[ 0 \to A \to B \to C, \]
    we have an exact sequence
    \[ 0 \to A^G \to B^G \to C^G. \]
    However, this may fail to be right-exact.
\end{rmk}

We may resolve this failure of right-exactness by constructing the derived functor of ${(-)}^G$. This will give us a long exact sequence.

\begin{defn}
    The \textit{group cohomology} $H^r(G, M)$ for any $r \geq 0$ is defined by be the functor $\Ext^r(G, M)$. This is the right derived functor of $\Hom_{\Z[G]}(\Z, M)$. It is characterized by
    \begin{enumerate}
        \item $H^0(G, M) = M^G$.
        \item A short exact sequence $0 \to A \to B \to C \to 0$ induces a long exact sequence in group cohomology
            \begin{equation*}
            \begin{tikzcd}
                0 \arrow{r} & A^G \ar{r} & B^G \ar{r} & C^G \ar[out=-20,in=160]{dll} \\
                            & H^1(G,A) \ar{r} & H^1(G,B) \ar{r} & H^1(C,G) \ar{r} & \cdots
            \end{tikzcd}
            \end{equation*}
        \item If $I \in \Mod_G$ is injective, then $H^r(G, I) = 0$ for all $r \geq 1$.
    \end{enumerate}
\end{defn}

\begin{rmk}
    More concretely, we can compute $H^r(G, M)$ using an injective resolution of $M$. If we consider an injective resolution
    \[ 0 \to M \to I^0 \to I^1 \to I^2 \to \cdots, \]
    then we apply ${(-)}^G$ to obtain a complex
    \[ {(I^0)}^G \to {(I^1)}^G \to {(I^2)}^G \to \cdots \]
    and then we have $H^r(G, M) = H^r({(I^{\bullet})}^G)$.
\end{rmk}

\begin{rmk}
    We can also compute $H^r(G, M)$ using a projective resolution of $\Z$. If we consider a projective resolution
    \[ \cdots \to P_2 \to P_1 \to P_0 \to \Z \to 0, \]
    then we apply $\Hom_G(-,M)$ to this and obtain a complex
    \[ \Hom_G(P_0, M) \to \Hom_G(P_1, M) \to \Hom_G(P_2, M) \to \cdots \]
    and then $H^{\bullet}(G, M)$ is just the cohomology of this complex.
\end{rmk}

Now we will give a description of the cohomology in low degree. In particular, what we will do is use an explicit free resolution of $\Z$.

\begin{defn}
    Define $P_r = \Z[\underbrace{G \times \cdots G}_{r+1}]$ where $G$ acts on $P_r$ by
    \[ g \cdot (g_0, \ldots, g_r) = (gg_0, \ldots, gg_r). \]
    Note that $P_r$ is a free $\Z[G]$-module with basis $\qty{(1, g_1, \ldots, g_r)}$. Now we define the morphism $P_r \to P_{r-1}$ by
    \[ (g_0, \ldots, g_r) \mapsto \sum_{i=0}^r {(-1)}^i (g_0, \ldots, \wh{g}_i, \ldots, g_r). \]
\end{defn}

\begin{lem}
    The previous definition gives a free resolution of $\Z$ in $\Mod_G$.
\end{lem}

\begin{defn}
    This is clearly a complex. To prove exactness, let 
    \[ k_r \colon P_r \to P_{r+1} \qquad (g_0, \ldots, g_r) \mapsto (1, g_0, \ldots, g_r). \]
    Then we can check that $\dd_r \circ k_r + k_{r-1} \circ \dd_{r-1} = \mr{id}$. Then taking the image of both sides of $\ker \dd_{r-1}$, we obtain $\dd_r \circ k_r (\ker \dd_{r-1}) = \ker \dd_{r-1}$ and thus $\ker \dd_{r-1} \subseteq \Im \dd_r$.
\end{defn}

\begin{cor}
    We can compute $H^r(G, M) = H^r(\Hom_G(P_{\bullet}, M))$.
\end{cor}

\begin{defn}
    We have an identification 
    \[ \Hom_G(P_r, M) = \qty{\varphi \colon G^{r+1} \to M \mid \varphi(g g_0, \ldots, g g_r) = g \varphi(g_0, \ldots, g_r)}. \]
    These are called the \textit{homogeneous $r$-cochains of $G$ with values in $M$} and are denoted by $\wt{C}^r(G, M)$. Then the differentials are given by
    \[ \wt{C}^r(G,M) \xrightarrow{\wt{\dd}^r} \wt{C}^{r+1}(G,M) \qquad (\wt{\dd}^r \varphi) (g_0, \ldots, g_{r+1}) = \sum_{i=0}^{r+1} {(-1)}^i \varphi (g_0, \ldots, \wh{g}_i, \ldots, r+1). \]
    Then we have an explicit cochain description
    \[ H^r(G, M) = \frac{\ker \wt{\dd}^r}{\Im \wt{\dd}^{r-1}} = \frac{\qty{\text{homogeneous $r$-cocycles}}}{\qty{\text{homogeneous $r$-coboundaries}}}. \]
\end{defn}

Note that homogeneous $r$-cocycles $\varphi \colon G^{r+1} \to M$ are determined by their values on elements of the form $(1, g_1, \ldots, g_r)$ for $g_i \in G$, or equivalently on elements of the form $(1, g_1, g_1g_2, \ldots, g_1 \cdots g_r)$. Therefore we may eliminate one degree of freedom. 

\begin{defn}
    Define the group of \textit{inhomogeneous $r$-cochains} to be the group
    \[ C^r(G, M) \coloneqq \qty{\varphi \colon G^r \to M\ \text{arbitrary function}}. \]
    Now we have an isomorphism $\wt{C}^r(G, M) \simeq C^r(G, M)$ given by
    \[ \wt{\varphi} \mapsto \varphi(g_1, \ldots, g_r) \coloneqq \wt{\varphi}(1, g_1, g_1g_2, \ldots, g_1 \cdots g_r). \]
    The differentials $C^r(G, M) \xrightarrow{\dd^{r+1}} C^{r+1}(G, M)$ are given by
    \begin{align*} 
        (\dd^r \varphi)(g_1, \ldots, g_{r+1}) &= g_1 \varphi(g_2, g_3, \ldots, g_r) \\
                                              &\ \ + \sum_{i=1}^r {(-1)}^i \varphi(g_1, g_2, \ldots, g_{i-1}, g_i g_{i+1}, g_{i+2}, \ldots, g_r) \\
                                              &\ \ + {(-1)}^r \varphi(g_1, g_2, \ldots, g_r). 
    \end{align*}
    Now we can define the \textit{$r$-cocycles} $Z^r(G, M)$ and \textit{$r$-coboundaries} $B^r(G, M)$ and the corresponding cohomology groups $H^r(G, M)$.
\end{defn}

\begin{exm}
    Suppose $r = 1$. Then we have 
    \begin{align*}
        Z^1(G, M) &= \qty{\varphi \colon G \to M \mid \dd \varphi = 0} \\
                  &= \qty{\varphi \colon G \to M \mid g_1 \varphi(g_2) - \varphi(g_1g_2) + \varphi(g_1) = 0} \\
                  &= \qty{\varphi \colon G \to M \mid \varphi(g_1g_2) = g_1 \varphi(g_2) + \varphi(g_1)}.
    \end{align*}
    Such functions are usually called \textit{crossed homomorphisms}. 

    On the other hand, we have
    \begin{align*}
        B^1(G, M) &= \qty{\dd^0 \varphi \mid \varphi \colon \qty{1} \to M} \\
                  &= \qty{(\dd^0\varphi)(g) = gm - m \mid m \in M} \\
                  &= \qty{\varphi \colon G \to M \mid \varphi(g) = gm - m\ \text{for some $m \in M$}}. 
    \end{align*}
    These functions are called \textit{principal crossed homomorphisms}. Therefore, we have
    \[ H^1(G, M) = \frac{\qty{\text{crossed homomorphisms}}}{\qty{\text{principal crossed homomorphisms}}}. \]
\end{exm}

\begin{exm}
    If $M$ acts trivially on $G$, then $Z^1(G, M) = \Hom_{\ms{Grp}}(G, M)$ and $B^1(G, M)$ is trivial. Thus $H^1(G, M) = \Hom_{\ms{Grp}}(G, M)$.
\end{exm}

\begin{rmk}
    There is an explicit cochain description for $H^r(G, M)$, but computing using this definition is extremely tedious. Instead, we will try to break $G$ and $M$ into smaller pieces and then piece them back together.
\end{rmk}

\section{Change of Groups}%
\label{sec:change_of_groups}

\begin{defn}
    Let $H \subseteq G$ be a subgroup and let $M \in \Mod_H$. Define the \textit{induced module}
    \[ \Ind_H^G M \coloneqq \Hom_{\Z[H]} (\Z[G], M) = \qty{\varphi \colon G \to M \mid \varphi(hg) = h \varphi(g)} \]
    with the action of $G$ given by
    \[ (g \varphi)(x) \coloneqq \varphi(xg). \]
\end{defn}

\begin{exm}
    For any $M \in \Mod_G$, we have $\Ind_G^G M = M$.
\end{exm}

\begin{defn}
    Let $M \in \Mod_G$. Then define the \textit{restriction} $\Res_H^G M$ to be the same $M$ but viewed as an $H$-module. 
\end{defn}

\begin{prop}\leavevmode
    \begin{enumerate}
        \item (Frobenius reciprocity) For any $M \in \Mod_G, N \in \Mod_H$, we have
            \[ \Hom_G(M, \Ind_H^G N) \cong \Hom_H(\Res_H^G M, N). \]
        \item The functor $\Ind_H^G$ is an exact functor.
        \item The functor $\Ind_H^G$ preserves injections.
    \end{enumerate}
\end{prop}

\begin{proof}\leavevmode
    \begin{enumerate}
        \item We will construct explicit mutual inverses. We will define
            \begin{align*} 
                \Hom_G(M, \Ind_H^G N) &\xleftrightarrow{\sim} \Hom_H(\Res_H^G M, N) \\
                            \alpha &\mapsto \beta(m) \coloneqq \alpha(m)(1_G) \\
                            \alpha(m)(g) \coloneqq \beta(gm) &\mapsfrom \beta
            \end{align*}
            It is easy to check that these are inverse to each other.
        \item Now we need to show that induction is right exact. Suppose $M \twoheadrightarrow N$ is a surjective map of $H$-modules. Now let $\varphi \in \Ind_H^G N$. Note that $\varphi$ is uniquely determined by its values on a complete set of representatives $s \in H \backslash G$. Then we lift $\varphi(s) \in N$ to $\wt{\varphi(s)} \in M$ using the surjection $M \twoheadrightarrow N$. Now define $\wt{\varphi}(hs) = h \wt{\varphi}(s)$ for all $h \in H$, and now we obtain an element $\wt{\varphi} \in \Ind_H^G M$ mapping to $\varphi$.
        \item Let $I \in \Mod_H$ be injective. Then $\Ind_H^G I$ is injective if and only if $\Hom_G(-, \Ind_H^G I)$ is exact. By Frobenius reciprocity, this is equivalent to exactness of $\Hom_H(\Res_H^G -, I)$, which is obvious. \qedhere
    \end{enumerate}
\end{proof}

\begin{prop}[Shapiro's lemma]
    Let $N \in \Mod_H$. Then $H^r(G, \Ind_H^G N) \simeq H^r(H, N)$ for all $r \geq 0$.
\end{prop}

\begin{proof}
    Choose an injective resolution $N \to I^{\bullet}$ in $\Mod_H$. But then exactness of induction and preservation of injectives imply that $\Ind_H^G \to \Ind_H^G I^{\bullet}$ is an injective resolution in $\Mod_G$. But now we see that
    \[ H^r(G, \Ind_H^G N) = H^r(\Hom_G (\Z, \Ind_H^G I^{\bullet})) = H^r(\Hom_H(\Z, I^{\bullet})) = H^r(H, N). \qedhere \]
\end{proof}

\begin{defn}
    A module $M \in \Mod_G$ is called \textit{induced} if $M = \Ind_1^G M_0$ for some abelian group $M_0$.
\end{defn}

\begin{cor}
    If $M$ is induced, then for all $r \geq 1$, $H^r(G, M) = 0$.
\end{cor}

\begin{proof}
    By Shapiro, this reduces to computing cohomology $H^r(1, M_0)$, but then we know that $\Hom(\Z, -) = (-)$ is exact, so all higher derived functors vanish.
\end{proof}

Now we will consider functorial properties of group cohomology with respect to change of groups. Given $H \subseteq G$ and $M \in \Mod_G$, we will define
\begin{enumerate}
    \item The \textit{restriction} functor $\Res \colon H^r(G, M) \to H^r(H, M)$.
    \item The \textit{corestriction} functor $\Cor \colon H^r(H, M) \to H^r(G, M)$ whenever $[G:H] < \infty$.
    \item The \textit{inflation} functor $\Inf \colon H^r(G/H, M^H) \to H^r(G, M)$ when $H$ is a normal subgroup of $G$.
\end{enumerate}

Suppose we are given $\alpha \colon G' \to G$ and $\Mod_G \ni M \xrightarrow{\beta} M' \in \Mod_{G'}$ that are compatible in the sense that
\[ \beta(\alpha(g')m) = g' \beta (m) \]
for all $g' \in G', m \in M$. Then we obtain a morphism of cochain complexes 
\[ C^r(G, M) \to C^r(G', M') \qquad (\varphi \colon G^r \to M) \to \beta \circ \varphi \circ \alpha^r \colon {(G')}^r \to G^r \to M \to M'. \]
This is compatible with the differentials, so we obtain a morphism $H^r(G, M) \to H^r(G', M')$. Now using this generality, we can define the three functors.

\begin{enumerate}
    \item (Restriction) We will set $\alpha \colon H \hookrightarrow G$ and $\beta \colon M \xrightarrow{\mr{id}} M$.
    \item (Corestriction) We will set $\alpha \colon G \xrightarrow{\mr{id}} G$ and 
        \[ \beta \colon \Ind_H^G M \to M \qquad \varphi \mapsto \sum_{g \in G/H} g \varphi(g^{-1}). \]
        This gives us the corestriction map by Shapiro.
    \item (Inflation) We take $\alpha \colon G \twoheadrightarrow G/H$ and $\beta \colon M^H \hookrightarrow M$.
\end{enumerate}

\begin{rmk}
    Suppose $r = 0$. Then the functors are
    \begin{align*}
        \Res \colon M^G &\hookrightarrow M^H \\
        \Cor \colon M^H &\xrightarrow{N_{G/H}} M^G \\
                      m &\mapsto \sum_{g \in G/H} gm.
    \end{align*}
\end{rmk}

\begin{prop}
    The map $\Cor \circ \Res \colon H^r(G, M) \to H^r(H, M) \to H^r(G, M)$ is given by multiplication by $[G:H]$.
\end{prop}

\begin{proof}
    Consider $M \to \Ind_H^G M \to M$. Then this map is given by
    \[ m \mapsto \varphi(g) = gm \mapsto \sum_{g \in G/H} g \varphi(g^{-1}) = \sum_{g \in G/H} m = [G:H]m. \qedhere \]
\end{proof}

\begin{cor}
    If $G$ is finite, then $H^r(G, M)$ is killed by $\abs{G}$.
\end{cor}

\begin{proof}
    Take $H = 1$ and apply the previous proposition together with the fact that all higher cohomology vanishes for $H = 1$.
\end{proof}

\begin{cor}
    If $G$ is finite and $M$ is a finitely generated abelian group, then $H^r(G, M)$ is finite.
\end{cor}

\begin{proof}
    If $M$ is finitely generated, then so is $H^r(G, M)$ because the cochain complex is finitely generated. But then $H^r(G, M)$ is torsion, so it must be finite.
\end{proof}

\begin{thm}[Inflation-restriction exact sequence]
    There is an exact sequence
    \[ 0 \to H^1(G/H, M^H) \xrightarrow{\Inf} H^1(G, M) \xrightarrow{\Res} H^1(H, M). \]
\end{thm}

\begin{proof}
    We will check this by hand.
    \begin{description}
        \item[First map is injective:] Let $\varphi \in Z^1(G/H, M^H)$. Now assume that 
            \[ \Inf(\varphi) \colon G \twoheadrightarrow G/H \xrightarrow{\varphi} M^H \hookrightarrow M \in Z^1(G, M)\]  
            is a coboundary. Thus there exists $m \in M$ such that $\Inf(\varphi) (g) = gm - m$ for all $g \in G$. But then $\varphi(\ol{g}) = gm - m$, but then for all $h \in H$, we see that $\varphi(\ol{h}) = hm-m = 0$ because $h$ fixes $M^H$. Therefore $\varphi$ is a coboundary.
        \item[Composition is zero:] Clearly we have $\Res \circ \Inf = 0$ because the composition
            \[ H \hookrightarrow G \twoheadrightarrow{G/H} \xrightarrow{\varphi} M^H \hookrightarrow M \]
            is trivial because it passes through $G/H$.
        \item[Exactness on right:] Assume that $\Res(\varphi) \in B^1(H, M)$. Then there exists $m \in M$ such that $\varphi(h) = hm - m$ for all $h \in H$. Now define 
            \[ \varphi' \in Z^1(G, M) \qquad \varphi'(g) \coloneqq \varphi(g) - (gm-m). \]
            Now $\varphi', \varphi$ are cohomologous. We will now write $\varphi'$ as an inflation. Because $\varphi'(h) = \varphi(h) - (hm-m) = 0$, we know that $\varphi'$ factors through $G/H$. Moreover, we see that
            \begin{align*}
                \varphi'(hg) &= h \varphi'(g) + \varphi'(h) = h \varphi'(g) \\
                             &= \varphi'(gh') = g \varphi'(h') + \varphi'(g) = \varphi'(g).
            \end{align*}
            This implies that $\varphi'$ is valued in $M^H$, so it must have been an inflation. \qedhere
    \end{description}
\end{proof}

\begin{thm}[Inflation-restriction, general case]
    Let $r \geq 1$. Assume $H^i(H, M) = 0$ for all $1 \leq i < r$. Then we have an exact sequence
    \[ 0 \to H^r(G/H, M^H) \xrightarrow{\Inf} H^r(G, M) \xrightarrow{\Res} H^r(H, M). \]
\end{thm}

\begin{proof}
    We will proceed by induction on $r$ using a dimension-shifting argument. Consider the induced $G$-module $M_* = \Ind_1^G M$. Then look at the exact sequence
    \[ 0 \to M \to M_* \to M' \to 0, \]
    where we have the map $m \mapsto (g \mapsto gm)$. But then the higher cohomology of $M_*$ vanishes, so we have isomorphisms $H^i(G, M') \simeq H^{i+1}(G, M)$ for all $i \geq 1$. But then $M_* = \Ind_H^G \Ind_1^H M$, so we have isomorphisms $H^i(H, M') \simeq H^{i+1}(H, M)$ for all $i \geq 1$. Similarly, because $H^1(H, M) = 0$, we have an exact sequence
    \[ 0 \to M^H \to H_*^H \to {(M')}^H \to 0, \]
    and therefore $M_*^H$ is also an induced $G/H$-module. Therefore $H^i(G/H, {(M')}^H) \simeq H^{i+1}(G/H, M^H)$ for all $i \geq 1$.

    Now, by assumption, we have $H^i(H, M) = 0$ for all $1 \leq i \leq r-1$. Therefore, $H^i(H, M') = 0$ for all $1 \leq i \leq r-2$ and now we can apply the inductive hypothesis to obtain an exact sequence
    \[ 0 \to H^{r-1}(G/H, {(M')}^H) \xrightarrow{\Inf} H^{r-1}(G, M') \xrightarrow{\Res} H^{r-1}(H, M'). \]
    This implies that
    \[ 0 \to H^r(G/H, M^H) \xrightarrow{\Inf} H^r(G, M') \xrightarrow{\Res} H^r(H, M) \]
    is also exact.
\end{proof}

\begin{rmk}
    We have a more general version of the dimension shift. Given an exact sequence
    \[ 0 \to M \to A^1 \to \cdots \to A^k \to N \to 0, \]
    where each $A^i$ is induced, then $H^r(G, M) \simeq H^{r+1}(G, N)$ for all $r \geq 1$.
\end{rmk}

\begin{rmk}
    The inflation-restriction exact sequence is a special case of the \textit{Hochschild-Serre spectral sequence.} We have
    \[ E_2^{p,q} = H^p(G/H, H^q(H, M)) \Rightarrow H^{p+q}(G, M). \]
    In our case, only the rows $q = 0, q = r$ are nontrivial.
\end{rmk}

\section{Group Homology}%
\label{sec:group_homology}

\begin{defn}
    Define the \textit{$G$-coinvariants} of $M$ by 
    \[ M_G \coloneqq M / \ev{gm - m \mid g \in G, m \in M}. \]
    This is the largest quotient module of $M$ where $G$ acts trivially.
\end{defn}

\begin{defn}
    The \textit{augmentation ideal} is defined to be
    \[ I_G \coloneqq \ker (\Z[G] \xrightarrow{\ep} \Z) = \Z[g-1 \mid g \neq 1]. \]
\end{defn}

Then we see that $M_G = M/I_G M = M \otimes_{\Z[G]} \Z[G]/I_G = M \otimes_{\Z[G]} \Z$. In particular, the coinvariants functor is right-exact. Applying the derived functor package, we obtain

\begin{defn}
    Define the \textit{group homology} by 
    \[ H_r(G, M) \coloneqq \Tor_r^{\Z[G]} (\Z, M) \]
    to be the left derived functor $\Tor_r^{\Z[G]}(\Z, M)$ of $M \mapsto M_G$.
\end{defn}

\begin{rmks}
    Group homology is characterized by
    \begin{enumerate}
        \item $H_0(G, M) = M_G$;
        \item A short exact sequence $0 \to A \to B \to C \to 0$ in $\Mod_G$ gives a long exact sequence
            \begin{equation*}
            \begin{tikzcd}
                \cdots \arrow{r} & H_r(G, A) \ar{r} & H_r(G, B) \ar{r} & H_r(G, C) \ar[out=-20,in=160]{dll} \\
                                 & H^{r-1}(G,A) \ar{r} & H^{r-1}(G,B) \ar{r} & H^{r-1}(C,G) \ar{r} & \cdots
            \end{tikzcd}
            \end{equation*}
        \item If $P \in \Mod_G$ is projective, then $H_r(G, P) = 0$ for all $r \geq 1$.
    \end{enumerate}
\end{rmks}

\begin{rmk}
    Let $P_{\bullet} \to M$ be a projective resolution. Then $H_r(G, M) = H^r({(P_{\bullet})}_G)$.
\end{rmk}

\begin{prop}
    For any $G$, we have $H_1(G, \Z) \simeq G^{\mr{ab}} = G/[G,G]$.
\end{prop}

\begin{proof}
    Consider the short exact sequence 
    \[ 0 \to I_G \to \Z[G] \xrightarrow{\ep} \to \Z \to 0. \]
    This gives a long exact sequence 
    \[ H_1(G, \Z[G]) \to H_1(G, \Z) \to H_0(G, I_G) \to H_0(G, \Z[G]) \to H_0(G, \Z) \to 0. \]
    This becomes
    \[ 0 \to H_1(G, \Z) \to I_G / I_G^2 \to \Z \xrightarrow{\sim} \Z \to 0. \]
    But this tells us that $H_1(G, \Z) \simeq I_G / I_G^2$. Now define the map
    \[ G \to I_G / I_G^2 \qquad g \mapsto g-1. \]
    By the identity
    \[ g_1 g_2 - 1 = (g_1 - 1) + (g_2 - 1) + (g_1 - 1)(g_2 - 1), \]
    we see that this is a group homomorphism. Because $I_G/I_G^2$ is abelian, the group homomorphism factors through $G^{\mr{ab}} \to I_G / I_G^2$. One can easily check that have an inverse $I_G / I_G^2 \to G^{\mr{ab}}$ given by $g-1 \mapsto g$.
\end{proof}

\begin{rmk}
    This gives a homological interpretation of ${\Gal(L/K)}^{\mr{ab}}$ as $H_1(\Gal(L/K), \Z)$.
\end{rmk}

\begin{rmk}
    This is analogous to the topological fact that if $X$ is path-connected, then $H_1(X, \Z) = {\pi_1(X)}^{\mr{ab}}$. In fact, for any group $G$, we can define a space $BG = K(G, 1)$ such that $\pi_1(BG) = G$ and $\pi_i(BG) = 0$ for $i \neq 1$. We have $H_r(BG, \Z) = H_r(G, \Z)$, where the LHS is singular homology and the RHS is group homology. If we choose $r = 1$, we recover $H_1(G, \Z) = G^{\mr{ab}}$.
\end{rmk}

\begin{exm}\leavevmode
    \begin{enumerate}
        \item If $G = \Z$, then we have $B\Z = \R / \Z = S^1$. 
        \item If $G = \underbrace{\Z * \Z * \cdots * \Z}_n$, then $BG = \underbrace{S^1 \vee \cdots \vee S^1}_n$.
    \end{enumerate}
\end{exm}

\section{Tate Cohomology}%
\label{sec:tate_cohomology}

The idea is that for a finite group $G$, we will patch both group cohomology and group homology into a single theory.

\begin{defn}
    Let $G$ be a finite group and $M \in \Mod_G$. Define the \textit{norm map} 
    \[ \Nm_G \colon M \to M \qquad m \mapsto \sum_{g \in G} gm. \]
\end{defn}

By definition, the image of the norm map lies in the invariants and $\Nm_G (gm - m) = 0$, so the kernel always contains $I_G M$. This gives us a commutative diagram
\begin{equation*}
\begin{tikzcd}
    M \ar{r}{\Nm_G} \ar[twoheadrightarrow]{d} & M \\
    M_G \ar{r}{\Nm_G} & M^G \ar[hookrightarrow]{u}
\end{tikzcd}
\end{equation*}
and an exact sequence
\[ 0 \to \frac{\ker \Nm_G}{I_G} \to M_G \xrightarrow{\Nm_G} M^G \to \frac{M^G}{\Im \Nm_G} \to 0. \]
In particular, this gives us
\[ 0 \to \wh{H}^{-1}(G, M) \to H_0(G, M) \to H^0(G, M) \to \wh{H}^0(G, M) \to 0, \]
where $\wh{H}$ just means a modified version of cohomology.

\begin{defn}
    For $r \in \Z$, define the \textit{Tate cohomology groups}
    \[ \wh{H}^r(G, M) \coloneqq \begin{cases}
        H^r(G, M) & r \geq 1 \\
        \frac{M^G}{\Im \Nm_G} & r = 0 \\
        \frac{\ker \Nm_G}{I_G M} & r = -1 \\
        H_{-(r+1)}(G, M) & r \leq -2.
    \end{cases} \]
\end{defn}

\begin{exm}
    Given a short exact sequence
    \[ 0 \to A \to B \to C \to 0 \]
    in $\Mod_G$, we obtain a long exact sequence
    \begin{equation*}
    \begin{tikzcd}
        \cdots \ar{r} & H_1(G, C) \ar{d} \ar{r} & H_0(G, A) \ar{r} \ar{d}{\Nm_G} & H_0(G, B) \ar{r} \ar{d}{\Nm_G} & H_0(G, C) \ar{r} \ar{d}{\Nm_G} & 0 \ar{d} \\
                      & 0 \ar{r} & H^0(G, A) \ar{r} & H^0(G, B) \ar{r} & H^0(G, C) \ar{r} & H^1(G, A) \ar{r} & \cdots
    \end{tikzcd}
    \end{equation*}
    and by the snake lemma, these can be patched to form a very long exact sequence
    \[ \cdots \to \wh{H}^{-1}(G, C) \to \wh{H}^{-1}(G, A) \to \wh{H}^{-1}(G, B) \to \wh{H}^{-1}(G, C) \to \wh{H}^0(G, A) \to \cdots \]
\end{exm}

\begin{rmk}\leavevmode
    \begin{enumerate}
        \item If $M \in \Mod_G$ is induced, then $\wh{H}^r(G, M) = 0$ for all $r \in \Z$.
        \item The dimension shift argument works for $\wh{H}^r$ in both directions. If $M \hookrightarrow \Ind_1^G M$, then $\wh{H}^r(M^1) = \wh{H}^{r+1}(M)$, where $M^1$ is the cokernel. Similarly, if $\Ind_1^G M \twoheadrightarrow M$, then $\wh{H}^r(M) = \wh{H}^{r+1}(M^1)$, where $M^1$ is the kernel.
        \item Shapiro's lemma still holds for $\wh{H}^r$.
        \item The property that $\Cor_H^G \circ \Res_H^G$ is multiplication by $[G:H]$ holds for $\wh{H}^r$.
    \end{enumerate}
\end{rmk}

\begin{rmk}
    Take $P_{\bullet} \to \Z$ to be a free resolution in $\Mod_G$. Taking the dual $M \mapsto M^* = \Hom_{\Z}(M, \Z)$ where $(g \varphi)(m) = \varphi(g^{-1}m)$. This gives a resolution $\Z \to P^{\bullet}$. If we take $P_{-n} \coloneqq P_{n-1}^*$, we obtain
    \[ \cdots \to P_2 \to P_1 \to P_1 \to P_{-1} = P_0^* \to P_{-2} \to \cdots \]
    Then we can compute $\wh{H}^r(G, M) = H^r(\Hom_G({(P_{\bullet})}, M))$.
\end{rmk}

Now we will compute Tate cohomology explicitly for finite cyclic groups $G = \ev{\sigma}$. We can very explicitly write down the norm map, invariants, and coinvariants:
\begin{align*}
    \Nm_G(m) &= \sum_{g \in G} g m = (1 + \sigma + \sigma^2 + \cdots + \sigma^{n-1}) m \\
    M^G &= M^{\sigma = 1} = \ker (\sigma - 1) \\
    I_G M &= \Im (\sigma - 1) \\
    M_G &= \coker (\sigma - 1).
\end{align*}
This implies that 
\begin{align*}
    \wh{H}^0(G, M) &= \frac{M^G}{\Im (\Nm_G)} = \frac{\ker (\sigma - 1)}{\Im (1 + \sigma + \cdots + \sigma^{n-1})} \\
    \wh{H}^{-1}(G, M) &= \frac{\ker (\Nm_G)}{I_G M} = \frac{\ker (1+ \sigma + \cdots + \sigma^{n-1})}{\Im (\sigma - 1)}.
\end{align*}

\begin{rmk}
    All remaining $\wh{H}^r(G, M)$ are determined by $\wh{H}^0, \wh{H}^{-1}$.
\end{rmk}

\begin{thm}[Period $2$]
    Let $G = \ev{\sigma}$. Then we have isomorphisms $\wh{H}^r(G, M) \simeq \wh{H}^{r+2}(G, M)$.
\end{thm}

\begin{proof}
    Note that we have a free resolution of $\Z$ in $\Mod_G$ given by
    \[ \cdots \to \Z[G] \xrightarrow{\sigma - 1} \Z[G] \xrightarrow{1 + \sigma + \cdots + \sigma^{n-1}} \Z[G] \xrightarrow{\sigma - 1} \Z[G] \xrightarrow{\ep} \Z. \]
    This computes the Tate cohomology and has period $2$, so the Tate cohomology also must have period $2$.
\end{proof}

\begin{defn}
    Define the \textit{Herbrand quotient} 
    \[ h(M) \coloneqq \frac{\abs{\wh{H}^0(G, M)}}{\abs{\wh{H}^1(G, M)}} \]
    if both $\wh{H}^0, \wh{H}^1$ are finite.
\end{defn}

\begin{prop}
    If $0 \to A \to B \to C \to 0$ is an exact sequence, then $h(B) = h(A) \cdot h(C)$.
\end{prop}

\begin{proof}
    Consider the long exact sequence
    \[ 0 \to K \to \wh{H^0}(G, A) \to \wh{H}^0(G, B) \to \wh{H}^-(G, C) \to \wh{H}^1(G, A) \to \wh{H}^1(G, B) \to \wh{H}^1(G, C) \to Q \to 0. \]
    This implies that 
    \[ \abs{K} \cdot \abs{\wh{H}^0(A)}^{-1} \cdot \abs{\wh{H}^0(B)} \cdot \abs{\wh{H}^0(C)}^{-1} \cdot \abs{\wh{H}^1(A)} \cdot \abs{\wh{H}^1(B)}^{-1} \cdot \abs{\wh{H}^1(C)} \cdot \abs{Q}^{-1} = 1. \]
    From this, we obtain
    \[ \frac{h(B)}{h(A)h(C)} = \frac{\abs{Q}}{\abs{K}}, \]
    so it suffices to show that $\abs{Q} = \abs{K}$. But here, we see that
    \[ Q = \coker (\wh{H}^1(B) \to \wh{H}^1(C)) = \coker (\wh{H}^{-1}(B) \to \wh{H}^{-1}(C)) = K \]
    because Tate cohomology has period $2$.
\end{proof}

\begin{prop}
    If $M$ is finite, then $h(M) = 1$.
\end{prop}

\begin{proof}
    Note that we have the exact sequence
    \[ 0 \to M^G \to M \xrightarrow{\sigma - 1} M \to M_G \to 0. \]
    This tells us that $\abs{M^G} = \abs{M_G}$. Next, we note that
    \[ 0 \to \wh{H}^{-1}(G, M) \to M_G \xrightarrow{\Nm_G} M^G \to \wh{H}^0(G, M) \to 0 \]
    is exact, so $\abs{\wh{H}^{-1}(G, M)} = \abs{\wh{H}^0(G, M)}$.
\end{proof}

\begin{cor}
    If $\alpha \colon M \to N$ has finite kernel and cokernel, then $h(M) = h(N)$.
\end{cor}

\begin{proof}
    Consider the exact sequence
    \[ 0 \to \ker \alpha \to M \to N \to \coker \alpha \to 0. \]
    This tells us that 
    \[ h(\ker \alpha) \cdot { h(M) }^{-1} \cdot h(N) \cdot { h(\coker \alpha) }^{-1} = 1. \]
    Because $\alpha$ has finite kernel and cokernel, $h(\ker \alpha) = h(\coker \alpha) = 1$, so $h(M) = h(N)$.
\end{proof}

Now we want to compute Tate cohomology for more general $G$.

\begin{thm}
    Let $G$ be a finite group. If for any subgroup $H \subseteq G$ we have $H^1(H, M) = H^2(H, M) = 0$, then $\wh{H}^r(G, M) = 0$ for all $r \in \Z$.
\end{thm}

\begin{proof}
    If $G$ is cyclic, then it follows from the period $2$ isomorphism. In the general case, if $G$ is solvable, we can induct on $\abs{G}$. If $H \triangleleft G$ is normal such that $G/H$ is cyclic. By the induction hypothesis, we know $\wh{H}^r(H, M) = 0$ for all $r \in \Z$. Applying inflation-restriction, we have an exact sequence
    \[ 0 \to H^r(G/H, M^H) \to H^r(G, M) \to H^r(H, M). \]
    By assumption, $H^r(G, M) = 0$ for $r = 1,2$. By exactness, we know $H^r(G/H, M^H) = 0$ for $r = 1,2$, so $\wh{H}^r(G/H, M^H) = 0$ for all $r \in \Z$ because $G/H$ is cyclic. By exactness and the inductive hypothesis, we know $\wh{H}^r(G, M) = 0$ for all $r \geq 1$. But now the hypotheses are invariant under dimension shift, so we obtain the desired result.

    Finally, if $G$ is an arbitrary finite group, reduce to the solvable case. If we consider the Sylow subgroups $G_p \subseteq G$, we know $G_p$ is solvable. Therefore $\wh{H}^r(G_p, M) = 0$ for all $r \in \Z$. By an exercise from the homework, we see $\wh{H}^r(G, M) = 0$ for all $r \in \Z$.
\end{proof}

Unfortunately, $H^2$ generally does not vanish, but we can still gain insight about the behavior of Tate cohomology in this case.

\begin{thm}[Tate]
    Let $G$ be a finite group. If for any subgroup $H \subseteq G$ we have
    \begin{enumerate}
        \item $H^1(H, M) = 0$;
        \item $H^2(H, M) = \Z/\abs{H} \Z$,
    \end{enumerate}
    then $\wh{H}^r(G, \Z) \simeq \wh{H}^{r+2}(G, M)$ for all $r \in \Z$.
\end{thm}

\begin{exm}
    We will apply Tate's theorem to construct the local Artin map. Let $G = \Gal(L/K)$ and $M = L^{\times}$ and set $r = -2$. Then both conditions of the theorem are satisfied, and so 
    \[ {\Gal(L/K)}^{\mr{ab}} = H_1(G, \Z) = \wh{H}^{-1}(G, \Z) \simeq \wh{H}^0(G, M) = M^G / \Im (Nm_G) = K^{\times} / N_{L/K}(L^{\times}). \]
    Now we have defined the inverse of the local Artin map!
\end{exm}

\begin{rmk}
    Chao says that Tate was very brave to go from the object $\wh{H}^0(G, M)$ that we want to understand to an object that did not exist at the time!
\end{rmk}

Before we prove Tate's theorem, we will give an explicit description of $H^2(G, M)$. Recall that
\[ Z^2(G, M) = \qty{\varphi \colon G^2 \to M \mid g_1 \varphi(g_2, g_3) - \varphi(g_1g_2, g_3) + \varphi(g_1, g_2g_3) - \varphi(g_1, g_2) = 0}. \]
We also know that
\[ B^2(G, M) = \qty{\varphi \colon G^2 \to M \mid \varphi(g_1, g_2) = g_1 \psi(g_2) - \psi(g_1g_2) + \psi(g_1), \psi \colon G \to M}. \]
\begin{prop}
    The cohomology group $H^2(G, M) = \frac{Z^2(G, M)}{B^2(G, M)}$ parameterizes extensions 
    \[ 0 \to M \to E \to G \to 0 \]
    such that the conjugation action of $E/M \cong G$ on $M$ agrees with the $G$-module structure. Moreover, $0 \in H^2(G, M)$ corresponds to the split extension.
\end{prop}

\begin{proof}
    Given such an extension, choose a splitting of $E = G \times M$ as \textbf{sets}. This gives us a section $s \colon E \to G$ Then for all $g_1, g_2 \in G$, we have $s(g_1) \cdot s(g_2) = \varphi(g_1,g_2) s(g_1g_2)$ for some $\varphi \colon G^2 \to M$.

    We check that $\varphi$ is actually a cocycle. To see this, note that
    \[ (s(g_1) s(g_2))s(g_3) = s(g_1) (s(g_2) s(g_3)) \]
    by associativity, which gives us
    \[ ( \varphi(g_1,g_2) + \varphi(g_1g_2, g_3) ) s(g_1g_2g_3) = (g_1 \varphi(g_2,g_3) + \varphi(g_1,g_2g_3)) s(g_1g_2g_3), \]
    and therefore $\varphi$ is a cocycle. We can check that any $\varphi$ gives such an extension together with a section $s$ and that different choices of the section $s$ give coboundary relations.
\end{proof}

\begin{defn}
    For any $\varphi \in Z^2(G, M)$, define its \textit{splitting module} to be
    \[ M(\varphi) \coloneqq M \oplus \qty(\bigoplus_{g \neq 1} \Z X_g) \]
    with the action of $G$ given by
    \[ g_1 \cdot X_{g_2} \coloneqq X_{g_1g_2} - X_{g_1} + \varphi(g_1,g_2) \]
    for any choice of $g_1, g_2 \in G$. By convention, we write $X_1 \coloneqq \varphi(1,1) \in M$.
\end{defn}

\begin{rmk}
    By construction, we have an exact sequence
    \[ 0 \to M \to M(\varphi) \to I_G \to 0, \]
    where $X_g \mapsto g-1$.
\end{rmk}

\begin{lem}
    The image of $\varphi$ under $H^2(G, M) \to H^2(G, M(\varphi))$ is $0$.
\end{lem}

\begin{proof}
    Note that $\varphi(g_1,g_2) = g_1 X_{g_2} - X_{g_1g_2} + X_{g_1}$ is the coboundary of $g \mapsto X_g$.
\end{proof}

\begin{rmk}
    The extension $E'$ associated to $\varphi$
    \begin{equation*}
    \begin{tikzcd}
        0 \ar{r} & M \ar{r} \ar{d} & E \ar{r} \ar{d} & G \ar{r} \ar[equals]{d} & 0 \\
        0 \ar{r} & M(\varphi) \ar{r} & E' \ar{r} & G \ar{r} & 0
    \end{tikzcd}
    \end{equation*}
    of $M(\varphi)$ is a split extension.
\end{rmk}

\begin{proof}[Proof of Tate's theorem]
    There are two steps to the proof.
    \begin{enumerate}
        \item We know that $H^2(G, M) = \Z/\abs{G}\Z = \ev{\varphi}$ with splitting module $M(\varphi)$. We also know that $H^1(H, M) = 0$. We will show that $H^1(H, M(\varphi)) = 0$ and $H^2(H, M(\varphi)) = 0$. Consider the short exact sequence
            \[ 0 \to M \to M(\varphi) \to I_G \to 0. \]
            This gives a long exact sequence
            \[ H^1(H, M) \to H^1(H, M(\varphi)) \to H^1(H, I_G) \to H^2(H, M) \to H^2(H, M(\varphi)) \to H^2(H, I_G). \]
            However, we know that $H^1(H_1, M) = 0, H^1(H, I_G) = \Z/\abs{H} \Z$, and $H^2(H, I_G) = 0$. We also know that $H^2(H, M) \to H^2(H, M(\varphi))$ is the zero map (the restriction of $\varphi$ to $H^2(H, M)$ is a generator), so the map $H^1(H, I_G) \to H^2(H, M)$ must be an isomorphism and therefore $H^1(H, M(\varphi)) = H^2(H, M(\varphi)) = 0$.
        \item Now we know that all Tate cohomology $\wh{H}^r(G, M(\varphi)) = 0$ for all $r \in \Z$. Now we use the four term exact sequence
            \[ 0 \to M \to M(\varphi) \to \Z[G] \xrightarrow{\ep} \Z \to 0. \]
            Because $\wh{H}^r(G, M(\varphi)) = 0$ and $\wh{H}^r(G, \Z[G]) = 0$ for all $r \in \Z$, By dimension shift, we have $\wh{H}^r(G, \Z) \cong \wh{H}^{r+2}(G, M)$. \qedhere
    \end{enumerate}
\end{proof}

\begin{rmk}
    There is an alternative description of $\wh{H}^r(G, \Z) \simeq \wh{H}^{r+2}(G, M)$ as a cup product with some generator $\varphi \in H^2(G, M)$. Here, if $M, N \in \Mod_G$, we can consider the tensor product $M \otimes N$ as a $G$-module with the action $g(m \otimes n) = gm \otimes gn$.

    Now a \textit{cup product} is a family of $\Z$-bilinear maps 
    \[ H^r(G, M) \otimes H^s(G, N) \to H^{r+s}(G, M \otimes N) \]
    given by the following properties:
    \begin{enumerate}
        \item The cup product is functorial in both $M, N$.
        \item For $r = s = 0$, it is given by the natural map $M^G \otimes N^G \to {(M \otimes N)}^G$.
        \item If $0 \to M' \to M \to M'' \to 0$ is a short exact sequence such that
            \[ 0 \to M' \otimes N \to M \otimes N \to M'' \otimes N \to 0 \]
            is also exact, then for all $m'' \in H^r(G, M'')$ and $n \in H^s(G, N)$, we have $\delta m'' \cup n = \delta (m'' \cup n) \in H^{r+s+1}(G, M' \otimes N)$.
        \item If $0 \to N' \to N \to N'' \to 0$ is a short exact sequence such that
            \[ 0 \to M \otimes N' \to M \otimes N \to M \otimes N'' \to 0 \]
            is also exact, then for all $m \in H^r(G, M), n'' \in H^s(G, N'')$, we have $m \cup \delta n'' = {(-1)}^r \delta(m \cup n'')$.
    \end{enumerate}
    The cup product exists and is unique. Given cocycles $\varphi, \psi$, we write
    \[ (\varphi \cup \psi) (g_1, \ldots, g_{r+s}) = \varphi(g_1, \ldots, g_r) \otimes g_1 \cdots g_r \psi(g_{r+1}, \ldots, g_{r+s}) \]
    and this is unique by dimension shift. If $G$ is finite, then the cup product extends to Tate cohomology.
\end{rmk}

\chapter{Local Class Field Theory}%
\label{cha:local_class_field_theory}

Now our goal is to understand $H^i(\Gal(L/K), L^{\times})$ for $i = 1,2$. This will allow us to construct the local Artin reciprocity map.

\section{Vanishing of first cohomology}%
\label{sec:vanishing_of_first_cohomology}

This result was proved by Hilbert in 1894 Zahlbericht for cyclic extensions and then by Noether for general extensions.

\begin{thm}[Hilbert Theorem 90]
    Let $L/K$ be a finite Galois extension of fields. Then
    \[ H^1(\Gal(L/K), L^{\times}) = 0. \]
\end{thm}

\begin{rmk}
    Note that this applies to extensions of arbitrary fields, not just local fields.
\end{rmk}

\begin{proof}
    Choose $\varphi \in Z^1 (\Gal(L/K), L^{\times})$. We want $\varphi(g) = \frac{gm}{m}$ for some $m \in L^{\times}$. For any $a \in L^{\times}$, construct
    \[ m \coloneqq \sum_{g \in \Gal(L/K)} \varphi(g) \cdot ga. \]
    Because $g \colon L^{\times} \to L^{\times}$ are distinct characters of $L^{\times}$ for $g \in G \coloneqq \Gal(L/K)$, we know that they are $L$-linearly independent, so the map
    \[ \sum_{g \in G} \varphi(g) \cdot g \]
    is not the zero map and thus there exists $a \in L^{\times}$ such that $m \neq 0$. But now we compute
    \begin{align*} 
        gm &= g \sum_{h \in G} \varphi(h) \cdot ha \\
           &= \sum_{h \in G} g \varphi(h) \cdot gha \\
           &= \sum_{h \in G} \frac{\varphi(gh)}{\varphi(g)} \cdot gha \\ 
           &= \frac{1}{\varphi(g)} \sum_{h \in G} \varphi(gh) \cdot gha \\
           &= \frac{1}{\varphi(g)} \sum_{h \in G} \varphi(h) \cdot ha \\
           &= \frac{m}{\varphi(g)},
    \end{align*}
    and thus $\varphi(g) = \frac{n}{gm} = \frac{gm^{-1}}{m^{-1}}$, so $\varphi \in B^1(G, L^{\times})$.
\end{proof}

\begin{exm}
    In the case where $L/K$ is a cyclic extension and $G = \ev{\sigma}$, then $\wh{H}^{-1}(G, L^{\times}) = H^1(G, L^{\times}) = 0$, and thus $\ker \Nm_G = \Im (\sigma - 1)$, which means that if $a \in L^{\times}$ satisfies $\Nm_{L/K}(a) = 1$, then $a = \frac{\sigma b}{b}$ for some $b \in L^{\times}$.
\end{exm}

\begin{exm}
    Let $L/K = \Q(i)/\Q$ be the simplest quadratic extension. Let $a = x+iy \in L^{\times}$ with $x,y = \Q$. Then we know $\Nm_{L/K}(a) = x^2 + y^2$. If $b = m + in \in L^{\times}$ for $m,n \in \Q$, we see that
    \[ \frac{\sigma b}{b} = \frac{m-in}{m+in} = \frac{m^2-n^2}{m^2+n^2} + \frac{2mn}{m^2+n^2} \cdot i. \]
    By Hilbert 90, if $x^2 + y^2 = 1$, then there exist $m,n \in \Z$ such that 
    \[ x = \frac{m^2 - n^2}{m^2 + n^2} \qquad y = \frac{2mn}{m^2 + n^2}. \]
    In particular, this tells us how to classify all Pythagorean triples.
\end{exm}

\begin{rmk}
    We may also consider infinite Galois extensions $L/K$, for example $L = K^{\mr{sep}}, K^{\mr{ab}}$. Then we know
    \[ \Gal(L/K) = \varprojlim_{\substack{L' \subseteq L \\ \text{$L'/K$ finite Galois}}} \Gal(L'/K). \]
    This allows us to define \textit{continuous Galois cohomology}
    \[ H^r_{\mr{cts}}(\Gal(L/K), L^{\times}) \coloneqq \varinjlim_{L'} H^r(\Gal(L'/K), {(L')}^{\times}). \]
    The $H^r_{\mr{cts}}$ can be computed using continuous cochains.
\end{rmk}

\begin{rmk}
    More generally, if $G = \varprojlim_{H} G/H$ is a profinite group and $M$ is a \textbf{discrete} $G$-module (which means $G \times M \to M$ is continuous with respect to the profinite topology on $G$ and the discrete topology on $M$), we may consider the continuous group cohomology
    \[ H^r_{\mr{cts}}(G, M) \coloneqq \varinjlim_H H^r(G/H, M^H), \]
    where the maps in the directed system are inflation.
\end{rmk}

\begin{notn}
    Let $G = \Gal(L/K)$. We will simply write 
    \[ H^r(L/K, M) \coloneqq H^r_{\mr{cts}}(\Gal(L/K), M) \qquad H^r(K, M) \coloneqq H^r_{\mr{cts}}(\Gal(K^{\mr{sep}}/L), M) \]
    for the continuous Galois cohomology.
\end{notn}

\begin{exm}
    Hilbert Theorem 90 implies that $H^1(K, {(K^{\mr{sep}})}^{\times}) = 0$. Equivalently, we have $H^1_{\text{\'et}}(\Spec K, \mathbb{G}_m) \cong \Pic \Spec K = 0$. In other words, all line bundles on a point are trivial.
\end{exm}

\section{Second Cohomology}%
\label{sec:second_cohomology}

Let $L/K$ be a finite unramified extension of local fields. Recall that 
\[ G = \Gal(L/K) \simeq \Gal(\ell / k) = \ev{\Frob_{L/K}}, \] 
where $\ell/k$ is the extension of residue fields. Our goal is to compute $H^2(L/K, L^{\times}) \cong \Z/n\Z$, where $n = [L:K]$. The idea is to break $L^{\times}$ into pieces, so choose a uniformizer $\pi_L$. Then we have $L^{\times} \cong \mc{O}_L^{\times} \times \pi_L^{\Z}$. Because $L/K$ is unramified, we may choose $\pi_L = \pi_K \in K^{\times}$. Therefore, as a Galois module, we have $L^{\times} \cong \mc{O}_L^{\times} \oplus \Z$.

\begin{defn}
    Let $U_K = \mc{O}_K^{\times}$. Then write $U_K^{(i)} \coloneqq 1 + \mf{m}_K^i$ for all $i \geq 1$. 
\end{defn}
This gives us a natural filtration
\[ U_K \supseteq U_K^{(i)} \supseteq U_K^{(2)} \supseteq U_K^{(3)} \supseteq \cdots \]
with short exact sequences
\[ 1 \to U_K^{(1)} \to U_K \to k^{\times} \to 1 \]
and similarly
\[ 1 \to U_K^{(i+1)} \to U_K^{(i)} \to k \to 1 \]
for $i \geq 1$.

\begin{lem}
    We have $\wh{H}^r(G, \ell^{\times}) = \wh{H}^r(G, \ell) = 0$ for all $r \in \Z$.
\end{lem}

\begin{proof}
    By Hilbert Theorem 90, we know $H^1(G, \ell^{\times}) = 0$. Because $G$ is cyclic and $\ell^{\times}$ is finite, we can understand the Herbrand quotient. Therefore $h(\ell^{\times}) = 1$, so $H^2(G, \ell^{\times}) = 0$. Because $G$ is cyclic, we may apply period $2$ to obtain the desired result.

    Now we consider $\ell$. We know that $\ell$ is actually an induced module by the homework, so all Tate cohomology vanishes by period $2$.
\end{proof}

\begin{cor}
    The norm map $\Nm_{\ell/k} \colon \ell^{\times} \to k^{\times}$ and the trace map $\Tr_{\ell/k} \colon \ell \to k$ are surjective.
\end{cor}

\begin{proof}
    We know that $\wh{H}^0(G, \ell^{\times}) = \frac{k^{\times}}{\Nm_{\ell/k}(\ell^{\times})} = 0$, so the norm map is surjective. Similarly, we know $\wh{H}^0(G, \ell) = \frac{k}{\Tr_{\ell/k}(\ell)} = 0$, so the trace is surjective.
\end{proof}

\begin{lem}
    The norm map $\Nm_{L/K} \colon U_L \to U_K$ is surjective.
\end{lem}

\begin{proof}
    Consider the commutative diagram
    \begin{equation*}
    \begin{tikzcd}
        U_L \ar{r}{\Nm} \ar{d} & U_K \ar{d} & U_L^{(i)} \ar{r}{\Nm} \ar{d} & U_K^{(i)} \ar{d} \\
        \ell^{\times} \ar{r}{\Nm} & k^{\times} & \ell \ar{r}{\Tr} & k.
    \end{tikzcd}
    \end{equation*}
    Given any $a \in U_K$, we want $b \in U_L$ such that $\Nm(b) = a$. Because $\Nm \colon \ell^{\times} \to k^{\times}$, we may find $b_0 \in U_L$ such that $\Nm(b_0) \equiv a \pmod {\pi_K}$. Now let $a_1 \coloneqq \frac{a}{\Nm(b_0)} \in U_K^{(1)}$. Because $\Tr \colon \ell \to k$ is surjective, we may find $b_1 \in U_L^{(1)}$ such that $a_2 \coloneqq \frac{a_1}{\Nm(b_1)} \in U_K^{(2)}$. Repeating this to infinity, we can find $b_i \in U_L^{(i)}$ such that $a_{i+1} \coloneqq \frac{a_i}{\Nm(b_i)} \in U_K^{(i+1)}$. Setting 
    \[ b \coloneqq \prod_{i=1}^{\infty} b_i, \] 
    we know 
    \[ \frac{a}{\Nm(b)} \in \bigcap_{i=0}^{\infty}U_K^{(i)} = \qty{1}, \] 
    so $a = \Nm(b)$.
\end{proof}

\begin{cor}
    For all $r \in \Z$, we have $\wh{H}^r(G, U_L) = 0$.
\end{cor}

\begin{proof}
    By the Lemma, we know that $\Nm \colon U_L \to U_K$ is surjective, so $\wh{H}^0(G, U_L) = 0$. But then we know $H^1(G, U_L) \subseteq H^1(G, L^{\times}) = 0$ is a direct factor, so $H^1(G, U_L) = 0$. The desired result is simply an application of period $2$.
\end{proof}

\begin{thm}
    If $L/K$ is an unramified extension, then $H^2(G, L^{\times}) \cong \Hom(G, \Q/\Z) \cong \frac{1}{n} \Z / \Z$.
\end{thm}

\begin{proof}
    We know $H^2(G, L^{\times}) \cong H^2(G, U_L) \oplus H^2(G, \Z) = H^2(G, \Z) = \Hom(G, \Q/\Z)$ by the homework. The final isomorphism is obvious.
\end{proof}

\begin{defn}
    Define the \textit{invariant map} 
    \[ \mr{inv}_{L/K} \colon H^2(L /K, L^{\times}) \simeq H^2(L/K, \Z) \simeq \Hom(G, \Q/\Z) \simeq \frac{1}{n} \Z / \Z \subseteq \Q/\Z. \]
    Taking the limit over all unramified extensions, we obtain a map
    \[ H^2(K^{\mr{ur}}/K, {(K^{\mr{ur}})}^{\times}) \xrightarrow{\sim} \bigcup_n \frac{1}{n} \Z / \Z = \Q/\Z. \]
\end{defn}

To simplify our notation, we will write $H^2(L/K) \coloneqq H^2(L/K, L^{\times})$, so we have 
\[ \mr{inv}_K \colon H^2(K^{\mr{ur}}/K) \simeq \Q/\Z. \] 
Our goal now will be to extend this to $H^2(K^{\mr{sep}}/K) \simeq \Q/\Z$.

\begin{rmk}
    We always have an injective inflation map $H^2(K^{\mr{ur}}/K) \xrightarrow{\Inf} H^2(K^{\mr{sep}}/K)$. The map exists, and injectivity is by Hilbert 90 and inflation-restriction. We want to prove that this is indeed an isomorphism.
\end{rmk}

\begin{lem}
    Let $L/K$ be a finite extension of degree $n$. Then the diagram
    \begin{equation*}
    \begin{tikzcd}
        H^2(K^{\mr{ur}}/K) \ar{d}{\mr{inv}_K} \ar{r}{\Res} & H^2(L^{\mr{ur}}/L) \ar{d}{\mr{inv}_L} \\
        \Q/\Z \ar{r}{\times n} & \Q/\Z
    \end{tikzcd}
    \end{equation*}
    commutes.
\end{lem}

\begin{rmk}
    Because $L^{\mr{ur}} = L \cdot K^{\mr{ur}}$, we may view $\Gal(L^{\mr{ur}}/L)$ as a subgroup of $\Gal(K^{\mr{ur}}/K)$, and the inclusion is comparible with ${(K^{\mr{ur}})}^{\times} \to {(L^{\mr{ur}})}^{\times}$, so restriction makes sense.
\end{rmk}

\begin{proof}
    Recall that $\mr{inv}_K$ is computed by $H^2(K^{\mr{ur}}/K) \simeq H^2(K^{\mr{ur}}/K, \Z) \simeq H^1(K^{\mr{ur}}/K, \Q/\Z) \simeq \Q/\Z$. Now we need to see the effect of restriction on the groups on the right of the chain of isomorphisms. Note that if $e = e(L/K)$ and $f = f(L/K)$, then $\pi_L^e = \pi_K$, so the diagram
    \begin{equation*}
    \begin{tikzcd}
        {(K^{\mr{ur}})}^{\times} \ar{d} \ar{r}{\mr{ord}_K} & \Z \ar{d}{\times e} \\
        {(L^{\mr{ur}})}^{\times} \ar{r}{\mr{ord}_L} & \Z
    \end{tikzcd}
    \end{equation*}
    commutes. Therefore, the maps on $H^2(K^{\mr{ur}}/K, \Z) \to H^2(L^{\mr{ur}}/L, \Z)$ are given by multiplication by $e$. Finally, for $\varphi \in H^1(K^{\mr{ur}}/K, \Q/\Z) = \Hom(\Gal(K^{\mr{ur}}/K), \Q/\Z)$, we see that $\varphi \mapsto \varphi(\Frob_K)$. Because $\Frob_L = \Frob_K^f$, we see that $\Res(\varphi)(\Frob_L) = f \varphi(\Frob_K)$. Therefore, the final map is simply $f \cdot e \cdot \Res = n \cdot \Res$.
\end{proof}

\begin{cor}
    Let $L/K$ be a finite Galois extension of degree $n$. Then $H^2(L/K)$ contains a subgroup of $\frac{1}{n} \Z/\Z$.
\end{cor}

\begin{proof}
    By Hilbert 90 and inflation-restriction, we have a commutative diagram
    \begin{equation*}
    \begin{tikzcd}
        0 \ar{r} & H^2(L/K) \ar{r}{\Inf} & H^2(K^{\mr{sep}}/K) \ar{r}{\Res} & H^2(L^{\mr{sep}}/L) \\
        0 \ar{r} & \frac{1}{n}\Z/\Z \ar{r} & H^2(K^{\mr{ur}}/K) \ar{r}{\Res} \ar[hookrightarrow]{u} & H^2(L^{\mr{ur}}/L) \ar[hookrightarrow]{u} \\
                 & & \Q/\Z \ar{r}{\times n} \ar{u}{\cong} & \Q/\Z \ar{u}{\cong}.
    \end{tikzcd}
    \end{equation*}
    Therefore, $\frac{1}{n}\Z/\Z$ injects into $H^2(L/K)$.
\end{proof}

\begin{prop}
    We have $\abs{H^2(L/K)} = n$. In particular, $H^2(L/K) \cong \frac{1}{n} \Z/\Z$.
\end{prop}

\begin{proof}
    We know this is true with $L/K$ is cyclic. We know $h(L^{\times}) = n$ and $H^1(L/K) = 0$, so $\abs{H^2(L/K)} = n$. In general, we will reduce to the cyclic case because $L/K$ is always solvable (because $K$ is a local field). We will induct on $[L:K]$. Choose a tower of fields $K \subset K' \subset L$ such that $K'/K$ is cyclic. Applying inflation-restriction, we have an exact sequence
    \[ 0 \to H^2(K'/K) \xrightarrow{\Inf} H^2(L/K) \xrightarrow{\Res} H^2(L/K'). \]
    We know that $\abs{H^2(K'/K)} = [K':K]$ because it is cyclic. We also know that $\abs{H^2(L/K')} = [L:K']$ by the inductive hypothesis, so $\abs{H^2(L/K)} \leq [K':K] \cdot [L:K'] = [L:K] = n$. But we know that $H^2(L/K)$ contains a subgroup of order $n$, so $\abs{H^2(L/K)} = n$.
\end{proof}

\begin{cor}
    We have a commutative diagram
    \begin{equation*}
    \begin{tikzcd}
        0 \ar{r} & H^2(L/K) \ar{r}{\Inf} & H^2(K^{\mr{sep}}/K) \ar{r}{\Res} & H^2(L^{\mr{sep}}/L) \\
        0 \ar{r} & \frac{1}{n}\Z/\Z \ar[equal]{u} \ar{r} & H^2(K^{\mr{ur}}/K) \ar{r}{\Res} \ar[hookrightarrow]{u} & H^2(L^{\mr{ur}}/L). \ar[hookrightarrow]{u} 
    \end{tikzcd}
    \end{equation*}
    In particular, we may view $H^2(L/K) \xrightarrow{\Inf} H^2(K^{\mr{ur}}/K)$.
\end{cor}

\begin{thm}
    There is an isomorphism $H^2(K^{\mr{ur}}/K) \xrightarrow{\Inf} H^2(K^{\mr{sep}}/K)$.
\end{thm}

\begin{proof}
    Recall that $H^2(L/K) \hookrightarrow H^2(K^{\mr{ur}}/K) \hookrightarrow H^2(K^{\mr{sep}}/K)$. Taking the limit over finite Galois extensions $L/K$, we have an injection $H^2(K^{\mr{sep}}/K) \hookrightarrow H^2(K^{\mr{ur}}/K) \hookrightarrow H^2(K^{\mr{sep}}/K)$, but the composition is the identity, so all inclusions are equalities.
\end{proof}

\begin{defn}
    We can extend the \textit{invariant map} to $\mr{inv}_K \colon H^2(K^{\mr{sep}}/K) \simeq \Q/\Z$. 
\end{defn}

\begin{rmk}
    For any field $K$, define the \textit{Brauer group} $\mr{Br}(K) \coloneqq H^2(K^{\mr{sep}}/K, {(K^{\mr{sep}})}^{\times})$. This is in fact $H^2_{\text{\'et}}(\Spec K, \mathbb{G}_m)$. The goal of local class field theory is to compute $\mr{Br}(K) \simeq \Q/\Z$ for a local field $K$.
\end{rmk}

\begin{rmk}
    Classically, the group $\mr{Br}(K)$ classifies \textit{central simple algebras} over $K$. In fact, it classifies central simple algebras up to the relation $A \sim B$ if and only if $A \otimes_K M_n(K) \simeq B \otimes M_m(K)$ for some $n, m$. The group structure is defined by $[A] \otimes [B] = [A \otimes_K B]$ with identity $0 = [M_n(k)]$.
\end{rmk}

\begin{thm}[Wedderburn]
    Each class of $\mr{Br}(K)$ is represented by a central division algebra over $K$.
\end{thm}

\begin{exm}
    If $K$ is a local field, then we may define the invariant $\mr{inv}_K(B)$ for any central division algebra $B$ over $K$. If $B = K$, then $\mr{inv}_K(K) = 0$. Now if $B$ is the quaternion algebra over $K$, we have $\mr{inv}_K(B) = \frac{1}{2}$. In fact there is an explicit construction of a central division algebra associated to $\frac{1}{n}$ for every $n$. If $L/K$ is a quadratic extension, then
    \[ \mr{inv}_L(B \otimes_K L) = [L:K] \mr{inv}_K(B) = 2 \cdot \frac{1}{2} = 0 \in \Q/\Z, \]
    so $B \otimes_K L = M_2(L)$.
\end{exm}

\begin{exm}
    Call the element $u_{L/K} \in H^2(L/K)$ with $\mr{inv}_K(u_{L/K}) = \frac{1}{n}$ the \textit{fundamental class} of $L/K$. 
\end{exm}

\begin{rmk}
    The local Artin reciprocity isomorphism $\wh{H}^{-2}(G, \Z) \to \wh{H}^0(G, L^{\times})$ can be realized as the cup product by the fundamental class $u_{L/K}$.
\end{rmk}

\section{Proof of Local Class Field Theory}%
\label{sec:proof_of_local_class_field_theory}

Our goal is to prove the following result:

\begin{thm}[Local class field theory]\leavevmode
    \begin{enumerate}
        \item (Local Artin reciprocity) There exists a unique homomorphism
            \[ \phi_K \colon K^{\times} \to {\Gal(K^{\mr{sep}}/K)}^{\mr{ab}} \]
            such that
            \begin{enumerate}
                \item For any finite Galois extension $L/K$, we have a commutative diagram
                    \begin{equation*}
                    \begin{tikzcd}
                        K^{\times} \ar{d} \ar{r}{\phi_K} & {\Gal(K^{\mr{sep}}/K)}^{\mr{ab}} \ar{d} \\
                        K^{\times}/\Nm(L^{\times}) \ar{r}{\phi_{L/K}}[swap]{\sim} & {\Gal(L/K)}^{\mr{ab}}.
                    \end{tikzcd}
                    \end{equation*}
                \item For any finite unramified extension, $\phi_{L/K}(\pi) = \Frob_{L/K}$, where $\pi$ is an uniformizer of $K$.
            \end{enumerate}
        \item (Local existence theorem) There is a bijection between norm subgroups of $K^{\times}$ and finite index open subgroups of $K^{\times}$. In addition, if $\operatorname{char} K = 0$, then the finite index open subgroups of $K^{\times}$ are precisely the finite index subgroups of $K^{\times}$.
    \end{enumerate}
\end{thm}

\subsection{Proof of local Artin reciprocity}%
\label{sub:proof_of_local_artin_reciprocity}

To construct $\phi_K$, we will use Tate's theorem. Recall that by Hilbert 90, we have $H^1(L/K, L^{\times}) = 0$. We also computed $H^2(L/K, L^{\times}) = \frac{1}{n} \Z/\Z$, where $n = [L:K]$. By Tate's theorem, we have an isomorphism $\wh{H}^r(L/K, \Z) \simeq \wh{H}^{r+2}(L/K, L^{\times})$ for all $r \in \Z$. We will specialize to the case where $r = -2$, and this gives us an isomorphism
\begin{equation*}
\begin{tikzcd}
    \wh{H}^{-2}(L/K, \Z) \ar[equal]{d} \ar{r}{\sim} & \wh{H}^0(L/K, L^{\times}) \ar[equal]{d} \\
    {\Gal(L/K)}^{\mr{ab}} \ar{r}{\sim} & K^{\times} / \Nm(L^{\times}).
\end{tikzcd}
\end{equation*}
Then we define $\phi_{L/K}^{\mr{ab}} \xrightarrow{\sim} {\Gal(L/K)}^{\mr{ab}}$ to be its inverse. Taking the limit over all finite Galois extensions $L/K$, we obtain a map
\[ \phi_K \colon K^{\times} \to {\Gal(K^{\mr{sep}}/K)}^{\mr{ab}}. \]
By construction, the property a) is satisfied.

To finish the proof, we need to check property b). Let $L/K$ be a finite unramified extension. Our goal is to explicitly describe the shift by $2$ isomorphism produced by Tate's theorem. Recall that Tate's theorem comes from $M = L^{\times}$ and $G = \Gal(L/K)$. Then we construct the splitting module for some $\varphi \in H^2(G, M) = H^2(L/K, L^{\times}) = \frac{1}{n} \Z/\Z$ by defining
\[ M(\varphi) = M \oplus \bigoplus_{g \neq 1} X_g \qquad g_1 X_{g_2} = X_{g_1g_2} - X_{g_1} + \varphi(g_1, g_2). \]
Then we have short exact sequences 
\[ 0 \to M \to M(\varphi)  \xrightarrow{X_g \mapsto g-1} I_g \to  0 \]
and
\[ 0 \to I_G \to \Z[G] \xrightarrow{g \mapsto 1} \Z \to 0. \]
Now we need to trace through the isomorphisms
\begin{equation*}
\begin{tikzcd}
    \wh{H}^{-2}(G, \Z) \ar{r}{\sim} \ar[equal]{d} & \wh{H}^{-1}(G, I_G) \ar{r}{\sim} \ar[equal]{d} & \wh{H}^0(G, M) \ar[equal]{d} \\
    G = \ev{\Frob_{L/K}} \ar{r}{\sim} & I_G/I_G^2 \ar{r}{\sim} & K^{\times}/\Nm(L^{\times}).
\end{tikzcd}
\end{equation*}
The first isomorphism is given by $\Frob_{L/K} \eqqcolon \sigma \mapsto \sigma - 1$. Therefore it remains to write down the connecting homomorphism $\wh{H}^{-1}(G, I_G) \to \wh{H}^0(G, M)$. This is described using the exact sequences
\begin{equation*}
\begin{tikzcd}
    & M_G \ar{r} \ar{d}{\Nm} & {M(\varphi)}_G \ar{r} \ar{d}{\Nm} & {(I_G)}_G = I_G/I_G^2 \ar{r} \ar{d}{\Nm} & 0 \\
    0 \ar{r} & M^G \ar{r} & {M(\varphi)}^G \ar{r} & {(I_G)}^G
\end{tikzcd}
\end{equation*}
by lifting $(\sigma - 1) \in I_G/I_G^2$ to ${M(\varphi)}_G$ and then taking $\Nm$. This will automatically land in $M^G$. An obvious list of $\sigma - 1$ to ${M(\varphi)}_G$ is $X_{\sigma}$, so we need to compute $\Nm(X_{\sigma})$. Therefore we have
\begin{align*}
    \Nm(X_{\sigma}) &= (1+ \sigma + \cdots + \sigma^{n-1}) X_{\sigma} \\
                    &= X_{\sigma} + \sigma X_{\sigma} + \cdots + \sigma^{n-1} X_{\sigma} \\
                    &= X_{\sigma} + (X_{\sigma^2}-X_{\sigma}+\varphi(\sigma, \sigma)) + \cdots + (X_{\sigma^n}-X_{\sigma^{n-1}}+\varphi(\sigma^{n-1}, \sigma)) \\
                    &= X_{\sigma^n} + \varphi(\sigma, \sigma) + \varphi(\sigma^2, \sigma) + \cdots + \varphi(\sigma^{n-1}, \sigma) \\
                    &= \varphi(1,1) + \varphi(\sigma, \sigma) + \varphi(\sigma^2, \sigma) + \cdots + \varphi(\sigma^{n-1}, \sigma).
\end{align*}
Now it remains to describe this explicitly for $\varphi \in H^2(G, M) = H^2(L/K, L^{\times}) = \frac{1}{n} \Z/\Z$.Recall that this comes from $H^2(L/K, L^{\times}) \xrightarrow{\sim} H^1(L/K, \Q/\Z) = \Hom(G, \Q/\Z)$ coming from the exact sequence
\[ 0 \to \Z \to \Q \to \Z/\Q \to 0. \]
Now if $f \in \Hom(G, \Q/\Z)$ such that $f(\sigma) = \frac{1}{n}$, then we can lift $f$ to $\wt{f} \in Z^1(G, \Q)$ and take $\dd{\wt{f}}$. This gives us $\wt{f}(\sigma^i) = \frac{i}{n} \in \Q$. If $\varphi = \dd{\wt{f}}$, then we have
\begin{align*}
    \varphi(\sigma^i, \sigma^j) &= \sigma^i \wt{f}(\sigma^j)- \wt{f}(\sigma^{i+j}) + \wt{f}(\sigma^i) \\
                                &= \frac{j}{n} - \frac{(i+j) \mod n}{n} + \frac{i}{n} \\
                                &= \begin{cases}
                                    0 & 0 \leq i+j \leq n-1 \\
                                    1 & i+j \geq n.
                                \end{cases}
\end{align*}
This gives us 
\begin{align*}
    \mr{ord}_K(\Nm(X_{\sigma})) &= \varphi(1,1) + \varphi(\sigma, \sigma) + \varphi(\sigma^2, \sigma) + \cdots + \varphi(\sigma^{n-1}, \sigma) \\
                              &= 0 + 0 + 0 + \cdots + 1 \\
                              &= 1
\end{align*}
and thus $\Nm(X_{\sigma}) = \pi \in K^{\times}/\Nm(L^{\times})$. \qed

\subsection{Proof of Local Existence}%
\label{sub:proof_of_local_existence}

We will assume that $\operatorname{char} K = 0$. We will establish a bijection
\[ \qty{\text{norm subgroups of $K^{\times}$}} \overset{1:1}{\longleftrightarrow} \qty{\text{finite index subgroups of $K^{\times}$}}. \]
Recall that any subgroup of $K^{\times}$ containing a norm subgroup is also a norm subgroup. Also note that any finite index subgroup of index $n$ contains ${(K^{\times})}^n$. Therefore it suffices to show that ${(K^{\times})}^n$ is a norm subgroup. Our goal is to study the $n$-th power may using the \textit{Kummer sequence}
\[ 1 \to \mu_n \to {(K^{\mr{sep}})}^{\times} \xrightarrow{x \mapsto x^n} {(K^{\mr{sep}})}^{\times} \to 1. \]
This is an exact sequence in $\Mod_G$ for $G = \Gal(K^{\mr{sep}}/K)$, so it induces a long exact sequence in cohomology
\[ 0 \to \mu_n(K) \to K^{\times} \xrightarrow{x \mapsto x^n} K^{\times} \to H^1(G, \mu_n) \to H^1(G, {(K^{\mr{sep}})}^{\times}) = 0. \]
This gives us the \textit{Kummer isomorphism} $K^{\times}/{(K^{\times})}^n \cong H^1(G, \mu_n)$.
\begin{enumerate}
    \item In the simplest case, when $\mu_n \subseteq K^{\times}$, then the isomorphism becomes 
        \[ K^{\times}/{(K^{\times})}^n \cong \Hom(G, \mu_n) \cong \Hom(\Gal(L/K), \mu_n), \]
        where $L/K$ is the maximal abelian extension of exponent $n$. Now the pairing
        \[ K^{\times}/{(K^{\times})}^n \times \Gal(L/K) \to \mu_n \qquad (b, \sigma) \mapsto \frac{\sigma(\sqrt[n]{b})}{\sqrt[n]{b}} \]
        is in fact a perfect pairing. Therefore $\abs{K^{\times}/{(K^{\times})}^n} = \abs{\Gal(L/K)}$, so by local Artin reciprocity, we have the map 
        \[ \phi_{L/K} \colon K^{\times}/\Nm(L^{\times}) \xrightarrow{\sim} \Gal(L/K). \]
        Also, we know $\Nm(L^{\times}) \supseteq {(K^{\times})}^n$, and therefore ${(K^{\times})}^n = \Nm(L^{\times})$.
    \item In general, if $\mu_n$ is not contained in $K^{\times}$, then take $K_1 = K(\mu_n)$. Now we can find $L_1/K_1$ such that $\Nm(L_1^{\times}) = {(K_1^{\times})}^n$. Now take $L/K$ Galois such that $L_1 \subseteq L$. Then
        \begin{align*} 
            \Nm_{L/K}(L^{\times}) &= \Nm_{K_1/K} ( \Nm_{L/K_1}(L^{\times}) ) \\
                                  &\subseteq \Nm_{K_1/K}(\Nm_{L_1/K_1}(L_1^{\times})) \\
                                  &= \Nm_{K_1/K}({(K_1^{\times})}^n) \\
                                  &\subseteq {(K^{\times})}^n. 
        \end{align*}
        Therefore ${(K^{\times})}^n$ contains a norm subgroup, and is thus also a norm subgroup. \qed
\end{enumerate}

\chapter{Global class field theory}%
\label{cha:global_class_field_theory}

Recall that a number field $K$ is a finite extension of $\Q$. The goal of global class field theory is to understand ${ \Gal(\ol{K}/K) }^{\mr{ab}}$, or equivalently to understand finite abelian extensions of $K$ in terms of \textbf{intrinsic} data associated to $K$. Analogously to the local case, we would like to construct a \textit{global Artin reciprocity map}
\[ \phi_K \colon C_k \to {\Gal(\ol{K}/K)}^{\mr{ab}} \]
which induces isomorphisms
\[ \phi_{L/K} \colon C_k/\Nm(C_L) \xrightarrow{\sim} \Gal(L/K). \]

\begin{rmk}
    When $K$ is a local field, we took $C_K = K^{\times}$. When $K$ is global, we can again try to take $C_K = K^{\times}$, but in fact this fails.
\end{rmk}

\begin{exm}
    Choose $L/K = \Q(i)/\Q$. Then $\Gal(L/K) = \Z/2$, but 
    \[ \Q^{\times}/\Nm({\Q(i)}^{\times}) = \Q^{\times}/\qty{x \in \Q^{\times} \mid x = a^2+b^2, a,b \in \Q} \neq \Z/2\Z. \]
    For example, any $p \equiv 3 \pmod 4$ cannot be written as $a^2+b^2$, and in fact the quotient is infinite.
\end{exm}

Instead of $K^{\times}$, we will take $C_K$ to be a generalization of the ideal class group $\Cl_K$, which is the quotient of all fractional ideals by the principal ones. Thus the class group measures the failure of unique factorization in $K$.

\section{Id\`eles}%
\label{sec:id`eles}

Our generalization of ideals will be the id\`ele class group of $K$, which is again given by
\[ C_K \coloneqq \frac{\mathbb{I}_K}{K^{\times}} \]
for some group $\I_K$. We need to define $\I_K$ and $\Nm_{L/K} \colon C_L \to C_K$.

\begin{notns}
    We will denote a (finite or infinite) place of $K$ by $v$ and $\abs{-}_v$ the absolute value associated to $v$ such that
    \[ \prod_v \abs{a}_v = 1 \]
    for all $a \in K^{\times}$. Then we will denote the completion of $K$ with respect to $\abs{-}_v$ by $K_v$. When $v$ is a finite place, we write $\mc{O}_V \subseteq K_V$ for the ring of integers, $\mf{p}_v \subseteq \mc{O}_K$ the corresponding maximal ideal, and $\wh{\mf{p}}_v \subseteq \mc{O}_v$ for the completion of the maximal ideal.
\end{notns}

\begin{rmk}
    We may attempt to take $\I_K = \prod_v K_v^{\times}$. The problem is that this group is too large in the sense that it is not locally compact.
\end{rmk}

Instead of this, we will allow only finitely many coordinates to be not a unit in $\mc{O}_v$. This will make our group locally compact.

\begin{defn}
    The group of \textit{id\`eles} (from the French \'el\'ement id\'eal for ideal element) is
    \[ \I_K \coloneqq \qty{{(a_v)}_v \in \prod_v K_v^{\times} \mid a_v \in \mc{O}_V^{\times} \text{ for all but finitely many }v}. \]
    This is the \textit{restricted product} of $\prod_v K_v^{\times}$ with respect to $\prod_v \mc{O}_V^{\times}$.
\end{defn}

\begin{rmk}
    The id\`eles are equipped with the restricted product topology with a basis of open neighborhoods given by
    \[ \prod_v U_v \subseteq \I_K, \]
    where $U_v$ is open in $K_v^{\times}$ for all $v$ and $U_v = \mc{O}_V^{\times}$ for all but finitely many $v$.
\end{rmk}

Let $S$ be a finite set of places containing $S_{\infty} = \qty{v\ \text{infinite}}$. Then define
\[ \I_{K,S} \coloneqq \prod_{v \in S} K_v^{\times} \times \prod_{v \notin S} \mc{O}_v^{\times} \subseteq \I_K. \]
Note the first factor is locally compact and the second product is compact, so $\I_{K,S}$ is locally compact. Then we can write
\[ \I_K = \bigcup_S \I_{K,S} \]
and each $\I_{K,S}$ is open in $\I_K$, so in particular $\I_K$ is locally compact.

\begin{rmk}
    The group $\I_K$ has a basis of open neighborhoods of $1$ given by
    \[ U(S, \ep) \coloneqq \qty{(a_v) \in \I_K \mid \abs{a_v - 1} < \ep \text{ for all } v \in S, \abs{a_v} = 1 \text{ for all } v \notin S}. \]
\end{rmk}

\begin{rmk}
    For all places $v$, there is a natural injection 
    \[ K_v^{\times} \hookrightarrow \I_K \qquad a \mapsto (1, \ldots, 1, a, 1, \ldots), \]
    where $a$ goes in the coordinaate corresponding to $v$. The induced topology on $K_v^{\times}$ agrees with the usual topology.
\end{rmk}

\begin{rmk}
    If $I_K$ is the group of fractional ideals of $K$, then there is a natural surjection
    \[ \I_K \twoheadrightarrow I_K \qquad (a_v) \mapsto \prod_{v \nmid \infty} \mf{p}_v^{\mr{ord}_v(a_v)} \]
    with kernel given by 
    \[ \prod_{v \mid \infty} K_v^{\times} \times \prod_{v \nmid \infty} \mc{O}_v^{\times} = \I_{K, S_{\infty}}. \]
    In particular, $\I_K$ can be thought of as an enlargement of $I_K$ with extra information from the infinite places and units at finite places.
\end{rmk}

\begin{rmk}
    The group $\I_K$ is completely local. 
\end{rmk}

\begin{prop}
    The natural map 
    \[ K^{\times} \hookrightarrow \I_K \qquad a \mapsto (a, a, \ldots, a, \ldots) \]
    has discrete image.
\end{prop}

\begin{proof}
    It suffices to show that $1 \in K^{\times}$ is open in its imaage. Equivalently, for some $S \supseteq S_{\infty}$ and some $\ep > 0$, we have $K^{\times} \cap U(s, \ep) = \qty{1}$. Let $a \in K^{\times} \cap U(S, \ep)$. Then $\abs{a-1}_v < \ep$ for all $v \in S$ and $\abs{a}_v = 1$ for all $v \notin S$. Note that $\abs{a-1}_v \leq 1$ for all $v \notin S$. Therefore
    \[ \prod_v \abs{a-1}_v < \ep^{\abs{s}}. \]
    On the other hand, by the product formula, we know $\prod_v \abs{a-1}_v = 1$ unless $a-1 = 0$. In particular, for $S$ sufficiently large and $\ep$ sufficiently small, we must have $a = 1$.
\end{proof}

\begin{rmk}
    Note that $K^{\times} \hookrightarrow K_v^{\times}$ has dense image for any single $v$. However, the product formula implies that for $a \in K^{\times}$, its absolute values $\abs{a}_v$ at different places $v$ ``repel'' each other and makes $K^{\times} \hookrightarrow \I_K$ have discrete image.
\end{rmk}

\begin{exm}
    Consider the ring $\Z[\sqrt{2}] \hookrightarrow \R$. Because $\sqrt{2}$ is irrational, this has dense image. However, the embedding
    \[ \Z[\sqrt{2}] \hookrightarrow \R \times \R \qquad a+b\sqrt{2} \mapsto (a+b\sqrt{2}, a-b\sqrt{2}) \]
    has discrete image. In fact, the image is a lattice in $\R^2$.

    Similarly, $\Z\qty[\frac{1}{p}] \hookrightarrow \R, \Q_p$ both have dense image, but 
    \[ \Z\qty[\frac{1}{p}] \hookrightarrow \R \times \Q_p \]
    has discrete image.

    Now it should not be surprising that $K^{\times} \hookrightarrow \I_K$ has discrete image. This is simply an infinite dimensional generalization of this phenomenon.
\end{exm}

\begin{defn}
    Define the \textit{id\`ele class group} $C_k \coloneqq \I_K / K^{\times}$ and equip it with the quotient topology. Because $K^{\times}$ is discrete, this group is locally compact. 
\end{defn}

\begin{rmk}
    The $\I_K \twoheadrightarrow I_K$ descends to $C_K \twoheadrightarrow \Cl_K$. In particular, $C_K$ can be viewed as an enlargement of $\Cl_K$.
\end{rmk}

\begin{defn}
    Let $L/K$ be a finite extension. Recall that 
    \[ L \otimes_K K_v \cong \prod_{w \mid v} L_w, \]
    given $K_v[x]/f(x) = \prod (K_v[x]/f_i(x))$, where $f_i$ are the irreducible factors of $f$ after the extension. Therefore, for all $\alpha \in L^{\times}$, we have 
    \[ \Nm_{L/K}(\alpha) = \prod_{w \mid v} \Nm_{L_w/K_v} (\alpha). \]
    Now we can define
    \[ \Nm_{L/K} \colon \I_L \to \I_L \qquad {(a_w)}_w \mapsto {(b_v)}_v \qquad b_v \coloneqq \prod_{w \mid v} \Nm_{L_w/K_v}(a_w). \]
\end{defn}

\begin{defn}
    By definition, we have a commutative diagram
    \begin{equation*}
    \begin{tikzcd}
        \I_L \ar{r}{\Nm_{L/K}} & \I_K \\
        L^{\times} \ar[hookrightarrow]{u} \ar{r}{\Nm_{L/K}} & L^{\times}. \ar[hookrightarrow]{u}
    \end{tikzcd}
    \end{equation*}
    Therefore, we have an induced map $\Nm_{L/K} \colon C_L \to C_K$.
\end{defn}

\section{Statement of global class field theory}%
\label{sec:statement_of_global_class_field_theory}

Our goal is now to construct the global Artin reciprocity map. We hope to do this using the local Artin reciprocity maps. First, we recall some properties from algebraic number theory.\footnote{The note taker does not actually know any number theory. Please send help.} If $L/K$ is a finite Galois extension of number fields, $v$ a place of $K$, and $w \mid v$ a place of $L$ above $v$, we can define the \textit{decomposition group}
\[ D(w) \coloneqq \qty{\sigma \in \Gal(L/K) \mid \sigma w = w} = \Gal(L_w/K_v). \]

\begin{rmk}
    Suppose that $w' \mid v$ is another place of $L$ above $v$. Then $D(w), D(w')$ are conjugate in $\Gal(L/K)$. In particular, if $\Gal(L/K)$ is abelian, then $D(w) = D(w')$ only depends on $v$. By local class field theory, we obtain 
    \[ \phi_v \colon K_v^{\times} \to \Gal(L_w/K_v) = D(w) \hookrightarrow \Gal(L/K) \]
    which only depends on $v$.
\end{rmk}

\begin{prop}
    There exists a unique continuous homomorphism
    \[ \phi_K \colon \I_K \to {\Gal(\ol{K}/K)}^{\mr{ab}} \]
    such that for any finite abelian extension $L/K$ and any place $v$ of $K$, the diagram
    \begin{equation*}
    \begin{tikzcd}
        \I_K \ar{r}{\phi_{L/K}} & \Gal(L/K) \\
        K_v^{\times} \ar{r}{\phi_v} \ar[hookrightarrow]{u} & \Gal(L_w/K_v) \ar[hookrightarrow]{u}
    \end{tikzcd}
    \end{equation*}
    commutes.
\end{prop}

\begin{proof}
    Let $a = (a_v) \in \I_K$. If $a_v \in \mc{O}_v^{\times}$ and $L_w/K_v$ is unramified (true for all but finitely many places), then $\phi_v(a_v) = 1 \in \Gal(L_w/K_v)$. Now $\phi_K$ is uniquely determined by a finite product
    \[ \phi_K((a_v)) = \prod_v \phi_v(a_v) \in \Gal(L/K), \]
    where $v$ runs over the places where $a_v \notin \mc{O}_v^{\times}$ or $L_v/K_v$ is ramified. Therefore, varying $L/K$ and taking a limit, this determines $\phi_K$ uniquely. 

    Now it remains to check that $\phi_K$ is continuous. In particular, we need the kernel to be open. Let $S$ be the set of ramified places of $L/K$. Then by local class field theory, the diagram
    \begin{equation*}
    \begin{tikzcd}
        \I_{K,S} \ar{r}{\phi_{L/K}} & \Gal(L/K) \\
        \I_{L,S} \ar{u}{\Nm_{L/K}} \ar{r}{\phi_{L/L}} & \Gal(L/L) \ar{u}
    \end{tikzcd}
    \end{equation*}
    commutes. Therefore $\ker(\phi_{L/K}) \supseteq \Nm_{L/K} (\I_{L,S})$, which is open. Therefore the kernel itself is also open.
\end{proof}

\begin{thm}[Global class field theory]\leavevmode
    \begin{enumerate}
        \item (Global Artin reciprocity) The homomorphism $\phi_K \colon \I_K \to \Gal(K^{\mr{ab}}/K)$ satisfies
            \begin{enumerate}
                \item $\phi_K(K^{\times}) = 1$. Therefore we obtain a global Artin map $\phi_K \colon C_k \to \Gal(K^{\mr{ab}}/K)$.
                \item For any finite abelian extension $L/K$, $\phi_K$ induces an isomorphism 
                    \[ \phi_{L/K} \colon C_k/\Nm(C_L) \simeq \Gal(L/K). \]
            \end{enumerate}
        \item (Global existence theorem) For any finite index open subgroup $N \subseteq C_K$, there exists a finite abelian extension $L/K$ such that $N = \Nm(C_L)$.
    \end{enumerate}
\end{thm}

\begin{cor}
    We have a bijection between finite abelian extensions $L/K$ and finite index open subgroups of $C_K$.
\end{cor}

Before we turn to the proof of global class field theory, we will describe some consequences of this bijection. We will begin with ray class fields.

\begin{defn}
    A \textit{modulus} of $K$ is a function
    \[ \mf{m} \colon \qty{\text{places of $K$}} \to \Z_{\geq 0} \]
    such that 
    \begin{enumerate}
        \item $\mf{m}(v) = 0$ for all but finitely many places $v$.
        \item $\mf{m}(v) = 0,1$ if $v$ is a real place.
        \item $\mf{m}(v) = 0$ if $v$ is a complex place.
    \end{enumerate}
\end{defn}


\begin{defn}
    For a modulus $\mf{m}$, define the \textit{principal congruence subgroup}
    \[ \I_K^m \coloneqq \prod_{v \nmid \infty} U_{v, \mf{m}(v)} \times_{v \mid \infty} K_{v, \mf{m}(v)}^{\times}, \]
    where $U_{v,0} = \mc{O}_v^{\times}$ and $U_{v,i} = 1 + \mf{p}_v^i$ for $i \geq 1$ for finite places. For infinite places, we have $K_{v,0}^{\times} = K_v^{\times}$, and $K_{v,1}^{\times} = \R_{>0}$.
\end{defn}

\begin{defn}
    For a modulus $\mf{m}$, define
    \[ C_k^{ \mf{m} } \coloneqq \frac{\I_K^{ \mf{m} } \cdot K^{\times}}{K^{\times}}. \]
    Then define the \textit{ray class group} to be
    \[ \Cl_{\mf{m}} \coloneqq C_K / C_K^m. \]
\end{defn}

Here, the term ``ray'' comes from a positivity condition $\R_{>0} \subset \R^{\times}$ at real places, which geometrically means specifying a ray.

\begin{defn}
    Write $\mf{m} = \prod_v \mf{p}_v^{\mf{m}(v)} = \mf{m}_0 \cdot \mf{m}_{\infty}$. Then we have a description
    \[ \Cl_{\mf{m}} \coloneqq \frac{ \qty{\text{fractional ideals coprime to $\mf{m}_0$}} }{\qty{x \in K^{\times}, x \in U_{v, \mf{m}(v)} (v \mid \mf{m}_0), x \in \R_{v,>0} (v \mid \mf{m}_{\infty})}}. \]
    In particular, if $\mf{m}(v) \equiv 0$, then $\mf{m} = \prod_v \mf{p}_v^0 = 1$ and $\Cl_1 = \Cl_K$. On the other hand, if $\mf{m} = \mf{m}_{\infty}$, then
    \[ \Cl_{\mf{m}_{\infty}} = \frac{\qty{\text{fractional ideals}}}{\qty{x \in K^{\times} \mid x\ \text{totally positive}}} = \Cl_K^+. \]
    This is called the \textit{narrow class group}. 
\end{defn}

\begin{defn}
    The abelian extension $L/K$ associated to $\Cl_{ \mf{m} } \cong \Gal(L/K)$ under global class field theory is called the \textit{ray class field of $\mf{m}$}. The case of $\mf{m}=1$ is called the \textit{Hilbert class field} and is denoted by $H$ with $\Cl_K = \Gal(H/K)$. The case where $\mf{m} = \mf{m}_{\infty}$ gives us the \textit{narrow Hilbert class field} and is denoted by $H^+$. This is characterized by $\Cl_K^+ = \Gal(H^+/K)$. In particular, $H \subseteq H^+$.  
\end{defn}

\begin{rmk}
    By local class field theory, the ramified places of the ray class field of $m$ is contained in $\mf{m}$. In particular, $H/K$ is the maximal abelian abelian extension unramified everywhere, and $H^+/K$ is the maximal abelian extension unramified at all finite places.
\end{rmk}

\begin{exm}
    If $K = \Q$, then $\Cl_K = \Cl_L^+ \cong \qty{1}$ and thus $H = H^+ = \Q$.
\end{exm}

\begin{exm}
    If $K = \Q(\sqrt{3})$, then $\Cl_K = \qty{1}$. However, we have
    \[ \Cl_K^+ = \frac{K^{\times}}{K_{>0}^{\times} \cdot \mc{O}_K^{\times}} = \Z/2. \]
    Therefore $H = K = \Q(\sqrt{3})$ and $H^+ = K(i) = \Q(\sqrt{3}, i)$.
\end{exm}

\begin{exm}
    Let $K = \Q$ and let $\mf{m} = (m)$ for some $m \in \Z$. Then
    \[ \Cl_{(m)} = \frac{\qty{\qty(\frac{r}{s}) \in \Q^{\times} \mid (r,m) = (s,m) = 1}}{\qty{\frac{r}{s} \in \Q^{\times} \mid \frac{r}{s} \equiv 1 \pmod m}} = {(\Z/m)}^{\times} / (\pm 1). \]
    If $\mf{m} = (m) \cdot \infty$, then $\Cl_{(m) \cdot \infty} = {(\Z/m)}^{\times}$.

    Now the ray class fields $L/\Q$ for either $2 \nmid m$ or $4 \mid m$ are given by $\Cl_{(m) \cdot \infty} \cong \Gal(L/\Q)$, where $L = \Q(\zeta_m)$, and the field given by $\Cl_{(m)}$ is $L = \Q(\zeta_m + \zeta_m^{-1})$. In particular, by varying $m$, we obtain the \textit{Kronecker-Weber theorem}. 
\end{exm}

\begin{thm}[Kronecker-Weber]
    We have $\Q^{\mr{ab}} = \bigcup_{m \geq 1} \Q(\zeta_m)$ and thus 
    \[ \Gal(\Q^{\mr{ab}}/\Q) = \varprojlim_m {(\Z/m)}^{\times} = \wt{Z}^{\times} = \prod_p \Z_p^{\times}. \]
\end{thm}

\begin{rmk}
    Hilbert's 12th problem asks for an explicit description of the ray class fields for a general number field. For $K = \Q$, we have the Kronecker-Weber theorem. For imaginary quadratic fields, we have modular functions and elliptic curves with complex multiplication. In general, this problem is still open.
\end{rmk}

We will use the following strategy to prove global Artin reciprocity:
\begin{enumerate}
    \item We will prove the first inequality\footnote{Some other references will call this the second inequality because they prove this second.} that $[C_K:\Nm(C_L)] \geq [L:K]$. This will be a consequence of the cohomology of $\I_K, C_K$. In particular, we will show that $h(C_K) = [L:K]$. This will imply that $\abs{\wh{H}^0(G, C_L)} \geq [L:K]$, but this is precisely the index we need.
    \item Next, we will prove the surjectivity of $\I_K \xrightarrow{\phi_{L/K}} \Gal(L/K)$. This is an application of the first inequality and the weak Chebotarev density theorem.
    \item We will prove the second inequality\footnote{In the first proof of class field theory, this was the first inequality.} $[C_K : \Nm(C_L)] \leq [L:K]$. This is a consequence of Chebotarev density and we will use the analytic tool of $L$-functions.
    \item We will prove that $\phi_K(K^{\times}) = 1$. This will follow from the determination of the Brauer groups of $K$.
\end{enumerate}

\section{Cohomology of Id\`eles and first inequality}%
\label{sec:cohomology_of_id`eles}

Let $L/K$ be a finite Galois extension of number fields. Recall that if $v$ is a place of $K$, then $L \otimes_K K_v = \prod_{w \mid v} L_w$. The action of $\Gal(L/K)$ on $L \otimes K_v$ induces an action on the product by permuting the places above $v$. If $\alpha = {(\alpha_w)}_{w \mid v}$ and $\sigma \in \Gal(L/K)$, then ${(\sigma \alpha)}_{\sigma w} = \sigma (\alpha_w) \in L_{\sigma w}$.

\begin{prop}
    Fix $w_0 \mid v$. Then there exists an isomorphism of $G \coloneqq \Gal(L/K)$-modules
    \[ \prod_{w \mid v} L_w \cong \Ind_{G_{w_0}}^G L_{w_0}, \]
    where $G_{w_0} = D(w_0)$ is the decomposition group.
\end{prop}

\begin{proof}
    Recall that $\Ind_{G_{w_0}}^G L_{w_0} \coloneqq \qty{f \colon G \to L_{w_0} \mid f(\tau \sigma) = \tau f(\sigma)\text{ for all }\tau \in G_{w_0}, \sigma \in G}$. For any $\alpha = {(\alpha_w)}_w$, we define
    \[ f_{\alpha} \colon G \to L_{w_0} \qquad \sigma \mapsto \sigma(\alpha_{\sigma^{-1} w_0}). \]
    Then it is easy to compute
    \[ f_{\alpha}(\tau \sigma) = \tau \sigma (\alpha_{\tau \sigma^{-1} w_0}) = \tau (\sigma \alpha_{\sigma^{-1} w_0}) = \tau f_{\alpha}(\sigma). \]
    Conversely, for $f \in \Ind_{G_{w_0}}^G L_{w_0}$, we define ${(\alpha_f)}_{\sigma w_0} \coloneqq \sigma f(\sigma^{-1})$. This is well-defined by $G_{w_0}$-equivariance, and we can check that these two assignments are inverse to each other.
\end{proof}

\begin{cor}
    By Shapiro's lemma, we have $H^r\qty(G, \prod_{w \mid v} L_w) = H^r(G_{w_0}, L_{w_0})$ for all $r \geq 0$.
\end{cor}

Similarly, we have
\begin{cor}
    For all $r \geq 0$, $H^r\qty(G, \prod_{w \mid v} L_w^{\times}) = H^r(G_{w_0}, L_{w_0}^{\times})$ and $H^r\qty(G, \prod_{w \mid v} U_w) = H^r(G_{w_0}, U_{w_0})$. 
\end{cor}

In particular, all of these are independent of the choice of $w_0 \mid v$. To stress the independence, we write $G^v \coloneqq G_{w_0}, L^v \coloneqq L_{w_0}$, and so on and so forth. 

\begin{prop}\leavevmode
    We have the following cohomology of the id\`eles:
    \begin{enumerate}
        \item $H^0(G, \I_L) = \I_K$.
        \item For all $r \in \Z$, $H^r(G, \I_L) = \bigoplus_v \wh{H}^r(G^v, {(L^v)}^{\times})$.
    \end{enumerate}
\end{prop}

\begin{proof}\leavevmode
    \begin{enumerate}
        \item We know that $a = {(a_w)} \in \I_L$ is $G$-invariant if and only if ${(a_w)}_{w \mid v}$ is $G$-invariant for all places $v$ of $K$ if and only if $a_w \in K_v^{\times}$ is independent of the choice of $w \mid v$ for all $v$, and this is equivalent to $a = {(a^v)}_v \in \I_K$.
        \item Let $S$ be a finite set of places of $K$ containing $S_{\infty}$. Write
            \[ \I_{L, S} = \qty(\prod_{v \in S} \prod_{w \mid v} L_w^{\times}) \times \qty(\prod_{v \notin S} \prod_{w \mid v} U_w). \]
            Then $\I_L = \varinjlim \I_{L, S}$ and therefore 
            \begin{align*}
                \wh{H}^r(G, \I_L) &= \varinjlim \wh{H}^r(G, \I_{L, S}) \\
                                  &= \varinjlim \qty(\prod_{v \in S} \wh{H}^r(G^v, {(L^v)}^{\times})) \times \qty(\prod_{v \notin S} \wh{H}^r(G^v, U^v)).
            \end{align*}
            If $S$ contains all places of $K$ that are ramified in $L$, then for all $v \notin S$, $L^v/K^v$ is unramified. By local class field theory, $\wh{H}^r(G^v, U^v) = 0$, so 
            \[ \wh{H}^r(G, \I_L) = \varinjlim \prod_{v \in S} \wh{H}^r(G^v, {(L^v)}^{\times}) = \bigoplus_v \wh{H}^r(G^v, {(L^v)}^{\times}). \qedhere \]
    \end{enumerate}
\end{proof}

\begin{cor}
    We have $H^1(G, \I_L) = 0$ and $H^2(G, \I_L) = \bigoplus_v \frac{1}{n_v} \Z/\Z$, where $n_v = [L^v: K_v]$.
\end{cor}

These statements follow from collecting all of the local information.

\begin{cor}
    Assume $L/K$ is cyclic. Assume $S$ contains all places of $K$ that are ramified in $L$ and the infinite places. Then $h(\I_L, S) = \prod_{v \in S} n_v$.
\end{cor}

\begin{proof}
    By local class field theory, we have
    \[ h(\I_{L, S}) = \prod_{v \in S} h({(L^v)}^{\times}) \prod_{v \notin S} h(U^v) = \prod_{v \in S} n_v \prod_{v \notin X} 1 = \prod_{v \in S} n_v. \qedhere \]
\end{proof}

\begin{defn}
    Let $T \coloneqq \qty{w \mid v : v \in S}$ be a finite set of places of $L$. Define the group of \textit{$T$-units} of $L$ to be
    \[ U(T) \coloneqq L^{\times} \cap \I_{L, S} = \qty{\alpha \in L^{\times} \mid \alpha \in U_w, w \notin T}. \]
\end{defn}

Our next goal is to compute $h(U(T))$ and use this to prove the first inequality. First, we must compute the cohomology of units. Let $L/K$ be a finite Galois extension of number fields and $S$ be a finite set of places containing $S_{\infty}$. Let $T$ be the set of places of $L$ above $S$.

\begin{prop}
    Assume $L/K$ is cyclic. Then 
    \[ h(U(T)) = \frac{\prod_{v \in S} n_v}{n}, \] 
    where $n_v = [L^v:K_v], n=[L:K]$.
\end{prop}

\begin{proof}
    We need to use the following comparison result for Herbrand quotients from the homework: let $G$ be a cyclic group and $V$ an $\R[G]$-module. If $M, N \subseteq V$ are two $G$-stable lattices, then $h(M) = h(N)$. We will apply this to the case where $V = \R^{\abs{T}} = \Hom_{\ms{Set}}(T, \R)$.
    \begin{enumerate}
        \item Consider $N = \Hom_{\ms{Set}}(T, \Z)$. This is clearly a $G$-stable lattice. As a $G$-module, we have
            \begin{align*}
                N &= \bigoplus_{v \in S} \Hom_{\ms{Set}}(G/G^v, \Z) \\
                  &= \bigoplus_{v \in S} \Ind_{G^v}^G \Z.
            \end{align*}
            In particular, we have
            \begin{align*}
                h(N) &= \prod_{v \in S} h(\Ind_{G^v}^G \Z) \\ 
                     &= \prod_{v \in S} h(\Z) \\
                     &= \prod_{v \in S} \abs{G^v} \\
                     &= \prod_{v \in S} n_v. 
            \end{align*}
        \item Now consider $\lambda \colon U(T) \to V = \R^{\abs{T}}$ given by $\alpha \mapsto {(\log \abs{\alpha}_w)}_{w \in T}$. If $M^0 = \Im(\lambda)$, Dirichlet's unit theorem for $U(T)$ says that $M^0$ is a lattice of rank $\abs{T} - 1$ in the hyperplane $\qty{\sum_{w \in T} x_w = 0}$. Then $M = M^0 \oplus \Z \cdot (1, \ldots, 1)$ is a $G$-stable lattice.

            Now we know that $h(M) = h(N)$, so $h(U(T)) = h(M^0)$ because the kernel of $\lambda$ is the set of roots of unity in $L$, which is finite. Now we obtain
            \[ h(U(T)) = h(M^0) = \frac{h(M)}{h(\Z)} = \frac{\prod_{v \in S} n_v}{n}, \]
            as desired. \qedhere
    \end{enumerate}
\end{proof}

Now we are able to compute the cohomology of the id\`ele class group $C_K$. 

\begin{lem}
    For $L/K$, we have $H^0(G, C_L) = C_K$.
\end{lem}

\begin{proof}
    The short exact sequence in $\Mod_G$
    \[ 1 \to L^{\times} \to \I_L \to C_L \to 1 \]
    gives a long exact sequence
    \[ 0 \to H^0(G, L^{\times}) \to H^0(G, \I_L) \to H^0(G, C_L) \to H^1(G, L^{\times}) = 0, \]
    where the last equality is by Hilbert 90. This implies that $H^0(G, C_L) = \I_K/K^{\times} = C_K$.
\end{proof}

We need to compute the Herbrand quotient of the id\`ele class group, so first we will rewrite $\I_K$ in terms of $S$-units.

\begin{lem}
    Choose $S$ containing a generating set of prime ideals in $\Cl_K$. Then $\I_K = K^{\times} \cdot \I_{K, S}$.
\end{lem}

\begin{proof}
    Recall that we have a surjection 
    \[ \I_K \twoheadrightarrow I_K \qquad {(a_v)}_v \mapsto \prod_{v \nmid \infty} \mf{p}_v^{\mr{ord}_v(a_v)}. \]
    with kernel $\I_{K, S_{\infty}}$. In particular, we can write
    \[ \Cl_K = \I_K / K^{\times} = \I_K / K^{\times} \I_{K, S_{\infty}}. \]
    By the choice of $S$, we have
    \[ \I_K/K^{\times} \I_{K,S} = \Cl_K/\ev{\mf{p} \in S} = \Cl_K / \Cl_K = 0. \qedhere \]
\end{proof}

\begin{thm}
    Assume $L/K$ is cyclic. Then $h(C_L) = [L:K]$. In particular, $[C_K: \Nm(C_L)] \geq [L:K]$.
\end{thm}

\begin{proof}
    Choose $S$ such that $T$ a generating set of prime ideals for $\Cl_L$. Then we can rewrite
    \[ C_L = \I_L/L^{\times} = L^{\times} \I_{L.T}/L^{\times} = \I_{L,T}/I_{L,T} \cap L^{\times} = \I_{L,T}/U(T). \]
    Now we have 
    \[ h(C_L) = \frac{h(\I_{L, T})}{h(U(T))} = \frac{\prod_{v \in S} n_v}{\frac{1}{n} \prod_{v \in S} n_v} = n = [L:K]. \qedhere \]
\end{proof}

As an application of the first inequality, we will prove

\begin{lem}
    Asssume $L/K$ is solvable. If there exists $D \subseteq \I_K$ such that $D \subseteq \Nm(\I_L)$ and $K^{\times} \cdot D \subseteq \I_K$ is dense, then $L = K$.
\end{lem}

\begin{proof}
    Assume $L \neq K$. Then we may find a tower $K \subseteq K' \subseteq L$ such that $K'/K$ is cyclic. Then we know $D \subseteq \Nm(\I_K')$. By local class field theory, $\Nm(\I_{K'}) \subseteq \I_K$ is an open subgroup, so $K^{\times} \cdot \Nm(\I_{K'})$ is also an open subgroup. Therefore $K^{\times} \cdot \Nm(\I_{K'})$ is also closed. By density, we see that $K^{\times} \Nm(I_{K'}) = \I_K$. In particular, $[C_K : \Nm(C_{K'})] = 1$, so $[K':K] = 1$, which is a contradiction. Thus $L = K$.
\end{proof}

\begin{prop}
    Assume $L/K$ is solvable. If all but finitely many places $v$ of $K$ split completely in $L$, then $L = K$. Equivalently, if $L \neq K$, then there are infinitely many places $v$ of $K$ that do not split completely in $L$.
\end{prop}

\begin{proof}
    Choose $S$ to contain the set of all $v$ that do not split completely in $L$. Also choose
    \[ D = \qty{{(a_v)}_v \mid a_v \in K_v^{\times} (v \in S), a_v = 1 (v \notin S)} \subseteq \I_K. \]
    Then if $v \notin S$, it splits completely in $L$, so $L^v / K_v$ is a trivial extension. Therefore $D \subseteq \Nm(\I_L)$. On the other hand, by weak approximation, any ${(a_v)}_{v \in S}$ can be approximated by a global element $a \in K^{\times}$, so $K^{\times} \cdot D$ is dense in $\I_K$. By the previous lemma, $L = K$.
\end{proof}

\begin{prop}
    Assume that $L/K$ is solvable. Then $\qty{\Frob_{w/v}}_{v\ \text{unramified}}$ generate $\Gal(L/K)$.
\end{prop}

\begin{proof}
    Take $H = \ev{\Frob_{w/v}} \subseteq \Gal(L/K)$. Choose $E = L^H$. If $v$ is unramified in $L$, then $v$ splits completely in $L^H$. But this means that $E = K$ because only finitely many places are ramified, so $H = \Gal(L/K)$.
\end{proof}

\begin{cor}
    Assume $L/K$ is abelian. Then the Artin map $\phi_K \colon \I_K \to \Gal(L/K)$ is surjective.
\end{cor}

\begin{proof}
    By local class field theory, if $v$ is unramified in $L$, then $\phi_K (1, 1, \ldots, \pi_v, 1, \ldots) = \Frob_{w/v}$. We know that the $\Frob_{w/v}$ generate $\Gal(L/K)$, so $\phi_K$ is surjective onto $\Gal(L/K)$.
\end{proof}











\end{document}
