\documentclass[leqno, openany]{memoir}
\setulmarginsandblock{3.5cm}{3.5cm}{*}
\setlrmarginsandblock{3cm}{3.5cm}{*}
\checkandfixthelayout

\usepackage{amsmath}
\usepackage{amssymb}
\usepackage{amsthm}
%\usepackage{MnSymbol}
\usepackage{bm}
\usepackage{accents}
\usepackage{mathtools}
\usepackage{tikz}
\usetikzlibrary{calc}
\usetikzlibrary{automata,positioning}
\usepackage{tikz-cd}
\usepackage{forest}
\usepackage{braket} 
\usepackage{listings}
\usepackage{mdframed}
\usepackage{verbatim}
\usepackage{physics}
\usepackage{stackengine} 

%font
\usepackage[sc]{mathpazo}
\usepackage{eulervm}
\usepackage[scaled=0.86]{berasans} 
\usepackage{inconsolata}
\usepackage{microtype}

%CS packages
\usepackage{algorithmicx}
\usepackage{algpseudocode}
\usepackage{algorithm}

% typeset and bib
\usepackage[english]{babel} 
\usepackage[utf8]{inputenc} 
\usepackage[T1]{fontenc} 
\usepackage[backend=biber, style=alphabetic]{biblatex}
\usepackage[bookmarks, colorlinks, breaklinks]{hyperref} 
\hypersetup{linkcolor=black,citecolor=black,filecolor=black,urlcolor=black}

% other formatting packages
\usepackage{float}
\usepackage{booktabs}
\usepackage{enumitem}
\usepackage{csquotes}
\usepackage{titlesec}
\usepackage{titling}
\usepackage{fancyhdr}
\usepackage{lastpage}
\usepackage{parskip}

\usepackage{lipsum}

% delimiters
\DeclarePairedDelimiter{\gen}{\langle}{\rangle}
\DeclarePairedDelimiter{\floor}{\lfloor}{\rfloor}
\DeclarePairedDelimiter{\ceil}{\lceil}{\rceil}


\newtheorem{thm}{Theorem}[section]
\newtheorem{cor}[thm]{Corollary}
\newtheorem{prop}[thm]{Proposition}
\newtheorem{lem}[thm]{Lemma}
\newtheorem{conj}[thm]{Conjecture}
\newtheorem{quest}[thm]{Question}

\theoremstyle{definition}
\newtheorem{defn}[thm]{Definition}
\newtheorem{defns}[thm]{Definitions}
\newtheorem{con}[thm]{Construction}
\newtheorem{exm}[thm]{Example}
\newtheorem{exms}[thm]{Examples}
\newtheorem{notn}[thm]{Notation}
\newtheorem{notns}[thm]{Notations}
\newtheorem{addm}[thm]{Addendum}
\newtheorem{exer}[thm]{Exercise}

\theoremstyle{remark}
\newtheorem{rmk}[thm]{Remark}
\newtheorem{rmks}[thm]{Remarks}
\newtheorem{warn}[thm]{Warning}
\newtheorem{sch}[thm]{Scholium}


% unnumbered theorems
\theoremstyle{plain}
\newtheorem*{thm*}{Theorem}
\newtheorem*{prop*}{Proposition}
\newtheorem*{lem*}{Lemma}
\newtheorem*{cor*}{Corollary}
\newtheorem*{conj*}{Conjecture}

% unnumbered definitions
\theoremstyle{definition}
\newtheorem*{defn*}{Definition}
\newtheorem*{exer*}{Exercise}
\newtheorem*{defns*}{Definitions}
\newtheorem*{con*}{Construction}
\newtheorem*{exm*}{Example}
\newtheorem*{exms*}{Examples}
\newtheorem*{notn*}{Notation}
\newtheorem*{notns*}{Notations}
\newtheorem*{addm*}{Addendum}


\theoremstyle{remark}
\newtheorem*{rmk*}{Remark}

% shortcuts
\newcommand{\Ima}{\mathrm{Im}}
\newcommand{\A}{\mathbb{A}}
\newcommand{\N}{\mathbb{N}}
\newcommand{\R}{\mathbb{R}}
\newcommand{\C}{\mathbb{C}}
\newcommand{\Z}{\mathbb{Z}}
\newcommand{\Q}{\mathbb{Q}}
\renewcommand{\k}{\Bbbk}
\renewcommand{\P}{\mathbb{P}}
\newcommand{\M}{\overline{M}}
\newcommand{\g}{\mathfrak{g}}
\newcommand{\h}{\mathfrak{h}}
\newcommand{\n}{\mathfrak{n}}
\renewcommand{\b}{\mathfrak{b}}
\newcommand{\ep}{\varepsilon}
\newcommand*{\dt}[1]{%
   \accentset{\mbox{\Huge\bfseries .}}{#1}}
\renewcommand{\abstractname}{Official Description}
\newcommand{\mc}[1]{\mathcal{#1}}
\newcommand{\T}{\mathbb{T}}
\newcommand{\mf}[1]{\mathfrak{#1}}
\newcommand{\mr}[1]{\mathrm{#1}}
\newcommand{\ms}[1]{\mathsf{#1}}
\newcommand{\ol}[1]{\overline{#1}}
\newcommand{\ul}[1]{\underline{#1}}
\newcommand{\wt}[1]{\widetilde{#1}}
\newcommand{\wh}[1]{\widehat{#1}}
\newcommand{\pt}{\mathrm{pt}}
\renewcommand{\op}{\mathrm{op}}

\DeclareMathOperator{\Der}{Der}
\DeclareMathOperator{\Hom}{Hom}
\DeclareMathOperator{\End}{End}
\DeclareMathOperator{\ad}{ad}
\DeclareMathOperator{\Aut}{Aut}
\DeclareMathOperator{\Gal}{Gal}
\DeclareMathOperator{\Rad}{Rad}
\DeclareMathOperator{\supp}{supp}
\DeclareMathOperator{\Supp}{Supp}
\DeclareMathOperator{\sgn}{sgn}
\DeclareMathOperator{\spec}{Spec}
\DeclareMathOperator{\Spec}{Spec}
\DeclareMathOperator{\Proj}{Proj}
\DeclareMathOperator{\Ext}{Ext}
\DeclareMathOperator{\Tor}{Tor}
\DeclareMathOperator{\Ann}{Ann}
\DeclareMathOperator{\Ass}{Ass}
\DeclareMathOperator{\dpth}{depth}
\DeclareMathOperator{\pdim}{proj.dim}
\DeclareMathOperator{\idim}{inj.dim}
\DeclareMathOperator{\gdim}{gl.dim}
\DeclareMathOperator{\Pic}{Pic}
\DeclareMathOperator{\codim}{codim}
\DeclareMathOperator{\Bl}{Bl}

% Section formatting
\titleformat{\section}
    {\Large\sffamily\scshape\bfseries}{\thesection}{1em}{}
\titleformat{\subsection}[runin]
    {\large\sffamily\bfseries}{\thesubsection}{1em}{}
\titleformat{\subsubsection}[runin]{\normalfont\itshape}{\thesubsubsection}{1em}{}

\title{COURSE TITLE}
\author{Lectures by INSTRUCTOR, Notes by NOTETAKER}
\date{SEMESTER}

\newcommand*{\titleSW}
    {\begingroup% Story of Writing
    \raggedleft
    \vspace*{\baselineskip}
    {\Huge\itshape Algebraic Geometry \\ Spring 2021}\\[\baselineskip]
    {\large\itshape Notes by Patrick Lei}\\[0.2\textheight]
    {\Large Lectures by Giulia Sacc\`a}\par
    \vfill
    {\Large \sffamily Columbia University}
    \vspace*{\baselineskip}
\endgroup}
\pagestyle{simple}

\chapterstyle{ell}


%\renewcommand{\cftchapterpagefont}{}
\renewcommand\cftchapterfont{\sffamily}
\renewcommand\cftsectionfont{\scshape}
\renewcommand*{\cftchapterleader}{}
\renewcommand*{\cftsectionleader}{}
\renewcommand*{\cftsubsectionleader}{}
\renewcommand*{\cftchapterformatpnum}[1]{~\textbullet~#1}
\renewcommand*{\cftsectionformatpnum}[1]{~\textbullet~#1}
\renewcommand*{\cftsubsectionformatpnum}[1]{~\textbullet~#1}
\renewcommand{\cftchapterafterpnum}{\cftparfillskip}
\renewcommand{\cftsectionafterpnum}{\cftparfillskip}
\renewcommand{\cftsubsectionafterpnum}{\cftparfillskip}
\setrmarg{3.55em plus 1fil}
\setsecnumdepth{subsection}
\maxsecnumdepth{subsection}
\settocdepth{subsection}

\begin{document}
    
\begin{titlingpage}
\titleSW
\end{titlingpage}

\thispagestyle{empty}
\section*{Disclaimer}%
\label{sec:disclaimer}

Unless otherwise noted, these notes were taken during lecture using the
\texttt{vimtex} package of the editor \texttt{neovim}.  Any errors are mine and
not the instructor's.  In addition, my notes are picture-free (but will include
commutative diagrams) and are a mix of my mathematical style and that of the
instructor.  If you find any errors, please contact me at
\texttt{plei@math.columbia.edu}.

Giulia's iPad died at the beginning of the final lecture, and the rest of the
lecture was spent with a phone camera pointed at a piece of paper, but the
phone locked itself and eventually died as well. Giulia sent notes, so the
transcription of these has replaced the live-{\TeX}ed notes.

\section*{Acknowledgements}% \label{sec:acknowledgements}

I would like to thank Nicol\'as Vilches for pointing out mistakes and typos in
these notes.

\newpage

\tableofcontents

\chapter{Schemes}% \label{cha:schemes}

\section{Affine Schemes}% \label{sec:affine_schemes}

Let $R$ be a commutative ring. We will define the scheme $\Spec R$ as a set, a
topological space, and finally as a locally ringed space. Our goal is for $R$
to be the ring of functions on $\Spec R$.

\begin{defn} We will define the \textbf{set} $\Spec R$ to be the set of prime
ideals $P \subset R$. Here, note that $R$ is not a prime ideal and that $(0)$
is prime if $R$ is a domain.  \end{defn}

\begin{exm} If $R = \Z$, then $\Spec \Z$ is the set of prime numbers together
with $0$. If $R = k$ is a field, then $\Spec k = \qty{(0)}$. If $R = k[t]$,
then $\Spec R$ is the set of irreducible polynomials.  \end{exm}

We will place the \textit{Zariski topology} on $\Spec R$ by declaring the
closed sets to be $V(S) = \qty{\mf{p} \mid \mf{p} \supset S}$ for any subset $S
\subset R$. Some easy properties of $V(S)$ are: \begin{enumerate} \item If $S
    \subset T$, then $V(S) \supset V(T)$.  \item If $\mf{a} = (S) \subseteq R$,
    then $V(S) = V(\mf{a})$.  \item $V(S) = \emptyset$ if and only if $1 \in
    (S)$ and $V((0)) = V(\qty{0}) = \Spec R$.  \item Given an ideal $\mf{a}
    \subset R$, we have $V(\mf{a}) = V(\sqrt{\mf{a}})$.  \item We verify that
    this forms a topology: \begin{itemize} \item First, $V\qty(\bigcup_{\alpha}
        S_{\alpha}) = \bigcap_{\alpha} V(S_{\alpha})$.  \item Second, $V(\mf{a}
        \cdot \mf{a}') = V(\mf{a} \cap \mf{a}') = V(\mf{a}) \cup V(\mf{a}')$.
\end{itemize} \end{enumerate}

Proof of all of these is a simple exercise. If $R$ is considered as the set of
functions on $\Spec R$, then for $f \in R$ and $x = \mf{p} \in \Spec R$, we
need to define $f(x)$. For this, we consider the field of fractions $k(x) =
k(\mf{p})$ of $R/\mf{p}$. This is called the \textit{residue field}. 

\begin{exm} If $R = \Z$ and $x = (p)$ for $p \neq 0$, then $k(p) = \Z/p\Z$. If
$x = (0)$, then we see that $k(0) = \Q$.  \end{exm}

Now we define $f(x)$ to be the image of $f$ under the map $R \to R/\mf{p} \to
K(R/\mf{p}) = k(x)$. Then clearly $\qty{x \mid f(x) = 0}$ is the closed subset
$V(f)$.

\begin{defn} Given $X = \Spec R$ and $f \in R$, we define $X_f = X \setminus
V(f) = \Spec R[1/f]$. These are called the \textit{principal} (or
distinguished) open subsets.  \end{defn}

\begin{lem} Principal open subsets form a basis for the Zariski topology and
are closed under finite intersections.  \end{lem}

\begin{proof} If $U$ is open, then we can write $U = \Spec R \setminus V(S) =
    V\qty(\sum_{f \in S} (f)) = \bigcap_{f \in S} V(f) = \bigcup_{f \in S}
    \Spec R \setminus V(f)$, as desired. The proof that principal open subsets
    are closed under finite intersection is clear.  \end{proof}

\begin{lem} Let $g, f_i \in R$. Then $X_g \subseteq X_{f_i}$ if and only if
$V(g) \supset V(\mf{a}) = V(\sqrt{\mf{a}})$, where $\mf{a} = \sum (f_i)$.
\end{lem}

\begin{proof} We know $X_g \subseteq \bigcup X_{f_i}$ if and only if $V(g)
\supseteq \bigcap V(f_i)$, which is equivalent to the right hand side.
\end{proof}

\begin{cor} If $g = 1$, then $X = \bigcup X_{f_i}$ if and only if $1 \in \sum
    (f_i)$. In particular, because $1 = \sum a_i f_i$ is a finite sum, and
    therefore $X$ is a finite union of some of the $X_{f_i}$. This implies that
    $\Spec R$ is a quasi-compact topological space.  \end{cor}

\begin{defn} Let $Y \subseteq \Spec R = X$. Then define \begin{align*} I(Y) &=
\qty{f \in R \mid f(x) = 0 \text{ for all }y \in Y} \\ &= \qty{f \in R \mid f
\in \mf{p}\text{ for all } \mf{p} \in Y} \\ &= \bigcap_{\mf{p} \in Y} \mf{p}.
\end{align*} \end{defn}

\begin{prop} \begin{enumerate} \item For all ideals $\mf{a} \subset R$, we have
    $I(V(\mf{a})) = \sqrt{\mf{a}}$.  \item $V$ and $I$ define inverse
    bijections \[ \qty{\text{radical ideals}} \xleftrightarrow{V,I}
    \qty{\text{closed subsets of }\Spec R}. \] \item If $Y \subset \Spec R$ is
    a subset, then $V(I(Y)) = \ol{Y}$, the Zariski closure of $Y$.
    \end{enumerate} \end{prop}

\begin{proof} \begin{enumerate} \item If $f \in I(V(\mf{a}))$, then $f \in
    \mf{p}$ for all $\mf{p} \supseteq \mf{a}$ and thus $f \in \sqrt{\mf{a}}$.
\item This is left as an exercise.  \item Note that $V(b) \supset Y$ if and
    only if $b \subseteq \bigcap_{\mf{p} \in Y} = I(Y)$.  \end{enumerate}
\end{proof}

In particular, we see that in general $\Spec R$ has points that are not closed.

\subsection{A Bit About Classical Varieties}%
\label{sub:a_bit_about_classical_varieties}

Let $k$ be an algebraically closed field and $R = k[t_1, \ldots, t_n]$. 

\begin{defn} A \textit{closed algebraic subset} of $k^n$ (or $\A^n(k))$ is the
    common set of zeros $V(f_1, \ldots, f_m)$ of a finite set of polynomials
    $f_1, \ldots, f_m$. Here, all of the same properties of these vanishing
    sets from $\Spec R$ hold.  \end{defn}

Now recall the Nullstellensatz from commutative algebra, which says that if $k$
is a field and $B$ a finite $k$-algebra, then $B$ is a field and a finite
extension of $k$.

\begin{cor} All maximal ideas of $k[t_1, \ldots, t_n]$ are $\mf{m} = (t_1 -
x_1, \ldots, t_n - x_n)$ for $x_i \in k$.  \end{cor}

\begin{cor} There is a bijection between radical ideals of $k[t_1, \ldots,
t_n]$ and closed algebraic subsets of $k^n$ given by $V$ and $I$.  \end{cor}

\subsection{Back to Affine Schemes}% \label{sub:back_to_affine_schemes}

\begin{exms} If $R$ is a PID, then we can write $0 \neq f = \prod_{i=1}^r
    p_i^{n_i}$ and therefore the closed subsets of $\Spec R$ are either $\Spec
    R$ or a finite union of closed points. If $R$ is also a local ring, then
    $\Spec R = \qty{(0), \mf{m}}$. If $\mf{a} \subset A$ and consider $R =
    A/\mf{a}$. Then $\Spec A/\mf{a} = V(\mf{a})$. If $f \neq 0$ is not
    nilpotent, then $\Spec R_f$ is the set of prime ideals not containing $f$,
    which is ${(\Spec R)}_f$.  \end{exms}

Suppose $\mf{p} \in \Spec R$. Then we know that $\ol{\mf{p}} = V(\mf{p}) \cong
\Spec R/\mf{p}$. This tells us that $x \in \Spec R$ is a closed point if and
only if it corresponds to a maximal ideal. 

\begin{rmk} Note that if $k$ is not algebraically closed, $k^n$ is
\textbf{different} from $\Spec k[t_1, \ldots, t_n]$.  \end{rmk}

\begin{exm} Let $R$ be a domain. Then we see that $(0) \in \Spec R$ is a
generic point (it is dense). We will see that it is the unique generic point.
\end{exm}

\begin{defn} Let $X$ be a topological space. A closed subset $Z \subseteq X$ is
called \textit{irreducible} if it is not the union of two proper closed
subsets.  \end{defn}

\begin{prop} A closed subset $Y \subseteq \Spec R$ is irreducible if and only
if $I(Y)$ is prime. Moreover, any closed irreducible subset has a unique
generic point.  \end{prop}

\begin{proof} Let $Y = V(\mf{a})$ and suppose $\mf{a} = \sqrt{\mf{a}} = I(Y)$.
    Then if $\mf{a} = \mf{p}$ then $\ol{\mf{p}} = Y$ and thus $Y$ is
    irreducible. In the other hand, if $fg \in I(Y)$, then $fg(x) = 0$ for all
    $x \in Y$, and this means either $f(x) = 0$ or $g(x) = 0$. This implies
    that $f \in \mf{p}$ for all $\mf{p} \in Y$ or $g \in \mf{p}$ for all
    $\mf{p} \in Y$. Then we can write $Y = ( V(f) \cap Y ) \cup ( V(g) \cap Y
    )$ and by irreducibility, we see that $Y = V(f) \cap Y$, which implies that
    $f \in I(Y)$.

    To prove the uniqueness of the generic point, we see that if there is more
than one, then their closures are the same, so they contain each other and thus
must be the same. For the existence of the generic point, we know $Y =
V(\mf{p})$ for a prime ideal $\mf{p}$ and thus $\mf{p}$ is the generic point.
\end{proof}

\begin{notn} For an irreducible closed subset $Y \subseteq \Spec R$ we will
denote by $\eta_Y$ the generic point of $Y$.  \end{notn}

Now recall that a ring $R$ is \textit{Noetherian} if it satisfies the ascending
chain condition of ideals.

\begin{defn} A topological space $X$ is called \textit{Noetherian} if any of
    the following conditions hold: \begin{itemize} \item Closed subsets satisfy
        the descending chain condition.  \item Open subsets satisfy the
        ascending chain condition.  \item Every open subset is quasi-compact.
\end{itemize} \end{defn}

\begin{lem} A ring $R$ is Noetherian if and only if $\Spec R$ is Noetherian,
and this implies that all open subsets of $\Spec R$ are quasi-compact.
\end{lem}

\subsection{The Structure Sheaf}% \label{sub:the_structure_sheaf}

Let $X$ be a topological space and $\mc{C}$ be a category. 

\begin{defn} A \textit{presheaf} on $X$ is a functor from the opposite category
of the poset category of open sets to $\mc{C}$.  \end{defn}

\begin{defn} A presheaf $\mc{F}$ on $X$ is called a \textit{sheaf} if for every
    open subset $U \subseteq X$ and for all open coverings $\qty{U_i}$ the
    sequence \[ \mc{F}(U) \to \prod \mc{F}(U_i) \rightrightarrows \prod
    \mc{F}(U_i \cap U_j) \] given by \begin{align*} s \mapsto (s_i) =
\qty(\eval{s}_{U_i}) &\mapsto \eval{s_i}_{U_i \cap U_j} \\ &\mapsto
\eval{s_j}_{U_i \cap U_j} \end{align*} is exact. Of course, if $\mc{C} =
\ms{Ab}$, then the second arrow can be replaced by $(s_i) \mapsto
\eval{s_i}_{U_i \cap U_j} - \eval{s_j}_{U_i \cap U_j}$. What this means is that
given two sections $s, s' \in \mc{F}(U)$ that agree on the restrictions, they
$s = s'$. Also, if there exist $s_i \in \mc{F}(U_i)$ such that $\eval{s_i}_{U_i
\cap U_j} = \eval{s_j}_{U_i \cap U_j}$ then there exists $s \in \mc{F}(U)$ that
globalizes the $s_i$.  \end{defn}

Now suppose $\mc{B} = \qty{U_i}$ is a basis of open sets of $X$. Given
$\mc{F}(U_i)$ for all $U_i \in \mc{B}$, we need to check when this defines a
(pre)sheaf. Here, on an arbitrary open set $V$, we will simply define \[
\mc{F}(V) = \lim_{\substack{\longleftarrow \\ V \supset U \in \mc{B}}}
\mc{F}(U). \] To check when this presheaf is actually a sheaf, then we only
need to check the gluing condition for $U, \qty{U_i} \in \mathcal{B}$.

Now we will define the structure sheaf on $X = \Spec R$. Here, we write
$\mc{O}_X(X_f) = R_f$ and $\mc{O}_X(X_g) \to \mc{O}_X(X_f)$ be given by
choosing $n$ such that $f^n = ag$ and then writing $\frac{b}{f^k} \mapsto
\frac{a^k b}{f^{nk}}$. Now we need to check the two gluing conditions. The
second is left as an exercise, so we will check the first.

If $\frac{b}{f^k} \mapsto 0$ for all $i$, then there exist $m_i$ such that
$f_i^{m_i} \cdot b = 0$ in $R$. But then because $X_f$ is quasi-compact, we can
assume the cover is finite and choose $n = \max m_i$. But then because $X_f =
\bigcup X_{f_i}$, we can write $1 = \sum a_i f_i^n$ and this implies $b = \sum
a_i f_i^n b = 0$.

The sheaf $\mc{O}_X$ that we have defined is called the \textit{structure
sheaf} of $X$.

\begin{defn} The pair $(X, \mc{O}_X)$ is called an \textit{affine scheme}.
\end{defn}

\begin{exm} For a field $k$, the space $\Spec k$ is a point, but $\mc{O}_X$ is
different for different fields.  \end{exm}

\begin{exm} If $X = \Spec D$ for $D$ a $DVR$ with uniformizer $t$, then $X_t =
\qty{0}$, we see that $\mc{O}_X(X_t) = D_t = K(D)$.  \end{exm}

\begin{prop} Let $X = \Spec R$.  \begin{enumerate} \item The stalks of
    $\mc{O}_X$ at $\mf{p} = x \in X$ are given by $\mc{O}_{X,x} = R_{\mf{p}}$.
\item For any $U \subseteq X$ open, we define $\mc{O}_X(U)$ to be the the set
    of $s_{\mf{p}} \in \prod_{\mf{p} \in U} R_{\mf{p}}$ such that whenever $U =
    \bigcup X_{f_i}$, there exist $s_i \in \mc{O}_X(X_{f_i})$ mapping to
    $s_{\mf{p}}$ whenever $\mf{p} \in X_{f_i}$.  \end{enumerate} \end{prop}

\begin{proof} First, the stalk $\mc{F}_x = \lim_{U \ni x} \mc{F}(U)$ and thus
    the stalk of the structure sheaf is easily computed to be the localization.
    For the second part, we note that \[ \mc{O}_X(U) =
    \lim_{\substack{\longleftarrow \\ X_f \subseteq U}} R_f \longrightarrow
\prod_{\mf{p} \in U} F_{\mf{p}}. \] \end{proof}

\begin{rmk} The same method used to construct $\mc{O}_X$ can be used to
    associate a sheaf for every $R$-module $M$. Here, we will define
    $\wt{M}(X_f) = M_f = M \otimes_R R_f$. Here, $\wt{M}$ is a sheaf of
    $\mc{O}_X$-modules. This means that for all $U \subseteq X$, $\wt{M}(U)$ is
    an $\mc{O}_X(U)$-module and the diagram \begin{equation*} \begin{tikzcd}
    \wt{M}(U) \times \mc{O}_X(U) \arrow{r} \arrow{d} & \wt{M}(U) \arrow{d} \\
\wt{M}(V) \times \mc{O}_X(V) \arrow{r} & \wt{M}(V) \end{tikzcd} \end{equation*}
commutes whenever $V \subseteq U$.  \end{rmk}

\begin{prop} $\Hom_R(M,N) \simeq \Hom_{\mc{O}_X}(\wt{M}, \wt{N})$.  \end{prop}

\begin{proof} Let $M \xrightarrow{\varphi} N$ be a map of $R$-modules. Now on
    $X_f$, we have a map $M_f \xrightarrow{\varphi_f} N_f$ by functoriality of
    localization, and then we can take limits to get a map on every open set.

    In the other direction, let $f \colon \wt{M} \to \wt{N}$ be a map of
sheaves. Then we simply apply the global sections functor to obtain a map $M
\to N$. Checking that the two maps defined are inverses is easy and uses
naturality of localiation.  \end{proof}

\section{General Schemes}% \label{sec:general_schemes}

\begin{defn} A \textit{scheme} is a locally ringed space $(X, \mc{O}_X)$ such
that there exists an open cover $\qty{U_i}$ of $X$ such that $\qty(U_i,
\eval{\mc{O}_X}_{U_i})$ is an affine scheme.  \end{defn}

\begin{lem} Let $R$ be a ring and $X = \Spec R$. Then for any $f \in R$, the
schemes $\qty(X_f, \eval{\mc{O}_X}_{X_f}), (\Spec R_f, \mc{O}_{\Spec R_f})$ are
isomorphic.  \end{lem}

\begin{proof} We check that the structure sheaves agree on principal open
subsets.  \end{proof}

\begin{prop} Let $(X, \mc{O}_X)$ be a scheme. Then for any open subset $U
\subseteq X$, the pair $\qty(U, \eval{\mc{O}_X}_{U})$ is also a scheme.
\end{prop}

\begin{proof} We need to show there exists an open affine covering of $U$. It
suffices to check for $X$ an affine scheme, but then $U$ is covered by
principal open subsets.  \end{proof}

\subsection{Morphisms of Schemes}% \label{sub:morphisms_of_schemes}

We will now define morphisms of schemes. Here, this will be a map of
topological spaces that is compatible with the structure sheaves. From this, we
will obtain a locally ringed space. In the category of topological spaces,
smooth manifolds, or complex manifolds, then $f \colon X \to Y$ is a regular
function if and only if the pullback of regular functions is regular. This
tells us that we have a morphism of sheaves $\mc{O}_Y \to f_* \mc{O}_X$. In
other words, we obtain a morphism $\mc{O}_Y(V) \to \mc{O}_X(f^{-1}(V))$ for any
open $V \subseteq Y$.

\begin{defn} A morphism $f \colon (X, \mc{O}_X) \to (Y, \mc{O}_Y)$ of schemes
    is the data of a continuous map $f \colon X \to Y$ and a morphism of
    sheaves $\mc{O}_Y \to f_* \mc{O}_X$ such that for every point $y \in Y$,
    the map $\mc{O}_{Y,y} \to {(f_* \mc{O}_X)}_y \to \mc{O}_{X,x}$ is a
    morphism of local rings. What this means is that the maximal ideal of
    $\mc{O}_{Y,y}$ is sent to the maximal ideal of $\mc{O}_{X,x}$. In
    particular, we obtain an extension $k(y) \hookrightarrow k(x)$.  \end{defn}

\begin{thm} Let $X$ be a scheme and $R$ a ring.  \begin{enumerate} \item The
    assigment $f \colon X \to \Spec R \mapsto \Gamma(f^*) \colon R \to
    \Gamma(X, \mc{O}_X)$ determines a bijection \[ \Hom_{\ms{Sch}}(X, \Spec R)
    = \Hom_{\ms{CRing}}(R, \Gamma(X, \mc{O}_X)). \] \item In particular, when
    $X = \Spec B$ this determines an anti-equivalence between the category of
    affine schemes and the category of commutative rings \[
    \Hom_{\ms{Sch}}(\Spec B, \Spec R) = \Hom_{\ms{CRing}}(R, B). \]
    \end{enumerate} \end{thm}

\begin{proof} First we will show that this assignment is injective. First, we
    will show that $f$ is determined (set-theoretically) by $\Gamma(f^*)$ and
    then we will show that $f_* \colon \mc{O}_Y \to f_* \mc{O}_X$ is determined
    by this.

    First, for $x \in X$, we recall that $I(f(x)) = \qty{h \in R \mid f(h(x)) =
    0} = { (f_x^*) }^{-1} \mf{m}_x$ and this gives us a prime ideal in $R$. To
    find the morphism of sheaves, we will simply consider principal open
    subsets $\Spec R_h$. Here, we have ring maps \begin{equation*}
        \begin{tikzcd} R \arrow{r} \arrow{d} & \Gamma(X, \mc{O}_X) \arrow{d} \\
        R_h \arrow{r} & \Gamma (f^{-1}(Y_h), \mc{O}_X) \end{tikzcd}
        \end{equation*} and thus this is uniquely determined.

    Now given a map $R \to \Gamma(X, \mc{O}_X)$, we want to construct a map of
    schemes. First, we will reduce to the affine case and then prove the
    theorem in the affine case. Cover $X = \bigcup U_{\alpha}$ by affines
    $U_{\alpha} = \Spec A_{\alpha}$. Then given $R \to \Gamma(\mc{O}_X) \to
    \Gamma(U_{\alpha}, \mc{O}_X) = A_{\alpha}$, we will prove the reduction to
    the affine case. For maps $R \xrightarrow{\varphi_{\alpha}}$ we obtain maps
    $\Spec A_{\alpha} \xrightarrow{f_{\alpha}} \Spec R$, so we want to show
    that these glue. It suffices to show that the diagram \begin{equation*}
        \begin{tikzcd} R \arrow{r}{\Gamma(f_{\alpha}^*)}
            \arrow{d}{\varphi_{\beta}} & \Gamma(U_{\alpha}, \mc{O}_X) \arrow{d}
            \\ \Gamma(U_{\alpha}, \mc{O}_X) \arrow{r} & \Gamma(U_{\alpha} \cap
            U_{\beta}, \mc{O}_X) \end{tikzcd} \end{equation*} commutes, which
            is obvious becayse these maps are all induced by $R \to \Gamma(X,
            \mc{O}_X) \to \Gamma(U, \mc{O}_X)$.

    Now given $\varphi \colon A \to B$, we will construct a map $f \colon \Spec
    B \to \Spec A$. This is given by $\mf{p} \mapsto \varphi^{-1}(\mf{p})$.
    This is continuous because \begin{align*} f^{-1}(V(\mf{a})) &= f^{-1}
        \qty{\mf{q} \supseteq \mf{a}} \\ &= \qty{\mf{p} \subseteq B \mid
    \varphi^{-1}(\mf{p}) \supseteq \mf{a}} \\ &= \qty{\mf{p} \supseteq
    \varphi(\mf{a}) \cdot B} \\ &= V(\varphi(\mf{a}) \cdot B).  \end{align*}
    Then we regard $B$ as an $A$-module via $\varphi \colon A \to B$, so
    $\wt{B} = f_* \mc{O}_X$ and we simply choose the map of sheaves to be the
    map of sheaves $\wt{A} = \mc{O}_X \to \wt{B}$ we defined previously.
    \end{proof}

\begin{cor} $\Spec \Z$ is the terminal object in the category of schemes. This
means that every scheme $X$ has a \textbf{unique} morphism $X \to \Spec \Z$.
\end{cor}

\begin{proof} Maps $X \to \Spec \Z$ are determined by maps of rings $\Z \to
\Gamma(X, \mc{O}_X)$, and clearly there is a unique such map of rings.
\end{proof}

\begin{rmk} There is an important variant. Let $S$ be a scheme. Then a scheme
    $X/S$ is a scheme $X$ together with a morphism $X \to S$. If $S = \Spec R$,
    then we can also write $X/R$. Of course, we can define the category of
    schemes over $S$, and the terminal object is $S$.  \end{rmk}

\begin{prop} Let $X$ be a scheme and $x \in X$. Then \begin{enumerate} \item
    There exists a canonical morphism $\Spec \mc{O}_{X,x} \xrightarrow{i_x} X$.
\item Let $X$ be a local domain. Then any morphism $\Spec R \to X$ that sends
    $0 \mapsto x$ factors uniquely via $i_x$.  \end{enumerate} \end{prop}

\begin{proof} Let $\mf{p} = x \in U \subseteq \in X$ and $U = \Spec A$. Then we
    have a map $A \to A_{\mf{p}}$ and clearly in the category of schemes, we
    have a commutative diagram \begin{equation*} \begin{tikzcd} \Spec
        A_{\mf{p}} \arrow{r} \arrow{d}{=} & \Spec A \arrow[hookrightarrow]{d}
        \\ \Spec \mc{O}_{X,x} \arrow{r} & X.  \end{tikzcd} \end{equation*} Of
        course, we should check that this is independent of the choice of open
        affine.

    For the second part, we have a map $\mc{O}_{X} = j_* \mc{O}_{\Spec R}$,
which is a map $\mc{O}_{X,x} \to \mf{O}_{\Spec R, \mf{m}} = R_{\mf{m}} = R$.
The other part of this is an exercise.  \end{proof}

\begin{cor} Let $k(x)$ be the residue field of $x$. Then there exists a map
$\Spec K \to X$ given by $0 \mapsto x$ if and only if $k(x) \hookrightarrow K$.
\end{cor}

\begin{rmk} The set $\Hom_{\ms{Sch}}(\Spec k[\ep]/\ep^2, (X, x))$ is in
bijection with the Zariski tangent space.  \end{rmk}

Now we will consider some examples. First, let $X = \Spec A$ and let $\mf{a}
\subseteq A$. Then $Z = \Spec A/\mf{a} \to \Spec A$ is a homeomorphism onto
$V(\mf{a})$, and the map $A \to A/\mf{a}$ corresponds to $\mc{O}_X \to i_*
\mc{O}_Z$.

Next, we can consider the ideal $\mf{a}^n$. Here, we note that $V(\mf{a}) =
V(\mf{a}^n)$, but the structure sheaves differ and so we can view $\Spec
A/\mf{a}^m \to \Spec A/\mf{a}^{m+1}$ as a closed subscheme.

\subsection{Gluing Schemes}% \label{sub:gluing_schemes}

Suppose we are given the following data: \begin{itemize} \item A set $I$.
\item For $i \in I$, a scheme $U_i$.  \item For all $i,j \in I$ an open subset
$U_{ij} \subseteq U_i$ \end{itemize} with compatibility conditions in the form
of isomorphisms $\varphi_{ij} \colon U_{ij} \to U_{ij}$ with $\varphi_{ii} =
\mr{id}$. We will also have triple compatibility conditions (cocycle
condition).

\begin{prop} Given the data above, there exists a scheme $X$ and morphisms $U_i
    \xrightarrow{\psi_i} X$ that are isomorphisms onto open subsets of $X$ such
    that $\psi_i(U_{ij}) = \psi_i(U_i) \cap \psi_j(U_j) = \psi_j(U_{ji})$ and
    $X = \bigcup \psi_i(U_i)$.  \end{prop}

\begin{exm} Let $R$ be a ring. Then we will denote by $\A^n_R \coloneqq \Spec
    R[t_1, \ldots, t_n]$. Here, we will take $U_1 = \A^1_R \supset U_{12} =
    \A^1_R \setminus \qty{0} = U_{21} \subseteq U_2 = \A_R^1$ and $\varphi_{12}
    = \mr{id}$. The scheme $X = U_1 \cup U_2$ is known as the \textit{affine
    line with double origin}.  \end{exm}

\begin{exm} Let $R$ be a ring. Then consider $U_i \coloneqq \Spec R \qty[
    \frac{x_0}{x_i}, \ldots, \wh{\frac{x_i}{x_i}}, \ldots, \frac{x_n}{x_i}]$
    and $U_{ij} = \qty{\frac{x_j}{x_i} \neq 0} \subseteq U_{ji}$. Then the
    scheme is $X = \P^n_R$.  \end{exm}

\begin{defn} Let $\mc{C}$ be a category and $S \in \mc{C}$ with morphisms $X
\xrightarrow{f} S, Y \xrightarrow{g} S$. Then the \textit{fiber product} $Z = X
\times_S Y$ is the limit of the diagram $X \xrightarrow{f} S \xleftarrow{g} Y$.
\end{defn}

\begin{rmk} If $\mc{C} = \ms{Set}$, then we can write $Z = \qty{(x,y) \in X
\times Y \mid f(x) = g(y)}$.  \end{rmk}

\begin{thm} Fiber products exist in the category of schemes.  \end{thm}

\begin{lem} If $X,Y,S$ are affine, then $X \times_S Y$ exists.  \end{lem}

\begin{proof} Write $X = \Spec A, Y = \Spec B, S = \Spec C$. Then $A \otimes_C
    B$ is the pushout of $A \gets C \to B$, and then we use the fact that
    affine schemes are opposite to commutative rings. Now we need to prove that
    the universal property holds for all schemes. But this simply reduces to
    the case of affine schemes by the next exercise.  \end{proof}

\begin{exer} For all schemes $T$, there exists an affine scheme $\mr{Aff}(T)$
that is universal with respect to morphisms $T \to \Spec A$.  \end{exer}


\begin{proof}[Proof of Theorem] First, we note that if $U \subseteq X$, then if
    $X \times_S Y$ exists, then for the map $p \colon X \times_Y S$, the
    preimage $p^{-1}(U)$ is the fiber product $U \times_S Y$. On the other
    hand, if $X = \bigcup U_i$ and $U_i \times_S Y$ exist for all $i$, then $X
    \times_S Y$ exists. To see this, we simply use gluing. 

    Next, suppose $S = \Spec C, Y = \Spec B$ are affine. Then if $X = \bigcup
    U_i$ is a cover by open affines then $U_i \times_S Y$ exist for all $i$, so
    $X \times_S Y$ exists. Third, we cover $Y = \bigcup V_i$ by open affines
    and then we now have $X \times_S Y$ for general $X,Y$.

    Finally, we cover $S = \bigcup W_i$ by open affines. Then if we consider
$X_i = f^{-1}(W_i), Y_i = g^{-1}(W_i)$, the fiber products $X_i \times_{W_i}
Y_i = X_i \times_S Y_i$ exist, so by gluing twice, we obtain the fiber product
$X \times_S Y$.  \end{proof}

\begin{rmk} $X \times_S Y$ has an affine open cover by open subsets of the form
$\Spec A \otimes_C B$.  \end{rmk}

\begin{exm} We have an identification $\A_R^n = \A_{\Z}^n \times_{\Spec Z}
\Spec R$. Similarly, we have $\A_R^{n+m} = \A_R^n \times_{\Spec R} \A_R^m$.
\end{exm}

\begin{defn} Let $X, S'$ be schemes over $S$. Then the fiber product $X
\times_S S' \to S'$ is called the \textit{base change} of $X/S$ to $S'$.
\end{defn}

\begin{exm} Suppose $k \subset K$ is a field extension and $X/k$ is a
$k$-scheme. Then $X_K = X \times_{\Spec k} \Spec K$ is a $K$-scheme.  \end{exm}

Fiber products allow us to consider the notion of the preimage of a closed
subset. For $s \in S$ and morphism $X \to S$, then the fiber product $X
\times_S \Spec k(s) \to \Spec k(s)$ is the fiber of $X \to S$ over $s$.

\begin{exm} Consider a closed subscheme $\Spec A/\mf{a} = Z \hookrightarrow S =
    \Spec A$. Then we may consider $f^{-1}(Z) = X \times_S Z$ for some $X \to
    S$. We may also consider the intersection of two closed subschemes $Z =
    \Spec A/\mf{a}, W = \Spec A/\mf{a'}$, which is simply the fiber product $Z
    \times_S W = \Spec A/(\mf{a} + \mf{a}')$.  \end{exm}

\begin{exm} Let $k = \ol{k}$ and $\operatorname{char} k \neq 2$ and consider
    the morphism $\Spec K[x,y,t] / (x^2-yt) = X \to S = \Spec k[t]$. So now for
    $s = (t-a) \in \Spec k[t]$, we see that $X_s = \Spec k[x,y] / (x^2=ay)$,
    and in particular $X_0 = \Spec k[x,y] / x^2$ is non-reduced. 

    On the other hand, if we consider $X \to \Spec k[x]$, we see that $X_0$ is
the union of two copies of $\A^1$ intersecting at a point.  \end{exm}

\section{Quasicoherent Sheaves and Relative Spec}% \label{sec:relative_spec}

We will relativize the construction of $\Spec R$ from a ring $R$. To do this,
we will replace $R$ with a sheaf of $\mc{O}_X$-algebras. Recall that if $X =
\Spec R$ and $M$ is an $R$-module, then the $\mc{B}$-sheaf $X_f \mapsto M_f$
defines a sheaf $\wt{M}$ on $X$. Then we know that for two $R$-modules $M,N$,
\[ \Hom_{\ms{R-mod}}(M,N) = \Hom_{\mc{O}_X}(\wt{M}, \wt{N}). \] This gives us a
fully faithful exact functor $\wt{(\cdot)}$ from $R$-modules to
$\mc{O}_X$-modules.

\begin{thm} The functor $M \mapsto \wt{M}$ commutes with kernels and cokernels.
In particular, it is exact.  \end{thm}

\begin{proof} Recall that localization is exact. This implies that if $K$ is
    the kernel of $M \to N$, then $\wt{K}$ is the kernel of $\wt{M} \to
    \wt{N}$. Next, for the cokernel of $M \to N$, we note that $\wt{C}$ and
    $\mr{coker}(\wt{M} \to \wt{N})$ are both sheaves extending the same
    presheaf.  \end{proof}

\begin{defn} A $R$-module $M$ is called \textit{finitely presented} if there is
an exact sequence $R^p \to R^q \to M \to 0$ for $p,q \geq 0$.  \end{defn}

\begin{prop}\leavevmode \begin{enumerate} \item If $M$ is finitely presented,
then $\mc{H}om_{\mc{O}_X}(\wt{M}, \wt{N}) = \wt{\Hom_R(M, N)}$.  \item The
functor $\wt{(\cdot )}$ commutes with arbitrary direct sums.  \end{enumerate}
\end{prop}

\begin{defn} Let $X$ be a scheme. Then a sheaf $\mc{F}$ of $\mc{O}_X$-modules
    is called \textit{quasi-coherent} if for all $x \in X$ there exists $U
    \subseteq X$ and an exact sequence \[ \eval{\mc{O}_X^J}_U \to
    \eval{\mc{O}_X^I}_U \to \eval{\mc{F}}_U \to 0. \] \end{defn}

\begin{prop} Let $X$ be a scheme and $\mc{F}$ and $\mc{O}_X$-module. Then the
    following are equivalent: \begin{enumerate} \item $\mc{F}$ is
        quasicoherent; \item For all affine open $U \subseteq X$,
        $\eval{\mc{F}}_U = \wt{M}$ for some $\mc{O}_X(U)$-module $M$ \item
there exists an affine open cover $\qty{U_{\alpha}}$ such that
$\eval{\mc{F}}_{U_{\alpha}} = \wt{M}_{\alpha}$ for some
$\mc{O}_X(U_{\alpha})$-module $M_{\alpha}$.  \end{enumerate} \end{prop}

\begin{proof} Clearly we see that \textbf{2} implies \textbf{3}, so we show
    that \textbf{3} implies \textbf{2}. Let $U \subseteq X$ be an open affine.
    Now we apply the exercise below to get a covering $\qty{U_i}$ such that
    $U_i \subseteq U$ and $U_i \subseteq U_{\alpha}$ is principal in both $U,
    U_{\alpha}$. Therefore $\eval{\mc{F}}_{U_i} = \wt{N}_i$ for some
    $\mc{O}_X(U_i)$-module $N_i$. Now if we write $U = \Spec R$ and $U_i =
    \Spec R_i$, we see that if $j_i \colon U_i \hookrightarrow U$, then $\eval{
    {(j_i)}_* \mc{F} }_{U_i} = \wt{N}_i$.

    Next, the sequence \[ \mc{F} \to \prod {(j_i)}_* \qty(\eval{\mc{F}}_{U_i})
        \to \prod {(j_i)}_* \qty(\eval{\mc{F}}_{U_i \cap U_j}) \] is exact by
        the sheaf axioms, so we are done because this is really an exact
        sequence \[ \mc{F} \to \prod \wt{N}_i \to \prod \wt{N}_{ij}. \] The
    implications \textbf{3} implies \textbf{1} and \textbf{1} implies
\textbf{2} are trivial.  \end{proof}

\begin{exer} Let $X$ be a scheme, $x \in X$, and $x \in U,V$ open subsets. Then
there exists an open $x \in W \subseteq U \cap V$ such that $W$ is principal in
both $U$ and $V$.  \end{exer}

\begin{exm} We will consider quasicoherent sheaves on $\Spec R$ for $R$ a
    discrete valuation ring. Then a sheaf on $X = \Spec R$ is a map $\mc{F}(X)
    \xrightarrow{\mr{res}} \mc{F}(X \setminus \qty{0})$. Now recall that
    $\mc{F}$ is quasicoherent if and only if it comes from an $R$-module $M$,
    so we see that $\mc{F}(X) = M$ and $\mc{F}(X \setminus \qty{0})$ is $M
    \otimes K$.  \end{exm}

\begin{rmk} Let $f \colon X \to Y$ be a morphism of schemes. Then $f_*
\mc{O}_X$ is a $\mc{O}_Y$-algebra.  \end{rmk}

\begin{exer} If $X$ is Noetherian or $f$ is quasicompact and $\mc{F}$ is
quasicoherent, then $f_* \mc{F}$ is quasicoherent.  \end{exer}

\begin{exm} If $f \colon \Spec A \to \Spec B$ is a morphism of affine schemes,
then $f_* \wt{M} = \wt{M}_B$ and is thus quasicoherent.  \end{exm}

\begin{thm} Let $Y$ be a scheme and $\mc{R}$ be a quasicoherent sheaf of
    $\mc{O}_Y$-algebras. Then there exists a scheme $X = \Spec_{\mc{O}_Y}
    \mc{R} \xrightarrow{\pi} Y$ such that $\pi_* \mc{O}_X = \mc{R}$ and for any
    $f \colon Z \to Y$ and morphism $\alpha \colon \mc{R} \to f_* \mc{O}_Z$,
    there exists a unique $g \colon Z \to X$ such that $\mc{R} = \pi_* \mc{O}_X
    \xrightarrow{\alpha} \pi_* g_* \mc{O}_Z = f_* \mc{O}_Z$.  \end{thm}

\begin{proof} If $Y = \Spec A$, then write $\mc{R} = \wt{R}$ and set $X = \Spec
    R$, and this has a natural morphism to $\Spec A$.

    In general, cover $Y = \bigcup U_{\alpha}$ by open affines. Then write
$\eval{\mc{R}}_{U_{\alpha}} = \wt{R}_{\alpha}$ for some
$\mc{O}_Y(U_{\alpha})$-module. Then set $X_{\alpha} = \Spec R_{\alpha}$. To
construct a transition map between $X_{\alpha}, X_{\beta}$, we simply consider
the restriction $R_{\beta} = \mc{R}(U_{\beta}) \to \Gamma(U_{\alpha} \cap
U_{\beta}, \mc{R})$, and this gives a morphism $\pi_{\alpha}^{-1}(U_{\alpha}
\cap U_{\beta}) \to X_{\beta}$ and this factors through the
$\pi_{\beta}^{-1}(U_{\alpha} \cap U_{\beta})$ because the latter is the fiber
product of $U_{\alpha} \cap U_{\beta}$ and $X_{\beta}$ over $Y$. The rest is
obvious.  \end{proof}

\begin{defn} A morphism $f \colon X \to Y$ of schemes is called \textit{affine}
if for every $U \subseteq Y$ open affine, the preimage $f^{-1}(U) \subseteq X$
is affine.  \end{defn}

\begin{exm} The morphism $\Spec_{\mc{O}_Y} \mc{R} \to Y$ is affine.  \end{exm}

\begin{prop} The following are equivalent: \begin{enumerate} \item $f \colon X
    \to Y$ is affine.  \item There exists an open covering $Y = \bigcup
    U_{\alpha}$ such that $f^{-1}(U_{\alpha})$ is affine.  \item $f \colon X
    \to Y$ can be written as $\Spec_{\mc{O}_Y} \mc{R} \to Y$.  \end{enumerate}
\end{prop}

\begin{proof} The first implication is by definition and \textbf{3} implies
    \textbf{1} by the construction, so assume there exists an open covering $Y
    = \bigcup U_{\alpha}$ by affines such that $f^{-1}(U_{\alpha})$ is affine.
    Set $\mc{R} = f_* \mc{O}_X$. By assumption, this is quasicoherent.
    Therefore there exists a morphism $g$ that makes \begin{equation*}
        \begin{tikzcd} X \arrow{rr}{g} \arrow{dr}{f} & & \Spec_{\mc{O}_Y}
        \mc{R} \arrow{dl} \\ & Y \end{tikzcd} \end{equation*} commute. But
        there exists a covering (the preimages of the $U_{\alpha}$) where $g$
        is an isomorphism, so $g$ is an isomorphism.  \end{proof}

\begin{thm} A scheme $X$ is \textit{locally noetherian} if for all $x \in X$,
there exists an affine neithborhood $\Spec R \ni x$ such that $R$ is a
Noetherian ring.  \end{thm}

If $X$ is locally noetherian and quasicompact, then it is Noetherian.

\begin{prop} $X$ is locally Noetherian if and only if for any affine open
    $\Spec R \subseteq X$, $R$ is Noetherian. Let $X = \bigcup \Spec
    R_{\alpha}$ where each $R_{\alpha}$ is Noetherian. Then if $\Spec R
    \subseteq X$ is any affine open, we want to show that $R$ is Noetherian.
    But here, we can choose $V \subseteq \Spec R \cap \Spec R_{\alpha}$, which
    is a principal open subset. Then $V = \Spec { (R_{\alpha}) }_{g_{\alpha}}$
    is Noetherian, so we can cover $\Spec R = \bigcup \Spec R_{f_{\alpha}}$ by
    Noetherian schemes. But then affines are quasicompact, so this becomes a
    finite cover and thus $\Spec R$ is Noetherian.  \end{prop}

\begin{prop}[Affine Communication Lemma] Let $\mc{P}$ be a property enjoyed by
    affine schemes. Suppose that \begin{enumerate} \item If $A$ has $\mc{P}$,
        then $A_f$ also has $\mc{P}$ for all $f \in A$.  \item If $f_i \in A$
        such that $(f_1, \ldots, f_n) = A$, then if $A_{f_i}$ have $\mc{P}$, so
        does $A$.  \end{enumerate} Then for any scheme $X$, if $\mc{P}$ holds
    for one affine open cover, it holds for all affine open covers.  \end{prop}

\begin{proof} Let $X = \bigcup \Spec A_i$ where the $A_i$ have $\mc{P}$. Then
there exists $V \subseteq \Spec A \cap \Spec A_i$ such that $V$ is a principal
open subset in both.  \end{proof}

\begin{defn} A morphism $f \colon X \to Y$ of schemes is called \textit{locally
    of finite type} if there exist an open affine cover $X = \bigcup
    U_{\alpha}$ and open subsets $V_{\alpha} \subseteq Y$ such that
    $f(U_{\alpha} = \Spec A_{\alpha}) \subseteq V_{\alpha} = \Spec B_{\alpha}$
    and $A_{\alpha}$ is a finitely generated $B_{\alpha}$-algebra.  \end{defn}

\begin{prop} A morphism $f \colon X \to Y$ is locally of finite type if and
    only if for every pair of affine open sets $U \subseteq X, V \subseteq Y$
    such that $f(U) \subseteq V$, $\mc{O}_X(U)$ is a finitely generated
    $\mc{O}_Y(V)$-algebra.  \end{prop}

\begin{proof} If $A$ is a finitely generated $B$-algebra, then for all $f \in
    A$, $A_f = A[1/f]$ is also a finitely-generated $B$-algebra. Next, if
    $A_{f_i}$ are finitely generated $B$ algebras and $(f_1, \ldots, f_n = A)$,
    we will show that $A$ is a finitely-generated $B$-algebra. Suppose the
    $A_{f_i}$ are generated by $\frac{a_{ij}}{f_i^{k_j}}$ and $\sum c_i f_i =
    1$. We will show that the $f_i, c_i, a_{ij}$ generate $A$ as a $B$-algebra.

    Let $r \in A$. Then in $A_{r_i}$, we see that $r =
    \frac{p_i(a_{ij})}{f_i^N}$, so by finiteness, we can assume that there
    exists $M \geq 0$ such that $f_i^{N+M} r = f_i^M p_i (a_{ij})$ for all
    $i,j$. Now we can write \[ 1 = \sum c_i f_i = { \qty(\sum c_i f_i)
    }^{(N+M)} = \sum Q_i(c_i, f_i) f_i^{N+M} \] and therefore \[ r = \sum
Q_i(c_i, c_i) f_i^{N+M} r = \sum Q_i(c_i, f_i) f_i^M p_i(a_{ij}), \] as
desired.  \end{proof}

\begin{defn} Let $X$ be a scheme and $\mc{F}$  sheaf of $\mc{O}_X$-modules is
    called \begin{enumerate} \item \textit{Locally of finite type} if for all
        $x \in X$, there exists $U \ni x$ and a surjection $\eval{\mc{O}^n_X}_U
        \to \eval{\mc{F}}_U \to 0$.  \item \textit{Locally of finite
        presentation} if for all $x \in X$, there exists $U \ni x$ open and an
        exact sequence \[ \eval{\mc{O}^m_X}_U \to \eval{\mc{O}_X^n}_U \to
        \eval{\mc{F}}_U \to 0. \] \item \textit{Locally free} if for all $x \in
        X$, there exists $U \ni x$ and an isomorphism $\eval{\mc{O}_X^n}_U
        \simeq \eval{\mc{F}}_U$.  \end{enumerate} \end{defn}

\begin{rmk} If $U = \Spec A$ and $\mc{F} = \wt{M}$, then $\mc{F}$ is locally of
    finite type if and only if $M$ is a finitely-generated $A$-module, locally
    of finite presentation if and only if $M$ is finitely presented, and
    locally free if and only if $M \simeq A^n$.  \end{rmk}

\begin{defn} Let $X$ be a scheme. An $\mc{O}_X$-module $\mc{F}$ is called
    \textit{coherent} if $\mc{F}$ is locally of finite type and for all $U$
    open and morphisms $\eval{\mc{O}_X^n}_U \xrightarrow{\alpha}
    \eval{\mc{F}}_U$, $\ker \alpha$ is of finite type.  \end{defn}

\begin{exm} Let $R = \prod R_n$, where $R_n = k[x_0, \ldots, x_n] / (x_0^2, x_0
x_1, \ldots, x_0 x_n)$ and $X = \Spec R$. Then $\mc{O}_X$ is not coherent.
Indeed, the map $\mc{O}_X \xrightarrow{x_0} \mc{O}_X$ is not of finite type.
\end{exm}

\begin{prop} A scheme $X$ is locally Noetherian if and only if $\mc{O}_X$ is
coherent.  \end{prop}

\begin{prop} Let $X$ be locally Noetherian. The following are equivalent for a
    sheaf $\mc{F}$ of $\mc{O}_X$-modules: \begin{enumerate} \item $\mc{F}$ is
        coherent.  \item $\mc{F}$ is locally of finite presentation.  \item
        $\mc{F}$ is quasicoherent and of finite type.  \end{enumerate}
    \end{prop}

\begin{proof} Suppose $\mc{F}$ is quasicoherent and finite type. Then let $U =
\Spec A$ be an open affine. Then $A$ is Noetherian, so $\mc{O}_X^n \to \mc{F}$
corresponds to $A^n \to M$ on $U$, and this has finitely-generated kernel.
\end{proof}

\begin{prop} Let $X$ be locally Noetherian. Then kernels and cokernels of maps
between coherent sheaves are coherent. This means that $\mr{Coh}(X)$ is an
abelian category.  \end{prop}

\begin{defn} Let $\mc{F}$ be a quasicoherent sheaf on a scheme $X$. For every
    point $x \in X$, the \textit{fiber} of $\mc{F}$ at $x$ is the $k(x)$-vector
    space $\mc{F}_x \otimes_{\mc{O}_{X,x}} k(x) \eqqcolon \mc{F}(x)$. The rank
    of $\mc{F}$ at $x$ is $\dim_{k(x)} \mc{F}(x) \eqqcolon r(x)$.  \end{defn}

\begin{exm} Let $p \in X$ be a closed point. Then suppose $i \colon \Spec
(k(p)) \to X$ and let $\mc{F}  = i_* k(p)$. Then $\mc{F}(x) = 0$ if and only if
$x \neq p$ and $\mc{F}(p) = k(p)$.  \end{exm}

\begin{lem}[Nakayama] Let $X$ be a scheme and $\mc{F}$ be a quasicoherent sheaf
locally of finite type. If $\mc{F}(x) = 0$ at $x \in X$, then there exists $U
\ni x$ open such that $\eval{\mc{F}}_U = 0$.  \end{lem}

\begin{proof} Let $\Spec A = V \ni x$ be an affine open neighborhood. Then
    $\eval{\mc{F}}_V = \wt{M}$. But this means that $\mc{F}_x
    \otimes_{\mc{O}_{X,x}} k(x) = 0$, so $\mf{m}_x \mc{F}_x = 0$, and thus
    $\mc{F}_x = 0$.

    Now if $m_1, \ldots, m_k$ are generators of $M$ as an $A$-module, then we
see that $m_i \in \mc{F}(V) \to \mc{F}_x$. By finiteness, up to restricting
$v$, we can assume that $m_i \in \mc{F}(V)$, and therefore $m_i = 0$, so
$\eval{\mc{F}}_V = 0$.  \end{proof}

\begin{cor} This tells us that $\Supp (\mc{F}) \subseteq X$ is closed.
\end{cor}

\begin{cor} Let $X$ be a scheme and $\mc{F}$ quasicoherent and locally of
    finite type. Now choose $x \in X$ and let $s_1, \ldots, s_k$ generate
    $\mc{F}(x)$ as a $k(x)$-vector space. Then there exists and open $U \ni x$
    and $\wt{s_i} \in \mc{F}(U)$ lifting the $s_i$ that generate
    $\eval{\mc{F}}_U$.  \end{cor}

\begin{proof} Clearly, we can lift the sections, so we consider the cokernel of
    \[ \eval{\mc{O}_X^n}_U \xrightarrow{\alpha} \mc{F}_U \to \mr{coker}(\alpha)
    \eqqcolon \mc{G} \to 0. \] We show that $\mc{G}(x) = 0$. If we localize at
    $x$, then we obtain an exact sequence \[ \mc{O}_{X,x}^n \to \mc{F}_x \to
    \mc{G}_x \to 0, \] and then by right-exactness of the tensor product, we
    have ${k(x)}^n \to \mc{F}(x) \to \mc{G}(x) \to 0$. But now the map
    ${k(x)}^n \to \mc{F}(x)$ was surjective, so $\mc{G}(x) = 0$.  \end{proof}

\begin{prop}[Upper Semicontinuity] Let $x \in X$ and $\mc{F}$ be quasicoherent
    and locally of finite type. Then \begin{enumerate} \item The function
        $\mr{rk} \colon X \to \Z$ sending $x \mapsto \mr{rk}(\mc{F}(x))$ is
        upper semicontinuous.  \item If $X$ is connected, reduced, and locally
noetherian, then $\mr{rk}(x) \equiv r$ if and only if $\mc{F}$ is locally free
of rank $r$.  \end{enumerate} \end{prop}

\begin{proof}\leavevmode \begin{enumerate} \item Let $p \in X$ and $\mr{rk}(p)
    \eqqcolon r$. Then there exists $U \ni p$ with a surjection
    $\eval{\mc{O}_X^r}_U \twoheadrightarrow \eval{\mc{F}}_U$, so by exactness
    of localization and right-exactness of tensor product, we obtain a
    surjection ${k(x)}^r \twoheadrightarrow \mc{F}(x)$ for $x \in U$. This
    tells us that $\mr{rk}(\mc{F}(x)) \leq r$ for all $x \in U$.  \item Assume
    that $x \mapsto \mr{rk}(\mc{F}(x)) \equiv r$. Then for $x \in X$, we can
    choose $\Spec A = U \ni x$, where $A$ is Noetherian. Then the exact
    sequence \[ 0 \to \mc{G} \to \eval{\mc{O}_X^r}_U \twoheadrightarrow
    \eval{\mc{F}}_U \to 0 \] corresponds to \[ 0 \to N \to A^r \to M \to 0. \]
    Now choose $\mf{p} \in \Spec A$ such that $A_{\mf{p}}$ is a field and for
    some $(a_1, \ldots, a_r) \in N$, at least one $a_i \notin \mf{p}$. Because
    $A$ is Noetherian and $X$ is reduced, there exist finitely many minimal
    primes, and now the sequence \[ 0 \to N_{\mf{p}} \to A_{\mf{p}}^r \to
M_{\mf{p}} \to 0 \] is exact and because $\mr{rk}(\mc{F}(p)) = r$, we see that
$N_{\mf{p}} = 0$. \qedhere \end{enumerate} \end{proof}

\begin{rmk} Passing to fibers does not preserve injections. For example, if we
consider a field $k$, then the map $0 \to \mc{O}_{\A^1} \xrightarrow{t}
\mc{O}_{\A^1} \to k(0) \to 0$ is exact.  \end{rmk}

\begin{exm} Let $X = \Spec k[t] / t^2$, we can produce nontrivial sheaves with
trivial fibers.  \end{exm}

Now let $X$ be a locally Noetherian scheme. Then if $\mc{F}, \mc{G}$ are
coherent, then $\mc{F} \otimes_{\mc{O}_X} \mc{G}$ and $\mc{H}om(\mc{F},
\mc{G})$ is coherent. In addition, any operation from multilinear algebra, in
particular symmetric and exterior powers, can be performed on coherent sheaves.

\begin{defn} Let $X$ be a scheme and $\mc{F}$ be a quasicoherent sheaf of
finite type. Then $\mc{F}$ is called \textit{invertible} if it is locally free
of rank $1$.  \end{defn}

\begin{exm} Let $k$ be a field and consider $\A^n$. Then for $f \in k[t_1,
\ldots, t_n]$, the sheaf $\wt{(f)}$ is invertible.  \end{exm}

The reason these are called invertible is because if $\mc{F}$ is invertible,
then there exists $\mc{F}'$ such that $\mc{F} \otimes_{\mc{O}_X} \mc{F}' \simeq
\mc{O}_X$.

\begin{defn} We will denote the \textbf{set} (although we will see this is a
group scheme with operation the tensor product) of isomorphism classes of
invertible sheaves on $X$ by $\Pic X$.  \end{defn}

\section{Functor of Points}% \label{sec:functor_of_points}

We will begin a discussion of something that will eventually allow us to define
moduli problems.

\begin{exm} Let $R$ be a ring and $I = (f_0, \ldots, f_n)$ be an ideal. We may
    consider the closed subscheme $X \coloneqq \Spec R[t_1, \ldots, t_n]/I
    \hookrightarrow \A^n_R$. Then we know that \[ \Hom(\Spec A, X) =
    \Hom(R[t_1, \ldots, t_n]/I, A) = \qty{(a_1, \ldots, a_n) \in A^n \mid
f_j(a_1, \ldots, a_n) = 0} \] for any $R$-algebra $A$.  \end{exm}

This generalizes to general schemes the idea that for $A = k = \ol{k}$, then
the closed points of $X$ are the same as morphisms $\Spec k \to X$. Can we
recover a scheme $X$ from the functor $\Hom(-,X)$?

Let $\mc{C} = \ms{Sch}_{/S}$ for a fixed scheme $X$. Then for any $X \in
\mc{C}$, consider the functor $h_X \colon \mc{C}^{\mr{op}} \to \ms{Set}$
defined by $h_X(-) = \Hom_S(-,X)$.

\begin{rmk} We can perform this construction for any category $\mc{C}$. For
example, we can recover a group $G$ as a set from $\Hom(\Z, G)$. Similarly, a
smooth manifold can be recovered (as a set) from $\Hom(\pt, M)$.  \end{rmk}

\begin{exm} Let $X = \A^n_{\Z}$. Then \[ \Hom(T, \A^n_{\Z}) = \Hom(\Z[t_1,
\ldots, t_n], \Gamma(T, \mc{O}_T)) = {\Gamma(T, \mc{O}_T)}^n. \] \end{exm}

\begin{exm} Let $X = \Spec R[t,t^{-1}]$. Then we see that \[ X(T) =
\Hom(R[t,t^{-1}], \Gamma(T, \mc{O}_T)) = {\Gamma(T, \mc{O}_T)}^{\times}. \]
Observe that for any $T$, $X(T)$ has the structure of a group, and this
procedure will define a \textit{group scheme}.  \end{exm}

\begin{exm} Fix a field $k$, let $X/k$, and let $K/k$ be a field extension.
    Then \[ X_k(K) = \qty{x \in X \mid k(x) \hookrightarrow K}. \] When $K =
k$, then $X(k) = \qty{x \in X \mid k = k(x)}$.  \end{exm}

\begin{exm} Let $X \xrightarrow{f} S$ be a scheme. Then $X_S(S)$ is the set of
    sections of $f$. For example, if $\mc{A} \to B$ is a family of abelian
    varieties over an integral scheme, then $MW(\pi) = \mc{A}_B(K(B))$.
    \textbf{Note taker:} The most elementary example of this is an elliptic
    surface, for example a K3 surface or a rational elliptic surface.
\end{exm}

Now we want to relate the functor of points to fiber products. By the universal
property of the fiber product, we see that \[ X_S(T) \times Y_S(T) = {(X
\times_S Y)}_S(T).\] Now observe that the assignment $X \mapsto h_X =
\Hom(-,X)$ is functorial in $X$! To see this, note that $\Hom(-,-)$ is
functorial in both arguments. This gives us a functor \[ h \colon \mc{C} \to
\Hom(\mc{C}^{\op}, \ms{Set}) \qquad X \mapsto h_X. \] After all of this
discussion, we have the following question.

\begin{quest} How much information is lost about $X$ after passing to $h_X$?
    \end{quest} In fact, we lose no information because the functor $X \mapsto
    \Hom(-,X)$ is fully faithful.  \begin{lem}[Yoneda] Let $X,Y \in C$. Then
        $\Hom(X,Y) = \Hom(h_X, h_Y)$. In fact, for any functor $F \colon
        \mc{C}^{\op} \to \ms{Set}$, we have $\Hom(h_X, F) \simeq F(X)$.
    \end{lem}

\begin{proof} Let $Y \in \mc{C}$. Then consider the system of natural
    transformations $\eta_Y \colon h_X(Y) \to F(Y)$. In particular, if $Y = X$,
    we have $\eta_X \colon h_X(X) \to F(X)$, and in particular the element
    $\eta_X(\mr{id}_X) \in F(X)$.

    Now given $\xi \in F(X)$. For every $Y \in \mc{C}$, we need to define
$\eta_Y \colon h_X(Y) \to F(Y)$. Given $f \in \Hom(Y,X)$, we have $F(f) \colon
F(X) \to F(Y)$, so we make the assignment $f \mapsto F(f)(\xi)$.  \end{proof}

\begin{cor} To give a morphism of schemes $X \to Y$ is the same as giving a
natural transformation $h_X \to h_Y$, which is the same as giving compatible
maps $X_S(T) \to Y_X(T)$ for all $T/S$.  \end{cor}

\begin{rmk} In fact, it is enough to consider a scheme as a functor on affine
schemes.  \end{rmk}

Now another natural question is the following: \begin{quest} Which functors in
    $\Hom(\mc{C}^{\op}, \ms{Set})$ are of the form $h_X$ for some $X \in
    \mc{C}$?  \end{quest} Functors of this form are called
    \textit{representable}.  \begin{prop} A functor $F$ is representable if and
        only if there exists $X \in \mc{C}$ and $u \in F(X)$ such that the map
        \[ \Hom(Z,X) \to F(Z) \qquad f \mapsto F(f)(u) \] is a bijection.
    \end{prop} If $F$ is representable, then $X,u$ are unique up to unique
    isomorphism.  \begin{proof} Consider $Z' \xrightarrow{g} Z$ and suppose $f
        \in F(f)$. Then we assign $(f \circ g) \mapsto F(f \circ g)(u)$, and
        this will make the diagram \begin{equation*} \begin{tikzcd} h_X(Z)
            \arrow{r} \arrow{d} & F(Z) \arrow{d} \\ h_X(Z') \arrow{r} & F(Z')
        \end{tikzcd} \end{equation*} commute.  \end{proof}

\begin{exm} Let $X,Y \in \ms{Sch}_{/S}$. Then consider $F \colon Z \to
\Hom_S(Z,X) \times \Hom_S(Z,Y)$. This is represented by the fiber product $X
\times_S Y$ with the two projections $X \times_S Y \rightrightarrows X,Y$.
\end{exm}

We will now give examples of presheaves on the category of schemes over a fixed
$S$.

\begin{exm} Consider $T \mapsto { \Gamma(T, \mc{O}_T) }^{\times}$. This is
represented by $\Spec_{ \mc{O}_S } \mc{O}_S[t, t^{-1}] \eqqcolon \mathbb{G}_{m,
S}$.  \end{exm}

\begin{exm} The functor $T \mapsto {(\Gamma(T, \mc{O}_T))}^n$ is represented by
$\A^n_S = \Spec_{\mc{O}_S} \mc{O}_S[t_1, \ldots, t_n]$.  \end{exm}

\begin{exm} The functor $T \mapsto GL_n(\Gamma(T, \mc{O}_T))$ is represented by
$\Spec_{\mc{O}_S} \mc{O}_S[t_{ij}, \det^{-1}]$.  \end{exm}

\begin{exm}[Projective Space] Fix a positive integer $n$. Define the functor $F
    \colon \mc{C}^{\mr{op}} \to \ms{Set}$ by \[ Z/S \mapsto \qty{\text{exact
    sequences}\ \mc{O}_Z^{n+1} \twoheadrightarrow \mc{L} \to 0 \mid \mc{L}\
    \text{invertible}}. \]

    To check that this is a functor, consider $Z' \xrightarrow{f} Z$. Then
    pullback defines a map $F(Z) \to F(Z')$ (by right-exactness). In fact, $F$
    is represented by $\P^n_S$. The universal object is the line bundle
    $\mc{O}_{\P^n}(1)$. We will $\mc{O}(1)$ as follows:

    Let $\P^n = \bigcup U_{\alpha}$ where $U_{\alpha} = \Spec R[x_0/x_{\alpha},
    \ldots, \wh{x_{\alpha}/x_{\alpha}}, \ldots, x_n/x_{\alpha}]$. We will set
    \[ \eval{\mc{O}(1)}_{U_{\alpha}} = \wt{\frac{1}{x_{\alpha}}
    F[x_0/x_{\alpha}, \ldots, x_n/x_{\alpha}]} = \frac{1}{x_{\alpha}}
\mc{O}_{U_{\alpha}}. \] Then we see that multiplication by
$x_{\beta}/x_{\alpha}$ carries $\eval{\mc{O}(1)}_{U_{\alpha}}$ to
$\eval{\mc{O}(1)}_{U_{\beta}}$. Now we will study the global sections. For any
homogeneous linear polynomial $L(x_0, \ldots, x_n)$. Then on each open set we
obtain a map of multiplication by $L(x_i/x_{\alpha})$. Gluing is obvious.

    Conversely, suppose $\mc{O}_{U_{\alpha}} \xrightarrow{s_{\alpha}}
    \mc{O}_{U_{\alpha}}(1)$ are morphisms that glue. Then the $s_{\alpha}$ are
    rational functions, and we can show that they must come from a polynomial
    of degree $1$.

    Now choose a basis $x_0, \ldots, x_n$ of $\Gamma(\P^n, \mc{O}_{\P^n}(1))$.
    Then these define a map $\mc{O}_{\P^n}^{n+1} \to \mc{O}(1)$. Then for any
    morphism $Z \to \P^n$, we may consider $\mc{O}_Z^{n+1} = f^*
    \mc{O}_{\P^n}^{n+1} \twoheadrightarrow f^* \mc{O}(1)$. Now given any
    $\mc{O}_Z^{n+1} \xrightarrow{\alpha} \mc{L}$, we view $\alpha = (s_0,
    \ldots, s_n)$. Then surjectivity implies that $Z = \bigcup Z_i$ for $Z_i =
    \qty{s(x) \neq 0}$. On each $Z_i$, we see that $\mc{O}_{Z_i}
    \xrightarrow{s_i} \mc{L}$ is surjective and is in fact an isomorphism. Now
    we will define \[ Z_i \to U_i \qquad \frac{x_j}{x_{i}} \mapsto
    \frac{s_j}{s_i}. \] By construction, these maps glue, and so we obtain a
morphism $f \colon Z \to \P^n$. We can check that $f^* [\mc{O}_{\P^n}^{n+1}
\twoheadrightarrow \mc{O}(1)] = [\mc{O}_Z^{n+1} \twoheadrightarrow \mc{L}]$.
\end{exm}

\begin{exm} If we precompose $\mc{O}_{Z}^{n+1} \xrightarrow{g} \mc{O}_Z^{n+1}
\to \mc{L}$ for some $g \in GL_n$, we transform the map $Z \to \P^n$ by a
projective transformation.  \end{exm}

\begin{exm}[Grassmannian] We can generalize $\P^n$ to the functor \[ F(Z) =
\qty{\mc{O}_Z^{n+1} \twoheadrightarrow \mc{E} \mid \mc{E}\ \text{locally free
of rank $k$}} \] and obtain the Grassmannian $\mr{Gr}(k, n+1)$.  \end{exm}

\begin{exm}[Picard Functor] Consider the ``Picard functor'' $T/S \mapsto
    \Pic(X_T)$ for a given $X/S$. This functor is \textbf{not} representable.
    If it was representable, then $F(-) = \Hom(-,X)$, but we know that $U
    \subseteq Z \mapsto \Hom(U,X)$ is a sheaf of sets over $Z$. However, when
    we apply this to $T \mapsto \Pic X_T$, then there are nontrivial line
    bundles on $X_T$ that come from $T$. If $\mc{L}$ is an invertible sheaf on
    $T$ such that $f^*_T \mc{L} \not\cong \mc{O}_{X_T}$, then let $T = \bigcup
    U_i$ be an open cover such that $\eval{\mc{L}}_{U_i} = \mc{O}_{U_i}$. But
    then \[ \Pic X_T \to \prod \Pic X_{U_i} \rightrightarrows \cdots \] sends
    $f_T^* \mc{L} \mapsto \prod \mc{O}_{X_{\mc{O}_{U_i}}}$ and so this sequence
    of sets is not exact.

    Instead, we consider the \textit{relative Picard functor} $\Pic_{X/S}$,
    which is defined by \[ T \mapsto \Pic X_T / f_T^* \Pic T. \] With
    additional assumptions on $X/S$ (for example projective, integral, etc), we
    can show that this functor is representable.  \end{exm}

\begin{exm}[Hilbert Scheme] Later, we will define the correct notion of a
    closed subscheme. Then for a fixed $X/S$, we consider the functor \[ T
    \mapsto \qty{\text{closed subschemes}\ Y \subseteq X_T\ \text{flat over
$T$}}. \] This is called the $\mr{Hilb}$ functor.  \end{exm}

\section{Properties of Schemes and Morphisms}%
\label{sec:properties_of_schemes_and_morphisms}

Recall that if $X$ is a scheme and $U \subseteq X$ is open, then $\qty(U,
\eval{\mc{O}_X}_U)$ is a scheme. We would like a similar definition for a
\textbf{closed} subscheme.

\begin{defn} A morphism $j \colon Y \to X$ is called an \textit{open immersion}
    if $j$ is a homeomorphism onto an open subset $U \subseteq X$ and the sheaf
    morphism $\mc{O}_X \to j_* \mc{O}_Y$ induces an isomorphism
    $\eval{\mc{O}_X}_U \simeq \eval{j_* \mc{O}_Y}_U$.  \end{defn}

\begin{exm} The maps $\A^n_R \to \P^n_R$ onto the standard open subsets are
open immersions. Similarly, if $X = \bigcup U_i$, then $U_i \to X$ is an open
immersion.  \end{exm}

\begin{defn} Let $X$ be a scheme. Then a \textit{closed subscheme} if a pair
    $(Z, \mc{I})$ of a closed subset $Z \subseteq X$ and a sheaf of ideals
    $\mc{I} \subseteq \mc{O}_X$ supported on $Z$ such that $(Z, \mc{O}_X /
    \mc{I})$ is a scheme.  \end{defn}

\begin{exm} Let $X = \Spec R$ and $I \subseteq R$ be an ideal. Then if we take
$\mc{I} = \wt{I}$, then $(Z, \mc{O}_X / \mc{I})$ is a scheme isomorphic to
$\Spec R/I$.  \end{exm}

\begin{rmk} Because $\mc{O}_X / \mc{I}$ is an $\mc{O}_X$-module of finite type,
we know that $\supp \mc{O}_X/\mc{I}$ is closed, and thus $\mc{I}$ determines
the closed subset $Z \subseteq X$.  \end{rmk}

\begin{exm} If $X$ is a scheme and $\mc{I}$ is a quasicoherent sheaf of ideals,
    then $Z = \supp \mc{O}_X / \mc{I}$ is a scheme with structure sheaf
    $\mc{O}_X/\mc{I}$.

    To see this, note that because $\mc{I}$ is quasicoherent, then we can
    consider an open cover $\qty{U_i = \Spec A_i}$ such that
    $\eval{\mc{I}}_{U_i} = \wt{I}_i$ for ideals $I_i \subseteq A_i$. But then
    we see that \[ \supp \mc{O}_X/\mc{I} \cap U_i = \supp \wt{A_i/I_i} = V(I_i)
    = \Spec A_i/I_i. \] and therefore the support is covered by affine open
schemes $Z \cap U_i$.  \end{exm}

\begin{exm}[Non-example] Let $0 \in \A^1_k$ be a closed point and let $U =
    \A^1_k \setminus \qty{0} \hookrightarrow \A^1_k$ be the open immersion. Now
    if $j \colon U \hookrightarrow X$ is open and $\mc{F}$ is a sheaf on $U$,
    then we can define \[ j_! (\mc{F})(V) = \begin{cases} \mc{F}(V) & V
    \subseteq U \\ 0 & V \not\subseteq U.  \end{cases} \] One can check that
    this is a sheaf.
    
    Now we see that $j_! \mc{O}_U \to \mc{O}_X$ is a sheaf of ideals and
$\mc{O}_X/j_! \mc{O}_U$ is supported at $0$. To see this, note that ${ ( j_!
\mc{O}_U ) }_x = \mc{O}_{X,x}$ away from $0$ and the stalk vanishes at $0$. But
now we know that $Z/\mc{O}_X/j_! \mc{O}_U$ is not a scheme because here $Z =
\qty{0}$. If $Z$ were a scheme, then $Z$ would be affine, but $\Gamma(Z,
\mc{O}_X/j_! \mc{O}_U) = { k[t] }_(t)$ is not a field.  \end{exm}

The problem in the previous example is that $j_! \mc{O}_U$ is \textbf{not}
quasi-coherent!

\begin{prop} Let $\mc{I} \subseteq \mc{O}_X$ be a sheaf of ideals and $Z =
\supp \mc{O}_X/\mc{I}$. If $(Z, \mc{O}_Z)$ is a closed subscheme, then $\mc{I}$
is a quasicoherent sheaf of ideals.  \end{prop}

\begin{cor} Any closed subscheme of an affine subscheme is affine.  \end{cor}

\begin{rmk} Using the fact that quasicoherent sheaves form an abelian category,
we see that $\mc{I} \subseteq \mc{O}_X$ is quasicoherent if and only if
$\mc{O}_X/\mc{I}$ is quasicoherent.  \end{rmk}

\begin{proof}[Proof of Proposition] If $U \subseteq X \setminus Z$, then there
is nothing to check. If $x \in Z$, then we pass to open affines $x \in U
\subseteq X$.  \end{proof}

\begin{defn} A morphism $f \colon Z \to X$ of schemes is a \textit{closed
immersion} if \begin{enumerate} \item $f$ is injective and a homeomorphism onto
a closed subset of $X$.  \item The map $\mc{O}_X \to f_* \mc{O}_Z$ is
surjective.  \end{enumerate} By definition, we have a bijection \[
\qty{\text{closed subschemes }Z \subseteq X} \longleftrightarrow
\qty{\text{closed immersions } f \colon Z \to X}. \] \end{defn}

\begin{prop} Let $i \colon Z \to X$ be a closed immersion. For any $U \subseteq
X$ open affine such that $U \cap Z \neq \emptyset$, the set $i^* (U) = Z \cap
U$ is an open affine subset of $Z$.  \end{prop}

\begin{proof} Fix $x \in Z$ and let $x \in U_1 \subseteq X$ be open affine.
    Then let $x \in V_1 \subseteq Z \cap U$ be open affine. Now $Z \subseteq
    V_1$ is a closed subset of $Z$ (and of $U_1$) and is disjoint from $x \in
    Z$. Now there exists $\alpha \in \Gamma(U_1, \mc{O}_{U_1})$ that vanishes
    on $Z \setminus V_1$ but not on $x$.\footnote{Giulia said something about
    GIT here. I will shamelessly plug the GIT seminar at
\url{http://www.math.columbia.edu/~plei/f20-GIT.html}} But now
${(U_1)}_{\alpha} \eqqcolon U$ is open affine, and therefore $U \cap Z =
{(U_1)}_{\alpha} \cap Z = {(V_1)}_{\alpha}$.  \end{proof}

This means that if $U = \Spec R$, then $i \colon U \cap Z \hookrightarrow U$ is
a map $\Spec S \to \Spec R$, and thus $\eval{ (Z, \mc{O}_Z) }_V = (\Spec S,
\wt{S})$. This implies that $i_* \mc{O}_Z = \wt{S}$. Therefore $\ker \mc{O}_X
\to \mc{O}_Z = \wt{I} = \wt{\ker (R \to S)}$ is quasicoherent and therefore we
have proved the bijection between closed subschemes and closed subschemes.

\begin{cor} The map $f \colon Z \to X$ is a closed immersion if and only if
    there exists an affine open covering $\qty{U_i}$ of $X$ such that
    $f^{-1}(U_i)$ is affine and $\Gamma(U_i, \mc{O}_{U_i}) \to
    \Gamma(f^{-1(U_i)}, \mc{O}_{f^{-1}(U_i)})$ is surjective.  \end{cor}

Of course, given a closed subset $Z \subseteq X$, there may be many different
quasicoherent sheaves of ideals that give $Z$ different scheme structures.

\begin{exm} Consider $0 \in \A^1_k$. Then the possible closed subschemes
    supported at $0$ are given by $\Spec k[x]/x^2$ corresponding to ideals $(x)
    \supseteq (t^2) \supseteq \cdots \supseteq (t^n) \supseteq \cdots$. Note
    that these Artinian rings are relevant in deformation theory.  \end{exm}

\begin{defn} \begin{enumerate} \item Let $X$ be a scheme. Then a
    \textit{subscheme} of $X$ is a pair $(Y, \mc{O}_Y)$ such that $Y \subseteq
    X$ is locally closed and if $U \subseteq X$ is the largest open subset of
    $X$ such that $Y \subseteq U$ is closed, then $Y \subseteq U$ is a closed
    subscheme.  \item An \textit{immersion} $f \colon Y \to X$ is a
    homeomorphism onto a locally closed subset such that for all $y \in Y$, the
    map $\mc{O}_{X,f(y)} \to \mc{O}_{Y,y}$ is surjective.  \end{enumerate}
\end{defn}

Now we may consider the image of a morphism of schemes. For example, we may
have closed immersions (here, $f(Z)$ is a closed subscheme) and open immersions
(here $f(Y)$ is an open subscheme).

Consider the map $\A^2_k \to \A^2_k$ given by $(x,y) \mapsto (x, xy)$. Then the
image of $f$ is not locally closed, but it is a constructible set. Recall that
if $X$ is a topological space, then a constructible set $S \subseteq X$ is a
finite union of locally closed subsets.

\begin{exm} Consider $\A^1_k$ and let $K = k(t)$. Then we have the inclusion of
the generic point $\Spec K \to \A^1_K$.  \end{exm}

\begin{defn} Let $X,Y$ be integral schemes. Then a morphism $f \colon X \to Y$
is called \textit{dominant} if $f(X) \subseteq Y$ is dense.  \end{defn}

\begin{exm} The morphism $\Spec K \to \A^1_k$ is dominant. If $U \subseteq Y$
is open, then the inclusion is dominant. Any surjective morphism is dominant.
The map $\qty{xy = 1} \subset \A^2_k \to \A_1$ is dominant.  \end{exm}

\begin{exer} Let $f \colon X \to Y$ is dominant if and only if $f(\eta_X) =
\eta_Y$.  \end{exer}

\begin{exm} If $A, B$ are domains, then $\Spec A \to \Spec B$ is dominant if
and only if $B \to A$ is injective.  \end{exm}

Now if $f \colon X \to Y$ is dominant, how bad can $f(X) \subseteq Y$ be? When
does it contain an open subset?

\begin{thm}[Chevalley] Let $f \colon X \to Y$ be a morphism of finite type and
$Y$ be Noetherian. Then for any constructible set $S \subseteq X$, $f(S)
\subseteq Y$ is constructible.  \end{thm}

\begin{defn} A morphism $f \colon X \to Y$ is \textit{of finite type} if it is
locally of finite type and quasi-compact.  \end{defn}

\begin{cor} If $f \colon X \to Y$ is of finite type, $Y$ is Noetherian, and $f$
is dominant, then $f(X) \subseteq Y$ contains an open subset.  \end{cor}

\begin{exm} If $X$ is a Noetherian topological space, then $C \subseteq X$ is
constructble if and only if for all closed irreducible $Z \subseteq X$, $Z \cap
C$ contains an open subset of $Z$ or $\ol{Z \cap C} \subsetneq Z$.  \end{exm}

\begin{cor} If $f$ is as above and dominant and $X,Y$ are integral, then $f(X)
\supseteq U$ for some open subset $U \subseteq Y$.  \end{cor}

\begin{proof}[Proof of Chevalley] Because $f$ is of finite type and $Y$ is
    Noetherian, there exists a finite cover $X = \bigcup \Spec A_{ij}$ and $Y =
    \bigcup \Spec B_i$, where $f(\Spec A_{ij}) \subseteq \Spec B_i$. Then we
    know $A_{ij}$ is a finitely-generated $B_i$-algebra. But then each $f(C
    \cap \Spec A_{ij})$ is constructible, we can assume that $X = \Spec R, Y =
    \Spec S$ are affine. Then we have a morphism of rings $S \to R = S[t_1,
    \ldots, t_k] / I$. We may also assume that $\sqrt{I} = I$ because this is a
    topological statement. In addition, we may also assume that $S$ is reduced.

    Now $X \to Y$ factors through $\A^n_S$, where $X \hookrightarrow \A^n_S$ is
    a closed immersion. Therefore we can assume $X = \A^n$. But then $\A^n \to
    \Spec S$ factors as \[ \A^n_S \to \A^{n-1}_S \to \cdots \to \A^1_S \to
    \Spec S, \] and therefore we can assume $X = \A^1_S$. Because $\Spec S$ is
    Noetherian, it has finitely many irreducible components $Z_i$, so now we
    may assume that $S$ is a domain. After this, we apply the following lemma.
\end{proof}

\begin{lem} Let $S$ be a domain and $f \colon \A^1_S \to \Spec S$. Then for all
    $C_0 \subseteq C \subseteq \A^1_S$ where $C_0 \subseteq C$ is open and $C
    \subseteq \A^1$ is closed and irreducible, there exists an open subset $U
    \subseteq \Spec S$ such that $f(C_0) \supseteq U$ or $f(C_0) \cap U =
    \emptyset$.  \end{lem}

\begin{proof} Let $\Spec S$ be integral and $\eta \in \Spec S$ be the generic
    point. Then let $K \coloneqq K(\eta)$. Then we have a commutative diagram
    \begin{equation*} \begin{tikzcd} \A^1_K \arrow{r} \ar{d} & \A^1_S \ar{d} \\
        \Spec K \ar{r} & \Spec S.  \end{tikzcd} \end{equation*} Using the
        following exercise, we see that either $C \to \Spec S$ is dominant, in
        which case $\eta_C \in C_{\eta} \neq \emptyset$, or not, in which case
        $\ol{f(C)} \subseteq \Spec S$ and thus there exists $U \subseteq \Spec
        S$ such that $f(C_0) \cap U = \emptyset$. Now there are two cases:
        \begin{enumerate} \item $f^{-1}(\eta) \cap C = \A^1_K$. In this case,
            choose $C = \A^1_S \supseteq C_0 \supseteq {(\A^1_S)}_g$ for some
            $0 \neq g = a_0 t^n + a_1 t^{n-1} + \cdots$. Therefore $0 \neq a_0
            \in S$, so now we show that $f(C_0) \supseteq U_{a_0} - \Spec
            S_{a_0}$. But here, for all $x \in \Spec C$, we have $f^{-1}(x) =
            \Spec k(x)[t] = \A^1_{k(x)}$, so \[ f^{-1}(x) \cap C_0 \supseteq
                f^{-1}(x) \cap {(\A^1_S)}_g = \qty{y \in \A^1_{j(x)} \mid
                \ol{g}(y) \neq 0}. \] But then if $x \in U_{a_0}$, then
                $\ol{a}_0 \neq 0$, so $\ol{g} \neq 0$. But now \[ f^{-1}(x)
                \cap C_0 \supseteq f^{-1}(x) \cap \qty{\ol{g} \neq 0}. \] But
                this is nonempty and thus $x \in f(C_0)$, so $U_{a_0} \subseteq
                f(C_0)$.  \item $f^{-1}(\eta) \cap C \eqqcolon C_{\eta} \in
                \A^1_K$ is a closed point. Then $C \subset V(\mf{p})$ for some
                prime ideal, and then $\mf{p} K[t] = (g)$ for some irreducible
                $g \in K[t]$. Up to inverting denominators, we may assume that
                $g \in S[t]$. But then $C_0 \subseteq C \subseteq V(G)
                \subseteq \A^1_S$. Now we see that \[ f^{-1}(\eta) \cap C_0 =
                f^{-1}(\eta) \cap C = f^{-1}(\eta) \cap V(g). \] But now $V(g)
                \setminus C_0$ is constructible, so we can write $\ol{V(g)
                \setminus C_0} = \bigcup W_i$ as a finite union of closed
                irreducible subsets, and $\ol{f(W_i)} \subsetneq \Spec S$.
                Therefore $\ol{f}(W_i) \subseteq V(\alpha)$ for some $0 \neq
                \alpha \in S$. Now consider $\Spec S \supseteq U_{\alpha a_0}
                \ni x$: \begin{enumerate} \item If $\alpha(x) \neq 0$, then $x
                    \notin \ol{f(W_i)}$, so $f^{-1}(x) \cap V(g) = f^{-1}(x)
                    \cap C_0$.  \item If $a_0(x) \neq 0$, then $\ol{g}(t) \in
                    k(x)[t]$ is nonzero of positive degree, so $V(\ol{g})
                    \subseteq \A^1_k$ is a nonempty closed subset.
            \end{enumerate} Therefore, for $x \in U_{\alpha, a_0}$, we have
            $f^{-1}(x) \cap C_0 = f^{-1}(x) \cap V(g) \neq \emptyset$, so
            $U_{\alpha, a_0} \subseteq f(C_0)$. \qedhere \end{enumerate}
        \end{proof}

\begin{exer} Let $f \colon X \to Y$ with $X,Y$ integral and $\eta_Y \in Y$ the
generic point. Then $X_{\eta_Y}$ is irreducible.  \end{exer}

We will use this to study closed points of schemes $X/k$ of finite type over a
field $k$.

\begin{cor} Let $X$ be of finite type over a field $k$. Then $x \in X$ is a
closed point if and only if $k(x)$ is an algebraic extension of $k$.  \end{cor}

\begin{proof} Suppose $x \in X$ is closed. Then $x \in U = \Spec R \subseteq X$
    is constructible in $U$. Then $x \in U \subseteq \A^n_k \to \A^1_k$, and we
    will denote the coordinates by $U \xrightarrow{f_i} \A^1_k$. Therefore
    $f_i(x)$ is a constructible set in $\A^1_k$, so it must be a closed point.
    But then $k(f_i(x))$ is an algebraic extension of $k$. But then the
    extension $k \subseteq k(x)$ by the $f_i(x)$ and is thus algebraic.

    Now suppose $k \subseteq k(x)$ is algebraic. If $x \in X$ is not closed,
then there exists $x \neq y \in \ol{\qty{x}}$. Then we can choose $U \ni x,y$
open affine, so $x$ is not closed in $U$. But now $x = \mf{p} \in \Spec R$, so
$k \subseteq R/\mf{p} \subseteq k(x)$. But then $R/\mf{p}$ is a finitely
generated integral extension of $k$, so $L$ is a field and thus $\mf{p}$ is
maximal and $x$ is closed.  \end{proof}

\begin{rmk} If $k$ is algebraically closed, then closed points are precisely
those with residue field $k$.  \end{rmk}

\begin{exm} Let $A$ be a local Noetherian ring. If $U = X \setminus \mf{m}$,
    then $U$ satisfies the descending chain condition for closed subsets, and
    therefore has closed points. However, none of these points are closed in
    $X$ because $X$ has a unique closed point.  \end{exm}

\begin{cor} Let $X$ be a scheme of finite type over $k$. Then if $U \subseteq
X$ is open and $x \in U$, then $x$ is closed in $U$ if and only if $x$ is
closed in $X$.  \end{cor}

\begin{cor} Let $X$ be of finite type over $k$. Then \begin{enumerate} \item
    For any $S \subseteq X$ closed, the closed points of $S$ are dense in $S$.
\item $X$ can be reconstructed as a topological space from the set of its
    closed points.  \end{enumerate} \end{cor}

\begin{proof} Let $S \subseteq X$ be closed. It suffices to show that for all
    open $U \subseteq X$, $U \cap S$ contains a closed point. Assuming $U =
    \Spec R$ is affine, then $S \cap U = V(I)$ for some ideal $I \subseteq R$,
    and the desired result follows from the existence of maximal ideals.
\end{proof}

\section{More on Varieties}% \label{sec:more_on_varieties}

\textit{Note: Notes were not taken in great detail for this section.} 

Let $k$ be an algebraically closed field. We know $\A^n(k) = k^n$. Then we will
define affine algebraic sets to be the common zero set of a set of polynomials.
Of course, we can declare the Zariski topology on $\A^n(k)$. Of course there is
a natural correspondence between radical ideals of $k[x_1, \ldots, x_n]$ and
closed subsets of $\A^n(k)$ giving a correspondence between maximal ideals and
points.

\begin{rmk} With the Zariski topology, $\A^n(k)$ is a Noetherian topological
space.  \end{rmk}

A morphism of algebraic sets is a map $\A^n \supseteq X \to Y \subseteq \A^m$
that is expressible in terms of polynomials $f = (f_1, \ldots, f_m) \in {
k[x_1,\ldots, x_n] }^n$. Dually, this defines a map \[ k[y_1, \ldots, y_m]/I(Y)
\to k[x_1, \ldots, x_n] \qquad y_i \mapsto f_i(x). \] References that use the
language of varieties are Chapter 1 of Hartshorne, Shafarevich,
Griffiths-Harris, etc.

Now if $X = V(I) \subseteq \A^n(k)$ for a radical ideal $I$, then we can define
\begin{defn} The \textit{affine coordinate ring} of $X$ is $k[X] = \Hom(X,
\A^1(k)) = k[x_1, \ldots, x_n] / I$.  \end{defn}

\begin{prop} If $X$ is an affine algebraic set, then $k[X]$ is a reduced
finitely-generated $k$-algebra. In addition, $X$ is irreducible if and only if
$k[X]$ is a domain.  \end{prop}

For any $x \in X$, we define the maximal ideal $\mf{m}_x = \ker (k[X]
\xrightarrow{\mr{ev}_x} k)$. More generally, if $Z \subseteq X$ is closed, then
$I(Z) = \qty{f \mid f(x) = 0} = \bigcap_{x \in Z} \mf{m}_x$. For an ideal $I
\subseteq K[X]$, we can define the closed set $V(I)$. For example, if $I =
(f)$, we can define the principal open subsets $X_f$.

Note that there is an equivalence of categories between irreducible affine
algebraic sets and finitely-generated $k$-algebras that are domains.

\subsection{Rational Functions}% \label{sub:rational_functions}

Let $X$ be an irreducible algebraic set in $\A^n$. We may consider the field of
fractions $k(X)$ of $k[X]$, and this will be called the field of rational
functions on $X$. If $\frac{f}{g} \in k(X)$, we have a map \[ [f,g] \colon X
\to \P^1(k), \] whatever $\P^1$ means. Really, we have a map $X \dashrightarrow
\A^1(k)$. This is of course regular on $X_g$.

\begin{rmk} If $f' = fh, g' = gh$, then $\frac{f}{g} = \frac{f'}{g'}$ in
$k(X)$. We see that $X_g \supseteq X_{g'}$ and for all $x \in X_{g'}$, the
fractions $\frac{f(x)}{g(x)} = \frac{f'(x)}{g'(x)}$ agree.  \end{rmk}

\begin{lem} If there exists $\frac{f}{g}, \frac{f'}{g'} \in k(X)$ such that
there exists $U \subseteq X$ where $\frac{f(x)}{g(x)} = \frac{f'(x)}{g'(x)}$
for all $x \in U$, then $\frac{f}{g} = \frac{f'}{g'}$.  \end{lem}

\begin{proof} Up to multiplying by something in $k[X]$, we may assume $g = g'$.
But this means that $f'(x) = f(x)$ for all $x \in U$, which means $V(f-f')
\supseteq U$. By irreducibility of $X$, we have $X = V(f-f')$, so $f = f'$.
\end{proof}

We now define a sheaf of regular functions, and we can upgrade affine algebraic
sets to ringed spaces.

\begin{defn} Define the sheaf $\mc{O}_X$ by \[ \mc{O}_X(U) = \bigcap_{x \in X}
{k[X]}_{\mf{m}_x}. \] \end{defn} It should be obvious what the restriction maps
are, and they are injective.

\begin{lem} If $f \in k[X]$, then $\Gamma(X_f, \mc{O}_X) = {k[X]}_f$.
\end{lem}

\begin{proof} One direction is obvious. In the other direction, if we define
    $\mf{a} = \qty{h \mid h g \in k[X]} \subseteq k[X]$, then we want to show
    that $f \in \sqrt{\mf{a}}$. If we choose representatives $\mf{g} = g_1 /
    g_2 \in \mc{O}_X(X_f)$, we see that $\mf{g}_2 \in \mf{a}$. Thus $g_2 \notin
    \mf{m}_X$ for $x \in X_f$, so if $x \in X_f$, then $x \notin V(\mf{a})$.
\end{proof}

\begin{prop} A map $f \colon X \to Y$ is a morphism of irreducible affine
algebraic sets if and only if for all $g \in k[Y]$, then $g \circ f \in k[X]$.
Equivalently, $f$ is continuous and induces a morphism of sheaves.  \end{prop}

\begin{defn} A locally ringed space $(X, \mc{O}_X)$ is called a
\textit{prevariety} if $X$ is connected and there exists a finite covering of
$X$ by irreducible affine algebraic sets.  \end{defn}

\begin{rmk} Any prevariety $X$ is Noetherian and irreducible.  \end{rmk}

\begin{defn} We define the \textit{function field} of a prevariety $(X,
\mc{O}_X)$ to be the fraction field of $\mc{O}_X(U)$ for any open affine $U
\subseteq X$.  \end{defn}

In particular, $K(U)$ is independent of the affine open subset $U \subseteq X$.
In particular, all restrictions are injective and $\mc{O}_X(U) \cap \mc{O}_X(V)
= \mc{O}_X(U \cap V)$.

\begin{defn} If $X, Y$ are prevarieties, then a morphism $f \colon X \to Y$ is
a morphism of locally ringed spaces.  \end{defn}

\begin{rmk} If $f \colon X \to Y$ is a morphism of prevarieties, then we do not
have a pullback of rational functions in general.  \end{rmk}

\begin{exm} Projective varieties are prevarieties.  \end{exm}

\begin{thm} Let $k$ be algebraically closed. Then there is an equivalence of
categories between integral schemes of finite type over $k$ and prevarieties
over $k$.  \end{thm}

\begin{proof} Given a scheme $(X, \mc{O}_X)$, we will consider the prevariety
    $(X(k), \mc{O}_{X(k)})$, where $X(k) = \Hom_k(\Spec k, X)$ is the set of
    closed points. Of course, if $\Spec A = U \subseteq X$ is open affine, then
    $U(k)$ is an affine algebraic set. Now, we simply define $\mc{O}_{X(k)}
    (U(k)) = \mc{O}_X(U)$, and this is a sheaf. We can view this as functions
    to $k$. Equivalently, we see that $\alpha^{-1}(\mc{O}_X)$ coincides locally
    with the sheaf of regular functions because $X$ is quasi-compact. Therefore
    we have defined a prevariety. To see that this is functorial, note that
    morphisms of schemes of finite type send closed points to closed points.

    Now in the other direction, given $X(k)$, we will simply aff a generic
    point $\eta_Z$ for every irreducible $Z \subseteq X(k)$. In fact, this
    defines a functor on topological spaces, called soberification. It is
    obvious what the topology should be. Of course, the inclusion $\alpha(X(k))
    \to t(X(k))$ is continuous and induces a bijection of open subsets. But now
    we note that if $X(k) \subseteq \A^n$ is an affine algebraic set with $A =
    k[X(k)]$, we note that $t(X(k)) = \Spec A$. Therefore $t(X(k))$ is covered
    by finitely many open affines. Now it remains to check that $\alpha_*
    \mc{O}_{X(k)}$ makes $t(X(k))$ a scheme. But this can be checked on
    affines. It is enough to show that if $U(k)$ is an affine algebraic set
    with $A = k(U(k))$, then $\Spec A = U \subseteq X$. But now, by definition,
    we have \[ \mc{O}_X(U) = \mc{O}_{X(k)}(U(k)), \] so when $U = \Spec A$, we
    have $\mc{O}_X(U) = A$. In addition, we have $\mc{O}_X(U_f) =
    \mc{O}_{U(k)}({U(k)}_f) = A_f$ for all $f \in k[U(k)]$, so we are done. To
    check functoriality, we note that soberification is functorial and then
    check that we obtain a map $\mc{O}_Y \to f_* \mc{O}_X$ from the map on
    $k$-points.  \end{proof}

\begin{rmk} Note that this means that a morphism of (integral) schemes of
finite type over $k$ is determined by its value on closed points.  \end{rmk}

\begin{rmk} Let $X$ be an integral scheme of finite type over $k$ and $X(k)$ be
    the corresponding prevariety. Then we have $K(X) = \mc{O}_{X, \eta}$ and
    the field $k(X(k)) = \mr{Frac}(k[U(k)])$ for any affine open $k(U(k))$, and
    these are the same field.  \end{rmk}

\subsection{Comparison with GAGA}% \label{sub:comparison_with_gaga}

Let $X$ be a projective scheme of finite type over $\C$. We may consider $X(\C)
\subseteq \P^n(\C)$. But now we may consider the analytic topology on
$\P^n(\C)$, and so we may consider the analytic space $X^{\mr{an}}(\C)$ with a
sheaf of holomorphic functions $\mc{O}_{X^{\mr{an}}}$. Of course, we may still
consider the continuous map \[ \gamma \colon X^{\mr{an}}(\C) \hookrightarrow X.
\] Next, to a coherent sheaf $\mc{F}$, we may consider a coherent analytic
sheaf $\mc{F}^{\mr{an}} = \gamma^{-1} \mc{F} \otimes_{\gamma^{-1} \mc{O}_X}
\mc{O}_X^{\mr{an}}$. Now Serre's GAGA tells us that this is an equivalence of
categories.

\subsection{Non-algebraically closed fields}%
\label{sub:non_algebraically_closed_fields}

We will now consider what happens if $k \subsetneq \ol{k}$. The most basic
example is $\A^1_k = \Spec k[t]$, and the closed points are given by $(p(t))$
for irreducible polynomials $p(t)$. Then any finite extension of $k$ that is
generated by a single element is the residue field of a closed point. In
particular, for any such field $k'$ we obtain a point in $\A^1_k(k')$. For
another example, consider a field extension $k \subsetneq k'$ with induced
morphism $X = \Spec k' \to \Spec k$. Then $X$ has no $k$-points. Returning to
affine space, any commutative diagram \begin{equation*} \begin{tikzcd} \Spec k'
\ar{r} \ar{dr} & \A^n_k \ar{d} \\ & \Spec k \end{tikzcd} \end{equation*} gives
us a $k'$-rational point of $\A^n_k \times_{\Spec k} \Spec k' = \A^n_{k'}$. In
particular, if $X = V(f_1, \ldots, f_m) \subseteq \A^n_k$, then $X(k') = \qty{x
\in {(k')}^n \mid f_i(x) = 0, i = 1, \ldots, m}$. We have a similar statement
for $\P^n_k$ and closed subschemes.

\begin{rmk} If $X = \bigcup U_i$ is a union of open affines, then $X(k') =
\bigcup U_i(k')$.  \end{rmk}

\begin{rmk} A $k'$-point $x \in X_k(X')$ determines a field extension $k
    \subseteq k(x) \subseteq k'$. If $\sigma \in \Aut(k'/k)$, then we can
    precompose $\Spec k' \xrightarrow{\sigma} \Spec k' \to X$ to get another
    $k'$-point. In addition, we have ${X_k(k')^{\sigma} = X_k(k^{\sigma})}$.
\end{rmk}

\begin{exm} Let $X$ be (locally) of finite type and $\ol{k}$ be the algebraic
    closure of $k$. Then we have the map \[ X_k(\ol{k}) \xrightarrow{\Sigma} X
    \qquad (\Spec \ol{k} \xrightarrow{i} X) \mapsto i(\Spec \ol{k}). \] From
    the characterization of the closed points, the image of $\ol{\Sigma}$
    consists of all closed points. Then we know that the absolute Galois group
    $G = \Aut(\ol{k} / k)$ of $\ol{k}$ acts on $X_k(\ol{k})$. But then the
    $G$-orbits of this action are contained in the fibers of $\Sigma$ and $G$
    acts transitively on the orbits, so the fibers are the $G$-orbits.
\end{exm}

\subsection{Classical projective geometry}%
\label{sub:classical_projective_geometry}

We will now consider open and closed prevarieties.  \begin{enumerate} \item An
    open subprevariety is simply an open subset with the structure sheaf
    restricted.  \item If $X$ is a prevariety and $Z \subseteq X$ is an
    irreducible closed subset, we can give $Z$ the structure of a prevariety.
    Then if $V \subseteq Z$ is open, we can define \[ \mc{O}_Z(V) = \qty{f
    \colon V \to k \mid \text{locally a restriction of a function on $X$}}. \]
    Note that if $Z \subseteq X \subseteq \A^n$, the structure of $Z$ as a
    subprevariety of $X$ is the same as the one of $Z \subseteq \A^n(k)$.
    \end{enumerate}

Now we will define projective varieties over algebraically closed fields. If we
consider the scheme $\P^n_k$, then $\P^n(k) = (k^{n+1} \setminus \qty{0}) /
k^{\times}$. Then the open affine charts give us the structure of a prevariety.
Now we define the sheaf of functions to be $\mc{O}_{\P^n}(U) = \mc{O}_{U_i}(U)$
when $U \subseteq U_i$ and in particular if we homogeneize, we have \[
\mc{O}_{\P^n}(U) = \qty{\varphi \colon U \to k \mid \varphi = \frac{F(x_0,
\ldots, x_n)}{G(x_0, \ldots, x_n)}, \deg F = \deg G}. \] As a consequence, we
have $k(\P^n(k)) = k(U_i) = k(x_0, \ldots, x_n)$. Now we need to consider the
global regular functions on $\P^n$. NOte that for a prevariety $X$ and $U,V$
open subsets, then $\mc{O}_X(U \cup V) = \mc{O}_X(U) \cap \mc{O}_X(V)$, so \[
\mc{O}_{\P^n}(\P^n) = \mc{O}_{\P^n} \qty(\bigcup U_i) = \bigcap
\mc{O}_{\P^n}(U_i) = \bigcap k\qty[ \frac{x_0}{x_i}, \ldots, \frac{x_n}{x_i} ]
= k. \]

\begin{rmk} Note that if $\P^n(\C)$ is considered as a complex manifold, then
$\Gamma(\P^n(\C), \mc{O}_{X^{\mr{an}}}) = \C$. More generally, there are no
nonconstant global holomorphic functions on any compact complex manifold.
\end{rmk}

Now closed subsets of projective space are given by the vanishing of
homogeneous polynomials $F_i$. This gives us the definition \begin{defn} A
    prevariety is called a \textit{projective variety}\footnote{Note that these
    are separated. Also consider that if $\text{variety} = \text{separated
prevariety}$, then $\text{separated} = \text{pre}^{-1}$ and thus we may
consider $\text{pre}^{-1}$schemes.} to a closed subprevariety of $\P^n(k)$.
\end{defn}

\begin{defn} A prevariety is called \textit{quasiprojective} if it is
isomorphic to an open subset of a projective variety.  \end{defn}

\begin{rmk} The structure of a prevariety does not depend on the embedding in
some ambient space.  \end{rmk}

We will now consider morphisms between (quasi)-projective varieties.

\begin{prop} Let $Y \subseteq \P^n(k)$ be a quasi-projective variety.
    \begin{enumerate} \item Given $f_0, \ldots, f_m \in k[x_0, \ldots, x_n]$
        homogeneous polynomials of the same degree such that $V_+(f_0, \ldots,
        f_m) \cap Y = \emptyset$, then the map \[ Y \to \P^m(k) \qquad y
        \mapsto [f_0(y), \ldots, f_m(y)] \] is a morphism of prevariety.
        Moreover, if $g_0, \ldots, g_m$ are homogeneous polynomails of the same
        degree with $V_+(g_0, \ldots, g_m) \cap Y = \emptyset$, they define the
        same morphism if and only if $g_i f_j = g_j f_i$ for all $i,j$.  \item
Conversely, given $\varphi \colon Y \to \P^m(k)$ a morphism of prevarieties,
then $\varphi$ is locally defined as above.  \end{enumerate} \end{prop}

To prove this result, simply consider the affine open subsets of $\P^m$ and the
sets $Y \cap \qty{f_i \neq 0}$.

\begin{rmk} Compare this to the statement that $\P^n_k$ represents the functor
taking a scheme $X$ to the set of isomorphism classes of line bundles $\mc{L}$
with linearly independent sections $s_0, \ldots, s_n$.  \end{rmk}

\begin{prop} If $\emptyset \neq V_+(f_0, \ldots, f_m) = Z \subsetneq Y$, then
there exists a morphism $\varphi \colon U \to \P^m(k)$, where $U = Y \setminus
Z$. This gives us a \textit{rational map} $Y \dashrightarrow \P^n(k)$.
\end{prop}

\begin{exm} The map \[ \A^{n+1}(k) \setminus \qty{0} \xrightarrow{\pi} \P^n(k)
\qquad (x_0, \ldots, x_n) \mapsto [x_0, \ldots, x_n] \] is a morphism of
prevarieties. Given $Z = V_+(I) \subseteq \P^n(k)$, define the \textit{cone}
over $Z$ to be $C(Z) = \ol{\pi^{-1}(Z)} = V(I) \subseteq \A^{n+1}(k)$. On the
other hand, if $I \subseteq k[x_0, \ldots, x_n]$ is a homogeneous ideal, then
$V(I) = C(V_+(I))$.  \end{exm}

\begin{rmk} The cone over $Z$ depends both on $Z$ and on the embedding in
    projective space. On the cone over $Z$, the origin is usually a singular
    point. Sometimes (for example when $Z$ is a point) it is not, but in
    general, the properties of the singularity depend on the properties of the
    embedding.  \end{rmk}

\begin{exm} Consider the map $\P^1 \to \P^2$ given by $[x,y] \mapsto [x,y,0]$
    with image $W_1 = \qty{z = 0}$ and $C(W_1) = \A^2 \subset \A^3$. Consider
    the other map $\P^1 \to \P^2$ given by $[x,y] \mapsto [x^2, xy, y^2]$ with
    image $W_2 = \qty{xz = y^2}$ and cone $C(W_2)$. But then $W_1 \cong W_2
    \cong \P^1$, but $W_2$ has an $A_1$ singularity at the origin while $W_1$
    is smooth.  \end{exm}

Suppose $g \in GL(n+1)$. Then the action of $g$ on $\A^{n+1}(k) \setminus
\qty{0}$ is scaling-invariant, so it descends to $\P^n(k)$. Of course, we have
an exact sequence \[ 1 \to k^{\times} \to GL(n+1) \to \Aut(\P^n(k)). \] In
fact, we will see that $\Aut(\P^n) = PGL(n+1)$.

\begin{rmk} The \textit{Cremona group} of birational transformations of $\P^n$
is massive.  \end{rmk}

If $I = (L_0, \ldots, L_m)$, then $V_+(I) \subset \P^n$ is isomorphic to some
$\P^{n-m}$. Therefore, $PGL(n+1)$ acts transitively on the set of
$m$-dimensional linear subspaces in $\P^n$. Of course, this has the structure
of an algebraic variety, the \textit{Grassmannian}. This represents a functor
$\mc{O}^{n+1} \twoheadrightarrow \mc{L}$, where $\mc{L}$ has rank $m$. 

A linear subspace is a \textit{hyperplane} when it is defined by a single
equation $V_+(L)$, and is a line if it is isomorphic to $\P^1$. We can also
define the \textit{linear} span of $Z \subseteq \P^n$ to be \[ \ev{Z} =
\bigcap_{Z \subseteq L} L. \] For two points $p,q$, the line containing $p,q$
is denoted $\ol{pq}$.

\begin{defn} Points $p_1, \ldots, p_m \in \P^n$ are said to be in
\textit{general position} if no $(k+1)$ of them lie on a $(k-1)$-plane.
\end{defn}

\begin{exm} Three points in $\P^2$ are in general position if and only if they
are not collinear.  \end{exm}

Let $H = \qty{x_0 = 0} \subseteq \P^n$ be a hyperplane and suppose $p =
[1,0,\ldots,0] \notin H$. For any closed subset $Z \subseteq H$, we can write
$\ol{pZ} = \bigcup_{z \in Z} \ol{pz} \subseteq \P^n$. This is a closed subset.
Thus if $Z = V_+(I)$ for some ideal in $k[x_1, \ldots, x_n]$, then $\ol{pZ} =
V_+(I \cdot K[X_0, \ldots, X_n])$.

\begin{rmk} Let $L_1, L_2 \subseteq \P^n$ be linear subspaces. Then $\ev{L_1,
L_2} = \P^3$ if and only if $L_1 \cap L_2 = \emptyset$.  \end{rmk}

\begin{exm} With the same assumptions as the remark, let $Z \subseteq L_1$.
Then $\ol{L_2 Z} = \bigcup_{z \in Z} \ev{L_2, Z} \subseteq \P^n$ is a closed
subset.  \end{exm}

\begin{defn} Let $F$ be a homogeneous polynomial of degree $e$ in $k[x_0,
\ldots, x_n]$. Then $V_+(F) = X \subseteq \P^n$ is called a
\textit{hypersurface of degree $d$}. If $d=2$, then it is called a
\textit{quadric}.  \end{defn}

\begin{exm} In $\P^2$, the quadrics are either conics ($x^2+y^2+z^2$), a pair
of lines ($x^2+y^2$), or a double line ($x^2$).  \end{exm}

In characteristic not equal to $2$, we know that if $Q = V_+(q)$ is a quadric,
then $q$ is given by a symmetric matrix. But then we know that $PGL(n+1)$ acts
on ${k[x_0, \ldots, x_n]}_d$ and thus on quadrics in $\P^n$ in a way that
preserves the rank. This tells us that $Q$ is irreducible if and only if $q$
has rank different from $2$. If $r = \operatorname{rk}(q) < n$, then $Q$ is an
(iterated) cone over a rank $r$ quadric in $\P^r$.

Consider the map $\P^1 \xrightarrow{v_3} \P^3$ given by $[x,y] \mapsto [x^3,
x^2y, xy^2, y^3]$. Clearly the image of $v_2$ satisfies the equations $AD = BC,
AC = B^2, C^2 = BD$. On the other hand, we can explicitly construct $x,y$ from
the equations in $A,B,C,D$, so on the open chart $W \cap U_A$, we have the map
$[A,B,C,D] \mapsto [A,B]$. Similarly, on $W \cap U_D$, we have the inverse
$[A,B,C,D] \mapsto [C,D]$. The image of $v_3$ is called a \textit{twisted
cubic}. It is easy to see that $\ev{v_2(\P^1)} = \P^3$.\footnote{In fact, there
is something even stronger than this, but I can't remember it right now.} 

More generally, the degree $d$ \textit{Veronese embedding} $\P^n \to
\P^{\binom{n+d}{d}-1}$ is given by \[ {[x_0, \ldots, x_n] \mapsto
[x_0^{i_0}x_1^{i_1}\cdots x_n^{i_n}]}_{\sum i_k = d}. \] Up to $PGL(n+1)$,
replacing $x_0, \ldots, x_d$ with a different basis of ${k[x_0, \ldots,
x_n]}_d$. In fact, the image is closed, and if we write $v_d(\P^n) =
V_+(\mf{a})$, then we know that $\mf{a}$ is a homogeneous prime ideal.

\begin{exm} Consider the degree $2$ Veronese embedding $v_2 \colon \P^2 \to
\P^5$. Then if $H$ is a hyperplane, $H \cap v_2(\P^2)$ is the image of a conic
in $\P^2$.  \end{exm}

\begin{exm} Let $F \in {K[x_0, \ldots, x_1]}_d$ and let $X = V_+(F)$. Then
there exists a hyperplane in $\P^{\binom{n+d}{d}-1}$ such that $H \cap
v_d(\P^n) = V_+(F)$. This tells us that $\P^n \setminus X$ is affine.
\end{exm}

Note the following facts: \begin{enumerate} \item If $k$ is a field and $X,Y$
    are $k$-schemes (locally) of finite type over $k$, then $X \times_k Y$ is
    (locally) of finite type over $k$.  \item If $k = \ol{k}$, then $X,Y$ are
    integral if and only if $X \times_k Y$ are integral. At the level of closed
    points, we note that $X(k) \times Y(k) = (X \times_k Y)(k)$.
    \end{enumerate}

Now observe that the product of projective spaces is a projective variety.
Define the map \[ \P^n \times \P^m \to \P^{(n+1)(m+1)-1} \qquad [\ldots,
x_i,\ldots], [\ldots, y_j, \ldots] \mapsto [\ldots, x_i y_j, \ldots]. \] This
is called the \textit{Segre embedding}. If the $(n+1) \times (m+1)$ matrix of
coordinates is given by $Z_{ij}$, then the image of the Segre embedding is
given by the vanishing of the $2 \times 2$ minors. For example, we note that
the Segre embedding \[ \P^1 \times \P^3 \to \P^3 \qquad [x,y], [u,v] \mapsto
[xu,xv,yu,yv] \] has image $\qty{AD = BC}$, which is a smooth quadric. In fact,
the images of $\P^1 \times \qty{p}, \qty{p} \times \P^1$ give us two families
of lines on $\Q$.

\section{Dimension}% \label{sec:dimension}

Dimension is a topological property. 

\begin{defn} Let $X$ be a topological space. Then we define the
    \textit{dimension} of $X$ to be \[ \dim X \coloneqq \sup \qty{\ell \mid X_0
    \supsetneq \cdots \supsetneq X_{\ell}}, \] where each $X_i$ is a closed
    irreducible subset of $X$.  \end{defn}

We will define the dimension of a scheme to be the dimension of its underlying
topological space. If $X = \emptyset$, then we set $\dim X = - \infty$. 

\begin{warn} Even for a Noetherian scheme $X$, we can have $\dim X = \infty$.
There is an example of Nagata in Vakil's notes or as tag
\href{https://stacks.math.columbia.edu/tag/02JC}{02JC} in the Stacks project.
\end{warn}

It is easy to see that if $X = \Spec A$, then $\dim X = \dim A$.

\begin{exm} Let $A$ be a principal ideal domain such that $A$ is not a field.
Then all chains of prime ideals are given by $0 \subseteq (t)$, and thus $\dim
A = 1$.  \end{exm}

\begin{exm} If $A$ is a ring and $\mf{p}_0 \subsetneq \cdots \subsetneq
    \mf{p}_{\ell}$ is a chain of prime ideals, then in $A[t]$ as have the chain
    of ideals \[ \mf{p}_0 A[t] \subsetneq \cdots \subsetneq \mf{p}_{\ell} A[t]
    \subsetneq (\mf{p}_{\ell}, t), \] and thus $\dim A[t] \geq \dim A + 1$. If
$A$ is Noetherian, then this is an equality.  \end{exm}

\begin{lem} Let $X$ be a topological space.  \begin{enumerate} \item If $Y
    \subseteq X$ has the subspace topology, then $\dim Y \leq \dim X$. If $X$
    is irreducible and $\dim X < \infty$, then this inequality is strict.
\item If $X = \bigcup U_{\alpha}$ is an open covering, then $\dim X =
    \sup_{\alpha} \dim U_{\alpha}$.  \item If $X = \bigcup X_i$ is a union of
    irreducible components, then $\dim X = \sup \dim X_i$.  \item If $X$ is a
    scheme, then $\dim X = \sup_{x \in X} \dim \mc{O}_{X,x}$.  \end{enumerate}
\end{lem}

\begin{lem} The first three properties are obvious. Now if $X = \bigcup \Spec
    A_{\alpha}$ is a cover by affines, then we know $\dim X = \sup_{\alpha}
    \dim A_{\alpha}$. But now we know that for any prime ideal $\mf{p}$,
    $\mr{ht}(\mf{p}) = \dim A_{\mf{p}}$, so we are done.  \end{lem}

\begin{cor} If $Y \hookrightarrow X$ is a closed immersion, then $\dim Y \leq
\dim X$. If $X$ is integral and $Y \subsetneq X$, then $\dim Y < \dim X$.
\end{cor}

\begin{exm} Let $A$ be a ring. Then $\dim A = 0$ if and only if all prime
ideals are maximal. If $A$ is Noetherian, then this holds if and only if $A$ is
Artinian, which is equivalent to $A$ being the product of its localizations.
\end{exm}

\begin{defn} A morphism $\Spec B \to \Spec A$ is \textit{integral} if $A \to B$
is an integral map of rings.  \end{defn}

Recall that $B$ is a finite $A$-module if and only if $\varphi$ is integral and
$B$ is a finitely-generated $A$-algebra. Recall that for integral morphisms if
$\mf{q}_1 \subseteq \mf{q}_2$ are prime ideals of $B$ such that $\mf{q}_1 \cap
A = \mf{q}_2 \cap A$, then $q_1 = q_2$. Of course, integral morphisms of rings
also satisfy going-up and going-down. The geometric interpretation of this is
\begin{prop} Let $\Spec B \to \Spec A$ be an integral morphism with $A \subset
B$. We already know that $f$ is closed and surjective. The three properties of
integral morphisms of rings imply that $\dim B = \dim A$.  \end{prop}

\begin{defn} Let $X$ be a topological space and $Z \subseteq X$ be closed and
    irreducible. Then we will define \[ \codim_X(Z) \coloneqq \sup \qty{\ell
    \mid z = z_{\ell} \subsetneq \cdots \subsetneq Z_0}, \] where each $Z_i$ is
    a closed irreducible subset of $X$.  \end{defn}

\begin{exm} If $X = \Spec A$ and $Z = V(\mf{p})$, then $\codim_X(Z) =
\mr{ht}(\mf{p}) = \dim A_{\mf{p}}$.  \end{exm}

\begin{exm} Let $X$ be a scheme and $Z \subseteq X$ be closed and irreducible.
    Then \[ \codim_X(Z) = \sup_{U \cap Z \neq \emptyset} \qty{\codim_U (Z \cap
    U)} = \sup_{U_{\alpha}} \dim \mc{O}_{X, \eta_{Z \cap U_{\alpha}}}. \]
\end{exm}

\begin{rmk} We always have the inequality $\dim Z + \codim_X Z \leq \dim X$.
Equality holds if all maximal chains of closed irreducible subsets have the
same length.  \end{rmk}

\begin{defn} A topological space $X$ is called \textit{catenary} if all maximal
chains of closed irreducible subsets have the same length.  \end{defn}

\begin{exm} If $A$ is a DVR over a field $k$ with $\mf{m} = (t)$, then $\dim
    \Spec A[x] \geq 2$. It is easy to see that the ideals $\mf{m}_1 = (tx-1)$
    and $\mf{m}_2 = (t,x)$ are both maximal, so if $Z_i = V(m_i)$, we know
    $\dim Z_i = 0$. However, we see that $\codim_X(Z_2) \geq 2$ while $\codim_X
    (Z_1) = 1$ because $\mf{m}_1$ is principal.  \end{exm}

\begin{thm}\label{thm:dimfinitetype} Let $X$ be an integral scheme locally of
    finite type over $k$.  \begin{enumerate} \item There is an equality $\dim X
        = \operatorname{tr.deg} k(\eta)$. Moreover, $X$ is catenary.  \item For
        all closed points $x \in X$, $\dim X = \dim \mc{O}_{X,x}$.  \item If
        $X, Y$ are finite type over $k$ with $Y$ integral and $f \colon Y \to
        X$ is dominant, then $\dim Y \geq \dim X$.  \item If $f \colon Y \to X$
        is a quasi-finite morphism of schemes of finite type over $k$, then
        $\dim Y \leq \dim X$.  \end{enumerate} \end{thm}

\begin{rmk} Let $A$ be a discrete valuation ring over $k$ and $X = \Spec A$.
Then $\Spec k \sqcup \Spec k(\eta_X) \to \Spec A$ is a bijection on points, but
$\dim (\Spec k \sqcup \Spec k(\eta_X)) = 0 < \dim X = 1$.  \end{rmk}

The proof of the Theorem will require Noether normalization:
\begin{thm}[Noether normalization lemma] Let $A \neq 0$ be a finitely generated
    integral $k$-algebra. Then there exists $t_1, \ldots, t_d \in A$ such that
    the morphism $k[t_1, \ldots, t_d] \to A$ is injective and integral.
\end{thm}

\begin{proof}[Proof of Theorem~\ref{thm:dimfinitetype}]\leavevmode
    \begin{enumerate} \item Let $X = \bigcup \Spec A$ where $A$ is a finitely
        generated $k$-algebra. Now we only need to prove the statements for
        $\Spec A$. By Noether normalization, there exists $t_1, \ldots, t_d$
        such that $k[t_1, \ldots, t_d] \hookrightarrow A$ and $X = \Spec A \to
        \A_k^d$ is a finite morphism. Therefore $\dim X = \dim \A^d_k$, so we
        need to prove that $\dim \A^d = d$. Clearly we have $\dim \A^n \geq n$,
        so suppose we have a maximal chain \[ (0) = \mf{p}_0 \subsetneq
        \mf{p}_1 \subsetneq \cdots \subsetneq \mf{p}_m. \] Choose some nonzero
        $f \in \mf{p}_1$. Up to passing to an irreducible factor, we may assume
        that $f$ is irreducible. Therefore we can replace $\mf{p}_1$ with
        $(f)$. Now we may consider $X = \Spec k[x_1, \ldots, x_n]/(f)$, and
        then its fraction field has transcendence degree $n-1$, so by induction
        we may assume that $\dim X = n-1$.

            Now we will prove that $X$ is catenary. It suffices to prove that
            if $Z \subsetneq X = \Spec A$ is a maximal proper closed
            irreducible subset, then $\dim Z = n-1$. If $X \xrightarrow{\pi}
            \A^n_k$ is the morphism obtained from Noether normalization, then
            we consider $\pi(Z) \subsetneq \A^n$. By going-down, we know
            $\pi(Z)$ is maximal, so we have now reduced to proving the
            statement for $\A^n$. But then $W = V(\mf{Q})$ for some prime ideal
            $\mf{Q}$, so let $f \in \mf{Q}$ be nonzero. If $g$ is an
            irreducible factor, then maximality of $W$ implies that $\mf{Q} =
            (g)$. But then the desired statement about dimension is simply a
            statement about the transcendance degree of $k[x_1, \ldots,
            x_n]/f$.  \item We can reduce to the case where $X = \Spec A$. Then
            we know $\dim X = \dim A = \dim A_{\mf{m}}$ for any maximal ideal
            $x = \mf{m} \in \Spec A$ because $X$ is catenary.  \item Let $y \in
            Y$ satisfy $f(y) = \eta_X$ and set $Z \coloneqq \ol{\qty{y}}$.
            Therefore $Z \to X$ is dominant, so $\eta_Z \to \eta_X$, so we have
            an extension $k(\eta_X) \subseteq k(\eta_Z)$ and therefore $\dim X
            \leq \dim Z \leq \dim Y$.  \item We reduce to the affine case. Up
            to passing to the closure of the image, we may assume $f$ is
            dominant. Then we have $f \colon \Spec A \to \Spec B$ and then
            $f^{-1}(\eta_Y)$ is a finite set, and in fact is a $0$-dimensional
            scheme of finite type over $k(\eta_Y)$. In particular, $\eta_X$ is
            a closed point of $\eta_Y$, and thus the field extension $k(\eta_Y)
            \subseteq \eta_X$ is finite. This means that $\dim Y \leq \dim X$.
            \qedhere \end{enumerate} \end{proof}

Here is a general statement about unique factorization domains $A$ with $\Spec
A = X$. If $Z \subseteq X$ is a closed subset all of whose components have
codimension $1$, then $Z = V(f)$ for some $f \in A$. The converse also holds.
If we remove the assumption of unique factorization, then the converse is still
true by Krull's principal ideal theorem. 

To prove the statement for a Noetherian UFD, then there are finitely many
irreducible components $Z_1, \ldots, Z_s$ and if $Z_i = V(f_i)$, then $Z =
V\qty(\prod f_i)$. Therefore we can assume $Z$ is irreducible, but then every
height $1$ prime ideal is principal.

In the other direction, we may assume that $f$ is irreducible. If there exists
a prime ideal $0 \subseteq \mf{p} \subseteq (f)$, then we can assume that
$\mr{ht}(\mf{p}) = 1$ by the Noetherian assumption. But then we know $p$ is
principal, so $\mf{p} = (f')$, and this implies that $f \mid f'$ and thus $(f)
= (f')$. Thus $\mr{ht}((f)) = 1$.

\begin{rmk} A Noetherian domain $A$ is a UFD if and only if every prime ideal
of height $1$ is principal.  \end{rmk}

\begin{rmk} Let $Q = V(xy = z^2) \subset \A^3_k$. Then $V(z)$ has two
irreducible components, and each line cannot be cut out by a principal ideal.
\end{rmk}

\begin{thm}[Krull principal ideal theorem] Let $A$ be Noetherian and $f \in A$
    be nonzero. Let $\mf{p} \ni f$ be a minimal prime ideal containing $f$.
    Then $\mr{ht}(\mf{p}) \leq 1$. In fact, if $f$ is not a zero divisor, then
    $\mr{ht}(\mf{p}) = 1$.  \end{thm}

\begin{rmk} If $X$ is locally of finite type, we can prove Krull by Noether
normalization and the UFD property.  \end{rmk}

\begin{thm} Let $X$ be locally Noetherian and $f \in \Gamma(X, \mc{O}_X)$. Then
every irreducible component of $V(f)$ has codimension $0$ or $1$.  \end{thm}

\begin{cor} Let $X$ be locally Noetherian and $f_1, \ldots, f_r \in \Gamma(X,
\mc{O}_X)$. Then every irreducible component of $V(f_1, \ldots, f_r)$ has
codimsion at most $r$.  \end{cor}

\begin{proof} We may assume that $X = \Spec A$ is affine with $A$ Noetherian.
    We will induct on $r$. When $r = 1$, then this is just Krull's principal
    ideal theorem. Now consider $V(f_1, \ldots, f_{r-1}) \supseteq V(f_1,
    \ldots, f_r)$ and now let $Z \subseteq V(f_1, \ldots, f_r)$ be an
    irreducible component. Then let $W \subseteq V(f_1, \ldots, f_{r-1})$ be an
    irreducible component containing $Z$. By induction, we know $W \cap V(f_r)
    \supseteq Z$. Then every irrreducible component of $W \cap V(f_r)$ has
    codimension $0$ or $1$ in $W$, so it has codimension at most $r$ in $X$.
    Therefore, $\codim_X (Z) \leq r$.  \end{proof}

\section{Separated Morphisms}% \label{sec:separated_morphisms}

\begin{defn} A topological space $X$ is \textit{Hausdorff} if $\Delta \subseteq
X \times X$ is closed.  \end{defn}

\begin{rmk} A scheme $X$ is almost never Hausdorff.  \end{rmk}

\begin{thm} In the topological definition, we are taking the product topology
on $X \times X$ and set-theoretically we have the Cartesian product.  \end{thm}

For schemes, we can consider the fiber product $X \times_S X$ endowed with the
Zariski topology as a scheme.  \begin{defn} Let $\pi \colon X \to S$ be a
    morphism. Then $\pi$ is \textit{separated} if the morphism $\delta_{\pi}
    \colon X \times_S X$ determined by the identity on each copy of $X$ is a
    closed immersion.  \begin{equation*} \begin{tikzcd} X \ar[bend
    left=20]{drr}{\mr{id}} \ar[bend right=20]{ddr}{\mr{id}}
\ar{dr}{\delta_{\pi}} \\ & X \times_S X \ar{r} \ar{d} & X \ar{d} \\ & X \ar{r}
                         & S \end{tikzcd} \end{equation*} A scheme $X$ is
                         \textit{separated} if $X \to \Spec \Z$ is separated,
                         and $\pi \colon X \to S$ is called
                         \textit{quasi-separated} if $\delta_{\pi}$ is
                         quasi-compact.  \end{defn}

\begin{defn} A morphism $f \colon X \to Y$ is a \textit{locally closed
    immersion} if $f$ is a closed immersion into some open subset $U \subseteq
    Y$. Equivalently, $f$ is a homeomorphism onto a locally closed subset of
    $Y$ and $\mc{O}_{Y, f(x)} \to \mc{O}_{X,x}$ is surjective.  \end{defn}

\begin{exm} Any morphism $\pi \colon X = \Spec A \to \Spec B = Y$ is separated.
Here, we note that $X \times_Y X = \Spec A \otimes_B A$, and the natural
morphism $A \otimes_B A \to A$ is surjective.  \end{exm}

\begin{prop} Let $\pi \colon X \to S$ be a morphism of schemes. Then
$\delta_{\pi} \colon X \to X \times_S X$ is always a locally closed immersion.
\end{prop}

As a consequence, $\pi \colon X \to S$ is separated if and only if
$\delta_{\pi}(X) \subseteq X \times_S X$ is closed.

\begin{rmk} Note that if $\pi$ is separated, then it is quasi-separated.
\end{rmk}

\begin{proof}[Proof of Proposition] Let $S = \bigcup V_{\beta}$ be a cover by
    open affines and $X = \bigcup U_{\alpha\beta}$ be a cover by open affines
    such that $f(U_{\alpha\beta}) \subseteq V_{\beta}$. Write $V_{\beta} =
    \Spec B_{\beta}$ and $U_{\alpha\beta} = \Spec A_{\alpha\beta}$. Now we set
    \[ \mc{U} = \bigcup U_{\alpha\beta} \times_{V_{\beta}} U_{\alpha\beta} =
    \bigcup \Spec A_{\alpha\beta} \otimes_{B_{\beta}} A_{\alpha\beta}. \] Note
    that $X \to \mc{U}$ is closed because $\Spec A_{\alpha\beta} \to \Spec
    A_{\alpha\beta} \otimes_{B_{\beta}} A_{\alpha\beta}$ is a closed immersion.
\end{proof}

\begin{prop} Affine morphisms are separated because morphisms of affine schemes
are separated.  \end{prop}

\begin{exm} Being separated is local on the target.  \end{exm}

\begin{cor} Closed immersions are separated.  \end{cor}

\begin{exm} Open immersions are separated.  \end{exm}

\begin{lem} If $\pi \colon X \to S$ is separated, then for all open affines $U,
    V \subseteq X$ that map to a common affine open subset of $S$, then $U \cap
    V$ is affine and $\mc{O}_X(U) \otimes_{\Z} \mc{O}_X(V) \twoheadrightarrow
    \mc{O}_X(U \cap V)$.  \end{lem}

\begin{exm} We have $U \cap V = U \times_S V \cap \delta_{\pi}(X) =
\delta_{\pi}^{-1}(U \times_S V)$.  \end{exm}

\begin{proof} We can assume that $S = \Spec R$ is affine. Write $U = \Spec A, V
    = \Spec B$. Then we know that $U \cap V \to U \times_S V = \Spec A
    \otimes_R B$ is a closed immersion. Therefore, we have a surjection $A
    \otimes_{\Z} B \twoheadrightarrow A \otimes_R B \twoheadrightarrow
    \mc{O}_X(U \cap V)$.  \end{proof}

\begin{prop} Let $S = \Spec R$. Then $\pi \colon X \to S$ is separated if and
    only if for all $U, V \subseteq X$ open affines, $U \cap V$ is affine and
    $\mc{O}_X(U) \otimes_R \mc{O}_X(V) \to \mc{O}_X(U \cap V)$ is surjective.
    Equivalently, there exists a covering of $X$ by open affines such that the
    conditions hold.  \end{prop}

\begin{cor} Let $S = \Spec R$. Then $\P^n_S \to S$ is separated.  \end{cor}

Recall that closed immersions are preserved under base change. This follows
from stability of affine morphisms and the fact that the tensor product is
right exact. 

\begin{prop} Being separated and quasi-separated are preserved under base
change.  \end{prop}

\begin{proof} Let $f \colon Y' \to Y$ be separated. Let $X \to Y$ be a morphism
    and $X' \to X$ be the base change of $f$. Now consider the Cartesian
    diagram \begin{equation*} \begin{tikzcd} X' \ar{r}{\Delta} \ar{d} & X'
        \times_X X' \ar{d} \\ Y' \ar{r} & Y' \times_Y Y'.  \end{tikzcd}
    \end{equation*} Because the bottom arrow is a closed immersion, so is the
top arrow. The proof for the quasi-separated case is similar.  \end{proof}

\begin{prop} Being separated (or quasi-separated) is closed under composition.
If $f \colon X \to Y, g \colon Y \to Z$ are separated, then so is $h = g \circ
f$.  \end{prop}

\begin{proof} Consider $X_i \xrightarrow{f_i} Y \to Z$ for $i = 1,2$. Then we
    want to show that the diagram \begin{equation*} \begin{tikzcd} X_1 \times_Y
        X_2 \ar{d}{\ep} \ar{r}{\gamma} & X_1 \times_Z X_2 \\ Y \ar{r}{\Delta} &
        Y \times_Z Y \end{tikzcd} \end{equation*} is Cartesian. Note that $\ep$
        is the morphism $X_1 \times_Y X_2 \rightrightarrows X_i \to Y$ and
        $\gamma$ is constructed using the universal property for $X_1 \times_Z
        X_2 \rightrightarrows X_i \to Z$. Now consider a scheme $T$ with
        diagram \begin{equation*} \begin{tikzcd} T \ar{drr}{\varphi}
        \ar{ddr}{\psi} \\ & X_1 \times_Y X_2 \ar{r}{\gamma} \ar{d}{\ep} & X_1
    \times_Z X_2 \ar{d} \\ & Y \ar{r}{\Delta} & Y \times_Z Y.  \end{tikzcd}
\end{equation*} Now a map $\varphi \colon T \to X_1 \times_Z X_2$ is given by
$\varphi = (\varphi_1, \varphi_2)$, and we know that $( f_1 \circ \varphi, f_2
\circ \varphi_2 ) = (\psi, \psi)$. But now the universal property of the fiber
product gives us the desired result.  \end{proof}

\begin{exer} Open immersions are separated.  \end{exer}

\begin{cor} Quasi-projective schemes are separated.  \end{cor}

\begin{proof}[Proof of Proposition] The square in the following diagram is
    Cartesian \begin{equation*} \begin{tikzcd} X \ar{r}{\Delta} & X \times_Y X
    \ar{d} \ar{r} & X \times_Z X \ar{d}{(f,f)} \\ & Y \ar{r}{\Delta} & Y
\times_Z Y \end{tikzcd} \end{equation*} and thus all horizontal arrows are
closed immersions.  \end{proof}

\begin{prop} Let $f \colon X \to Y$ be quasi-compact and quasi-separated and
$\mc{F}$ be a quasicoherent sheaf on $X$. Then $f_* \mc{F}$ is quasicoherent.
\end{prop}

Consider a morphism $f \colon X \to Y$ of $S$-schemes. Then we define the
\textit{graph morphism} $\Gamma_f \colon X \to X \times_F Y$.

\begin{prop} Let $f \colon X \to Y$ be a morphism of $S$-schemes.
\begin{enumerate} \item $\Gamma_f \colon X \to X \times_S Y$ is locally closed.
\item If $Y \to S$ is separated, then $\Gamma_f \colon X \times_S Y$ is closed.
\end{enumerate} \end{prop}

\begin{proof} Consider the Cartesian diagram \begin{equation*} \begin{tikzcd} X
\ar{r}{\Gamma_f} \ar{d} & X \times_S Y \ar{d}{(f, \mr{id})} \\ Y \ar{r}{\Delta}
                        & Y \times_S Y.  \end{tikzcd} \end{equation*} Because
                        $\Delta$ is a (locally) closed immersion, so is
                        $\Gamma_f$.  \end{proof}

Now let $f,g \rightrightarrows Y$ be two morphisms of $S$-schemes. We define
the \textit{equalizer} to be the limit of this diagram if it exists. We need a
scheme structure on the equalizer, and in the diagram \begin{equation*}
    \begin{tikzcd} X \ar{r}{(f,g)} & Y \times_S Y \\ {(f,g)}^{-1}(\Delta_Y)
    \ar{r} \ar[hookrightarrow]{u} & Y, \ar{u}{\Delta} \end{tikzcd}
    \end{equation*} we see that the equalizer is simply $Y \times_{(Y \times_S
    Y)} X \subseteq X$. This is a locally closed subscheme. Proving that this
    is actually the equalizer is simply and we simply note that $(f \circ \rho,
    g \circ \rho)$ factors through $Y \xrightarrow{\Delta} Y \times_S Y$ and
    therefore factors through the fiber product $Y \times_{(Y \times_S Y)} X$.

\begin{exm} Consider the morphisms $f,g \colon \A^1 \rightrightarrows \A^1$
    given by $x \mapsto 0, x \mapsto x^2$. Then the diagrams of (schemes,
    rings) are \begin{equation*} \begin{tikzcd} \mr{Eq} \ar{r} \ar{d} & \A^1
        \ar{d}{\Delta} & \mr{Eq} & k[t] \ar{l} \\ \A^1 \ar{r}{(f,g)} & \A^2 &
    k[t] & \k[x,y] \ar{l}{\overset{x \mapsto 0}{y \mapsto t^2}} \ar{u}{x,y
\mapsto t} \end{tikzcd} \end{equation*} and it is easy to see that $\mr{Eq} =
\Spec k[t]/t^2$.  \end{exm}

\begin{cor} Let $f,g \colon X \rightrightarrows Y$ be a morphism and $X$ be
reduced and $Y$ separated. Suppose there exists a dense open $U \subseteq X$
such that $\eval{f}_U = \eval{g}_U$. Then $f = g$ everywhere on $X$.  \end{cor}

\begin{proof} We know $U \subseteq \mr{Eq}(f,g)$ is closed because $Y$ is
separated, and thus $\mr{Eq}(f,g) = X$.  \end{proof}

\begin{exm} The affine line with two origins is \textbf{not} separated! Note
that the inclusions of the two copies of $\A^1$ coincide on $\A^1 \setminus
\qty{0}$, but globally are different morphisms.  \end{exm}

\begin{exm} Consider the maps $\Spec k[x]/x^2 \rightrightarrows \Spec k[x]/x^2$
given by the identity and killing the maximal ideal. These agree
set-theoretically but are different as morphisms of schemes.  \end{exm}

We return to rational functions on integral schemes of finite type over an
algebraically closed field $k$. More generally, if $X$ is a reduced locally
Noetherian scheme, we say that \begin{defn} A \textit{rational function} on $X$
    is an \textbf{equivalence class} of $(U, f)$, where $U \subseteq X$ is
    dense and open and $f \in \mc{O}_X(U)$. We declare $(U, f) \sim (U', f')$
    if $f,f'$ agree on $U \cap U'$.  \end{defn}

\begin{defn} We say $f$ is \textit{regular} at $x \in X$ if there exists a
representative $(U, f)$ such that $x \in U$.  \end{defn}

\begin{lem} If $\qty{(U, f)}$ is a rational function, there is a
\textbf{maximal} open subset of regular points. This is called the
\textit{domain of definition}.  \end{lem}

Proof of this is simply the sheaf axioms. Now the set of rational functions on
$X$ form a ring. If $X = \Spec A$, this is the total ring of fractions of $A$.
If $X$ is integral, this is $k(\eta_X)$.

\begin{exm} On $X = \Spec k[x,y]/xy$, we see that $\frac{1}{(x-1)}{(y-3)}$ is a
rational function but $\frac{1}{x(y-3)}$ is not.  \end{exm}

Now let $X$ be reduced and $Y$ be a scheme.  \begin{defn} A \textit{rational
    map} $f \colon X \dashrightarrow Y$ is an \textbf{equivalence class} of
    pairs ${(U, f)}$ where $U \subseteq X$ is open and dense and $f \colon U
    \to Y$ is a morphism. We declare $(U, f) \sim (U', f')$ when there exists
    $V \subseteq U \cap U'$ open and dense such that $f,f'$ agree on $V$.
\end{defn}

In particular, if $Y$ is separated, then we can take $V = U \cap U'$.

\begin{exm} Consider $\P^n \dashrightarrow \P^{n-1}$ given by projection from a
point. This sends $[x_0, \ldots, x_n] \mapsto [x_1, \ldots, x_n]$ and is
defined everywhere except $[1,0,\ldots,0]$.  \end{exm}

\begin{exm} Consider the map $\P^2 \dashrightarrow \P^2$ given by $[x,y,z]
    \mapsto \qty[\frac{1}{x}, \frac{1}{y}, \frac{1}{z}]$. This is called the
    \textit{Cremona transformation} and is defined everywhere besides $[1,0,0],
    [0,1,0], [0,0,1]$.  \end{exm}

\begin{exm} The graph of the rational map $\P^2 \dashrightarrow \P^1$ given by
projection from a point is simply the blowup of $\P^2$ in a point.  \end{exm}

\begin{lem} The set of regular points of a rational map is open. If $Y$ is
separated, then there exists a morphism $f \colon U^{\mr{reg}} \to Y$
representing the rational map.  \end{lem}

\begin{defn} Let $f \colon X \dashrightarrow Y$ be a rational map over $S$ with
    $Y$ separated. Then let $U \subseteq X$ be the set of regular points. Then
    the \textit{graph} of $f$ is the closed subscheme $\Gamma_f \subseteq X
    \times_S Y$ given by the closure of $\eval{f}_U$. In fact, the graph is
    independent of the dense open subset $U$ chosen.  \end{defn}

\begin{defn} A rational map $f \colon X \dashrightarrow Y$ over $S$ is called
\textit{dominant} if there exists a representative $(U, f)$ such that $f \colon
U \to Y$ is dominant.  \end{defn}

\begin{exm} The map $\P^n \to \A^n$ is dominant (choose a distinguished open
subset, then take the identity to $\A^1$).  \end{exm}

\begin{defn} Let $X,Y$ be reduced. A rational map $f \colon X \dashrightarrow
    Y$ is called \textit{birational} if it is dominant and there exists a
    rational dominant map $g \colon Y \dashrightarrow X$ that is inverse to $f$
    as rational maps.  \end{defn}

\begin{prop} Let $X, Y$ be reduced and $f \colon X \dashrightarrow Y$ over $S$
be birational. Then there exist dense open subsets $U \subseteq X, V \subseteq
Y$ such that $\eval{f}_U \colon U \to V$ is an isomorphism.  \end{prop}

\begin{proof} Let $g \colon Y \dashrightarrow X$ be the inverse. Then we may
    assume that $U, V$ are affine. Next, if we write $Z = Y \setminus V$, then
    we can replace $U$ by an open affine in $U \setminus f^{-1}(Z)$. Thus we
    can assume $X,Y$ are affine. Call $U' \coloneqq f^{-1}(V)$, Then $U'
    \xrightarrow{f} V \xrightarrow{g} X$. This means that $g \circ f$ is the
    inclusion of $U'$ in $V$. Now clearly we may replace $V$ with $g^{-1}(U')
    \eqqcolon V'$, and now $f,g$ are inverse on $U', V'$.  \end{proof}

\begin{exm} Let $f \colon X \dashrightarrow Y$ be a rational map. Then the
projection $\pi \colon \Gamma_f \to X$ is birational, and the maximal open
subset where $\rho = \pi^{-1}$ is defined the domain of definition of $f$.
\end{exm}

\begin{exm} Consider $\A^2 \dashrightarrow \P^1$ given by $(x,y) \mapsto
[x,y]$. Then the graph is the blowup $\Bl_0 \A^2$ of $\A^2$ at the origin.
\end{exm}

\section{Proper Morphisms}% \label{sec:proper_morphisms}

\begin{defn} A morphism $f \colon X \to Y$ is called \textit{proper} if $f$ is
of finite type, separated, and universally closed (closed and being closed is
preserved under base change).  \end{defn}

\begin{exm} Closed immersions are proper. Clearly they are finite type (because
they are affine of the form $A \to A/I$), separated, and clearly universally
closed because closed immersions are closed and stable under base change.
\end{exm}

\begin{exm} The map $\A^1 \to \Spec k$ is closed, but not universally closed.
For example, the map $\A^2 = \A^1 \times_k \A^1 \to \A^1$ is not closed (take
the closed subset defined by $xy=1$).  \end{exm}

\begin{exm} Let $f \colon X \to Y$ be an integral morphism. Then $f$ is affine
    and thus separated. Also, integral morphisms are closed and stable under
    base change, so $f$ is universally closed. Then $f$ is of finite type if
    and only if it is finite, so finite morphisms are proper.  \end{exm}

\begin{prop} Being proper is stable under base change, stable under
composition, and local on the target.  \end{prop}

\begin{rmk} Let $\mc{P}$ be a property of a morphism of schemes that is stable
    under base change and composition. Suppose closed immersions satisfy
    $\mc{P}$. Then for all morphisms $f \colon X \to Y$, if $X \to X$ has
    $\mc{P}$ and $Y \to X$ is separated, then $f$ has $\mc{P}$.  \end{rmk}

\begin{proof} Note that $f \colon X \xrightarrow{\Gamma_f} X \times_S Y \to Y$.
Then $X \times_S Y \to Y$ has $\mc{P}$, and $\Gamma_f$ is a closed immersion
(because $Y$ is separated) and thus has $\mc{P}$, so $f$ has $\mc{P}$.
\end{proof}

\begin{rmk} The property $\mc{P}$ can be taken to be proper, separated, closed
immersion, etc.  \end{rmk}

\begin{prop}[Image of proper to separated is proper] Let $f \colon X \to Y$ be
a surjective morphism and suppose $X$ is proper and $Y$ is separated of finite
type. Then $Y$ is proper.  \end{prop}

\begin{proof} We only need to check that $Y \to S$ is universally closed. We
    show that $Y \to S$ is closed. Because $f$ is surjective, then for $Z
    \subseteq Y$, we know $Z = f(f^{-1}(Z))$, so the image of $Z$ in $S$ is
    closed. To show that $f$ is universally closed, we simply base change the
    entire diagram.  \end{proof}

\begin{rmk} We can eliminate the condition that $f$ is surjective by replacing
$Y$ with the scheme-theoretic image of $f$.  \end{rmk}

\begin{prop} Let $X$ be a reduced scheme and $f \colon X \to Y$ be a morphism.
Then the scheme theoretic image coincides with the closure of the set-theoretic
image.  \end{prop}

\begin{prop} Let $X$ be a proper connected reduced scheme over a field $k$.
Then $k \subseteq \Gamma(X, \mc{O}_X)$ is integral. If $k$ is algebraically
closed, then $\Gamma(X, \mc{O}_X) = k$.  \end{prop}

\begin{proof} Let $f \in \Gamma(X, \mc{O}_X)$. Then we can view $f \colon X \to
    \A^1$. This is proper, so $f(X)$ is a closed connected reduced subscheme of
    $\A^1$. Thus it suffices to show that $f(X) \neq \A^1$, but this is simply
    because $\A^1$ is not proper (despite being separated of finite type). Thus
    $f(X)$ must be a closed point, so the map $k[t] \to \Gamma(X, \mc{O}_X)$
    factors through $k(f(X))$. But then $f(X) = V(p(t))$ for some irreducible
    polynomial $p$, so $f$ also satisfies $p$.  \end{proof}

\begin{thm} Let $R$ be a ring. Then $\P^n_R \to \Spec R$ is proper.  \end{thm}

\begin{proof} We already know $\P^n$ is finite type and separated, so we need
    to show it is universally closed. Therefore it suffices to prove that
    $\P^n_{\Z}$ is universally closed (because being proper is preserved by
    base change). Now we need to show that $\P^n \times_{\Z} X \to X$ is closed
    for all schemes $X$. In fact, we can check this locally, so we only need to
    show that $\P^n_{A} \to \Spec A$ is closed.  \end{proof}

\section{Proj construction}% \label{sec:proj_construction}

We want to think of $\P^n$ as $\Proj(k[x_0, \ldots, x_n])$ under some
definition of $\Proj$. This construction should work for any graded ring $R =
\bigoplus_{i \geq 0} R_i$. Then there is an \textit{irrelevant ideal} $R_+ =
\bigoplus_{i > 0} R_i$. 

We will define $\Proj R$ as a \textbf{set} as the set of all graded prime
ideals $\mf{p} \subseteq R$ such that $\mf{p} \not\supseteq R_+$. Now we want
to consider $\Proj R$ has a \textbf{topological space}, and we may consider the
Zariski topology as in the case of $\Spec R$. 

\begin{lem} Let $\mf{a} \subseteq R$ be a homogeneous ideal. Then
$\sqrt{\mf{a}}$ is the intersection of all homogeneous prime ideals containing
$\mf{a}$ and $V(\mf{a}) = \emptyset$ if and only if $\sqrt{\mf{a}} = R_+$.
\end{lem}

Finally we are ready to define $\Proj R$ as a \textbf{locally ringed space}.
For any $f \in R$, define $U_f \coloneqq {(\Proj R)}_f = \Proj R \setminus
V(f)$. 

\begin{lem} The sets $\qty{U_f}_{f \in R_k, k \geq 1}$ form a basis for the
topology of $\Proj R$.  \end{lem}

\begin{rmk}\leavevmode We have an isomorphism $U_f = { (\Proj R) }_f \simeq
    \Spec R_{(f)}$, where $R_{(f)}$ is the degree zero localization of $R$ at
    $f$. Here, the maps are given by \[ f \notin \mf{p} \mapsto \mf{p} R_f \cap
    R_{(f)} \] and \[ \Spec R_{(f)} \ni \mf{Q} \mapsto \bigoplus \qty{x \in R_k
\mid \frac{x^{\deg f}}{f^k} \in \mf{Q}}. \] \end{rmk}

Finally we are able to describe the structure sheaf on each open subset. Here,
we simply define $\mc{O}_{\Proj R}({(\Proj R)}_f) = R_{(f)}$, and so we need to
check the gluing axioms.  \begin{enumerate} \item Let $f, g_i \in R_+$ be
    homogeneous and suppose ${(\Proj R)}_f = \bigcup {(\Proj R)}_{g_i}$. This
    is the same as $f \in \sqrt{\sum g_i R}$, so there exists $n$ such that
    $f^n = \sum a_i g_i$ for some $a_i \in R$.  \item If ${(\Proj R)}_f
    \subseteq {(\Proj R)}_g$, then $f^n = a \cdot g$ and thus there exists a
    canonical map $R_{(g)} \to R_{(f)}$.  \item We have $\bigcup {(\Proj
    R)}_{g_i} = \Proj R$ if and only if for all homogeneous primes $\mf{p}$ not
    containing the irrelevant ideal, there exists $g_i \notin \mf{p}$. This is
    equivalent to $V\qty(\sum g_i R) = \emptyset$, which is equivalent to
    $\sqrt{\sum g_i R} = R_+$.  \end{enumerate}

Now we not only obtain the structure sheaf but also a sheaf $\wt{M}$ associated
to any graded $R$-module $M = \bigoplus_{k \geq 0} M_k$. In particular, $\Proj
R$ is a scheme. If $R = R_0 \oplus R_+$, then there is of course a natural
continuous map $\Proj R \to \Spec R_0$. Note that we have a commutative diagram
\begin{equation*} \begin{tikzcd} R \ar{r} & R_f \\ R_0 \ar{r} \ar{u} & R_{(f)}
\ar[hookrightarrow]{u} \end{tikzcd} \end{equation*}

\begin{rmk} If $R$ is finitely generated as an $R_0$-algebra, then $\pi \colon
\Proj R \to \Spec R_0$ is of finite type.  \end{rmk}

\begin{prop} The map $\pi \colon \Proj R \to \Spec R_0$ is separated.
\end{prop}

\begin{proof} Apply the criterion that if $U, V \subseteq X$ is open affine,
    then $U \cap V$ is affine and $\mc{O}_X(U) \otimes_R \mc{O}_X(V) \to
    \mc{O}_X(U \cap V)$ is surjective implies that $f$ is separated. Now apply
    this to distinguished open subsets.  \end{proof}

We would like to define functoriality for this construction. Let $R, S$ be
graded rings and $\varphi \colon S \to R$ be a morphism of graded rings. Can we
define a map $\Proj R \to \Proj S$?

\begin{exm} Let $S = k[x_1, \ldots, x_n] \overset{\varphi}{ \hookrightarrow } R
    = k[x_0, \ldots, x_n]$. Then $\Proj R = \P^n$ and $\Proj S = \P^{n-1}$.
    This is not defined globally because $(x_1, \ldots, x_n) \cap S = S_+$ is
    the irrelevant ideal. Instead, we consider $V(\varphi(S_+)) = Z$, and then
    we obtain a morphism $\psi \colon \Proj R \setminus Z \to \Proj S$. In
    fact, here we obtain projection from a point.  \end{exm}

\begin{rmk} $\psi$ is an affine morphism.  \end{rmk}

\begin{exm} If $\varphi(S_+) = R_+$, then $Z \neq \emptyset$, so we have a
morphism $\Proj R \to \Proj S$.  \end{exm}

\begin{exm} For example, if $\varphi$ is surjective, then $\ker \varphi$ is a
homogeneous ideal and in fact we have $\Proj R = \Proj S/\ker\varphi \subseteq
\Proj S$, and the last inclusion is a closed immersion.  \end{exm}

\begin{exm} The morphism $k[x_0, \ldots, x_n] \xrightarrow{\varphi} k[x,t] = R$
given by  $x_i \mapsto s^{n-i}t^i$ defines the Veronese embedding.  \end{exm}

\begin{exm} Suppose $R$ is finitely generated as an $R_0$-algebra by finitely
many elements in degree $1$. Then we have a surjection $R_0[x_1, \ldots, x_n]
\twoheadrightarrow R$, and thus $\Proj R$ is a closed subscheme of
$\P^n_{R_0}$.  \end{exm}

Now we would like to consider what happens under base change. Let $R = R_0
\oplus R_+$ and consider a morphism $S_0 \to R_0$. Then we can consider the
scheme $\Proj R \times_{R_0} \Spec S_0$. Alternatively, we may consider the
graded ring $R' = R \otimes_{R_0} S_0 = \bigoplus R_i \otimes_{R_0} S_0$. Of
course we have a morphism $\Proj R' \to \Proj R \times_{R_0} \Spec S_0$, and in
fact on open subsets, this defines an isomorphism.

As a corollary, let $X$ be a scheme and $\mc{R} = \bigoplus \mc{R}_i$ be a
sheaf of graded algebras with $\mc{R}_i$ suasicoherent for all $i \geq 0$. Then
there exists a scheme $\Proj_{\mc{O}_X}(\mc{R}) \to X$ such that over an affine
open $\Spec A$ where $\mc{R}_i = R_i$, we have $\Proj_A \bigoplus R_i \to \Spec
A$.

\begin{exm} If $\mc{F}$ is a finitely generated quasicoherent sheaf, then we
    may consider $\mc{R} = \bigoplus \mr{Sym}^n \mc{F}$. Then we write
    $\P_X(\mc{F}) \coloneqq \Proj_{\mc{O_X}}(\mc{R})$. In particular, if
    $\mc{F} = \mc{O}_X \otimes V$, we obtain $\P_X(V^{\vee})$.  \end{exm}

Now if $R$ is a graded ring, we can consider the ring $R(d) = \bigoplus_{k \geq
0} R_{kd}$, where $R_{kd}$ now has degree $k$. Then we have an inclusion $\Proj
R \to \Proj R(d)$. But then if $R_+ \not\subset \mf{p}$, we know that if
$\mf{p}_{kd} \supseteq R_{kd}$, we know that $R_k \subseteq \mf{p}$ for all $k
> 0$. In particular, we obtain an honest morphism $\psi \colon \Proj R \to
\Proj R(d)$. We can check on principal open subsets that $\psi$ is an
isomorphism.

\begin{cor} Let $R, R'$ be rings such that $R_0 = R_0'$ and there exists $N$
    such that \[ \bigoplus_{k \geq N} R_k \simeq \bigoplus_{k \geq N} R_k', \]
then $\Proj R \simeq \Proj R'$.  \end{cor}

\begin{proof} Replace $R, R'$ with $R(d), R'(d)$ for sufficiently large $d$.
\end{proof}

Now let $R, S$ be graded rings with $R_0 = S_0$. Now of course we have the
fiber product \begin{equation*} \begin{tikzcd} \Proj R \times_{R_0} \Proj S
\ar{r} \ar{d} & \Proj R \ar{d} \\ \Proj S \ar{r} & \Spec R_0.  \end{tikzcd}
\end{equation*} Then if $\R' = \bigoplus R_i \otimes_{R_0} S_i$, we can show
that $\Proj R' = \Proj R \times_{R_0} \Proj S$. 

\begin{thm} For any graded ring $R$, the morphism $\Proj R \to \Spec R_0$ is
closed.  \end{thm}

This follows from the fact that $\P^n \to \Spec R_0$ is universally closed and
$\Proj R \subseteq \P^n$ is a closed immersion.

\begin{exer} Up to taking $R(d)$ for an appropriate $d$, we can assume that $R$
is finitely generated in degree $1$.  \end{exer}

\begin{proof} We want to show that $\pi \colon \P^n_A \to \Spec A$ is closed.
    Let $Z = V(I)$. If $\mf{p}$ is such that $\P^n_{k(\mf{p})} \cap V(I) =
    \emptyset$, then there exists an open subset $U \ni \mf{p}$ of $\Spec A$
    such that $\pi^{-1}(U) \cap V(I) = \emptyset$. If $I = (f_1, \ldots, f_k)$,
    then $\ol{f}_1, \ldots, \ol{f}_k$ have no nontrivial solutions in
    $k(\mf{p})[x_0, \ldots, x_n]$. The idea here is to use Nakayama.
\end{proof}

We now want to consider sheaves on $X \coloneqq \Proj R$. We want a theory such
that $\mc{O}_X = \wt{R}$. More generally, for any graded $R$-module $M$, define
the sheaf $\wt{M}$ by \[ \wt{M}(X_f) = M_{(f)} = \M \otimes_R R_{(f)}. \] This
gives an exact functor, which is not faithful.

\begin{rmk} Compare this to the affine case where we had an equivalence of
categories.  \end{rmk}

\begin{rmk} The grading is important for this construction. We can construct
the sheaves ${(M(d))}_k = M_{d+k}$, and in particular, we have the sheaf
$\wt{R(d)} \eqqcolon \mc{O}_X(d)$.  \end{rmk}

\begin{prop} If $R$ is finitely generated in degree $1$, the sheaves
$\mc{O}_X(d)$ are locally free of rank $1$.  \end{prop}

\begin{proof} We know that $X = \bigcup_{f \in R_1} X_f$. For $f \in R_1$, we
    know $\wt{R(1)}(X_f) = {R(1)}_(f)$, which consists of elements of the form
    $\frac{h}{f^n}$, where $\deg h = 1+n$. This implies that multiplication by
    $f$ defines an isomorphism $R_{(f)} \to {R(1)}_{(f)}$.  \end{proof}

\begin{rmk} Note that multiplication gives a map $R \to R(1)$, which after
    sheafification gives us a map $\mc{O}_X \to \mc{O}_X(1)$. Combining these,
    we obtain a map $R_1 \to \Gamma(X, \mc{O}_X(1))$. However, we cannot say
    much about this map in general. All of this can be generalized to $R_d \to
    \Gamma(X, \mc{O}_X(d))$.  \end{rmk}

\begin{rmk} Consider $R = A[x_0, \ldots, x_n]$ and write $\Proj R = \P^n_A =
    \bigcup U_i$. We can write $\mc{O}_X(1)$ in terms of Cech cocycles. On each
    $U_i$, we have maps $\eval{\mc{O}_X}_{U_i} \xrightarrow{x_i}
    \eval{\mc{O}_X(1)}_{U_i}$. On overlaps, the transitions are given by
    $x_j/x_i$, so we obtain $\mc{O}_X(1)$ by gluing copies of $\mc{O}_X$ with
    these gluing functions.  \end{rmk}

\begin{prop}\leavevmode \begin{enumerate} \item For all graded modules $M$ and
    $d \in \Z$, $\wt{M(d)} \cong \wt{M} \otimes_{\mc{O}_X} \mc{O}_X(d)$.  \item
    For all $m,n$, $\mc{O}_X(n) \otimes_{\mc{O}_X} \mc{O}_X(m) \simeq
    \mc{O}_X(n+m)$. In particualr, when $m = -n$, we have ${\mc{O}_X(n)}^{\vee}
    = \mc{O}_X(-n)$.  \end{enumerate} \end{prop}

This is proved by considering the maps $M \otimes_R R(1) \to M(1)$. Similarly
to above, we also have maps $M_d \to \Gamma(X, \wt{M(d)})$ for all $d \in \Z$.

\begin{defn} For any sheaf $\mc{F}$ of $\mc{O}_X$-modules, define the graded
    module \[ \Gamma_*(\mc{F}) = \bigoplus_{d \in \Z} \Gamma(X, \mc{F}(d)), \]
    which is a graded $\Gamma_*(\mc{O}_X)$-module. For a graded module $M$, we
    obtain a module $\Gamma_*(\wt{M})$.  \end{defn}

\begin{prop} If $R = A[x_0, \ldots, x_n]$, then $R = \Gamma_*(\wt{R})$.
Therefore, for all $d \geq 0$, we have $R_d = {A[x_0,\ldots,x_n]}_d \simeq
\Gamma(\P^n_A, \mc{O}(d))$.  \end{prop}

\begin{prop} Let $\mc{F}$ be a quasicoherent sheaf on $X = \Proj R$. Then there
exists a natural isomorphism $\mc{F} = \wt{\Gamma_*(\mc{F})}$.  \end{prop}

\begin{cor} Any closed subscheme of $X = \Proj R$ is defined by some graded
ideal $I \subseteq R$.  \end{cor}

\begin{rmk} If we replace $R = R' = \bigoplus_{i \geq 0} R_{id}$, we know $X =
\Proj R \xrightarrow{\varphi}{\sim} \Proj R' = Y$, and $\varphi^*\mc{O}(1) =
\mc{O}(d)$.  \end{rmk}

\begin{rmk} Let $R \to S$ be a morphism of graded algebras of degree $0$. Then
we have a map $\Proj S \supseteq V \xrightarrow{\varphi} \Proj S$, and
$\varphi^*(\mc{O}(1)) = \mc{O}(1)$.  \end{rmk}

We may also define global versions of this on an arbitrary scheme $S$. Let
$\mc{R} = \bigoplus \mc{R}_i$ be a graded $\mc{O}_S$-algebra. We assume that
$\mc{R}$ is generated as an $\mc{R}_0$-algebra by $\mc{R}_1$. This gives us a
morphism $\Proj_S(\mc{R}) \eqqcolon X \xrightarrow{\pi} S$. Also, we have a
natural surjection $S^* \mc{R}_1 \twoheadrightarrow \mc{R}$, so we have an
embedding $X \hookrightarrow \P \mc{R}_1$. Now if $\mc{M}$ is a quasicoherent
graded $\mc{R}$-module, we obtain a module $\wt{\mc{M}}$ on $X$. If $\mc{M} =
\mc{R}(d)$, we have $\wt{\mc{R}(d)} \eqqcolon \mc{O}_{\pi}(d)$.

\begin{prop} If $\mc{E}$ is locally free over $S$, then $\P \mc{E} =
\Proj_S(S^* \mc{E}) \xrightarrow{\pi}$, and $\pi_* \mc{O}_{\P \mc{E}}(d) = S^d
\mc{E}$.  \end{prop}

\begin{rmk} If $\mc{L}$ is a line bundle, then $\P \mc{E} = \P (\mc{E} \otimes
\mc{L})$.  \end{rmk}

Now recall that $\Hom_{\ms{Sch}}(X, \P^n_{\Z})$ is in bijection with the set of
surjections $\mc{O}_X^{n+1} \twoheadrightarrow \mc{L}$ for line bundles
$\mc{L}$. Now if we consider $\Proj R$, where $R$ is finitely generated in
degree $1$, then we obtain a map $\mc{O}_{\P^n} \otimes R_1 \twoheadrightarrow
\mc{O}_{\P^n}(1)$, so we obtain a map $R_1 \to \Gamma(X, \mc{L})$. Now we want
to consider $\Hom_{S = \Spec R_0}(X, \Proj R)$ for arbitrary $R$.

\begin{thm} The set $\Hom_S(X, \Proj S^* R_1)$ is in bijection with the set of
    invertible sheaves $\mc{L}$ on $X$ equipped with a map $R_1
    \xrightarrow{\varphi} \Gamma(X, \mc{L})$ which globally generate $\mc{L}$.
    The bijection is given by $\mc{L} = f^*(\mc{O}(1))$ and $R_1 =
    \Gamma(\mc{O}(1)) \to \Gamma(X, f^*(\mc{O}(1)))$.  \end{thm}

\begin{proof} Given a line bundle $\mc{L}$ on $X$ and $\varphi \colon R_1 \to
    \Gamma(X, \mc{L})$ which globally generates $\mc{L}$, write $R_1 \ni f
    \mapsto s_f \in \Gamma(X, \mc{L})$. Then we know $X = \bigcup X_{s_f}$.
    Locally, we have the section $\mc{O}_X \xrightarrow{s} \mc{L}$, so
    $\eval{\mc{L}}_{X_s} = \eval{\mc{O}_X}_{X_s}$. To define $f \colon X \to
    \Proj R \eqqcolon \P$, we define it locally. In fact, we give morphism
    $X_{s_f} \to \P_f = \Spec R_{(f)}$. This is equivalent to giving a morphism
    $R_{(f)} \to \Gamma(X_{s_f}, \mc{O}_X)$. But this is simply given on $R_1$
    by $f \mapsto 1$, and so the map $S^*(R_1) \to \Gamma(X_{s_f}, \mc{O}_X)$
    factors through ${(S^* R_1)}_f$, and thus we have a map ${(S^* R_1)}_{(f)}
    \to \Gamma(X_{s_f}, \mc{O}_{X_{s_f}})$.  \end{proof}

\begin{rmk} The map $f$ is uniquely determined by $\mc{L}$ and $\varphi$, and
thus gluing is given by this uniqueness.  \end{rmk}

\begin{rmk} Given a surjection $S^* R_1 \twoheadrightarrow R$ which induces
    $\Proj R \hookrightarrow \Proj S^* R_1$, a map $X \to \Proj(S^* R_1)$
    factors through $\Proj R$ if and only if the map $R_1 \to \Gamma(X,
    \mc{L})$ satisfies the property that $S^* R_1 \to \Gamma_* \mc{L}$ factors
    through $R$.  \end{rmk}

\begin{thm} Let $f \colon X \to S$ be a morphism of schemes and $\mc{R} = S^*
    \mc{R}_1$ be a quasicoherent graded $\mc{O}_S$-algebra. Then $\Hom_S(X,
    \Proj_S \mc{R})$ is in bijection with the set of line bundles $\mc{L}$ on
    $X$ equipped with a surjection $\varphi \colon f^* \mc{R}_1
    \twoheadrightarrow \mc{L}$.  \end{thm}

\begin{cor} Let $\mc{E} \to X$ be locally free. Then sections of $\P \mc{E} \to
X$ are in bijection with surjections $\mc{E} \to \mc{L}$, where $\mc{L}$ is a
line bundle on $X$.  \end{cor}

\begin{exm} Let $X = \P^n_A$ and $f \colon \mc{O}_X \hookrightarrow
    \mc{O}_X(1)$ for $f \in { A[x_0, \ldots, x_n] }_1$. If we tensor with
    $\mc{O}_X(-1)$, we obtain a map $f \colon \mc{O}_X(-1) \hookrightarrow
    \mc{O}_X$. This realizes $\mc{O}_X(-1)$ as an ideal sheaf of $\mc{O}_X$
    associated to $f$. Of course, the scheme associated to this ideal sheaf is
    $V(f)$.  \end{exm}

\begin{exm} Let $X$ be locally factorial and $Z \subseteq X$ be a codimension
$1$ subscheme. Then the ideal sheaf $\mc{I}_Z$ is locally principal.  \end{exm}

Now let $X$ be a scheme, $\mc{L}$ be a line bundle, and $s_0, \ldots, s_n \in
\Gamma(X, \mc{L})$. This defines a rational map $X \dashrightarrow \P^n$. We
would like to put a scheme structure on $X \setminus U \eqqcolon Z$.
\begin{defn} The \textit{base locus} $\mr{Bs}(s_0, \ldots, s_n) \subseteq X$ is
    the closed subscheme defined by the following ideal sheaf. Write $V
    \coloneqq \ev{s_0, \ldots, s_n} \subseteq \Gamma(X, \mc{L})$. Then we
    obtain a morphism $V \otimes \mc{O}_S \xrightarrow{\mr{ev}} \mc{L}$.
    Tensoring with $\mc{L}^{-1}$, we obtain a map $V \otimes \mc{L}^{-1} \to
    \mc{O}_X$. Now we define the ideal sheaf $\mc{I}_V \subseteq \mc{O}_X$ to
    be the image of this morphism.  \end{defn}

We would like to now modify $X$ such that we can define an actual morphism $X
\to \P^n$. Before we do the construction, we give some examples.

\begin{exm} Suppose that the ideal sheaf $I$ is invertible and locally
    principal. If $\mc{L} = \mc{O}_{\P^n}(1)$ and $s_0 = x_1$, then the base
    locus is $V(x_1)$. Up to passing from $\mc{L}$ to $\mc{L} \otimes \mc{I}$,
    we may assume that the line bundle is globally generated by $V$, so our map
    extends.  \end{exm}

We are now ready to consider blowups of closed subschemes $Y \subseteq X$.
Here, $X$ is a scheme and $Y \subseteq X$ to be a closed subscheme with ideal
sheaf $\mc{I}$. Write $\mc{R} = \bigoplus_{n \geq 0} \mc{J}^n$

\begin{defn} The \textit{blowup} of $X$ along $Y$ is the scheme
$\Proj_X(\mc{R}) \xrightarrow{\pi} X$.  \end{defn}

Note that $\pi$ is proper. Now note that if $f \colon X \to Y$, and $\mc{I}
\subseteq \mc{O}_Y$ is an ideal sheaf, the map $f^* \mc{I} \to f^* \mc{O}_Y =
\mc{O}_X$ is not injective, but we can consider the image, which is an ideal
sheaf that we will call $f^{-1}\mc{I}$.

\begin{exm} Consider the map $\A^2 \to \A^2$ given by $k[s,t] \to k[x,y]$ given
    by $(s,t) \mapsto (x, xy)$. Then \[ f^*(s,t) = (s,t) \otimes_{k[s,t]}
    k[x,y] \to k[x,y] \] is not injective, because $f^{-1}(s,t) = (x)$.
\end{exm}

\begin{rmk} If $\mc{I}$ is quasicoherent, so it $f^{-1}\mc{I}$.  \end{rmk}

\begin{rmk} If $X \xrightarrow{f} Y$ and $\mc{I}_Z \subseteq \mc{O}_Y$ is the
ideal of a closed subscheme $Z$, then $f^{-1}(Z)$ has ideal $f^{-1} \mc{I}_Z$.
\end{rmk}

\begin{exm} We will consider the blowup of $0 \in \A^2_k$. Here, if $A
    \coloneqq k[x,y]$, then the blowup $X \subseteq \P^1_{\A^2}$ is defined by
    $(xT = yS)$, where $\P^1_{\A^2} = \Proj A[S,T]$. Now we want to check what
    $f^{-1}(x,y)$ is, and we can do this locally. If we consider the chart $U_S
    = \Spec k[x,y,t]$, then $X \cap U_S$ is given by $tx = y$. Thus $\eval{
    f^{-1}(x,y) }_{U_S} = (x)$. On the other hand, on the chart $U_T$, we have
    the equation $x = ys$, so $\eval{f^{-1}(x,y)}_{U_T} = (y)$. In fact, $X =
    \Bl_0 \A^2_k$. To see this, we know that \[ \Bl_0 \A^2 = \Proj R \qquad R =
    \bigoplus_{n \geq 0} {(x,y)}^n = \Proj k[x,y](S,T) / (xT = yS). \]
\end{exm}

\begin{thm} Let $X, Y, \mc{I}$ be as above. Then write $\Bl_Y X \eqqcolon
    \wt{X} \xrightarrow{\pi} X$ and set $U = X \setminus Y$.  \begin{enumerate}
        \item $\pi$ induces an isomorphism $\pi \colon \pi^{-1}(U) \to U$.
        \item The sheaf $f^{-1} \mc{I} \subseteq \mc{O}_{\wt{X}}$ is invertible
            and corresponds to $\mc{O}_{\wt{X}}(1)$.  \end{enumerate} \end{thm}

\begin{proof} On $X \setminus Y = U$, $\mc{I}$ is trivial, so $\eval{\mc{R}}_U
    = \bigoplus \mc{O}_U = \mc{O}_U[T]$, so we are done. This proves the first
    part. For the second part, note $\mc{J} \cdot \mc{R} = \bigoplus_{n \geq 1}
    \mc{J}^n = \mc{R}(1)$.  \end{proof}

\begin{rmk} If $\mc{I}$ is locally principal (thus locally trivial), then in
fact $\Bl_Y X = X$.  \end{rmk}

\begin{rmk} When we have $\P^n_A = \Proj A[x_0, \ldots, x_n]$, $\mc{O}(1)$ has
    sections ${A[x_0, \ldots, x_n]}_1$, and here $\mc{O}(-1) \subseteq \mc{O}$
    is an ideal sheaf. Here something was said about self intersections of
    exceptional divisors on surfaces (if you blow up a smooth point, you get a
    curve with self intersection $-1$).  \end{rmk}

We know that $f^{-1} \mc{I}$ is a locally principal ideal sheaf on $\wt{X}$. If
$E \subseteq \wt{X}$ is the subscheme defined by $f^{-1} \mc{I}$, then
$f^{-1}(Y) = E$. Because $\mc{I}_E$ is locally principal, then $\eval{E}_U =
(f_u)$ is a closed subscheme of codimension at most $1$ for any open subset $U
\subseteq \wt{X}$, and the codimension is $1$ if $f_u$ is not a zero divisor.
Later, we will see that $E$ has pure codimension $1$ and is a Cartier divisor,
called the \textit{exceptional divisor}. 

We will check this affine locally on $X$. We may assume that $X = \Spec A$ and
$R = \bigoplus I^n$. Then let $(x_1, \ldots, x_r) = I$, so $\wt{X} = \bigcup
U_{x_i}$. Now we have a map \[ \varphi \colon A[T_1, \ldots, T_r]
\twoheadrightarrow \bigoplus_{n \geq 0} I^n = R, \] and we always have the
relations $x_i T_j = x_j T_i$ for all $i,j$. On $U_{x_i} = \Spec R_{(x_i)}$, we
still have a morphism ${ A[T_1, \ldots, T_r] }_{(T_i)} \twoheadrightarrow
R_{(x_i)}$. Now we consider $I R_{(x_i)}$ and note that \[ R_{(x_i)} = A \oplus
I \cdot \frac{1}{x_i} \oplus I^2 \frac{1}{x_i^2} \oplus \cdots \] Because $x_j
= x_i \varphi \qty(\frac{T_j}{T_i})$, it follows that $I R_{(x_i)}$ is
generated by $x_i$ and is thus principal. Now we need to show that $x_i$ is not
a zero divisor, but this is clear by the localization process.

\begin{prop} If $X$ is integral, so is $\wt{X}$. Also, if $X \to S$ is
separated or proper, so is $\wt{X} \to S$. Finally, if $X \to S$ if $X$ is
Noetherian, then $\mc{I}$ is coherent, and $\wt{X}$ is also Noetherian.
\end{prop}

Returning to extending morphisms, let $X$ be integral and $\mc{L}$ be an
invertible sheaf. Then choose $s_0, \ldots, s_n \in \Gamma(X, \mc{L})$ and
write $V = \ev{s_0, \ldots, s_n} \subseteq \Gamma(X, \mc{L})$. Then let $Y$ be
the base locus and $\mc{I}_Y = [V \otimes \mc{L}^{-1} \twoheadrightarrow
\mc{I}_Y \subseteq \mc{O}_X]$ be the ideal sheaf. Then we would like to extend
the morphism $U \coloneqq X \setminus Y \to \P^n$. Last time, we saw that we
could extend the morphism if $\mc{I}_Y$ was an invertible sheaf. Now let
$\wt{X} = \Bl_Y X$ and $\pi \colon \wt{X} \to X$. But now we have a surjection
$V \otimes \pi^* \mc{L}^{-1} \twoheadrightarrow \pi^{-1} \mc{I}_Y$ onto an
invertible sheaf, so we obtain a surjection $V \otimes \mc{O}_{\wt{X}}
\twoheadrightarrow \pi^{-1} \mc{I}_Y \otimes \pi^* \mc{L}$, and so now we
obtain a regular morphism $\wt{X} \to \P^n_S$. Note that we also write
$\pi^{-1} \mc{I}_Y = \mc{O}_{\wt{X}}(-E)$.

\begin{exer} We have an identification of $\wt{X}$ with the graph of $X
\dashrightarrow \P^n$.  \end{exer}

\begin{thm}[Universal property of blowups] Let $X, Y, \mc{I}, \wt{X}
    \xrightarrow{\pi} X$ be as before. Then for all $f \colon Z \to X$ such
    that $f^{-1} \mc{I}_Y \subseteq \mc{O}_Z$ is an invertible sheaf, then
    there exists a unique $g \colon Z \to \Bl_Y X$ making the diagram
    \begin{equation*} \begin{tikzcd} Z \ar[dashrightarrow]{rr}{g} \ar{dr}{f} &
    & \Bl_Y X \ar{dl}{\pi} \\ & X \end{tikzcd} \end{equation*} commute.
    \end{thm}

\begin{proof} We use the characterization of morphisms to $\Proj R$. Here, a
    map $X \to \Proj S^* \mc{R}_1$ was an invertible sheaf $\mc{L}$ on $X$ and
    a surjection $\mc{h}^* \mc{R}_1 \to \mc{L}$. If $\mc{R}$ is generated by
    $\mc{R}_1$, then we need this surjection to factor through $\mc{h}^*
    \mc{R}$. 

    Now set $\mc{L} = f^{-1} \mc{I}_Y$. By definition, there exists a
    surjective morphism $f^* \mc{I}_Y \to \mc{L}$, so of course we obtain a
    morphism \[ S^* f^* \mc{I}_Y \to \mc{R} \to \mc{L}. \] This gives us the
    morphism $Z \to \wt{X} = \Proj \mc{R} \to X$. The proof of uniqueness is
    omitted.  \end{proof}

\begin{cor} Consider $\pi \colon \Bl_Y X \eqqcolon \wt{X} \to X$. Then for all
$f \colon Z \to X$, let $\wt{Z} \to Z$ be the blowup of $Z$ along $f^{-1}
\mc{I}_Y$, then there exists $\wt{f} \colon \wt{Z} \to \wt{X}$ lifting $f$.
\end{cor}

\begin{rmk} If $f$ is a closed embedding, then so is $\wt{f}$. When this is the
case, then $\wt{Z}$ is called the \textit{proper transform} of $Z$ in $\wt{X}$.
\end{rmk}

\begin{cor} If $Z$ is integral and $Z \subseteq X$ is a closed embedding, then
$\wt{Z} = \ol{\pi^{-1}(X \setminus Y) \cap Z} \subseteq \wt{X}$.  \end{cor}

\begin{exer} Consider proper transforms of nodal cubic curves in $\Bl_0 \A^2$.
\end{exer}

Now let $E$ be the exceptional divisor of $\pi^{-1}(Y)$ under the blowup $\pi
\colon \wt{X} \to X$. Then we know \[ E = \Proj \qty(\eval{\mc{R}}_Y) = \Proj
\mc{R} \otimes_{\mc{O}_X} \mc{O}_Y = \Proj \mc{R} / \mc{I}_Y \cdot \mc{R} =
\bigoplus_{n \geq 0} \mc{I}^n / \mc{I}^{n+1}. \] For example, the exceptional
divisor of $\Bl_0 \A^2 \to \A^2$ is \[ E = \Proj \bigoplus {(x,y)}^n /
{(x,y)}^{n+1} = \Proj k[s,t] = \P^1. \] More generally, the blowup $\Bl_0 \A^n$
has exceptional divisor $\P^{n-1}$.

Now let $A$ be a Noetherian ring and $I \subseteq A$ be an ideal. Suppose $I$
is generated by a regular sequence of length $r$.  \begin{lem} If $I$ is
    generated by a regular sequence of length $r$, then $I/I^2$ is a free
    $A/I$-module of rank $r$, and \[ \bigoplus_{n \geq 0} I^n / I^{n+1} = S^*
    I/I^2. \] \end{lem}

\begin{defn} Let $X$ be a Noetherian scheme and $Y \subseteq X$ be a closed
subscheme. Then $Y$ is called \textit{locally complete intersection} of
codimension $r$ if $\mc{I}_Y$ is locally generated by a regular sequence of
length $r$.  \end{defn}

\begin{exm} Let $X = \Spec A$ where $A$ is a unique factorization domain. Then
$Y \subseteq X$ of codimension $1$ is a locally complete intersection.
\end{exm}

\begin{exm} If $E \subseteq \wt{X}$ is the exceptional divisor of a blowup,
then $E$ is a locally compelte intersection.  \end{exm}

\begin{exm} The twisted cubic is locally a complete intersection.  \end{exm}

\begin{prop} Let $X$ be Noetherian and $Y \subseteq X$ be locally complete
    intersection of codimension $r$. Then \begin{itemize} \item $\mc{I} /
        \mc{I}^2$ is a locally free coherent sheaf of rank $r$; \item The map
        $E = \R \mc{I}/\mc{I}^2 \to Y$ is a $\P^{r-1}$-bundle.  \end{itemize}
    \end{prop}

\begin{rmk} Locally on $\wt{X} \subseteq \P^n_X$ this is defined by $f_i T_j =
f_j T_i$, where $I = (f_1, \ldots, f_r)$.  \end{rmk}

\section{Projective morphisms and (very) ample line bundles}%
\label{sec:projective_morphisms_and_very_ample_line_bundles}

\begin{defn} A morphism $f \colon X \to Y$ is called \textit{projective} if $X
    \simeq \Proj_Y \mc{R}$ for some $\mc{R} = \bigoplus \mc{R}_i$, where
    $\mc{R}$ is a quasicoherent graded algebra, $\mc{R}_0 = \mc{O}_Y$, and
    $\mc{R}$ is finitely generated in degree $1$, so $\mc{R}_1$ is locally of
    finite type.  \end{defn}

This is equivalent to the existence of a factorization $f \colon X
\hookrightarrow \P \mc{F} \to Y$, where $\mc{F}$ is a quasicoherent sheaf
locally of finite type. If either of these conditions are satisfied, $X$ is
said to be \textit{projective} over $Y$.

\begin{rmk} If $f \colon X \to Y$ is projective, then on $X$ we have a line
    bundle $\mc{L} = \mc{O}(1) = \wt{\mc{R}(1)}$. Conversely, if there exists
    $\mc{L}$ and a quasicoherent sheaf $\mc{R}_1$ of finite type with $f^*
    \mc{R}_1 \twoheadrightarrow \mc{L}$ such that $f$ factors as $X
    \hookrightarrow \Proj_Y S^* \mc{R}_1 \to Y$, then $f$ is projective.
\end{rmk}

\begin{exm} Blowups are projective if $X$ is Noetherian and $\mc{I}$ is
coherent.  \end{exm}

\begin{exm} Clearly closed embeddings are projective.  \end{exm}

\begin{rmk} Projective morphisms are proper and stable under base change.
\end{rmk}

\begin{warn} Hartshorne defines projective morphisms as being $X
\hookrightarrow \P^n_Y \to Y$. This is strictly stronger than our definition.
\end{warn}

\begin{defn} If $Y$ is quasicompact, then $f \colon X \to Y$ is called
\textit{quasiprojective} if it can be factored as a quasicompact embedding
followed by a projective morphism.  \end{defn}

\begin{rmk} Projective morphisms are separated and proper. Also, there exists
some invertible sheaf $\mc{L} = i^* \mc{O}_{\P \mc{F}}(1)$.  \end{rmk}

Now we want consitions for an invertible sheaf $\mc{L}$ on $X$ to define an
immersion $X \to \P \mc{F}$.

\begin{defn} Let $f \colon X \to Y$ be a morphism of finite type with $Y$
    Noetherian. Then $\mc{L}$ is called \textit{very ample} over $Y$ if there
    exists a coherent sheaf $\mc{F}$ on $Y$ such that $f$ factors via an
    immersion as $f = X \hookrightarrow \P \mc{F} \to Y$ such that $\mc{L} =
    i^* \mc{O}_{\P \mc{F}}(1)$.  \end{defn}

\begin{rmk} If $Z \subset X$ is a closed subscheme and $\mc{L}$ is very ample,
then so is $i^* \mc{L}$.  \end{rmk}

\begin{exm} Let $f$ be as above. Then $\mc{O}_{\P \mc{F}}(1)$ is very ample
over $Y$.  \end{exm}

\begin{rmk} This is a relative notions. If $X = \Bl_x \P^2$, then $\mc{O}_X(1)$
is very ample over $\P^2$ but not ample over $\Spec k$ (?).\footnote{Giulia
deleted this remark during the lecture.} \end{rmk}

\begin{rmk} If $f$ is proper, then the morphism $X \to \P \mc{F}$ is
necessarily a closed immersion.  \end{rmk}

\begin{rmk} Vakil defines the notion of being very ample only for proper
morphisms. There is yet another different definition in G\"ortz-Wedhorn, but I
cannot be bothered to copy it.  \end{rmk}

\begin{rmk} By definition, if there exists a very ample $\mc{L}$, then $f$ is
separated.  \end{rmk}

\begin{defn} Let $X$ be Noetherian and $\mc{L}$ be an invertible sheaf on $X$.
    Then $\mc{L}$ is \textit{ample} if there exists $n_0$ such that $\mc{F}
    \otimes \mc{L}^n$ is globally generated for all $n \geq n_0$ and all
    coherent $\mc{F}$.  \end{defn}

\begin{exm} If $X = \Spec A$ is affine, then any invertible sheaf is ample.
\end{exm}

\begin{prop} Let $X$ be a quasicompact and quasiseparated scheme, $\mc{L}$ be a
    line bundle, $\mc{F}$ quasicoherent, and $f \in \Gamma(X, \mc{L})$.
    \begin{enumerate} \item Let $s \in \Gamma(X, \mc{F})$ such that
        $\eval{s}_{X_f} = 0$. Then there exists $n > 0$ such that $f^n s = 0$
        in $\Gamma(X, \mc{F} \otimes \mc{L}^n)$.  \item Let $t \in \Gamma(X_f,
\mc{F})$. Then there exists $n > 0$ such that $f^n t$ lifts to a section of
$\mc{F} \otimes \mc{L}^n$.  \end{enumerate} \end{prop}

\begin{thm}[Serre] Let $\mc{F}$ be quasicoherent on $X = \Proj R$, where $R$ is
    finitely generated in degree $1$. Then there exists $n_0 > 0$ such that for
    all $n \geq n_0$, $\mc{F}(n)$ is generated by a finite number of global
    sections.  \end{thm}

\begin{proof} By assumption, $\Proj R$ is quasicompact and quasiseparated.
    Also. Now $\mc{F} = \wt{M}_i$ on $X_{f_i}$, and these are all finitely
    generated. There are also finitely many $f_i$, so we can choose generators
    and lift to $\mc{F} \otimes \mc{O}(n)$.  \end{proof}

\begin{rmk} The sheaf $\mc{O}_{\Proj R}(1)$ is ample.  \end{rmk}

\begin{lem} The following are equivalent for a Noetherian scheme $X$:
\begin{enumerate} \item $\mc{L}$ is ample.  \item There exists $n > 0$ such
that $\mc{L}^n$ is ample.  \item $\mc{L}^n$ is ample for all $n>0$.
\end{enumerate} \end{lem}

This means that ample line bundles form a cone. They are invariant under
passing to positive powers. Now here is a useful fact. Let $R$ be as above and
assume $R_0$ is Noetherian. Then for all $\mc{F}$ quasicoherent on $X$,
$\Gamma_* \mc{F}$ is finitely presented over $R_0$.

\begin{cor} Let $f \colon X \to Y$ be a projective morphism with $Y$
Noetherian. then for all $\mc{F}$ coherent on $X$, $f_* \mc{F}$ is coherent on
$Y$.  \end{cor}

In general the pushforward of a coherent sheaf is not coherent (consider open
immersions).

\begin{prop}[Coherent extension] Let $X$ be Noetherian and $\mc{F}$ be
    quasicoherent on $X$. Suppose $U \subseteq X$ is open and $\mc{G}_U \subset
    \eval{\mc{F}}_U$ is a coherent subsheaf. Then there exists a coherent
    subsheaf $\mc{G} \subset \mc{F}$ such that $\eval{\mc{G}}_U = G_U$.
\end{prop}

\begin{cor} Let $X$ be Noetherian, $U \subseteq X$ be open and $\mc{L}$ be
ample on $X$. Then $\eval{\mc{L}}_U$ is ample.  \end{cor}

\begin{proof} For all $\mc{F}$ coherent on $U$, $j_* \mc{F}$ is quasicoherent
    on $X$. Then use coherent extension to construct an extension $\ol{\mc{F}}$
    on $X$ extending $\mc{F}$. If $\ol{\mc{F}} \otimes \mc{L}^n$ is globally
    generated, so is $\eval{\mc{F} \otimes \mc{L}^n}_U$.  \end{proof}

\begin{proof}[Proof of proposition] Consider the partially ordered set $\qty{
    (\mc{G}_W,W) }$ of coherent extensions of $W$ ordered in the obvious way.
    Then by Zorn, there exists a maximal element, and after reducing to the
    affine case, we can see that the maximal element is defined on all of $X$.
\end{proof}

\begin{prop} Let $X$ be Noetherian (or qcqs) and $\mc{L}$ be an invertible
    sheaf. Then the following are equivalent: \begin{enumerate} \item $\mc{L}$
        is ample.  \item For all coherent ideals $\mc{I}$, there exists $n_0 >
        0$ such that $\mc{I} \otimes \mc{L}^n$ is globally generated for all $n
        \geq n_0$.  \item The open subsets of the form $X_f, f \in \Gamma(X,
        \mc{L}^n)$ for some $n > 0$ form a basis for the topology of $X$.
    \item There exists $n_0$ and $f_1, \ldots, f_N \in \Gamma(X, \mc{L}^{n_0})$
        such that $X_{f_i}$ are affine and $X = \bigcup X_{f_i}$.
\end{enumerate} \end{prop}

\begin{proof}\leavevmode \begin{description} \item[1 implies 2:] This is by
    definition.  \item[2 implies 3:] Let $U \subseteq X$ be open, $x \in U$ be
    a closed point, and $Y = X \setminus U$ have the reduced structure. Finally
    let $\mc{I}_Y \subseteq \mc{O}_X$ be the ideal sheaf of $Y$. Then there
    exists $n$ such that $\mc{I}_Y \otimes \mc{L}^n$ is globally generated.
    Thus if $f \in \Gamma(X, \mc{I}_Y \otimes \mc{L}^n)$, so $X_f \subseteq U$.
\item[3 implies 4:] By quasicompactness, there exists $n$ such that $f_1,
    \ldots, f_n \in \Gamma(X, \mc{L}^n)$ are such that $X = \bigcup X_{f_i}$.
    Then $\mc{X}_{f_i} \subseteq U_i$ are affine and $\eval{\mc{L}}_{U_i} =
    \mc{O}_{U_i}$. Thus $X_{f_i}$ is a principal open subset of an affine, so
    it is affine.  \item[4 implies 1:] This is the exact same proof as Serre's
    theorem. \qedhere \end{description} \end{proof}

\begin{prop} Let $X$ be quasicompact and quasiseparated and $\mc{L}$ be ample.
Then write $R = \bigoplus \Gamma(X, \mc{L}^n)$. Then $X \to \Spec \Gamma(X,
\mc{O}_X)$ factors through the open immersion $X \hookrightarrow \Proj R$.
\end{prop}

\begin{thm} Let $f \colon X \to \Spec A$ be of finite type with $A$ Noetherian
    and $\mc{L}$ be an invertible sheaf on $X$. Then $\mc{L}$ is ample if and
    only if there exists $n \geq 0$ such that $\mc{L}^n$ is very ample.
    Moreover, if this is the case, then the immersion $X \hookrightarrow \P
    \mc{F}$ can be taken to be $X \hookrightarrow \P^N_A$.  \end{thm}

\begin{proof} Assume $\mc{L}^n$ is very ample. Then there exists $\mc{F}$
    coherent and also $\mc{O}_X^{N+1} \twoheadrightarrow \mc{F}$, so we obtain
    an immersion $j \colon X \hookrightarrow \P^N_A$. But now if $\mc{F}$ is
    coherent on $X$, we can apply coherent extension to find a coherent
    subsheaf $\ol{\mc{F}} \subset j_* \mc{F}$ on $\ol{X}$. Finally $i_*
    \mc{\ol{F}}$ is coherent on $\P^N$, so $\mc{L}$ is ample.

    Conversely, if $\mc{L}$ is ample, then there exist $s_1, \ldots, s_r \in
\Gamma(X, \mc{L}^n)$ such that $X = \bigcup X_{s_i}$ with $X_{s_i} = \Spec B_i$
affine. Also note that $B_i$ is a finitely generated $A$-algebra. Now let
$b_{ij}$ be generators of $B_i$ over $A$. Then there exists $N$ such that
$t_{ij} = s_i^N b_{ij}$ lift to $\Gamma(X, \mc{L}^N)$. Now we have finitely
many sections $s_i^N, t_{ij} \in \Gamma(X, \mc{L}^n)$, and these define a
regular morphism $\psi \colon X \to \P^m$. On each $X_{s_i} \to
{(\P^m)}_{T_i}$, we see that the map ${A[T_1, \ldots, T_r]}_{(T_i)} \to B_i$ is
surjective. Thus $\psi$ is a closed immersion into an open subset of $\P^n$.
\end{proof}

\begin{rmk} If $X$ is quasicompact and quasiseparated, $i \colon Z \to X$ is a
quasicompact immersion, then the pullback of an ample line bundle is ample.
\end{rmk}

\begin{thm} Let $X$ be proper over an algebraically closed field $k$ Now let
    $s_0, \ldots, s_n \in \Gamma(X, \mc{L})$ and set $V = \ev{s_0, \ldots,
    s_n}$. Then $V$ defines a closed immersion if and only if \begin{enumerate}
        \item $V$ separates closed points. This means for all $x \neq x'$ there
            exists $s \in V$ such that $s(x) = 0, s(x') \neq 0$.  \item $V$
            separates tangent directions. This means that for a closed point
            $x$ and a tangent vector $t \in T_x X$, there exists $s \in V$ such
            that $x \in \Supp s$ but $t \notin T_x V(s)$.\footnote{I copied
            this from Hartshorne, so it may not be the same as Giulia's
        statement.} \end{enumerate} \end{thm}

\begin{thm} Let $f \colon X \to Y$ be a quasiseparated morphism of finite type
    with $X, Y$ Noetherian. The following are equivalent: \begin{enumerate}
        \item $\mc{L}^n$ is very ample for some $n > 0$.  \item There exists an
            affine open covering $\qty{V_i}$ of $Y$ such that
            $\eval{\mc{L}}_{X_{V_i}}$ is very ample for all $i$.  \item For all
    $V \subseteq Y$ affine open, $\eval{\mc{L}}_{X_V}$ is ample.
    \end{enumerate} \end{thm}

\begin{proof}\leavevmode \begin{description} \item[1 implies 3:] Note that if
    $\mc{L}^n$ is very ample, then $\eval{\mc{L}^n}_{X_V}$ is very ample, so
    $\eval{\mc{L}}_{X_V}$ is ample.  \item[2 implies 1:] We have a morphism
    $\psi_i \colon X_{V_i} \P^{N_i}_{V_i}$ such that $\psi_i^* \mc{O}(1) =
    \eval{\mc{L}^{n_i}}_{X_{V_i}}$. Up to passing to Veronese embeddings, we
    may assume $n_i = n$. Then we have a surjection $f^* \mc{O}_{V_i}^{N_i+1}
    \twoheadrightarrow \eval{\mc{L}^n}_{V_i}$. By the adjoint, we obtain a
    morphism $\mc{O}_{V_i}^{N_i+1} \to f_* \eval{\mc{L}^n}_{V_i}$. Now the
    image of this morphism is coherent, so let $\mc{G}_i$ be a coherent
    extension of the $\mc{F}_i$ to $Y$. Then we obtain a morphism $\bigoplus
    \mc{G}_i \to f_* \mc{L}^n$, which gives us $f^* \bigoplus \mc{G}_i \to
    \mc{L}^n$. This is surjective because it is surjective on each $X_{V_i}$.
    Therefore we obtain a morphism $\psi \colon X \hookrightarrow \P \bigoplus
    \mc{G}_i$, and we need to show this is an immersion. For each $i$, we have
    an immersion $X_{V_i} \to \P_{Y}(\mc{G}_i)$. Now up to passing to an open
    subset, $\psi$ factors through $\P \bigoplus \mc{G}_i$, and so we need to
    show that if $g \circ f$ is an immersion, so is $f$. This is left as an
    exercise. \qedhere \end{description} \end{proof}

\section{Cartier and Weil divisors}% \label{sec:cartier_and_weil_divisors}

Let $\mc{L}$ be an invertible sheaf. Then there exists an open cover
$\qty{U_i}$ of $X$ such that $\eval{\mc{L}}_{U_i} \xrightarrow{\varphi_i}
\mc{O}_{U_i}$, and on $U_i \cap U_j$ we have an isomorphism $\mc{O}_{U_i \cap
U_j} \to \mc{O}_{U_i \cap U_j}$. Of course, this $\varphi_{ij} \in
{\mc{O}_X(U_i \cap U_j)}^{\times}$. On $U_i \cap U_j \cap U_k$, we have
$\varphi_{ij} \varphi_{jk} = \varphi_{ik}$. This is a {\v{C}}ech $1$-cocycle.
Then inside the set $Z^1(\mc{U}, \mc{O}_X^{\times})$ we have a subgroup of
boundaries $B^1(\mc{U}, \mc{O}_X^{\times})$. 

If $\mc{L} = \mc{M}$ is an isomorphism of invertible sheaves with cocycles
$\varphi_{ij}, \psi_{ij}$, there exist $f_i \in {\mc{O}_X(U_i)}^{\times}$ such
that $\varphi_{ij} = \psi_{ij} f_i f_j^{-1}$. Therefore, we can define the
\textit{Picard group} of a scheme to be the subgroup of isomorphism classes of
line bundles. This is isomorphic to the cohomology $H^1(X, \mc{O}_X^{\times})$.

Now consider sections $s \in \Gamma(X, \mc{L})$. This is the same as a morphism
$s \colon \mc{O}_X \to \mc{L}$, and if $X$ is integral, this is always
injective if $s \neq 0$. Dualizing, we have an injection $\mc{L}^{\vee}
\subseteq \mc{O}_X$, so we realize $\mc{L}^{\vee}$ as an ideal sheaf. Every
section may determine a different embedding $\mc{L}^{\vee} \subset \mc{O}_X$,
and thus a different closed subscheme of $X$.

\begin{exm} Consider $\mc{O}(d)$ on $X = \P^n$. Then $s \in \Gamma(X,
    \mc{O}(d)) = {k[x_0, \ldots, x_n]}_d$, and we obtain the exact sequence \[
    0 \to \mc{O}(-d) \to \mc{O} \to \mc{O}_H \to 0, \] where $H$ is a degree
$d$ hypersurface.  \end{exm}

\begin{rmk} Let $X$ be proper over a field $k$ and $\mc{L}$ is very ample. Then
    if $X \hookrightarrow \P^n$ is the embedding given by $\mc{L}$, then $s \in
    \Gamma(X, \mc{L})$ corresponds to a section $s \in \Gamma(\P^n,
    \mc{O}(1))$, and $V(s) = X \cap H$, where $H$ is the hyperplane cut out by
    $s$.  \end{rmk}

Now let $X$ be a scheme. For an open $U \subseteq X$, consider the set $S$ of
elements of $\Gamma(U, \mc{O}_X)$ that are not zero divisors and now write
$K(U) = S^{-1} \Gamma(U, \mc{O}_X)$. Now we define a presheaf $U \mapsto K(U)$,
and after sheafifying we obtain sheaves $\mc{K}(X), {\mc{K}(X)}^{\times}$.

\begin{exm} When $X$ is integral, this is a constant sheaf.  \end{exm}

Of course, we have a injection $\mc{O}_X^{\times} \to {\mc{K}_X^{\times}}$, and
now we have an exact sequence of sheaves \[ 0 \to \mc{O}_X^{\times} \to
\mc{K}_X^{\times} \to \mc{K}_X^{\times} / \mc{O}_X^{\times} \to 0. \]

\begin{exm} A \textit{Cartier divisor} is an element $D \in \Gamma(X,
\mc{K}_X^{\times} / \mc{O}_X^{\times})$.  \end{exm}

Concretely, we obtain a collection of compatible pairs $(U,f)$, where $U
\subset X$ is open and $f \in \Gamma(U, \mc{K}_X^{\times})$ such that $(U, f)
\sim (U', f')$ if $f' f^{-1} \in {\mc{O}_X(U \cap U')}^{\times}$. Now to a
Cartier divisor $D$, we can consider $\Supp(D) = \qty{x \mid D_x \neq 1} =
\qty{x \mid f_x \notin \mc{O}_{X,x}^{\times}}$. Now if $D = \qty{(U_i, f_i)},
\qty{(U_i, g_i)}$, we denote $D \pm E$ the Cartier divisor defined by
$\qty{(U_i, f_i g_i^{\pm 1})}$.

\begin{defn} A Cartier divisor is called \textit{principal} if it is in the
    image of $\Gamma(X, \mc{L}_X^{\times})$ given by $D = {(X, f)}$. Here, we
    write $D = (f)$. Two Cartier divisors are called \textit{linearly
    equivalent} if $D-E = (f)$ is principal.  \end{defn}

\begin{defn} A Cartier divisor $D$ is called \textit{effective} if $D =
    \qty{(U_i, f_i)}$, where $f_i \in \mc{O}_X(U_i) \cap
    \mc{K}_X^{\times}(U_i)$. This is a regular function that is not a zero
    divisor, and so an effective Cartier divisor defines an ideal sheaf
    $\mc{I}_D \subseteq \mc{O}_X$. Clearly this is locally free of rank $1$.
\end{defn}

Therefore, we see that effective Cartier divisors are the same as invertible
ideal sheaves. Also, we will write $D \geq E$ if and only if $D - E \geq 0$.
Now to a Cartier divisor $D$ we will associate the sheaf $\mc{O}_X(\pm D)$
locally defined by $f_i^{\mp 1}$. Also, we see that $\mc{O}_X(-D) \subseteq
\mc{O}_X$ if and only if $D \geq X$, so we have a group homomorphism from
Cartier divisors to the Picard group $\Pic X$.

\begin{prop} This assignment factors through the group of Cartier divisors
modulo principal divisors and in fact, the kernel is precisely the set of
principal divisors. If $X$ is integral, the assignment is surjective.
\end{prop}

\begin{proof} If $D = (f)$, where $f \in \Gamma(X, \mc{K}_X^{\times})$, $f$
    determines an isomorphism $\mc{O}_X \to \mc{O}_X((f))$. On the other hand,
    we need to show that if $\mc{O}_X(D) \simeq \mc{O}_X$, then $D$ can be
    represented by $(U, 1)$. Finally, if $X$ is integral, then the sheaf
    $\mc{K}(X)$ is constant, so if $\mc{L}$ corresponds to
    $\qty{\varphi_{ij}}$, then we fix $j$ and think of $f_i \coloneqq
    \varphi_{ij} \in \Gamma(U_i, \mc{K}_X^{\times})$, and then everything will
    glue.  \end{proof}

\begin{prop} If $X$ is integral and $\mc{L}$ is invertible, then effective
    Cartier divisors Cartier divisors $D$ such that $\mc{O}_X(D) = \mc{L}$
    correspond exactly to nonzero global sections of $\mc{L}$ modulo invertible
    functions $\Gamma(X, \mc{O}_X^{\times})$. In particular, if $X$ is proper
    over an algebraically closed field $k$, then $\dim_k \Gamma(X, \mc{L}) <
    \infty$.  \end{prop}

\begin{proof} Suppose $D \geq 0$. Then $\mc{O}_X(-D) \hookrightarrow \mc{O}_X$,
    and tensoring with $\mc{L}$, we obtain a section. Conversely, given a
    section, we obtain an invertible ideal sheaf $\mc{L}^{\vee} \subset
    \mc{O}_X$, and a Cartier divisor $D$.  \end{proof}

Now we will consider Weil divisors. Let $X$ be an integral Noetherian scheme.
Now a \textit{Weil divisor} is a finite sum $\sum a_i D_i$, where $a_i \in \Z$
and $D_i$ is an integral subscheme of codimension $1$. We need to assume that
$X$ is regular in codimension $1$. Note that a local Noetherian ring of
dimension $1$ is regular if and only if it is a discrete valuation ring. 

\begin{defn} Let $f \in K^{\times}(X)$ and $W \subseteq X$ be a prime divisor.
    Then $\mc{O}_{X,W}$ is a discrete valuation ring, and thus we obtain a
    valuation $v_W \colon K(X) \to \Z$. Then $f$ is said to have a \textit{pole
    or zero along $W$} if $v_W(f)$ is negative (or positive). Now we define a
    function $\Gamma(X, \mc{K}_X^{\times}) = K^{\times}(X) \to Z^1(X)$ to the
    group of Weil divisors by $f \mapsto \sum v_W(f) \eqqcolon (f)$.
\end{defn}

\begin{defn} Given $f$, there exist finitely many prime divisors $W$ such that
$v_W(f) \neq 0$.  \end{defn}

\begin{defn} A \textit{Weil divisor} $D = \sum a_i W_i$ is called
\textit{principal} if there exists $f \in K^{\times}(X)$ such that $D = (f)$.
\end{defn}

In fact, we can define a map $\Gamma(X, \mc{K}_X^{\times} / \mc{O}_X^{\times})
\to Z^1(X)$ sending principal Cartier divisors to principal Weil divisors. This
induces an injection $\Pic(X) \hookrightarrow Z^1(X)/K^{\times}(X)$. If $X$ is
locally factorial, this is an isomorphism.

\begin{exm} Consider $\P^n$. Then it is easy to see that $\Pic(\P^n) = \Z$ (if
we think of everything as a Weil divisor $\sum n_i Y_i$, then equivalence
classes are classified by $\sum n_i$).  \end{exm}

\chapter{Cohomology}% \label{cha:cohomology}

\section{Derived Functors}% \label{sec:derived_functors}

Let $\mc{A}, \mc{B}$ be abelian categories and $F \colon \mc{A} \to \mc{B}$ be
a left exact functor. For example, if $X$ is a topological space, we can have
$\mc{A}$ be the category of modules, $\mc{B} = \ms{Ab}$, and $F = \Gamma(X,-)$.
This is left exact but not exact in general.

\begin{exm} Let $A$ be a ring and $\mc{A} = \ms{Mod}_A$. Then $\Hom_A(M,-)$ is
left exact.  \end{exm}

\begin{exm} Let $X$ be a scheme and $p \neq q$ be closed points. Then we have
    the exact sequence \[ 0 \to \mc{I}_{p,q} \to \mc{O}_X \to k(p) \oplus k(q)
    \to 0. \] If $X$ is proper over an algebraically closed field $k$, then
    $\Gamma(X, \mc{O}_X) \to k \oplus k$ cannot be surjective.  \end{exm}

\begin{exm} If $X = \Spec A$ is affine, then $\Gamma(X, -)$ is exact on
$\ms{Qcoh}(X)$.  \end{exm}

Now we will construct a sequence of functors $R^i F$ for $i \geq 0$ such that
$R^0 F = F$. In some semse, this will measure the failure of exactness of $F$.
To do this, we will replace $\mc{A}$ with the derived category $D(\mc{A})$ of
complexes localized at quasi-isomorphisms. I have discussed chain homotopies
and quasi-isomorphisms in my notes for several other courses,\footnote{For
example, see my algebraic topology notes at
\url{https://math.columbia.edu/~plei/docs/AT1.pdf}.} so I will omit the
discussion here. In order to construct derived functors, we need to replace $A
\in \mc{A}$ with a quasi-isomorphic complex $I^{\bullet}$ of injective objects.

\begin{defn} An object $I \in \mc{A}$ is called \textit{injective} if
$\Hom_{\mc{A}}(-, I)$ is exact.  \end{defn}

\begin{exm} In $\ms{Vect}_k$, every object is injective.  \end{exm}

\begin{exer} If $0 \to A_1 \to A_2 \to A_3 \to 0$ is a short exact sequence,
then if $A$ is injective, the sequence is split.  \end{exer}

\begin{defn} An abelian category $\mc{A}$ \textit{has enough injectives} if for
all $A \in \mc{A}$, there exists an injective object $I$ and injection $A
\hookrightarrow I$.  \end{defn}

An injective resolution of $A \in \mc{A}$ a complex $A \to I^{\bullet}$ that is
a long exact sequence. Clearly, $\mc{A}$ has enough injectives if and only if
injective resolutions always exist.

\begin{lem} For $A \in \mc{A}$, any two injective resolutions are
quasi-isomorphic.  \end{lem}

If $A$ is a ring, then $\ms{Mod}_A$ has enough injectives.

\begin{cor}\leavevmode \begin{enumerate} \item Let $(X, \mc{O}_X)$ be a ringed
space. Then $\ms{Mod}(X)$ has enough injectives.  \item Let $X$ be a
topological space. Then $\ms{Ab}(X)$ has enough injectives.  \end{enumerate}
\end{cor}

\begin{proof} For all $x \in X$, we have an injective module $I_x$ with
    $\mc{F}_x \hookrightarrow I_x$. But then we obtain an injection \[ \mc{F}
    \hookrightarrow \prod_{x \in X} {(j_x)}_* I_x. \] The target is injective,
    so we are done. In the second case, simply take the ringed space $(X, \Z)$.
\end{proof}

Now we may define the right derived functors of $F$.  \begin{defn} The
    \textit{right derived functors} of $F$ are the functors \[ R^i F \colon
    \mc{A} \to \mc{B} \qquad A \mapsto H^i(F(I^{\bullet})). \] Alternatively,
    we may consider the complex $RF(A) = F(I^{\bullet})$, which is well-defined
    up to quasi-isomorphism.  \end{defn}

\begin{rmk} Because $F$ is left-exact, we $H^0(F(I)) = R^0 F(A) = F(A)$ as
desired.  \end{rmk}

The crucial observation is that this definition does not depend on the choice
of injective resolution.

\begin{exm} Let $X$ be a topological space and $F = \Gamma(X,-)$. Then $R^i
\Gamma(X,-) \eqqcolon H^i(X,-)$ are called the \textit{cohomology functors} of
$X$.  \end{exm}

\begin{thm} Let $F \colon \mc{A} \to \mc{B}$ be an exact functor as above. Then
    \begin{enumerate} \setcounter{enumi}{0} \item The derived functors $R^i F$
        are well-defined and additive.  \item If $0 \to A' \to A \to A'' \to 0$
        is a short exact sequence in $\mc{A}$, we have a long exact sequence \[
        \cdots \to R^i F(A') \to R^i F(A) \to R^i F(A'') \xrightarrow{\delta_i}
    R^{i+1}F(A') \to \cdots \] \item Given two short exact sequences and
    morphism $f \colon A^{\bullet} \to B^{\bullet}$ in $\mc{A}$, the diagram
    \begin{equation*} \begin{tikzcd} R^i (A'') \ar{r}{\delta} \ar{d}{R^i f} &
    R^{i+1}(A') \ar{d}{R^{i+1}f} \\ R^i(B'') \ar{r}{\delta} & R^{i+1}(B')
\end{tikzcd} \end{equation*} commutes.  \end{enumerate} \end{thm}

\begin{exm} If $I$ is injective, then $R^i F(I) = 0$ for all $i > 0$.
\end{exm}

Injective resolutions are hard to compute, so we will try construct a
resolution that is easier to compute. 

\begin{defn} $A \in \mc{A}$ is called \textit{$F$-acyclic} if $R^i F(A) = 0$
for all $i > 0$.  \end{defn}

\begin{exm} Injective objects are acyclic for all left-exact functors.
\end{exm}

\begin{prop} If $A \to J^{\bullet}$ is ann $F$-acyclic resolution, then $R^i
F(A) = H^i(F(J))$.  \end{prop}

\begin{proof} Consider the diagram of exact sequences \begin{equation*}
    \begin{tikzcd} A \ar{r} \ar[equal]{d} & J^0 \ar{r} \ar{d} & J^1 \ar{r}
        \ar{d} & \cdots \\ A \ar{r} & I^0 \ar{r} \ar{d} & I^1 \ar{r} \ar{d} &
        \cdots \\ & J^0 \ar{r} & K^1 \ar{r} & \cdots.  \end{tikzcd}
    \end{equation*} Becuase the $I^j$ are injective, the map $J^{\bullet} \to
    I^{\bullet}$ is a quasi-isomorphism. Next, $K^{\bullet}$ is exact. Because
    $J^i, I^i$ are acyclic, so are the $K^i$. But this means that the the map
    $F(J^{\bullet}) \to F(I^{\bullet})$ is a quasi-isomorphism. To complete
    this, it is an exercise that if $C^{\bullet}$ is an exact complex of
    acyclic objects, then $F(C^{\bullet})$ is exact.  \end{proof}

\begin{defn} A sheaf $\mc{F} \in \ms{Ab}(X)$ is called \textit{flasque} if for
all $V \subseteq U$ the restriction $\mc{F}(U) \to \mc{F}(V)$ is surjective.
\end{defn}

\begin{exer} Let $0 \to \mc{F}' \to \mc{F} \to \mc{F}'' \to 0$ be a short exact
sequence. If $\mc{F}'$ is flasque, then $\Gamma(X, \mc{F}) \twoheadrightarrow
\Gamma(X, \mc{F}'')$ is surjective.  \end{exer}

\begin{exm} If $(X, \mc{O}_X)$ is a ringed space, then injective sheaves are
flasque.  \end{exm}

\begin{prop} Let $X$ be a topological space and $\mc{F} \in \ms{Ab}(X)$ be
flasque. Then $\mc{F}$ is $\Gamma(X,-)$-acyclic.  \end{prop}

\begin{proof} Let $0 \to \mc{F} \to I \to \mc{G} \to 0$ be an exact sequence
    with $I$ injective. Then because $\mc{F}$ is flasque, we obtain an exact
    sequence \[ 0 \to \Gamma(X, \mc{F}) \to \Gamma(X, I) \to \Gamma(X, \mc{G})
    \to 0. \] Because $H^i(X, I) = 0$ for all $i > 0$, we know $H^1(X, \mc{F})
    = 0$ and $H^i(X, \mc{G}) \simeq H^{i+1}(X, \mc{F})$. By induction, we see
    that $\mc{F}$ is acyclic.  \end{proof}

\begin{rmk} If $(X, \mc{O}_X)$ is a ringed space, cohomology in $\ms{Mod}(X)$
is the same as cohomology in $\ms{Ab}(X)$.  \end{rmk}

\begin{thm} Let $X$ be a Noetherian topological space of dimension $n$. Then
$H^i(X, \mc{F}) = 0$ for $i > n$.  \end{thm}

\begin{lem} Let $i \colon Y \hookrightarrow X$ be a closed immersion. Then
$H^i(Y, \mc{F}) = H^i(X, i_* \mc{F})$.  \end{lem}

\section{Cohomology of Noetherian Schemes}%
\label{sec:cohomology_of_noetherian_schemes}

\begin{thm} Let $X = \Spec A$ be a Noetherian scheme and $\mc{F}$ be
quasicoherent. Then $\mc{F}$ is $\Gamma(X,-)$-acyclic. In other words, $H^i(X,
\mc{F}) = 0$ for all $i > 0$.  \end{thm}

This result is implied by the following proposition: \begin{prop} Let $I$ be a
injective $A$=module. Then $\wt{I}$ is flasque.  \end{prop}

To prove the theoremm from this proposition, write $\mc{F} = \wt{M}$ and let $M
\to I^{\bullet}$ be an injective resolution in $\ms{Mod}(A)$. Then $\wt{M} \to
\wt{I}^{\bullet}$ is a flasque resolution, so it computes the cohomology.

\begin{cor} Let $X$ be a Noetherian scheme and $\mc{F}$ be quasicoherent. Then
for $\mc{F}$ coherent, there exists $\mc{G}$ quasicoherent and flasque with
$\mc{F} \hookrightarrow \mc{G}$.  \end{cor}

\begin{proof} Cover $X$ be affines $U_i$. Then we have an injection \[ \mc{F}
\hookrightarrow \bigoplus j_I \eval{\mc{F}}_{U_i} \hookrightarrow \bigoplus j_*
\wt{I}_i. \qedhere \] \end{proof}

\begin{thm}[Serre] Let $X$ be Noetherian. The following are equivalent:
    \begin{enumerate} \item $X$ is affine.  \item $H^i(X, \mc{F}) = 0$ for all
        quasicoherent $\mc{F}$ and $i>0$.  \item $H^1(X, \mc{I}) = 0$ for all
        coherent ideal sheaves $\mc{I} \subseteq \mc{O}_X$.  \end{enumerate}
    \end{thm}

\begin{proof} Clearly 1 implies 2 implies 3, so now choose $p \in X$ a closed
    point and $U \ni p$ be an open neighborhood. Then if $Y = X \setminus U$
    and $Z = Y \cup p$, we have an exact sequence \[ 0 \to \mc{I}_Z \to
    \mc{O}_x \to \mc{O}_Z \to 0. \] From the vanishing of $H^1$ for coherent
    ideal sheaves, there exists $f \in \Gamma(X, \mc{O}_X)$ such that $f(p)
    \neq 0$. Then $X_f \subseteq U$ is affine, so now we need to show that
    $\ev{f_1, \ldots, f_k} = \Gamma(X, \mc{O}_X)$.  \end{proof}

\section{\v{C}ech cohomology}% \label{sec:cech_cohomology}

Let $X$ be a topological space and $\mc{U} = \qty{U_i}_{i \in I}$ and fix an
ordering of $I$. Then let $\mc{F} \in \ms{Ab}(X)$. For all $p \geq 0$, define
the \textbf{sheaf} \[ \mc{C}^p(\mc{U}, \mc{F}) =  \prod_{i_1 < \cdots < i_p}
\eval{\mc{F}}_{U_{i_1} \cap \cdots \cap U_{i_p}} \] and the \textbf{group} \[
C^p(\mc{U}, \mc{F}) = \prod \mc{F}(U_{i_0} \cap \cdots \cap U_{i_p}). \] Now we
may define a complex by \[ \dd \colon \mc{C}^p(\mc{U}, \mc{F}) \to
\mc{C}^{p+1}(\mc{U}, \mc{F}) \qquad {s_{i_0\ldots i_p}} \mapsto
\sum_{j=0}^{p+1} {(-1)}^j \eval{ s_{i_0 \ldots i_p} }_{\cdots}. \]

\begin{exm} The kernel of $d^0$ is precisely the global sections.  \end{exm}

\begin{exer} The complex $\mc{F} \to \mc{C}^0(\mc{U}, \mc{F}) \to
\mc{C}^1(\mc{U}, \mc{F}) \to \cdots$ is a resolution of $\mc{F}$.  \end{exer}

\begin{defn} Let $X, \mc{U}, \mc{F}$ be as above. Then the \textit{\v{C}ech
    cohomology} of the covering is defined as \[ \check{H}^i(\mc{U}, \mc{F}) =
    H^i(C^{\bullet}(\mc{U}, \mc{F})). \] then the \v{C}ech cohomology of $X$ is
    defined as $\check{H}^i(X, \mc{F}) = \varinjlim_{\mc{U}}
    \check{H}^i(\mc{U}, \mc{F})$.  \end{defn}

Now we will compare \v{C}ech cohomology and derived functor cohomology.
\begin{enumerate} \item Flasque sheaves have no higher \v{C}ech cohomology. In
    particular, if $\mc{F}$ is flasque then $\check{H}^i(\mc{F}, \mc{F}) = 0$
    for all $\mc{U}$ and all $i > 0$.

        To see this, note that $\eval{\mc{F}}_U$ is flasque, so $\mc{F} \to
        \mc{C}^{\bullet}(\mc{U}, \mc{F})$ is a flasque resolution. Therefore
        the \v{C}ech cohomology is the same as the usual cohomology, so by
        flasqueness, they must vanish.  \item Let $X$ be a topological space
        and $\mc{U}$ be an open covering. Then there exists a functorial map
        $\check{H}^i(\mc{U}, \mc{F}) \to H^i(X, \mc{F})$.  \item If $X$ is
        Noetherian and separated, then for every $\mc{F}$ quasicoherent and
        open affine cover $\mc{U}$, the map $\check{H}^i(\mc{U}, \mc{F}) \to
        H^i(X, \mc{F})$ is an isomorphism.

        Because $X$ is separated, $U_i \cap U_j$ is affine, so if we denote
        $U_{\alpha} = U_{i_0} \cap \cdots \cap U_{i_p}$ for any multi-index
        $\alpha = i_0 \ldots i_p$, then we can consider an exact sequence \[ 0
        \to \mc{F} \to \mc{G} \to \mc{E} \to 0. \] Because $\mc{G}$ is
        quasicoherent and flasque, it has no derived functor cohomology and no
        \v{C}ech cohomology. Because the $U_{\alpha}$ are affine, then \[ 0 \to
        \mc{F}(U_{\alpha}) \to \mc{G}(U_{\alpha}) \to \mc{E}(U_{\alpha}) \to 0
    \] is exact, so we obtain an exact sequence \[ 0 \to C^{\bullet}(\mc{U},
\mc{F}) \to C^{\bullet}(\mc{U}, \mc{G}) \to C^{\bullet}(\mc{U}, \mc{E}) \to 0
\] of complexes. Now we obtain a long exact sequence in the $H^i(X,-)$, so now
$H^i(X, \mc{E}) = H^{i+1}(X, \mc{F})$. Using the snake lemma, we haave the same
result for the \v{C}ech cohomology. Because the desired result holds for $i=0$,
we use induction to obtain it for all $i$.  \end{enumerate}

\section{Cohomology of projective schemes}%
\label{sec:cohomology_of_projective_schemes}

\begin{thm} Let $A$ be a Noetherian ring and $X = \P^r_A$.  \begin{enumerate}
    \item As graded rings, $\bigoplus H^0(X, \mc{O}_X(n)) \simeq A[x_0, \ldots,
        x_r]$.  \item For all $n \in \Z$ and $0 < i < r$, $H^i(X, \mc{O}_X(n))
        = 0$.  \item $H^r(X, \mc{O}_X(-r-1)) = A$.  \item The map $H^0(X,
\mc{O}_X(n)) \times H^r(X, \mc{O}_X(-r-1-n)) \to H^r(X, \mc{O}_X(-r-1))$ is a
perfect pairing.  \end{enumerate} \end{thm}

\begin{rmk} The bundle $\mc{O}_X(-r-1) \simeq \omega_X$ is the canonical bundle
$\det \Omega^1_{\P^n_A} = \bigwedge^r \Omega^1_{\P^n_A}$. Compare the third
result to $h^r(\P^r_{\C}, \Omega^r_{\C}) = h^{r,r}(\P^r_{\C}) = 1$.  \end{rmk}

\begin{rmk} These are particular instances of \textit{Serre duality}, which
    says that when $A = k$, then \begin{enumerate} \item $H^r(\omega_X) = k$;
        \item The map $\Hom(\mc{F}, \omega_X) \times H^r(X, \mc{F}) \to H^r(X,
            \omega_X) = k$ is a perfect pairing; \item $\Ext^i(\mc{F},
    \omega_X) \simeq {H^{r-i}(X, \mc{F})}^{\vee}$, \end{enumerate} which holds
    when $X$ is a projective scheme over $k$ and $\omega_X$ is the
    \textit{dualizing sheaf}. In nice cases, for example when $X$ is smooth,
    $\omega_X$ is just the canonical bundle.  \end{rmk}

The proof of the theorem is a direct computation using \v{C}ech cohomology. Now
we will state important finite results for cohomology of coherent sheaves on
Noetherian projective schemes.

\begin{thm}[Serre] Let $X \to \Spec A$ be a projective scheme of finite type
    over $A$ with $A$ Noetherian. Let $\mc{F}$ be a coherent sheaf on $X$ and
    $\mc{O}_X(1)$ be very ample. Then \begin{enumerate} \item $H^i(X, \mc{F})$
        is a finitely generated $A$-module for all $i$; \item There exists
$n_0$ such that for all $i > 0$ and $n \geq n_0$, $H^i(X, \mc{F}(n)) = 0$.
\end{enumerate} \end{thm}

\begin{proof} We can reduce this to the case of $X = \P^r_A$. For $i > r$, we
    know $H^i(X, \mc{F}) = 0$ because $\P^r_A$ is covered by $r+1$ affine open
    subsets and thus $\mc{C}^p(\mc{U}, \mc{F}) = 0$ for $p > r+1$. For the
    second part, we use descending induction on $i$, using the fact that there
    exists a surjection $\mc{O}_X^N(n) \twoheadrightarrow \mc{F}$ for $n \geq
    n_0$ and what we already know about $H^i(\mc{O}_X(n))$.  \end{proof}

\begin{rmk} The $n_0$ depends very much on $\mc{F}$. A crucial point in the
construction of Hilbert schemes is to find an $n_0$ that works for any sheaf of
ideals $\mc{I}_Z$ as long as we fix the Hilbert polynomial of $Z$.  \end{rmk}

\begin{defn} Let $X$ be projective over $k$ with $\mc{O}_X(i)$ ample and
    $\mc{F}$ be coherent. Then define the \textit{Euler characteristic} \[
    \chi(\mc{F}) \coloneqq \sum {(-1)}^i \dim_k H^i(X, \mc{F}). \] This is
    additive on short exact sequences of coherent sheaves.  \end{defn}

\begin{defn} The function $n \mapsto \chi(X, \mc{F}(n))$ is a polynomial with
rational coefficients, called the \textit{Hilbert polynomial} $p_{\mc{F}}(n)$.
\end{defn}

\begin{exm} If $X = \P^r$, then $p_{\mc{O}_X}(n) = p_X(n) = \binom{n+r}{r}$.
\end{exm}

\begin{exer} Compute the Hilbert polynomial of a degree $d$ hypersurface $Y
\subseteq \P^r$.  \end{exer}

\begin{rmk} Given $\mc{F}$, the coefficients of the Hilbert polynomial
$p_{\mc{F}}(n)$ are important invariants of $\mc{F}$.  \end{rmk}

\begin{rmk} We will see later that the Hilbert polynomial is constant in flat
families.  \end{rmk}

\section{Higher direct images}% \label{sec:higher_direct_images}

Let $f \colon X \to Y$ be a continuous map of topological spaces. Then $f_*
\colon \mf{Ab}(X) \to \ms{Ab}(Y)$ is left exact. We also know that $\ms{Ab}(X)$
has enough injectives, so we may consider the right derived functors.

\begin{defn} The functors $R^i f_* \colon \ms{Ab}(X) \to \ms{Ab}(Y)$ are the
\textit{higher direct image} functors.  \end{defn}

\begin{prop} The higher direct image $R^i f_* \mc{F}$ is the sheaf associated
    to the presheaf \[ V \mapsto H^i\qty(f^{-1}(V), \eval{\mc{F}}_{f^{-1}(V)}).
    \] In particular, $\eval{R^i f_* \mc{F}}_V = R^i
    {\qty(\eval{f}_{f^{-1}(V)})}_* \qty(\eval{\mf{F}}_{f^{-1}(V)})$ and if
    $\mc{F}$ is flasque, then $R^i f_* \mc{F} = 0$ for all $i > 0$. This means
    that flasque sheaves are $f_*$-acyclic, so they may be used to compute
    higher direct images. Also computing higher direct images is the same in
    $\ms{Ab}(X)$ and in $\ms{Mod}(X)$.  \end{prop}

\begin{prop} Let $f \colon X \to Y = \Spec A$ with $X$ Noetherian. Let $\mc{F}$
    be quasicoherent. Then $R^i f_* \mc{F} = \wt{H^i(X, \mc{F})}$. Therefore
    $R^i f_* \mc{F}$ is quasicoherent, and if $f$ is projective with $A$
    Noetherian, then $R^i f_*$ preserves coherent sheaves.  \end{prop}

\begin{prop} Let $f \colon X \to Y$ be a morphism of separated Noetherian
    schemes, $\mc{F}$ be quasicoherent on $X$, and $\mc{U} = \qty{U_i}$ be an
    open affine cover. Let $\mc{C}^{\bullet}(\mc{U}, \mc{F})$ be the \v{C}ech
    resolution of $\mc{F}$. Then $R^i f_* \mc{F} \simeq H^i(f_*
    \mc{C}^{\bullet}(\mc{U}, \mc{F}))$.  \end{prop}

\begin{proof} Because $Y$ is separated, for all $V \subseteq Y$ affine,
    $f^{-1}(V) \cap U_i$ is affine. Therefore we may assume that $Y = \Spec A$
    is affine. But now \begin{align*} R^i f_* \mc{F} &= \wt{H^i(X, \mc{F})} \\
    &= \wt{H^i(\Gamma(X, \mc{C}^{\bullet}(\mc{U}, \mc{F})))} \\ &=
    \wt{H^i(\Gamma(Y, f_* \mc{C}^{\bullet}(\mc{U}, \mc{F})))} \\ &= H^i(f_*
C^{\bullet}(\mc{U}, \mf{F})).  \end{align*} \end{proof}

\begin{rmk} It is also often useful to use the long exact sequence for right
derived functors.  \end{rmk}

\begin{thm} Let $f \colon X \to Y$ be a projective morphism of finite type with
    $X, Y$ Noetherian. Suppose $\mc{O}_X(1)$ is very ample over $Y$ and let
    $\mc{F}$ be coherent on $X$.  \begin{enumerate} \item There exists $n_0$
        such that for $n \geq n_0$ the map $f^* f_* \mc{F}(n) \to \mc{F}(n)$ is
        surjective; \item $R^i f_* \mc{F}$ is coherent; \item   There exists
$n_0$ such that for all $n \geq n_0$ and $i > 0$, $R^i f_* \mc{F}(n) = 0$ for
$i > 0$.  \end{enumerate} \end{thm}

\begin{proof} Because $Y$ is quasicompact, we may reduce to the affine case.
\end{proof}

Now we will consider base change of $R^i f_* \mc{F}$ along general morphisms
$X' \to Y$.

\begin{prop} Let $X, Y$ be Noetherian and separated schemes, $f \colon X \to Y$
    be of finite type, and $Y'$ be Noetherian. Suppose $\mc{F}$ is
    quasicoherent and let \begin{equation*} \begin{tikzcd} X' \ar{r}{u}
        \ar{d}{f'} & X \ar{d}{f} \\ Y' \ar{r}{v} & Y \end{tikzcd}
        \end{equation*} be Cartesian. Then there exists a base change morphism
        \[ v^* R^i f_* \mc{F} \to R^i f'_* (u^* \mc{F}) \] which is an
    isomorphism if $v$ is flat.  \end{prop}

\begin{proof} We may assume $Y = \Spec A, Y' = \Spec A'$ are affine. Then $R^i
    f_* \mc{F} = \wt{H^i(X, \mc{F})}$, so \[ v^* R^i f_* \mc{F} = \wt{H^i(X,
    \mc{F})} \otimes_A A' = H^i(\mc{C}^{\bullet}(\mc{U}, \mc{F})) \otimes_A A'.
\] On the other hand, if we cover $X'$ by $U_i \times_A \Spec A' \eqqcolon
U_i'$, then $\mc{C}^{\bullet}(\mc{U}', u^* \mc{F}) = u^*
\mc{C}^{\bullet}(\mc{U}, \mc{F})$ and $C^{\bullet}(\mc{U}', u^* \mc{F}) =
C^*(\mc{U}, \mc{F}) \otimes_A A'$. Therefore we obtain a map \[
H^i(C^{\bullet}(\mc{U}, \mc{F})) \otimes_A A' \to H^i(C^{\bullet}(\mc{U},
\mc{F}) \otimes_A A') \] which is an isomorphism if $A \to A'$ is flat.
\end{proof}

\begin{rmk} If $y \in Y$ is a point, then $y \to Y$ is in general not flat, so
it is not easy to compare ${(R^i f_* \mc{F})}_y$ and $H^i(X_y, \mc{F}_y)$.
\end{rmk}

\begin{exm} Let $C$ be a smooth curve over $k$ that is irrational. Then
    consider $\pi = p_2 \colon C \times C \to C$ and let $\mc{L} = \mc{O}_{C
    \times C}(\Sigma - \Delta)$ where $\Sigma = p_0 \times C$ and $\Delta$ is
    the diagonal. Then $\pi_* \mc{L}$ is torsion free on $C$ and thus locally
    free. However, \[ { ( \pi_* \mc{L} ) }_p = \begin{cases} \mc{O}_C & p = p_0
    \\ \mc{O}_C(p-p_0) \neq \mc{O}_C & p \neq p_0 \end{cases}. \] Therefore the
rank of $H^0(C, { ( \pi_* \mc{L} ) }_p)$ changes.  \end{exm}

\begin{rmk} When $f \colon X \to Y$ is flat, we will see some criteria to
understand what happens.  \end{rmk}

\section{Flatness and base change}% \label{sec:flatness_and_base_change}

\begin{defn} Let $f \colon X \to Y$ be a morphism of schemees. then $\mc{F}$ is
    \textit{flat} over $Y$ if for all $x \in X$ the stalk $\mc{F}_x$ is a flat
    $\mc{O}_{Y, f(x)}$-module. If $\mc{O}_X$ is flat over $Y$, then $f$ is said
    to be a \textit{flat morphism}.  \end{defn}

Here are some important results about flat morphisms: \begin{enumerate} \item
    Flat morphisms of locally Noetherian schemes are equidimensional. This
    means that for all $x \in X$, $\dim_x (X_y) + \dim_y Y = \dim_x X$. To
    prove this, use going down.  \item If $X$ is integral of dimension $1$ and
    $Y$ is regular, then $f$ is flat if and only if it is dominant. In fact,
    without assuming that $X$ is integral, $f$ is flat if and only if every
    associaated point of $X$ dominates $Y$.  \item If $Y$ is regular of
    dimension $1$, $p$ is a closed point, and $U \coloneqq Y \setminus p$, then
    for all $X_U \subseteq \P^n_U$ flat over $U$, there exists a \textit{flat
    limit} $\ol{X}_U \colon X \to Y$ sending $X_p$ to $p$.  \end{enumerate}

Now for any morphism $f \colon X \to Y$, we know that the \v{C}ech resolution
$\mc{C}^{\bullet}(\mc{U}, \mc{F})$ computes $R^i f_* \mc{F}$, and this is
compatible with base change to an open subset of $Y$ or with flat base change.
For an arbitrary base change, the \v{C}ech resolution does not work, but if
$\mc{F}$ is flat over $Y$, we can cook up a complex that computes cohomology
compatibly with base change.

\begin{thm} Let $f \colon X \to Y$ be a projective morphism of Noetherian
    schemes with $Y = \Spec A$. Let $\mc{F}$ be coherent on $X$. Then there
    exists a finite complex of finitely generated projective $A$-modules \[ 0
    \to K^0 \to K^1 \to \cdots \to K^N \to 0 \] such that for all $A \to A'$,
    there exists a natural isomorphism $H^i(X', u^* \mc{F}) = H^i(K^{\bullet}
    \otimes_A A')$, where \begin{equation*} \begin{tikzcd} X' \ar{r}{u}
        \ar{d}{f'} & X \ar{d}{f} \\ \Spec A' \ar{r}{v} & \Spec A.  \end{tikzcd}
    \end{equation*} \end{thm}

\begin{cor} There exists a complex $\wt{K}^{\bullet}$ of locally free coherent
    sheaves such that \[ R^i f'_* u^* \mc{F} \simeq H^i(u^* \wt{K}^{\bullet}).
    \] \end{cor}

\begin{rmk} The proof of this only uses coherence of $R^i f_* \mc{F}$ which
holds more generally for proper morphisms.  \end{rmk}

\begin{cor} The map $y \mapsto \dim_{k(Y)} H^i(X_y, \mc{F}_y)$ is upper
semicontinuous. In addition, $y \mapsto \chi(X_y, \mc{F}_y)$ is locally
constant.  \end{cor}

\begin{proof} Note that \begin{align*} h^i(X_y, \mc{F}_y) &= h^i(K^{\bullet}
\otimes k(y)) \\ &= \dim \ker (\dd^i \otimes k(y)) - \dim \Im (\dd^{i-1}
\otimes k(y)) \\ &= \dim K^i \otimes k(y) - \dim \Im \dd_y^i - \dim \Im
\dd_y^{i-1}, \end{align*} and the last two terms are lower semicontinuous
    because the $K^j$ are locally free. For the second part, note that
    \begin{align*} \chi(\mc{F}_y) &= \sum {(-1)}^i \dim K_y^i -
        \mr{rk}(\dd_y^i) - \mr{rk}(\dd_y^{i-1}) \\ &= \sum {(-1)}^i \dim K_y^i
                                                \\ &= \sum {(-1)}^i
                                            \mr{rk}(K^i), \end{align*} which is
                                        locally constant on $Y$.  \end{proof}

\begin{cor} The Hilbert polynomial is constant in flat families.  \end{cor}

\begin{cor} Assume that $Y$ is reduced. The following are equivalent:
    \begin{enumerate} \item The map $y \mapsto \dim H^i(X_y, \mc{F}_y)$ is
        constant.  \item The sheaf $R^i f_* \mc{F}$ is locally free and $R^i
        f_* \mc{F} \otimes k(y) \to H^i(X_y, \mc{F}_y)$ is an isomorphism.
    \end{enumerate} Moreover, if these conditions are satisfied, then \[
R^{i-1} f_* \mc{F} \otimes k(y) \to H^i(X_y, \mc{F}_y) \] is also an
isomorphism.  \end{cor}

\begin{proof} Use the following two facts: \begin{enumerate} \item If $\mc{F}$
    is coherent on $Y$, then $\dim_{k(y)} (\mc{F} \otimes k(y)) \equiv r$ if
    and only if $\mc{F}$ is locally free of rank $r$.  \item If $\mc{F},
    \mc{G}$ are locally free on $Y$ and $\varphi \colon \mc{F} \to \mc{G}$ is a
    morphism such that $\mr{rk}(\varphi_y) = r$ for all $y$, then locally on
    $Y$, there exists a splitting $\mc{F} = \mc{F}_1 \oplus \mc{F}_2, \mc{F} =
    \mc{G}_1 \oplus \mc{G}_2$ with all $\mc{F}_i, \mc{G}_i$ locally free such
    that \[ \varphi = \mqty( 0 & \psi \\ 0 & 0 ) \] with $\psi \colon \mc{F}_2
    \to \mc{G}_1$ an isomorphism. \qedhere \end{enumerate} \end{proof}

\begin{cor} Let $Y$ be reduced. If $H^i(X_y, \mc{F}_y) = 0$ for all $y \in Y$,
then $R^i f_* \mc{F} = 0$.  \end{cor}

\begin{cor} Let $Y$ be reduced. If $R^i f_* \mc{F} = 0$ for all $i \geq i_0$,
then $H^i(X_y, \mc{F}_y) = 0$ for all $i \geq i_0$ and $y \in Y$.  \end{cor}

In fact, we can prove a stronger result.

\begin{thm} Let $f \colon X \to Y$ be a projective morphism of fintie type
    between Noetherian schemes. Let $\mc{F}$ be flat over $Y$. Then
    \begin{enumerate} \item If $\varphi^i_Y \colon  R^i f_* \mc{F} \otimes k(y)
        \to H^i(X_y, \mc{F}_y)$ is surjective at $y$, then it is an isomorphism
        at $y$ and the same is true in a neighborhood of $y \in Y$.  \item If
        $\varphi_y^i$ is surjective, then the following are equivalent:
        \begin{enumerate} \item $\varphi_y^{i-1}$ is also surjective; \item
            $R^i f_* \mc{F}$ is locally free in a neighborhood of $y$.
    \end{enumerate} \end{enumerate} \end{thm}

\begin{cor} If $H^i(X_y, \mc{F}_y) = 0$ for all $y \in Y$, then $\varphi_y^i$
    is surjective for all $y$, and thus $\varphi_y^i$ is an isomorphism, so
    ${(R^i f_* \mc{F})}_y = 0$. This implies that $R^i f_* \mc{F} = 0$ around
    $y$, and thus $\varphi_y^{i-1}$ is also surjective.  \end{cor}

\begin{exer} Let $X, Y$ be Noetherian and $f \colon X \to Y$ be flat and
    proper. Suppose that for all $y \in Y$, $X_y \simeq \P^n_{k(y)}$. If
    $\mc{L}$ is invertible such that $\eval{\mc{L}}_{X_y} = \mc{O}_{X_y}$ for
    all $y \in Y$, then there exists an invertible sheaf $\mc{M}$ such that
    $\mc{L} = f^* \mc{M}$.

    Hint: Set $\mc{M} = f_* \mc{L}$ and prove that $f^* f_* \mc{L} \to \mc{L}$
is an isomorphism.  \end{exer}

\begin{exer} Let $X$ be Noetherian and connected. Show that $\Pic(X \times
    \P^n_{\Z}) \simeq \Pic X \times \Z$.

    Hint: Show that the map $\Pic X \times \Pic \P^n \to \Pic (X \times \P^n)$
is an isomorphism.  \end{exer}



\end{document}
