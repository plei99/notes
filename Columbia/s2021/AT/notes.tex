\documentclass[leqno, openany]{memoir}
\setulmarginsandblock{3.5cm}{3.5cm}{*}
\setlrmarginsandblock{3cm}{3.5cm}{*}
\checkandfixthelayout

\usepackage{amsmath}
\usepackage{amssymb}
\usepackage{amsthm}
%\usepackage{MnSymbol}
\usepackage{bm}
\usepackage{accents}
\usepackage{mathtools}
\usepackage{tikz}
\usetikzlibrary{calc}
\usetikzlibrary{automata,positioning}
\usepackage{tikz-cd}
\usepackage{forest}
\usepackage{braket} 
\usepackage{listings}
\usepackage{mdframed}
\usepackage{verbatim}
\usepackage{physics}
\usepackage{spectralsequences} 
%\usepackage{/home/patrickl/homework/macaulay2}

%font
\usepackage[osf]{mathpazo}
\usepackage{microtype}

%CS packages
\usepackage{algorithmicx}
\usepackage{algpseudocode}
\usepackage{algorithm}

% typeset and bib
\usepackage[english]{babel} 
\usepackage[utf8]{inputenc} 
\usepackage[backend=biber, style=alphabetic]{biblatex}
\usepackage[bookmarks, colorlinks, breaklinks]{hyperref} 
\hypersetup{linkcolor=black,citecolor=black,filecolor=black,urlcolor=black}

% other formatting packages
\usepackage{float}
\usepackage{booktabs}
\usepackage{enumitem}
\usepackage{csquotes}
\usepackage{titlesec}
\usepackage{titling}
\usepackage{fancyhdr}
\usepackage{lastpage}
\usepackage{parskip}

\usepackage{lipsum}

% delimiters
\DeclarePairedDelimiter{\gen}{\langle}{\rangle}
\DeclarePairedDelimiter{\floor}{\lfloor}{\rfloor}
\DeclarePairedDelimiter{\ceil}{\lceil}{\rceil}


\newtheorem{thm}{Theorem}[section]
\newtheorem{cor}[thm]{Corollary}
\newtheorem{prop}[thm]{Proposition}
\newtheorem{lem}[thm]{Lemma}
\newtheorem{conj}[thm]{Conjecture}
\newtheorem{quest}[thm]{Question}

\theoremstyle{definition}
\newtheorem{defn}[thm]{Definition}
\newtheorem{defns}[thm]{Definitions}
\newtheorem{con}[thm]{Construction}
\newtheorem{exm}[thm]{Example}
\newtheorem{exms}[thm]{Examples}
\newtheorem{notn}[thm]{Notation}
\newtheorem{notns}[thm]{Notations}
\newtheorem{addm}[thm]{Addendum}
\newtheorem{exer}[thm]{Exercise}

\theoremstyle{remark}
\newtheorem{rmk}[thm]{Remark}
\newtheorem{rmks}[thm]{Remarks}
\newtheorem{warn}[thm]{Warning}
\newtheorem{sch}[thm]{Scholium}


% unnumbered theorems
\theoremstyle{plain}
\newtheorem*{thm*}{Theorem}
\newtheorem*{prop*}{Proposition}
\newtheorem*{lem*}{Lemma}
\newtheorem*{cor*}{Corollary}
\newtheorem*{conj*}{Conjecture}

% unnumbered definitions
\theoremstyle{definition}
\newtheorem*{defn*}{Definition}
\newtheorem*{exer*}{Exercise}
\newtheorem*{defns*}{Definitions}
\newtheorem*{con*}{Construction}
\newtheorem*{exm*}{Example}
\newtheorem*{exms*}{Examples}
\newtheorem*{notn*}{Notation}
\newtheorem*{notns*}{Notations}
\newtheorem*{addm*}{Addendum}


\theoremstyle{remark}
\newtheorem*{rmk*}{Remark}

% shortcuts
\newcommand{\Ima}{\mathrm{Im}}
\newcommand{\A}{\mathbb{A}}
\newcommand{\N}{\mathbb{N}}
\newcommand{\R}{\mathbb{R}}
\renewcommand{\H}{\mathbb{H}}
\newcommand{\C}{\mathbb{C}}
\newcommand{\Z}{\mathbb{Z}}
\newcommand{\F}{\mathbb{F}}
\newcommand{\Q}{\mathbb{Q}}
\renewcommand{\k}{\Bbbk}
\renewcommand{\P}{\mathbb{P}}
\newcommand{\M}{\overline{M}}
\newcommand{\g}{\mathfrak{g}}
\newcommand{\h}{\mathfrak{h}}
\newcommand{\n}{\mathfrak{n}}
\renewcommand{\b}{\mathfrak{b}}
\newcommand{\ep}{\varepsilon}
\newcommand*{\dt}[1]{%
   \accentset{\mbox{\Huge\bfseries .}}{#1}}
\renewcommand{\abstractname}{Official Description}
\newcommand{\mc}[1]{\mathcal{#1}}
\newcommand{\T}{\mathbb{T}}
\newcommand{\mf}[1]{\mathfrak{#1}}
\newcommand{\mr}[1]{\mathrm{#1}}
\newcommand{\ms}[1]{\mathsf{#1}}
\newcommand{\ol}[1]{\overline{#1}}
\newcommand{\wt}[1]{\widetilde{#1}}
\newcommand{\wh}[1]{\widehat{#1}}

\DeclareMathOperator{\Der}{Der}
\DeclareMathOperator{\Sq}{Sq}
\DeclareMathOperator{\Hom}{Hom}
\DeclareMathOperator{\colim}{colim}
\DeclareMathOperator{\coker}{coker}
\DeclareMathOperator{\End}{End}
\DeclareMathOperator{\ad}{ad}
\DeclareMathOperator{\Aut}{Aut}
\DeclareMathOperator{\Rad}{Rad}
\DeclareMathOperator{\supp}{supp}
\DeclareMathOperator{\sgn}{sgn}
\DeclareMathOperator{\spec}{Spec}

% Section formatting
\titleformat{\section}
    {\Large\sffamily\scshape\bfseries}{\thesection}{1em}{}
\titleformat{\subsection}[runin]
    {\large\sffamily\bfseries}{\thesubsection}{1em}{}
\titleformat{\subsubsection}[runin]{\normalfont\itshape}{\thesubsubsection}{1em}{}

\title{COURSE TITLE}
\author{Lectures by INSTRUCTOR, Notes by NOTETAKER}
\date{SEMESTER}

\newcommand*{\titleSW}
    {\begingroup% Story of Writing
    \raggedleft
    \vspace*{\baselineskip}
    {\Huge\itshape Algebraic Topology \\ Spring 2021}\\[\baselineskip]
    {\large\itshape Notes by Patrick Lei}\\[0.2\textheight]
    {\Large Lectures by Francesco Lin}\par
    \vfill
    {\Large \sffamily Columbia University}
    \vspace*{\baselineskip}
\endgroup}
\pagestyle{simple}

\chapterstyle{ell}


%\renewcommand{\cftchapterpagefont}{}
\renewcommand\cftchapterfont{\sffamily}
\renewcommand\cftsectionfont{\scshape}
\renewcommand*{\cftchapterleader}{}
\renewcommand*{\cftsectionleader}{}
\renewcommand*{\cftsubsectionleader}{}
\renewcommand*{\cftchapterformatpnum}[1]{~\textbullet~#1}
\renewcommand*{\cftsectionformatpnum}[1]{~\textbullet~#1}
\renewcommand*{\cftsubsectionformatpnum}[1]{~\textbullet~#1}
\renewcommand{\cftchapterafterpnum}{\cftparfillskip}
\renewcommand{\cftsectionafterpnum}{\cftparfillskip}
\renewcommand{\cftsubsectionafterpnum}{\cftparfillskip}
\setrmarg{3.55em plus 1fil}
\setsecnumdepth{subsection}
\maxsecnumdepth{subsection}
\settocdepth{subsection}

\begin{document}
    
\begin{titlingpage}
\titleSW
\end{titlingpage}

\thispagestyle{empty}
\section*{Disclaimer}%
\label{sec:disclaimer}

These notes were taken during lecture using the \texttt{vimtex} package of the editor \texttt{neovim}. 
Any errors are mine and not the instructor's. 
In addition, my notes are picture-free (but will include commutative diagrams) and are a mix of my mathematical style and that of the instructor.
If you find any errors, please contact me at \texttt{plei@math.columbia.edu}.
\newpage


\tableofcontents

\chapter{Spectral Sequences}%
\label{cha:spectral_sequences}

Our goal in this chapter will be to compute the homotopy groups of spheres, but we are not algebraic topologists so we don't actually care about that. What we know from the basic theory is that $\pi_i(S^n) = 0$ if $i < n$ and $\pi_n(S^n) = \Z$, and this isomorphism is given by the degree. We also know that $\pi_3(S^2) = \Z$ and is generated by the Hopf fibration $S^1 \hookrightarrow S^3 \to S^2$. Note that this map is the attaching map of the $4$-cell of $\C\P^2$.

There is an analogous decomposition $\H\P^2 = \mr{pt} \cup D^4 \cup D^8$. Then we obtain a fibration $f \colon S^7 \to S^4$, which is an $S^3$-fibration. Using the long exact sequence of the fibration, we see that $\pi_7(S^4) \supseteq \Z$.

Next, we may consider $\mathbb{O}\P^2$, the octonionic projective plane. This has a cell decomposition $\mr{pt} \cup D^8 \cup D^{16}$ and from this we obtain a fibration $S^7 \hookrightarrow S^{15} \to S^8$ and compute that $\pi_{15}(S^8) \supseteq \Z$. We should note, however, that $\mathbb{O}\P^2$ is \textbf{not} the set of lines in $\mathbb{O}^3$ because the octonions do not form an associative algebra. There is enough associativity to define some sort of $\mathbb{O}\P^2$ but not $\mathbb{O}\P^n$ for $n \geq 3$. In fact, we will prove that no space with the expected cohomology ring exists.

More generally, we can study $f \colon S^{2n-1} \to S^n$ as follows: Consider the mapping cone $S^n \cup_f D^{2n} \eqqcolon C_f$. Then the cohomology is
\[ H^i(C_f) = \begin{cases}
    \Z & i = 0, n, 2n \\
    0 & \text{otherwise}
\end{cases}. \]
Then choose generators $\alpha \in H^n, \beta \in H^{2n}$. We then have $\alpha^2 = H(f) \beta \in H^{2n}$ for some integer $H(f)$, called the \textit{Hopf invariant}.

\begin{exm}
    The attaching maps of $\C\P^2, \H\P^2, \mathbb{O}\P^2$ all have $H(f) = 1$. This implies that $H^* = \Z[\alpha]/(\alpha^3)$.
\end{exm}

We know that if $f,g$ are homotopic, then $H(f) = H(g)$ because the mapping cones are homotopy equivalent. Then we obtain a homomorphism $H \colon \pi_{2n-1}(S^n) \to \Z$. Note that when $n$ is odd, we have $\alpha^2 = -\alpha^2$ by graded commutativity, and so $H(f) = 0$.

\begin{exm}
    If $n = 2k$ is even, there exists $f \colon S^{4k-1} \to S^{2k}$ with $H(f) = 2$.
\end{exm}

Now given a pointed space $(X,e)$ define $J_2(X) = X \times X / (x,e) \sim (e,x)$. This is somewhere between the product and the smash product, and so $J_2(S^{2k})$ is a CW complex with a single cell in dimensions $0, 2k, 4k$.

\begin{exer}
    The attaching map of the $4k$-cell has $H(f) = 2$.
\end{exer}

\begin{cor}
    $\pi_{4k-1}(S^{2k})$ admits a $\Z$-summand. This is because $H$ surjects onto either $2\Z$ or $\Z$, and so it splits.
\end{cor}

We have the following results:

\begin{thm}[Serre]
    $\pi_i(S^n)$ is a finitely generated abelian group. It has rank $1$ if $i = n$ or $i = 2n-1$ and $n$ is even.
\end{thm}

\begin{thm}[Adams]
    If $[f] \in \pi_{2n-1}(S^n)$ with $H(f) = 1$, then $n = 2,4,8$.
\end{thm}

\begin{rmk}
    This is related to the following. Suppose $\R^n$ admits the structure of a division algebra. This is a bilinear map $* \colon \R^n \otimes \R^n \to \R^n$ that is invertible for $a \neq 0$. Then $n = 1,2,4,8$. We will prove this result using K-theory.
\end{rmk}

\section{The Simplest Case}%
\label{sec:the_simplest_case}

Here, we will consider $\pi_{n+1}(S^n)$ for $n \geq 3$. By the Freudenthal suspension theorem, we have maps
\[ \pi_3(S^2) \xrightarrow{\Sigma} \pi_4(S^3) \xrightarrow[\sim]{\Sigma} \pi_5(S^4) \to \cdots \]
Therefore the group we need to compute is $\pi_4(S^3)$. The key strategy is to exploit the interaction between homotopy and homology. Here, we will consider the \textit{Hurewicz map}
\[ h \colon \pi_n(X) \to H_n(X) \qquad [f] \mapsto f_* [S^n]. \]

\begin{thm}[Hurewicz]
    Suppose $n \geq 2$ and $X$ is $(n-1)$-connected. Then $\wt{H}_0(X) = H_1(X) = \cdots = H_{n-1}(X) = 0$ and $h \colon \pi_n(X) \simeq H_n(X)$.
\end{thm}

\begin{rmks}
    When $n = 1$, $H_1(X)$ is the abelianization of $\pi_1(X)$. Also, there is a relative version of the Hurewicz theorem.
\end{rmks}

\begin{proof}[Sketch of Proof]
    We can assume $X$ is a CW complex with a single $0$-cell and no cells in dimension $1, \ldots, n-1$. Then we can replace $X$ with $X^{n+1}$, so we can write
    \[ X = \qty( \bigvee_{\alpha} S_{\alpha}^n ) \cup_{\beta} D_{\beta}^{n+1}. \]
    By homotopy excision, we have
    \[ \pi_n(X) = \mr{coker}(d \colon \pi_{n+1}(X, X^n) \to \pi_n(X^n)) \]
    and this is exactly $C_{n+1}(X) \to C_n(X) \to 0$.
\end{proof}

Our strategy to compute $\pi_4(S^3)$ will be to construct a space whose homology is $\pi_4(S^3)$. Recall that $\pi_3(K(\Z, 3)) = \Z$ by definition, and so consider $f \colon S^3 \to K(\Z, 3)$ be a generator. Then $f_* \colon \pi_3(S^3) \xrightarrow{\sim} \pi_3(K(\Z, 3))$ is an isomorphism. Now we can turn $f$ into a fibration $F \hookrightarrow S^3 \to K(\Z, 3)$. Considering the long exact sequence of the fibration, we have
\[ 0 \to \pi_4(F) \to \pi_4(S^3) \to 0 \to \pi_3(F) \to \pi_3(S^3) \to \pi_3(K(\Z, 3)) \to \cdots \]
and therefore $\pi_0(F) = \pi_1(F) = \pi_2(F) = \pi_3(F) = 0$. We also have $\pi_n(F) \cong \pi_n(S^3)$ for $n \geq 4$. In particular, we have $\pi_4(S^3) = \pi_4(F) = H_4(F)$.If we can understand $F$ in some reasonable way, then we will be done.

If we turn $F \hookrightarrow S^3$ into a fibration, then the homotopy fiber is now a $K(\Z, 2) = \C\P^{\infty}$. Now we have a fibration and we know the homology of both $S^3$ and $\C\P^{\infty}$, so the question is to compute $H_4$ from the information we have.

\begin{quest}
    Given $F \hookrightarrow E \to B$ a fibration, is there a relationship between $H_*(B), H_*(E), H_*(F)$?
\end{quest}

Recall that in the case when $E \cong F \times B$ this relation is given by the K\"unneth formula. Here, we see that $C_*(E) = C_*(F) \otimes C_*(B)$ as chain complexes, and so we reduce the problem to homological algebra. This gives us

\begin{thm}
    Let $R$ be a principal ideal domain. Then there is a natural short exact sequence
    \[ 0 \to \bigoplus H_i(F,R) \otimes H_{n-i}(B, R) \to H_n(F \times B, R) \to \bigoplus \mr{Tor}^1_R(H_i(F,R), H_{n-1-i}(B, R)) \to 0 \]
    and this sequence splits (but not naturally).
\end{thm}

\begin{exm}
    We can compute that $H_*(S^1 \times S^2) = \Z$ in dimensions $0,1,2,3$.
\end{exm}

However, we note that the Kunneth theorem does not hold for fibrations in general. To see this, consider the Hopf fibration.

\begin{exm}
    Consider the fibration $K(\Z, n-1) \hookrightarrow * \to K(\Z, n)$. Both the base and fiber have nontrivial homology, but clearly the total space is contractible, so it has trivial homology. In both examples, $H_*(E)$ is \textbf{smaller} than what we would get from K\"unneth. 
\end{exm}

\section{Spectral Sequences}%
\label{sec:spectral_sequences}

\begin{defn}
    A \textit{spectral sequence} is a sequence $(E^r, d_r)$, where $E^r$ is an $R$-module and $d_r \colon E^r \to E^r$ is a differential. In addition, we require that $E^{r+1} = H_*(E^r, d_r)$. 
\end{defn}

\begin{rmk}
    Note that $(E^r, d_r)$ determines $E^{r+1}$, but not $d_{r+1}$ in general.
\end{rmk}

Now assume that $d_r = 0$ for $r \gg 0$ so that $E^r = E^{r+1} = \cdots \eqqcolon E^{\infty}$.

\begin{defn}
    We say that $(E^r, d_r) \Rightarrow G$, or $(E_r, d_r)$ \textit{converges to $G$}, if $G$ admits a filtration
    \[ 0 = G_{-1} \subset G_0 \subset G_1 \subset \cdots \subset G_n = G \] such that $\bigoplus G_i / G_{i-1} \cong E^{\infty}$.
\end{defn}

\begin{rmk}
    This says that $G$ is recovered by $(E^r, d_r)$ up to extension problems.
\end{rmk}

\begin{exm}
    Consider a short exact sequence $0 \to A_* \to C_* \to C_*/A_* \to 0$ of chain complexes. Then we will see that there is a spectral sequence with $E^1 = H_*(A) \oplus H_*(C/A)$ and $(E^r, d_r) \Rightarrow H_*(C)$. 

    First, consider the long exact sequence in homology. If we consider the boundary homomorphism $\delta_i$, we obtain a long exact sequence
    \[ 0 \to \operatorname{coker} \delta_{i+1} \to H_i(C) \to \ker \delta_i \to 0. \]
    Now define $E^1 = H_*(A) \oplus H_*(C / A)$ with $d_1 = \bigoplus \delta_i$. Then we have 
    \[ H_*(E^1, d_1) = \qty( \bigoplus \ker \delta_i ) \oplus \qty(\bigoplus \operatorname{coker} \delta_i) \eqqcolon E^2, \]
    as desired. Then we let $d_2 = 0$.
\end{exm}

This tells us that if $H_*(A) = H_*(C/A) = 0$, then $H_*(C) = 0$. Also, if $H_*(C) = 0$, then $H_*(A) \cong H_*(C/A)$ with a shift.

\begin{thm}[Serre]
    Let $F \hookrightarrow E \to B$ be a fibration and let $\pi_1(B) = 1$. Then there is a spectral sequence $(E^r, d_r)$ such that
    \begin{itemize}
        \item $E^2 \cong H_*(B; H_*(F))$;
        \item $(E^r, d_r) \Rightarrow H_*(E)$.
    \end{itemize}
\end{thm}

Note here that we need to define what it means to converge in this setting. Also, note that $E^3$ is smaller than $E^2$, so this formalizes the notion that $H_*(E)$ is smaller than what we naively expect.

\begin{rmk}
    We actually do not know what $d_2$ is, so it is impossible to compute the homology in general.
\end{rmk}

Fortunately, there is more structure:
\begin{itemize}
    \item $E^r_{*,*}$ is bigraded.
    \item $d_r$ has bidegree $(-r, r-1)$.
    \item $E_{p,q}^2 = H_p(B; H_q(F))$.
    \item $E_{p,q}^r = E_{p,q}^{\infty}$ for $r \gg 0$ depending on $p,q$.
    \item $H_n(E)$ has a filtration with associated graded $\bigoplus E_{i, n-i}^{\infty}$.
\end{itemize}

\begin{exm}
    Consider the example $S^1 \hookrightarrow * \to \C\P^{\infty}$. Then we know that $E_{p,q}^2 = H_p(B, H_q(S^1))$ and therefore the spectral sequence is
    \begin{center}
    \begin{sseqdata}[classes={draw=none},name=cpinf,homological Serre grading,xscale=1.8, y axis gap = 2em]
        \class["H_0(B)"](0,0)
        \class["H_1(B)"](1,0)
        \class["H_2(B)"](2,0)
        \class["H_3(B)"](3,0)
        \class["\ldots"](4,0)
        \class["H_0(B)"](0,1)
        \class["H_1(B)"](1,1)
        \class["H_2(B)"](2,1)
        \class["H_3(B)"](3,1)
        \class["\ldots"](4,1)
        \d[->]2(2,0)
        \d[->]2(3,0)
        \d[->]2(4,0)
    \end{sseqdata}
    \printpage[name=cpinf,page=2]
    \end{center}
    This gives us $\delta_i \colon H_{2+i}(B) \to H_i(B)$, and so $d_3$ has degree $(-3, 2)$ and thus it has to be zero. By degree reasons, we see that $E^4 = E^3$ and $d_4 = 0$, so $E^{\infty} = E^3$. Then, the total space is a point, and so writing the $E^3$-page
    \begin{center}
        \begin{sseqdata}[classes={draw=none},name=cpinf3,homological Serre grading,xscale=1.8, y axis gap = 2em]
            \class["\coker \delta_0"](0,1)
            \class["\coker \delta_1"](1,1)
            \class["\coker \delta_2"](2,1)
            \class["H_0(B)"](0,0)
            \class["H_1(B)"](1,0)
            \class["\ker \delta_0"](2,0)
            \class["\ker \delta_1"](3,0)
            \class["\ker \delta_2"](4,0)
        \end{sseqdata}
        \printpage[name=cpinf3, page=3]
    \end{center}
    we see that $H_0(B) = \Z, H_1(B) = 0$ and that $\ker \delta_i = \operatorname{coker} \delta_i = 0$. This recovers the usual computation of the homology of $\C\P^{\infty}$.
\end{exm}

\begin{exm}
    We want to compute $H_*(\Omega S^2)$. Here, we will use the fiber sequence $\Omega S^2 \hookrightarrow * \to S^2$. Here, we know that $E_{p,q}^2 = H_p(S^2, H_q(F))$, so the $E^2$-page looks like
    \begin{center}
        \begin{sseqdata}[classes={draw=none}, name=omegas2, homological Serre grading, xscale=1.5, y axis gap = 2em]
            \class["H_0(F)"](0,0)
            \class["H_1(F)"](0,1)
            \class["H_2(F)"](0,2)
            \class["H_3(F)"](0,3)
            \class["H_0(F)"](2,0)
            \class["H_1(F)"](2,1)
            \class["H_2(F)"](2,2)
            \class["H_3(F)"](2,3)
            \d["g_0", ->]2(2,0)
            \d["g_1", ->]2(2,1)
            \d["g_2", ->]2(2,2)
        \end{sseqdata}
        \printpage[name=omegas2, page=2]
    \end{center}
    Then the $E^3$-page looks like
    \begin{center}
        \begin{sseqdata}[classes={draw=none}, name=omegas23, homological Serre grading, xscale=1.5, y axis gap = 2em]
            \class["H_0(F)"](0,0)
            \class["\coker g_0"](0,1)
            \class["\coker g_1"](0,2)
            \class["\coker g_2"](0,3)
            \class["\ker g_0"](2,0)
            \class["\ker g_1"](2,1)
            \class["\ker g_2"](2,2)
            \class["\ker g_3"](2,3)
        \end{sseqdata}
        \printpage[name=omegas23, page=3]
    \end{center}
    and the differential has degree $(-3,2)$, so $d_3 = 0$. This implies that $E^{\infty} = E_3$, and so the associated graded pieces are $\Z,0,0,0$. We see that $H_0(F) = \Z$ and each $g_i$ is an isomorphism. This tells us that $H_i(\Omega S^2) \cong \Z$ for all $i$.
\end{exm}

\begin{exm}
    We can enhance this example to $\Omega S^n \hookrightarrow * \to S^n$. In this case, the first possible nontrivial differential is $d_n$, which has degree $(-n, n-1)$. This gives us $E^{n+1} = E^{\infty}$ for degree reasons, so we can compute
    \[ H_i(\Omega S^n) = \begin{cases}
        \Z & (n-1) \mid i \\
        0 & \text{otherwise}.
    \end{cases} \]
    because $\delta_i \colon H_i(F) \to H_{i+n-1}(F)$ is an isomorphism.
\end{exm}

Recall that our goal was to study $\C\P^{\infty} \hookrightarrow F \to S^3$. However, we cannot compute this yet, so we will need to introduce even more structure. To do this, we will need to do some homological algebra.

Considered a filtered chain complex, which is an abelian group $C$ with a map $d \colon C \to C$ such that $d^2 = 0$. To this, we attach a filtration $0 = F_{-1} C \subseteq F_0 C \subseteq \cdots \subseteq F_n C = C$ such that $d(F_i C) \subseteq F_i C$.

\begin{rmk}
    $(F_i C, D)$ is a subcomplex of $C$.
\end{rmk}

Now if $X$ is a topological space with filtration $X_{-1} = \emptyset \subset X_0 \subset \cdots \subset X_n = X$. Then if $C_*(X)$, we can choose $F_i C = C_*(X_i)$, and this is a filtration.

Given $(C,d)$ and a filtration, we can associate two objects:
\begin{enumerate}
    \item $\mr{Gr}_* C = \bigoplus F_i C / F_{i-1} C$, the \textit{associated graded} complex. Then we may consider the homology of this complex, which is a graded object.
    \item Given a subcomplex $(F_i C, d) \hookrightarrow (C,d)$, we have a map $H(F_i C, d) \to H(C, d)$. The image of this is $F_i H(C,d)$, so we get a filtration on the homology $H(C)$. So then we obtain a graded
        \[ \mr{Gr}_* H(C,d) = \bigoplus F_i H(C) / F_{i-1} H(C). \]
\end{enumerate}
Now we want to consider therelationship between the two. Note that the first one is bigger, because $x \in \mr{Gr}_i C$ is a cycle if $\dd{x} \in F_{i-1} C$, while $x \in C$ is a cycle if $\dd{x} = 0$.

\begin{thm}[Leray]
    There exists a spectral sequence $(E_*^r, d_r)$ such that
    \begin{itemize}
        \item $E_*^1 \cong H_* (\mr{Gr}_* C, d_0)$.
        \item The spectral sequence converges to $H(C)$, or more precisely, $E_*^{\infty} \cong \mr{Gr}_* H(C)$.
    \end{itemize}
\end{thm}

\begin{proof}
    Consider the group
    \[ E_p^r = \frac{F_p C \cap \dd^{-1}(F_{p-r} C)}{(F_{p-1} C \cap \dd^{-1} F_{p-r} C) + (F_p C \cap \dd(F_{q+r-1} C))}. \]
    Notice that
    \begin{align*}
        E_p^0 &= \frac{F_p C}{F_{p-1} C + d(F_{p-1} C)} = \frac{F_p C}{F_{p-1} C} \\
        E_p^1 &= \frac{F_p C \cap d^{-1} F_{p-1} C}{F_{p-1}C + d(F_p C)} = H_p(\mr{Gr}_* C, \dd_0) \\
        E_p^{\infty} &= \frac{F_p C \cap \ker \dd}{(F_{p-1} \cap \ker d) + (F_p C \cap \Im d)} = \frac{F_p H(C)}{F_{p-1} H(C)},
    \end{align*}
    so all the groups are as expected. For the differential, we define $\dd{r} \colon E_*^r \to E_r^*$ and in fact $\dd_r \colon E_p^r \to E_{p-r}^r$ and thus has degree $-r$. Choose $\alpha \in E_p^r$ and choose $a \in F_p C \cap \dd^{-1}(F_{p-r} C)$ a representative. Then $\dd{\dd{a}} = 0$ implies that $\dd{a} \in \dd^{-1}(0) \subset \dd^{-1}(F_{p-2r} C)$ and therefore $\dd{a} \in F_{p-r} C \cap d^{-1}(F_{p-2r} C)$. Therefore $\dd{a}$ defines an element in $E_{p-r}^r$, so we set $\dd^r \alpha = [\dd{a}] \in E_{p-r}^r$.

    The things that need to be checked are that this is well-defined, ${(\dd^r)}^2 = 0$, and that $H_*(E_*^r, \dd{r}) \cong E_*^{r+1}$ canonically. All of this is painful homological algebra and is omitted.
\end{proof}

However, we will need something even more painful. Now we will considered a filter graded chain complex $(C_*, \dd)$, where $\dd$ has degree $-1$. Now for a filtration, we can consider the bigraded
\[ \mr{Gr}_{*,*} = \bigoplus_{p,m} \frac{F_p C_m}{F_{p-1} C_m}. \]
Also, we note that $H_m(C)$ is naturally filtered with $F_p H_m(C) = \Im(H_m (F_p C) \to H_m(C))$.

\begin{thm}[Leray]
    There is a spectral sequence $(E_{*,*}^r, \dd{r})$ such that
    \begin{itemize}
        \item $E^1_{*,*} \cong H_{*,*}(\mr{Gr} C)$.
        \item The spectral sequence converges to $H(C)$, where
            \[ E_{p,q}^{\infty} = \frac{F_p H_{p+q}(C)}{F_{p-1}H_{p+q}(C)}. \]
    \end{itemize}
\end{thm}

\begin{proof}
    We write
    \[ E_{p,q}^r = \frac{F_p C_{p+q} \cap \dd^{-1} (F_{p-r} C_{p+q-1})}{[F_{p-1} C_{p+q} \cap \dd^{-1}(F_{p-q}C_{p+q-1})] + [F_p C_{p+q} \cap \dd(F_{p+r-1} C_{p+q+1})]}. \]
    The differential has degree $(-r, r-1)$. Note that it decreases the filtration by $r$ and the total degree by $1$. For $\alpha \in E_{p,q}^r$, choose a representative $a \in F_p C_{p+q}$ with $\dd{a} \in F_{p-r} C_{p+q-1}$. Then $\dd^2 a = 0$ and therefore $\dd{a} \in \dd^{-1}(0) \subseteq \dd^{-1}(F_{p-2r} C_{p+q-1})$, so we set $\dd^r \alpha = [\dd{a}] \in E_{p-r, q+r-1}^r$.
\end{proof}

Returning to topology, consider a filtration $\emptyset = X_{-1} \subseteq X_0 \subseteq \cdots \subseteq X_n = X$, which givs a filtration on $C_*(X)$. This now defines a spectral sequence by setting $F_p C_{p+q}(X) = C_{p+q}(X_p)$. The $E^0$ page of this is simply
\[ E_{p,q}^0 = \frac{F_p C_{p+q}}{F_{p-1} C_{p+q}} = \frac{C_{p+q}(X_p)}{C_{p+q}(X_{p-1})} = C_{p+q}(X_p, X_{p-1}) \]
and $d_{p,q}^0$ is $\partial \colon C_{p+1} \to C_{p+q-1}$.

The $E^1$-page of the spectral sequence is
\[ E^1_{p,q} = H_{p+q}(X_p, X_{p-1})\] 
and $d^1 \colon H_{p+1}(X_p, X_{p-1}) \to H_{p+q-1}(X_{p-1}, X_{p-2})$ is the connecting homomorphism for the triple $(X_p, X_{p-1}, X_{p-2})$. 

The $E^{\infty}$-page of the spectral sequence is
\[ E_{p,q}^{\infty} = \frac{\Im(H_{p+q}(X_p) \to H_{p+q}(X))}{\Im(H_{p+1}(X_{p-1}) \to H_{p+1}(X))}, \]
which is the associated graded of $H_{p+q}(X)$ for the filtration $X_p \hookrightarrow X$.

\begin{exm}
    Consider the cellular homology of $X$. Set $X_p = X^p$, the $p$-skeleton. Then we note that $E_{p,q}^1$ is the homology $H_{p+1}(X_p, X_{p-1})$, and now if $\mc{C}_p(X)$ is the cellular complex, the $E^2$-page is precisely the cellular homology of $X$ and is the same as the $E^{\infty}$-page. This proves that cellular homology is the same as singular homology.
\end{exm}

\begin{rmk}
    This discussion works for infinite filtrations as long as $X = \bigcup X_n$ and $X$ has the weak topology induced by the filtration.
\end{rmk}

If you are interested in working through the pain\footnote{Francesco's words} of spectral sequences, the book \textit{A User's Guide to Spectral Sequences} is a good resource. In this course, we will not open this Pandora's box.

\begin{proof}[Proof of Serre Spectral Sequence]
    Consider $F \hookrightarrow E \xrightarrow{\pi} B$ and assume $\pi_1(B) = 1$. For simplicity, assume that $B$ is a CW complex and $\pi$ is a fiber bundle. Now $B$ has a filtration by skeleta
    \[ \varphi = B_{-1} \subseteq B_0 \subseteq \cdots \]
    and therefore $E$ has a filtration induced by pulling back $\pi$. Therefore there is a spectral sequence converging to $H_*(E)$ with 
    \[ E_{p,q}^1 = H_{p+q}(\pi^{-1}(B_p), \pi^{-1}(B_{p-1})). \]
    We want to compute the $E^2$-page, so by excision we have
    \[ E_{p,q}^1 = \bigoplus_{\text{$p$-cells}} H_{p+q}(\pi^{-1}(e_i), \pi^{-1}(\partial e_i)). \]
    By contractibility of $e_i$, so 
    \[ H_{p+q}(\pi^{-1}(e_i), \pi^{-1}(\partial e_i)) \cong H_{p+q}(D^p \times F, S^{p-1} \times F) = H_p(F). \]
    Unfortunately, this identification is not canonical, which is why we need the assumption that $B$ is simply-connected. In general, for a path $\gamma$, transport (homotopy lifting) gives us a map $\gamma_* \colon H_*(F_0) \simeq H_*(F_1)$ depending only on the relative homotopy class. For example, if we consider $S^1 \xrightarrow{2} S^1$, then the two paths between $p_0, p_1$ give different identifications. Therefore, we have a \textbf{canonical} identification
    \[ E_{p,q}^1 = \bigoplus_{\text{$p$-cells}} H_q(F) = \mc{C}_p(B, H_q(F)). \]
    Finally, we note that $d_{p,q}^1 \colon E_{p,q}^1 \to E_{p-1,q}^1$ is exactly the cellular boundary map $\partial \times 1_{H_q(F)}$, so $E_{p,q}^2 = H_p(B; H_q(F))$, as desired.
\end{proof}

\begin{rmk}
    The theorem holds provided the action of $\pi_1(B)$ on $H_*(F)$ is trivial, which means that the fibration is \textit{homologically simple}. 
\end{rmk}

\begin{exm}
    Consider a sphere bundle $S^n \hookrightarrow E \to B$. This is homologically simple if and only if it is orientable.
\end{exm}

Even more generally, we need to use \textit{homology with local coefficients}. Here, we will take
\[ E_{p,q}^1 = \bigoplus_{\text{$p$-cells}} H_q(F_i) \]
and $d_{p,q}^1$ will have components given by composing $\delta$ with transport. An alternative interpretation is to consider the universal cover $\wt{B} \to B$. Now $\pi_1(B)$ acts on $\mc{C}_*(\wt{B})$ and on $H_*(F)$, so we obtain modules over the group ring $\Z[\pi_1(B)]$. Therefore we have
\[ E^1_* = \mc{C}_*(\wt{B}) \otimes_{\Z[\pi_1(B)]} H_*(F). \]

Now we will use groups on the edges of each page to compute $H_*(F) \to H_*(E). H_*(E) \to H_*(B)$.  Degree reasons tell us that $H_n(F) = E_{0,n}^2 \twoheadrightarrow E_{0,n}^3 \twoheadrightarrow \cdots \twoheadrightarrow E_{0,n}^{\infty} \subseteq H_n(E)$.

\begin{prop}
    $E_{0,n}^{\infty} \subseteq H_n(E)$ is the image of $H_n(F) \to H_n(E)$.
\end{prop}

Similarly, we can consider the groups $H_0(B), \ldots, H_n(B)$ on the bottom of the $E^2$ page. This tells us that $H_n(B) = E_{n,0}^2 \supseteq \cdots \supseteq E_{n,0}^{\infty}$, which is a quotient of $H_n(E)$.

\begin{prop}
    $E_{n,0}^{\infty} \subseteq H_n(B)$ is the image of $H_n(E) \to H_n(B)$.
\end{prop}

Returning to algebra, suppose we have a chain map $f_* \colon (C_*, d) \to (\wt{C}_*, \wt{d})$ that preserves filtration. This gives a morphism of spectral sequences $f_{*,*}^r \colon (E_{*,*}^r, d^r) \to (\wt{E}_{*,*}^r, \wt{d}^r)$ such that $f^1, f^{\infty}$ are the maps on the associated graded objects and $f^r$ is a chain map such that the induced map on homology is $f^{r+1}$.

\begin{exm}
    An example here is filtered spaces with $f \colon X \to \wt{X}$ such that $f(X_p) \subseteq \wt{X}_p$.
\end{exm}

Note that the propositions follow by looking at the maps on spectral sequences induced by
\begin{equation*}
\begin{tikzcd}
    F \arrow[hookrightarrow]{r} \arrow{d} & E \arrow{d}{\pi} & E \arrow{r}{\pi} \arrow{d}{\pi} & B \arrow{d} \\
    {*} \arrow{r} & B & B \arrow{r} & B.
\end{tikzcd}
\end{equation*}

Now consider the map $d_m \colon E_{m,0}^m \to E_{0,m-1}^m$. Note that $E_{0,m-1}^m$ is a quotient of $H_{m-1}(F)$ and $E_{m,0}^m$ is a subgroup of $H_m(B)$, so we have $d_m \colon H_m(B) \dashrightarrow H_{m-1}(F)$, which is the analogue of $\pi_m(B) \xrightarrow{\partial_x} \pi_{m-1}(F)$.

\begin{defn}
    We call this the \textit{transgression map}.
\end{defn}

\begin{prop}
    The transgression is given by $H_m(B) = H_m(B, \mr{pt}) \dashrightarrow H_m(E, F) \to H_{m-1}(F)$.
\end{prop}

\begin{exm}
    Let $X^n$ be a closed smooth oriented manifold. Then there is a fibration $S^{n-1} \hookrightarrow SX \to X$, where $SX$ is the unit sphere bundle. This is homologically simple, and then we have $d_n \colon H_n(X) \to H_{n-1}(S^{n-1})$, which is multiplication by $\chi(X)$.
\end{exm}

\begin{rmk}
    We can prove this result if we know the relationship between $\chi$ and zeroes of vector fields.
\end{rmk}

\begin{exm}
    Consider the fibration $\Omega X \to * \to X$. If $X$ is $(n-1)$-connected, then $\Omega X$ is $(n-2)$-connected. Then $H_n(X), H_{n-1}(\Omega X)$ are the first possibly nonzero homology groups. Now we consider the spectral sequence
    \begin{center}
    \begin{sseqdata}[name=lol,homological Serre grading, classes = {draw=none}, y axis gap = 3em]
        \class["\Z"](0,0)
        \class(0,1)
        \class(1,0)
        \class["H_{n-1}(\Omega X)"](0,3)
        \class["H_n(X)"](4,0)
        \d[->]4(4,0)
    \end{sseqdata}
    \printpage[name=lol,page=4]
    \end{center}
    and thus the map $d_r = \tau \colon H_r(X) \to H_{r-1}(\Omega X)$ is an isomorphism for $r \leq 2n-2$.
\end{exm}

\begin{rmk}
    We can interpret this very explicitly as $\pi \colon \Sigma \Omega X \to X$, and this is the inverse of
    \[ H_{r-1}(\Omega X) \xrightarrow{\Sigma} H_r(\Sigma \Omega X) \xrightarrow{\pi_*} H_r(X). \]
\end{rmk}

\section{Spectral Sequences in Cohomology}%
\label{sec:spectral_sequences_in_cohomology}

\begin{thm}
    Let $F \hookrightarrow E \to B$ and $\pi(B) = 1$. Then there exists $(E_r^{*,*}, d_r) \Rightarrow H^*(E)$ and $E_2^{p,q} = H^p(B, H^q(F))$ and $d_r$ has degree $(r, 1-r)$. Furthermore,
    \begin{enumerate}
        \item There is a multiplication $E_{r}^{p,q} \otimes E_r^{p',q'} \to E_r^{p+p',q+q'}$
        \item $d_r \colon E_r^{*,*} \to E_r^{*,*}$ is a derivation, which means
            \[ d_r(\alpha \cdot \beta) = d_r(\alpha) \cdot \beta + {(-1)}^{p+q} \alpha \cdot d_r(\beta). \]
        \item The multiplication on $E_{r+1}^{*,*}$ is induced by the one on $E_r^{*,*}$.
        \item The multiplication on $E_{\infty}$ is compatible with the one on $H^*(E)$. This means that $E_{\infty}$ is the associated graded of the cohomology.
    \end{enumerate}
\end{thm}

\begin{exm}
    Consider $\C\P^{\infty} \hookrightarrow F \to S^3$. We want to compute $H_4(F)$, which will compute $\pi_4(S^3)$. Now the $E_2$-page of the spectral sequence for cohomology looks like
    \begin{center}
    \begin{sseqdata}[cohomological Serre grading, name = sphere, classes = {draw=none}]
        \class["\Z_{x^3}"](0,6)
        \class["\Z_{x^2}"](0,4)
        \class["\Z_{x}"](0,2)
        \class["\Z_1"](0,0)
        \class["\Z_y"](3,0)
        \class["\Z_{ xy }"](3,2)
        \class["\Z_{ x^2y }"](3,4)
        \class["\Z_{x^3y}"](3,6)
        \d[->]3(0,6)
        \d[->]3(0,4)
        \d[->]3(0,2)
    \end{sseqdata}
    \printpage[name=sphere,page=3]
    \end{center}
    and now $d_3(x) = y$ because $H^2(F) = H^3(F) = 0$. But then 
    \[ d_3(x^2) = (d_3 x) x + x (d_3 x) = yx + xy = 2xy, \qquad d_3(x^m) = m x^{m-1}y. \]
    This tells us that the $E_4$-page is
    \begin{center}
    \begin{sseqdata}[cohomological Serre grading, name = sphere4, classes = {draw=none}]
        \class["\Z"](0,0)
        \class["\Z_2"](3,2)
        \class["\Z_3"](3,4)
    \end{sseqdata}
    \printpage[name=sphere4,page=3]
    \end{center}
    and therefore $H^1(F) = \cdots = H^4(F) = 0$ and $H^5(F) = \Z_2$. By the universal coefficients theorem, we see that $H_4(F) = \Z_2$.
\end{exm}

\begin{cor}
    For $n \geq 3$, we have $\pi_{n+1}(S^n) = \Z/2\Z$ and the group is generated by the suspension of the Hopf map.
\end{cor}

\begin{thm}
    $H^*(SU(n), \Z) \cong \bigwedge_{\Z}[x_3, \ldots, x_{2n-1}]$ and $\abs{x_i} = i$.
\end{thm}

\begin{proof}
    Let $n = 2$. We know that $SU(2) \cong S^3$. Now for general $n$, consider the fiber bundle $SU(n-1) \hookrightarrow SU(n) \to S^{2n-1}$. Now assume that $H^*(SU(n-1)) = \bigwedge_{\Z} [x_3, \ldots, x_{2n-3}]$. Then the $E_{2n-1}$-page of the spectral sequence is
    \begin{center}
    \begin{sseqdata}[name=sun, classes = {draw=none}, cohomological Serre grading, y axis gap = 4em, x axis extend end = 4em]
        \class["H^*(SU(n-1))"](0,6)
        \class["H^*(SU(n-1))"](7,0)
        \d[->]7(0,6)
    \end{sseqdata}
    \printpage[name=sun, page=7]
    \end{center}
    and therefore we see that $d_{2n-1}(x_3) = \cdots = d_{2n-1}(x_{2n-3}) = 0$ by grading reasons. In particular, we see that $d_{2n-1}(x_3 x_5) = 0$, and in general we see that $d_{2n-1} = 0$. Therefore we see that $E_{\infty} = H^*(SU(n-1)) \otimes S^*(S^{2n-1}) = \bigwedge_{\Z} [x_3, \ldots, x_{2n-1}]$.

    To conclude, we know that $E_{\infty} = \mr{Gr} H^*(E)$, so for some $x_j \in E_{\infty}$, we can choose a lift in $H^*(E)$. Because $H^*$ is torsion free and for degree reasons, the lifts satisfy the desired identities.
\end{proof}

\begin{rmk}
    The ring $E^{\infty}$ does not determine $H^*(E)$. There are $S^2 \hookrightarrow E \to S^2$ and $S^2 \hookrightarrow E' \to S^2$ with $H^*(E) \cong H^*(E')$. For example, we can consider $\P^1 \times \P^1$ and $\operatorname{Bl}_1 \P^2$ (the intersection forms are different).
\end{rmk}

\begin{exm}
    Let $\operatorname{char} k = 0$. Then $H^*(SO(2m+1), k) \cong H^*(S^3 \times S^7 \times \cdots \times S^{4m-1})$ and $H^*(SO(2m), k) \cong H^*(S^3 \times \cdots \times S^{4m-5} \times S^{2m-1})$. However, we have
    \[ H^*(SO(n), \Z_2) \cong H^*(S^1 \times S^2 \times \cdots \times S^{n-1}; \Z_2) \]
    as groups, but not as rings.

    To study this, we will consider the \textit{Stiefel manifold} $V(n,k)$ of $k$-orthonormal frames in $\R^n$. Note that $V(n.k) = SO(n) / SO(n-k)$. In the simplest case, $V(n,1) = S^{n-1}$ and $V(n,2) = S(S^{n-1})$, the unit sphere bundle of $S^{n-1}$. Now we have $S^{n-1} \hookrightarrow V(n,2) \to S^{n-1}$, so we have a spectral sequence
    \begin{center}
    \begin{sseqdata}[name=stiefel, homological Serre grading, classes={draw=none}]
        \class["\Z"](0,0)
        \class["\Z"](0,2)
        \class["\Z"](3,0)
        \class["\Z"](3,2)
        \d[->]3(3,0)
    \end{sseqdata}
    \printpage[name=stiefel,page=3]
    \end{center}
\end{exm}

Observe that a finite CW complex might not have even finitely generated homotopy groups. For example, we have $\pi_2(S^1 \vee S^2) \cong \bigoplus_{i \in \Z} \Z$.

\begin{thm}[Serre]
    Let $X$ be a $1$-connected CW complex such that $H_q(X, \Z)$ is finitely generated for all $q$. Then $\pi_q(X)$ is finitely generated for all $q$.
\end{thm}

The key ingredients to the proof are:
\begin{enumerate}
    \item Show that for $K(G, n)$ with $G$ finitely generated (resp. finite), then $H_*$ is finitely generated (resp. finite).
    \item Inductively, kill homotopy groups using fibrations. This generalizes $F \hookrightarrow S^3 \to K(\Z_3)$.
\end{enumerate}

Consider the \textit{Whitehead tower} of $X$, which is a sequence $\cdots \to X_3 \to X_2 \to X_1 \to X$, where $X_n$ is $n$-connected and $X_n \to X$ is an isomorphism in $\pi_i$ for $i \geq n+1$. Also, we require $X_{n+1} \to X_n$ to be a fibration with fiber $K(\pi_{n+1}(X), n)$. For example, we had $F = {(S_3)}_3$ and $S^3 = {(S^3)}_2$. We have $\pi_{n+1}(X) = \pi_{n+1}(X_n) = H_{n+1}(X_n)$.

To construct the Whitehead tower, we work inductively. Suppose we have $X_n$. Then $K_{\pi_{n+1}(X), n+1}$ is obtained by attaching cells to $X_n$, so we have a map $X_n \to K(\pi_{n+1}(X), n+1)$ that is an isomorphism on $\pi_{n+1}$. Turning this into a fibration, we take $X_{n+1}$ to be the homotopy fiber.

\begin{lem}
    Consider a fibration $F \hookrightarrow E \to B$ such that $\pi_1(B) = 1$. Then if $H_q(F), H_p(B)$ are finitely generated, then $H_n(E)$ is finitely generated for all $n$.
\end{lem}

\begin{proof}
    Consider the Serre spectral sequence. Then $E_{p,q}^2$ is finitely generated for all $p,q$, so because $\Z$ is Noetherian, then $E_{p,q}^{\infty}$ must also be finitely generated. This implies that $H_n(E)$ has a filtration by finitely generated objects, so it must be finitely generated.
\end{proof}

We will apply this to $K(\pi_{n+1}(X)) \hookrightarrow X_{n+1} \to X_n$.

\begin{prop}
    $H_p(K(G, n))$ is finitely generated for $p>0$ whenever $G$ is finitely generated.
\end{prop}

\begin{proof}
    Write $G = \bigoplus \Z^r \oplus \bigoplus \Z/p^i \Z$. Now up to homotopy, we have
    \[ K(G, n) = {K(\Z, n)}^r \times \cdots \times K(\Z/p^i \Z, n). \]
    By K\"unneth, we may assume that $G$ is cyclic, so consider the fibration $K(G, n-1) \hookrightarrow * \to K(G,n)$. Note that $K(\Z,1) = S^1$ and $K(\Z_m,1)$ is an infinite lens space, so only the inductive step remains.

    Consider $E_{p,q}^2 = H_p(K(G,n), H_q(K(G, n-1)))$ and assume $H_M(K(G, n))$ is not finitely generated but $H_i(K(G, n))$ is finitely generated for $i < M$. But then $E_{M,0}^3$ is not finitely generated (it is the kernel of a map to something that is finitely generated), so $E_{M,0}^{\infty}$ is not finitely generated, a contradiction.
\end{proof}

\begin{rmk}
    This holds more generally for classes for classes of groups called \textit{Serre classes}, for example finite abelian $p$-groups.
\end{rmk}

\section{Homotopy groups of spheres}%
\label{sec:homotopy_groups_of_spheres}

We know that $\pi_m(S^n)$ is a finitely generated abelian group.

\begin{thm}[Serre]
    The rank of $\pi_q(S^n)$ is
        $1$ is $q = n$ or $q = 2n-1$ for $n$ even and
        $0$ otherwise.
\end{thm}

The key computation is the rational homology $H^*(K(\Z, n); \Q)$. This is given by
\[ H^*(K(\Z, n), \Z) = \begin{cases}
    \bigwedge_{\Q}[x], \abs{x} = n & n\ \text{odd} \\
    \Q[x], \abs{x} = n & n\ \text{even}.
\end{cases} \]
This is true for $K(\Z, 1) = S^1, K(\Z, 2) = \C\P^{\infty}$. We will do the case when $n$ is even, and consider $K(\Z, n-1) \hookrightarrow * \to K(\Z, n)$. Then the Serre spectral sequence in cohomology is
\begin{center}
    \begin{sseqdata}[name=eil, classes={draw=none}, cohomological Serre grading]
        \class["y"](0,3)
        \class["1"](0,0)
        \class["x"](4,0)
        \d4(0,3)
    \end{sseqdata}
    \printpage[name=eil, page=4]
\end{center}
and then $d_n(xy) = d_n(x) \cdot y + x d_n(y) = x^2$.

\begin{lem}
    Let $X$ be $1$-connected and suppose $H^q(X, \Z)$ is finitely generated for all $q$. Also, suppose that $H^*(X,\Q) \cong H^*(S^m, \Q)$ for $m$ odd. Then
    \[ \operatorname{rk} \pi_q(X) = \begin{cases}
        0 & q \neq m \\
        1 & q = m. 
    \end{cases} \]
\end{lem}

\begin{proof}
    Note that $H^m(X, \Q) = \Q$, so $[X, K(\Z,m)] = H^m(X,\Z) = \Z \oplus \text{torsion}$. Then there exists $f \colon X \to K(\Z, m)$ such that $f^*(\iota_m) = 1$, where $\iota_m$ generates $H^m(K(\Z, m), \Z) = \Z$. This tells us that $f^* \colon H^*(K(\Z, m), \Z) \simeq H^*(X, \Q)$ is an isomorphism. Now if $F$ is the homotopy fiber of $f$, then $H^q(F, \Q) = 0$ for $q > 0$, so $H^q(F, \Z)$ for all $q > 0$ (by finite generation), and therefore $\pi_q(F)$ is finite for all $q$. Now $f_* \colon \pi_q(X) \to \pi_q(K(\Z, m))$ fits in the long exact sequence with $\pi_i(F)$, so $f_*$ has finite kernel and cokernel, and thus $\rank \pi_q(X) =\rank \pi_q(K(\Z,m))$.
\end{proof}

\begin{proof}[Proof of Serre]
    We may assume that $n$ is even. Consider $K(\Z, n-1) \hookrightarrow {(S^n)}_n \to S^n$. This is the $n$th stage of the Whitehead tower. Consider the Serre spectral sequence
\begin{center}
    \begin{sseqdata}[name=even, classes={draw=none}, cohomological Serre grading]
        \class["\Q"](0,3)
        \class["\Q"](0,0)
        \class["\Q"](4,0)
        \d4(0,3)
    \end{sseqdata}
    \printpage[name=even, page=4]
\end{center}
and note that $d_n$ is an isomorphism because $H_{n-1}({(S^n)}_n) = 0$. But then we see that $H^*({(S^n)}_n) = H^*(S^{2n-1}, \Q)$, so the desired result follows.
\end{proof}

An ``easy'' generalization is
\begin{thm}[Cartan-Serre]
    Let $X$ be $1$-connected with finitely generated homology groups such that 
    \[ H^*(X, \Q) \cong \bigwedge_{\Q}[x_1, \ldots, x_m] \otimes \Q[y_1, \ldots, y_n]. \]
    Then $\rank \pi_q(X)$ is the number of generators $x_i, y_j$ in degree $q$.
\end{thm}

\begin{exm}
    Consider $X = SU(n), SO(n)$.
\end{exm}

\begin{rmk}
    In general, $H^*(X, \Q) \cong H^*(X', \Q)$ does not imply that $\rank \pi_q(X) = \rank \pi_q(X')$.
\end{rmk}

Now the point of rational homotopy theory is to compute $\rank \pi_q(X)$ using $H^*(X, \Q)$ and extra information (for example Massey products).

We have computed the ranks of the homotopy groups of spheres, so now we will consider the torsion. The tools we have developed are enough to show that
\begin{thm}
    Let $n \geq 3$. For any prime $p$, the group $\pi_i(S^n)$ has no $p$-torsion for $i < n + 2p-3$ and the $p$-primary part of $\pi_{n+2n-3}(S^n)$ is $\Z/p\Z$.
\end{thm}

\begin{cor}
    In the stable range, $\pi_{n+2}(S^n)$ is a $2$-group and $\pi_{n+3}(S^n)$ is the direct sum of a $2$-group and $\Z/3\Z$.
\end{cor}

The idea of the proof is induction on the Whitehead tower with the fibration $K(\pi_{n-1}(X), n) \hookrightarrow X_{n+1} \to X_n$ and $H_{n+2}(X_{n+1}) = \pi_{n+2}(X)$. We can study this group one prime at a time because $H_*(K(\Z/p, 1), \Z/p')$ vanishes if $p \neq p'$ and is very interesting if $p = p'$. Then there is a large gap in cohomology depending on $p$. If we study the $\Z$-cohomology of $K(\Z/p, 2)$, then we may consider the fibration $K(\Z/p, 1) \hookrightarrow * \to K(\Z/p, 2)$. Then we know that
$H^*(K(\Z/p, 1), \Z) = \Z[x]/(px)$ where $\abs{x} = 2$, and so the Serre spectral sequence of the fibration is
\begin{center}
    \begin{sseqdata}[name=kzp2, classes={draw=none}, cohomological Serre grading]
        \class["1"](0,0)
        \class["x"](0,2)
        \class["x^2"](0,4)
        \class["y_3"](3,0)
        \class["xy_3"](3,2)
        \d3(0,2)
        \d3(0,4)
    \end{sseqdata}
    \printpage[name=kzp2, page=3]
\end{center}
and then $d(x^m) = mx^{m-1} y_3$ and thus $d(x^p) = 0$ and $d(x^i) \neq 0$ for $i<p$. For more detail, see Fuchs-Fomenko.

Now we will compute the $2$-primary part of these homotopy groups. Consider the fibration $K(\Z, n-1) \hookrightarrow {(S^n)}_n \to S^n$ and the next step $k(\Z/2, n) \hookrightarrow {(S^n)}_{n+1} \to {(S^n)}_n$. We want to compute the $2$-primary part of $H_{n+2}({(S^n)}_{n+1})$. This means we need to understand $H^*(K(\Z/2, n); \Z/2)$. We can understand it as cohomology operations
\[ H^n(-, \Z/2) \to H^n(-, \Z). \]

\section{Cohomolgy Operations}%
\label{sec:cohomolgy_operations}

\begin{defn}
    A \textit{cohomology operation} $\varphi$ between $H^n(-,G) \to H^n(-,K)$ is a natural transformation between the two functors viewed as $\ms{CW} \to \ms{Set}$.
\end{defn}

\begin{exm}
    Let $R$ be a ring. Then the cup product $H^n(-,R) \to H^{2n}(-,R)$ given by $\alpha \mapsto \alpha \cup \alpha$ is a cohomology operation. Note that this is not a homomorphism in general.
\end{exm}

Now $H^n(-,G) = [-, K(G, n)]$ and so natural transformations are just homotopy classes of maps $K(G, n) \to K(K, m)$, which form the group $H^m(K(G, n), K)$.

\begin{exm}[Bockstein homomorphism]
    Consider a short exact sequence $0 \to A \to B \to C \to 0$. Then we get a short exact sequence
    \[ 0 \to C^*(X, A) \to C^*(X,B) \to C^*(X,C) \to 0 \]
    of chain complexes, and this induces a long exact sequence in cohomology. Then the map
    \[ \beta_n \colon H^n(X, C) \to H^{n+1}(X, A) \]
    is a cohomology operation, called the \textit{Bockstein homomorphism}.

    An interesting case of this is the exact sequence $0 \to \Z/p \to \Z/p^2 \to \Z/p \to 0$. Therefore we obtain an interesting map
    \[ \beta_n \colon H^n(X, \Z/p) \to H^{n+1}(X, \Z/p), \]
    so there is an interesting element in $H^{n+1}(K(\Z/p, n), \Z/p)$. This is actually useful in the homotopy classification of $3$-dimensional lens spaces $L(p,q)$. In fact, we can show that $L(p,q) \simeq L(p,q')$ if and only if $q' \equiv \pm k^2 q \pmod p$ for some $k$. For example, $L(5,1) \not\simeq L(5,2)$.

    To show this, consider the map 
    \[ Q \colon H^1(L(p,q), \Z/p) \to \Z/p \qquad x \mapsto \ev{x \smile \beta_1(x), [L(p,q)]}. \]
    This is well-defined up to a choice of sign. There is a standard generator $\alpha \in H_1(L(p,q))$, which is the image of the arc $(1,0) \to (e^{2\pi i/p}, 0)$. Then $Q(\alpha^*) = q$, where $\alpha^*$ is the Poincar\'e dual of $\alpha$. Then if we have a homotopy equivalence $L(p,q) \xrightarrow{f} L(p,q')$, we see that $\alpha_{p,q} \mapsto k \alpha_{p,q'}$ and then $Q(\alpha_{p,q}) = \pm k^2 Q(\alpha_{p,q'})$ by naturality.
\end{exm}

\begin{exm}[Steenrod square]
    For all $n$, we will construct a cohomology operation $\Sq^i \colon H^n(-, \Z/2) \to H^{n+i}(-, \Z/2)$. Some properties are
    \begin{itemize}
        \item Steenrod squares are additive.
        \item We have
            \[ \Sq^i(x) = \begin{cases}
                x & i = 0 \\
                x^2 & i = \dim X \\
                0 & i > \dim x.
            \end{cases} \]
        \item $\Sq^i \colon H^n(-,\Z/2) \to H^{n+1}(-,\Z/2)$ is the Bockstein homomorphism for 
            \[ 0 \to \Z/2 \to \Z/4 \to \Z/2 \to 0. \]
        \item $\Sq^i(\alpha \cup \beta) = \sum_{j+k=i} \Sq^j(\alpha) \cup \Sq^k(\beta)$ (the Cartan relation).
    \end{itemize}
    We will take these to be axioms for the Steenrod squares.
\end{exm}

\begin{thm}\label{thm:steenrodsquare}
    There exist unique cohomology operations with these properties.
\end{thm}

\begin{rmk}
    Define the total squaring operation by
    \begin{align*}
        \Sq(x) \coloneqq \sum \Sq^i(x) = x + \sum_{i = \abs{x}+1}^{2 \abs{x}-1} \Sq^i(x) + x^2 \in H^*(X, \Z/2).
    \end{align*}
    Here, the Cartan relation says that $\Sq$ is multiplicative.
\end{rmk}

\begin{exm}
    Let $X = \R\P^{\infty}$. Then $H^*(X, \Z/2) = \Z/2[\alpha]$. Then 
    \[ \Sq(\alpha) = \Sq^0(\alpha) + \Sq^1(\alpha) = \alpha + \alpha^2. \] 
    This implies that $\Sq(\alpha^k) = {\Sq(\alpha)}^2 = \alpha^k {(1+\alpha)}^k$. In particular, we see that $\Sq^i(\alpha^k) = \binom{k}{i} \alpha^{k+i}$.
\end{exm}

\begin{prop}\label{prop:suspension}
    The Steenrod squares $\Sq^i$ commute with the suspension isomorphisms $\Sigma \colon H^n(X, \Z/2) \to H^{n+1}(\Sigma X, \Z/2)$.
\end{prop}

This is interesting because $\Sigma X$ has trivial cup products.

\begin{exm}
    Consider $f \colon S^{15} \to S^8$ with $H(f) = 1$. Then $\Sigma^n f \colon S^{15+n} \to S^{8+n}$ is nontrivial. Consider the mappinc cone $C_f = S^8 \cup_f D^{18}$ where $\alpha^2 = \beta$, where $\alpha$ is the $8$-cell and $\beta$ is the $16$-cell. Then we see that $C_{\Sigma f} = \Sigma C_f$, so by dimension reasons, all cup products are trivial. However, $\Sq^8(\Sigma \alpha) = \Sigma \Sq^8(\alpha) = \Sigma \beta$. Otherwise, if $\Sigma f$ is trivial, then $C_{\Sigma f} = S^9 \cup S^{17}$.
\end{exm}

\begin{proof}[Proof of Proposition\autoref{prop:suspension}]
    Note that $\Sigma$ is obtained by
    \[ H^1(S^1) \otimes H^n(X) \to H^{n+1}(S^1 \wedge X). \]
    This implies that
    \[ \Sq^i(\Sigma \alpha) = \Sq^i(t \otimes \alpha) = t \otimes \Sq^i(\alpha) = \Sigma \Sq^i(\alpha). \qedhere \]
\end{proof}

Now we will consider the following question. How many pointwise linearly independent vector fields $v_1, \ldots, v_{k-1}$ can there be on $S^{n-1}$. When are there $n-1$?

Some basic obervations are the following:
\begin{itemize}
    \item If $n-1$ is even, then $\chi(S^{n-1}) = 2$, so any vector field has a zero.
    \item By Gram-Schmidt, we can assume that $v_1, \ldots, v_{k-1}$ is orthonormal at each point.
\end{itemize}

\begin{thm}[Steenrod-Whitehead]
    If $n = 2^r (2s+1)$, then $S^{n-1}$ has at most $2^{r}-1$ linearly independent vector fields.
\end{thm}

\begin{exm}
    If $n-1$ is even, then $r=0$, so there are no linearly independent vector fields.
\end{exm}

\begin{cor}
    If $S^{n-1}$ is parallelizable, then $n$ is a power of $2$.
\end{cor}

\begin{rmk}
    Adams gave a complete solution. We will see using $K$-theory that if $S^{n-1}$ is parallelizable, then $n = 1,2,4,8$.
\end{rmk}

\begin{proof}[Sketch of Proof]
    Let $V_{n,k}$ be the Stiefel manifold of orthonormal $k$-frames in $\R^n$. Then we have the natural map 
    \[ p \colon V_{n,k} \to S^{n-1} \qquad (v_1, \ldots, v_k) \mapsto v_k. \]
    Now if $v_1(x), \ldots, v_{k-1}(x)$ are orthonormal vector fields on $S^{n-1}$, then we get a section $f$ of $p$ given by
    \[ f(x) = (v_1(x), \ldots, v_{k-1}(x), x). \]
    Now the question is reduced to that of the existence of a section of $p$. We will discuss obstructions using $\Sq^i$. Recall that $V_{n,k} = SO(n)/SO(n-k)$. Now consider
    \[ \R\P^{n-1} \hookrightarrow SO(n) \qquad \ell \mapsto \mr{refl}_{\ev{e_1}^{\perp}} \circ \mr{refl}_{\ell^{\perp}}. \]
    This induces a map $\R\P^{n-1}/\R\P^{n-k-1} \hookrightarrow SO(n) / SO(n-k) = V_{n,k}$. Now if $2k-1 \leq n$, there is a cell decomposition of $V_{n,k}$ for which $\R\P^{n-1}/\R\P^{n-k-1}$ is the $(n-1)$-skeleton. Now suppose there is a section $f$ of $p \colon V_{n,k} \to S^{n-1}$. Then $f^* p^* \colon H^{n-1}(S^{n-1}) \simeq H^{n-1}(S^{n-1})$ is an isomorphism. After homotopy, we can make $f$ a cellular map, so $\wt{f} \colon S^{n-1} \to {(V_{n,k})}^{n-1} = \R\P^{n-1}/\R\P^{n-k-1}$. Therefore $f^*$ factors through $H^*(\R\P^{n-1}/\R\P^{n-k-1})$ and induces an isomorphism in degree $n-1$ (using $\Z/2\Z$-coefficients). In degree $n-k$, $f^*$ induces a map $\Z/2 \to 0$. But now we have a commutative diagram
    \begin{equation*}
    \begin{tikzcd}
        H^{n-1}(\R\P^{n-1}/\R\P^{n-k-1}) \arrow{r}{f^*}[swap]{\sim} & H^{n-1}(S^{n-1}) \\
        H^{n-k}(\R\P^{n-1}/\R\P^{n-k-1}) \arrow{r}{f^*} \arrow{u}{\Sq^{k-1}} & H^{n-k}(S^{n-1}) \arrow{u}{\Sq^{k-1}},
    \end{tikzcd}
    \end{equation*}
    and now we see that $\Sq^{k-1}$ cannot be an isomorphism. Computing $\Sq^{k-1}$ using naturality, we see that $k = 2^r + 1$ works.
\end{proof}

\subsection{Construction of Steenrod Squares}%
\label{sub:construction_of_steenrod_squares}

Here, we are following Chapter 4.L in Hatcher. The idea is that the cup product is commutative on $H^*(-, \Z/2)$. However, this is \textbf{not} commutative on the chain level, so $\Sq^i$ will measure this failure. We work at the level of spaces.

Let $X$ be a pointed space and consider $X \wedge X$ with the swap map $T \colon X \wedge X \to X \wedge X$. Now we consider the homotopy quotient of $T$. For example, we know that $\Z/2\Z$ acts freely on $S^{\infty}$ with quotient $\R\P^{\infty}$. Then we define $\Gamma^{\infty} X = S^{\infty} \times (X \wedge X) / (\Z/2\Z)$. This has a projection to $\R\P^{\infty}$ with fiber $X \wedge X$. Then the inclusion $s \colon \R\P^{\infty} \to \Gamma^{\infty} X$ of $\R\P^{\infty}$ at the basepoint of $X \wedge X$ is a section, so set $\bigwedge^{\infty} X = \Gamma^{\infty} X / s(\R\P^{\infty})$ (doing this kills many unnecessary cells in low dimension).

Now given $\alpha \in H^n(X, \Z_2)$, we can associate $\lambda(\alpha) \in H^{2n}\qty(\bigwedge^{\infty} X; \Z)$ such that 
\[ \eval{\lambda(\alpha)}_{X \wedge X} = \alpha \otimes \alpha \in H^{2n}(X \wedge X, \Z_2). \] 
Then $S^{\infty} \times X \hookrightarrow S^{\infty} \times (X \wedge X)$ induces a map
\[ \R\P^{\infty} \times X \to \Gamma^{\infty} X \twoheadrightarrow {\bigwedge}^{\infty} X. \]
But now given such $\lambda(\alpha)$, this maps to a sum of the form
\[ \sum_i \omega^{n-i} \otimes \Sq^{n+i}(\alpha). \]

\begin{rmk}
    The assignments $\bigwedge^{\infty}(-), \Gamma^{\infty}(-)$ are functorial.
\end{rmk}

Now because $H^n(-, \Z_2) = [-, K(\Z_2, n)]$, we only need to construct $\lambda(\iota)$ for the nontrivial
\[ \iota \in H^n(K(\Z/2, n); \Z_2) \cong \Z_2. \]
Here, we will set $K_n \coloneqq K(\Z_2, n)$, and therefore we need to construct the map $\bigwedge^{\infty} K_n \to K_{2n}$ step by step. Consider the map
\[ \iota \otimes \iota \colon K_n \wedge K_n \to K_{2n}. \]
Then if we consider $K_n \wedge K_n \xrightarrow{T} K_n \wedge K_n \xrightarrow{\iota \otimes \iota} K_{2n}$, this must be homotopic to $\iota \otimes \iota$ by some homotopy $h_t$. Now we can use $h_t$ to define
\[ I \times (K_n \wedge K_n) / (0,x) \sim (1,T(x)) = \Gamma^1 K_n \xrightarrow{h_t} K_{2n}. \]
Because this is basepoint preserving, it descends to $\bigwedge^1 K_n \to K_{2n}$. But now $\bigwedge^{\infty} K_n$ is obtained from $\bigwedge^1 K_n$ by attaching cells of dimension stricly larger than $2n+1$, so we can extend and obtain a map $\lambda(\iota) \colon \bigwedge^{\infty} K_n \to K_{2n}$.

\begin{rmk}
    Note that $\lambda(\iota)$ is uniquely determined because the map
    \[ H^{2n}\qty({ \bigwedge }^{\infty} K_n) \hookrightarrow H^{2n}(K_n \wedge K_n) \]
    is injective.
\end{rmk}

There is an alternative easier way. Consider $\alpha \in H^n(X, \Z_2)$ and $\alpha \times \alpha \in H^{2n}(X \times X, \Z_2)$. This gives us a classifying map $X \times X \xrightarrow{f} K_{2n}$, and again let $T$ be the swap map on $X \times X$. Now we see that $f \sim f \circ T$ with homotopy $f_t$, so now we have a map $S^1 \times X \times X \to K_{2n}$ such that $(s,x_1, x_2), (-s, x_2, x_1)$ map to the same point. This map extends to $D^2$ and therefore to $S^2$. This now gives us a map $S^{\infty} \times X \times X \to K_{2n}$ which is $\Z_2$-invariant, and precomposing with $S^{\infty} \times X \to S^{\infty} \times X \times X$ gives us a map $\R\P^{\infty} \times X \to K_{2n}$.

\begin{thm}[Cartan]
    For any $\alpha, \beta$, we have $\lambda(\alpha \cup \beta) = \lambda(\alpha) \cup \lambda(\beta)$.
\end{thm}

\subsection{Stable Cohomology Operations}%
\label{sub:stable_cohomology_operations}

\begin{defn}
    A \textit{stable cohomology operation} (SCO) of degree $r$ is a map
    \[ \varphi \colon H^n(-, G) \to H^{n+r}(-, H) \]
    for all $n$ that commutes with suspension maps, i.e. the diagram
    \begin{equation*}
    \begin{tikzcd}
        H^n(X, G) \ar{r}{\varphi} \ar{d}{\Sigma} & H^{n+r}(X, H) \ar{d}{\Sigma} \\
        H^{n+1}(X, G) \ar{r}{\varphi} & H^{n+r+1}(X, H)
    \end{tikzcd}
    \end{equation*}
\end{defn}

\begin{exm}
    The Steenrod squares and compositions of Steenrod squares are all stable cohomology operations.
\end{exm}

\begin{lem}
    Let $\varphi$ be a stable cohomology operation. Then the diagram
    \begin{equation*}
    \begin{tikzcd}
        H^n(X,A,G) \ar{r} \ar{d}{\varphi}& H^n(A, G) \ar{r}{\delta^*} \ar{d}{\varphi}& H^{n+1}(X,A,G) \ar{r} \ar{d}{\varphi}& H^{n+1}(X, G) \ar{d}{\varphi}\\
        H^{n+r}(X,A,G) \ar{r} & H^{n+r}(A, G) \ar{r}{\delta^*} & H^{n+r+1}(X,A,G) \ar{r} & H^{n+r+1}(X, G) 
    \end{tikzcd}
    \end{equation*}
    commutes.
\end{lem}

\begin{proof}
    The left and right squares are clear because they come from maps of spaces. At $\delta^*$, we consider 
    \[ \delta^* \colon H^n(A) \to H^{n+1}(X, A) = H^{n+1}(X/A). \]
    Because $A \subset X$ is a cofibration, then $X \cup C A \cong X/A$. Then we may collapse the $X$ to obtain $\Sigma A$, and then we see that $\delta^*$ is given by $H^n(A) \xrightarrow{\Sigma} H^{n+1}(\Sigma A) \xrightarrow{p^*} H^{n+1}(X/A)$, and both arrows in the composition commute with $\varphi$.
\end{proof}

This gives us the following slogan:

\begin{quotation}
    \textit{Stable cohomology operations commute with transgression.} 
\end{quotation}

\begin{prop}
    Let $\varphi$ be a stable cohomology operation of degree $e$. If $\alpha \in H^m(F)$ is transgressive, then so is $\varphi(\alpha) \in H^{m+r}$ (which means that $\dd_2 \varphi(\alpha) = \cdots = \dd_{m+r} \varphi(\alpha) = 0$). If $\tau(\alpha) \in E^{m+1, 0}_{m+1} = H^{m+1}(B)/\sim$ is represented by $\beta \in H^{n+1}(B)$, then $\tau (\varphi(\alpha)) \in H^{m+r+1}(B) /\sim$.
\end{prop}

\begin{proof}
    The transgression map is given by 
    \[ H^m(F) \xrightarrow{\delta^*} H^{m+1}(E, F) \overset{{(p^*)}^{-1}}{\dashrightarrow} H^{m+1}(B, \mr{pt}) = H^{m+1}(B). \]
    Then $\varphi$ commutes with $\delta^*$ and $p^*$.
\end{proof}

\begin{rmk}
    If $G = H$, then we can compose stable cohomology operations. This is a noncommutative algebra!
\end{rmk}

\begin{thm}
    The \textit{Steenrod algebra}
    \[ \mc{A}_2 = \qty{\text{stable cohomology operations on $\Z_2$-cohomology}} \]
    is generated as an algebra by $\qty{\Sq^i}_{i \in \mathbb{N}}$.
\end{thm}

This is a very complicated algebra. For example, if $a < 2b$, we have the \textit{Adem relations} 
\[ \Sq^a \circ \Sq^b = \sum_j \binom{b-j-1}{a-2j} \Sq^{a+b-j} \circ \Sq^j. \]
For a proof of this, see Hatcher.

\begin{exm}
    For any $b$, we have $\Sq^1 \Sq^b = (b-1) \Sq^{b+1}$. For example, $\Sq^1 \Sq^2 = \Sq^3$. We say that $\Sq^3$ is decomposable. This means it can be written in terms of elements of lower degree.
\end{exm}

\begin{exm}
    The elements of the form $\Sq^{2^k}$ are indecomposable. Consider the action on $H^*(\R\P^{\infty})$. Then we see that
    \[ \Sq^{2^k}(\alpha^{2^k}) = {(\alpha^{2^k})}^2 = \alpha^{2^{k+1}} \neq 0 \]
    but $\Sq^i (\alpha^{2^k})=  0$ for $0 < i < 2^k$. The idea is that $\binom{2^k}{i} \equiv 0 \mod 2$ for all $i$.
\end{exm}

We want to interpret stable cohomology operations using $K(G, n)$. Then if $\varphi = \qty{\varphi_n}$, we see that
\[ \varphi_n \in [K(\Z_2, n),K(\Z_2, n+1)] = H^{n+r}(K(\Z_2, n), \Z_2). \]
But then stability tells us that $H^n(\Sigma K(\Z_2, n-1), \Z_2) = \Z_2$, and if we consider the nontrivial $\iota \colon \Sigma K(\Z_2, n-1) \to K(\Z/2, n)$, then we obtain a map
\[ H^{n+r}(K(\Z_2, n), \Z_2) \xrightarrow{\iota^*} H^{n+r}(\Sigma K(\Z_2, n-1), \Z_2) \to H^{n+r-1}(K(\Z_2, n-1), \Z_2). \]
Then stability is equivalent to the fact that $\varphi_n \mapsto \varphi_{n-1}$. Alternatively, we may use the suspension-loop adjunction to see that
\[ [K(\Z_2, n), K(\Z_2, n+r)] \xrightarrow{\Omega} [K(\Z_2, n-1), K(\Z_2, n+r-1)] \qquad \varphi_n \mapsto \varphi_{n-1}. \]

\begin{prop}
    For any $r$, we have 
    \[ \mc{A}_2^r = \varprojlim_{r} H^{n+r}(K(\Z_2, n); \Z_2). \]
\end{prop}

We will now compute $H^*(K(\Z_2, n), \Z_2)$. We will use the fibration $K(\Z_2, 1) \hookrightarrow * \to K(\Z_2, n)$ and hope to proceed by induction.

\begin{exm}
    Let $n = 2$. Then we know $K(\Z_2, 1) = \R \P^{\infty}$. Then the spectral sequence is given by
    \begin{center}
        \begin{sseqdata}[classes={draw=none}, name=kz22, cohomological Serre grading]
            \class["1"](0,0)
            \class["\alpha"](0,1)
            \class["\alpha^2"](0,2)
            \class["\alpha^3"](0,3)
            \class["\alpha^2"](0,4)
            \class["e"](2,0)
            \class["\Sq^1 e"](3,0)
            \d2(0,1)
        \end{sseqdata}
        \printpage[name=kz22, page=2]
    \end{center}
    Then we see that $\dd_2(\alpha^2) = 2 \alpha \dd_2(\alpha) = 0$. But then $\dd_3(\alpha^2) \neq 0$, so we see that $\dd_3(\Sq^1 \alpha) = \Sq^1 \dd_2(\alpha) = \Sq^1 e$. Inductively, we see that $\dd_5(\alpha^4) = \Sq^2 \Sq^1 e$ and $\dd_2(\alpha e) = \dd_2 (\alpha) \cdot e = e^2$. In the end, we will see that
    \[ H^*(K(\Z_2, 2); \Z_2) = \Z_2 [ e, \Sq^1 e, \Sq^2 \Sq^1 e, \Sq^4 \Sq^2 \Sq^1 e, \ldots ]. \]
\end{exm}

\begin{defn}
    A set $I = (i_1, \ldots, i_k)$ (possibly empty) is \textit{admissible} if $i_1 \geq 2 i_2, i_2 \geq 2 i_3, \ldots$. Then we will write $\Sq^I = \Sq^{i_1} \circ \Sq^{i_2} \circ \cdots \circ \Sq^{i_k}$.
\end{defn}

Note that if $I$ is not admissible, then $\Sq^I$ can be simplified using Adem relations.

\begin{defn}
    Define the \textit{excess} of an admissible $I$ to be 
    \[ \mr{exc}(I) = (i_1 - 2 i_2) + (i_2 - 2 i_3) + \cdots + (i_{k-1} - 2 i_k) + i_k = i_1 - (i_2 + \cdots + i_k). \]
\end{defn}

For example, if $\mr{exc}(I) = 1$, then $I = (2^k, 2^{k-1}, \ldots, 4, 2, 1)$. If $\mr{exc}(I) = 0$, then $I = \emptyset$.

\begin{thm}[Serre]
    We have
    \[ H^*(K\Z_2, n) = \Z_2 [\Sq^I (e_n) \mid I, \text{admissible},\ \mr{exc}(I) < n]. \]
\end{thm}

\begin{exm}
    We have $H^*(K(\Z_2, 1)) = \Z_2[e_1]$. When $n=3$, the generators are $e_3, \Sq^1 e_3, \Sq^2 e_3, \Sq^3 \Sq^1 e_3, \ldots$. When $n$ increases, the description becomes more and more complicated.
\end{exm}

\begin{thm}[Borel]
    The set $\qty{\tau({(\Sq^I e_{n-1})}^{2^k}) \mid \mr{exc}(I) < n-1}$ are generators of $H^*(K(\Z_2, n))$. Now $\tau(e_{n-1}) = e_n$.
\end{thm}

\begin{exm}
    When $n = 2$, we have ${( \Sq^0 e_2 )}^2 = e_2^2 = \Sq^2 e_2$ and has excess $2$. Similarly, we see that ${(\Sq^0 e_2)}^4 = \Sq^4 \Sq^2 e_2$ and has excess $2$. We can also see that ${(\Sq^2 \Sq^1 e_2)}^2 = \Sq^5 \Sq^2 \Sq^1 (e_2)$, and this also has excess $2$.
\end{exm}








\end{document}
