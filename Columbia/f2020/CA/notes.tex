\documentclass[leqno, openany]{memoir}
\setulmarginsandblock{3.5cm}{3.5cm}{*}
\setlrmarginsandblock{3cm}{3.5cm}{*}
\checkandfixthelayout

\usepackage{amsmath}
\usepackage{amssymb}
\usepackage{amsthm}
%\usepackage{MnSymbol}
\usepackage{bm}
\usepackage{accents}
\usepackage{mathtools}
\usepackage{tikz}
\usetikzlibrary{calc}
\usetikzlibrary{automata,positioning}
\usepackage{tikz-cd}
\usepackage{forest}
\usepackage{braket} 
\usepackage{listings}
\usepackage{mdframed}
\usepackage{verbatim}
\usepackage{physics}
%\usepackage{/home/patrickl/homework/macaulay2}

%font
\usepackage[osf]{mathpazo}
\usepackage{microtype}

%CS packages
\usepackage{algorithmicx}
\usepackage{algpseudocode}
\usepackage{algorithm}

% typeset and bib
\usepackage[english]{babel} 
\usepackage[utf8]{inputenc} 
\usepackage[backend=biber, style=alphabetic]{biblatex}
\usepackage[bookmarks, colorlinks, breaklinks]{hyperref} 
\hypersetup{linkcolor=black,citecolor=black,filecolor=black,urlcolor=black}

% other formatting packages
\usepackage{float}
\usepackage{booktabs}
\usepackage{enumitem}
\usepackage{csquotes}
\usepackage{titlesec}
\usepackage{titling}
\usepackage{fancyhdr}
\usepackage{lastpage}
\usepackage{parskip}

\usepackage{lipsum}

% delimiters
\DeclarePairedDelimiter{\gen}{\langle}{\rangle}
\DeclarePairedDelimiter{\floor}{\lfloor}{\rfloor}
\DeclarePairedDelimiter{\ceil}{\lceil}{\rceil}


\newtheorem{thm}{Theorem}[section]
\newtheorem{cor}[thm]{Corollary}
\newtheorem{prop}[thm]{Proposition}
\newtheorem{lem}[thm]{Lemma}
\newtheorem{conj}[thm]{Conjecture}
\newtheorem{quest}[thm]{Question}

\theoremstyle{definition}
\newtheorem{defn}[thm]{Definition}
\newtheorem{defns}[thm]{Definitions}
\newtheorem{con}[thm]{Construction}
\newtheorem{exm}[thm]{Example}
\newtheorem{exms}[thm]{Examples}
\newtheorem{notn}[thm]{Notation}
\newtheorem{notns}[thm]{Notations}
\newtheorem{addm}[thm]{Addendum}
\newtheorem{exer}[thm]{Exercise}

\theoremstyle{remark}
\newtheorem{rmk}[thm]{Remark}
\newtheorem{rmks}[thm]{Remarks}
\newtheorem{warn}[thm]{Warning}
\newtheorem{sch}[thm]{Scholium}


% unnumbered theorems
\theoremstyle{plain}
\newtheorem*{thm*}{Theorem}
\newtheorem*{prop*}{Proposition}
\newtheorem*{lem*}{Lemma}
\newtheorem*{cor*}{Corollary}
\newtheorem*{conj*}{Conjecture}

% unnumbered definitions
\theoremstyle{definition}
\newtheorem*{defn*}{Definition}
\newtheorem*{exer*}{Exercise}
\newtheorem*{defns*}{Definitions}
\newtheorem*{con*}{Construction}
\newtheorem*{exm*}{Example}
\newtheorem*{exms*}{Examples}
\newtheorem*{notn*}{Notation}
\newtheorem*{notns*}{Notations}
\newtheorem*{addm*}{Addendum}


\theoremstyle{remark}
\newtheorem*{rmk*}{Remark}

% shortcuts
\newcommand{\Ima}{\mathrm{Im}}
\newcommand{\A}{\mathbb{A}}
\newcommand{\N}{\mathbb{N}}
\newcommand{\R}{\mathbb{R}}
\newcommand{\C}{\mathbb{C}}
\newcommand{\Z}{\mathbb{Z}}
\newcommand{\Q}{\mathbb{Q}}
\renewcommand{\k}{\Bbbk}
\renewcommand{\P}{\mathbb{P}}
\newcommand{\M}{\overline{M}}
\newcommand{\g}{\mathfrak{g}}
\newcommand{\h}{\mathfrak{h}}
\newcommand{\n}{\mathfrak{n}}
\renewcommand{\b}{\mathfrak{b}}
\newcommand{\ep}{\varepsilon}
\newcommand*{\dt}[1]{%
   \accentset{\mbox{\Huge\bfseries .}}{#1}}
\renewcommand{\abstractname}{Official Description}
\newcommand{\mc}[1]{\mathcal{#1}}
\newcommand{\T}{\mathbb{T}}
\newcommand{\mf}[1]{\mathfrak{#1}}
\newcommand{\mr}[1]{\mathrm{#1}}
\newcommand{\ms}[1]{\mathsf{#1}}
\newcommand{\ol}[1]{\overline{#1}}
\newcommand{\wt}[1]{\widetilde{#1}}

\DeclareMathOperator{\Der}{Der}
\DeclareMathOperator{\Hom}{Hom}
\DeclareMathOperator{\End}{End}
\DeclareMathOperator{\ad}{ad}
\DeclareMathOperator{\Aut}{Aut}
\DeclareMathOperator{\Gal}{Gal}
\DeclareMathOperator{\Rad}{Rad}
\DeclareMathOperator{\supp}{supp}
\DeclareMathOperator{\Supp}{Supp}
\DeclareMathOperator{\sgn}{sgn}
\DeclareMathOperator{\spec}{Spec}
\DeclareMathOperator{\Spec}{Spec}
\DeclareMathOperator{\Ext}{Ext}
\DeclareMathOperator{\Tor}{Tor}
\DeclareMathOperator{\Ann}{Ann}
\DeclareMathOperator{\Ass}{Ass}

% Section formatting
\titleformat{\section}
    {\Large\sffamily\scshape\bfseries}{\thesection}{1em}{}
\titleformat{\subsection}[runin]
    {\large\sffamily\bfseries}{\thesubsection}{1em}{}
\titleformat{\subsubsection}[runin]{\normalfont\itshape}{\thesubsubsection}{1em}{}

\title{COURSE TITLE}
\author{Lectures by INSTRUCTOR, Notes by NOTETAKER}
\date{SEMESTER}

\newcommand*{\titleSW}
    {\begingroup% Story of Writing
    \raggedleft
    \vspace*{\baselineskip}
    {\Huge\itshape Commutative Algebra \\ Fall 2020}\\[\baselineskip]
    {\large\itshape Notes by Patrick Lei}\\[0.2\textheight]
    {\Large Lectures by Eric Urban}\par
    \vfill
    {\Large \sffamily Columbia University}
    \vspace*{\baselineskip}
\endgroup}
\pagestyle{simple}

\chapterstyle{ell}


%\renewcommand{\cftchapterpagefont}{}
\renewcommand\cftchapterfont{\sffamily}
\renewcommand\cftsectionfont{\scshape}
\renewcommand*{\cftchapterleader}{}
\renewcommand*{\cftsectionleader}{}
\renewcommand*{\cftsubsectionleader}{}
\renewcommand*{\cftchapterformatpnum}[1]{~\textbullet~#1}
\renewcommand*{\cftsectionformatpnum}[1]{~\textbullet~#1}
\renewcommand*{\cftsubsectionformatpnum}[1]{~\textbullet~#1}
\renewcommand{\cftchapterafterpnum}{\cftparfillskip}
\renewcommand{\cftsectionafterpnum}{\cftparfillskip}
\renewcommand{\cftsubsectionafterpnum}{\cftparfillskip}
\setrmarg{3.55em plus 1fil}
\setsecnumdepth{subsection}
\maxsecnumdepth{subsection}
\settocdepth{subsection}

\begin{document}
    
\begin{titlingpage}
\titleSW
\end{titlingpage}

\thispagestyle{empty}
\section*{Disclaimer}%
\label{sec:disclaimer}

These notes were taken during lecture using the \texttt{vimtex} package of the editor \texttt{neovim}. 
Any errors are mine and not the instructor's. 
In addition, my notes are picture-free (but will include commutative diagrams) and are a mix of my mathematical style and that of the instructor.
If you find any errors, please contact me at \texttt{plei@math.columbia.edu}.
\newpage



\tableofcontents

\chapter{Basic Notions}%
\label{cha:basic_notions}

The references we will use in this course are Matsumura's \textit{Commutative Algebra} and Serre's \textit{Alg\`ebre Locale, Multiplicit\'es}. There is an English translation of Serre.
We will begin with general results on rings and modules. We will assume all rings are commutative and unital. Recall that and ideal $I$ of a ring $A$ is prime if and only if $A/I$ is a domain, and $I$ is maximal if and only if $A/I$ is a field.

\section{Basics of Ideals}%
\label{sec:basics_of_ideals}


\begin{defn}
    Let $I \subset A$ be an ideal. Then the \textit{radical} $\sqrt{I}$ of $I$ is the set 
    \[ \sqrt{I} \coloneqq \qty{x \in A \mid x^a \in I \text{ for some } a \in \N}. \]
\end{defn}

\begin{defn}
    An ideal $I \subset A$ is \textit{primary} if $I \neq A$ and the zero divisors in $A/I$ are nilpotent. Thus if $xy \in I$ and $x \notin I$, then $y^n \in I$ for some $n$.
\end{defn}

\begin{prop}
    \label{prop:radical_of_primary}
    If $Q \subset A$ is primary, then $\sqrt{Q}$ is a prime ideal.
\end{prop}

\begin{proof}
    If $xy \in \sqrt{Q}$, then $x^ny^n \in Q$. If $x^n \notin Q$, then $y \in \sqrt{q}$ because $(y^n)^a \in Q$.
\end{proof}

\begin{rmk}
    The converse to Proposition \ref{prop:radical_of_primary} is false in general.
\end{rmk}

\begin{defn}
    Let $A$ be a ring. Then the \textit{spectrum} $\spec A$ of $A$ as a set is the set of prime ideals of $A$. We may place the Zariski topology on this set, where the basis of open sets is given by $D_f = \spec A \setminus V_f$, where $V_f$ is the set of prime ideals containing $f$.
\end{defn}

If $\varphi:A \to B$ is a morphism of rings, the morphism $\varphi^*: \spec B \to \spec A$ is continuous in the Zariski topology. 

\begin{exer}
    In particular, if $\pi: A \to A/I$, then $\pi^*$ is an embedding.
\end{exer}

\begin{exer}
    Let $I \subset A$ be an ideal. Then let $P_1, \ldots, P_r$ be ideals of $A$ that are all prime except possibly two of them. Show that if $I \not\subset P_i$ for all $i$, then $I \not\subset \bigcup_i P_i$.
\end{exer}

\begin{exer}
    Let $ \mathfrak{a}_1, \ldots, \mathfrak{a}_r$ be ideals of $A$ such that $ \mathfrak{a}_i + \mathfrak{a}_j = A $. Then 
    \begin{enumerate}
        \item $\bigcap_i \mathfrak{a}_i = \mathfrak{a}_1 \cdots \mathfrak{a}_r$
        \item There is an isomorphism of rings $A/ \bigcap_i \mf{a}_i \cong \prod_i A/\mf{a}_i$.
    \end{enumerate}
\end{exer}

\section{Localization}%
\label{sec:localization}

Let $S \subset A$ be a multiplicative subset. The main examples are $S_f = \qty{1,f,f^2, \ldots}$ and $S_{\mf{p}} = A \setminus \mf{p}$ for a prime ideal $\mf{p}$. Then if $0 \notin S$, there is at least one ideal $\mf{a}$ such that $\mf{a} \cap S = \emptyset$. Denote the set of such $\mf{a}$ by $\mc{M}_S$. Then any maximal element of $\mc{M}_S$ is a prime ideal in $A$. Existence of a maximal element is seen using Zorn's lemma.

To see that maximal elements of $\mc{M}_S$ are prime ideals, note that $(x) + P$ is not in $\mc{M}_S$, so if $x,y \notin P$, there exist $a,b \in A$ and $s,s' \in S$ such that $ax \equiv s \mod P$ and $by \equiv s' \mod P$. Therefore $abxy \notin P$, so $xy$ is not in $P$

\begin{lem}
Let $\operatorname{nil} A$ be the set of all nilpotent elements. Then 
\[ \operatorname{nil} A = \bigcap_{\substack{P \subset A \\ P \text{ prime}}} P. \]
\end{lem}

\begin{proof}
    One direction is easy, so let $x$ be contained in all prime ideals. Then consider the set $S_x$. If $0 \notin S_x$, then $\mc{M}_{S_x}$ is nonempty, so it has a maximal element. This is a prime ideal, which implies $x$ is not contained in some prime.
\end{proof}

\begin{cor}
    Let $Q$ be an ideal of $A$. Then $\sqrt{Q}$ is the intersection of all prime ideals containing $Q$.
\end{cor}

Now fix a multiplicative subset $S$. Then we will define an equivalence relation on $A \times S$. We write
\[ (a,s) \sim (b,r) \]
if there exists $t \in S$ such that $t(ar - bs) = 0$. If $A$ is a domain, then this says that $\frac{a}{s} = \frac{b}{r}$. Now we will define the \textit{localization} $S^{-1}A$ to be the set of equivalence classes for this relation. Note there is a natural morphism $A \to S^{-1}A$ that sends $a \mapsto \frac{a}{1}$.

Note that the localization has a universal property: If $\varphi:A \to B$ is a morphism such that $\varphi(S) \subset B^{\times}$, then $\varphi$ factors uniquely through $S^{-1}A$.

Localization gives a map $\spec S^{-1}A \to \spec A$, and in particular, if $S = \qty{1, f, f^2, \ldots}$, we recover the set $D_f = \spec A_f$.

\section{Modules}%
\label{sec:modules}

Let $A$ be a ring. Then an $A$-module $M$ is an abelian group with an action of $A$. If $M$ is an $A$-module and $S \subset A$ is a multiplicative set, then $S^{-1}M$ is the set of equivalence classes for $(m,s) \sim (m',s')$ if there exists $t \in S$ such that $t(s'm - sm') = 0$. This is an $S^{-1}A$-module.

\begin{lem}
    Let $M$ be an $A$-module. Then the map 
    \[ M \to \prod_{\substack{ \mf{p} \subset A \\ \mf{p} \text{ maximal}}} M_{\mf{p}} \]
    is injective.
\end{lem}

\begin{proof}
    Let $x \in M$ be nonzero. Then the annihilator of $x$ is a proper ideal of $A$, so it is contained in a maximal ideal. This implies that $x_p \in M_p$ is nonzero.
\end{proof}

\begin{cor}
    Let $A$ be a domain. Then $A = \bigcap_{\mf{p}} A_{\mf{p}}$, where this intersection makes sense inside the fraction field of $A$.
\end{cor}

\begin{proof}
    Apply the previous lemma to $M = K/A$.
\end{proof}

\begin{defn}
    Let $M$ be an $A$-module and $x \in A$. Then $x$ is \textit{$M$-regular} if the morphism $m \mapsto xm$ is injective. Additionally, if $x$ is $A$-regular, then it is called \textit{regular}.
\end{defn}

The set $S_0$ of all regular elements in $A$ is multiplicative, and the ring $S_0^{-1}A$ is called the \textit{total ring of fractions}. If $A$ is a domain, then $S_0 = A \setminus \{0 \}$, and $S_0^{-1}A$ is the field of fractions.

\begin{defn}
    A ring $A$ is a \textit{local ring} if $A$ has only one maximal ideal. In this case, all elements not in the maximal ideal are units.
\end{defn}

\begin{rmk}
    If $I \subset A$ is an ideal such that $A \setminus I = A^{\times}$, then $A$ is a local ring and $I$ its maximal ideal.
\end{rmk}

\begin{exm}
Now let $A$ be a general ring and $\mf{p} \in \spec A$. Then $A_{\mf{p}}$ is a local ring with maximal ideal $\mf{p} A_{\mf{p}}$.
\end{exm}

\begin{defn}
    Now suppose $A,B$ are local rings. Then a morphism $\varphi:A \to B$ of rings is \textit{local} if $\varphi(\mf{m}_A) \subset \mf{m}_B$. This means we have a commutative diagram
    \begin{equation}
    \begin{tikzcd}
        A \arrow{r}{\varphi} \arrow{d} & B \arrow{d} \\
        k_A \arrow{r} & k_B
    \end{tikzcd}
    \end{equation}
    where $k_A = A/ \mf{m}_A$ is the residue field of $A$.
\end{defn}

Recall that the nilradical is the set of all nilpotent elements, or equivalently the intersection of all prime ideals. Then the \textit{Jacobson radical} $\operatorname{rad} A$ is defined to be the intersection of all maximal ideals.

\begin{prop}
    Let $x \in A$. Then $x \in \operatorname{rad} A$ if and only if $1+xa$ is a unit for any $a \in A$.
\end{prop}

\begin{proof}
    If $(1+x)A \neq A$, then $1+x$ is contained in some maximal ideal $mf{m}$, which implies $1 \in \mf{m}$. In the other direction, suppose there exists some maximal ideal $\mf{m}$ such that $x \notin \mathfrak{m}$. Then $x$ is nonzero in $A/ \mathfrak{M}$. Thus there exists $b$ such that $1-xb \in m$, which contradicts the assumption that $1+xa$ is a unit for any $a$.
\end{proof}

\begin{lem}[Nakayama's Lemma]
    Let $M$ be a finitely generated $A$-module. Then let $I$ be an ideal such that $IM = M$. Then there exists $x \in I$ such that $(1+x) M = 0$. In particular, if $I \subseteq \operatorname{rad} A$, then $M = 0$.
\end{lem}

\begin{proof}
    We will induct on the number of generators. If $M = A.m$, then $m = xm$ for some $x \in I$, and thus $(1-x)m = 0$. Now suppose $M = Am_1 + \cdots + Am_r$. Let $M' = M / Am_r$. By the inductive hypothesis, $(1+x)M' = 0$ for some $x \in I$.  Therefore $(1+x)M \subset A m_r$, so $(1+x) IM = (1+x)M \subset I m_r$. Therefore $(1+x) m_r = y m_r$ for some $y \in I$, and thus $(1+x - y) m_r = 0$. Thus $(1+x)(1+x-y) M \subset (1+x-y)A m_r = 0$.
\end{proof}

\begin{cor}
    Let $N, N' \subset M$ and $I \subset A$ such that $M = N + I N'$. Then if either
    \begin{enumerate}
        \item $I$ is nilpotent;
        \item $I \subset \operatorname{rad} A$ and $N'$ is finitely generated,
    \end{enumerate}
    then $M = N$.
\end{cor}

\begin{proof}
    \begin{enumerate}
        \item Suppose $I$ is nilpotent. Then
            \begin{align*}
                M &= N + IN' = N+IM \\
                  &= N + I (N + IM) \\
                  &= N + I^2 M \\
                  &\ \vdots \\
                  &= N + I^n M \\
                  &= N
            \end{align*}
            because $I$ is nilpotent.
        \item Let $I \subseteq \operatorname{rad} A$ and $N'$ be finitely generated. Then set $M_0 = M/N = I N_0'$, where $N_0'$ is the image of $N'$ inside $M_0$. Because $N_0'$ is finitely generated, so is $M_0$. Therefore $M_0 = I M_0 = 0$, so $M = N$.
    \end{enumerate}
\end{proof}

\begin{rmk}
    Most of the time, we apply this result when $A$ is local and $I$ is the maximal ideal of $A$. In this case, $M / \mf{m} M$ is a finite-dimensional vector space over $A / \mf{m}$.
\end{rmk}

\section{Artinian and Noetherian Rings}%
\label{sec:artinian_and_noetherian_rings}

\begin{defn}
    We say that an $A$-module $M$ satisfies the \textit{ascending chain condition} if any ascending chain of submodules of $M$ becomes stationary. Similarly, $M$ satisfies the \textit{descending chain condition} if any descending chain of submodules becomes stationary. If $M$ satisfies the ascending chain condition, it is called \textit{Noetherian}, and if $M$ satisfies the descending chain condition, it is \textit{Artinian}.
\end{defn}

\begin{prop}
    Assume we have a short exact sequence of $A$ modules
    \begin{equation}
    \begin{tikzcd}
        0 \arrow{r} & N \arrow{r} & M \arrow{r} & P \arrow{r} & 0.
    \end{tikzcd}
    \end{equation}
    Then $M$ is Noetherian (resp. Artinian) if and only if $N$ and $P$ are.
\end{prop}

\begin{proof}
    Proving that if $M$ is Noetherian, then $N$ and $P$ are is left to the reader. Now consider a chain
    \[ M_1 \subset M_2 \subset \cdots M_n \subset \cdots \]
    Then let $P_i$ be the image of the $M_i$ in $P$ and $N_i = N \cap M_i$. Then we have an exact sequence
    \[ 0 \to N_i \to M_i \to P_i \to 0. \]
    Because $(N_i)$ and $(P_i)$ stabilize, so must $M_i$ from the exact sequence.
\end{proof}

\begin{cor}
    If $A$ is Noetherian (resp. Artinian), then any finitely generated $A$-module is Noetherian (resp. Artinian).
\end{cor}

\begin{cor}
    Assume $A$ is Noetherian. Then any finitely generated $A$-module $M$ has a projective resolution by finite free $A$-modules. In other worse, there exists an exact sequence 
    \[ \cdots \to F_n \to F_{n-1} \to \cdots \to F_1 \to F_0 \to M \to 0 \]
    such that each $F_i = A^{m_i}$.
\end{cor}

\begin{proof}
    Suppose $M$ is finitely generated. Then $M = A m_1 + \cdots + A m_r$, so we have a sequence
    \[ A^r \xrightarrow{\varphi_0} M \to 0. \]
    Then $\ker \varphi_0 = N_0$ and $F_0 = A^r$. Then we repeat this process with $N_0$ taking the role of $M$.
\end{proof}

\begin{prop}
    An $A$-module $M$ is noetherian if and only if any submodule of $M$ is finitely generated.
\end{prop}

\begin{proof}
    Let $N \subseteq M$. Then choose $n_1 \in N$. Then if $A n_1 \neq N$, choose $n_2 \in N \setminus A n_1$. This process will stop because $M$ is Noetherian, so $N$ is finitely generated.

    Now suppose any submodule is finitely generated. Given a chain
    \[ M_1 \subset M_2 \subset \cdots \subset M_n \subset \cdots, \]
    set $N = \bigcup_i M_i$. This is finitely generated and is also equal to the first $M_i$ that contains all of the generators.
\end{proof}

This means that a ring $A$ is noetherian if and only if all ideals of $A$ are finitely generated. In particular, fields and principal ideal domains are Noetherian.

\begin{prop}
    Let $M$ be Noetherian and suppose $S$ is a multiplicative subset of $A$. Then $S^{-1}M$ is Noetherian.
\end{prop}

\begin{proof}
    Consider the morphism $M \to S^{-1}M$. Then let $N_i$ be a chain of $S^{-1}A$-modules in $S^{-1}M$. Their preimages $M_i$ form a chain, and they are stationary, so $N_i$ is also stationary.
\end{proof}

\begin{thm}
    Let $A$ be a Noetherian ring. Then $A[X]$ is Noetherian.
\end{thm}

\begin{proof}
    Let $I \subset A[X]$. Then $\mf{A}_n \subset A$ be generated by the dominant coefficients of polynomials in $I$ of degree at most $n$. Then we can write $a \in \mf{A}_n$ as $a = \sum \alpha_i \beta_i$ where $\alpha_i \in A$ and $\beta_i$ a dominant coefficient of a polynomial of degree at most $n$ in $I$. Thus the $\mf{A}_n$ form a chain of ideals of $A$ that stabilizer for $n \geq N$. Then $\mf{A}_N = (\beta_1, \ldots, \beta_r)$. Set $Q_i = \beta_i X^N + \cdots \in I$. If $P \in I$, then there exists $S$ such that $P = QS+R$ such that $Q \in A Q_1 + \cdots + A Q_r$ and $\deg R < N$.

    Therefore $P \in (Q_1, \ldots, Q_r) + A[X]_{N-1} \cap I$, so $I \subset (Q_1, \ldots, Q_r) + A[X]_{N-1} \cap I$ and is thus finitely generated.
\end{proof}

\begin{cor}
    Let $B$ be a finitely-generated $A$-algebra. Then if $A$ is Noetherian, $B$ is also Noetherian.
\end{cor}

\begin{cor}
    Any finitely generated algebra over a field is Noetherian.
\end{cor}

\begin{rmk}
    Suppose $A$ is Noetherian and $M$ an $A$-module. If $M$ is finitely generated, then $M$ is Noetherian, but submodules are not necessarily Noetherian. However, they are finitely generated.
\end{rmk}

Suppose $A \subset B$ is an inclusion of rings. Then we say that $x \in B$ is 
\textit{integral over $A$} if there exists a monic polynomial $Q \in A[t]$ such that $Q(x) = 0$.

\begin{prop}
    \label{prop:integral}
    The following are equivalent:
    \begin{enumerate}
        \item $x \in B$ is integral over $A$;
        \item $A[x]$ is a finitely-generated $A$-module;
        \item There exists $A[x] \subset C \subset B$ such that $C$ is a finitely-generated $A$-module.
        \item There exists a faithful $A[x]$-module $M$ which is finitely generated over $A$.
    \end{enumerate}
\end{prop}

\begin{proof}
    \begin{description}
        \item[1 implies 2] Note that $A[x]$ is generated by $1,x,x^2, \ldots, x^m$, where $Q$ has degree $m$.
        \item[2 implies 3] Set $C = A[x]$.
        \item[3 implies 4] Choose $M = C$.
        \item[4 implies 1] Write $M = Am_1 + \cdots + A m_r$. $M$ is an $A[x]$-module, so we can consider $x . M \subset M$. Then for all $i$, we have $x m_i = \sum a_{ij} m_j$, so if we write consider the matrix $T = (a_{ij})$, then this matrix represents the map given by multiplication by $x$. Therefore we have
            \[ \det (T - x I_r) \cdot \begin{pmatrix}
                m_1 \\ \vdots \\ m_r
            \end{pmatrix} = 0, \]
            so set $Q = \det(T - x I_r)$. This is monic and $Q(x) . m_i = 0$ for all $i$, and therefore $Q(x)$ acts by $0$ on $M$. Because $M$ is a faithful $A[x]$-module, we have $Q(x) = 0$.
    \end{description}
\end{proof}

\begin{exer}
    Let $x,y \in B \supset A$. Show that if $x$ and $y$ are integral over $A$ then so are $x+y,xy$.
\end{exer}

\begin{prop}
    Let $A \subset B \subset C$. Assume that $A$ is Noetherian and that $C$ is a finitely-generated $A$-algebra. If $C$ is a finitely-generated $B$-module, then $B$ is a finitely-generated $A$-algebra.
\end{prop}

\begin{proof}
    Write $C = B c_1 + \cdots + B c_r$. Also, we can write $C = A[x_1, \ldots, x_m]$ for some $x_i \in C$. Then we can write $x_i = \sum b_{ij} c_j$ and $c_ic_j = \sum b_{ijk} c_k$ for $b_{ij}, b_{ijk} \in B$. Then $B_0 = A[b_{ij}, b_{ijk}]$ is a finitely-generated $A$-algebra. Any element of $C$ is a polynomial in the $x_i$ with coefficients in $A$, so $C$ is a finitely-generated $B_0$-module. In particular, $B_0$ is Noetherian. Because $B \subset C$, this implies that $B$ is a finitely generated $B_0$-module, so it is a finitely-generated $A$-algebra.
\end{proof}

\begin{cor}
    Let $k$ be a field and $E$ a finitely-generated $k$-algebra. If $E$ is a field, then $E$ is a finite extension of $k$.
\end{cor}

\begin{proof}
    Let $E$ be a finitely-generated $k$-algebra. Then there exist $x_1, \ldots, x_r \in E$ that are algebraically independent over $k$. Then $E$ is algebraic over $k(x_1, \ldots, x_r)$, which is the field of fractions of $k[x_1, \ldots, x_r]$. However, this gives an inclusion $k \subset F \subset E$, where $E$ is a finitely-generated $k$-algebra and $E$ is algebraic over $F$.

    By the proposition, $F$ is a finitely-generated $k$-algebra. Therefore, we can write $F = h[y_1, \ldots, y_s]$, where $y_i = \frac{f_i}{g_i}$. Because $k[x_1, \ldots, x_n]$ is a UFD, then we can write
    \[ h = \prod_{i=1}^s g_i + 1 \in k[x_1, \ldots, x_n]. \]
    $h$ is repatively prime to all of the $g_i$, so $\frac{1}{h} \notin k[y_!, \ldots, y_s]$. This gives a contradiction, so $E$ must be algebraic over $k$.
\end{proof}

\subsection{Primary Decomposition in Noetherian Rings}%
\label{sub:existence_of_}

\begin{defn}
    An ideal $\mf{a} \subset A$ is \textit{irreducible} if for any decomposition $\mf{a} = \mf{b} \cap \mf{c}$, then either $\mf{a} = \mf{b}$ or $\mf{a} = \mr{c}$.
\end{defn}

\begin{exm}
    If $\mf{a}$ is a prime ideal, then $\mf{a}$ is irreducible. To see this, if $\mf{a} \mid \mf{bc}$, then $\mf{a}$ contains one of $\mf{b}, \mf{c}$, and so either $\mf{a} = \mf{b}$ or $\mf{a} = \mf{c}$.
\end{exm}

\begin{rmk}
    Suppose $\mf{m} \subset A$ is a maximal ideal. Then any power $\mf{m}^n$ of $\mf{m}$ is primary.
\end{rmk}

\begin{proof}
    We want to prove that the zero divisors of $A / \mf{m}^n$ are nilpotent. Because $\mf{m}$ is maximal, then $A / \mf{m}^n$ is a local ring with maximal ideal $\mf{m} / m^n$. But then $A / \mf{m}^n \setminus \mf{m} / \mf{m}^n$ are all units, so everything in $\mf{m}$ is nilpotent.
\end{proof}

\begin{lem}
    If $A$ is Noetherian, then every irreducible ideal is primary.
\end{lem}

\begin{proof}
    Let $\mf{a} \subset A$ be irreducible. Then we can pass to the quotient, so we may asusme $\mf{a} = 0$. Let $x,y$ be nonzero with $xy = 0$. We want to show that $x$ is nilpotent.

    Because $A$ is Noetherian, then there exists $n$ such that $\operatorname{Ann} x^n = \operatorname{Ann} x^{n+1}$. We want to show that $(x^n) \cap (y) = 0$, so choose $z = ax^n = by$. Then $zx = ax^{n+1} = byx = 0$, so $a \in \operatorname{Ann} x^{n+1} = \operatorname{Ann} x^n$. However, this means $z = 0$. Because $0$ is irreducible, then $(x^n) = 0$, so $x^n = 0$.
\end{proof}

\begin{cor}
    If $A$ is Noetherian, then every ideal of $A$ has a primary decomposition. In other words, we can write $I = \mf{a}_1 \cap  \cdots \cap \mathfrak{a}_r$, where each $\mf{a_i}$ is primary. 
\end{cor}

\begin{proof}
    Let $S$ be the set of ideals with no primary decomposition. If $S$ is nonempty, then $S$ has a maximal element $I$. To see this, we can use the fact that $A$ is Noetherian, so any chain of ideals in $S$ eventually stabilizes. We know that $I$ is not irreducible, we can write $I = \mf{a} \cap \mathfrak{b}$ such that $I \neq \mf{a}, \mf{b}$. In addition, $\mf{a}, \mf{b} \notin S$, so they have a primary decomposition. This implies that $\mf{a} \cap \mf{b} = I$ has a primary decomposition.
\end{proof}

\begin{rmk}
    This decomposition is not unique. For example, consider $I = \ev{ x^2, xy } \subset k[x,y]$. Then $I = \ev{x} \cap \ev{x^2, xy, y^n}$ for all $n > 0$.
\end{rmk}

\subsection{Artinian Rings}%
\label{sub:artinian_rings}

\begin{prop}
    Assume that $A$ is Artinian.
    \begin{enumerate}
        \item Every prime ideal of $A$ is maximal.
        \item $A$ has finitely many maximal ideals.
        \item The Jacobson radical of $A$ is nilpotent.
    \end{enumerate}
\end{prop}

\begin{proof}
    \begin{enumerate}
        \item Fix a prime ideal $\mf{p}$ and consider the domain $B = A / \mf{p}$. Choose $B \ni x \neq 0$ and consider the decreasing chain $(x^n)$ of ideals. This stabilizes, so there exists $(x^n) = (x^{n+1})$, so we can write $x^n = x^{n+1}y$ for some $y \in B$, and therefore $1 = xy$ because $B$ is a domain. Therefore $x$ has an inverse, so $B$ is a field. Thus $\mf{p}$ is maximal.
        \item Suppose we have infinitely many maximal ideals $\mf{p}_1, \ldots, \mf{p}_n, \ldots$ that are pairwise distinct. Then we form a chain
            \[ \mf{p}_1 \supset \mf{p}_1\mf{p}_2 \supset \cdots \]
            which becomes stationary. Therefore $\mf{p}_1 \cdots \mf{p}_n \subset \mf{p}_{n+1}$, so $\mf{p}_{n+1}$ contains some $\mf{p}_i$. Because these ideals are maximal, this is a contradiction.
        \item Consider $I = \mr{rad}(A) = \mf{m}_1 \cap \cdots \cap \mf{m}_n$. Then the chain $I \supset I^2 \supset \cdots$ stabilizes, so $I^n = I^{n+1}$ for some $n$. Let $J = ((0): I^n)$.\footnote{Here, $(\mf{a}: \mf{b}) = \qty{x \in A \mid x\mf{b} \in \mf{a}}$.} We will show that $J = A$. If not, let $J' \supsetneq J$ such that $J'$ is minimal for this property. Such a $J'$ exists because $A$ is Artinian.

            Let $x \in J' \setminus J$ and consider the ideal $Ax + J$ By minimality of $J'$, we see that $Ix + J \subsetneq J'$ (otherwise $J = J'$ by Nakayama's lemma). Therefore $Ix + J = J$, so $Ix \subset J$ and thus $x \in (J:I)$. Therefore, $I^{n+1}x \subset I^n J = (0)$. This implies $I^nx = 0$, so $x \in J$ and thus $J' = J$. \qedhere
    \end{enumerate}
\end{proof}

\begin{defn}
    An $A$-module $M$ is called \textit{irreducible} if $0$ and $M$ are the only submodules of $M$.
\end{defn}

\begin{defn}
    An $A$-module $M$ is said to be \textit{of finite length}  if there exists a (finite) decreasing sequence of submodules 
    \[ M = M_0 \supsetneq M_1 \supsetneq \cdots \supsetneq M_{n+1} = 0 \]
    such that $M_i / M_{i+1}$ is irreducible for $i = 0, \ldots, n$. In this case, $n$ is actually unique and depends only on $M$. We will call $n$ the \textit{length} of $M$. 
\end{defn}

\begin{prop}
    Let $A$ be a ring. Then $A$ is Artinian if and only if $A$ is of finite length as an $A$-module.
\end{prop}

\begin{proof}
    If $A$ is of finite length, then we have a sequence $A = M_0 \supsetneq \cdots \supsetneq M_{n+1} = 0$ where $M_i / M_{i+1}$ is irreducible. If $\mf{a}_1 \supset \mf{a}_2 \supset \cdots$ is a decreasing chain of ideals, so $\mf{a}_i \cap M_n$ is a decreasing chain of ideals. However, each is either $M_n$ or $0$, so this chain stabilizes. Similarly, the chain $(M_j \cap \mf{a}_i) / (M_{j+1} \cap \mf{a}_i)$ also stabilizes for all $j$. Therefore, there exists $N$ such that for all $i > N$, $M_j \cap \mf{a}_i / (M_{j+1} \cap \mf{a}_i)$ is constant for all $j$, so $\mf{a}_i$ is constant for all $i > N$.

    Now suppose that $A$ is Artinian. Choose $I = \mr{rad}(A) = \mf{m}_1 \cap \cdots \cap \mf{m}_m$, where the $\mf{m}_i$ are the maximal ideals of $A$. Then $I$ is nilpotent, so there exists $n > 0$ such that
    \[ 0 = I^n = \mf{m}_1^n \cdots \mf{m}_m^n. \]
    Then $A = A / I^n = \prod A/ \mf{m}_j^n$ by the Chinese remainder theorem, so $A / \mf{m}_j^n$ is clearly a local ring and is of finite length as an $A$-module. Note that the $A/\mf{m}_j$-vector space $\mf{m}_j^i / \mf{m}_j^{i+1}$ is finite-dimensional because $A$ is Artinian. Therefore $\mf{m}_j^i / \mf{m}_j^{i+1}$ is of finite length.
\end{proof}

\begin{exer}
    If there is an exact sequence $0 \to N \to M \to P \to 0$ of $A$-modules, then $M$ is of finite length if and only if $N$ and $P$ are of finite length. Moreover, $\ell(M) = \ell(P) + \ell(N)$.
\end{exer}

\begin{thm}
    $A$ is Artinian if and only if $A$ is Noetherian and $\dim A = 0$.
\end{thm}

\begin{proof}
    If $A$ is Artinian, we have already proved that $\dim A = 0$. By the previous proposition, because $A$ is of finite length, $A$ is Noetherian. To see this, for a chain $\mf{a}_1 \subset \mf{a}_2 \subset \cdots$, note that $\mf{a}_m \cap M_i / \mf{a}_m \cap M_{i+1}$ stabilizes. We can do this for each $i$, so any increasing chain stabilizes.

    Now assume $A$ is Noetherian and has dimension $0$. We know that $(0)$ has a primary decomposition, so we can write $(0) = \mf{q}_1 \cap \cdots \cap \mf{q}_r$, where each $\mf{q}_i$ is primary. Then $\mf{m}_i = \sqrt{\mf{q}_i}$ is a prime ideal, so it is maximal because $\dim A = 0$. Because $A$ is Noetherian and for all $x \in \mf{m}_i$, $x^n \in \mf{q}_i$ for $n \gg 0$, so there exists $N$ such that $\mf{m}_i^N \subset \mf{q}_i$ for each $i$. Therefore 
    \[ \mf{m}_1^N \cdots \mf{m}_r^N \subset \mf{q}_1 \cdots \mf{q}_r \subset \mf{q}_1 \cap \cdots \cap \mf{q}_r = 0, \]
    so $\mf{m}_1^N \cdots \mf{m}_1^N = 0$. Therefore, $A \cong A / \mf{m}_1^N \times \cdots \times A / \mf{m}_r^N$. Each $A / \mf{m}_i^N$ is of finite length (because each $\mf{m}_i^j / \mf{m}_i^{j+1}$ is a finite-dimensional vector space), so $A$ is of finite length.
\end{proof}

\begin{prop}
    Let $A$ be a Noetherian local ring with maximal ideal $\mf{m}$. Then one of the following holds:
    \begin{enumerate}[label=(\alph*)]
        \item Either $\mf{m}^n \supsetneq \mf{m}^{n+1}$ for all $n$, or;
        \item $\mf{m}^n = 0$ for $n \gg 0$ and in this case, $A$ is Artininian.
    \end{enumerate}
\end{prop}

\begin{proof}
    If $\mf{m}^n = \mf{m}^{n+1}$, then $\mf{m}^n = 0$ by Nakayama's lemma. This implies that $A = A / \mf{m}^n$ is of finite length. Then if $\mf{p}$ is prime, then $\mf{m}^n = (0) \subset \mf{p}$, so $\mf{m} \subset \mf{p}$. Because $\mf{m}$ is maximal, $\mf{m} = \mf{p}$, so $\dim A = 0$.
\end{proof}

\begin{thm}[Structure Theorem for Artinian Rings]
    An Artinian ring is uniquely up to isomorphism a finite product of Artinian local rings.
\end{thm}

\begin{proof}
    Previously, we proved that $A = \prod A / \mf{m}_i^N$. Each of these is a local Artinian ring.
\end{proof}

\chapter{Linear Algebra of Modules}%
\label{cha:linear_algebra_of_modules}

\begin{prop}
    Assume $M, N, P$ are $A$-modules.
    \begin{enumerate}
        \item The sequence $N \to M \to P \to 0$ is exact if and only if for all $A$-modules $Q$, the sequence
            \[ 0 \to \Hom(P,Q) \to \Hom(M,Q) \to \Hom(N,Q) \]
            is exact.
        \item The sequence $0 \to N \to M \to P $ is exact if and only if for all $A$-modules $Q$, the sequence
            \[ 0 \to \Hom(Q,N) \to \Hom(Q,M) \to \Hom(Q,P) \]
            is exact.
    \end{enumerate}
\end{prop}

\begin{proof}
    This is left as an exercise.
\end{proof}

\begin{rmk}
    In general, if $0 \to M \to N \to P \to 0$ is exact, then
    \[ 0 \to \Hom(P,Q) \to \Hom(N,Q) \to \Hom(M,Q) \\
    0 \to  \Hom(Q,M) \to \Hom(Q,N) \to \Hom(Q,P) \]
    are exact but the last morphism is not necessarily surjective.
\end{rmk}

\begin{defn}
    A module $Q$ is \textit{projective} if the functor $\Hom(Q,-)$ is exact. Here, exact means that short exact sequences are preserved. Similarly, a module $I$ is \textit{injective} if the functor $\Hom(-,I)$ is exact.
\end{defn}

\begin{prop}
    A module $Q$ is projective if and only if $Q$ is a direct factor of a free module. In other words, there exists a free module $F$ and $A$-module $Q'$ such that $F = Q \oplus Q'$.
\end{prop}

\begin{proof}
    Suppose $Q$ is projective. Then there is a surjection $\pi: A^{(S)} \to Q \to 0$. Because $Q$ is projective, there exists a map $\theta$ such that $\pi \circ \theta = \mr{id}$. Therefore $A^{(S)} \cong Q  \oplus Q'$, where $Q'$ is the kernel of $\pi$.

    On the other hand, if $A^{(S)} = Q \oplus Q'$, then for any diagram of the form
    \begin{equation}
    \begin{tikzcd}
        M \arrow{r} & P \arrow{r} & 0 \\
                    & Q \arrow{u}
    \end{tikzcd}
    \end{equation}
    we can embed $Q$ in $A^{(S)}$ and then use projectivity of free modules (because $\Hom(A,M) = M$).
\end{proof}

\begin{rmk}
    If $M$ is projective and finitely generated, then it is a direct factor of a finite free module.
\end{rmk}

\begin{defn}
    A \textit{projective resolution} of an $A$-module $M$ is a right bounded complex
    \[ \cdots \to P_{n+1} \to P_n \to \cdots \to P_0 \to 0 \]
    of projective modules such that there exists $P_0 \to M$ such that
    \[ \cdots P_{n+1} \to P_n \to \cdots \to P_0 \to M \to 0 \]
    is exact.
\end{defn}

\begin{exer}
    Show that any module has a projective resolution (\textbf{Hint:}  construct a free resolution). In addition, any two projective resolutions are homotopic.
\end{exer}

\begin{defn}
    %A map of complexes $\phi: M_{.} \to N_{.}$ is a \textit{homotopy} if there exist maps 
    $k_q: M_q \to N_{q+1}$ such that 
    $\phi_q = d^N \circ k_q + k_{q-1} \circ d^M$. 
\end{defn}

\section{Tor and Ext Functors}%
\label{sec:derived_functors}

Note that for a complex, we can compute the \textit{homology} 
$H_q(M_{\bullet}) \coloneqq \ker d_q / \Im d_{q+1}$. 
This measures the defect of the complex from being exact. 
For functors that are not exact, we can construct \textit{derived functors} that measure the defect of exactness. 
Let $F: A\text{-}\ms{Mod} \to A\text{-}\ms{Mod}$ be right exact. 
Then for any $M$, we can consider a projective resolution 
$P_{\bullet} \to M \to 0$. 
Applying $F$ to $P_{.}$, then the \textit{left derived functor} 
$L_{\bullet} F(M)$ is defined by $L_{\bullet} F(M) = H_{\bullet}(F(P_{\bullet}))$.

\begin{prop}
    If $0 \to M \to N \to P \to 0$ is exact, then we have a long exact sequence
    \[ \cdot \to L_1F(M) \to L_qF(N) \to L_q F(P) \to L_{q-1} F(M) \to \cdots \to L_0 F(M) \to L_0 F(N) \to L_0 F(P) \to 0. \]
\end{prop}
    
Recall that the tensor product $M \otimes N$ of two modules $M,N$ is an $A$-module with a bilinear map $M \times N \to M \otimes N$ such that all bilinear maps $M \times N \to P$ factor through $M \otimes N$.

\begin{prop}
    The functors $- \otimes N, \Hom(N,-)$ are an adjoint pair.
\end{prop}

\begin{cor}
    If $N \to M \to P \to 0$ is exact, then 
    \[ N \otimes Q \to M \otimes Q \to P \otimes Q \to 0 \]
    is exact.
\end{cor}

\begin{defn}
    A module $Q$ is \textit{flat} if $- \otimes Q$ is exact. 
\end{defn}

We can defined the left derived functors $\operatorname{Tor}_q(Q,M)$ of the tensor product.

\begin{prop}
    Any projective module is flat.
\end{prop}

\begin{proof}
    Clearly free modules are flat, so write $Q \oplus Q' = A^{(S)}$ and then note that the tensor product distributes over the sum.
\end{proof}

\begin{prop}
    The following are equivalent:
    \begin{enumerate}
        \item $M$ is flat over $A$.
        \item If $N' \hookrightarrow N$, then $M \otimes N' \hookrightarrow M \otimes N$.
        \item For all finitely generated ideals $I \subset A$, $I \otimes M \hookrightarrow M$.
        \item For any finitely generated ideals $I \subset A$, $\operatorname{Tor}_1(M,A/I) = 0$.
        \item For any finitely generated module $N$, we have $\operatorname{Tor}(M,N) = 0$.
        \item For all $a_i \in A$ and $x_i \in M$ such that $\sum a_i x_i = 0$ there exist $y_1, \ldots, y_s \in M$ and $b_{ij}$ such that $x_i = \sum b_{ij} y_j$.
    \end{enumerate}
\end{prop}

\begin{proof}
    It is clear that 1 is equivalent to 2 implies 3 implies 4 implies 5. The directions 3 implies 2 and 4 implies 3 are left to the reader. 
    \begin{description}
        \item[1 implies 6] Choose $a_i \in A, x_i \in M$ such that $\sum_{i=1}^r a_i x_i = 0$. Then define a map $A^r \xrightarrow{f} A$ by
            \[ f(b_1, \ldots, b_r) = \sum_{i=1}^r a_i b_i \]
            and define $K = \ker f$. 
            Because $M$ is $A$-flat, we have an exact sequence
            \[ 0 \to K \otimes M \to M^r \to M. \]
            Then $(x_1, \ldots, x_r) \in \ker f \otimes \mr{id}_M$. Therefore there exists $b_1, \ldots, b_s \in K$ and $y_1, \ldots, y_s \in M$ such that 
            \[ (x_1, \ldots, x_r) = \sum_{j=1}^s b_j \otimes y_j. \]
            Writing $b_j = (b_{1j}, \ldots, b_{rj})$, we obtain the identity
            \[ \sum_{i=1}^r b_{ij} a_i = 0 \]
            and thus $x_i = \sum b_{ji} y_j$.
        \item[6 implies 3] Choose an ideal $I \subset A$. Consider the map $0 \to I \otimes M \to M$. Then for any element in the kernel, we can write 
            \[ \sum_i a_i \otimes x_i \mapsto \sum a_i x_i = 0. \]
            Then we can write $x_i = \sum b_{ij} y_j$ and so
            \[ \sum a_i \otimes x_i = \sum \sum a_i \otimes b_{ij}y_j = \sum \qty(\sum a_i b_{ij}) \otimes y_j = 0 \]
            and thus $I \otimes M \to M$ is injective. \qedhere
    \end{description}
\end{proof}

Let $\phi: A \to B$ be a map of rings and let $M$ be a $B$-module. Define $\phi$ to be \textit{flat} if $B$ is flat as an $A$-module.

\begin{prop}
    If $\phi: A \to B$ is flat and $M$ is a flat $B$-module, then $M$ is also flat as an $A$-module.
\end{prop}

\begin{proof}
    Let $S$ be an $A$-module. Then $S \otimes_A M = S \otimes_A (B \otimes_B M) = (S \otimes_A B) \otimes_B M$. If $0 \to N_1 \to N_2$ is an exact sequence of $A$-module, by flatness of $B$ as an $A$-module, then
    \[ 0 \to N_1 \otimes_A B \to N_2 \otimes_A B\]
    is exact. Because $M$ is flat over $B$, we see that
    \[ 0 \to (N_1 \otimes_A B) \otimes_B M \to (N_2 \otimes_A B) \otimes_B M \]
    is exact, as desired.
\end{proof}

Now let $M$ be an $A$-module. Then for any map $A \xrightarrow{\phi} B$, we can consider the $B$-module $M_{(B)} \coloneqq M \otimes_A B$.

\begin{prop}
    If $M$ is $A$-flat, then $M_{(B)}$ is $B$-flat.
\end{prop}

\begin{proof}
    For a $B$-module $S$, write
    \begin{align*}
        S \otimes_B M_{(B)} &= S \otimes_B (M \otimes_A B) \\
                            &\cong S \otimes_B (B \otimes_A M) \\
                            &\cong (S \otimes_B B) \otimes_A M \\
                            &\cong S \otimes_A M.
    \end{align*}
    Thus if $0 \to S_1 \to S_2$ is exact, then $0 \to S_1 \otimes_A M \to S_2 \otimes_A M$ is exact because $M$ is $A$-flat, as desired.
\end{proof}

\begin{prop}
    Let $S \subset A$ be a multiplicative subset of $A$. Then the morphism of rings $A \to S^{-1}A$ is flat.
\end{prop}

The proof is left to the reader. This can be reformulated as $M \otimes_A S^{-1} A \cong S^{-1}M$.

Now we will give some remarks about the $\operatorname{Ext}$ functors. For any left exact functor, we may define the right derived functors $R^{\bullet} F$ by
\[ R^i F(M) = H^i(F(I^{\bullet})) \]
where $M \to I^{\bullet}$ is an injective resolution. Then we will define the right derived functors of $\Hom_A(N,-)$ by $\Ext_A^i(N,-)$.

\begin{prop}
    If $M$ is injective, then $\Ext_A^i(N,M) = 0$ for all $i > 0$. Similarly, if $N$ is projective, then $\Ext_A^i(N,M) = 0$ for all $I > 0$.
\end{prop}

\begin{rmk}
    We can compute $Ext^i(N,M)$ using a projective resolution of $N$.
\end{rmk}

\begin{prop}
    Let $A \to B$ be a morphism of rings and let $M, N$ be $A$-modules. Then let $M_{(B)}, N_{(B)}$ be their base changes to $B$. Then we have
    \[ \Ext_B^i (M_{(B)}, N_{(B)}) = \Ext_A^i(M,N)_{(B)} \]
    and 
    \[ \Tor_i^B(M_{(A)}, N_{(B)}) = \Tor_i^A(M,N)_{(B)} \]
    if $B$ is $A$-flat.
\end{prop}

\begin{proof}
    This follows from the definition of $\Ext, \Tor$ using projective resolutions using the following facts.
    \begin{enumerate}
        \item If $M$ is $A$-projective, then $M_{(B)}$ is $B$-projective.
        \item Since $B$ is $A$-flat, for any complex $X^{\bullet}$ of $A$-modules, then $H^{\bullet}(X_{(B)}^{\bullet}) = H^{\bullet}(X^{\bullet})_{(B)}$.
    \end{enumerate}
\end{proof}

\section{Flatness}%
\label{sec:flatness}

\begin{prop}
    Let $A$ be a local ring. Then any finitely generated flat $A$-module is free. In particular, free, projective, and flat are equivalent for $A$-modules.
\end{prop}

\begin{proof}
    We know that free implies projective implies flat. Therefore we will show that if $M$ is flat, then it is free. Assume that $M$ is finitely generated and $A$-flat. Let $k = A / \mf{m}$ be the residue field of $A$. Define $\ol{M} = M \otimes_A k$, which is a vector space of finite dimension over $k$. Then there exists $x_1, \ldots, x_r \in M$ that descend to a basis of $\mf{M}$.

    Then the map $A^r \to M, (a_i) \mapsto \sum a_i x_i$ is surjective by Nakayama's lemma. We will prove that this map is injective by induction. If $r = 1$, then suppose $a x_1 = 0$. Then there exist $y_1, \ldots, y_s, b_{11}, \ldots, b_{1s}$ such that
    \[ x_1 = \sum_{j=1}^s b_{ij} y_j \]
    where $ab_{ij} = 0$ for all $j = 1, \ldots, s$. Because $\ol{x}_1 \neq 0$, there exists $j$ such that $\ol{b}_{1j} \neq 0$ so $b_{1j}$ is invertible in $A$. Thus $a = 0$.

    Now suppose $a_1 x_1 + \cdots + a_r x_r = 0$. Then there exist $y_1, \ldots, y_s$ and $b_{ij}$ such that
    \[ x_i = \sum b_{ij} y_j \]
    and \[ \sum a_i \begin{pmatrix}
        b_{i1} \\ \vdots \\ b_{ij}
    \end{pmatrix} = 0. \]
    Because $\ol{x}_r \neq 0$, we see that $\ol{b}_{rj} \neq 0$ for some $j$ and thus $b_{ij}$ is a unit. Then $a_1 b_{1j} + \cdots + a_r b_{rj} = 0$, so we can write
    \[ \sum a_i (x_i - c_i x_r) = 0. \]
    We know that $\ol{x}_1 - c_1 \ol{x}_r, \ldots, \ol{x}_{r-1} - c_{r-1} \ol{x}_r$ are linearly independent over $k$, so from the induction $a_1 = \cdots = a_r = 0$ and thus $a_r = 0$.
\end{proof}


\begin{rmk}
    If $M$ is not finitely generated, the proposition is false. An example is given by taking the field of fractions of a local domain.
\end{rmk}

When proving the proposition, we in fact proved that
\begin{lem}
    If $x_1, \ldots, x_r \in M$ with $M$ a flat $A$-module for $A$ a local ring and $\ol{x}_1, \ldots, \mf{x}_r$ are linearly independent in $M \otimes_A k$, then $x_1, \ldots, x_r$ are linearly independent in $M$.
\end{lem}

\begin{prop}
    Suppose that $A \to B$ is flat and $I_1, I_2$ are ideals of $A$. Then
    \begin{enumerate}
        \item $(I_1 \cap I_2)B = I_1 B \cap I_2 B$;
        \item If $I_2$ is finitely generated, then $(I_1 : I_2)B = (I_1 B : I_2 B)$.
    \end{enumerate}
\end{prop}

\begin{proof}
    The proof is a formal consequence of flatness.
    \begin{enumerate}
        \item Consider the exact sequence $0 \to I_1 \cap I_2 \to A \to A/I_1 \times A/I_2$. Tensoring with $B$, we obtain an exact sequence
            \[ 0 \to (I_1 \cap I_2) \otimes B \to B \to B/I_1 B \times B/I_2 B. \]
            But then $(I_1 \cap I_2) \otimes B = (I_1 \cap I_2)B$, but the kernel of the last map is clearly $I_1 B \cap I_2 B$.
        \item Set $I_2 = (x_1, \ldots, x_r)$. Then because
            \[ (I_1 : I_2) = \bigcap_{i=1}^r (I_1 : x_i A), \]
            it suffices to prove the result for $I_2$ a principal ideal. We have an exact sequence
            \[ 0 \to (I_1 : x A) \to A \xrightarrow{\times x} A/I_1. \]
            Tensoring by $B$, we obtain
            \[ 0 \to (I_1 : x A) \otimes B \to B \to B/I_1 B, \]
            and by analysing the kernel, we see that $(I_1: xA) B = (I_1 B: x B)$. By repeated application of the previous part, the desired result follows. \qedhere
    \end{enumerate}
\end{proof}

\begin{exm}
    We will give an example where the previous proposition is not true in general. Let $A = k[x,y]$ and $B = A/xA = k[y]$. Then choose $I_1 = (x+y), I_2 = (y)$, so $I_1 \cap I_2 = I_1 I_2 = ( (x+y)y )$. But then we have $(I_1 \cap I_2) B = y^2 B$, but $I_1 B \cap I_2 B = yB$.

    Another example is $A = k[x,y], B = k[x,y,z]/(xz-y) \cong k[x,z], I_1 = xA, I_2 = yA$. Here we can check that $(I_1 \cap I_2) = (xy)$, that $(I_1 \cap I_2)B = x^2z B$, but $I_1 B \cap I_2 B = xz B$. Viewing this geometrically as $\Spec B \to \Spec A$, we can check the fiber over $(0,0)$ and see that the map is not flat.
\end{exm}

\begin{prop}
    Let $A \xrightarrow{\varphi} B$ be a ring homomorphism. The following are equivalent:
    \begin{enumerate}
        \item $B$ is flat over $A$;
        \item $B_{\mf{P}}$ is flat over $A_{\mf{p}}$ for all $\mf{P} \in \Spec B$ and $\mf{p} = \varphi^{-1} (\mf{P})$.
        \item $B_{\mf{P}}$ is flat over $A_{\mf{p}}$ for any $\mf{P}$ maximal.
    \end{enumerate}
\end{prop}

\begin{proof}
    \begin{description}
        \item[1 implies 2:] We know that $B_{\mf{p}}$ is flat over $A_{\mf{p}}$. But then $B_{\mf{P}}$ is flat over $B_{\mf{p}}$ because it is a localization. By transitivity of flatness, $B_{\mf{P}}$ is flat over $A_{\mf{p}}$.
        \item[2 implies 3:] This is obvious.
        \item[3 implies 1:] Note that for all $\mf{P}$ maximal, $\Tor_i^A(B,N)_{\mf{P}} = 0$ for $i > 0$. This implies that $\Tor_i^!(B,N) = 0$, and thus $B$ is flat over $A$. To get that the first $\Tor$ is zero we need to use the lemma belos. \qedhere
    \end{description}
\end{proof}

\begin{lem}
    Let $\varphi: A \to B$ be a morphism of rings and choose $\mf{P} \in \Spec B$. Then let $\mf{p} = \varphi^{-1}(\mf{P})$ and $N$ an $A$-module. Then $\Tor_i^A(B,N)$ is a $B$-module and $\Tor_i^A(B,N)_{\mf{P}} = \Tor_i^{A_{\mf{p}}}(B_\mf{P},N_{\mf{p}})$.
\end{lem}

\begin{proof}
    Let $X_{\bullet} \to N$ be a projective resolution. Then $\Tor$ is computed by the homology of the complex $B \otimes_A X_{\bullet}$. When we localize, we localize the homology at the $B$ term. However, $B_{\mf{p}} \otimes_A X_{\bullet} = B_{\mf{P}} \otimes_{A_{\mf{p}}} (X_{\bullet})_{\mf{p}}$, so because $X_i$ is $A$-projective, then $X_{\mf{p}}$ is $A_{\mf{p}}$-projective, and thus $(X_{\bullet})_{\mf{p}}$ is a projective resolution of $N_{\mf{p}}$. Thus the complex $B_{\mf{P}} \otimes_{A_{\mf{p}}} (X_{\bullet})_{\mf{p}}$ computes the $\Tor$ as desired.
\end{proof}

\begin{defn}
    An $A$-module $N$ is said to be \textit{faithfully flat} if
    \begin{enumerate}
        \item $N$ is $A$-flat;
        \item For any sequence $P \to Q \to R$ of $A$-modules, if $P \otimes N \to Q \otimes N \to R \otimes N$ is exact, then $P \to Q \to R$ is exact.
    \end{enumerate}
\end{defn}

\begin{thm}
    Let $M$ be an $A$-module. Then the following are equivalent:
    \begin{enumerate}
        \item $M$ is faithfully flat over $A$;
        \item $M$ is flat and for any nonzero $N$, $M \otimes N \neq 0$;
        \item $M$ is flat and for all maximal ideals $\mf{m} \subset A$, $\mf{m} . M \neq M$.
    \end{enumerate}
\end{thm}

\begin{proof}
    \begin{description}
        \item[1 implies 2:] Choose the sequence $0 \to N \to 0$. Then tensor with  $M$. If $N \otimes M = 0$, the sequence is now exact, then the original sequence is exact, and thus $N = 0$.
        \item[2 implies 3:] Consider $N = A/\mf{m}$. Then $N \otimes M = M / \mf{m}M \neq 0$, so $M \neq \mf{m}M$.
        \item[3 implies 2:] Choose $0 \neq x \in N$ and set $I = \operatorname{Ann}(x) \subset \mf{m}$ for some maximal ideal $\mf{m}$. Then $Ax = A/I$, so $Ax \otimes M \cong M / IM \neq 0$. Because $Ax$ injects into $N$, $Ax \otimes M$ injects into $N \otimes M$, which must be nonzero.
        \item[2 implies 1:] Consider the sequence $P \xrightarrow{f} Q \xrightarrow{g} R$. Then because $M$ is flat, $\ker(g \otimes \mr{id}_M) = \ker(g) \otimes M$ and $\Im(f \otimes \mr{id}_M) = \Im(f) \otimes M$. If $g \circ f = 0$, then $\Im(g \circ f) = 0$, which happens iff $\Im(g \circ f) \otimes M = 0$. Then $\Im((g \otimes \mr{id}_M) \circ (f \otimes \mr{id}_M)) = 0$. 
            
            If $P \otimes M \to Q \otimes M \to R \otimes M$ is exact, then $P \to Q \to R$ is a complex. Finally, we need that $\ker g = \Im f$. By flatness of $M$, we can tensor to find that $\ker g / \Im f \otimes M = 0$ and then we see that $\ker g / \Im f = 0$. \qedhere
    \end{description}
\end{proof}

\begin{cor}
    Let $A \to B$ be a local homomorphism and let $M$ be a finitely-generated $B$=module. Then $M$ if flat over $A$ if and only if $M$ is faithfully flat over $A$.
\end{cor}

\begin{proof}
    Clearly faithfully flat implies flat. Then we need to show that $M \neq \mf{m}_A. M$. By Nakayama's lemma, we know that $M \otimes k_B \neq 0$, so $\mf{m_B} M \neq M$. In particular, $\mf{m}_A M \neq M$. In particular, this implies that item \textbf{3} of the previous theorem holds. Thus $M$ is faithfully flat over $A$.  
\end{proof}

\begin{rmk}
    This also shows that flat and faithfully flat are equivalent over local rings. Alternatively, we can use the equivalence of flat and free.
\end{rmk}

\begin{rmk}
    Faithful flatness is transitive. In addition, if $A \to B$ is a morphism of rings, and $M$ is faithfully flat over $A$, then $M \otimes_A B$ is faithfully flat over $B$.
\end{rmk}

\begin{prop}
    Let $M$ be a faithfully flat $B$-module which is faithfully flat over $A$. Then $B$ is faithfully flat over $A$.
\end{prop}

\begin{proof}
    Let $N$ be an $A$-module. Then $(B \otimes_A N) \otimes_B M = M \otimes_A N \neq 0$ if $N \neq 0$. This implies that $B \otimes_A N$ is nonzero. Now it suffices to show that $B$ is flat over $A$.

    Let $(S)$ be an exact sequence of $A$-modules. Then if we consider $((S) \otimes_A M) = (S) \otimes_A M$, this is exact by flatness of $M$ over $A$. By faithful flatness of $M$ over $B$, this implies that $(S) \otimes_A B$ is exact.
\end{proof}

\begin{prop}
    Let $A \to B$ be faithfully flat. Then
    \begin{enumerate}
        \item For any $A$-module $N$, the map $N \to N \otimes_A B$ is injective;
        \item If $I \subset A$ is an ideal, then $IB \cap A = I$;
        \item The map $\Spec B \to \Spec A$ is surjective.
    \end{enumerate}
\end{prop}

\begin{proof}
    \begin{enumerate}
        \item Let $0 \neq x \in N$. Then $Ax \otimes B \hookrightarrow N \otimes B$. Because $B$ is faithfully flat, $Ax \otimes B \neq 0$.
        \item Recall that $B/IB = B \otimes A/I$. Then the map $A/I \to B/IB$ is injective. Therefore we have a map $A \to B/IB$ which has kernel $I = IB \cap A$.
        \item Choose $\mf{p} \in \Spec A$. Then $B_{\mf{p}} = B \otimes_A A_{\mf{p}}$ and $A_{\mf{p}} \to B_{\mf{p}}$ is faithfully flat by base change. This means that $\mf{p} B_{\mf{p}} \subsetneq B_{\mf{p}}$. Thus if we choose $\mf{m}$ to be a maximal ideal of $B_{\mf{p}}$ containing $\mf{p} B_{\mf{p}}$, we see that $\mf{m} \cap A_{\mf{p}} \supset \mf{p}A_{\mf{p}}$ and thus they are equal by maximality of $\mf{p}A_{\mf{p}}$. Then if we choose $\mf{P} = \mf{m} \cap B$, we see that 
            \begin{align*}
                \mf{P} \cap A &= \mf{m} \cap A \\
                              &= \mf{m} \cap A_{\mf{p}} \cap A \\
                              &= \mf{p} A_{\mf{p}} \cap A \\
                              &= \mf{p}.
            \end{align*}
            Thus the image of $\mf{P}$ is $\mf{p}$. \qedhere
    \end{enumerate}
\end{proof}

\begin{thm}
    Let $\varphi: A \to B$ be a map of rings. The following are equivalent:
    \begin{enumerate}
        \item The map $\varphi$ is faithfully flat.
        \item The map $\varphi$ is flat and $\Spec B \to \Spec A$ is surjective.
        \item The map $\varphi$ is flat and for all maximal ideals $\mf{m}$ of $A$, there exists some maximal ideal $\mf{m}'$ of $B$ such that $\varphi^{-1}(\mf{m}') = \mf{m}$.
    \end{enumerate}
\end{thm}

\begin{proof}
    \begin{description}
        \item[1 implies 2:] This is the previous proposition.
        \item[2 implies 3:] Choose a maximal ideal $\mf{m} \subset A$. Then there exists $\mf{P} \in \Spec B$ such that $\varphi^{-1}(\mf{P}) = \mf{m}$. But then if $\mf{m}'$ is any maximal ideal containing $\mf{P}$, we see that $\varphi^{-1}(\mf{m}') = \varphi^{-1}(\mf{P}) = \mf{m}$ by maximality of $\mf{m}$.
        \item[3 implies 1:] We want to prove that $B \neq \mf{m}B$ for any maximal ideal $\mf{m}$ of $A$. Then there exists $\mf{m}'$ such that $\varphi^{-1}(\mf{m}') = \mf{m}$. But then $B \supsetneq \mf{m}' \supset \mf{m}B$. \qedhere
    \end{description}
\end{proof}

\begin{prop}[Descent]
    Let $A \to B$ be faithfully flat and $M$ be an $A$-module. Then
    \begin{enumerate}
        \item $M$ is flat (resp. faithfully flat) if and only if $M \otimes_A B$ is $B$-flat (resp. $B$-faithfully flat).
        \item Assume $A$ is a local ring and $M$ is finitely-generated. Then $M$ is free if and only if $M \otimes_A B$ is $B$-free.
    \end{enumerate}
\end{prop}

\begin{proof}
    \begin{enumerate}
        \item Let $(S)$ be an exact sequence. Then $(S) \otimes_A B$ is exact, so $S \otimes_A B \otimes M \otimes_A B = (S \otimes_A M) \otimes_A B$ is exact. By faithful flatness of $B$, $(S) \otimes_A M$ is exact. Now if $N \neq 0$ is another $A$-module, we know that $M_{(B)} \otimes N_{(B)} \neq 0$, but this is the same as $(M \otimes_A N)_{(B)}$, so $M \otimes_A N$ is nonzero.
        \item Assume that $A$ is local. Then suppose $M \otimes_A B$ is free. Therefore $M \otimes_A B$ is faithfully flat. But then, $M$ is faithfully flat over $A$, which means that $M$ is free because $M$ is finitely generated. \qedhere
    \end{enumerate}
\end{proof}

\begin{exer}
    Let $A \subset B$ be integral domains. Assume that $A$ and $B$ have the same field of fractions. Prove that $A \hookrightarrow B$ is faithfully flat if and only if $A = B$.
\end{exer}

\section{More on Integral Dependence}%
\label{sec:more_on_integral_dependence}

Recall Proposition \textbf{\ref{prop:integral}}. 

\begin{cor}
    Let $x_1, \ldots, x_n \in B$. If each $x_i$ is integral over $A$, then $A[x_1, \ldots, x_n]$ is a finitely-generated $A$-module.
\end{cor}

\begin{cor}
    Let $C \subset B$ be the set of integral elements over $A$. Then $C$ is a subring of $B$.
\end{cor}

\begin{proof}
    Note that $x + y, xy \in A[x][y] \in A[x,y]$, which is a finitely-generated $A$ module. Therefore they are integral over $A$.
\end{proof}

\begin{rmk}
    The ring $C$ is not necessarily finitely-generated over $A$. For an example, choose $\Z \ol{\Z} \subset \ol{\Q}$.
\end{rmk}

\begin{defn}
    Let $A \subset B$. Then we say that $B$ is \textit{integral over $A$} if all elements of $B$ are integral over $A$.
\end{defn}

\begin{cor}
    Let $A \subset B \subset C$ be extensions of rings. Then if $B$ is integral over $A$ and $C$ is integral over $B$, then $C$ is integral over $A$.
\end{cor}

Proof of this is left to the reader.

\begin{defn}
    Let $A$ be an integral domain. We say that $A$ is \textit{integrally closed} if for all $x \in K = \operatorname{Frac} A$, then $x$ is integral over $A$ if and only if $x \in A$.
\end{defn}

\begin{defn}
    Assume that $A \subset B$ is an inclusion of rings. Then the \textit{integral closure of $A$ inside $B$} is the set of all elements of $B$ that are integral over $A$. 
\end{defn}

\begin{exm}
    A typical example of this situation is when $A$ is a domain, $K$ is its fraction field, and $L / K$ is a field extension. Then we can consider the integral closure $B$ of $A$ inside $L$. In number theory, if $K$ is a number field, we define its \textit{ring of integers} $\mc{O}_K$ to be the integral closure of $\Z$ in $K$.
\end{exm}

\begin{exer}
    If $K = \Q(\zeta_p)$, prove that $\mc{O}_K = \Z[\zeta_p]$.
\end{exer}

\begin{exer}
    Let $B$ be an integral domain and $A \subset B$. Prove that the integral closure of $A$ inside $B$ is integrally closed.
\end{exer}

\begin{lem}
    Let $B$ be a domain that is integral over $A$. Then $A$ is a field if and only if $B$ is a field.
\end{lem}

\begin{proof}
    Assume that $A$ is a field. Now choose $0 \neq x \in B$. But then we know that 
    \[ x^n + a_{n-1}x^{n-1} + \cdots + a_0 = 0 \]
    for some $a_i \in A$ and $a_0 \neq 0$. But then we have
    \[ x^{-1} = -a_0^{-1} \qty( \sum_{i=1}^n a_i x^i ) \]
    and so $x^{-1} \in B$.

    Now assume that $B$ is a field. Choose $0 \neq x in A$. Then $x^{-1} \in B$. This means that $x^{-1}$ is integral over $A$, which means that
    \[ a^{-n} + a_{n-1} x^{-(n-1)} + \cdots + a_0 = 0 \]
    for some $a_i \in A, a_0 \neq 0$. Then if we multiply by $a^{n-1}$, we obtain
    \[ a^{-1} + a_{n-1} + a_{n-2} x + \cdots + a_0 x^{n-1} = 0, \]
    which means $x^{-1} \in A$.
\end{proof}

\begin{cor}
    Let $A \subset B$ and $B$ be integral over $A$. Let $\mf{P} \in \Spec B$ and $\mf{p} = A \cap \mf{P}$. Then $\mf{p}$ is maximal if and only if $\mf{P}$ is maximal.
\end{cor}

\begin{proof}
    Note that $B/ \mf{P}$ is integral over $A / \mf{p}$. Then we apply the lemma to $A / \mf{p} \subset B / \mf{P}$.
\end{proof}

We can refine this into going-up and going-down. Let $\phi: A \to B$ be a morphism of rings and let $\phi^*: \Spec B \to \Spec A$ be the induced map of spaces.

\begin{defn}[Going-up]
    A ring homomorphism $\phi$ satisfies the \textit{Going-up property} if the following holds:

    Let $\mf{p} \subset \mf{p}'$ be prime ideals of $A$ and suppose that $\phi^*(\mf{P}) = \mf{p}$. Then there exists $\mf{P}' \subset \mf{P}$ such that $\phi^*(\mf{P}') = \mf{p}'$.
\end{defn}

\begin{defn}[Going-down]
    A ring homomorphism $\phi$ satisfies the \textit{Going-down property} if the following holds:

    Let $\mf{p} \subset \mf{p}'$ be prime ideals of $A$. Then let $\mf{P}' \in \Spec B$ satisfy $\phi^*(\mf{P}') = \mf{p}'$. Then there exists $\mf{P} \subset \mf{P}'$ with $\phi^*(\mf{P}) = \mf{p}$.
\end{defn}

\begin{lem}
    \label{lem:goingdown}
    The going-down property is equivalent to the following:

    For all $\mf{p} \in \Spec A$ and $\mf{P}$ a minimal prime ideal of $B$ containing $\mf{p}B$, we have $\mf{P} \cap A = \mf{p}$.
\end{lem}

\begin{proof}
    First suppose that going-down holds. Then choose $\mf{P}$ be a minimal prime containing $\mf{p}B$. Then $\mf{p}' = \phi^{-1} (\mf{P}) \supset \mf{p}$. If $\mf{p}' \neq \mf{p}$, then there exists $\mf{P}_0 \subset \mf{P}$ such that $\varphi^{-1}(\mf{P}_0) = \mf{P}$, which contradicts minimality.

    Now suppose the other condition holds. Suppose $\mf{P}'$ goes to $\mf{p}' \supset \mf{p}$. Then we know that $\mf{p} B \subset \mf{p}' B \subset \mf{P}'$. If we fix $\mf{P}_0$ to be the minimal prime containing $\mf{p}B$, then we see that $\mf{P}_0 \cap A = \mf{p}$.
\end{proof}

\begin{thm}
    If $\phi: A \to B$ is flat, then going-down holds.
\end{thm}

\begin{proof}
    Fix $\mf{p} \subset \mf{p}'$ and let $\mf{P}'$ lie over $\mf{p}'$. Then we know that $B_{\mf{P}'}$ is flat over $\mf{A}_{\mf{p}'}$. Because $A_{\mf{p}'}$ is local and the map $A_{\mf{p}'} \to \mf{B}_{\mf{P}'}$ is local, it is faithfully flat. This implies that the map $\Spec B_{\mf{P}'} \to \Spec A_{\mf{p}'}$ is surjective, so there exists $\mf{P}_1 \in \Spec B_{\mf{P}'}$ such that $\phi^{-1}(\mf{P}_1) = \mf{p} A_{\mf{p}'}$. 

    Now set $\mf{P} \coloneqq \mf{P}_1 \cap B$. Then we see that
    \begin{align*}
        \phi^{-1}(\mf{P}) &= \phi^{-1}(\mf{P}_1 \cap B) \\
                          &= \phi^{-1}(\mf{P}_1) \cap A \\
                          &= \mf{p A_{\mf{p}'}} \cap A \\
                          &= \mf{p}. \qedhere
    \end{align*}
\end{proof}

We will see consequences of this result in algebraic geometry.

We will now consider integral ring extensions $A \subset B$.

\begin{thm}[Cohen-Seidenberg]
    Suppose $A \subset B$ is an integral extension. Then the following hold:
    \begin{enumerate}
        \item The map $\Spec B \to \Spec A$ is surjective.
        \item There are no inclusion relations between the prime ideals of $B$ which are above a fixed prime ideal of $A$.
        \item Going-up holds for $A \subset B$.
        \item If $A$ is local with maximal ideal $\mf{m}$, then the prime ideals of $B$ lying over $\mf{m}$ are precisely the maximal ideals of $B$.
        \item Assume further that $A$ and $B$ are integral domains and that $A$ is integrally closed. Then going-down holds for $A \subset B$.
        \item If $B$ is the integral closure of $A$ in a normal extension of field $L$ of $K \coloneqq \operatorname{Frac} A$, then any two prime ideals of $B$ lying over the same prime ideal of $A$ are conjugate by an element of $\Aut(L/K)$.
    \end{enumerate}
\end{thm}

\begin{proof}
    We prove \textbf{4}, then \textbf{1}, \textbf{2}, and \textbf{3}, then \textbf{6}, and finally \textbf{5}.       
    \begin{enumerate}
        \item Let $\mf{p} \in \Spec A$. Then $B_{\mf{p}} = B \otimes_A A_{\mf{p}}$ is integral over $A_{\mf{p}}$. Applying \textbf{4}, we obtain the desired result.
        \item Consider $B_{\mf{p}}$ again. By \textbf{4}, because ideals lying over $\mf{p}$ are maximal, there cannot be inclusion relations between them.
        \item Let $\mf{p} \subset \mf{p}'$ and $\mf{P} \in \Spec B$ lying over $\mf{p}$. Then $B / \mf{P}$ is integral over $A / \mf{p}$. By \textbf{1}, we know that $\Spec B/\mf{P} \to \Spec A / \mf{p}$ is surjective. Thus there exists $\ol{\mf{P}}' \in \Spec B / \mf{P}$ lying over $\ol{\mf{p}}' = \mf{p}' / \mf{p}$. Then we know that $\ol{\mf{P}}' = \mf{P}' / \mf{P}$ for some prime ideal $\mf{P}'$ of $B$, and this is the ideal we are looking for. 
        \item This is a consequence of Lemma \ref{lem:goingdown}.
        \item Write $L = \operatorname{Frac} B \supset K = \operatorname{Frac} A$. Then let $L_1$ be the normal closure of $L / K$. Then let $\mf{p} \subset \mf{p}'$ in $A$ and $\mf{P}'$ in $B$ lie over $\mf{p}'$. Then let $\mf{P}_1 \subset \mf{P}'_1$ in $B_1$ the integral closure of $A$ in $L_1$. These exist thanks to \textbf{1} and \textbf{3}.  

            Let $\mf{P}_1''$ in $B_1$ such that $\mf{P}_1'' \cap B = \mf{P}'$. Then there exists $\sigma$ such that $\mf{P}_1'' = \sigma(\mf{P}_1')$ because both ideals are above $\mf{p}'$. Then we can choose 
            \[ \mf{P} \coloneqq \sigma(\mf{P}_1) \cap B \subset \mf{P}_1'' \cap B = \mf{P}'. \]
            We need to show that $\mf{P} \cap A = \mf{p}$. But this is simply
            \begin{align*}
                \mf{P} \cap A &= \sigma(\mf{P}_1) \cap A \\
                              &= \sigma(\mf{P}_1 \cap A) \\
                              &= \sigma(\mf{p}) \\
                              &= \mf{p}.
            \end{align*}
        \item We know that $A$ is integrally closed in $K$. Then let $L/K$ be a finite Galois (we can always reduce to this case) extension and $B$ the integral closure of $A$ in $L$. Then let $\mf{P}, \mf{P}' \in \Spec B$ lie above $\mf{p} \in \Spec A$. We will show there exists $\sigma \in \Gal(L/K)$ such that $\sigma(\mf{P}) = \mf{P}'$.

            Suppose that no such $\sigma$ exists. Then for all $\sigma \in \Gal(L/K)$, $\mf{P}' \neq \sigma(\mf{P})$. In particular, $\mf{P}' \not\subset \sigma(\mf{P})$. Then there exists $x \in \mf{P}'$ which is not in any $\sigma(\mf{P})$ then we see that
            \[ y \coloneqq \prod_{\sigma \in \Gal(L/K)} \sigma(x) \in K \]
            is integral over $A$, so $y \in A$. Also, $y \notin \mf{P}$ because $x \notin \sigma(\mf{P})$, so $x \in \mf{P}'$ and thus $y \in \mf{P}'$, so $y \in \mf{p} \subset \mf{P}$. This gives a contradiction. \qedhere
    \end{enumerate}
\end{proof}

\begin{cor}
    Assume that $B$ is integral over $A$. 
    \begin{enumerate}
        \item If $\mf{P}_0 \subsetneq \mf{P}_1 \subsetneq \cdots \subsetneq \mf{P}_r$ is a chain of prime ideals of $B$, the the $\mf{p}_i \coloneqq \mf{P}_i \cap A$ for a chain of prime ideals of $A$.
        \item If $p_0 \subsetneq \cdots \subsetneq \mf{p}_r$ is a chain of prime ideals of $A$, then there exists a chain $\mf{P}_0 \subsetneq \cdots \subsetneq \mf{P}_r$ of prime ideals of $B$ above it.
        \item If $A$ is integrally closed and $B$ is a domain, then for any chain $\mf{p}_0 \subsetneq \cdots \subsetneq \mf{p}_r$ and $\mf{P}_r \in \Spec B$ above $\mf{p}_r$, there exists a chain $\mf{P}_0 \subsetneq \cdots \subsetneq \mf{P}_r$ above the chain in $A$.
    \end{enumerate}
\end{cor}

\begin{proof}
    The proof is clear and left to the reader.
\end{proof}

\begin{defn}
    Let $\mf{p} \in \Spec A$. Then define the \textit{height}  of $\mf{p}$ by
    \[ \mr{ht}(\mf{p}) = \max \qty{n \geq 0 \mid \exists \mf{p}_0 \subsetneq \cdots \subsetneq \mf{p}_n = \mf{p}}. \]
    Then define the \textit{dimension} of $A$ by
    \[ \dim A = \max \qty{ \mr{ht}(\mf{p}) \mid p \in \Spec A }. \]
\end{defn}

\begin{cor}
    Let $A \subset B$ be an integral extension. Then
    \begin{enumerate}
        \item Suppose $\mf{P} \in \spec B$ lies above $\mf{p} \in \Spec A$. Then $\mr{ht}(\mf{P}) \leq \mr{ht}(\mf{p})$.
        \item $\dim A = \dim B$.
        \item If $A$ is integrally closed and $B$ is a domain, then we have $\mr{ht}(\mf{P}) = \mr{ht}(\mf{p})$. 
    \end{enumerate}
\end{cor}

\begin{proof}
    This is an immediate consequence of the previous corollary.
\end{proof}

\section{Associated Primes}%
\label{sec:associated_primes}

Let $M$ be an $A$-module and $\mf{p} \in \Spec A$.

\begin{defn}
    We say that $\mf{p}$ is an \textit{associated prime} of $M$ if one of the two following equivalent conditions hold.
    \begin{enumerate}
        \item There exists $x \in M$ such that $\mr{Ann}_A(x) = \mf{p}$;
        \item There is an injection $A / \mf{p} \hookrightarrow M$.
    \end{enumerate}
\end{defn}

We will denote the set of associated primes using the unfortunate notation $\mr{Ass}_A(M)$. Then the set of primes $\mf{p}$ such that $M_{\mf{p}} \neq 0$ will be denoted $\mr{Supp}_A(M)$.

\begin{prop}
    Let $\mf{p}$ be a maximal element of $\{ \Ann(x) \mid x \in M, x \neq 0 \}$. Then $\mf{p} \in \Ass_A(M)$.
\end{prop}

\begin{proof}
    We will show that such a maximal element is actually a prime ideal. Suppose $ab \in \mf{p}$. Then $\mf{p} = \Ann(x)$ for some nonzero $x$, so $b.x \neq 0$. Then $\Ann(x) \subset \Ann(bx) \neq A$. By maximality, $\Ann(x) = \Ann(bx)$. Because $abx = 0$, then $a \in \Ann(bx) = \mf{p}$.
\end{proof}

\begin{cor}
    Let $A$ be Noetherian.
    \begin{enumerate}
        \item $M$ is nonzero if and only if $\Ass_A(M)$ is nonempty.
        \item The set of zero divisors for $M$ is the union of the associated primes of $M$.
    \end{enumerate}
\end{cor}

\begin{proof}
    \begin{enumerate}
        \item If there is some associated prime, then clearly $M \neq 0$. In the other direction, the set of annihilators has a maximal element because $A$ is Noetherian, so there must be an associated prime.
        \item Let $a \in \Ann(x)$ for some nonzero $x \in M$. Then $\Ann(x) \subset \mf{p}$ is contained in some associated prime (because it is contained in some maximal element), and thus every zero divisor is contained in an associated prime. The other direction is obvious. \qedhere
    \end{enumerate}
\end{proof}

\begin{lem}
    Let $S \subset A$ be a multiplicative set and $M$ an $A$-module. Then
    \[ \Ass_A(S^{-1})M = \varphi^*(\Ass_{S^{-1}A}(S^{-1}M)). \]
\end{lem}

\begin{proof}
    Let $\mf{p} \in \Ass_A(S^{-1}M)$. Then $\mf{p} = \Ann_A \frac{x}{1}$ for some $x \in M$, so $\mf{p} \cap S$ must be empty. Next, we see that the set $\qty{\Ann_A(sx) \mid s \in S}$ contains some maximal element $\mf{m}$ because $A$ is Noetherian. But then $\mf{m} = \Ann_A(s_0 \cdot x) = \mf{p}$.

    On the other hand, if $a \in \mf{p}$, then $\frac{ax}{1} = 0$, which means $asx = 0$ for some $s \in S$. Then $a \in \Ann_A(sx) \subset \Ann(s_0 sx) = \Ann(s_0 x)$. Thus $\mf{p} \subset \Ann(s_0 x)$. Thus we have shown that
    \[ \Ass_A(S^{-1}M) \subset \varphi^* \Ass_{S^{-1}A} (S^{-1}M). \]
    The other inclusion is clear.
\end{proof}

\begin{thm}
    Let $A$ be Noetherian and $M$ and $A$-module. Then $\Ass_A(M) \subset \Supp_A(M)$ and any minimal element of $\Supp_A(M)$ is inside $\Ass_A(M)$.
\end{thm}

\begin{proof}
    Let $\mf{p} \in \Ass_A(M)$. Then $A/\mf{p}$ injects in $M$, so we have an injection $A_{\mf{p}} / \mf{p}A_{\mf{p}} \hookrightarrow M_{\mf{p}}$. Thus $\mf{p} \in \Supp_A(M)$.

    Now choose a minimal $\mf{p} \in \Supp_A(M)$. Thus $M_{\mf{p}}$ is nontrivial, so there exists a prime ideal $\mf{q} \subset \mf{p}$ such that $\mf{q} A_{\mf{p}} \in \Ass_{A_{\mf{p}}}(M_{\mf{p}})$. Thus $M_{\mf{q}} = (M_{\mf{p}})_{\mf{q}A_{\mf{p}}}$ is nonzero, so $\mf{q} \in \Supp(M)$. By minimality, $\mf{q} = \mf{p}$ and thus $\mf{p}A_{\mf{p}} \in \Ass_{A_{\mf{p}}} (M_{\mf{p}})$. Therefore $\mf{p} \in \Ass_A(M)$.
\end{proof}

\begin{defn}
    If $\mf{p} \in \Ass_A(M)$, then $\mf{p}$ is not necessarily minimal in the support of $M$. Then such a prime is called an \textit{embedded prime}. 
\end{defn}

\begin{prop}
    Let $A$ be Noetherian and $M$ a finitely-generated $A$-module. Then
    \begin{enumerate}
        \item There exists a chain
            \[ 0 = M_0 \subset M_1 \subset \cdots \subset M_n = M \]
            such that $M_i / M_{i-1} \cong A / \mf{p}_i$ for $i = 1, \ldots, n$ and $\mf{p}_i \in \Supp_A(M)$.
        \item Given such a sequence, we have $\Ass_A(M) \subset \{ \mf{p}_1, \ldots, \mf{p}_n \}$. In particular this set is finite.
    \end{enumerate}
\end{prop}

\begin{proof}
    \begin{enumerate}
        \item Suppose $M \neq 0$ Then choose $\mf{p}_1 \in \Ass_A(M)$ and let $M_1$ be the image of $A / \mf{p}_1$ in $M$. Then if $M / M_1$ is nonzero, choose $\mf{p}_1 \in \Ass_A(M / M_1)$ and $M_2$ defined analogously to $M_1$. This gives a sequence of submodules of $M$ such that $A / \mf{p}_i \cong M_i / M_{i-1}$. Because $M$ is Noetherian, this sequence becomes stationary. Thus there exists $n$ such that $M_n = M$.
        \item This is a consequence of the next lemma.
    \end{enumerate}
\end{proof}

\begin{rmk}
    In general the support of a module is \textbf{not} finite. 
\end{rmk}

\begin{lem}
    Assume we have an exact sequence of modules $0 \to M' \to M \to M''$. Then $\Ass(M) \subset \Ass(M') \cup \Ass(M'')$.
\end{lem}

\begin{proof}
    If $\mf{p} \in \Ass(M)$, there exists $N \subset M$ such that $N \cong A / \mf{p}$. Then if $N \cap M' = 0$, $N \hookrightarrow M''$ and $\mf{p} \in \Ass(M'')$. If the intersection is nonzero, then there exists some nonzero $x \in N \cap M'$ such that $\Ann_A(x) = p$ because $A/\mf{p}$ is a domain. Thus $\mf{p} \in \Ass(M')$.
\end{proof}

\begin{defn}
    We say that $M$ is \textit{coprimary} if $\Ass_A(M) = \qty{\mf{p}}$. 
\end{defn}

\begin{defn}
    Let $N \subset M$. Then we say that $N$ is \textit{$\mf{p}$-primary} if $\Ass_A(M/N) = \{ \mf{p} \}$. Alternatively, we say that $N$ \textit{belongs to} $\mf{p}$.  
\end{defn}

\begin{lem}
    A module $M$ is coprimary if and only if $M$ is nonzero and any zero divisor for $M$ is locally nilpotent (for all $x \in M$, there exists $n > 0$ such that $a^n . x = 0$).
\end{lem}

\begin{proof}
    Suppose that $M$ is coprimary. Now suppose that $a \in \mf{p}$ and $x \in M$. Then $\Ass(Ax) = \{ \mf{p} \}$, so $\mf{p}$ is minimal in the support of $A_x$, which is $V(\Ann(x))$. Therefore, $\mf{p} = \sqrt{\Ann(x)}$. Thus for $a \in \mf{p}$, $a^n \in \Ann(x)$.
    
    In the other direction, let $\mf{p}$ be the set of locally nilpotent elements with respect to $M$. This is clearly an ideal of $A$. Then let $\mf{q} \in \Ass_A(M)$. Then $x \in M$, so $\mf{q} = \Ann(x)$. Therefore $\mf{p} \subset \mf{q}$ because $\mf{q}$ is a prime ideal. However, $\mf{q}$ is contained in the set of zero divisors, which is precisely $\mf{p}$.
\end{proof}

\begin{rmk}
    Let $I \subset A$ be an ideal. Then $\Ass_A(A/I) = \{ \mf{p} \}$ if and only if the zero divisors of $A/I$ are locally nilpotent. This is equivalent to $I$ being primary.
\end{rmk}

\begin{lem}
    \begin{enumerate}
        \item Let $Q_1, Q_2 \subset M$ be $\mf{p}$-primary submodules. Then $Q_1 \cap Q_2$ is $\mf{p}$-primary.
        \item Let $N = Q_1 \cap \cdots \cap Q_r$ be an irredundant decomposition (i.e. $Q_i$ is $\mf{p}_i$-primary) for distinct $\mf{p}_i$. Then $\Ass_A(M/N) = \qty{\mf{p}_1, \ldots, \mf{p}_r}$. 
    \end{enumerate}
\end{lem}

\begin{proof}
    \begin{enumerate}
        \item Note that $M / Q_1 \cap Q_2$ injects in $M / Q_1 \oplus M / Q_2$. The desired result follows from the previous lemma.
        \item First, note that $M/N \hookrightarrow \bigoplus M / Q_i$. Then suppose $\Ass(M/N) = \{ \mf{p}_1, \ldots, \mf{p}_r \}$. Then we have an injection
            \[ \frac{Q_2 \cap \cdots \cap Q_r}{N} \hookrightarrow M/N \]
            and thus $\Ass((Q_1 \cap  \cdots \cap Q_r)/N)$ is contained in $\Ass(M/N)$. By the exact sequence
            \[ 0 \to N \to Q_2 \cap \cdots \cap Q_r \to M/Q_1, \]
            we see that $\Ass(Q_2, \cap \cdots \cap Q_r / N) = \{ \mf{p}_1 \}$. \qedhere
    \end{enumerate}
\end{proof}

\begin{thm}
    Let $M$ be a module over a Noetherian ring $A$. Then for all $\mf{p} \in \Ass(M)$, there exists a $\mf{p}$-primary submodule $Q(\mf{p}) \subset M$ such that
    \[ \bigcap_{\mf{p} \in \Ass(M)} Q(\mf{p}) = \{0 \}. \]
\end{thm}

\begin{proof}
    Fix $\mf{p} \in \Ass(M)$. Consider the set 
    \[ \mc{S}_{\mf{p}} = \qty{ N \subseteq M \mid \mf{o} \notin \Ass(N) }. \]
    This set is nonempty because $0 \in \mc{S}_{\mf{p}}$. Next, if $N_{\lambda} \in \mc{S}_{\mf{p}}$ is a chain, then the module $N = \bigcup N_{\lambda}$ is a submodule of $M$. In addition, $\Ass(N) \subset \bigcup \Ass(N_{\lambda})$. This implies that $\mc{S}_{\mf{p}}$ contains a maximal element by Zorn's lemma. Choose such a maximal element $Q(\mf{p})$.

    We will show that $M/Q(\mf{p})$ is coprimary. By the exact sequence 
    \[ 0 \to Q(\mf{p}) \to M \to M/\mf{Q}(\mf{p}), \]
    if $\mf{p}' \in \Ass(M/Q(\mf{p}))$, then $\mf{p}' = \mf{p}$ because otherwise $A/ \mf{p}'$ would inject in $M / Q(\mf{p})$ as $Q'/Q(\mf{p})$. Then $\Ass(Q') \subset \Ass(Q(\mf{p})) \cup \Ass(Q' / Q(\mf{p}))$, so $Q' \supsetneq Q(\mf{p})$, contradicting minimality. Thus $\Ass(M / Q(\mf{p})) = \qty{\mf{p}}$. 

    The second part of the claim follows immediately from the fact that $\Ass\qty(\bigcap Q(\mf{p})) = \bigcap \Ass(Q(\mf{p})) = \emptyset$.
\end{proof}

\begin{cor}
    Let $M$ be an $A$-module of finite type. Then any $N \subset M$ has a primary decomposition
    \[ N = Q_1 \cap \cdots \cap Q_r \]
    such that
    \begin{enumerate}
        \item The $Q_i$ are $\mf{p}_i$-primary;
        \item No $Q_i$ can be omitted;
        \item This decomposition is irredundant: $\Ass(M/N) = \qty{\mf{p}_1, \ldots, \mf{p}_r}$.
    \end{enumerate}
\end{cor}

\begin{proof}
    Apply the previous theorem to $M/N$. Because $M/N$ is of finite type, $\Ass(M/N)$ is finite. Then use the previous lemma.
\end{proof}

\begin{exer}
    Let $A \xrightarrow{\varphi} B$ be a morphism of rings and let $M$ be a $B$-module. Then prove that
    \[ \varphi^*(\Ass_B(M)) = \Ass_A(M) \]
    where $\varphi^* \colon \Spec B \to \Spec A$ is the induced map of spaces.
\end{exer}

\chapter{Dimension Theory}%
\label{cha:dimension_theory}

\section{Graded Rings and Modules}%
\label{sec:graded_rings_and_modules}

Let $A = \bigoplus_{n \geq 0} A_n$ be a graded ring. This means that $A_n \cdot A_m \subset A_{n+m}$. Then an $A$-module $M$ is a graded module if
\[ M = \bigoplus_{n \in \Z} M_n \]
such that $A_n \cdot M_m \subset M_{n+m}$. We will call $M_m$ the homogeneous elements of degree $m$ on $M$.

Now let $N \subset M$ be a submodule. We say that $N$ is a graded submodule if $N = \bigoplus N \cap M_m$. $N$ is also called homogeneous. A homogeneous element of $M$ is an element of some $M_m$. Being a graded submodule is the same as every element being a sum of homogeneous elements.

\begin{lem}
    The following are equivalent:
    \begin{enumerate}
        \item $N$ is a homogeneous submodule.
        \item $N$ is generated by homogeneous elements.
        \item If $x = x_r + \cdots + x_n \in N$ for $x_i \in M_i$, then for all $i$, $x_i \in N$.
    \end{enumerate}
    Moreover, if $N \subset M$ is homogeneous, then so is $M/N$, and
    \[ M/N = \bigoplus_m M_m / N_m. \]
\end{lem}

\begin{proof}
    The proof is left as an exercise to the reader.
\end{proof}

\begin{exm}
    Let $k$ be any ring. Then the ring $A = k[x_1, \ldots, x_r]$ is a graded ring where the grading is by the degree of each monomial. In particular, $A_0 = k$. Then an ideal $I \subset A$ is graded if $I = \bigoplus_n I_n$ where $I_n = I \cap A_n$. In addition, $A/I$ is a graded ring.
\end{exm}

\begin{prop}
    Let $A$ be a Noetherian graded ring and $M$ a graded $A$-module. Then
    \begin{enumerate}
        \item If $\mf{p} \in \Ass(M)$, then $\mf{p}$ is a graded ideal of $A$ and there exists a homogeneous $x \in M$ such that $\mf{p} = \Ann(x)$.
        \item One can choose a $\mf{p}$-primary graded submodule $Q(\mf{p})$ such that $0 = \bigcap_{\mf{p} \in \Ass(M)} Q(\mf{p})$.
    \end{enumerate}
\end{prop}

\begin{proof}
    Let $x \in M$ and set $\mf{p} = \Ann(M)$. Then write $x = x_e + x_{e-1} + \cdots + x_0$. Then for $f \in \mf{p}$, write $f = f_r + \cdots + f_s$. If $fx = 0$, then we can write
    \[ 0 = fx + f_r x_e + (f_{r-1} x_e + f_r x_{e-1}) + \cdots \]
    and deduce that $0 = f_r x_e = f_r^2 x_{e-1} = \cdots$. Then $f_r^e \in \mf{p}$, so $f_r \in \mf{p}$. By induction, all $f_i \in \mf{o}$, so $\mf{p}$ is graded.

    The proof of the second part is simply the following lemma.
\end{proof}

\begin{lem}
    Let $\mf{p}$ be a graded prime ideal and $Q \subset M$ such that $Q$ is $\mf{p}$-primary. Let $Q' \subset Q$ be the submodule of $Q$ generated by the homogeneous elements of $Q$. Then $Q'$ is $\mf{p}$-primary.
\end{lem}

\begin{proof}
    This will be proved later.
\end{proof}

We will now discuss filtrations of rings. A \textit{filtration} is a sequence of subgroups
\[ A = J_0 \supset J_1 \supset J_2 \cdots \]
such that $J_n \cdot J_m = J_{n+m}$. If we set
\[ A' = \bigoplus_{n=0}^{\infty} J_n / J_{n+1}, \]
then $A'$ is a graded ring.

The basic example is $J_m = I^m$ for some fixed ideal $I \subset A$. in this case, the filtration is called the $I$-adic filtration.

\begin{lem}
    Let $A$ be a Noetherian ring and set $I \subset A$. Then 
    \[ \operatorname{gr}^I A = \bigoplus_{n=0}^{\infty} I^n / I^{n+1} \]
    is a Noetherian graded ring.
\end{lem}

\begin{proof}
    Because $I$ is finitely-generated, then $I / I^2$ is a finitely-generated $A/I$-module. Thus $\operatorname{gr}^I(A)$ is a finitely-generated $A/I$-algebra. If $x_1, \ldots, x_r$ is a set of generators of $I$, then
    \[ A/I[x_1, \ldots, x_r] \to \operatorname{gr}^I A \]
    is surjective, so because $A/I$ is Noetherian, so is $A/I[x_1, \ldots, x_r]$ and thus so is $\operatorname{gr}^IA$.
\end{proof}

Let $A$ be an Artinian ring and $B = A[x_1, \ldots, x_r]$ be a graded ring. Then let $M$ be a finitely-generated graded $B$-module. Each graded piece $M_n$ is an $A$-module, so write $F_M(n) = \ell_A(M_n)$. Because $M$ is finitely generated, we have a map
\[ \bigoplus_{i=1}^r B(d_i) \twoheadrightarrow M. \]
Here, $B(d_i) = B$ as a $B$-module with the gradation $B(d)_n = B_{n-d}$. Thus $M$ is generated by homogeneous elements $x_{d_i}$ of degree $d_i$. This gives us the map
\begin{align*}
\bigoplus_{i=1}^r B_{m-d_i} \hookrightarrow M_m \\
(f_i) \mapsto \sum_{i=1}^r f_i x_{d_i} 
\end{align*}
and thus $\ell_A(M_m) \leq \sum_{i=1}^r \ell_A(B_{m-d_i})$. But then $B_m$ is a free $A$-module, and thus 
\[ \ell_A(B_m) \leq \binom{r+m-1}{m-1} \ell(A). \]

\begin{thm}
    Let $A, B,M$ be as above. Then there is a polynomial $f_M(x) \in \Q[x]$ such that
    \[ \ell_A(M_n) = f_M(n) \]
    for $n \gg 0$. This is called the \textit{Hilbert-Samuel polynomial} for $M$. The degree of this polynomial will give the first definition for the dimension of $M$. 
\end{thm}

\begin{proof}
    Say that $M$ satisfies the property $P(M)$ if there exists $f \in \Q(x)$ such that $\ell(M_n) = f(n)$ for $n \gg 0$. 
    \begin{enumerate}
        \item First, we will show that if $N_1, N_2 \subset M$ and $P(M/N_1), P(M/N)2$ hold, then $P(M/N_1 \cap N_2)$ holds.
        \item Second, if $N$ is irreducible, then $P(M/N)$ holds.
    \end{enumerate}
    If we prove these two things, then we simply use a primary decomposition of $0 \subset M$. Then the second statement implies that $P(M/N_i)$ holds and then we simply repeatedly apply the first to see that $P(M)$ holds.
    \begin{enumerate}
        \item Suppose $N = N_1 \cap N_2$ with $N, N_1, N_2$ graded. We then have an exact sequence
            \[ 0 \to N_1 / (N_1 \cap N_2) \to M / (N_1 \cap N_2) \to M/N_1 \to 0 \]
            and we know that $N_1 / (N_1 \cap N_2) = (N_1 + N_2) / N_1$ is graded. Then we see that $F_{M/(N_1 \cap N_2)} = F_{M/N} + F_{(N_1 + N_2)/N_1}$, so we only need to prove that $F_{N_1 + N_2 / N_1}$ exists. But then $F_{N_1 + N_2 / N_1} = F_{M / N_2} - F_{M/N_1 + N_2}$ and because $P(M/N_1 + N_2)$ holds, so does $P(M/N_1 \cap N_2)$.
        \item Let $N$ be irreducible. We know that $M' = M/N$ is coprimary, so $N$ is $\mf{p}$-primary for some prime ideal $\mf{p} \subset B$. Write $I = (x_1, \ldots, x_m)$. If $I \subset \mf{p}$, then $M'_m = 0$ for $n \gg 0$. Indeed, if $d$ is the maximal degree of a system of generators of $M'$, then $M'_{n+d} = I^n \cdot M'_d$. On the other hand, because $M'$ is $\mf{p}$-primary, then elements in $\mf{p}$ are locally nilpotent. Thus there exists $k \gg 0$ such that $p^k \cdot M'_d = 0$ and thus $M'_{n+d} = 0$ for $n \geq k$. Thus $F_{M/N}$ exists and is identically zero.

            In the second case, $I \not\subset \mf{p}$. Then suppose that $x_1 \notin \mf{p}$. Thus $x_1$ is not a zero divisor for $M'$. Thus, we have an exact sequence
            \[ 0 \to M' \to x_1 \to M' \to M'/x_1M' \to 0 \]
            which then gives
            \[ 0 \to (M/N)_{n-1} \to (M/N)_n \to (M/N+x_1 M)_n \to 0 \]
            when restricting to a single graded piece. Thus $N \subsetneq N + x_1 M$. This implies that $f_{M/N + x_1 M}$ exists because above, we proved that if $P(M/N')$ holds for any $N' \supsetneq N$ implies that $P(M/N)$ holds. Then for $n \geq n_0$, we have $\ell((M/N)_m) - \ell((M/N)_{n-1}) = f_{M/N + x_1 M}(n)$. This implies that
            \[ \ell((M/N)_m) = f(n) + f(n-1) + \cdots + \ell(M/N)_{n_0}. \]
            Then $f(n) + \cdots + f(n_0) = g(n)$ for some polynomial $g$ of degree $\deg f + 1$ and then $f_{M/N} = g + \ell((M/N)_{n_0})$.\qedhere
    \end{enumerate}
\end{proof}

Now let $A \supset I$ and $M$ be an $A$-module with filtration
\[ M_0 = M \supset M_1 \supset \cdots \supset M_n \supset \cdots \]
We say that the filtration is
\begin{enumerate}
    \item \textit{$I$-admissible} if $I M_n \subset M_{n+1}$ for all $n \gg 0$;
    \item \textit{$I$-acic} if $I M_n = M_{n+1}$ for all $n \geq 0$;
    \item \textit{essentially $I$-adic} if $I M_n = M_{n+1}$ for $n \gg 0$. 
\end{enumerate}

\begin{rmk}
    A filtration on $M$ defines a topology on $M$ so that $M$ is a topological group. Here, a system of neighborhoods of $0$ is $(M_n)_{n \gg 0}$. If $\bigcap M_n = 0$, then the topology is Hausdorff. If the filtration is essentially $I$-adic, then the topology is called the $I$-adic topology.
\end{rmk}

\begin{lem}
    Let $A$ and $I$ be as before. Let $M$ be an $A$-module with an admissible filtration. Let $A' = \bigoplus_{n=1}^{\infty} I^n x^n \subset A[x]$ and 
    \[ M' = \bigoplus M_n \otimes_A Ax^n = \bigoplus M_n x^n. \]
    \begin{enumerate}
        \item $M'$ is a $A'$-module.
        \item The filtration is essentially $I$-adic if and only if $M'$ is a finitely-generate $A'$-module.
    \end{enumerate}
\end{lem}

\begin{proof}
    \begin{enumerate}
        \item This is trivial.
        \item Note that $M'$ is a graded $A'$-module. If $M'$ is finitely generated, then write $M' = A' m_1 + \cdots + A' m_r$. Then we see that
            $ M'_n = I x M'_{n-1}$ 
            for $n > \max \qty{ \deg m_i }$. Thus $M_n$ is essentially $I$-adic.

            Conversely, if $M_n = I^{n - n_0} M_{n_0}$ for $n \geq n_0$, then, then it is clear that $M'$ is generated by $M_{n_0} x^{n_0} + \cdots + M_1 x + M_0$ and is thus finitely generated. \qedhere
    \end{enumerate}
\end{proof}

\begin{thm}[Artin-Rees]
    Let $A$ be a Noetherian ring and $I \subset A$. Then let $M$ be a finitely-generated $A$-module and $N \subset M$ be a submodule. Then there exists $r > 0$ such that
    \[ I^n M \cap N = I^{n-r} (I^r M \cap N) \]
    for all $n \geq r$.
\end{thm}

\begin{proof}
    Let $M_n = I^n M$ be the $I$-adic filtration. Then $N_n = I^n M \cap N$ is $I$-admissible. Then both $N' \subset M'$ are both $A'$-modules. We know that $A'$ is Noetherian, so because $M_n$, $M'$ is finitely generated. Thus $N'$ is also Noetherian, so it is also finitely-generated. This implies that $N_m$ is essentially $I$-adic, as desired.
\end{proof}

\begin{rmk}
    This theorem is saying that the filtration $I^n M \cap N$ is essentially $I$-adic.
\end{rmk}

\begin{cor}[Krull Intersection Theorem]
    Let $A, I, M$ as above.
    \begin{enumerate}
        \item If $N = \bigcap_{n=0}^{\infty } I^n M$, then $IN = N$.
        \item If $I \subset \mr{rad}(A)$, then $\bigcap_{n=0}^{\infty} I^n M = 0$.
    \end{enumerate}
\end{cor}

\begin{proof}
    \begin{enumerate}
        \item Note that $N \subset M$. Then apply the Artin-Rees theorem to $N = I^n M \cap N$.
        \item Apply Nakayama's lemma. \qedhere
    \end{enumerate}
\end{proof}

\begin{cor}
    Let $A$ be a Noetherian domain and let $I \subset A$ be a proper ideal. Then $\bigcap_n I^n = 0$.
\end{cor}

\begin{proof}
    Let $N = \bigcap_{n=0}^{\infty} I^n$. By the previous corollary, $IN = N$. Then $N$ is finitely generated because $A$ is Noetherian. Thus there exists $x \in I$ such that $(1+x)N = 0$, which implies that $N = 0$ because $A$ is a domain.
\end{proof}

\begin{exer}
    Let $A$ be Noetherian and $M$ a finitely-generated $A$-module. Then let $I, J$ be generated by $M$-regular elements. Then there exists $r > 0$ such that $(I^n M : J) = I^{n-r} (I^r M : J)$. Here, $(N:J) = \qty{ m \in M \mid Jm \subset N }$.
\end{exer}




\end{document}
