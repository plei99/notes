\documentclass[leqno, openany]{memoir}
\setulmarginsandblock{3.5cm}{3.5cm}{*}
\setlrmarginsandblock{3cm}{3.5cm}{*}
\checkandfixthelayout

\usepackage{amsmath}
\usepackage{amssymb}
\usepackage{amsthm}
%\usepackage{MnSymbol}
\usepackage{bm}
\usepackage{accents}
\usepackage{mathtools}
\usepackage{tikz}
\usetikzlibrary{calc}
\usetikzlibrary{automata,positioning}
\usepackage{tikz-cd}
\usepackage{forest}
\usepackage{braket} 
\usepackage{listings}
\usepackage{mdframed}
\usepackage{verbatim}
\usepackage{physics}
\usepackage{stackengine} 

%font
\usepackage[sc]{mathpazo}
\usepackage{eulervm}
\usepackage[scaled=0.86]{berasans}
\usepackage{inconsolata}
\usepackage{microtype}

%CS packages
\usepackage{algorithmicx}
\usepackage{algpseudocode}
\usepackage{algorithm}

% typeset and bib
\usepackage[english]{babel} 
\usepackage[utf8]{inputenc} 
\usepackage[T1]{fontenc}
\usepackage[backend=biber, style=alphabetic]{biblatex}
\usepackage[bookmarks, colorlinks, breaklinks]{hyperref} 
\hypersetup{linkcolor=black,citecolor=black,filecolor=black,urlcolor=black}

% other formatting packages
\usepackage{float}
\usepackage{booktabs}
\usepackage{enumitem}
\usepackage{csquotes}
\usepackage{titlesec}
\usepackage{titling}
\usepackage{fancyhdr}
\usepackage{lastpage}
\usepackage{parskip}

\usepackage{lipsum}

% delimiters
\DeclarePairedDelimiter{\gen}{\langle}{\rangle}
\DeclarePairedDelimiter{\floor}{\lfloor}{\rfloor}
\DeclarePairedDelimiter{\ceil}{\lceil}{\rceil}


\newtheorem{thm}{Theorem}[section]
\newtheorem{cor}[thm]{Corollary}
\newtheorem{prop}[thm]{Proposition}
\newtheorem{lem}[thm]{Lemma}
\newtheorem{conj}[thm]{Conjecture}
\newtheorem{quest}[thm]{Question}

\theoremstyle{definition}
\newtheorem{defn}[thm]{Definition}
\newtheorem{defns}[thm]{Definitions}
\newtheorem{con}[thm]{Construction}
\newtheorem{exm}[thm]{Example}
\newtheorem{exms}[thm]{Examples}
\newtheorem{notn}[thm]{Notation}
\newtheorem{notns}[thm]{Notations}
\newtheorem{addm}[thm]{Addendum}
\newtheorem{exer}[thm]{Exercise}

\theoremstyle{remark}
\newtheorem{rmk}[thm]{Remark}
\newtheorem{rmks}[thm]{Remarks}
\newtheorem{warn}[thm]{Warning}
\newtheorem{sch}[thm]{Scholium}


% unnumbered theorems
\theoremstyle{plain}
\newtheorem*{thm*}{Theorem}
\newtheorem*{prop*}{Proposition}
\newtheorem*{lem*}{Lemma}
\newtheorem*{cor*}{Corollary}
\newtheorem*{conj*}{Conjecture}

% unnumbered definitions
\theoremstyle{definition}
\newtheorem*{defn*}{Definition}
\newtheorem*{exer*}{Exercise}
\newtheorem*{defns*}{Definitions}
\newtheorem*{con*}{Construction}
\newtheorem*{exm*}{Example}
\newtheorem*{exms*}{Examples}
\newtheorem*{notn*}{Notation}
\newtheorem*{notns*}{Notations}
\newtheorem*{addm*}{Addendum}


\theoremstyle{remark}
\newtheorem*{rmk*}{Remark}

% shortcuts
\newcommand{\Ima}{\mathrm{Im}}
\newcommand{\A}{\mathbb{A}}
\newcommand{\N}{\mathbb{N}}
\newcommand{\R}{\mathbb{R}}
\newcommand{\C}{\mathbb{C}}
\newcommand{\Z}{\mathbb{Z}}
\newcommand{\Q}{\mathbb{Q}}
\renewcommand{\k}{\Bbbk}
\renewcommand{\P}{\mathbb{P}}
\newcommand{\M}{\overline{M}}
\newcommand{\g}{\mathfrak{g}}
\newcommand{\h}{\mathfrak{h}}
\newcommand{\n}{\mathfrak{n}}
\renewcommand{\b}{\mathfrak{b}}
\newcommand{\ep}{\varepsilon}
\newcommand*{\dt}[1]{%
   \accentset{\mbox{\Huge\bfseries .}}{#1}}
\renewcommand{\abstractname}{Official Description}
\newcommand{\mc}[1]{\mathcal{#1}}
\newcommand{\T}{\mathbb{T}}
\newcommand{\mf}[1]{\mathfrak{#1}}
\newcommand{\mr}[1]{\mathrm{#1}}
\newcommand{\ms}[1]{\mathsf{#1}}
\newcommand{\ol}[1]{\overline{#1}}
\newcommand{\ul}[1]{\underline{#1}}
\newcommand{\wt}[1]{\widetilde{#1}}
\newcommand{\wh}[1]{\widehat{#1}}

\DeclareMathOperator{\Der}{Der}
\DeclareMathOperator{\Hom}{Hom}
\DeclareMathOperator{\End}{End}
\DeclareMathOperator{\ad}{ad}
\DeclareMathOperator{\Aut}{Aut}
\DeclareMathOperator{\Gal}{Gal}
\DeclareMathOperator{\Rad}{Rad}
\DeclareMathOperator{\supp}{supp}
\DeclareMathOperator{\Supp}{Supp}
\DeclareMathOperator{\sgn}{sgn}
\DeclareMathOperator{\spec}{Spec}
\DeclareMathOperator{\Spec}{Spec}
\DeclareMathOperator{\Ext}{Ext}
\DeclareMathOperator{\Tor}{Tor}
\DeclareMathOperator{\Ann}{Ann}
\DeclareMathOperator{\Ass}{Ass}
\DeclareMathOperator{\dpth}{depth}
\DeclareMathOperator{\pdim}{proj.dim}
\DeclareMathOperator{\idim}{inj.dim}
\DeclareMathOperator{\gdim}{gl.dim}

% Section formatting
\titleformat{\section}
    {\Large\sffamily\scshape\bfseries}{\thesection}{1em}{}
\titleformat{\subsection}[runin]
    {\large\sffamily\bfseries}{\thesubsection}{1em}{}
\titleformat{\subsubsection}[runin]{\normalfont\itshape}{\thesubsubsection}{1em}{}

\title{COURSE TITLE}
\author{Lectures by INSTRUCTOR, Notes by NOTETAKER}
\date{SEMESTER}

\newcommand*{\titleSW}
    {\begingroup% Story of Writing
    \raggedleft
    \vspace*{\baselineskip}
    {\Huge\itshape Commutative Algebra \\ Fall 2020}\\[\baselineskip]
    {\large\itshape Notes by Patrick Lei}\\[0.2\textheight]
    {\Large Lectures by Eric Urban}\par
    \vfill
    {\Large \sffamily Columbia University}
    \vspace*{\baselineskip}
\endgroup}
\pagestyle{simple}

\chapterstyle{ell}


%\renewcommand{\cftchapterpagefont}{}
\renewcommand\cftchapterfont{\sffamily}
\renewcommand\cftsectionfont{\scshape}
\renewcommand*{\cftchapterleader}{}
\renewcommand*{\cftsectionleader}{}
\renewcommand*{\cftsubsectionleader}{}
\renewcommand*{\cftchapterformatpnum}[1]{~\textbullet~#1}
\renewcommand*{\cftsectionformatpnum}[1]{~\textbullet~#1}
\renewcommand*{\cftsubsectionformatpnum}[1]{~\textbullet~#1}
\renewcommand{\cftchapterafterpnum}{\cftparfillskip}
\renewcommand{\cftsectionafterpnum}{\cftparfillskip}
\renewcommand{\cftsubsectionafterpnum}{\cftparfillskip}
\setrmarg{3.55em plus 1fil}
\setsecnumdepth{subsection}
\maxsecnumdepth{subsection}
\settocdepth{subsection}

\begin{document}
    
\begin{titlingpage}
\titleSW
\end{titlingpage}

\thispagestyle{empty}
\section*{Disclaimer}%
\label{sec:disclaimer}

These notes were taken during lecture using the \texttt{vimtex} package of the
editor \texttt{neovim}.  Any errors are mine and not the instructor's.  In
addition, my notes are picture-free (but will include commutative diagrams) and
are a mix of my mathematical style and that of the instructor.  If you find any
errors, please contact me at \texttt{plei@math.columbia.edu}.

Any similarties to Matsumura's \textit{Commutative Algebra} are due to the
instructor.

\section*{Acknowledgements}% \label{sec:acknowledgements}

I would like to thank Caleb Ji, Nicol\'as Vilches, and Alex Xu for pointing out
typos in these notes.

\newpage



\tableofcontents

\chapter{Basic Notions}% \label{cha:basic_notions}

The references we will use in this course are Matsumura's \textit{Commutative
Algebra} and Serre's \textit{Alg\`ebre Locale, Multiplicit\'es}. There is an
English translation of Serre.  We will begin with general results on rings and
modules. We will assume all rings are commutative and unital. Recall that and
ideal $I$ of a ring $A$ is prime if and only if $A/I$ is a domain, and $I$ is
maximal if and only if $A/I$ is a field.

\section{Basics of Ideals}% \label{sec:basics_of_ideals}


\begin{defn} Let $I \subset A$ be an ideal. Then the \textit{radical}
    $\sqrt{I}$ of $I$ is the set \[ \sqrt{I} \coloneqq \qty{x \in A \mid x^a
    \in I \text{ for some } a \in \N}. \] \end{defn}

\begin{defn} An ideal $I \subset A$ is \textit{primary} if $I \neq A$ and the
zero divisors in $A/I$ are nilpotent. Thus if $xy \in I$ and $x \notin I$, then
$y^n \in I$ for some $n$.  \end{defn}

\begin{prop} \label{prop:radical_of_primary} If $Q \subset A$ is primary, then
$\sqrt{Q}$ is a prime ideal.  \end{prop}

\begin{proof} If $xy \in \sqrt{Q}$, then $x^ny^n \in Q$. If $x^n \notin Q$,
then $y \in \sqrt{q}$ because $(y^n)^a \in Q$.  \end{proof}

\begin{rmk} The converse to Proposition \ref{prop:radical_of_primary} is false
in general.  \end{rmk}

\begin{defn} Let $A$ be a ring. Then the \textit{spectrum} $\spec A$ of $A$ as
    a set is the set of prime ideals of $A$. We may place the Zariski topology
    on this set, where the basis of open sets is given by $D_f = \spec A
    \setminus V_f$, where $V_f$ is the set of prime ideals containing $f$.
\end{defn}

If $\varphi:A \to B$ is a morphism of rings, the morphism $\varphi^*: \spec B
\to \spec A$ is continuous in the Zariski topology. 

\begin{exer} In particular, if $\pi: A \to A/I$, then $\pi^*$ is an embedding.
\end{exer}

\begin{exer} Let $I \subset A$ be an ideal. Then let $P_1, \ldots, P_r$ be
ideals of $A$ that are all prime except possibly two of them. Show that if $I
\not\subset P_i$ for all $i$, then $I \not\subset \bigcup_i P_i$.  \end{exer}

\begin{exer} Let $ \mathfrak{a}_1, \ldots, \mathfrak{a}_r$ be ideals of $A$
    such that $ \mathfrak{a}_i + \mathfrak{a}_j = A $. Then \begin{enumerate}
        \item $\bigcap_i \mathfrak{a}_i = \mathfrak{a}_1 \cdots \mathfrak{a}_r$
        \item There is an isomorphism of rings $A/ \bigcap_i \mf{a}_i \cong
            \prod_i A/\mf{a}_i$.  \end{enumerate} \end{exer}

\section{Localization}% \label{sec:localization}

Let $S \subset A$ be a multiplicative subset. The main examples are $S_f =
\qty{1,f,f^2, \ldots}$ and $S_{\mf{p}} = A \setminus \mf{p}$ for a prime ideal
$\mf{p}$. Then if $0 \notin S$, there is at least one ideal $\mf{a}$ such that
$\mf{a} \cap S = \emptyset$. Denote the set of such $\mf{a}$ by $\mc{M}_S$.
Then any maximal element of $\mc{M}_S$ is a prime ideal in $A$. Existence of a
maximal element is seen using Zorn's lemma.

To see that maximal elements of $\mc{M}_S$ are prime ideals, note that $(x) +
P$ is not in $\mc{M}_S$, so if $x,y \notin P$, there exist $a,b \in A$ and
$s,s' \in S$ such that $ax \equiv s \mod P$ and $by \equiv s' \mod P$.
Therefore $abxy \notin P$, so $xy$ is not in $P$

\begin{lem} Let $\operatorname{nil} A$ be the set of all nilpotent elements.
    Then \[ \operatorname{nil} A = \bigcap_{\substack{P \subset A \\ P \text{
    prime}}} P. \] \end{lem}

\begin{proof} One direction is easy, so let $x$ be contained in all prime
    ideals. Then consider the set $S_x$. If $0 \notin S_x$, then $\mc{M}_{S_x}$
    is nonempty, so it has a maximal element. This is a prime ideal, which
    implies $x$ is not contained in some prime.  \end{proof}

\begin{cor} Let $Q$ be an ideal of $A$. Then $\sqrt{Q}$ is the intersection of
all prime ideals containing $Q$.  \end{cor}

Now fix a multiplicative subset $S$. Then we will define an equivalence
relation on $A \times S$. We write \[ (a,s) \sim (b,r) \] if there exists $t
\in S$ such that $t(ar - bs) = 0$. If $A$ is a domain, then this says that
$\frac{a}{s} = \frac{b}{r}$. Now we will define the \textit{localization}
$S^{-1}A$ to be the set of equivalence classes for this relation. Note there is
a natural morphism $A \to S^{-1}A$ that sends $a \mapsto \frac{a}{1}$.

Note that the localization has a universal property: If $\varphi:A \to B$ is a
morphism such that $\varphi(S) \subset B^{\times}$, then $\varphi$ factors
uniquely through $S^{-1}A$.

Localization gives a map $\spec S^{-1}A \to \spec A$, and in particular, if $S
= \qty{1, f, f^2, \ldots}$, we recover the set $D_f = \spec A_f$.

\section{Modules}% \label{sec:modules}

Let $A$ be a ring. Then an $A$-module $M$ is an abelian group with an action of
$A$. If $M$ is an $A$-module and $S \subset A$ is a multiplicative set, then
$S^{-1}M$ is the set of equivalence classes for $(m,s) \sim (m',s')$ if there
exists $t \in S$ such that $t(s'm - sm') = 0$. This is an $S^{-1}A$-module.

\begin{lem} Let $M$ be an $A$-module. Then the map \[ M \to \prod_{\substack{
\mf{p} \subset A \\ \mf{p} \text{ maximal}}} M_{\mf{p}} \] is injective.
\end{lem}

\begin{proof} Let $x \in M$ be nonzero. Then the annihilator of $x$ is a proper
ideal of $A$, so it is contained in a maximal ideal. This implies that $x_p \in
M_p$ is nonzero.  \end{proof}

\begin{cor} Let $A$ be a domain. Then $A = \bigcap_{\mf{p}} A_{\mf{p}}$, where
this intersection makes sense inside the fraction field of $A$.  \end{cor}

\begin{proof} Apply the previous lemma to $M = K/A$.  \end{proof}

\begin{defn} Let $M$ be an $A$-module and $x \in A$. Then $x$ is
\textit{$M$-regular} if the morphism $m \mapsto xm$ is injective. Additionally,
if $x$ is $A$-regular, then it is called \textit{regular}.  \end{defn}

The set $S_0$ of all regular elements in $A$ is multiplicative, and the ring
$S_0^{-1}A$ is called the \textit{total ring of fractions}. If $A$ is a domain,
then $S_0 = A \setminus \{0 \}$, and $S_0^{-1}A$ is the field of fractions.

\begin{defn} A ring $A$ is a \textit{local ring} if $A$ has only one maximal
ideal. In this case, all elements not in the maximal ideal are units.
\end{defn}

\begin{rmk} If $I \subset A$ is an ideal such that $A \setminus I =
A^{\times}$, then $A$ is a local ring and $I$ its maximal ideal.  \end{rmk}

\begin{exm} Now let $A$ be a general ring and $\mf{p} \in \spec A$. Then
$A_{\mf{p}}$ is a local ring with maximal ideal $\mf{p} A_{\mf{p}}$.  \end{exm}

\begin{defn} Now suppose $A,B$ are local rings. Then a morphism $\varphi:A \to
    B$ of rings is \textit{local} if $\varphi(\mf{m}_A) \subset \mf{m}_B$. This
    means we have a commutative diagram \begin{equation} \begin{tikzcd} A
        \arrow{r}{\varphi} \arrow{d} & B \arrow{d} \\ k_A \arrow{r} & k_B
    \end{tikzcd} \end{equation} where $k_A = A/ \mf{m}_A$ is the residue field
of $A$.  \end{defn}

Recall that the nilradical is the set of all nilpotent elements, or
equivalently the intersection of all prime ideals. Then the \textit{Jacobson
radical} $\operatorname{rad} A$ is defined to be the intersection of all
maximal ideals.

\begin{prop} Let $x \in A$. Then $x \in \operatorname{rad} A$ if and only if
$1+xa$ is a unit for any $a \in A$.  \end{prop}

\begin{proof} If $(1+x)A \neq A$, then $1+x$ is contained in some maximal ideal
    $mf{m}$, which implies $1 \in \mf{m}$. In the other direction, suppose
    there exists some maximal ideal $\mf{m}$ such that $x \notin \mathfrak{m}$.
    Then $x$ is nonzero in $A/ \mathfrak{M}$. Thus there exists $b$ such that
    $1-xb \in m$, which contradicts the assumption that $1+xa$ is a unit for
    any $a$.  \end{proof}

\begin{lem}[Nakayama's Lemma] Let $M$ be a finitely generated $A$-module. Then
    let $I$ be an ideal such that $IM = M$. Then there exists $x \in I$ such
    that $(1+x) M = 0$. In particular, if $I \subseteq \operatorname{rad} A$,
    then $M = 0$.  \end{lem}

\begin{proof} We will induct on the number of generators. If $M = A.m$, then $m
    = xm$ for some $x \in I$, and thus $(1-x)m = 0$. Now suppose $M = Am_1 +
    \cdots + Am_r$. Let $M' = M / Am_r$. By the inductive hypothesis, $(1+x)M'
    = 0$ for some $x \in I$.  Therefore $(1+x)M \subset A m_r$, so $(1+x) IM =
    (1+x)M \subset I m_r$. Therefore $(1+x) m_r = y m_r$ for some $y \in I$,
    and thus $(1+x - y) m_r = 0$. Thus $(1+x)(1+x-y) M \subset (1+x-y)A m_r =
    0$.  \end{proof}

\begin{cor} Let $N, N' \subset M$ and $I \subset A$ such that $M = N + I N'$.
Then if either \begin{enumerate} \item $I$ is nilpotent; \item $I \subset
\operatorname{rad} A$ and $N'$ is finitely generated, \end{enumerate} then $M =
N$.  \end{cor}

\begin{proof} \begin{enumerate} \item Suppose $I$ is nilpotent. Then
    \begin{align*} M &= N + IN' = N+IM \\ &= N + I (N + IM) \\ &= N + I^2 M \\
    &\ \vdots \\ &= N + I^n M \\ &= N \end{align*} because $I$ is nilpotent.
\item Let $I \subseteq \operatorname{rad} A$ and $N'$ be finitely generated.
    Then set $M_0 = M/N = I N_0'$, where $N_0'$ is the image of $N'$ inside
    $M_0$. Because $N_0'$ is finitely generated, so is $M_0$. Therefore $M_0 =
    I M_0 = 0$, so $M = N$.  \end{enumerate} \end{proof}

\begin{rmk} Most of the time, we apply this result when $A$ is local and $I$ is
the maximal ideal of $A$. In this case, $M / \mf{m} M$ is a finite-dimensional
vector space over $A / \mf{m}$.  \end{rmk}

\section{Artinian and Noetherian Rings}%
\label{sec:artinian_and_noetherian_rings}

\begin{defn} We say that an $A$-module $M$ satisfies the \textit{ascending
    chain condition} if any ascending chain of submodules of $M$ becomes
    stationary. Similarly, $M$ satisfies the \textit{descending chain
    condition} if any descending chain of submodules becomes stationary. If $M$
    satisfies the ascending chain condition, it is called \textit{Noetherian},
    and if $M$ satisfies the descending chain condition, it is
    \textit{Artinian}.  \end{defn}

\begin{prop} Assume we have a short exact sequence of $A$ modules
    \begin{equation} \begin{tikzcd} 0 \arrow{r} & N \arrow{r} & M \arrow{r} & P
    \arrow{r} & 0.  \end{tikzcd} \end{equation} Then $M$ is Noetherian (resp.
    Artinian) if and only if $N$ and $P$ are.  \end{prop}

\begin{proof} Proving that if $M$ is Noetherian, then $N$ and $P$ are is left
    to the reader. Now consider a chain \[ M_1 \subset M_2 \subset \cdots M_n
    \subset \cdots \] Then let $P_i$ be the image of the $M_i$ in $P$ and $N_i
    = N \cap M_i$. Then we have an exact sequence \[ 0 \to N_i \to M_i \to P_i
    \to 0. \] Because $(N_i)$ and $(P_i)$ stabilize, so must $M_i$ from the
exact sequence.  \end{proof}

\begin{cor} If $A$ is Noetherian (resp. Artinian), then any finitely generated
$A$-module is Noetherian (resp. Artinian).  \end{cor}

\begin{cor} Assume $A$ is Noetherian. Then any finitely generated $A$-module
    $M$ has a projective resolution by finite free $A$-modules. In other worse,
    there exists an exact sequence \[ \cdots \to F_n \to F_{n-1} \to \cdots \to
    F_1 \to F_0 \to M \to 0 \] such that each $F_i = A^{m_i}$.  \end{cor}

\begin{proof} Suppose $M$ is finitely generated. Then $M = A m_1 + \cdots + A
    m_r$, so we have a sequence \[ A^r \xrightarrow{\varphi_0} M \to 0. \] Then
    $\ker \varphi_0 = N_0$ and $F_0 = A^r$. Then we repeat this process with
    $N_0$ taking the role of $M$.  \end{proof}

\begin{prop} An $A$-module $M$ is noetherian if and only if any submodule of
$M$ is finitely generated.  \end{prop}

\begin{proof} Let $N \subseteq M$. Then choose $n_1 \in N$. Then if $A n_1 \neq
    N$, choose $n_2 \in N \setminus A n_1$. This process will stop because $M$
    is Noetherian, so $N$ is finitely generated.

    Now suppose any submodule is finitely generated. Given a chain \[ M_1
    \subset M_2 \subset \cdots \subset M_n \subset \cdots, \] set $N =
    \bigcup_i M_i$. This is finitely generated and is also equal to the first
    $M_i$ that contains all of the generators.  \end{proof}

This means that a ring $A$ is noetherian if and only if all ideals of $A$ are
finitely generated. In particular, fields and principal ideal domains are
Noetherian.

\begin{prop} Let $M$ be Noetherian and suppose $S$ is a multiplicative subset
of $A$. Then $S^{-1}M$ is Noetherian.  \end{prop}

\begin{proof} Consider the morphism $M \to S^{-1}M$. Then let $N_i$ be a chain
of $S^{-1}A$-modules in $S^{-1}M$. Their preimages $M_i$ form a chain, and they
are stationary, so $N_i$ is also stationary.  \end{proof}

\begin{thm} Let $A$ be a Noetherian ring. Then $A[X]$ is Noetherian.  \end{thm}

\begin{proof} Let $I \subset A[X]$. Then $\mf{A}_n \subset A$ be generated by
    the dominant coefficients of polynomials in $I$ of degree at most $n$. Then
    we can write $a \in \mf{A}_n$ as $a = \sum \alpha_i \beta_i$ where
    $\alpha_i \in A$ and $\beta_i$ a dominant coefficient of a polynomial of
    degree at most $n$ in $I$. Thus the $\mf{A}_n$ form a chain of ideals of
    $A$ that stabilizer for $n \geq N$. Then $\mf{A}_N = (\beta_1, \ldots,
    \beta_r)$. Set $Q_i = \beta_i X^N + \cdots \in I$. If $P \in I$, then there
    exists $S$ such that $P = QS+R$ such that $Q \in A Q_1 + \cdots + A Q_r$
    and $\deg R < N$.

    Therefore $P \in (Q_1, \ldots, Q_r) + A[X]_{N-1} \cap I$, so $I \subset
(Q_1, \ldots, Q_r) + A[X]_{N-1} \cap I$ and is thus finitely generated.
\end{proof}

\begin{cor} Let $B$ be a finitely-generated $A$-algebra. Then if $A$ is
Noetherian, $B$ is also Noetherian.  \end{cor}

\begin{cor} Any finitely generated algebra over a field is Noetherian.
\end{cor}

\begin{rmk} Suppose $A$ is Noetherian and $M$ an $A$-module. If $M$ is finitely
generated, then $M$ is Noetherian, but submodules are not necessarily
Noetherian. However, they are finitely generated.  \end{rmk}

Suppose $A \subset B$ is an inclusion of rings. Then we say that $x \in B$ is
\textit{integral over $A$} if there exists a monic polynomial $Q \in A[t]$ such
that $Q(x) = 0$.

\begin{prop} \label{prop:integral} The following are equivalent:
    \begin{enumerate} \item $x \in B$ is integral over $A$; \item $A[x]$ is a
        finitely-generated $A$-module; \item There exists $A[x] \subset C
        \subset B$ such that $C$ is a finitely-generated $A$-module.  \item
There exists a faithful $A[x]$-module $M$ which is finitely generated over $A$.
\end{enumerate} \end{prop}

\begin{proof} \begin{description} \item[1 implies 2] Note that $A[x]$ is
    generated by $1,x,x^2, \ldots, x^m$, where $Q$ has degree $m$.  \item[2
    implies 3] Set $C = A[x]$.  \item[3 implies 4] Choose $M = C$.  \item[4
    implies 1] Write $M = Am_1 + \cdots + A m_r$. $M$ is an $A[x]$-module, so
    we can consider $x . M \subset M$. Then for all $i$, we have $x m_i = \sum
    a_{ij} m_j$, so if we write consider the matrix $T = (a_{ij})$, then this
    matrix represents the map given by multiplication by $x$. Therefore we have
    \[ \det (T - x I_r) \cdot \begin{pmatrix} m_1 \\ \vdots \\ m_r
    \end{pmatrix} = 0, \] so set $Q = \det(T - x I_r)$. This is monic and $Q(x)
    . m_i = 0$ for all $i$, and therefore $Q(x)$ acts by $0$ on $M$. Because
    $M$ is a faithful $A[x]$-module, we have $Q(x) = 0$.  \end{description}
\end{proof}

\begin{exer} Let $x,y \in B \supset A$. Show that if $x$ and $y$ are integral
over $A$ then so are $x+y,xy$.  \end{exer}

\begin{prop} Let $A \subset B \subset C$. Assume that $A$ is Noetherian and
that $C$ is a finitely-generated $A$-algebra. If $C$ is a finitely-generated
$B$-module, then $B$ is a finitely-generated $A$-algebra.  \end{prop}

\begin{proof} Write $C = B c_1 + \cdots + B c_r$. Also, we can write $C =
    A[x_1, \ldots, x_m]$ for some $x_i \in C$. Then we can write $x_i = \sum
    b_{ij} c_j$ and $c_ic_j = \sum b_{ijk} c_k$ for $b_{ij}, b_{ijk} \in B$.
    Then $B_0 = A[b_{ij}, b_{ijk}]$ is a finitely-generated $A$-algebra. Any
    element of $C$ is a polynomial in the $x_i$ with coefficients in $A$, so
    $C$ is a finitely-generated $B_0$-module. In particular, $B_0$ is
    Noetherian. Because $B \subset C$, this implies that $B$ is a finitely
    generated $B_0$-module, so it is a finitely-generated $A$-algebra.
\end{proof}

\begin{cor} Let $k$ be a field and $E$ a finitely-generated $k$-algebra. If $E$
is a field, then $E$ is a finite extension of $k$.  \end{cor}

\begin{proof} Let $E$ be a finitely-generated $k$-algebra. Then there exist
    $x_1, \ldots, x_r \in E$ that are algebraically independent over $k$. Then
    $E$ is algebraic over $k(x_1, \ldots, x_r)$, which is the field of
    fractions of $k[x_1, \ldots, x_r]$. However, this gives an inclusion $k
    \subset F \subset E$, where $E$ is a finitely-generated $k$-algebra and $E$
    is algebraic over $F$.

    By the proposition, $F$ is a finitely-generated $k$-algebra. Therefore, we
    can write $F = h[y_1, \ldots, y_s]$, where $y_i = \frac{f_i}{g_i}$. Because
    $k[x_1, \ldots, x_n]$ is a UFD, then we can write \[ h = \prod_{i=1}^s g_i
    + 1 \in k[x_1, \ldots, x_n]. \] $h$ is repatively prime to all of the
$g_i$, so $\frac{1}{h} \notin k[y_!, \ldots, y_s]$. This gives a contradiction,
so $E$ must be algebraic over $k$.  \end{proof}

\subsection{Primary Decomposition in Noetherian Rings}%
\label{sub:existence_of_}

\begin{defn} An ideal $\mf{a} \subset A$ is \textit{irreducible} if for any
decomposition $\mf{a} = \mf{b} \cap \mf{c}$, then either $\mf{a} = \mf{b}$ or
$\mf{a} = \mr{c}$.  \end{defn}

\begin{exm} If $\mf{a}$ is a prime ideal, then $\mf{a}$ is irreducible. To see
this, if $\mf{a} \mid \mf{bc}$, then $\mf{a}$ contains one of $\mf{b}, \mf{c}$,
and so either $\mf{a} = \mf{b}$ or $\mf{a} = \mf{c}$.  \end{exm}

\begin{rmk} Suppose $\mf{m} \subset A$ is a maximal ideal. Then any power
$\mf{m}^n$ of $\mf{m}$ is primary.  \end{rmk}

\begin{proof} We want to prove that the zero divisors of $A / \mf{m}^n$ are
    nilpotent. Because $\mf{m}$ is maximal, then $A / \mf{m}^n$ is a local ring
    with maximal ideal $\mf{m} / m^n$. But then $A / \mf{m}^n \setminus \mf{m}
    / \mf{m}^n$ are all units, so everything in $\mf{m}$ is nilpotent.
\end{proof}

\begin{lem} If $A$ is Noetherian, then every irreducible ideal is primary.
\end{lem}

\begin{proof} Let $\mf{a} \subset A$ be irreducible. Then we can pass to the
    quotient, so we may asusme $\mf{a} = 0$. Let $x,y$ be nonzero with $xy =
    0$. We want to show that $x$ is nilpotent.

    Because $A$ is Noetherian, then there exists $n$ such that
$\operatorname{Ann} x^n = \operatorname{Ann} x^{n+1}$. We want to show that
$(x^n) \cap (y) = 0$, so choose $z = ax^n = by$. Then $zx = ax^{n+1} = byx =
0$, so $a \in \operatorname{Ann} x^{n+1} = \operatorname{Ann} x^n$. However,
this means $z = 0$. Because $0$ is irreducible, then $(x^n) = 0$, so $x^n = 0$.
\end{proof}

\begin{cor} If $A$ is Noetherian, then every ideal of $A$ has a primary
decomposition. In other words, we can write $I = \mf{a}_1 \cap  \cdots \cap
\mathfrak{a}_r$, where each $\mf{a_i}$ is primary.  \end{cor}

\begin{proof} Let $S$ be the set of ideals with no primary decomposition. If
    $S$ is nonempty, then $S$ has a maximal element $I$. To see this, we can
    use the fact that $A$ is Noetherian, so any chain of ideals in $S$
    eventually stabilizes. We know that $I$ is not irreducible, we can write $I
    = \mf{a} \cap \mathfrak{b}$ such that $I \neq \mf{a}, \mf{b}$. In addition,
    $\mf{a}, \mf{b} \notin S$, so they have a primary decomposition. This
    implies that $\mf{a} \cap \mf{b} = I$ has a primary decomposition.
\end{proof}

\begin{rmk} This decomposition is not unique. For example, consider $I = \ev{
x^2, xy } \subset k[x,y]$. Then $I = \ev{x} \cap \ev{x^2, xy, y^n}$ for all $n
> 0$.  \end{rmk}

\subsection{Artinian Rings}% \label{sub:artinian_rings}

\begin{prop} Assume that $A$ is Artinian.  \begin{enumerate} \item Every prime
ideal of $A$ is maximal.  \item $A$ has finitely many maximal ideals.  \item
The Jacobson radical of $A$ is nilpotent.  \end{enumerate} \end{prop}

\begin{proof} \begin{enumerate} \item Fix a prime ideal $\mf{p}$ and consider
    the domain $B = A / \mf{p}$. Choose $B \ni x \neq 0$ and consider the
    decreasing chain $(x^n)$ of ideals. This stabilizes, so there exists $(x^n)
    = (x^{n+1})$, so we can write $x^n = x^{n+1}y$ for some $y \in B$, and
    therefore $1 = xy$ because $B$ is a domain. Therefore $x$ has an inverse,
    so $B$ is a field. Thus $\mf{p}$ is maximal.  \item Suppose we have
    infinitely many maximal ideals $\mf{p}_1, \ldots, \mf{p}_n, \ldots$ that
    are pairwise distinct. Then we form a chain \[ \mf{p}_1 \supset
    \mf{p}_1\mf{p}_2 \supset \cdots \] which becomes stationary. Therefore
    $\mf{p}_1 \cdots \mf{p}_n \subset \mf{p}_{n+1}$, so $\mf{p}_{n+1}$ contains
    some $\mf{p}_i$. Because these ideals are maximal, this is a contradiction.
\item Consider $I = \mr{rad}(A) = \mf{m}_1 \cap \cdots \cap \mf{m}_n$. Then the
    chain $I \supset I^2 \supset \cdots$ stabilizes, so $I^n = I^{n+1}$ for
    some $n$. Let $J = ((0): I^n)$.\footnote{Here, $(\mf{a}: \mf{b}) = \qty{x
    \in A \mid x\mf{b} \in \mf{a}}$.} We will show that $J = A$. If not, let
    $J' \supsetneq J$ such that $J'$ is minimal for this property. Such a $J'$
    exists because $A$ is Artinian.

            Let $x \in J' \setminus J$ and consider the ideal $Ax + J$ By
    minimality of $J'$, we see that $Ix + J \subsetneq J'$ (otherwise $J = J'$
    by Nakayama's lemma). Therefore $Ix + J = J$, so $Ix \subset J$ and thus $x
    \in (J:I)$. Therefore, $I^{n+1}x \subset I^n J = (0)$. This implies $I^nx =
    0$, so $x \in J$ and thus $J' = J$. \qedhere \end{enumerate} \end{proof}

\begin{defn} An $A$-module $M$ is called \textit{irreducible} if $0$ and $M$
are the only submodules of $M$.  \end{defn}

\begin{defn} An $A$-module $M$ is said to be \textit{of finite length}  if
    there exists a (finite) decreasing sequence of submodules \[ M = M_0
    \supsetneq M_1 \supsetneq \cdots \supsetneq M_{n+1} = 0 \] such that $M_i /
    M_{i+1}$ is irreducible for $i = 0, \ldots, n$. In this case, $n$ is
    actually unique and depends only on $M$. We will call $n$ the
    \textit{length} of $M$.  \end{defn}

\begin{prop} Let $A$ be a ring. Then $A$ is Artinian if and only if $A$ is of
finite length as an $A$-module.  \end{prop}

\begin{proof} If $A$ is of finite length, then we have a sequence $A = M_0
    \supsetneq \cdots \supsetneq M_{n+1} = 0$ where $M_i / M_{i+1}$ is
    irreducible. If $\mf{a}_1 \supset \mf{a}_2 \supset \cdots$ is a decreasing
    chain of ideals, so $\mf{a}_i \cap M_n$ is a decreasing chain of ideals.
    However, each is either $M_n$ or $0$, so this chain stabilizes. Similarly,
    the chain $(M_j \cap \mf{a}_i) / (M_{j+1} \cap \mf{a}_i)$ also stabilizes
    for all $j$. Therefore, there exists $N$ such that for all $i > N$, $M_j
    \cap \mf{a}_i / (M_{j+1} \cap \mf{a}_i)$ is constant for all $j$, so
    $\mf{a}_i$ is constant for all $i > N$.

    Now suppose that $A$ is Artinian. Choose $I = \mr{rad}(A) = \mf{m}_1 \cap
    \cdots \cap \mf{m}_m$, where the $\mf{m}_i$ are the maximal ideals of $A$.
    Then $I$ is nilpotent, so there exists $n > 0$ such that \[ 0 = I^n =
    \mf{m}_1^n \cdots \mf{m}_m^n. \] Then $A = A / I^n = \prod A/ \mf{m}_j^n$
    by the Chinese remainder theorem, so $A / \mf{m}_j^n$ is clearly a local
    ring and is of finite length as an $A$-module. Note that the
    $A/\mf{m}_j$-vector space $\mf{m}_j^i / \mf{m}_j^{i+1}$ is
    finite-dimensional because $A$ is Artinian. Therefore $\mf{m}_j^i /
    \mf{m}_j^{i+1}$ is of finite length.  \end{proof}

\begin{exer} If there is an exact sequence $0 \to N \to M \to P \to 0$ of
$A$-modules, then $M$ is of finite length if and only if $N$ and $P$ are of
finite length. Moreover, $\ell(M) = \ell(P) + \ell(N)$.  \end{exer}

\begin{thm} $A$ is Artinian if and only if $A$ is Noetherian and $\dim A = 0$.
\end{thm}

\begin{proof} If $A$ is Artinian, we have already proved that $\dim A = 0$. By
    the previous proposition, because $A$ is of finite length, $A$ is
    Noetherian. To see this, for a chain $\mf{a}_1 \subset \mf{a}_2 \subset
    \cdots$, note that $\mf{a}_m \cap M_i / \mf{a}_m \cap M_{i+1}$ stabilizes.
    We can do this for each $i$, so any increasing chain stabilizes.

    Now assume $A$ is Noetherian and has dimension $0$. We know that $(0)$ has
    a primary decomposition, so we can write $(0) = \mf{q}_1 \cap \cdots \cap
    \mf{q}_r$, where each $\mf{q}_i$ is primary. Then $\mf{m}_i =
    \sqrt{\mf{q}_i}$ is a prime ideal, so it is maximal because $\dim A = 0$.
    Because $A$ is Noetherian and for all $x \in \mf{m}_i$, $x^n \in \mf{q}_i$
    for $n \gg 0$, so there exists $N$ such that $\mf{m}_i^N \subset \mf{q}_i$
    for each $i$. Therefore \[ \mf{m}_1^N \cdots \mf{m}_r^N \subset \mf{q}_1
    \cdots \mf{q}_r \subset \mf{q}_1 \cap \cdots \cap \mf{q}_r = 0, \] so
    $\mf{m}_1^N \cdots \mf{m}_1^N = 0$. Therefore, $A \cong A / \mf{m}_1^N
    \times \cdots \times A / \mf{m}_r^N$. Each $A / \mf{m}_i^N$ is of finite
    length (because each $\mf{m}_i^j / \mf{m}_i^{j+1}$ is a finite-dimensional
    vector space), so $A$ is of finite length.  \end{proof}

\begin{prop} Let $A$ be a Noetherian local ring with maximal ideal $\mf{m}$.
    Then one of the following holds: \begin{enumerate}[label=(\alph*)] \item
        Either $\mf{m}^n \supsetneq \mf{m}^{n+1}$ for all $n$, or; \item
        $\mf{m}^n = 0$ for $n \gg 0$ and in this case, $A$ is Artininian.
\end{enumerate} \end{prop}

\begin{proof} If $\mf{m}^n = \mf{m}^{n+1}$, then $\mf{m}^n = 0$ by Nakayama's
    lemma. This implies that $A = A / \mf{m}^n$ is of finite length. Then if
    $\mf{p}$ is prime, then $\mf{m}^n = (0) \subset \mf{p}$, so $\mf{m} \subset
    \mf{p}$. Because $\mf{m}$ is maximal, $\mf{m} = \mf{p}$, so $\dim A = 0$.
\end{proof}

\begin{thm}[Structure Theorem for Artinian Rings] An Artinian ring is uniquely
up to isomorphism a finite product of Artinian local rings.  \end{thm}

\begin{proof} Previously, we proved that $A = \prod A / \mf{m}_i^N$. Each of
these is a local Artinian ring.  \end{proof}

\chapter{Linear Algebra of Modules}% \label{cha:linear_algebra_of_modules}

\begin{prop} Assume $M, N, P$ are $A$-modules.  \begin{enumerate} \item The
    sequence $N \to M \to P \to 0$ is exact if and only if for all $A$-modules
    $Q$, the sequence \[ 0 \to \Hom(P,Q) \to \Hom(M,Q) \to \Hom(N,Q) \] is
    exact.  \item The sequence $0 \to N \to M \to P $ is exact if and only if
    for all $A$-modules $Q$, the sequence \[ 0 \to \Hom(Q,N) \to \Hom(Q,M) \to
\Hom(Q,P) \] is exact.  \end{enumerate} \end{prop}

\begin{proof} This is left as an exercise.  \end{proof}

\begin{rmk} In general, if $0 \to M \to N \to P \to 0$ is exact, then
    \begin{align*} 0 \to \Hom(P,Q) \to \Hom(N,Q) \to \Hom(M,Q) \\ 0 \to
    \Hom(Q,M) \to \Hom(Q,N) \to \Hom(Q,P) \end{align*} are exact but the last
    morphism is not necessarily surjective.  \end{rmk}

\begin{defn} A module $Q$ is \textit{projective} if the functor $\Hom(Q,-)$ is
exact. Here, exact means that short exact sequences are preserved. Similarly, a
module $I$ is \textit{injective} if the functor $\Hom(-,I)$ is exact.
\end{defn}

\begin{prop} A module $Q$ is projective if and only if $Q$ is a direct factor
of a free module. In other words, there exists a free module $F$ and $A$-module
$Q'$ such that $F = Q \oplus Q'$.  \end{prop}

\begin{proof} Suppose $Q$ is projective. Then there is a surjection $\pi:
    A^{(S)} \to Q \to 0$. Because $Q$ is projective, there exists a map
    $\theta$ such that $\pi \circ \theta = \mr{id}$. Therefore $A^{(S)} \cong Q
    \oplus Q'$, where $Q'$ is the kernel of $\pi$.

    On the other hand, if $A^{(S)} = Q \oplus Q'$, then for any diagram of the
    form \begin{equation} \begin{tikzcd} M \arrow{r} & P \arrow{r} & 0 \\ & Q
    \arrow{u} \end{tikzcd} \end{equation} we can embed $Q$ in $A^{(S)}$ and
    then use projectivity of free modules (because $\Hom(A,M) = M$).
    \end{proof}

\begin{rmk} If $M$ is projective and finitely generated, then it is a direct
factor of a finite free module.  \end{rmk}

\begin{defn} A \textit{projective resolution} of an $A$-module $M$ is a right
    bounded complex \[ \cdots \to P_{n+1} \to P_n \to \cdots \to P_0 \to 0 \]
    of projective modules such that there exists $P_0 \to M$ such that \[
    \cdots P_{n+1} \to P_n \to \cdots \to P_0 \to M \to 0 \] is exact.
\end{defn}

\begin{exer} Show that any module has a projective resolution (\textbf{Hint:}
construct a free resolution). In addition, any two projective resolutions are
homotopic.  \end{exer}

\begin{defn}
    %A map of complexes $\phi: M_{.} \to N_{.}$ is a \textit{homotopy} if there
    %exist maps 
    $k_q: M_q \to N_{q+1}$ such that $\phi_q = d^N \circ k_q + k_{q-1} \circ
d^M$.  \end{defn}

\section{Tor and Ext Functors}% \label{sec:derived_functors}

Note that for a complex, we can compute the \textit{homology} $H_q(M_{\bullet})
\coloneqq \ker d_q / \Im d_{q+1}$.  This measures the defect of the complex
from being exact.  For functors that are not exact, we can construct
\textit{derived functors} that measure the defect of exactness.  Let $F:
A\text{-}\ms{Mod} \to A\text{-}\ms{Mod}$ be right exact.  Then for any $M$, we
can consider a projective resolution $P_{\bullet} \to M \to 0$.  Applying $F$
to $P_{.}$, then the \textit{left derived functor} $L_{\bullet} F(M)$ is
defined by $L_{\bullet} F(M) = H_{\bullet}(F(P_{\bullet}))$.

\begin{prop} If $0 \to M \to N \to P \to 0$ is exact, then we have a long exact
    sequence \[ \cdot \to L_1F(M) \to L_qF(N) \to L_q F(P) \to L_{q-1} F(M) \to
    \cdots \to L_0 F(M) \to L_0 F(N) \to L_0 F(P) \to 0. \] \end{prop}
    
Recall that the tensor product $M \otimes N$ of two modules $M,N$ is an
$A$-module with a bilinear map $M \times N \to M \otimes N$ such that all
bilinear maps $M \times N \to P$ factor through $M \otimes N$.

\begin{prop} The functors $- \otimes N, \Hom(N,-)$ are an adjoint pair.
\end{prop}

\begin{cor} If $N \to M \to P \to 0$ is exact, then \[ N \otimes Q \to M
\otimes Q \to P \otimes Q \to 0 \] is exact.  \end{cor}

\begin{defn} A module $Q$ is \textit{flat} if $- \otimes Q$ is exact.
\end{defn}

We can defined the left derived functors $\operatorname{Tor}_q(Q,M)$ of the
tensor product.

\begin{prop} Any projective module is flat.  \end{prop}

\begin{proof} Clearly free modules are flat, so write $Q \oplus Q' = A^{(S)}$
and then note that the tensor product distributes over the sum.  \end{proof}

\begin{prop} The following are equivalent: \begin{enumerate} \item $M$ is flat
    over $A$.  \item If $N' \hookrightarrow N$, then $M \otimes N'
    \hookrightarrow M \otimes N$.  \item For all finitely generated ideals $I
    \subset A$, $I \otimes M \hookrightarrow M$.  \item For any finitely
    generated ideals $I \subset A$, $\operatorname{Tor}_1(M,A/I) = 0$.  \item
    For any finitely generated module $N$, we have $\operatorname{Tor}(M,N) =
    0$.  \item For all $a_i \in A$ and $x_i \in M$ such that $\sum a_i x_i = 0$
    there exist $y_1, \ldots, y_s \in M$ and $b_{ij}$ such that $x_i = \sum
    b_{ij} y_j$.  \end{enumerate} \end{prop}

\begin{proof} It is clear that 1 is equivalent to 2 implies 3 implies 4 implies
    5. The directions 3 implies 2 and 4 implies 3 are left to the reader.
    \begin{description} \item[1 implies 6] Choose $a_i \in A, x_i \in M$ such
        that $\sum_{i=1}^r a_i x_i = 0$. Then define a map $A^r \xrightarrow{f}
        A$ by \[ f(b_1, \ldots, b_r) = \sum_{i=1}^r a_i b_i \] and define $K =
        \ker f$.  Because $M$ is $A$-flat, we have an exact sequence \[ 0 \to K
        \otimes M \to M^r \to M. \] Then $(x_1, \ldots, x_r) \in \ker f \otimes
        \mr{id}_M$. Therefore there exists $b_1, \ldots, b_s \in K$ and $y_1,
        \ldots, y_s \in M$ such that \[ (x_1, \ldots, x_r) = \sum_{j=1}^s b_j
            \otimes y_j. \] Writing $b_j = (b_{1j}, \ldots, b_{rj})$, we obtain
            the identity \[ \sum_{i=1}^r b_{ij} a_i = 0 \] and thus $x_i = \sum
            b_{ji} y_j$.  \item[6 implies 3] Choose an ideal $I \subset A$.
            Consider the map $0 \to I \otimes M \to M$. Then for any element in
            the kernel, we can write \[ \sum_i a_i \otimes x_i \mapsto \sum a_i
                x_i = 0. \] Then we can write $x_i = \sum b_{ij} y_j$ and so \[
            \sum a_i \otimes x_i = \sum \sum a_i \otimes b_{ij}y_j = \sum
        \qty(\sum a_i b_{ij}) \otimes y_j = 0 \] and thus $I \otimes M \to M$
is injective. \qedhere \end{description} \end{proof}

Let $\phi: A \to B$ be a map of rings and let $M$ be a $B$-module. Define
$\phi$ to be \textit{flat} if $B$ is flat as an $A$-module.

\begin{prop} If $\phi: A \to B$ is flat and $M$ is a flat $B$-module, then $M$
is also flat as an $A$-module.  \end{prop}

\begin{proof} Let $S$ be an $A$-module. Then $S \otimes_A M = S \otimes_A (B
    \otimes_B M) = (S \otimes_A B) \otimes_B M$. If $0 \to N_1 \to N_2$ is an
    exact sequence of $A$-module, by flatness of $B$ as an $A$-module, then \[
    0 \to N_1 \otimes_A B \to N_2 \otimes_A B\] is exact. Because $M$ is flat
    over $B$, we see that \[ 0 \to (N_1 \otimes_A B) \otimes_B M \to (N_2
    \otimes_A B) \otimes_B M \] is exact, as desired.  \end{proof}

Now let $M$ be an $A$-module. Then for any map $A \xrightarrow{\phi} B$, we can
consider the $B$-module $M_{(B)} \coloneqq M \otimes_A B$.

\begin{prop} If $M$ is $A$-flat, then $M_{(B)}$ is $B$-flat.  \end{prop}

\begin{proof} For a $B$-module $S$, write \begin{align*} S \otimes_B M_{(B)} &=
S \otimes_B (M \otimes_A B) \\ &\cong S \otimes_B (B \otimes_A M) \\ &\cong (S
\otimes_B B) \otimes_A M \\ &\cong S \otimes_A M.  \end{align*} Thus if $0 \to
S_1 \to S_2$ is exact, then $0 \to S_1 \otimes_A M \to S_2 \otimes_A M$ is
exact because $M$ is $A$-flat, as desired.  \end{proof}

\begin{prop} Let $S \subset A$ be a multiplicative subset of $A$. Then the
morphism of rings $A \to S^{-1}A$ is flat.  \end{prop}

The proof is left to the reader. This can be reformulated as $M \otimes_A
S^{-1} A \cong S^{-1}M$.

Now we will give some remarks about the $\operatorname{Ext}$ functors. For any
left exact functor, we may define the right derived functors $R^{\bullet} F$ by
\[ R^i F(M) = H^i(F(I^{\bullet})) \] where $M \to I^{\bullet}$ is an injective
resolution. Then we will define the right derived functors of $\Hom_A(N,-)$ by
$\Ext_A^i(N,-)$.

\begin{prop} If $M$ is injective, then $\Ext_A^i(N,M) = 0$ for all $i > 0$.
Similarly, if $N$ is projective, then $\Ext_A^i(N,M) = 0$ for all $I > 0$.
\end{prop}

\begin{rmk} We can compute $Ext^i(N,M)$ using a projective resolution of $N$.
\end{rmk}

\begin{prop} Let $A \to B$ be a morphism of rings and let $M, N$ be
    $A$-modules. Then let $M_{(B)}, N_{(B)}$ be their base changes to $B$. Then
    we have \[ \Ext_B^i (M_{(B)}, N_{(B)}) = \Ext_A^i(M,N)_{(B)} \] and \[
\Tor_i^B(M_{(A)}, N_{(B)}) = \Tor_i^A(M,N)_{(B)} \] if $B$ is $A$-flat.
\end{prop}

\begin{proof} This follows from the definition of $\Ext, \Tor$ using projective
    resolutions using the following facts.  \begin{enumerate} \item If $M$ is
        $A$-projective, then $M_{(B)}$ is $B$-projective.  \item Since $B$ is
        $A$-flat, for any complex $X^{\bullet}$ of $A$-modules, then
        $H^{\bullet}(X_{(B)}^{\bullet}) = H^{\bullet}(X^{\bullet})_{(B)}$.
\end{enumerate} \end{proof}

\section{Flatness}% \label{sec:flatness}

\begin{prop} Let $A$ be a local ring. Then any finitely generated flat
$A$-module is free. In particular, free, projective, and flat are equivalent
for $A$-modules.  \end{prop}

\begin{proof} We know that free implies projective implies flat. Therefore we
    will show that if $M$ is flat, then it is free. Assume that $M$ is finitely
    generated and $A$-flat. Let $k = A / \mf{m}$ be the residue field of $A$.
    Define $\ol{M} = M \otimes_A k$, which is a vector space of finite
    dimension over $k$. Then there exists $x_1, \ldots, x_r \in M$ that descend
    to a basis of $\mf{M}$.

    Then the map $A^r \to M, (a_i) \mapsto \sum a_i x_i$ is surjective by
    Nakayama's lemma. We will prove that this map is injective by induction. If
    $r = 1$, then suppose $a x_1 = 0$. Then there exist $y_1, \ldots, y_s,
    b_{11}, \ldots, b_{1s}$ such that \[ x_1 = \sum_{j=1}^s b_{ij} y_j \] where
    $ab_{ij} = 0$ for all $j = 1, \ldots, s$. Because $\ol{x}_1 \neq 0$, there
    exists $j$ such that $\ol{b}_{1j} \neq 0$ so $b_{1j}$ is invertible in $A$.
    Thus $a = 0$.

    Now suppose $a_1 x_1 + \cdots + a_r x_r = 0$. Then there exist $y_1,
    \ldots, y_s$ and $b_{ij}$ such that \[ x_i = \sum b_{ij} y_j \] and \[ \sum
    a_i \begin{pmatrix} b_{i1} \\ \vdots \\ b_{ij} \end{pmatrix} = 0. \]
    Because $\ol{x}_r \neq 0$, we see that $\ol{b}_{rj} \neq 0$ for some $j$
    and thus $b_{ij}$ is a unit. Then $a_1 b_{1j} + \cdots + a_r b_{rj} = 0$,
    so we can write \[ \sum a_i (x_i - c_i x_r) = 0. \] We know that $\ol{x}_1
- c_1 \ol{x}_r, \ldots, \ol{x}_{r-1} - c_{r-1} \ol{x}_r$ are linearly
independent over $k$, so from the induction $a_1 = \cdots = a_r = 0$ and thus
$a_r = 0$.  \end{proof}


\begin{rmk} If $M$ is not finitely generated, the proposition is false. An
example is given by taking the field of fractions of a local domain.  \end{rmk}

When proving the proposition, we in fact proved that \begin{lem} If $x_1,
    \ldots, x_r \in M$ with $M$ a flat $A$-module for $A$ a local ring and
    $\ol{x}_1, \ldots, \mf{x}_r$ are linearly independent in $M \otimes_A k$,
    then $x_1, \ldots, x_r$ are linearly independent in $M$.  \end{lem}

\begin{prop} Suppose that $A \to B$ is flat and $I_1, I_2$ are ideals of $A$.
Then \begin{enumerate} \item $(I_1 \cap I_2)B = I_1 B \cap I_2 B$; \item If
$I_2$ is finitely generated, then $(I_1 : I_2)B = (I_1 B : I_2 B)$.
\end{enumerate} \end{prop}

\begin{proof} The proof is a formal consequence of flatness.  \begin{enumerate}
    \item Consider the exact sequence $0 \to I_1 \cap I_2 \to A \to A/I_1
        \times A/I_2$. Tensoring with $B$, we obtain an exact sequence \[ 0 \to
        (I_1 \cap I_2) \otimes B \to B \to B/I_1 B \times B/I_2 B. \] But then
        $(I_1 \cap I_2) \otimes B = (I_1 \cap I_2)B$, but the kernel of the
        last map is clearly $I_1 B \cap I_2 B$.  \item Set $I_2 = (x_1, \ldots,
        x_r)$. Then because \[ (I_1 : I_2) = \bigcap_{i=1}^r (I_1 : x_i A), \]
        it suffices to prove the result for $I_2$ a principal ideal. We have an
        exact sequence \[ 0 \to (I_1 : x A) \to A \xrightarrow{\times x} A/I_1.
        \] Tensoring by $B$, we obtain \[ 0 \to (I_1 : x A) \otimes B \to B \to
    B/I_1 B, \] and by analysing the kernel, we see that $(I_1: xA) B = (I_1 B:
    x B)$. By repeated application of the previous part, the desired result
    follows. \qedhere \end{enumerate} \end{proof}

\begin{exm} We will give an example where the previous proposition is not true
    in general. Let $A = k[x,y]$ and $B = A/xA = k[y]$. Then choose $I_1 =
    (x+y), I_2 = (y)$, so $I_1 \cap I_2 = I_1 I_2 = ( (x+y)y )$. But then we
    have $(I_1 \cap I_2) B = y^2 B$, but $I_1 B \cap I_2 B = yB$.

    Another example is $A = k[x,y], B = k[x,y,z]/(xz-y) \cong k[x,z], I_1 = xA,
I_2 = yA$. Here we can check that $(I_1 \cap I_2) = (xy)$, that $(I_1 \cap
I_2)B = x^2z B$, but $I_1 B \cap I_2 B = xz B$. Viewing this geometrically as
$\Spec B \to \Spec A$, we can check the fiber over $(0,0)$ and see that the map
is not flat.  \end{exm}

\begin{prop} Let $A \xrightarrow{\varphi} B$ be a ring homomorphism. The
    following are equivalent: \begin{enumerate} \item $B$ is flat over $A$;
        \item $B_{\mf{P}}$ is flat over $A_{\mf{p}}$ for all $\mf{P} \in \Spec
            B$ and $\mf{p} = \varphi^{-1} (\mf{P})$.  \item $B_{\mf{P}}$ is
    flat over $A_{\mf{p}}$ for any $\mf{P}$ maximal.  \end{enumerate}
\end{prop}

\begin{proof} \begin{description} \item[1 implies 2:] We know that $B_{\mf{p}}$
    is flat over $A_{\mf{p}}$. But then $B_{\mf{P}}$ is flat over $B_{\mf{p}}$
    because it is a localization. By transitivity of flatness, $B_{\mf{P}}$ is
    flat over $A_{\mf{p}}$.  \item[2 implies 3:] This is obvious.  \item[3
    implies 1:] Note that for all $\mf{P}$ maximal, $\Tor_i^A(B,N)_{\mf{P}} =
    0$ for $i > 0$. This implies that $\Tor_i^!(B,N) = 0$, and thus $B$ is flat
    over $A$. To get that the first $\Tor$ is zero we need to use the lemma
    belos. \qedhere \end{description} \end{proof}

\begin{lem} Let $\varphi: A \to B$ be a morphism of rings and choose $\mf{P}
    \in \Spec B$. Then let $\mf{p} = \varphi^{-1}(\mf{P})$ and $N$ an
    $A$-module. Then $\Tor_i^A(B,N)$ is a $B$-module and
    $\Tor_i^A(B,N)_{\mf{P}} = \Tor_i^{A_{\mf{p}}}(B_\mf{P},N_{\mf{p}})$.
\end{lem}

\begin{proof} Let $X_{\bullet} \to N$ be a projective resolution. Then $\Tor$
    is computed by the homology of the complex $B \otimes_A X_{\bullet}$. When
    we localize, we localize the homology at the $B$ term. However, $B_{\mf{p}}
    \otimes_A X_{\bullet} = B_{\mf{P}} \otimes_{A_{\mf{p}}}
    (X_{\bullet})_{\mf{p}}$, so because $X_i$ is $A$-projective, then
    $X_{\mf{p}}$ is $A_{\mf{p}}$-projective, and thus $(X_{\bullet})_{\mf{p}}$
    is a projective resolution of $N_{\mf{p}}$. Thus the complex $B_{\mf{P}}
    \otimes_{A_{\mf{p}}} (X_{\bullet})_{\mf{p}}$ computes the $\Tor$ as
    desired.  \end{proof}

\begin{defn} An $A$-module $N$ is said to be \textit{faithfully flat} if
    \begin{enumerate} \item $N$ is $A$-flat; \item For any sequence $P \to Q
        \to R$ of $A$-modules, if $P \otimes N \to Q \otimes N \to R \otimes N$
        is exact, then $P \to Q \to R$ is exact.  \end{enumerate} \end{defn}

\begin{thm} Let $M$ be an $A$-module. Then the following are equivalent:
    \begin{enumerate} \item $M$ is faithfully flat over $A$; \item $M$ is flat
        and for any nonzero $N$, $M \otimes N \neq 0$; \item $M$ is flat and
        for all maximal ideals $\mf{m} \subset A$, $\mf{m} . M \neq M$.
\end{enumerate} \end{thm}

\begin{proof} \begin{description} \item[1 implies 2:] Choose the sequence $0
    \to N \to 0$. Then tensor with  $M$. If $N \otimes M = 0$, the sequence is
    now exact, then the original sequence is exact, and thus $N = 0$.  \item[2
    implies 3:] Consider $N = A/\mf{m}$. Then $N \otimes M = M / \mf{m}M \neq
    0$, so $M \neq \mf{m}M$.  \item[3 implies 2:] Choose $0 \neq x \in N$ and
    set $I = \operatorname{Ann}(x) \subset \mf{m}$ for some maximal ideal
    $\mf{m}$. Then $Ax = A/I$, so $Ax \otimes M \cong M / IM \neq 0$. Because
    $Ax$ injects into $N$, $Ax \otimes M$ injects into $N \otimes M$, which
    must be nonzero.  \item[2 implies 1:] Consider the sequence $P
    \xrightarrow{f} Q \xrightarrow{g} R$. Then because $M$ is flat, $\ker(g
    \otimes \mr{id}_M) = \ker(g) \otimes M$ and $\Im(f \otimes \mr{id}_M) =
    \Im(f) \otimes M$. If $g \circ f = 0$, then $\Im(g \circ f) = 0$, which
    happens iff $\Im(g \circ f) \otimes M = 0$. Then $\Im((g \otimes \mr{id}_M)
    \circ (f \otimes \mr{id}_M)) = 0$. 
            
            If $P \otimes M \to Q \otimes M \to R \otimes M$ is exact, then $P
    \to Q \to R$ is a complex. Finally, we need that $\ker g = \Im f$. By
    flatness of $M$, we can tensor to find that $\ker g / \Im f \otimes M = 0$
    and then we see that $\ker g / \Im f = 0$. \qedhere \end{description}
\end{proof}

\begin{cor} Let $A \to B$ be a local homomorphism and let $M$ be a
finitely-generated $B$=module. Then $M$ if flat over $A$ if and only if $M$ is
faithfully flat over $A$.  \end{cor}

\begin{proof} Clearly faithfully flat implies flat. Then we need to show that
    $M \neq \mf{m}_A. M$. By Nakayama's lemma, we know that $M \otimes k_B \neq
    0$, so $\mf{m_B} M \neq M$. In particular, $\mf{m}_A M \neq M$. In
    particular, this implies that item \textbf{3} of the previous theorem
    holds. Thus $M$ is faithfully flat over $A$.  \end{proof}

\begin{rmk} This also shows that flat and faithfully flat are equivalent over
local rings. Alternatively, we can use the equivalence of flat and free.
\end{rmk}

\begin{rmk} Faithful flatness is transitive. In addition, if $A \to B$ is a
morphism of rings, and $M$ is faithfully flat over $A$, then $M \otimes_A B$ is
faithfully flat over $B$.  \end{rmk}

\begin{prop} Let $M$ be a faithfully flat $B$-module which is faithfully flat
over $A$. Then $B$ is faithfully flat over $A$.  \end{prop}

\begin{proof} Let $N$ be an $A$-module. Then $(B \otimes_A N) \otimes_B M = M
    \otimes_A N \neq 0$ if $N \neq 0$. This implies that $B \otimes_A N$ is
    nonzero. Now it suffices to show that $B$ is flat over $A$.

    Let $(S)$ be an exact sequence of $A$-modules. Then if we consider $((S)
\otimes_A M) = (S) \otimes_A M$, this is exact by flatness of $M$ over $A$. By
faithful flatness of $M$ over $B$, this implies that $(S) \otimes_A B$ is
exact.  \end{proof}

\begin{prop} Let $A \to B$ be faithfully flat. Then \begin{enumerate} \item For
    any $A$-module $N$, the map $N \to N \otimes_A B$ is injective; \item If $I
    \subset A$ is an ideal, then $IB \cap A = I$; \item The map $\Spec B \to
    \Spec A$ is surjective.  \end{enumerate} \end{prop}

\begin{proof} \begin{enumerate} \item Let $0 \neq x \in N$. Then $Ax \otimes B
    \hookrightarrow N \otimes B$. Because $B$ is faithfully flat, $Ax \otimes B
    \neq 0$.  \item Recall that $B/IB = B \otimes A/I$. Then the map $A/I \to
    B/IB$ is injective. Therefore we have a map $A \to B/IB$ which has kernel
    $I = IB \cap A$.  \item Choose $\mf{p} \in \Spec A$. Then $B_{\mf{p}} = B
    \otimes_A A_{\mf{p}}$ and $A_{\mf{p}} \to B_{\mf{p}}$ is faithfully flat by
    base change. This means that $\mf{p} B_{\mf{p}} \subsetneq B_{\mf{p}}$.
    Thus if we choose $\mf{m}$ to be a maximal ideal of $B_{\mf{p}}$ containing
    $\mf{p} B_{\mf{p}}$, we see that $\mf{m} \cap A_{\mf{p}} \supset
    \mf{p}A_{\mf{p}}$ and thus they are equal by maximality of
    $\mf{p}A_{\mf{p}}$. Then if we choose $\mf{P} = \mf{m} \cap B$, we see that
    \begin{align*} \mf{P} \cap A &= \mf{m} \cap A \\ &= \mf{m} \cap A_{\mf{p}}
    \cap A \\ &= \mf{p} A_{\mf{p}} \cap A \\ &= \mf{p}.  \end{align*} Thus the
    image of $\mf{P}$ is $\mf{p}$. \qedhere \end{enumerate} \end{proof}

\begin{thm} Let $\varphi: A \to B$ be a map of rings. The following are
    equivalent: \begin{enumerate} \item The map $\varphi$ is faithfully flat.
        \item The map $\varphi$ is flat and $\Spec B \to \Spec A$ is
            surjective.  \item The map $\varphi$ is flat and for all maximal
            ideals $\mf{m}$ of $A$, there exists some maximal ideal $\mf{m}'$
            of $B$ such that $\varphi^{-1}(\mf{m}') = \mf{m}$.  \end{enumerate}
        \end{thm}

\begin{proof} \begin{description} \item[1 implies 2:] This is the previous
    proposition.  \item[2 implies 3:] Choose a maximal ideal $\mf{m} \subset
    A$. Then there exists $\mf{P} \in \Spec B$ such that $\varphi^{-1}(\mf{P})
    = \mf{m}$. But then if $\mf{m}'$ is any maximal ideal containing $\mf{P}$,
    we see that $\varphi^{-1}(\mf{m}') = \varphi^{-1}(\mf{P}) = \mf{m}$ by
    maximality of $\mf{m}$.  \item[3 implies 1:] We want to prove that $B \neq
    \mf{m}B$ for any maximal ideal $\mf{m}$ of $A$. Then there exists $\mf{m}'$
    such that $\varphi^{-1}(\mf{m}') = \mf{m}$. But then $B \supsetneq \mf{m}'
    \supset \mf{m}B$. \qedhere \end{description} \end{proof}

\begin{prop}[Descent] Let $A \to B$ be faithfully flat and $M$ be an
    $A$-module. Then \begin{enumerate} \item $M$ is flat (resp. faithfully
        flat) if and only if $M \otimes_A B$ is $B$-flat (resp. $B$-faithfully
        flat).  \item Assume $A$ is a local ring and $M$ is finitely-generated.
        Then $M$ is free if and only if $M \otimes_A B$ is $B$-free.
\end{enumerate} \end{prop}

\begin{proof} \begin{enumerate} \item Let $(S)$ be an exact sequence. Then $(S)
    \otimes_A B$ is exact, so $S \otimes_A B \otimes M \otimes_A B = (S
    \otimes_A M) \otimes_A B$ is exact. By faithful flatness of $B$, $(S)
    \otimes_A M$ is exact. Now if $N \neq 0$ is another $A$-module, we know
    that $M_{(B)} \otimes N_{(B)} \neq 0$, but this is the same as $(M
    \otimes_A N)_{(B)}$, so $M \otimes_A N$ is nonzero.  \item Assume that $A$
    is local. Then suppose $M \otimes_A B$ is free. Therefore $M \otimes_A B$
    is faithfully flat. But then, $M$ is faithfully flat over $A$, which means
    that $M$ is free because $M$ is finitely generated. \qedhere
    \end{enumerate} \end{proof}

\begin{exer} Let $A \subset B$ be integral domains. Assume that $A$ and $B$
have the same field of fractions. Prove that $A \hookrightarrow B$ is
faithfully flat if and only if $A = B$.  \end{exer}

\section{More on Integral Dependence}% \label{sec:more_on_integral_dependence}

Recall Proposition \textbf{\ref{prop:integral}}. 

\begin{cor} Let $x_1, \ldots, x_n \in B$. If each $x_i$ is integral over $A$,
then $A[x_1, \ldots, x_n]$ is a finitely-generated $A$-module.  \end{cor}

\begin{cor} Let $C \subset B$ be the set of integral elements over $A$. Then
$C$ is a subring of $B$.  \end{cor}

\begin{proof} Note that $x + y, xy \in A[x][y] \in A[x,y]$, which is a
finitely-generated $A$ module. Therefore they are integral over $A$.
\end{proof}

\begin{rmk} The ring $C$ is not necessarily finitely-generated over $A$. For an
example, choose $\Z \ol{\Z} \subset \ol{\Q}$.  \end{rmk}

\begin{defn} Let $A \subset B$. Then we say that $B$ is \textit{integral over
$A$} if all elements of $B$ are integral over $A$.  \end{defn}

\begin{cor} Let $A \subset B \subset C$ be extensions of rings. Then if $B$ is
integral over $A$ and $C$ is integral over $B$, then $C$ is integral over $A$.
\end{cor}

Proof of this is left to the reader.

\begin{defn} Let $A$ be an integral domain. We say that $A$ is
\textit{integrally closed} if for all $x \in K = \operatorname{Frac} A$, then
$x$ is integral over $A$ if and only if $x \in A$.  \end{defn}

\begin{defn} Assume that $A \subset B$ is an inclusion of rings. Then the
\textit{integral closure of $A$ inside $B$} is the set of all elements of $B$
that are integral over $A$.  \end{defn}

\begin{exm} A typical example of this situation is when $A$ is a domain, $K$ is
    its fraction field, and $L / K$ is a field extension. Then we can consider
    the integral closure $B$ of $A$ inside $L$. In number theory, if $K$ is a
    number field, we define its \textit{ring of integers} $\mc{O}_K$ to be the
    integral closure of $\Z$ in $K$.  \end{exm}

\begin{exer} If $K = \Q(\zeta_p)$, prove that $\mc{O}_K = \Z[\zeta_p]$.
\end{exer}

\begin{exer} Let $B$ be an integral domain and $A \subset B$. Prove that the
integral closure of $A$ inside $B$ is integrally closed.  \end{exer}

\begin{lem} Let $B$ be a domain that is integral over $A$. Then $A$ is a field
if and only if $B$ is a field.  \end{lem}

\begin{proof} Assume that $A$ is a field. Now choose $0 \neq x \in B$. But then
    we know that \[ x^n + a_{n-1}x^{n-1} + \cdots + a_0 = 0 \] for some $a_i
    \in A$ and $a_0 \neq 0$. But then we have \[ x^{-1} = -a_0^{-1} \qty(
    \sum_{i=1}^n a_i x^i ) \] and so $x^{-1} \in B$.

    Now assume that $B$ is a field. Choose $0 \neq x in A$. Then $x^{-1} \in
    B$. This means that $x^{-1}$ is integral over $A$, which means that \[
    a^{-n} + a_{n-1} x^{-(n-1)} + \cdots + a_0 = 0 \] for some $a_i \in A, a_0
    \neq 0$. Then if we multiply by $a^{n-1}$, we obtain \[ a^{-1} + a_{n-1} +
    a_{n-2} x + \cdots + a_0 x^{n-1} = 0, \] which means $x^{-1} \in A$.
\end{proof}

\begin{cor} Let $A \subset B$ and $B$ be integral over $A$. Let $\mf{P} \in
\Spec B$ and $\mf{p} = A \cap \mf{P}$. Then $\mf{p}$ is maximal if and only if
$\mf{P}$ is maximal.  \end{cor}

\begin{proof} Note that $B/ \mf{P}$ is integral over $A / \mf{p}$. Then we
apply the lemma to $A / \mf{p} \subset B / \mf{P}$.  \end{proof}

We can refine this into going-up and going-down. Let $\phi: A \to B$ be a
morphism of rings and let $\phi^*: \Spec B \to \Spec A$ be the induced map of
spaces.

\begin{defn}[Going-up] A ring homomorphism $\phi$ satisfies the
    \textit{Going-up property} if the following holds:

    Let $\mf{p} \subset \mf{p}'$ be prime ideals of $A$ and suppose that
$\phi^*(\mf{P}) = \mf{p}$. Then there exists $\mf{P}' \supset \mf{P}$ such that
$\phi^*(\mf{P}') = \mf{p}'$.  \end{defn}

\begin{defn}[Going-down] A ring homomorphism $\phi$ satisfies the
    \textit{Going-down property} if the following holds:

    Let $\mf{p} \subset \mf{p}'$ be prime ideals of $A$. Then let $\mf{P}' \in
\Spec B$ satisfy $\phi^*(\mf{P}') = \mf{p}'$. Then there exists $\mf{P} \subset
\mf{P}'$ with $\phi^*(\mf{P}) = \mf{p}$.  \end{defn}

\begin{lem} \label{lem:goingdown} The going-down property is equivalent to the
    following:

    For all $\mf{p} \in \Spec A$ and $\mf{P}$ a minimal prime ideal of $B$
containing $\mf{p}B$, we have $\mf{P} \cap A = \mf{p}$.  \end{lem}

\begin{proof} First suppose that going-down holds. Then choose $\mf{P}$ be a
    minimal prime containing $\mf{p}B$. Then $\mf{p}' = \phi^{-1} (\mf{P})
    \supset \mf{p}$. If $\mf{p}' \neq \mf{p}$, then there exists $\mf{P}_0
    \subset \mf{P}$ such that $\varphi^{-1}(\mf{P}_0) = \mf{P}$, which
    contradicts minimality.

    Now suppose the other condition holds. Suppose $\mf{P}'$ goes to $\mf{p}'
\supset \mf{p}$. Then we know that $\mf{p} B \subset \mf{p}' B \subset
\mf{P}'$. If we fix $\mf{P}_0$ to be the minimal prime containing $\mf{p}B$,
then we see that $\mf{P}_0 \cap A = \mf{p}$.  \end{proof}

\begin{thm} If $\phi: A \to B$ is flat, then going-down holds.  \end{thm}

\begin{proof} Fix $\mf{p} \subset \mf{p}'$ and let $\mf{P}'$ lie over
    $\mf{p}'$. Then we know that $B_{\mf{P}'}$ is flat over $\mf{A}_{\mf{p}'}$.
    Because $A_{\mf{p}'}$ is local and the map $A_{\mf{p}'} \to
    \mf{B}_{\mf{P}'}$ is local, it is faithfully flat. This implies that the
    map $\Spec B_{\mf{P}'} \to \Spec A_{\mf{p}'}$ is surjective, so there
    exists $\mf{P}_1 \in \Spec B_{\mf{P}'}$ such that $\phi^{-1}(\mf{P}_1) =
    \mf{p} A_{\mf{p}'}$. 

    Now set $\mf{P} \coloneqq \mf{P}_1 \cap B$. Then we see that \begin{align*}
    \phi^{-1}(\mf{P}) &= \phi^{-1}(\mf{P}_1 \cap B) \\ &= \phi^{-1}(\mf{P}_1)
\cap A \\ &= \mf{p A_{\mf{p}'}} \cap A \\ &= \mf{p}. \qedhere \end{align*}
\end{proof}

We will see consequences of this result in algebraic geometry.

We will now consider integral ring extensions $A \subset B$.

\begin{thm}[Cohen-Seidenberg] Suppose $A \subset B$ is an integral extension.
    Then the following hold: \begin{enumerate} \item The map $\Spec B \to \Spec
        A$ is surjective.  \item There are no inclusion relations between the
        prime ideals of $B$ which are above a fixed prime ideal of $A$.  \item
        Going-up holds for $A \subset B$.  \item If $A$ is local with maximal
        ideal $\mf{m}$, then the prime ideals of $B$ lying over $\mf{m}$ are
        precisely the maximal ideals of $B$.  \item Assume further that $A$ and
        $B$ are integral domains and that $A$ is integrally closed. Then
        going-down holds for $A \subset B$.  \item If $B$ is the integral
        closure of $A$ in a normal extension of field $L$ of $K \coloneqq
        \operatorname{Frac} A$, then any two prime ideals of $B$ lying over the
        same prime ideal of $A$ are conjugate by an element of $\Aut(L/K)$.
\end{enumerate} \end{thm}

\begin{proof} We prove \textbf{4}, then \textbf{1}, \textbf{2}, and \textbf{3},
    then \textbf{6}, and finally \textbf{5}.       \begin{enumerate} \item Let
        $\mf{p} \in \Spec A$. Then $B_{\mf{p}} = B \otimes_A A_{\mf{p}}$ is
        integral over $A_{\mf{p}}$. Applying \textbf{4}, we obtain the desired
        result.  \item Consider $B_{\mf{p}}$ again. By \textbf{4}, because
        ideals lying over $\mf{p}$ are maximal, there cannot be inclusion
        relations between them.  \item Let $\mf{p} \subset \mf{p}'$ and $\mf{P}
        \in \Spec B$ lying over $\mf{p}$. Then $B / \mf{P}$ is integral over $A
        / \mf{p}$. By \textbf{1}, we know that $\Spec B/\mf{P} \to \Spec A /
        \mf{p}$ is surjective. Thus there exists $\ol{\mf{P}}' \in \Spec B /
        \mf{P}$ lying over $\ol{\mf{p}}' = \mf{p}' / \mf{p}$. Then we know that
        $\ol{\mf{P}}' = \mf{P}' / \mf{P}$ for some prime ideal $\mf{P}'$ of
        $B$, and this is the ideal we are looking for.  \item This is a
        consequence of Lemma \ref{lem:goingdown}.  \item Write $L =
        \operatorname{Frac} B \supset K = \operatorname{Frac} A$. Then let
        $L_1$ be the normal closure of $L / K$. Then let $\mf{p} \subset
        \mf{p}'$ in $A$ and $\mf{P}'$ in $B$ lie over $\mf{p}'$. Then let
        $\mf{P}_1 \subset \mf{P}'_1$ in $B_1$ the integral closure of $A$ in
        $L_1$. These exist thanks to \textbf{1} and \textbf{3}.  

            Let $\mf{P}_1''$ in $B_1$ such that $\mf{P}_1'' \cap B = \mf{P}'$.
            Then there exists $\sigma$ such that $\mf{P}_1'' =
            \sigma(\mf{P}_1')$ because both ideals are above $\mf{p}'$. Then we
            can choose \[ \mf{P} \coloneqq \sigma(\mf{P}_1) \cap B \subset
            \mf{P}_1'' \cap B = \mf{P}'. \] We need to show that $\mf{P} \cap A
            = \mf{p}$. But this is simply \begin{align*} \mf{P} \cap A &=
            \sigma(\mf{P}_1) \cap A \\ &= \sigma(\mf{P}_1 \cap A) \\ &=
        \sigma(\mf{p}) \\ &= \mf{p}.  \end{align*} \item We know that $A$ is
        integrally closed in $K$. Then let $L/K$ be a finite Galois (we can
        always reduce to this case) extension and $B$ the integral closure of
        $A$ in $L$. Then let $\mf{P}, \mf{P}' \in \Spec B$ lie above $\mf{p}
        \in \Spec A$. We will show there exists $\sigma \in \Gal(L/K)$ such
        that $\sigma(\mf{P}) = \mf{P}'$.

            Suppose that no such $\sigma$ exists. Then for all $\sigma \in
            \Gal(L/K)$, $\mf{P}' \neq \sigma(\mf{P})$. In particular, $\mf{P}'
            \not\subset \sigma(\mf{P})$. Then there exists $x \in \mf{P}'$
            which is not in any $\sigma(\mf{P})$ then we see that \[ y
            \coloneqq \prod_{\sigma \in \Gal(L/K)} \sigma(x) \in K \] is
            integral over $A$, so $y \in A$. Also, $y \notin \mf{P}$ because $x
            \notin \sigma(\mf{P})$, so $x \in \mf{P}'$ and thus $y \in
            \mf{P}'$, so $y \in \mf{p} \subset \mf{P}$. This gives a
            contradiction. \qedhere \end{enumerate} \end{proof}

\begin{cor} Assume that $B$ is integral over $A$.  \begin{enumerate} \item If
    $\mf{P}_0 \subsetneq \mf{P}_1 \subsetneq \cdots \subsetneq \mf{P}_r$ is a
    chain of prime ideals of $B$, the the $\mf{p}_i \coloneqq \mf{P}_i \cap A$
    for a chain of prime ideals of $A$.  \item If $p_0 \subsetneq \cdots
    \subsetneq \mf{p}_r$ is a chain of prime ideals of $A$, then there exists a
    chain $\mf{P}_0 \subsetneq \cdots \subsetneq \mf{P}_r$ of prime ideals of
    $B$ above it.  \item If $A$ is integrally closed and $B$ is a domain, then
    for any chain $\mf{p}_0 \subsetneq \cdots \subsetneq \mf{p}_r$ and
    $\mf{P}_r \in \Spec B$ above $\mf{p}_r$, there exists a chain $\mf{P}_0
    \subsetneq \cdots \subsetneq \mf{P}_r$ above the chain in $A$.
    \end{enumerate} \end{cor}

\begin{proof} The proof is clear and left to the reader.  \end{proof}

\begin{defn} Let $\mf{p} \in \Spec A$. Then define the \textit{height}  of
    $\mf{p}$ by \[ \mr{ht}(\mf{p}) = \max \qty{n \geq 0 \mid \exists \mf{p}_0
    \subsetneq \cdots \subsetneq \mf{p}_n = \mf{p}}. \] Then define the
    \textit{dimension} of $A$ by \[ \dim A = \max \qty{ \mr{ht}(\mf{p}) \mid p
    \in \Spec A }. \] \end{defn}

\begin{cor} Let $A \subset B$ be an integral extension. Then \begin{enumerate}
    \item Suppose $\mf{P} \in \spec B$ lies above $\mf{p} \in \Spec A$. Then
        $\mr{ht}(\mf{P}) \leq \mr{ht}(\mf{p})$.  \item $\dim A = \dim B$.
    \item If $A$ is integrally closed and $B$ is a domain, then we have
        $\mr{ht}(\mf{P}) = \mr{ht}(\mf{p})$.  \end{enumerate} \end{cor}

\begin{proof} This is an immediate consequence of the previous corollary.
\end{proof}

\section{Associated Primes}% \label{sec:associated_primes}

Let $M$ be an $A$-module and $\mf{p} \in \Spec A$.

\begin{defn} We say that $\mf{p}$ is an \textit{associated prime} of $M$ if one
    of the two following equivalent conditions hold.  \begin{enumerate} \item
        There exists $x \in M$ such that $\mr{Ann}_A(x) = \mf{p}$; \item There
        is an injection $A / \mf{p} \hookrightarrow M$.  \end{enumerate}
    \end{defn}

We will denote the set of associated primes using the unfortunate notation
$\mr{Ass}_A(M)$. Then the set of primes $\mf{p}$ such that $M_{\mf{p}} \neq 0$
will be denoted $\mr{Supp}_A(M)$.

\begin{prop} Let $\mf{p}$ be a maximal element of $\{ \Ann(x) \mid x \in M, x
\neq 0 \}$. Then $\mf{p} \in \Ass_A(M)$.  \end{prop}

\begin{proof} We will show that such a maximal element is actually a prime
    ideal. Suppose $ab \in \mf{p}$. Then $\mf{p} = \Ann(x)$ for some nonzero
    $x$, so $b.x \neq 0$. Then $\Ann(x) \subset \Ann(bx) \neq A$. By
    maximality, $\Ann(x) = \Ann(bx)$. Because $abx = 0$, then $a \in \Ann(bx) =
    \mf{p}$.  \end{proof}

\begin{cor} Let $A$ be Noetherian.  \begin{enumerate} \item $M$ is nonzero if
and only if $\Ass_A(M)$ is nonempty.  \item The set of zero divisors for $M$ is
the union of the associated primes of $M$.  \end{enumerate} \end{cor}

\begin{proof} \begin{enumerate} \item If there is some associated prime, then
    clearly $M \neq 0$. In the other direction, the set of annihilators has a
    maximal element because $A$ is Noetherian, so there must be an associated
    prime.  \item Let $a \in \Ann(x)$ for some nonzero $x \in M$. Then $\Ann(x)
    \subset \mf{p}$ is contained in some associated prime (because it is
    contained in some maximal element), and thus every zero divisor is
    contained in an associated prime. The other direction is obvious. \qedhere
    \end{enumerate} \end{proof}

\begin{lem} Let $S \subset A$ be a multiplicative set and $M$ an $A$-module.
    Then \[ \Ass_A(S^{-1})M = \varphi^*(\Ass_{S^{-1}A}(S^{-1}M)). \] \end{lem}

\begin{proof} Let $\mf{p} \in \Ass_A(S^{-1}M)$. Then $\mf{p} = \Ann_A
    \frac{x}{1}$ for some $x \in M$, so $\mf{p} \cap S$ must be empty. Next, we
    see that the set $\qty{\Ann_A(sx) \mid s \in S}$ contains some maximal
    element $\mf{m}$ because $A$ is Noetherian. But then $\mf{m} = \Ann_A(s_0
    \cdot x) = \mf{p}$.

    On the other hand, if $a \in \mf{p}$, then $\frac{ax}{1} = 0$, which means
    $asx = 0$ for some $s \in S$. Then $a \in \Ann_A(sx) \subset \Ann(s_0 sx) =
    \Ann(s_0 x)$. Thus $\mf{p} \subset \Ann(s_0 x)$. Thus we have shown that \[
    \Ass_A(S^{-1}M) \subset \varphi^* \Ass_{S^{-1}A} (S^{-1}M). \] The other
inclusion is clear.  \end{proof}

\begin{thm} Let $A$ be Noetherian and $M$ and $A$-module. Then $\Ass_A(M)
\subset \Supp_A(M)$ and any minimal element of $\Supp_A(M)$ is inside
$\Ass_A(M)$.  \end{thm}

\begin{proof} Let $\mf{p} \in \Ass_A(M)$. Then $A/\mf{p}$ injects in $M$, so we
    have an injection $A_{\mf{p}} / \mf{p}A_{\mf{p}} \hookrightarrow
    M_{\mf{p}}$. Thus $\mf{p} \in \Supp_A(M)$.

    Now choose a minimal $\mf{p} \in \Supp_A(M)$. Thus $M_{\mf{p}}$ is
nontrivial, so there exists a prime ideal $\mf{q} \subset \mf{p}$ such that
$\mf{q} A_{\mf{p}} \in \Ass_{A_{\mf{p}}}(M_{\mf{p}})$. Thus $M_{\mf{q}} =
(M_{\mf{p}})_{\mf{q}A_{\mf{p}}}$ is nonzero, so $\mf{q} \in \Supp(M)$. By
minimality, $\mf{q} = \mf{p}$ and thus $\mf{p}A_{\mf{p}} \in \Ass_{A_{\mf{p}}}
(M_{\mf{p}})$. Therefore $\mf{p} \in \Ass_A(M)$.  \end{proof}

\begin{defn} If $\mf{p} \in \Ass_A(M)$, then $\mf{p}$ is not necessarily
minimal in the support of $M$. Then such a prime is called an \textit{embedded
prime}.  \end{defn}

\begin{prop} Let $A$ be Noetherian and $M$ a finitely-generated $A$-module.
    Then \begin{enumerate} \item There exists a chain \[ 0 = M_0 \subset M_1
    \subset \cdots \subset M_n = M \] such that $M_i / M_{i-1} \cong A /
    \mf{p}_i$ for $i = 1, \ldots, n$ and $\mf{p}_i \in \Supp_A(M)$.  \item
    Given such a sequence, we have $\Ass_A(M) \subset \{ \mf{p}_1, \ldots,
    \mf{p}_n \}$. In particular this set is finite.  \end{enumerate} \end{prop}

\begin{proof} \begin{enumerate} \item Suppose $M \neq 0$ Then choose $\mf{p}_1
    \in \Ass_A(M)$ and let $M_1$ be the image of $A / \mf{p}_1$ in $M$. Then if
    $M / M_1$ is nonzero, choose $\mf{p}_1 \in \Ass_A(M / M_1)$ and $M_2$
    defined analogously to $M_1$. This gives a sequence of submodules of $M$
    such that $A / \mf{p}_i \cong M_i / M_{i-1}$. Because $M$ is Noetherian,
    this sequence becomes stationary. Thus there exists $n$ such that $M_n =
    M$.  \item This is a consequence of the next lemma.  \end{enumerate}
\end{proof}

\begin{rmk} In general the support of a module is \textbf{not} finite.
\end{rmk}

\begin{lem} Assume we have an exact sequence of modules $0 \to M' \to M \to
M''$. Then $\Ass(M) \subset \Ass(M') \cup \Ass(M'')$.  \end{lem}

\begin{proof} If $\mf{p} \in \Ass(M)$, there exists $N \subset M$ such that $N
    \cong A / \mf{p}$. Then if $N \cap M' = 0$, $N \hookrightarrow M''$ and
    $\mf{p} \in \Ass(M'')$. If the intersection is nonzero, then there exists
    some nonzero $x \in N \cap M'$ such that $\Ann_A(x) = p$ because $A/\mf{p}$
    is a domain. Thus $\mf{p} \in \Ass(M')$.  \end{proof}

\begin{defn} We say that $M$ is \textit{coprimary} if $\Ass_A(M) =
\qty{\mf{p}}$.  \end{defn}

\begin{defn} Let $N \subset M$. Then we say that $N$ is
\textit{$\mf{p}$-primary} if $\Ass_A(M/N) = \{ \mf{p} \}$. Alternatively, we
say that $N$ \textit{belongs to} $\mf{p}$.  \end{defn}

\begin{lem} A module $M$ is coprimary if and only if $M$ is nonzero and any
zero divisor for $M$ is locally nilpotent (for all $x \in M$, there exists $n >
0$ such that $a^n . x = 0$).  \end{lem}

\begin{proof} Suppose that $M$ is coprimary. Now suppose that $a \in \mf{p}$
    and $x \in M$. Then $\Ass(Ax) = \{ \mf{p} \}$, so $\mf{p}$ is minimal in
    the support of $A_x$, which is $V(\Ann(x))$. Therefore, $\mf{p} =
    \sqrt{\Ann(x)}$. Thus for $a \in \mf{p}$, $a^n \in \Ann(x)$.
    
    In the other direction, let $\mf{p}$ be the set of locally nilpotent
elements with respect to $M$. This is clearly an ideal of $A$. Then let $\mf{q}
\in \Ass_A(M)$. Then $x \in M$, so $\mf{q} = \Ann(x)$. Therefore $\mf{p}
\subset \mf{q}$ because $\mf{q}$ is a prime ideal. However, $\mf{q}$ is
contained in the set of zero divisors, which is precisely $\mf{p}$.
\end{proof}

\begin{rmk} Let $I \subset A$ be an ideal. Then $\Ass_A(A/I) = \{ \mf{p} \}$ if
and only if the zero divisors of $A/I$ are locally nilpotent. This is
equivalent to $I$ being primary.  \end{rmk}

\begin{lem} \begin{enumerate} \item Let $Q_1, Q_2 \subset M$ be
    $\mf{p}$-primary submodules. Then $Q_1 \cap Q_2$ is $\mf{p}$-primary.
\item Let $N = Q_1 \cap \cdots \cap Q_r$ be an irredundant decomposition (i.e.
    $Q_i$ is $\mf{p}_i$-primary) for distinct $\mf{p}_i$. Then $\Ass_A(M/N) =
    \qty{\mf{p}_1, \ldots, \mf{p}_r}$.  \end{enumerate} \end{lem}

\begin{proof} \begin{enumerate} \item Note that $M / Q_1 \cap Q_2$ injects in
    $M / Q_1 \oplus M / Q_2$. The desired result follows from the previous
    lemma.  \item First, note that $M/N \hookrightarrow \bigoplus M / Q_i$.
    Then suppose $\Ass(M/N) = \{ \mf{p}_1, \ldots, \mf{p}_r \}$. Then we have
    an injection \[ \frac{Q_2 \cap \cdots \cap Q_r}{N} \hookrightarrow M/N \]
    and thus $\Ass((Q_1 \cap  \cdots \cap Q_r)/N)$ is contained in $\Ass(M/N)$.
    By the exact sequence \[ 0 \to N \to Q_2 \cap \cdots \cap Q_r \to M/Q_1, \]
    we see that $\Ass(Q_2, \cap \cdots \cap Q_r / N) = \{ \mf{p}_1 \}$.
    \qedhere \end{enumerate} \end{proof}

\begin{thm} Let $M$ be a module over a Noetherian ring $A$. Then for all
    $\mf{p} \in \Ass(M)$, there exists a $\mf{p}$-primary submodule $Q(\mf{p})
    \subset M$ such that \[ \bigcap_{\mf{p} \in \Ass(M)} Q(\mf{p}) = \{0 \}. \]
\end{thm}

\begin{proof} Fix $\mf{p} \in \Ass(M)$. Consider the set \[ \mc{S}_{\mf{p}} =
\qty{ N \subseteq M \mid \mf{o} \notin \Ass(N) }. \] This set is nonempty
because $0 \in \mc{S}_{\mf{p}}$. Next, if $N_{\lambda} \in \mc{S}_{\mf{p}}$ is
a chain, then the module $N = \bigcup N_{\lambda}$ is a submodule of $M$. In
addition, $\Ass(N) \subset \bigcup \Ass(N_{\lambda})$. This implies that
$\mc{S}_{\mf{p}}$ contains a maximal element by Zorn's lemma. Choose such a
maximal element $Q(\mf{p})$.

    We will show that $M/Q(\mf{p})$ is coprimary. By the exact sequence \[ 0
    \to Q(\mf{p}) \to M \to M/\mf{Q}(\mf{p}), \] if $\mf{p}' \in
    \Ass(M/Q(\mf{p}))$, then $\mf{p}' = \mf{p}$ because otherwise $A/ \mf{p}'$
    would inject in $M / Q(\mf{p})$ as $Q'/Q(\mf{p})$. Then $\Ass(Q') \subset
    \Ass(Q(\mf{p})) \cup \Ass(Q' / Q(\mf{p}))$, so $Q' \supsetneq Q(\mf{p})$,
    contradicting minimality. Thus $\Ass(M / Q(\mf{p})) = \qty{\mf{p}}$. 

    The second part of the claim follows immediately from the fact that
$\Ass\qty(\bigcap Q(\mf{p})) = \bigcap \Ass(Q(\mf{p})) = \emptyset$.
\end{proof}

\begin{cor} Let $M$ be an $A$-module of finite type. Then any $N \subset M$ has
    a primary decomposition \[ N = Q_1 \cap \cdots \cap Q_r \] such that
    \begin{enumerate} \item The $Q_i$ are $\mf{p}_i$-primary; \item No $Q_i$
        can be omitted; \item This decomposition is irredundant: $\Ass(M/N) =
        \qty{\mf{p}_1, \ldots, \mf{p}_r}$.  \end{enumerate} \end{cor}

\begin{proof} Apply the previous theorem to $M/N$. Because $M/N$ is of finite
type, $\Ass(M/N)$ is finite. Then use the previous lemma.  \end{proof}

\begin{exer} Let $A \xrightarrow{\varphi} B$ be a morphism of rings and let $M$
    be a $B$-module. Then prove that \[ \varphi^*(\Ass_B(M)) = \Ass_A(M) \]
where $\varphi^* \colon \Spec B \to \Spec A$ is the induced map of spaces.
\end{exer}

\chapter{Dimension Theory}% \label{cha:dimension_theory}

\section{Graded Rings and Modules}% \label{sec:graded_rings_and_modules}

Let $A = \bigoplus_{n \geq 0} A_n$ be a graded ring. This means that $A_n \cdot
A_m \subset A_{n+m}$. Then an $A$-module $M$ is a graded module if \[ M =
\bigoplus_{n \in \Z} M_n \] such that $A_n \cdot M_m \subset M_{n+m}$. We will
call $M_m$ the homogeneous elements of degree $m$ on $M$.

Now let $N \subset M$ be a submodule. We say that $N$ is a graded submodule if
$N = \bigoplus N \cap M_m$. $N$ is also called homogeneous. A homogeneous
element of $M$ is an element of some $M_m$. Being a graded submodule is the
same as every element being a sum of homogeneous elements.

\begin{lem} The following are equivalent: \begin{enumerate} \item $N$ is a
homogeneous submodule.  \item $N$ is generated by homogeneous elements.  \item
If $x = x_r + \cdots + x_n \in N$ for $x_i \in M_i$, then for all $i$, $x_i \in
N$.  \end{enumerate} Moreover, if $N \subset M$ is homogeneous, then so is
$M/N$, and \[ M/N = \bigoplus_m M_m / N_m. \] \end{lem}

\begin{proof} The proof is left as an exercise to the reader.  \end{proof}

\begin{exm} Let $k$ be any ring. Then the ring $A = k[x_1, \ldots, x_r]$ is a
    graded ring where the grading is by the degree of each monomial. In
    particular, $A_0 = k$. Then an ideal $I \subset A$ is graded if $I =
    \bigoplus_n I_n$ where $I_n = I \cap A_n$. In addition, $A/I$ is a graded
    ring.  \end{exm}

\begin{prop} Let $A$ be a Noetherian graded ring and $M$ a graded $A$-module.
    Then \begin{enumerate} \item If $\mf{p} \in \Ass(M)$, then $\mf{p}$ is a
        graded ideal of $A$ and there exists a homogeneous $x \in M$ such that
        $\mf{p} = \Ann(x)$.  \item One can choose a $\mf{p}$-primary graded
submodule $Q(\mf{p})$ such that $0 = \bigcap_{\mf{p} \in \Ass(M)} Q(\mf{p})$.
\end{enumerate} \end{prop}

\begin{proof} Let $x \in M$ and set $\mf{p} = \Ann(M)$. Then write $x = x_e +
    x_{e-1} + \cdots + x_0$. Then for $f \in \mf{p}$, write $f = f_r + \cdots +
    f_s$. If $fx = 0$, then we can write \[ 0 = fx + f_r x_e + (f_{r-1} x_e +
    f_r x_{e-1}) + \cdots \] and deduce that $0 = f_r x_e = f_r^2 x_{e-1} =
    \cdots$. Then $f_r^e \in \mf{p}$, so $f_r \in \mf{p}$. By induction, all
    $f_i \in \mf{o}$, so $\mf{p}$ is graded.

    The proof of the second part is simply the following lemma.  \end{proof}

\begin{lem} Let $\mf{p}$ be a graded prime ideal and $Q \subset M$ such that
$Q$ is $\mf{p}$-primary. Let $Q' \subset Q$ be the submodule of $Q$ generated
by the homogeneous elements of $Q$. Then $Q'$ is $\mf{p}$-primary.  \end{lem}

\begin{proof} This will be proved later.  \end{proof}

We will now discuss filtrations of rings. A \textit{filtration} is a sequence
of subgroups \[ A = J_0 \supset J_1 \supset J_2 \cdots \] such that $J_n \cdot
J_m = J_{n+m}$. If we set \[ A' = \bigoplus_{n=0}^{\infty} J_n / J_{n+1}, \]
then $A'$ is a graded ring.

The basic example is $J_m = I^m$ for some fixed ideal $I \subset A$. in this
case, the filtration is called the $I$-adic filtration.

\begin{lem} Let $A$ be a Noetherian ring and set $I \subset A$. Then \[
\operatorname{gr}^I A = \bigoplus_{n=0}^{\infty} I^n / I^{n+1} \] is a
Noetherian graded ring.  \end{lem}

\begin{proof} Because $I$ is finitely-generated, then $I / I^2$ is a
    finitely-generated $A/I$-module. Thus $\operatorname{gr}^I(A)$ is a
    finitely-generated $A/I$-algebra. If $x_1, \ldots, x_r$ is a set of
    generators of $I$, then \[ A/I[x_1, \ldots, x_r] \to \operatorname{gr}^I A
    \] is surjective, so because $A/I$ is Noetherian, so is $A/I[x_1, \ldots,
x_r]$ and thus so is $\operatorname{gr}^IA$.  \end{proof}

Let $A$ be an Artinian ring and $B = A[x_1, \ldots, x_r]$ be a graded ring.
Then let $M$ be a finitely-generated graded $B$-module. Each graded piece $M_n$
is an $A$-module, so write $F_M(n) = \ell_A(M_n)$. Because $M$ is finitely
generated, we have a map \[ \bigoplus_{i=1}^r B(d_i) \twoheadrightarrow M. \]
Here, $B(d_i) = B$ as a $B$-module with the gradation $B(d)_n = B_{n-d}$. Thus
$M$ is generated by homogeneous elements $x_{d_i}$ of degree $d_i$. This gives
us the map \begin{align*} \bigoplus_{i=1}^r B_{m-d_i} \hookrightarrow M_m \\
    (f_i) \mapsto \sum_{i=1}^r f_i x_{d_i} \end{align*} and thus $\ell_A(M_m)
    \leq \sum_{i=1}^r \ell_A(B_{m-d_i})$. But then $B_m$ is a free $A$-module,
    and thus \[ \ell_A(B_m) \leq \binom{r+m-1}{m-1} \ell(A). \]

\begin{thm} Let $A, B,M$ be as above. Then there is a polynomial $f_M(x) \in
    \Q[x]$ such that \[ \ell_A(M_n) = f_M(n) \] for $n \gg 0$. This is called
    the \textit{Hilbert-Samuel polynomial} for $M$. The degree of this
    polynomial will give the first definition for the dimension of $M$.
\end{thm}

\begin{proof} Say that $M$ satisfies the property $P(M)$ if there exists $f \in
    \Q(x)$ such that $\ell(M_n) = f(n)$ for $n \gg 0$.  \begin{enumerate} \item
        First, we will show that if $N_1, N_2 \subset M$ and $P(M/N_1),
        P(M/N)2$ hold, then $P(M/N_1 \cap N_2)$ holds.  \item Second, if $N$ is
irreducible, then $P(M/N)$ holds.  \end{enumerate} If we prove these two
things, then we simply use a primary decomposition of $0 \subset M$. Then the
second statement implies that $P(M/N_i)$ holds and then we simply repeatedly
apply the first to see that $P(M)$ holds.  \begin{enumerate} \item Suppose $N =
    N_1 \cap N_2$ with $N, N_1, N_2$ graded. We then have an exact sequence \[
    0 \to N_1 / (N_1 \cap N_2) \to M / (N_1 \cap N_2) \to M/N_1 \to 0 \] and we
    know that $N_1 / (N_1 \cap N_2) = (N_1 + N_2) / N_1$ is graded. Then we see
    that $F_{M/(N_1 \cap N_2)} = F_{M/N} + F_{(N_1 + N_2)/N_1}$, so we only
    need to prove that $F_{N_1 + N_2 / N_1}$ exists. But then $F_{N_1 + N_2 /
    N_1} = F_{M / N_2} - F_{M/N_1 + N_2}$ and because $P(M/N_1 + N_2)$ holds,
    so does $P(M/N_1 \cap N_2)$.  \item Let $N$ be irreducible. We know that
    $M' = M/N$ is coprimary, so $N$ is $\mf{p}$-primary for some prime ideal
    $\mf{p} \subset B$. Write $I = (x_1, \ldots, x_m)$. If $I \subset \mf{p}$,
    then $M'_m = 0$ for $n \gg 0$. Indeed, if $d$ is the maximal degree of a
    system of generators of $M'$, then $M'_{n+d} = I^n \cdot M'_d$. On the
    other hand, because $M'$ is $\mf{p}$-primary, then elements in $\mf{p}$ are
    locally nilpotent. Thus there exists $k \gg 0$ such that $p^k \cdot M'_d =
    0$ and thus $M'_{n+d} = 0$ for $n \geq k$. Thus $F_{M/N}$ exists and is
    identically zero.

            In the second case, $I \not\subset \mf{p}$. Then suppose that $x_1
            \notin \mf{p}$. Thus $x_1$ is not a zero divisor for $M'$. Thus, we
            have an exact sequence \[ 0 \to M' \to x_1 \to M' \to M'/x_1M' \to
                0 \] which then gives \[ 0 \to (M/N)_{n-1} \to (M/N)_n \to
            (M/N+x_1 M)_n \to 0 \] when restricting to a single graded piece.
            Thus $N \subsetneq N + x_1 M$. This implies that $f_{M/N + x_1 M}$
            exists because above, we proved that if $P(M/N')$ holds for any $N'
            \supsetneq N$ implies that $P(M/N)$ holds. Then for $n \geq n_0$,
            we have $\ell((M/N)_m) - \ell((M/N)_{n-1}) = f_{M/N + x_1 M}(n)$.
            This implies that \[ \ell((M/N)_m) = f(n) + f(n-1) + \cdots +
        \ell(M/N)_{n_0}. \] Then $f(n) + \cdots + f(n_0) = g(n)$ for some
polynomial $g$ of degree $\deg f + 1$ and then $f_{M/N} = g +
\ell((M/N)_{n_0})$.\qedhere \end{enumerate} \end{proof}

Now let $A \supset I$ and $M$ be an $A$-module with filtration \[ M_0 = M
    \supset M_1 \supset \cdots \supset M_n \supset \cdots \] We say that the
    filtration is \begin{enumerate} \item \textit{$I$-admissible} if $I M_n
        \subset M_{n+1}$ for all $n \gg 0$; \item \textit{$I$-acic} if $I M_n =
M_{n+1}$ for all $n \geq 0$; \item \textit{essentially $I$-adic} if $I M_n =
M_{n+1}$ for $n \gg 0$.  \end{enumerate}

\begin{rmk} A filtration on $M$ defines a topology on $M$ so that $M$ is a
    topological group. Here, a system of neighborhoods of $0$ is $(M_n)_{n \gg
    0}$. If $\bigcap M_n = 0$, then the topology is Hausdorff. If the
    filtration is essentially $I$-adic, then the topology is called the
    $I$-adic topology.  \end{rmk}

\begin{lem} Let $A$ and $I$ be as before. Let $M$ be an $A$-module with an
    admissible filtration. Let $A' = \bigoplus_{n=1}^{\infty} I^n x^n \subset
    A[x]$ and \[ M' = \bigoplus M_n \otimes_A Ax^n = \bigoplus M_n x^n. \]
    \begin{enumerate} \item $M'$ is a $A'$-module.  \item The filtration is
    essentially $I$-adic if and only if $M'$ is a finitely-generate
    $A'$-module.  \end{enumerate} \end{lem}

\begin{proof} \begin{enumerate} \item This is trivial.  \item Note that $M'$ is
    a graded $A'$-module. If $M'$ is finitely generated, then write $M' = A'
    m_1 + \cdots + A' m_r$. Then we see that $ M'_n = I x M'_{n-1}$ for $n >
    \max \qty{ \deg m_i }$. Thus $M_n$ is essentially $I$-adic.

            Conversely, if $M_n = I^{n - n_0} M_{n_0}$ for $n \geq n_0$, then,
    then it is clear that $M'$ is generated by $M_{n_0} x^{n_0} + \cdots + M_1
    x + M_0$ and is thus finitely generated. \qedhere \end{enumerate}
\end{proof}

\begin{thm}[Artin-Rees] Let $A$ be a Noetherian ring and $I \subset A$. Then
    let $M$ be a finitely-generated $A$-module and $N \subset M$ be a
    submodule. Then there exists $r > 0$ such that \[ I^n M \cap N = I^{n-r}
    (I^r M \cap N) \] for all $n \geq r$.  \end{thm}

\begin{proof} Let $M_n = I^n M$ be the $I$-adic filtration. Then $N_n = I^n M
    \cap N$ is $I$-admissible. Then both $N' \subset M'$ are both $A'$-modules.
    We know that $A'$ is Noetherian, so because $M_n$, $M'$ is finitely
    generated. Thus $N'$ is also Noetherian, so it is also finitely-generated.
    This implies that $N_m$ is essentially $I$-adic, as desired.  \end{proof}

\begin{rmk} This theorem is saying that the filtration $I^n M \cap N$ is
essentially $I$-adic.  \end{rmk}

\begin{cor}[Krull Intersection Theorem] Let $A, I, M$ as above.
\begin{enumerate} \item If $N = \bigcap_{n=0}^{\infty } I^n M$, then $IN = N$.
\item If $I \subset \mr{rad}(A)$, then $\bigcap_{n=0}^{\infty} I^n M = 0$.
\end{enumerate} \end{cor}

\begin{proof} \begin{enumerate} \item Note that $N \subset M$. Then apply the
Artin-Rees theorem to $N = I^n M \cap N$.  \item Apply Nakayama's lemma.
\qedhere \end{enumerate} \end{proof}

\begin{cor} Let $A$ be a Noetherian domain and let $I \subset A$ be a proper
ideal. Then $\bigcap_n I^n = 0$.  \end{cor}

\begin{proof} Let $N = \bigcap_{n=0}^{\infty} I^n$. By the previous corollary,
    $IN = N$. Then $N$ is finitely generated because $A$ is Noetherian. Thus
    there exists $x \in I$ such that $(1+x)N = 0$, which implies that $N = 0$
    because $A$ is a domain.  \end{proof}

\begin{exer} Let $A$ be Noetherian and $M$ a finitely-generated $A$-module.
    Then let $I, J$ be generated by $M$-regular elements. Then there exists $r
    > 0$ such that $(I^n M : J) = I^{n-r} (I^r M : J)$. Here, $(N:J) = \qty{ m
    \in M \mid Jm \subset N }$.  \end{exer}

\section{Other Notions of Dimension}% \label{sec:other_notions_of_dimension}

Let $A$ be a ring. Then we define the \textit{Krull dimension} \[ \dim A = \sup
    \qty{ \mr{ht}(\mf{p}) \mid \mf{p} \in \Spec A }. \] Then for any ideal $I
    \subset A$, define the height of $I$ to be \[ \mr{ht}(I) = \inf
    \qty{\mr{ht}(\mf{p}) \mid \mf{p} \supset I}. \]

\begin{prop} For any ideal $I$, we have $\dim ( A/I ) + \mr{ht}(I) \leq \dim
A$.  \end{prop}

\begin{proof} Consider a chain \[ \mf{p}_{d'} \supsetneq \cdots \supsetneq
\mf{p}_0 \supset I \] where $d' = \dim(A/I)$. Then we see that $\dim A \geq d'
+ \mr{ht}(\mf{p}_0) \geq d' + \mr{ht}(I)$, as desired.  \end{proof}

Then if $M$ is an $A$-module, define $\dim M = \dim ( A / \Ann_A M )$.

\begin{prop} Assume that $A$ is Noetherian and $M$ is a finite $A$-module. Then
the following are equivalent: \begin{enumerate} \item $M$ is of finite length.
\item $A / \Ann_A M$ is Artinian.  \item $\dim M = 0$.  \end{enumerate}
\end{prop}

\begin{proof} Clearly conditions 2 and 3 are equivalent. Then $M$ is a quotient
    of $(A/\Ann_A M)^r$, so 2 implies 1. Thus we need to prove that 1 implies
    3.

    Assume that $\ell(M) < \infty$. If we write $A' = A/\Ann_A(M)$, then $M$ is
a finite $A'$-module. If $\dim A' > 0$, then there exists $\mf{p} \subset A'$
that is minimal but not maximal. Then because $\Ann_{A'}(M) = 0$, we have
$\mf{p} \in V((0)) = \Supp(M)$. But then $\mf{p} \in \Ass_{A'}(M)$ and thus we
have an embedding $A' / \mf{p} \hookrightarrow M$. But then $\dim M' / \mf{p} >
0$, so $\ell(A' / \mf{p}) = \infty$ and thus $\ell(M) = \infty$.  \end{proof}

Now let $A$ be a semilocal ring. Let $\mf{m} = \operatorname{Rad} A$. Then an
ideal $I \subset A$ is called an \textit{ideal of definition} of $A$ if there
exists $s > 0$ such that $\mf{m}^s \subset I \subset \mf{m}$.

\begin{rmk} $I$ is an ideal of definition if and only if $A/I$ is Artinian.
\end{rmk}

Let $A^* = \mr{gr}^I(A) = \bigoplus_{n=0}^{\infty} I^n / I^{n+1}$ be the graded
ring with respect to the $I$-adic filtration and let $M^* = \mr{gr}^I(M)$ be
the corresponding graded $A^*$-module. Then if $I = (x_1, \ldots, x_i) \subset
A$, define \[ B = A/I[x_1, \ldots, x_r]. \] Then we have a map $B
\twoheadrightarrow A^*$, so $M^*$ is a $B$-module. Now define $\chi(M,I,n)
\coloneqq \ell(M / I^nM)$. If $M$ is a finite $A$-module, then $M/I^n M$ is of
finite length (because $A/I^n$ is Artinian) and thus \[ \ell(M/I^nM) =
\ell(M/IM) + \ell(IM/I^2M) + \cdots + \ell(I^{n-1}M / I^n M). \] Then if
$\ell(I^sM / I^{s+1}M)$ is a polynomial in $s$ of degree at most $r-1$ for $s
\gg 0$, then $\ell(M/I^nM)$ is a polynomial of degree at most $r$ for $n \gg
0$.

Now if $J$ is another ideal of definition, then there exists $s$ such that $J^s
\subset I$ and thus $\chi (M, J, ns) \geq \chi(M,I,n)$. Therefore \[
d^{\bullet} \chi(M,J,n) \geq d^{\bullet} \chi(M,I,n) \] and so the degree of
$\chi(M,I,n)$ is independent of $I$. Denote this degree by $d(M)$. We know that
$d(M) \geq r$, which is the number of generators of $I$.

\begin{lem} Assume we have an exact sequence of finite $A$-modules \[ 0 \to M'
\to M \to M'' \to 0. \] Then $d(M) = \max \qty{ d(M'), d(M'') }$ and $\chi(M,
I,n) - \chi(M', I, n) - \chi(M'', I, n)$ is a polynomial of degree strictly
less that $d(M'')$.  \end{lem}

\begin{proof} For each $n$ we have an exact sequence \[ 0 \to \frac{M'}{I^n M
\cap M'} \to \frac{M}{I^nM} \to \frac{M''}{I^n M''} \to 0. \] Then $\Delta
\coloneqq \chi(M,I,n) - \chi(M'', I, n) = \ell(M/I^nM) - \ell(M'' / I^n M) =
\ell(M' / I^n M \cap M')$. By Artin-Rees, there exists $r$ such that $M' \cap
I^n M = I^{n-r} \cdot (M' \cap I^r M)$. But then \[ \chi(M',I, n-r) \leq \Delta
\leq \chi(M', I, n) \] because $\chi(M', I,n) - \chi'(M', I, n-r)$ has degree
strictly less than $d(M')$, and the desired result follows.  \end{proof}

\begin{lem} Let $A$ be a Noetherian semilocal ring. Then $d(A) \geq \dim A$. In
particular, $\dim A < \infty$.  \end{lem}

\begin{proof} We will induct on $d(A)$. If $d(A) = 0$, then $\mf{m}^n =
    \mf{m}^{n+1}$ for $n \gg 0$. By Nakayama, we see that $\mf{m}^n = 0$, so
    $\ell(A) < \infty$ and thus $A$ is Artinian.

    Assume that $d(A) > 0$ and $\dim A > 0$. Let \[ \mf{p}_0 \supset \mf{p}_1
    \supset \cdots \supset \mf{p}_{\ell} = \mf{p} \] be a chain of prime ideals
    of length $\ell > 0$. Choose $x \in \mf{p}_{\ell - 1} \setminus
    \mf{p}_{\ell}$. Then $\dim (A / \mf{p} + xA) \geq \ell - 1$. Because we
    have the exact sequence \[ 0 \to A/\mf{p} \xrightarrow{\times x} A/\mf{p}
    \to A/\mf{p} + xA \to 0, \] we have $d(A/\mf{p}) = \max (d(A/\mf{p}),
    d(A/\mf{p}+xA))$ and that $ \chi(A/\mf{p}+xA, I,n)$ has degree less than
    $d(A/\mf{p})$. Therefore \[ d(A/\mf{p}+xA) < d(A/\mf{p}) \leq d(A). \] By
    induction, $\dim(A/\mf{p}+xA) \leq d(A/\mf{p}+xA)$ and thus $\ell - 1 \leq
    d(A/\mf{p}+xA) \leq d(A) - 1$. This holds for any chain of ideals, so $\dim
    A \leq d(A)$.  \end{proof}

\begin{cor} If $A$ is Noetherian and $\mf{p} \in \Spec A$, then
$\mf{ht}(\mf{p}) < \infty$.  \end{cor}

\begin{proof} $\mr{ht}(\mf{p}) = \dim A_{\mf{p}} < \infty$ by the previous
lemma.  \end{proof}

\begin{lem} Let $A$ be a Noetherian semilocal ring and $M$ a finite $A$-module.
Choose $x \in \mr{Rad}(A)$. Then $d(M) \geq d(M/xM) \geq d(M) - 1$.  \end{lem}

\begin{proof} Let $I \subset A$ be an ideal of definition such that $x \in I$.
    Then \[ \chi(M/xM,I,n) = \ell(M/xM + I^nM) = \ell(M / I^nM) -
    \ell\qty(\frac{xM + I^nM}{I^nM}). \] Then because $x \in I$, we see that
    $I^{n-1}M \subset (I^n M : x)$, so \[ \ell \qty(\frac{xM + i^nM}{I^nM})
    \leq \ell(M / I^{n-1}M). \] This implies that $\chi(M / xM, I, n) \geq
\chi(M, I, m) - \chi(M, I, n-1)$, so $d(M/xM) \geq d(M) - 1$.  \end{proof}

\begin{lem} Let $A$ and $M$ be as before. Let $r = \dim M > 0$. Then there
exists $x_1, \ldots, x_r \in \mr{Rad}(A)$ such that $\ell(M / x_1 M + \cdots +
x_r M) < \infty$.  \end{lem}

\begin{proof} Let $\mf{p}_1, \ldots, \mf{p}_t$ be the minimal prime ideals
    containing $\Ann_A(M)$ such that $\dim(A/\mf{p}_i) = r$. Because $r > 0$,
    then the $\mf{p}_i$ are not maximal and therefore $\mr{Rad}(A) \not\subset
    \mf{p}_i$. In particular, it is not contained in $\bigcup \mf{p}_i$. Choose
    $x_1 \in \mr{Rad}(A) \setminus \bigcup \mf{p}_i$.

    If $\mf{q} \supset \Ann(M / x_1 X) \supset \Ann(M) + x_1A$ is prime and
    minimal, then $\mf{q} \notin \mf{p}_i$ because $x_1 \notin \mf{p}_i$. This
    implies $\dim A/\mf{q} \leq r-1$. By induction, we can then find $x_2,
    \ldots, x_r$ such that \[ \ell(\ol{M} / x_2 \ol{M} + \cdots + x_r \ol{M}) <
    \infty, \] where $\ol{M} = M / x_1M$.  \end{proof}

\begin{thm} Let $A$ be semilocal and $M$ a finite $A$-module. Then $d(M) = \dim
M$ is the smallest integer $r$ such that there exists $x_1, \ldots, x_r \in
\mr{Rad}(A)$ such that $\ell(M / x_1 M + \cdots + x_r M) < \infty$.  \end{thm}

\begin{proof} Choose $x_1, \ldots, x_r \in \mf{m} = \mr{Rad}(A)$. If $\ell(M /
    x_1 M + \cdots + x_r M) < \infty$, then we know that $d(M / x_1 M + \cdots
    | x_r M) \geq d(M) - r$. Then because $M / x_1 M + \cdots + x_r M$ has
    finite length, its dimsnion is zero and thus $r \geq d(M)$. Then let $r_0$
    be the smallest such integer. By the previous lemma, we deduce that $\dim M
    \geq r_0 \geq d(M)$.

    We will show that $d(M) \geq \dim M$. Consider a sequence \[ M = M_0
    \supset M_1 \supset \cdots \supset M_{n+1} = 0 \] such that $M_i / M_{i+1}
    \cong A / \mf{p}_i$ for some prime ideals $\mf{p}_i$. Then $\Ass(M) \subset
    \qty{ \mf{p}_0, \ldots, \mf{p}_n } \subset \operatorname{Supp} M$ are the
    minimal primes containing $\Ann_A(M)$, so we see that \[ d(M) = \max \qty{d
    ( A.\mf{p}_i )} \geq \max \qty{\dim (A / \mf{p}_i)} = \dim(A / \Ann_A M).
\qedhere \] \end{proof}

\begin{rmk} If $M = A$, then $d(M)$ is the smallest integer $r$ such that there
exists $x_1, \ldots, x_r \in \mr{Rad}(A)$ such that $(x_1, \ldots, x_r)$ is an
ideal of definition.  \end{rmk}

\begin{cor} If $A$ is Noetherian and $I = (x_1, \ldots, x_r) \subset A$, then
any minimal prime ideal $\mf{p}$ containing $I$ has height at most $r$. In
particular, $\mr{ht}(I) \leq r$.  \end{cor}

\begin{proof} First, note that $A_{\mf{p}} / IA_{\mf{p}}$ is Artinian because
    the image of $\mf{p}A_{\mf{p}}$ is both maximal and minimal. Then
    $\ell(A_{\mf{p}} / x_1 A_{\mf{p}} + \cdots + x_r A_{\mf{p}}) < \infty$ and
    thus $\mr{ht}(\mf{p}) = \dim(A_{\mf{p}}) \leq r$.  \end{proof}

Now let $M$ be a finitely generated $A$-module. Define $\wh{M} = \lim_{\gets} M
/ I^n M$ for an ideal of definition $I$. We call $\wh{M}$ the $I$-adic
completion of $M$.

\begin{cor} $\dim \wh{M} = \dim M$.  \end{cor}

\begin{proof} We know that $\wh{M} / I^n \wh{M} = M / I^nM$. Thus the two
modules have the same Hilbert-Samuel polynomial.  \end{proof}

\begin{cor} Let $A$ be Noetherian with $\mf{p} \in \Spec A$. Let $n$ be an
    integer. The following are equivalent: \begin{enumerate} \item
        $\mr{ht}(\mf{p}) \leq n$.  \item There exits $I$ generated by $n$
        elements such that $\mf{p}$ is minimal in $V(I)$.  \end{enumerate}
    \end{cor}

\begin{proof} \begin{description} \item[1 implies 2] Suppose that
    $\mr{ht}(\mf{p}) \leq n$. Then there exists an ideal of definition $J$ of
    $A_{\mf{p}}$ generated by $n$ elements. If $J = \qty( \frac{x_1}{s},
    \ldots, \frac{x_n}{s} )$, then $I = \qty(x_1, \ldots, x_n) \subset \mf{p}$
    and $\mf{p}$ is minimal containing $I$.  \item[2 implies 1] Let $I = (x_1,
    \ldots, x_n)$ such that $\mf{p} \supset I$ is minimal. Therefore
    $A_{\mf{p}} / IA_{\mf{p}}$ is Artinian, so it has finite length. Thus $\dim
    A_{\mf{p}} \leq n$, as desired. \qedhere \end{description} \end{proof}

\begin{defn} A \textit{system of parameters} for $M$ is a set of elements $x_1,
\ldots, x_s \in \mr{Rad}(A)$ such that \begin{itemize} \item $\ell(M / x_1 M +
\cdots + x_s M) < \infty$; \item $s = \dim M$.  \end{itemize} \end{defn}

\begin{prop} Let $x_1, \ldots, x_r \in \mr{Rad}(A)$. Then $\dim (M / (x_1,
\ldots, x_i)M) \geq \dim M - r$ and we have equality if and only if $x_1,
\ldots, x_r$ belong to a system of parameters for $M$.  \end{prop}

\begin{proof} By induction $d(M / xM) \geq d(M) - 1$ for any $x \in \Rad (A)$.
    Then we know that \[ d(M / (x_1, \ldots, x_r)M) \geq d(M) - r = \dim M - r.
    \]

    Then assume that we have equality. Let $y_1, \ldots, y_p$ be a system of
    parameters for $M / (x_1, \ldots, x_r)M$. Then $\dim(M / (x_1, \ldots,
    x_r)M) = p = \dim M - r$. However, if $\ol{M} = M / (x_1, \ldots, x_r)$,
    then \[ \ell(\ol{M} / (y_1, \ldots, y_p) \ol{M}) = \ell(M / (y_1, \ldots,
    y_p, x_1, \ldots, x_r)M) < \infty \] and thus $y_1, \ldots, y_p, x_1,
    \ldots, x_r$ is a system of parameters for $M$.

    Conversely suppose that $x_1, \ldots, x_r, y_1, \ldots, y_p$ is a system of
    parameters for $M$. Then \[ \dim(M / (x_1, \ldots, x_r)M) \geq \dim M - r =
        p, \] but we have equality because \[ 0 = d(M / x_1, \ldots, x_r, y_1,
    \ldots, y_p) \geq d(M / (x_1, \ldots, x_r)M) - p, \] and so $p \geq d(M /
(x_1, \ldots, x_r)M)$.  \end{proof}

Now we turn to the case of local Noetherian rings $A$ with maximal ideal
$\mf{m}$ and residue field $k$. Then if $d = \dim A$, any ideal of definition
has at least $d$ generators. Then let $x_1, \ldots, x_d \in \mf{m}$ such that
$\ell(A / (x_1, \ldots, x_d)) < \infty$. Thus $I = (x_1, \ldots, x_d)$ is an
ideal of definition and $(x_1, \ldots, x_d)$ is a system of parameters of $A$.

\begin{defn} A local ring $A$ is a \textit{regular local ring} if there is a
system of parameters generating the maximal ideal of $A$. Such a system is
called a \textit{regular system of parameters}.  \end{defn}

Note that $\dim A \leq \dim_k \mf{m} / \mf{m}^2$ and that $A$ is regular if and
only if $\dim A = \dim_k \mf{m} / \mf{m}^2$.

\begin{prop} Let $(A, \mf{m}, k)$ be a local Noetherian ring. Let $(x_1,
    \ldots, x_d)$ be a system of parameters of $A$. Then $\dim(A / (x_1,
    \ldots, x_i)A) = d-i$ and the image of $(x_{i+1}, \ldots, x_d)$ in $A /
    (x_1, \ldots, x_i)A$ is a system of parameters of this quotient.
\end{prop}

\section{Dimension in the Relative Setting}%
\label{sec:dimension_in_the_relative_setting}

Consider a morphism $A \xrightarrow{\varphi} B$. We have the pullback
$\varphi^* \colon \Spec B \to \Spec A$. For $\mf{p} \in \Spec A$, we have a
bijection between $(\varphi^*)^{-1}(\mf{p})$ and $\Spec(B_{\mf{p}} /
\mf{p}B_{\mf{p}})$. This latter ring is isomorphic to $B \otimes k(\mf{p})$,
where $k (\mf{p})$ is the residue field of $A_{\mf{p}}$.

\begin{thm} Let $\mf{P} \in \Spec B$ lie over $\mf{p}$. Then \begin{enumerate}
    \item $\mr{ht}(\mf{P}) \leq \mr{ht}(\mf{p}) + \mr{ht}(\mf{P} /
        \mf{p}\mf{P})$. Equivalently, $\dim B_{\mf{P}} \leq \dim A_{\mf{p}} +
        \dim (B_{\mf{P}} / \mf{p}B_{\mf{P}})$.  \item Equality holds is
        equivalent to the going-down property for $\varphi$ and in particular
        if $\varphi$ is flat.  \item If $\varphi^*$ is surjective and the
        going-down property holds, then $\dim B \geq \dim A$ and $\mr{ht}(I) =
        \mr{ht}(IB)$ for $I \subset A$.  \end{enumerate} \end{thm}

\begin{proof} \begin{enumerate} \item Set $A = A_{\mf{p}}, B = B_{\mf{P}}$. We
    need to prove that $\dim B \leq \dim A + \dim B / \mf{p}B$, where $\mf{p}$
    is the maximal ideal of $A$. Write $r = \dim A$ and let $x_1, \ldots, x_r$
    be a system of parameters for $A$. Then $I = (x_1, \ldots, x_r)$ is an
    ideal of definition, so $\mf{p}^n \subset I \subset \mf{p}$ for some $n$.
    Thus $\mf{p}^n B \subset IB \subset \mf{p}B$ and all of these ideals have
    the same nilradical. Therefore \[ \dim B / OB = \dim B / \mf{p}^n B = \dim
    B / \mf{p}B = s \] for some integer $s$. If $y_1, \ldots, y_s$ is a system
    of paramaters for $B / IB$, then $x_1, \ldots, x_r, y_1, \ldots, y_s$
    generate an ideal of definition for $B$, so $r + s \geq \dim B$.  \item Let
    $\mf{P} = \mf{P}_0 \supsetneq \mf{P}_1 \supsetneq \cdots \supsetneq
    \mf{P}_s$ be a chain of ideals of $B / \mf{p}B$ of length $s = \dim B /
    \mf{p}B$. Then for $i = 0, \ldots, s$ we know $\mf{p} \subset
    \varphi^*(\mf{P}_i)$ and thus $\varphi^*(\mf{P}_i) = \mf{p}$ for all $i$.
    Now by the Going Down property we can find \[ \mf{P}_s \supsetneq \cdots
        \supsetneq \mf{P}_{r+s} \] such that $\mf{p}_i =
        \varphi^{-1}(\mf{P}_{s+i})$. Thus we have \[ \mf{p} = \mf{p}_0
        \supsetneq \mf{p}_1 \supsetneq \cdots \supsetneq \mf{p}_r \] where $r =
        \dim A$. This gives us the chain \[ \mf{P}_0 \supsetneq \cdots
        \supsetneq \mf{P}_{r+s} \] and thus $\dim B \geq r+s$.  \item The first
        inequality follows from 2. Note that $\dim B = \dim A + \dim(B /
        \mf{p}B) \geq \dim A$.

            To prove the equality, let $\mf{P} \in V(IB)$ be minimal such that
            $\mr{ht}(\mf{P}) = \mr{ht}(IB)$. Let $\mf{p} = \varphi^*(\mf{P})$.
            Then $\mf{p} \supset I$ and $\mf{P} / \mf{p}B$ is minimal, so
            $\mr{ht}(\mf{P}/\mf{p}B) = 0$. This tells us that $\dim B_{\mf{P}}
            = \dim A_{\mf{p}}$ and thus $\mr{ht}(\mf{P}) = \mr{ht}(\mf{p})$.
            Thus $\mr{ht}(\mf{P}) \geq \mr{ht}(I)$.

            Conversely, let $\mf{p} \supset I$ be minimal with $\mr{ht}(\mf{p})
            = \mr{ht}(I)$. Let $\mf{P} \in \Spec B$ such that
            $\varphi^*(\mf{P}) = \mf{p}$. Then $\mf{P} \supset \mf{p}B \supset
            IB$ and so we may suppose it is minimal for this property. Then we
            see that \[ \mr{ht}(IB) \leq \mr{ht}(\mf{P}) = \mr{ht}(\mf{p}) =
            \mr{ht}(I) \] as desired. \qedhere \end{enumerate} \end{proof}

\begin{cor} Let $B \supset A$ be Noetherian rings such that $B$ is integral
    over $A$.  \begin{enumerate} \item $\dim A = \dim B$; \item For all $\mf{P}
        \in \Spec B$, $\mr{ht}(\mf{P}) \leq \mr{ht}(\mf{P} \cap A)$.  \item If
        the going-down property holds, then for any ideal $J \subset B$, have
        $\mr{ht}(J) = \mr{ht}(J \cap A)$.  \end{enumerate} \end{cor}

\begin{proof} The proof of this is left as an exercise to the reader.
\end{proof}

\begin{exer} Let $A \xrightarrow{\phi} B$ be a morphism of rings and assume
that going-down holds for $\phi$. Let $\mf{p} \supset \mf{q}$ be prime ideals
of $A$. Prove that $\dim(B \otimes k(\mf{p})) \geq \dim(B \otimes k(\mf{q}))$.
\end{exer}

Now we will consider finitely generated extensions of rings. Here $B$ will be a
finitely-generated $A$-algebra.

\begin{thm} Let $A$ be Noetherian. Then $\dim A[X] = \dim A + 1$.  \end{thm}

\begin{proof} Let $\mf{p} \in \Spec A$ and let $\mf{P} \in \Spec B$ such that
    $\mf{P} \cap A = \mf{p}$. Choose $\mf{P}$ to be maximal for this property.
    We will show that $\mr{ht}(\mf{P} / \mf{p}B) = 1$. After localization at
    $\mf{p}$, we may assume that $\mf{p}$ is maximal and $A$ is local. Then $B
    / \mf{p}B = A/\mf{p}[X]$, and $A / \mf{p}$ is a field. Thus $B / \mf{p}B$
    is a PID, so $\mf{P} / \mf{p}B$ is a nonzero principal ideal, so it must
    have height exactly equal to $1$.

    Previous we have seen that because $B$ is flat, $\dim B_{\mf{P}} = \dim
A_{\mf{p}} + 1$, and thus $\mr{ht}\mf{P} = \mr{ht}(\mf{p}) + 1$, and we obtain
the desired result.  \end{proof}

\begin{cor} \begin{enumerate} \item $\dim A[x_1, \ldots, x_m] = \dim A + m$.
\item If $k$ is a field, then $\dim k[x_1, \ldots, x_m] = m$. Moreover,
$\mr{ht}((x_1, \ldots, x_i)) = i$.  \end{enumerate} \end{cor}

\begin{proof} We only need to prove the part about the height of $(x_1, \ldots,
    x_i)$. Then we have \[ 0 \subsetneq (x_1) \subsetneq (x_1, x_2) \subsetneq
    \cdots \subsetneq (x_1, \ldots, x_i) \subsetneq \cdots \subsetneq (x_1,
\ldots, x_n). \] Then clearly $\mr{ht}((x_1, \ldots, x_i)) \geq i$ and the
inequality cannot be strict because otherwise $(x_1, \ldots, x_n)$ has height
strictly larger than $n$.  \end{proof}

\begin{exer} Let $A$ be Noetherian, $I \subset A$, and $I' \subset A[X]$.
Suppose $I' = I[X]$. Show that $\mr{ht}(I') = \mr{ht}(I)$.  \end{exer}

\begin{thm}[Noether Normalization] Let $A$ be a finitely generated $k$-algebra
    over a field $k$. Let $\mf{a}_1 \subsetneq \mf{a}_2 \subsetneq \cdots
    \subsetneq \mf{a}_p$ be a chain of prime ideals of $A$. Then there exist
    elements $x_1, \ldots, x_m \in A$ algebraically independent such that:
    \begin{enumerate} \item $A$ is integral over $B = k[x_1, \ldots, x_m]$;
        \item For all $i = 1, \ldots, p$, there exists an integer $h(i) \geq 0$
            such that $\mf{a}_i \cap B = (x_1, x_2, \ldots, x_{h(i)})$. In
            particular, $\mr{ht}(\mf{a}_i) = h(i)$.  \end{enumerate} \end{thm}

\begin{rmk} Recall that $x_1, \ldots, x_m$ are algebraically independent means
that the map $k[X_1, \ldots, X_m] \to A$ sending $X_i \mapsto x_i$ is
injective.  \end{rmk}

\begin{proof}[Proof of Theorem] We will treat the case where $A = k[y_1,
    \ldots, y_m]$ because any finitely generated algebra over $k$ which is an
    integral domain is integral over such a ring. We will prove the result by
    induction on $m$ and $p$. The case when $m = 1$ is clear because $k[y_1]$
    is a PID.

    We will assume the result is true for $m-1$. Now we will form an induction
    on $p$. In the case $p=1$, first assume that $\mf{a}_1 = (x_1)$, where $x_1
    \notin k$. Then $x_1 = f(y_1, \ldots, y_m) \in k[y_1, \ldots, y_m]$. For $i
    = 2, \ldots, m$ we will introduce $x_i = y_i - y_1^{r_i}$ for some integer
    $r_i$. We want to choose the $r_i$ such that $y_1$ is integral over $k[x_1,
    \ldots, x_m]$. Writing \begin{align*} x_1 &= f(y_1, \ldots, y_m) \\ &=
    \sum_{\ul{p}} a_{\ul{p}} y^{\ul{p}} \\ &= \sum_{\ul{p}} a_{\ul{p}}
y_1^{p_1} (x_2 + y_1^{r_2})^{p_2} \cdots (x_m + y_1^{r_m})^{p_m}, \end{align*}
we see that $f(\ul{p}) = p_1 + r_2 p_2 + \cdots r_m p_m$ is the maximal degree
of $y_1$ in this expression. Then it is possible to choose $r_2, \ldots, r_m$
such that $f(\ul{p})$ are all distinct, for example $r_i = k^i$ for $k > \max
\qty{p_i}$, where the max is taken over all $p_i$ that occur in the polynomial.

    Then choosing the $\ul{p}$ for which $f(\ul{p})$ is maximal, we can write
    \[ x_1 = a_{\ul{p}} y_1^{f(\ul{p})} + \sum_{j \leq f(\ul{p})} Q_j(\ul{x})
    y_1^j. \] Thus $y_1$ is integral over $k[x_1, \ldots, x_m]$, so $y_i = x_i
    + y_1^{r_i}$ is integral over $k[x_1, \ldots, x_m]$. Therefore $A$ is
    integral over $k[x_1, \ldots, x_m]$. Finally, $x_1, \ldots, x_m$ are
    algebraically independent because otherwise the transcendance degree of
    $\mr{Frac}(A)$ is smaller than $m$. We now show that $\mf{a}_1 \cap B =
    (x_1)$. If $y \in \mf{a}_1 \cap B$, then write $y = b'x_1$ for some $b' \in
    A$. But then $b' \in A \cap \mr{Frac}(B)$, so because $B$ is integrally
    closed, $b' \in B$ and thus $y \in B x_1$.

    For the general case, suppose $\mf{a}_1$ is generated by more than one
    element. Choose $x_1 \in \mf{a}_1 \setminus k$ and choose $t_2, \ldots,
    t_m$ such that $A$ is integral over $C = k[x_1, t_2, \ldots, t_m]$ and $x_1
    A \cap C = C x_1$. By the induction hypothesis on $m$, there exist $x_2,
    \ldots, x_m$ such that $k[t_2, \ldots, t_m]$ is integral over $k[x_1,
    \ldots, x_m]$ and $\mf{a_1} \cap k[t_1, \ldots, t_m] \cap k[x_2, \ldots,
    x_m] = (x_2, \ldots, x_h)$. To see this, choose $z \in \mf{a}_1 \cap k[x_1,
    \ldots, x_h]$. Then there exist $h_j \in k[x_2, \ldots, x_m]$ such that \[
    z = \sum_{j=1}^d h_j x_1^j \] bceause $x_1 \in \mf{a}_1 \cap k[x_1, \ldots,
    x_m]$. Thus $h_0 \in \mf{a}_1 \cap k[x_1, \ldots, x_m] = (x_2, \ldots,
    x_h)$ and thus $z \in (x_1, \ldots, x_h)$. This finishes the case $p = 1$.

    Now we complete the induction on $p$. Suppose we have a chain of prime
    ideals $\mf{a}_1 \subsetneq \cdots \subsetneq \mf{a}_p$ in $A$. Then we
    choose $t_1, \ldots, t_m$ such that \begin{itemize} \item $A$ is integral
        over $k[t_1, \ldots, t_m]$ \item $\mf{a}_i \cap k[t_1, \ldots, t_m] =
(t_1, \ldots, t_{h(i)})$ for $i \leq p-1$.  \end{itemize} Now we apply the case
$p = 1$ to the ideal $\mf{a}_p \cap k[t_{r+1}, \ldots, t_m]$ where $r =
h(p-1)$. Thus there exist $x_{r+1}, \ldots, x_m$ such that $k[t_{r+1}, \ldots,
t_m]$ is integral over $k[x_1, \ldots, x_m]$ and $\mf{a}_p \cap k[x_{r+1}
\ldots, x_m] = (x_{r+1}, \ldots, x_h)$.

    First, it is clear that $A$ is integral over $k[t_1, \ldots, t_r, x_{r+1},
    \ldots, x_m]$. If we set $x_i = t_i$ for $i \leq i$, then we will show that
    \[ \mf{a}_p \cap k[x_1, \ldots, x_m] = (x_1, \ldots, x_h). \] One direction
    is obvious from the inductive hypothesis. In the other direction, if we
    write $y = \sum a_{\ul{h}} \ul{x}^{\ul{h}}$, then because $x_1, \ldots, x_r
    \in \mf{a}_{p-1} \in \mf{a}_{p-1} \cap k[x_1, \ldots, x_m]$, we see that \[
    a_{\ul{0}} \in \mf{a}_p \cap k[x_{r+1}, \ldots, x_m] = (x_{r+1}, \ldots,
x_h). \] Thus $y \in (x_1, \ldots, x_h)$, as desired.  \end{proof}

\begin{cor} Let $A$ be an integral domain of finite type over a field $k$. Then
$\dim A$ equals the transcendence degree of the fraction field of $A$.
\end{cor}

\begin{proof} There exist $x_1, \ldots, x_m$ such that $A$ is integral over
    $k[x_1, \ldots, x_m]$. Then $\mr{Frac}(A)$ is algebraic over $k(x_1,
    \ldots, x_m)$. On the other hand, we know that $\dim A = m$, which is the
    transcendance degree of $k(x_1, \ldots, x_m)$.  \end{proof}

\begin{cor}[Nullstellensatz] Let $A$ be an algebra of finite type over a field
$k$. Then for any maximal ideal $\mf{m} \subset A$, $A / \mf{m}$ is algebraic
over $k$.  \end{cor}

\begin{proof} Note that $\dim A/\mf{m} = 0$, but this is also the transcendence
degree over $k$ by the previous corollary.  \end{proof}

\begin{prop} Let $A$ be an integral domain of finite type over a field $k$.
Then for any prime ideal $\mf{p} \in \Spec A$, we have $\mr{ht}(\mf{p}) + \dim
A/\mf{p} = \dim A$.  \end{prop}

\begin{proof} Let $\mf{p} \in \Spec A$ and $h$be its height. Then by Noether
    normalization, there exist $x_1, \ldots, x_m$, where $m = \dim A$, such
    that $A$ is integral over $A' = k[x_1, \ldots, x_m]$ and $\mf{p}' = \mf{p}
    \cap A' = (x_1, \ldots, x_h)$. Then we know that $A'/\mf{p}' \cong
    k[x_{h+1}, \ldots, x_n]$. Because $A/\mf{p}$ is integral over $A'/\mf{p}'$,
    we see that $\dim A/\mf{p} = \dim A'/\mf{p}' = n-h$. But then
    $\mr{ht}(\mf{p}) = \mr{ht}(\mf{p}') = h$ because $A'$ is integrally closed
    and $A$ is integral over $A'$. Therefore, \[ \mr{ht}(\mf{p}) + \dim
    A/\mf{p} = h + n-h = n = \dim A. \qedhere \] \end{proof}

\begin{rmk} Let $A$ be a finitely generated $k$-algebra. Then for any maximal
    ideal $\mf{m} \subset A$, we know that $A / \mf{m}$ is an algebraic
    extension of $k$. Therefore we have a correspondence \[ \qty{ \Centerstack{
    {Maximal ideals} {of $\Spec A$} } } \longrightarrow \qty{\Centerstack{
    {Galois orbits of points in $\ol{k}^n$} {satisfying certain algebraic
    equations} }}. \] In addition, any prime ideal $\mf{p}$ defines a
subvariety $\Spec A/\mf{p} = V(\mf{p})$ of $\Spec A$.  \end{rmk}

\begin{prop} Let $A,A'$ be two finitely-generated $k$-algebras that are
domains. Then for any minimal prime ideal $\mf{p} \subset A \otimes_k A'$, we
have $\dim A \otimes_k A' / \mf{p} = \dim A + \dim A'$.  \end{prop}

\begin{proof} Choose $B,B'$ polynomials over $k$ such that $A$ (resp $A'$) is
    integral over $B$ (resp $B'$). Then write $d,d' = \dim A, \dim A'$. Then $A
    \otimes A'$ is torsion free over $B \otimes B'$. Then because $\mf{p}
    \subset \Spec (A \otimes A')$ is minimal, we see that $B \otimes B' \cap
    \mf{p} = 0$. Therefore $A \otimes A' / \mf{p}$ is integral over $B \otimes
    B'$ and thus the desired result follows using integrality.  \end{proof}

\begin{rmk} We can think of $\Spec A \otimes A'$ as the product $\Spec A \times
\Spec A'$. The proposition says that irreducible components of the product
variety have the expected dimension.  \end{rmk}

\begin{prop}[Hilbert's Nullstellensatz] Let $k$ be a field, $A$ be a
finitely-generated $k$-algebra and $I \subsetneq A$ be a proper ideal. Then
$\sqrt{I} = \bigcap_{\substack{\mf{m} \supset I \\ \mf{m} \text{ maximal}}}
\mf{m}$.  \end{prop}

\begin{proof} One direction is obvious because $\sqrt{I}$ is the intersection
    of all prime ideals containing $I$. Let $a \in J = \bigcap \mf{m} \setminus
    \sqrt{I}$. Then $S = \qty{1, a, a^2, \ldots } \cap I = \emptyset$ and thus
    $S^{-1}I \subset S^{-1}A$ is a proper ideal. Thus there exists a maximal
    ideal of $S^{-1}A$ such that $S^{-1}I \subset \mf{m}_0$. Because $S^{-1}A$
    is a finitely generated $k$-algebra, we have \[ \dim (S^{-1}A / \mf{m}_0) =
    \operatorname{trdeg}_k S^{-1A / \mf{m_0}} = 0. \] Then writing $\mf{m} =
    \mf{m}_0 \cap A$, we see that $k \subset A/\mf{m} \subset S^{-1} A /
    \mf{m}_0$ and thus $\dim A / \mf{m} = 0$ and therefore $\mf{m} \supset I$
    is maximal. However, $a \notin \mf{m}$ by hypothesis, which gives us a
    contradiction.  \end{proof}

\section{Rings of Dimension $1$}% \label{sec:rings_of_dimension_1_}

\begin{defn} A local ring $A$ is called a \textit{discrete valuation ring} if
it is a principal ideal domain and has a nonzero prime ideal.  \end{defn}

This prime ideal is naturally maximal becuase if $\mf{p} \subset \mf{m} \subset
A$ given by $(a) \subset (b)$, then we know $a = bs$ for some $s \in A$, but
then $s \in \mf{p}$, so $s = as'$ and thus $a = bas'$, so $bs' = 1$. In
particular, $\dim A = 1$.

\begin{defn} A \textit{discrete valuation} on $A$ is a surjective function $v
\colon A^* \to \Z$ such that \begin{itemize} \item $v(xy) = v(x) + v(y)$; \item
$v(x+y) \geq \min \qty{v(x), v(y)}$.  \end{itemize} We define $v(0) = \infty$.
\end{defn}

If $A$ is a DVR, then choose $x \neq 0$. We define $v(x) = \sup \qty{n \geq 0
\mid x \in (\pi^n)}$, where $\pi$ generates the maximal ideal of $A$. Then
$v(x)$ is well-defined because $\bigcap_0 (\pi^n) = 0$ by the Krull
intersection theorem. We can extend this valuation to $K = \operatorname{Frac}
A$ by $v(x/y) = v(x) - v(y)$. 

\begin{prop} If $K$ is a field and $v \colon K^* \to \Z$ is a valuation, then
$A = \qty{x \in K \mid v(x) \geq 0} \cup \qty{0}$ is a discrete valuation ring.
\end{prop}

Proof of this is left as an exercise.

\begin{exm} Consider the ring $\Z_{(p)}$ where $p$ is a prime number. Then for
    any $\frac{x}{y}$ with $y$ coprime to $p$, define $v_p\qty(\frac{x}{y})$ to
    be the maximal power of $p$ dividing $x$.

    Now let $k$ be a field and let $k[[T]]$ be the ring of formal power series
in $T$. Then any series $a_0 + a_1 T + \cdots$ is invertible iff $a_0 \neq 0$,
and thus any element is a product of $T^n$ and a unit for some $n$. Thus $v(F)$
is the degree of the first monomial with nonzero coefficient.  \end{exm}

\begin{prop} Let $A$ be a ring. Then the following are equivalent:
\begin{enumerate} \item $A$ is a DVR.  \item $A$ is a local noetherian ring and
$\mf{m}_A$ is generated by an element $\pi$ which is not nilpotent.
\end{enumerate} \end{prop}

\begin{proof} One direction is clear. If $A$ is a DVR, then it is a domain, and
    thus the generator of the prime ideal is not nilpotent.

    In the other direction, suppose $\mf{m} = (\pi)$. Then by the Krull
intersection theorem, $\bigcap_n (\pi^n) = 0$. Then for $0 \neq x \in A$, there
exists a maximal $n$ such that $x \in (\pi^n)$, so $x = \pi^n u$ for some $u
\in A^{\times}$. But then for $y \in A, y \neq 0$, write $y = \pi^m v$ for $v
\in A^{\times}$. Thus $xy = \pi^{m+n} uv \neq 0$ because $\pi$ is not
nilpotent. Therefore $A$ is a domain and thus is a DVR.  \end{proof}

\begin{prop} Let $A$ be a local Noetherian ring. Then $A$ is a DVR if and only
if \begin{enumerate} \item $A$ is integrally closed; \item $A$ has a unique
nonzero prime ideal.  \end{enumerate} \end{prop}

\begin{proof} One direction is clear. In the other direction, assume $A$ is
    integrally closed and has a unique nonzero prime. First, we note that $A$
    must be local and $\mf{m} \neq 0$. Then if we write \[ \mf{m}' = \qty{x \in
    K \mid x \mf{m} \subset A} \supset A, \] this is an $A$-module. If we write
    $y \in \mf{m}$, then $y \mf{m}' \subset A$ and thus $\mf{m}' \subset
    Ay^{-1}$. This implies that $\mf{m}'$ is finitely generated. Then we have
    $\mf{m} \subset \mf{m} \mf{m}' \subset A$. We will show that this cannot
    equal $\mf{m}$ by contradiction. 

    Let $x \in \mf{m}'$. Then $x \mf{m} \subset \mf{m}$ and thus $x$ is
    integral over $A$. Because $A$ is integrally closed, $x \in A$. Thus
    $\mf{m}' = A$. Now we set $S = \qty{1, x, x^2, \ldots}$. Then $S^{-1}A = K$
    because it has no nonzero prime ideals. If we choose $z \in A \setminus 0$,
    then we can write $\frac{1}{z} = \frac{y}{x^n}$ for some $n \geq 0$. This
    tells us that $x^n \in (z)$. Because $\mf{m}$ is finitely generated, we see
    that $\mf{m}^N \subset (z)$. Then let $N_0$ be the smallestt integer such
    that $\mf{m}^{N_0} \subset (z) \subset \mf{m}$ and let $y \in \mf{m}^{N_0 -
    1} \setminus (z)$. Then we have $\mf{m}y \subset \mf{m}^{N_0} \subset (z)$
    and thus $\frac{y}{z} \in \mf{m}'$. This implies that $\mf{m}' \supsetneq
    A$ and thus $\mf{m} \mf{m}' = A$. Therefore we can write \[ 1 = \sum x_i
    y_i^{-1} \] where $x_i \in \mf{m}, y_i \in \mf{m}'$. Therefore there exists
$i$ such that $x_i y_i^{-1} \notin \mf{m}$ and thus $y_i^{-1} \mf{m} = A$.
Therefore $\mf{m} = (y_i)$ is a principal ideal, so $A$ is a DVR.  \end{proof}

\begin{prop} Let $A$ be a Noetherian domain. The following are equivalent:
    \begin{enumerate} \item For all $0 \neq \mf{p} \in \Spec A$, the
        localization $A_{\mf{p}}$ is a discrete valuation ring.  \item $A$ is
        integrally closed and of dimension $1$.  \end{enumerate} \end{prop}

\begin{proof} \begin{description} \item[1 implies 2:] Let $x \in K$ be integral
    over $A$. But then $x \in A_{\mf{p}}$ for all $\mf{p} \neq 0$ and thus $x
    \in \bigcap_{\mf{p} \neq 0} A_{\mf{p}} = A$. Being of dimension $1$ is
    easy. If $0 \neq \mf{p} \subset \mf{m} \subset A$, we localize at $\mf{m}$
    and see that $\mf{p} A_{\mf{m}} = \mf{m} A_{\mf{m}}$, so $\mf{p} = \mf{m}$.
    \item[2 implies 1:] For all $\mf{p}$, we know that $A_{\mf{p}}$ is
integrally closed. Because it has dimension $1$, it must be a DVR.
\end{description} \end{proof}

\begin{defn} A ring $A$ is called a \textit{Dedekind domain} if it is a domain
satisfying the properties of the previous proposition.  \end{defn}

\begin{exm} $\Z$ is a Dedekind domain. It is clearly a domain, having dimension
$1$ follows from being a PID, and being integrally closed is obvious. More
generally, any principal ideal domain is a Dedekind domain.  \end{exm}

\begin{exm} Let $A$ be a Dedekind domain and $K = \mr{Frac}(A)$. Then let $L/K$
    be a finite extension and $B$ be the integral closure of $A$ in $L$. Then
    $B$ is a Dedekind domain. In particular, if $K$ is a number field, then its
    ring of integers $\mc{O}_K$ is a Dedekind domain.

    Here is another example of this. Let $k$ be a field and $A =  k[x], k =
k(X)$. Then if $L/K$ is a field extension and $B$ is the integral closure of
$A$ in $L$, then $\Spec B \to \Spec A$ is a smooth affine curve with a map to
$\A^1$.  \end{exm}

\begin{rmk} All smooth curves can be obtained in this way (of taking the
integral closure of some ring). Also, normalization resolves all singluarities
of curves.  \end{rmk}

\begin{defn} A \textit{fractional ideal} $\mf{a} \subset K = \mr{Frac}(A)$ is
an $A$-submodule of finite type.  \end{defn}

Note that for $\mf{a}, \mf{b}$ nonzero fractional ideals, then $\mf{a} \cdot
\mf{b}$ is a fractional ideal.

\begin{prop} If $A$ is a Dedekind domain, then the set of nonzero fractional
ideals form an abelian group.  \end{prop}

\begin{proof} It is easy to see that the multiplication is associative and
    commutative. Now we need to show that inverses exist. For $\mf{a} \subset
    K$ a fractional ideal, we need to find another fractional ideal $\mf{b}$
    such that $\mf{a} \mf{b} = A$. Set \[ \mf{b} = \mf{a}^{-1} \coloneqq \qty{
    x \in K \mid x \cdot \mf{a} \subset A }. \] Clearly we know that for $0
    \neq x \in \mf{a}$, we have $\mf{a}^{=1} \subset x^{-1}A$ and thus
    $\mf{a}^{-1}$ is of finite type. Then for any prime ideal $\mf{p}$ we know
    that $\mf{a}_{\mf{p}} \cdot \mf{b}_{\mf{p}} = (\mf{a \cdot b})_{\mf{p}}$,
    and thus $\mf{a}_{\mf{p}}^{-1} = (\mf{a}^{-1})_{\mf{p}}$. Therefore we have
    \[ (\mf{a} \cdot \mf{a}^{-1})_{\mf{p}} = \mf{a}_{\mf{p}} \cdot
    \mf{a}_{\mf{p}}^{-1} = A_{\mf{p}}, \] and thus $\mf{a} \cdot \mf{a}^{-1} =
A$.  \end{proof}

Now if $A$ is a Dedekind domain and $0 \neq \mf{p} \neq \mf{p}' \subset A$,
then $\mf{p} + \mf{p}' = A$, so $\mf{p} \mf{p}' = \mf{p} \cap \mf{p}'$. This
implies that for any $0 \neq \mf{a} \subset A$ that $\mf{p} \cdot \mf{a}
\subsetneq \mf{a}$ because otherwise $\mf{p} = A$.

\begin{lem} If $x \in A$ is nonzero, then there are only finitely many maximal
ideals $\mf{p}$ such that $x \in \mf{p}$.  \end{lem}

\begin{proof} Let $x \in \mf{p}$. Then $\mf{a}^{-1} \subset x^{-1}A$, so if $x
    \in \mf{p}_1, \mf{p}_2, \ldots$ for infinitely many maximal ideals, then \[
    x \in \mf{p}_1 \supsetneq \mf{p}_1 \mf{p}_2 \supset \cdots \subset
\prod_{i=1}^n \mf{p}_i \supset \cdots \] is a strictly decreasing infinite
chain of ideals containing $x$. Thus we have \[ \mf{p}_1^{-1} \subset
\mf{p}_1^{-1} \mf{p}_2^{-1} \subset \cdots \subset \prod_{i=1}^n \mf{p}_i^{-1}
\subset \cdots \subset  x^{-1} A, \] but this is impossible because $x^{-1}A$
is Noetherian.  \end{proof}

\begin{rmk} For any nonzero ideal $I \subset A$, there are only finitely many
prime ideals $\mf{p}$ such that $I \subset \mf{p}$.  \end{rmk}

\begin{defn} Let $\mf{p}$ be a maximal ideal of $A$. Then let $v_{\mf{p}}(I)$
    be the $n_{\mf{p}}$ such that $I A_{\mf{p}} = \omega_{\mf{p}}^{n_{\mf{p}}}
    A_{\mf{p}}$. Thus $I \subset \mf{p}^{n_{\mf{p}}}$ but $I \not \subseteq
    \mf{p}^{n_{\mf{p}} + 1}$.  \end{defn}

\begin{cor} Let $I$ be a fractional ideal of $A$, where $A$ is a Dede,ind
    domain. Then \[ I = \prod_{\mf{p}} \mf{p}^{v_{\mf{p}}(I)}. \] \end{cor}

\begin{proof} If we denote the product by $J$, then $I_{\mf{p}} = J_{\mf{p}}$
for all $\mf{p}$ and thus $I = J$.  \end{proof}

\begin{rmk} This corollary gives the primary decomposition of an ideal in a
Dedekind domain. In number theory, this replaces the prime factorization of an
integer.  \end{rmk}

\section{Depth}% \label{sec:depth}

Let $A$ be a ring and $M$ be an $A$-module. Then for $a_1, \ldots, a_r \in A$,
write $\ul{a} = (a_1, \ldots, a_r) \subset A$. 

\begin{defn} The sequence $a_1, \ldots, a_r$ is an \textit{$M$-regular
    sequence} if it satisfies the following: \begin{enumerate} \item $a_i$ is
        not a zero divisor of $M / (a_1, \ldots, a_{i-1})M$ for $i = 1, \ldots,
        r$.  \item $\ul{a} \cdot M \subsetneq M$. Therefore we have \[ a_1 M
    \subsetneq (a_1,a_2) M \subsetneq \cdots \subsetneq (a_1, \ldots, a_r) M.
    \] \end{enumerate} \end{defn}

\begin{lem} Assume that $\ul{a} = (a_1, \ldots, a_r)$ is $M$-regular and let
$m_1, \ldots, m_r \in M$ such that $\sum_{i=1}^r a_i m_i = 0$. Then $m_i \in
\ul{a} \cdot M$ for all $i$.  \end{lem}

\begin{proof} We will induct on $r$. If $r = 1$, then $a_1 m_1 = 0$ implies
    $m_1 = 0$. Now assume $a_1 m_1 + \cdots + a_r m_r = 0$ implies that $a_r
    \ol{m}_r = 0$ in $M / (a_1, \ldots, a_{r-1}) M$. Then there exists $n_1,
    \ldots, n_{r-1}$ such that $m_r = a_1 n_1 + \cdots + a_{r-1} n_{r-1}$, and
    thus \[ \sum_{i=1}^{r-1} a_i (m_i + a_r n_i) = 0. \] Thus $m_i + a_r n_i
\in (a_1, \ldots, a_{r-1})M$ and thus $m_i \in (a_1, \ldots, a_r)M$.
\end{proof}

\begin{thm} Assume that $(a_1, \ldots, a_r)$ is an $M$-regular sequence. Then
for any integers $n_1, \ldots, n_r$, the sequence $a_1^{n_1}, \ldots,
a_r^{n_r}$ is $M$-regular.  \end{thm}

\begin{proof} It is sufficient to prove that $a_1^n, a_2, \ldots, a_r$ is an
    $M$-regular sequence. We will induct on $n$. Assume that $a_1^{n-1},
    \ldots, a_2, a_r$ is $M$-regular. First, multiplication by $a_1^n$ is
    clearly injective. 

    Now if $a_1^n, a_2, \ldots, a_{j-1}$ is $M$-regular, then let $m \in M$
    such that \[ a_j m = a_1^n m_1 + \cdots + a_{j-1} m_{j-1}. \] By induction
    on $n$, we can write \[ m = a_1^{n-1} m_1' + \cdots + a_{j-1} m_{j-1}'. \]
    Multiplying this by $a_j$ and combining the two equations, we obtain \[ 0 =
    a_1^{n-1}(a_1 m_1 - a_jm_1') + a_2(m_2 - a_jm_2') + \cdots +
a_{j-1}(m_{j-1} - a_j m_{j-1}'). \] By the previous lemma, we see that $a_1 m_1
- a_j m_1' \in (a_1^{n-1}, \ldots, a_{j-1})M$. Therefore, $a_j m_1' \in (a_1,
\ldots, a_{j-1})M$, so $m_1' \in (a_1, \ldots, a_{j-1})M$. This implies that $m
\in (a_1^n, \ldots, a_{j-1})M$.  \end{proof}

\begin{defn} The sequence $(a_1, \ldots, a_r)$ is said to be
    \textit{$M$-quasi-regular} if one of the following equivalent conditions
    holds: \begin{itemize} \item For all $F(x_1, \ldots, x_r) \in M[x_1,
        \ldots, x_r] = A[x_1, \ldots, x_r] \otimes_A M$ homogeneous of degree
        $n$ such that $F(a_1, \ldots, a_r) \in I^{n+1}M$, this implies that
        $F(x_1, \ldots, x_r) \in IM[x_1, \ldots, x_r]$, where $I = (a_1,
        \ldots, a_r)$.  \item If $F(x_1, \ldots, x_r) \in M[x_1, \ldots, x_r]$
        is homogeneous and such that $F(a_1, \ldots, a_r) = 0$, then $F \in
        IM[x_1, \ldots, x_r]$ \item The map \[ M/IM[x_1, \ldots, x_r] \to
        \mr{gr}^I M = \bigoplus_{n=0}^{\infty} I^n M / I^{n+1}M,\ F \mapsto
    F(a_1, \ldots, a_r) \] is an isomorphism.  \end{itemize} \end{defn}

\begin{lem} Assume that $(a_1, \ldots, a_r)$ is $M$-quasi-regular and $x \in
A$. Then if $(IM : x) = IM$, then $(I^n M : x) = I^n M$ for all $n \geq 1$.
\end{lem}

\begin{proof} We will induct on $n$. Suppose that $m \in (I^n M : x)$. Then $xm
    \in I^n M \subset I^{n-1}$ and thus $m \in I^{n-1}M$. Therefore there
    exists $g(X_1, \ldots, X_r)$ homogeneous of degree $n-1$ such that $m =
    g(a_1, \ldots, a_r)$. This implies that $x g(X_1, \ldots, X_r) \in
    M[\ul{X}]$, so \[ xg(a_1, \ldots, a_r) = xm \in I^n M \] and then
    quasi-regularity gives us that $xg(X_1, \ldots, X_r) \in IM[\ul{X}]$. This
    implies that \[ g(X_1, \ldots, X_r) \in (IM : X)[\ul{X}] = IM[\ul{X}], \]
as desired.  \end{proof}

\begin{prop} Using the same notation, if $(a_1, \ldots, a_r)$ is $M$-regular,
    then it is $M$-quasi-regular. Conversely, if $(a_1, \ldots, a_r)$ is
    $M$-quasi-regular and $M, M/a_1 M, \ldots, M/(a_1, \ldots, a_r)M$ are
    Hausdorff in the $I$-adic topology, then $(a_1, \ldots, a_r)$ is
    $M$-regular.  \end{prop}

\begin{proof} First we prove that regular implies quasi-regular by induction on
    $r$. Clearly $r=1$ is obvious. Suppose $g(x_1) \in M[x]$ is homogeneous of
    degree $n$. Then if $g(a_1) \in a_1^{n+1M}$, we have $a_1^{n+1}m' = a_1^n
    m$ and thus $a_1^n(m-a_1 m') = 0$. By regularity, we have $m = a_1 m'$ and
    thus $g(x_1) \in a_1 M[x_1]$.

    For the inductive step, suppose that $(a_1, \ldots, a_r)$ is regular. Then
    we know that $(a_1, \ldots, a_{r-1})$ is $M$-quasi-regular, so now choose
    $F(x_1, \ldots, x_r) \in M[x_1, \ldots, x_r]$ homogeneous of degree $q$ and
    such that $F(a_1, \ldots, a_r) = 0$. Then we can write \[ F(x_1, \ldots,
    x_r) = G(x_1, \ldots, x_{r-1}) + x_r H(x_1, \ldots, x_r) \] where $H$ is
    homogeneous of degree $q-1$. Then $G(a_1, \ldots, a_{r-1}) \in I_0^q M$,
    where $I_0 = (a_1, \ldots, a_{r-1})$. This implies that $a_r H(a_1, \ldots,
    a_{r-1}) \in I_0^q M$, which implies that $H(a_1, \ldots, a_{r-1}) \in
    (I_0^q M : a_r)$. Because $a_1, \ldots, a_{r-1}$ is quasi-regular and $a_1,
    \ldots, a_r$ is regular, we have $(I_0 M : a_r) = I_0 M$. This implies that
    $(I_0^q M : a_r) = I_0^q M$ and thus $H(a_1, \ldots, a_r) \in I_0^q M$.
    Then $H(a_1, \ldots, a_r) = h(a_1, \ldots, a_{r-1})$ where $h$ is
    homogeneous of degree $q$. Now let \[ g(x_1, \ldots, x_{r-1}) = G(x_1,
    \ldots, x_{r-1}) + a_r h(x_1, \ldots, x_{r-1}). \] Because $g(a_1, \ldots,
    a_{r-1}) = F(a_1, \ldots, a_r) = 0$, we see that $g(x_1, \ldots, x_{r-1})
    \in I_0 M [ x_1, \ldots, x_{r-1} ]$ by induction. We conclude that $G \in
    IM[x_1, \ldots, x_{r-1}]$. Because $H \in I_0 M[x_1, \ldots, x_r]$, we have
    $F \in IM[x_1, \ldots, x_r]$.

    Now in the other direction, we will induct on $r$. If $r=1$, assume that
    $a_1$ is $M$-quasi-regular. We need to show that $m \mapsto a_1 m$ is
    injective. Suppose that $a_1 m = 0$. Then if we consider the polynomial
    $g_0(x) = m$, we see that $a_1 g_0(a_1) = 0$ and thus $x_1 g_0(x_1) \in a_1
    M[x_1]$. This means that $g_0(x_1) \in a_1 M[x_1]$ and thus $m \in a_1 M$.
    Then there exists $g_1$ homogeneous of degree $1$ such that $m = g_1(a_1)$.
    Then we see that $x_1 g_1(x_1) \in IM[x_1]$ and thus $g_1(x_1) \in
    IM[x_1]$, so $m = g_1(a_1) \in I^2 M$. Then there exists $g_2 \in M[x_1]$
    homogeneous of degree $2$ such that $m = g_2(a_1)$. Then $a_1 m = a_1
    g_2(a_1) = 0$, so we deduce that $x_1 g_2(x_1) \in IM$ and thus $g_2(x_1)
    \in IM$. Evaluating at $a_1$, we see that $m \in I^3 M$. In particular, we
    see that $m \in \bigcap_{n \geq 1} I^n M = 0$, where the last equality uses
    the Hausdorff condition, and thus $m = 0$.

    Now for the induction, we know that $a_1$ is $M$-regular. We need to show
    that $(a_2, \ldots, a_r)$ is $M/a_1 M$-regular. This follows from the
    inductive hypothesis if we check that $(a_2, \ldots, a_r$ is $M/a_1 M
    \eqqcolon \ol{M})$-quasi-regular. Choose $F(x_2, \ldots, x_r) \in M[x_2,
    \ldots, x_r]$ homogeneous of degree $n$ such that $F(a_2, \ldots, a_r) \in
    a_1 M$. Then we can write $F(a_1, \ldots, a_r) = a_1 m$, so let $i$ be such
    that $m \in I^i M$. Then let $G \in M[x_1, \ldots, x_r]$ be homogeneous of
    degree $i$ and satisfy $m = G(a_1, \ldots, a_r)$. Then the polynomial \[
    F(x_2, \ldots, x_r) - x_1 G(x_1, \ldots, x_r) \] vanishes at $(a_1, \ldots,
    a_r)$. If $i < n-1$, then $a_1 G(a_1, \ldots, a_r) = F(a_2, \ldots, a_r)
    \in I^n M \subset I^{i+2} M$. But then $x_1 G(x_1, \ldots, x_r)$ is
    homogeneous of degree $i+1$ and thus $x_1 G(x_1, \ldots, x_r) \in IM[x_1,
    \ldots, x_r]$ by quasi-regularity. This implies that \[ m = G(a_1, \ldots,
        a_n) \in I^{i+1}M. \] We can repeat this until $m \in I^{n-1}M$ and $G$
        is of degree $n-1$. Then \[ g(x) = F(x_2, \ldots, x_r) - x_1 G(x_1,
        \ldots, x_r) \] is homogeneous of degree $n$ and $g(a_1, \ldots, a_r) =
        0$. This implies that $g(x_1, \ldots, x_r) \in IM[x_1, \ldots, x_r]$
        and thus $F(x_2, \ldots, x_r) \in IM[x_2, \ldots, x_r]$. Thus \[
        \ol{F}(x_2, \ldots, x_r) \in I \ol{M}[x_2, \ldots, x_r]. \] Then $(a_2,
    \ldots, a_r)$ is $\ol{M}$-quasi-regular, so they are $\ol{M}$-regular by
induction.  \end{proof}

\begin{rmk} For $A$ Noetherian and $M$ of finite type, regular and
quasi-regular are equivalent.  \end{rmk}

\begin{defn} Let $I \subset A$ be an ideal and $M$ and $A$-module. Then the
\textit{$I$-depth} of $M$ is the (possibly infinite) length of the longest
$M$-regular sequence in $I$.  \end{defn}

Before we continue, recall that $\Ext^{\bullet}$ are the right derived functors
of $\Hom$ and are computed by taking projective resolutions of the first
argument or injective resolution of the second argument. In particular, if $0
\to A \to B \to C \to 0$ is an exact sequence, we have an exact sequence \[
\cdots \to \Ext^i(M, A) \to \Ext^i(M, B) \to \Ext^i(M, C) \to \Ext^{i+1}(M,A)
\to \cdots \] and similarly, \[ \cdots \to \Ext^i(C,N) \to \Ext^i(B,N) \to
\Ext^i(A,N) \to \Ext^{i+1}(C,N) \to \cdots \]

\begin{thm} \label{thm:depth} Assume that $A$ is Noetherian, $M$ a finite
    $A$-module, and $I \subset A$ an ideal such that $IM \neq M$. Let $m \in
    \Z_{>0}$. Then the following are equivalent: \begin{enumerate} \item
        $\Ext^n(N,M) = 0$ for all $i < n$ and any finitely $A$-module $N$ such
        that $\supp(N) \subset V(I)$.  \item $\Ext^i(A/I,M) = 0$ for all $i <
        n$.  \item $\Ext^i(N,M) = 0$ for all $i < n$ for some finite $A$-module
        $N$ such that $\supp(N) = V(I)$.  \item There exists a $M$-regular
        sequence $(a_1, \ldots, a_n)$ of length $n$ inside $I$.
\end{enumerate} \end{thm}

\begin{proof} Clearly \textbf{1} implies \textbf{2} implies \textbf{3}. Now we
    prove \textbf{3} implies \textbf{4} implies \textbf{1}.
    \begin{description} \item[3 implies 4:] Assume that $\Hom(N,M) = 0$. If $I$
        does not contain any $M$-regular element, then $I \subset
        \bigcup_{\mf{p} \in \Ass M} \mf{p}$ and thus $I \subset \mf{p}$ for
        some $\mf{p} \in \Ass(M)$. Then $A/\mf{p} \hookrightarrow M$, which is
        equivalent to $A_{\mf{p}} / \mr{p} A_{\mf{p}} \hookrightarrow
        M_{\mf{p}} $. On the other hand, we know $\mf{p} \in V(I) = \supp(N)$,
        so $N_{\mf{p}} \neq 0$. By Nakayama, we see that $N_{\mf{p}} \otimes
        k(\mf{p}) \neq 0$, and thus $\Hom_{A_{\mf{p}}}(N_{\mf{p}}, M_{\mf{p}})
        \neq 0$, but this is simply $\Hom_A(N,M)_{\mf{p}}$. Thus $\Hom(N,M)
        \neq 0$ and thus there exists $a_1 \in I$ that is $M$-regular. Then we
        have an exact sequence \[ 0 \to M \to M \to M/a_1 M \to 0. \] Writing
        $M_1 = M/a_1 M$, we have proved the case $n=1$ and now proceed by
        induction on $n$. Applying $\Ext^n(N,-)$ to the above exact sequence,
        we have the exact sequence \[ \cdots \to \Ext^i(N,M) \to \Ext^i(N,M_1)
        \to \Ext^{i+1}(N,M) \] and deduce that $\Ext^i(N,M_1) = 0$ for $i <
        n-1$. Applying the case $n-1$ to $M_1$, we obtain an $M_1$-regular
        sequence $a_2, \ldots, a_n \in I$ and thus $(a_1, \ldots, a_n)$ is
        $M$-regular.  \item[4 implies 1:] We will induct on $n$. We have a
        sequence $a_1, \ldots, a_n \in I$ that is $M$-regular. Then set $M_1 =
        M/a_1 M$, so we have an exact sequence \[ 0 \to M \to M \to M_1 \to 0.
        \] Then choose $N$ such that $\supp(N) \subset V(I)$. Then we have an
        exact sequence \[ \Ext^{i-1}(N, M_1) \to \Ext^i(N,M) \to \Ext^i(N,M).
        \] If $i < n$, then $i-1 < n-1$. By the inductive hypothesis, this
        implies that $\Ext^i-1(N, M_1) = 0$. Thus the map $\Ext^i(N,M)
        \xrightarrow{\times a_1} \Ext^i(N,M)$ is injective, so because
        $\supp(N) \subset V(I)$, we have $I \subset \sqrt{\Ann(N)}$. Thus there
        exists $m$ such that $a_1^m \in \Ann(N)$. From the exact sequence, we
        know $\Ext^i(N,M) \xrightarrow{\times a_1^m} \Ext^i(N,M)$ is injective,
        which means that $\Ext^i(N,M) = 0$. \qedhere \end{description}
    \end{proof}

\begin{defn} If $A$ is a local Noetherian ring and $M$ is an $A$-module, then
    we define the depth of $M$ to be \[ \operatorname{depth}(M) \coloneqq
    \mf{m}\text{-depth of }M. \] This is the same as the maximal length of an
    $M$-regular sequence in $\mf{m}$.  \end{defn}

\begin{cor} Let $A$ be local Noetherian and $M$ be a finitely-generated
    $A$-module. Then $\mr{depth}(M) = n$ if and only if there exists a
    $M$-regular sequence $a_1, \ldots, a_n$ such that $\Ext^i(k, M) = 0$ for
    all $i < n$ and $\Ext^n(k,M) = \Hom(k,\ol{M})$, where $\ol{M} = M/(a_1,
    \ldots, a_m)M$ and $k = A/\mf{m}$.  \end{cor}

\begin{proof} We know that the equivalence of \textbf{2} and \textbf{4} from
    the theorem implies the corollary except for $\Ext^n(k,M) = \Hom(k,
    \ol{M})$, but this fact be proved by induction on $n$ using the fact that
    \[ 0 \to M \xrightarrow{\times a_1} M \to M/a_1 M \to 0 \] is exact because
this implies that $\Ext^{n-1}(k, M/a_1 M) = \Ext^n(k,M)$. This gives the
desired result.  \end{proof}

\begin{lem} Let $A$ be a local Noetherian and $M,N$ finite $A$-modules. Then if
$k = \mr{depth}(M)$ and $r = \dim(N)$, then $\Ext^i(N,M) = 0$ for all $i <
k-r$.  \end{lem}

\begin{proof} We use induction on $r$. For the case $r = 0$, we know that
    $\supp(N) = \qty{\mf{m}}$ and this follows by the previous theorem. Now for
    the inductive hypothesis, we may assume that $N = A/\mf{p}$ and $\dim
    A/\mf{p} = r$. This is possible because we can consider a filtration on $N$
    with successive quotients of the form $A/\mf{p}_i$ with $\dim A/\mf{p}_i
    \leq r$. We know that $\mf{m} \neq \mf{p}$, so there exists $x \in \mf{m}
    \setminus \mf{p}$, and this $x$ is $N$-regular. Then we have the exact
    sequence \[ 0 \to N \xrightarrow{\cdot x} N \to N' \to 0, \] where $N' =
    N/xN$. Because $\ol{x}$ is not in any minimal prime ideal of $A/\mf{p}$, we
    know that $r' \coloneqq \dim N' < \dim A/\mf{p}$. By induction, we know
    that $\Ext^j(N',M) = 0$ for $k < k-r'$. Now if $i < k-r$, we know that $i+1
    < k-r'$ and thus $\Ext^{i+1}(N',M) = 0$. Now considering the exact sequence
    of $\Ext$ groups, we have \[ \Ext^i(N',M) \to \Ext^i(N,M) \to \Ext^i(N,M)
    \to \Ext^{i+1}(N',M) \] and thus $\Ext^i(N,M) / x \Ext^i(N,M) = 0$ and thus
$\Ext^i(N,M) = 0$ by Nakayama's lemma.  \end{proof}

\begin{rmk} If $N,M$ are finitely-generated, then $\Ext^i(N,M)$ is also
finitely-generated.  \end{rmk}

\begin{thm} Let $A$ be local Noetherian and $M$ be a finitely-generated
$A$-module. Then for all $\mf{p} \in \Ass(M)$, we have $\operatorname{depth} M
\leq \dim A/\mf{p}$.  \end{thm}

\begin{proof} We know $A/\mf{p} \hookrightarrow M$, so $\Hom(A/\mf{p}, M) \neq
0$. Thus $0 \geq \operatorname{depth} M - \dim A/\mf{p}$ by the lemma.
\end{proof}

\begin{cor} Let $A$ be local Noetherian. Then $\dpth A \leq \dim A$.  \end{cor}

In general, this inequality is strict, so we will later study the rings for
which this is an equality.

\begin{lem} Let $A$ be a local Noetherian ring. Then let $M$ be a
finitely-generated $A$-module and $(a_1, \ldots, a_r)$ be an $M$-regular
sequence. Then $\dim M/(a_1, \ldots, a_r)< = \dim M - r$.  \end{lem}

\begin{proof} We prove this by induction on $M$. It suffices to do this for $r
    = 1$, so choose $x \in A$ an $M$-regular element. We know that $\dim M/xM
    \geq \dim M - 1$ for any $x \in A$, so we need to prove this is an
    equality. Then we know $\supp(M/xM) = \supp(M) \cap V(x)$, and thus $x$ is
    not contained in any minimal prime ideal in $\supp(M)$ by regularity.
    Therefore $\supp(M/xM)$ does not contain any minimal ideal of $V(\Ann(M)) =
    \supp(M)$. In particular, this means that $\dim M/xM < \dim M$, as desired.
\end{proof}

\begin{lem} Let $M,N$ be finitely-generated $A$-modules. Then $\supp(M \otimes
N) = \supp(M) \cap \Supp(N)$.  \end{lem}

\begin{proof} Let $\mf{p} \in \Supp(M) \cap \Supp(N)$. Then $M_{\mf{p}},
    N_{\mf{p}} \neq 0$. By Nakayama, we have $M_{\mf{p}} \otimes k(\mf{p}) \neq
    0$ and $N_{\mf{p}} \otimes k(\mf{p}) \neq 0$ where $k(\mf{p}) = A_{\mf{p}}
    / \mf{p} A_{\mf{p}}$. This implies that \[ M_{\mf{p}} \otimes k(\mf{p})
    \otimes_{k(\mf{p})} \otimes N_{\mf{p}} \otimes k(\mf{p}) \neq 0 \] and
    therefore $M_{\mf{p}} \otimes k_{\mf{p}} N_{\mf{p}} = (M \otimes_A
    N)_{\mf{p}} \neq 0$. Thus $\supp(M) \cap \supp(N) \subset \supp(M \otimes
    N)$. Note that for $N = A/xA$, we have $\supp(M/xM) = \supp(M) \cap V(x)$
    because $M \otimes N = M/xM$.  \end{proof}

\chapter{Geometric Aspects of Commutative Algebra}%
\label{cha:normal_and_regular_rings}

\section{Cohen-Macaulay Rings and Modules}%
\label{sec:cohen_macaulay_rings_and_modules}

Let $A$ be a local Noetherian ring and $M$ be a finitely-generated $A$-module. 

\begin{defn} Recall that $\dim M \geq \dpth M$. Then $M$ is
\textit{Cohen-Macaulay} if $\dim M = \dpth M$.  \end{defn}

\begin{thm} Let $A$ be local Noetherian and $M$ be finitely generated.
    \begin{enumerate} \item If $M$ is Cohen-Macaulay, then for any $\mf{p} \in
        \Ass(M)$, $\dpth M = \dim A/\mf{p}$.  \item If $f \in A$ is $M$-regular
        and $M' = M/fM$, then $M$ is Cohen-Macaulay if and only if $M'$ is.
        \item If $M$ is Cohen-Macaulay, then for all $\mf{p} \in \Spec A$,
    $M_{\mf{p}}$ is Cohen-Macaulay and $\dpth_{\mf{p}} M = \dpth_{A_{\mf{p}}}
    M_{\mf{p}}$.  \end{enumerate} \end{thm}

\begin{proof} \begin{enumerate} \item Let $M \neq 0$ and $\dim M = \dpth M$.
    Then let $\mf{p} \in \Ass(M) \subset \supp(M)$. This implies that $\dim
    A/\mf{p} \leq \dim M$, but also $\dim A/\mf{p} \geq \dpth(M) = \dim M$.
\item Let $f$ be $M$-regular. Then we know $\dpth M/fM = \dpth M - 1$ (this
    follows from the Theorem~\ref{thm:depth}, applying $\Ext(k,-)$ to the exact
    sequence $0 \to M \to M \to M/fM \to 0$). But then we proved earlier that
    $\dim M/fM = \dim M - 1$, and thus $M$ is C-M if and only if $M/fM$ is C-M.
\item Let $\mf{p} \in \supp(M)$. Then $\mf{p} \supset \Ann(M)$ and $M_{\mf{p}}
    \neq 0$. But then if $x_1, \ldots, x_r \in \mf{p}$ is $M$-regular where $r
    = \dpth_{\mf{p}} M$, then $x_1, \ldots, x_r \in \mf{p} A_{\mf{p}}$ is
    $M_{\mf{p}}$-regular by exactness of localization. This implies that $\dpth
    M_{\mf{p}} \geq r = \dpth_{\mf{p}} M$. We know that $\dim M_{\mf{p}} \geq
    \dpth M_{\mf{p}}$, so we need to prove that $r = \dim M_{\mf{p}}$. We do
    this by induction on $\dpth_{\mf{p}} M$. If $r = 0$, then we know
    $\Hom(A/\mf{p}, M) \neq 0$. Thus there exists $\mf{p}' \supset \mf{p}$ such
    that $A/\mf{p}' \hookrightarrow M$, so $\mf{p}' \in \Ass M$. By minimality
    of associated primes, we have $\mf{p}' = \mf{p}$. Now $\dim M_{\mf{p}'} =
    0$ because $\mf{p}$ is maximal in $A_{\mf{p}}$ and minimal in $\supp(M)$.

            Now in general, assume $\dpth_{\mf{p}}(M) > 0$. Let $a \in \mf{p}$
    be $M$-regular. Thus $a$ is $M_{\mf{p}}$-regular, so set $M_1 = M/aM$. We
    know that $\dim (M_1)_{\mf{p}} = \dpth_{\mf{p}} M_1$ by the inductive
    hypothesis. This implies that $\dim M_{\mf{p}} = \dpth_{\mf{p}} M$ because
    $\dim (M_1)_{\mf{p}} = \dim M_{\mf{p}} - 1$ and $\dpth_{\mf{p}} M_1 =
    \dpth_{\mf{p}} M - 1$. \qedhere \end{enumerate} \end{proof}

\begin{thm} Let $A$ be local Noetherian and Cohen-Macaulay. Then
    \begin{enumerate} \item For all proper ideals $I \subsetneq A$, we have
        $\mr{ht}(I) + \dim A/I = \dim A$.  \item For all $P \supset Q$ prime
        ideals, then $\mr{ht}(P) - \mr{ht}(Q) = \mr{ht}(P/Q)$, where the last
        height is taken in $A/Q$.  \item Let $x_1, \ldots, x_r \in \mf{m}$.
        Then the following are equivalent: \begin{enumerate} \item $x_1,
            \ldots, x_r$ are $A$-regular.  \item $\mr{ht}(x_1, \ldots, x_i) =
            i$ for all $i \leq r$.  \item $\mr{ht}(x_1, \ldots, x_r) = r$.
\item There exists $x_{r+1}, \ldots, x_n \in \mf{m}$ with $n = \dim A$ such
that $x_1, \ldots, x_n$ is a system of parameters for $A$.  \end{enumerate}
\end{enumerate} \end{thm}

\begin{proof} We will first prove \textbf{3} and then use this to prove
    \textbf{1} and \textbf{2}.  \begin{description} \item[a implies b:] We
        already know that $\mr{ht}(x_1, \ldots, x_i) \leq i$. First, we know
        that $x_1$ does not belong to any minimal prime ideal of $A$ because
        $x_1$ is $A$-regular. Thus $\mr{ht}(x_1) = 1$. Then in $A/x_1$, the
        sequence $\ol{x}_1, \ldots, \ol{x}_r$ is $A/x_1$-regular, and thus
        $\mr{ht}(\ol{x}_2, \ldots, \ol{x}_i) = i-1$. Thus we can find \[
        \mf{p}_i \supsetneq \mf{p}_{i-1} \subsetneq \cdots \supsetneq \mf{p}_1
    \supset (x_1) \] such that $\ol{\mf{p}}_i \supsetneq \cdots \supsetneq
    \ol{\mf{p}}_1 \supsetneq 0$. This implies that $\mr{ht}(\mf{p}_i) \geq 1$
    and thus $\mr{ht}(x_1, \ldots, x_r) \geq i$.  \item[b implies c:] This is
    trivial.  \item[c implies d:] Assume $r < \dim A$. Then there exists
    $x_{r+1} \in \mf{m}$ that does not belong to any minimal prime ideal
    containing $(x_1, \ldots, x_r)$. Therefore $\mr{ht}(x_1, \ldots, x_{r+1}) =
    r+1$ and then we can find $x_{r+2}, \ldots, x_n$ such that $\mr{ht}(x_1,
    \ldots, x_n) = n$. But this implies that $I = (x_1, \ldots, x_n)$ is an
    ideal of definition for $A$ and so $(x_1, \ldots, x_n)$ is a system of
    parameters.  \item[d implies a:] Note that so far we have not used the fact
    that $A$ is Cohen-Macaulay. We show that a system of parameters is
    $A$-regular. Let $\mf{p} \in \Ass(A)$. Then $\dim A/\mf{p} = n$. This
    implies that $\mf{p}$ is minimal and therefore $\ol{x}_1 \neq 0$ in
    $A/\mf{p}$ because otherwise $\ol{x}_1, \ldots, \ol{x}_n$ would be a system
    of parameters in $A/\mf{p}$. But then this means $x_1 \notin \mf{p}$ for
    all $\mf{p} \in \Ass(A)$. This implies that $x_1$ is $A$-regular, and thus
    $A/x_1 A$ is Cohen-Macaulay of dimension $n-1$. On the other hand,
    $\ol{x}_2, \ldots, \ol{x}_n$ is a system of parameters for $A/x_1 A$ and
    the desired result follows by induction.  \end{description} Now we will
    prove \textbf{1} and \textbf{2}.  \begin{enumerate} \item Let $I \subsetneq
        A$ be an ideal with height $\mr{ht}(I) = r$. Then we can choose $a_1
        \ldots, a_r \in I$ such that $\mr{ht}(a_1, \ldots, a_r) = r$. By
        \textbf{3a}, this means that $a_1, \ldots, a_r$ is $A$-regular and thus
        $r \leq \dpth_I (A)$. On the other hand, if we have $b_1, \ldots, b_s
        \in I$ an $A$-regular sequence, we see that $s \leq \mr{ht}(I)$. This
        tells us that $\mr{ht}(I) = \dpth_I A$. But then recall that \[
            \mr{ht}(I) = \inf \qty{\mr{ht}(\mr{p}) \mid \mr{p} \supset I} \]
            and that \[ \dim A/I = \sup \qty{ \dim A/\mf{p}, \mf{p} \supset I
            }. \] This means we can assume that $I = \mr{p}$ is a prime ideal
            with $\mf{p} = r$. In this case, we know that $A_{\mf{p}}$ is
            Cohen-Macaulay and \[ \mr{ht}(\mr{p}) = \dim A_{\mf{p}} =
            \dpth_{\mf{p}} (A) \] so there exists $x_1, \ldots, x_r \in I$ an
            $A$-regular sequence such that $A/(x_1, \ldots, x_r)$ is
            Cohen-Macaulay of dimension $n-r$. Then $\ol{\mf{p}}$ is minimal in
            $A/(x_1, \ldots, x_r)$ because otherwise we would be able to fine
            $x_{r+1} \in \mf{p}$ which is not a zero divisor in $A/(x_1,
            \ldots, x_r)$ and then $\dpth(\mf{p}) \geq r+1$. This implies \[
            \dim A/\mf{p} = \dim A/(x_1, \ldots, x_r) = n-r \] and thus $\dim
            A/\mf{p} + \mr{ht}(\mf{p}) = n-r+r = n = \dim A$.  \item Suppose $P
            \supset Q$ are prime ideals. We know $A_P$ is Cohen-Macaulay. Then
            we know that $\dim A_P = \mr{ht}(QA_P) + \dim (A_P / QA_P)$ by
            \textbf{1} for $A_P$. But then we have the desired result because
            $\dim A_P = \mr{ht}(P)$, $\mr{ht}(Q A_P) = \mr{ht}(Q)$, and $\dim
            (A_P / QA_P) = \mr{ht}(P/Q)$. \qedhere \end{enumerate} \end{proof}


\section{Normal and Regular Rings}% \label{sec:normal_rings}

Let $A$ be a domain and $K = \mr{Frac}(A)$. If $x \in K$, we say that it is
\textit{almost integral} if there exists $a \in A$ such that $ax^m \in A$ for
all $n \geq 1$.

\begin{rmk} Clearly if $x$ is integral, then $x$ is almost integral. Also, if
    $x$ is almost integral and $A$ is Noetherian, $x$ is integral.

    To prove the first part, write \[ x^N = a_0 + a_1 x + \cdots +
    a_{N-1}x^{N-1}. \] Then choose $a$ such that $a x^i \in A$ for $i \leq N-1$
    and then by induction we have $ax^i \in A$ for $i \geq n$.

    To prove the second part, note that if $ax^i \in A$ for all $a \geq 1$,
then $A[x] \subset a^{-1}A$. Because $A$ is noetherian, then $A[x]$ is a
finitely-generated $A$-module and then use the characterization of integral
elements over a ring.  \end{rmk}

\begin{defn} We say that $A$ is a \textit{normal domain (rep. completely normal
domain)} if every integral element (resp. almost integral element) of $K$ over
$A$ belongs to $A$.  \end{defn}

\begin{rmk} Thering $A$ is normal if and only if $A_{\mf{p}}$ is normal for all
prime ideals $\mf{p}$. Additionally, for any normal ring $A$, all localizations
$S^{-1}A$ are normal.  \end{rmk}

We also note that if $A$ is normal (resp.~completely normal) then so is $A[x]$
when $A$ is Noetherian. The proof is easy and uses the fact that $K[x]$ is a
UFD and is thus completely normal. Then for $f \in K(x)$ is almost integral, it
belongs to $K[x]$ and then we write \[ f = a_r x^r + \cdots + a_0 \] where $a_i
\in K$. Then there exists $Q \in A[x]$ such that $Q f^n \in A[x]$ for all $n
\geq 1$ and thus $\alpha_d a_r^n \in A$ for all $n \geq 1$, where $Q = \alpha_d
x^d + \cdots + \alpha_0$. This means that $a_r$ is almost integral and
therefore $a_r \in A$. Then $f - a_r x^r$ is almost integral and finally we
proceed by induction to see that $f \in A[x]$.

Now we consider the normal rings of dimension $1$.  \begin{prop} Let $A$ be a
    local Noetherian ring. Then the following are equivalent: \begin{enumerate}
        \item $A$ is a discrete valuation ring.  \item $A$ is normal and $\dim
            A = 1$.  \item $A$ is normal and there exists $0 \neq a \in \mf{m}$
            such that $(a)$ is $\mf{m}$-primary.  \item The maximal ideal
    $\mf{m}$ of $A$ is principal.  \end{enumerate} \end{prop}

\begin{proof} We will proceed quickly.  \begin{description} \item[1 implies 2:]
    This was done previously.  \item[2 implies 3:] Note that $\mf{m}$ is the
    only prime ideal containing $(a)$ because $A$ has dimension $1$ and is a
    local domain. This means that $(a)$ is $\mf{m}$-primary.  \item[3 implies
    4:] We know that $\mf{m} \in \Ass(A/(a))$ and thus $\mf{m} =
    \Ann_A(\ol{x})$ for some $0 \neq \ol{x} \in A/aA$. If $x \in A$ is a lift
    of $\ol{x}$, then $xa^{-1} \notin A$. However, we know that $\mf{m} \cdot x
    \subseteq (a)$, so $\mf{m}xa^{-1} \subset A$. This tells us that $\mf{m}
    \subsetneq \mf{m} xa^{-1}$ is a strict inclusion because otherwise
    $xa^{-1}$ is integral over $A$ and thus $xa^{-1} \in A$. Therefore $\mf{m}
    xa^{-1} = A$, so there exists $t \in \mf{m}$ such that $txa^{-1} = 1$ and
    thus $tA \subset \mf{m} \ni y$ and thus $y = txa^{-1}y = t(xa^{-1}y)$,
    where $xa^{-1}y \in A$, and thus $y \in tA$, so $\mf{m} = tA$ is principal.
\item[4 implies 1:] If $\mf{m} = (t)$, then we know that \[ \bigcap_{n \geq 0}
    \mf{m}^n = 0 \] and so for all $x \neq 0$ there exists $m$ such that $x = u
    t^m$ where $u \in A^{\times}$, and thus $A$ is a DVR. \qedhere
    \end{description} \end{proof}

Now let $A$ be a ring and $I \subset A$ such that $\bigcap_n I^n = (0)$ and $A$
is $I$-separated. Then for all $0 \neq a \in A$, there exists $n$ such that $a
\in I^n \setminus I^{n+1}$. Then define $a^*$ to be the class of $a$ in $I^n /
I^{n+1}$ and \[ A^* = \mr{gr}^I(A) = \bigoplus_{n=0}^{\infty} I^n / I^{n+1}. \]

\begin{thm} Let $A, I$ be as before. Then \begin{enumerate} \item If $A^*$ is a
domain, then so is $A$.  \item If $A$ is Noetherian and $I \subset
\mr{rad}(A)$, then if $A^*$ is a normal domain, so is $A$.  \end{enumerate}
\end{thm}

\begin{proof} \begin{enumerate} \item Choose $a,b \in A$ nonzero. Then $a^*,
    b^* \neq 0$ and so $a^* b^* \neq 0$ because $A^*$ is a domain. This means
    that if $a \in I^n \setminus I^{n+1}, b \in I^q \setminus I^{q+1}$, we know
    that $ab \in I^{n+q} \setminus I^{n+q+1}$ and thus $ab \neq 0$.  \item Let
    $x \in K$ be integral over $A$. Write $x = \frac{a}{b}$ where $a,b \in A$.
    Then $A/bA$ is Noetherian and therefore by the Krull intersection theorem
    this is the same thing as \[ \bigcap_{n \geq 0} I^n + bA = bA. \] To show
    that $x \in A$, we will prove that $a \in bA$ or that $a \in I^n + bA$ for
    all $n \geq 0$. If $a \in I^{n-1} + bA$, then we know that $a = a' + bs$,
    where $a' \in I^{n-1}$ and $s \in A$. Then we know that because
    $\frac{a}{b}$ is integral, so is $\frac{a'}{b}$. Then we may assume that $a
    \in I^{n-1}$ and $\frac{a}{b}$ is almost integral. Then there exists $c \in
    A$ such that $ca^m \in b^m A$ for all $m \geq 1$ and thus because $a^* \neq
    0$, we know that $(a^*)^m = (a^m)^* \neq 0$ and thus $c^* (a^*)^m \in
    (b^*)^m A$. Because $A^*$ is Noetherian and normal, it is completely
    normal, and thus $\frac{a^*}{b^*} \in A^*$ and thus $a^* = b^* d^*$ for
    some $0 \neq d \in A$. This means that $a-bd \in I^{N+1}$ for $N$ such that
    $a \in I^N \setminus I^{N+1}$. But we know this for $N = n-1$, and thus
    $a-bd \in I^n$, so $a \in I^n + bA$. \qedhere \end{enumerate} \end{proof}

Recall that if $A$ is local Noetherian, we say that $A$ is \textit{regular} if
there exists a system of parameters that generates the maximal idea $\mf{m}$ of
$A$.

\begin{thm} Let $A$ be local Noetherian, $\mf{m}$ be its maximal ideal, and $k$
    be the residue field. Then $A$ is regular if and only if \[
    \mr{gr}^{\mf{m}}(A) \cong k[x_1, \ldots, x_d], \] where $d = \dim A$.
\end{thm}

\begin{proof} Assume that $A$ is regular. Then $\mf{m} = (x_1, \ldots, x_d)$ is
    a system of parameters because $\dim_k \mf{m} / \mf{m}^2 = \dim A = d$.
    Therefore we have a map \[ k[x_1, \ldots, x_d] \longrightarrow
    \mr{gr}^{\mf{m}}(A) \qquad Q(X_1, \ldots, X_n) \mapsto Q(x_1, \ldots, x_d).
\] This map is clearly surjective, so we need to show that it is injective. If
$Q \neq 0$ is in the kernel, then let $q = \deg Q$. Then $(Q) \subset \ker$. We
know that \[ \ell(A / \mf{m}^{n+1}) \leq \binom{n+d}{d} - \binom{n-q+d}{d} \]
and thus $\dim A = d(A) \leq d-1$, a contradiction.

    Now assume that $\mr{gr}^{\mf{m}}(A) \cong k[x_1, \ldots, x_d]$. By
definition, this means that $d(A) = d$. Also $\dim \mf{m} / \mf{m}^2$ is the
dimension of the space of homogeneous polynomials of degree $1$, so $\dim
\mf{m} / \mf{m}^2 = d$. Thus $A$ is regular.  \end{proof}

\begin{cor} Let $A$ be a local Noetherian ring. Then the following are
equivalent: \begin{enumerate} \item $A$ is a discrete valuation ring.  \item
$A$ is a normal ring of dimension $1$.  \item $A$ is regular of dimension $1$.
\end{enumerate} \end{cor}

\begin{proof} We have seen before that \textbf{1} is equivalent to \textbf{2}
    which implies \textbf{3}. To prove that \textbf{3} implies \textbf{1}, we
    need to show that $A$ is a domain. But then $\mr{gr}^{\mf{m}}(A) \cong
    k[X]$ is a domain and thus $A$ is a domain.    \end{proof}

\begin{cor} If $A$ is regular, then $A$ is a normal domain.  \end{cor}

\begin{thm} Let $A$ be a regular local Noetherian ring and $(x_1, \ldots, x_d)$
    be a regular system of parameters. Then \begin{enumerate} \item $A$ is a
        normal domain.  \item $x_1, \ldots, x_d$ is an $A$-regular sequence and
        $A$ is Cohen-Macaulay.  \item The ideal $\mf{p}_i = (x_1, \ldots, x_i)$
        is a prime ideal of height $i$ and $A/ \mf{p}_i$ is regular of
        dimension $d-i$.  \item If $\mf{p} \in \Spec(A)$ and $A/ \mf{p}$ is
        regular, then there exists a system of parameters $y_1, \ldots, y_d$
        such that $\mf{p} = (y_1, \ldots, y_i)$ and $\dim A/\mf{p} = d-i$.
\end{enumerate} \end{thm}

\begin{proof} We prove each part.  \begin{enumerate} \item We already proved
    that the graded ring $\mr{gr}^{\mf{m}}(A)$ is a normal domain.  \item  Use
    the previous theorem. Recall that $k[x_1, \ldots, x_d] \simeq
    \mr{gr}^{\mf{m}}(A)$. This means that $x_1, \ldots, x_d$ is an
    $A$-quasi-regular sequence. Because $A$ is Noetherian, then $x_1, \ldots,
    x_d$ is an $A$-regular sequence. This implies that \[ d \leq \dpth(A) \leq
    d \] and thus $A$ is Cohen-Macaulay.  \item The previous implies that $\dim
    A / (x_1, \ldots, x_i) = d-i$. Then the ideal $\ol{\mf{m}} = \mf{m} /
    \mf{p}_i$ is generated by the $d-i$ elements $\ol{x}_{i+1}, \ldots,
    \ol{x}_d$ and these form a system of parameters. By definition, $A/
    \mf{p}_i$ is regular.  \item Write $\ol{\mf{m}} = \mf{m} / \mf{p}$. Then
    let $i \leq d$ such that $\dim \ol{\mf{m}} / \ol{\mf{m}}^2 = d-i$. This
    means that $i = \dim ( \mf{p}+\mf{m}^2  )/ \mf{m^2}$, so we can choose
    $y_1, \ldots, y_i \in \mf{p}$ such that $\ol{y}_1, \ldots, \ol{y}_i$ is a
    basis of $(\mf{p} + \mf{m}^2 / \mf{m}^2)$. We then complete this to a basis
    of $\mf{m} / \mf{m}^2$. Thus $y_1, \ldots, y_d$ is a regular system of
    parameters for $A$, and is thus is regular sequence. Then $\mf{p}' = (y_1,
    \ldots, y_i) \subset \mf{p}$ is a prime ideal and $\dim A/\mf{p}' = d-i$.
    But then $\dim A/\mf{p} = p-i$ because $A/\mf{p}$ is regular by assumption,
    so we must have $\mf{p} = \mf{p}'$. \qedhere \end{enumerate} \end{proof}

We now return to some definitions.

\begin{defn} Let $M$ be an $A$-module. Then $\mf{p} \in \Ass(M)$ is called an
\textit{embedded prime} if it is not minimal in the support.  \end{defn}

Recall that if $M$ is Cohen-Macaulay, then for any $\mf{p} \in \Ass(M)$, then
$\dim A/\mf{p} = \dim M$. Therefore $M$ has no embedded primes.

\begin{defn} Let $I$ be an ideal of $A$. We say that $I$ is \textit{unmixed} if
for all $\mf{p} \in \Ass(A/I)$ we have $\mr{ht}(\mf{p}) = \mr{ht}(I)$. Note
this implies that $A/I$ has no embedded primes.  \end{defn}

\begin{defn} We say that the \textit{unmixed theorem holds for $A$} if for all
$I = (a_1, \ldots, a_r)$ with $r = \mr{ht}(I)$, then $I$ is unmixed.
\end{defn}

\begin{rmk} This is equivalent to saying that for all $I = (x_1, \ldots, x_r)$
    of height $\mr{ht}(I) = r$, $A/I$ has no embedded primes.

    To see this, note that if $A/I$ has no embedded primes, choose a minimal $Q
    \in \Ass(A/I)$. We know that $\mr{ht}(Q) \leq r$ because $Q$ is minimal
    containing $I$. However, this is at least \[ \inf \qty{ \mr{ht}(\mf{p})
    \mid \mr{p} \supset I } = \mr{ht}(I) = r. \] \end{rmk}

\begin{thm} Let $A$ be Noetherian. Then $A$ is Cohen-Macaulay if and only if
the unmixed theorem holds for $A$.  \end{thm}

\begin{proof} Suppose that the unmixed theorem holds for $A$. Let $\mf{p}$ be a
    prime ideal. Then we know $r = \mr{ht}(\mf{p}) = \dim A_{\mr{p}}$, so we
    can choose $a_1, \ldots, a_r \in \mf{p}$ such that $\mr{ht}((a_1, \ldots,
    a_i)) = i$. However, if $A$ satisfies the unmixed theorem, then $a_{i+1}$
    is not in any prime ideal of $\Ass(A/(a_1, \ldots, a_i))$. Here, we use the
    fact that $A/(a_1, \ldots, a_i)$ has no embedded primes because $(a_1,
    \ldots, a_i)$ is unmixed. Thus $a_1, \ldots, a_r$ is an $A$-regular
    sequence and is also $A_{\mf{p}}$-regular. This implies that $r \leq
    \dpth(A_{\mf{p}}) \leq \dim A_{\mr{p}} = \mr{ht}(\mr{p}) = r$ and thus
    $A_{\mf{p}}$ is Cohen-Macaulay.

    Now assume that $A$ is Cohen-Macaulay. Then if $I = (a_1, \ldots, a_r)$ has
height $r$, then $a_1, \ldots, a_r$ is an $A$-regular sequence. Thus $A/(a_1,
\ldots, a_r)$ is Cohen-Macaulay and thus has no embedded primes, so $(a_1,
\ldots, a_r)$ is unmixed.  \end{proof}

Returning to normal domains, we have the following result: \begin{thm} Let $A$
    be a Noetherian normal domain. Then any nonzero principal ideal is unmixed
    and \[ A = \bigcap_{\mr{ht}(\mr{p}) = 1} A_{\mf{p}}. \] In particular, $A$
    is Cohen-Macaulay if $\dim A \leq 2$.  \end{thm}

\begin{proof} Let $0 \neq a \in A$. To show that $(a)$ is unmixed, we need to
    show that $\mr{ht}((a)) = 1$. Let $\mf{p} \in \Ass(A/(a))$. Then $\mf{p} =
    \Ann_A(\ol{b})$ for some $b \neq 0$. This means that \begin{align*} \mf{p}
    &= \qty{x \in K \mid xb \subset (a)} \\ &= \qty{x \in K \mid \frac{xb}{a}
    \in A}.  \end{align*} This means that \[ \mf{p}^{-1} = \qty{y \in K \mid y
    \mf{p} \subset A} \] and thus we can replace $A$ by $A_{\mf{p}}$. This
    implies that $\mf{p}^{-1} A_{\mf{p}} \neq A_{\mf{p}}$, so \[ \mf{p}
    A_{\mf{p}} \subset \mf{p}^{-1} A_{\mf{p}} \cdot \mf{p} A_{\mf{p}} \subset
A_{\mf{p}}. \] Then one of these two inclusions is an equality, so if the first
inclusion is an equality, we know that elements of $\mf{p}^{-1}A_{\mf{p}}$ are
integral over $A_{\mf{p}}$ and thus $\mf{p}^{-1}A_{\mf{p}} \subset A_{\mf{p}}$
because $A$ is a normal domain. This is impossible, so $\mf{p}^{-1}A_{\mf{p}}
\cdot \mf{p}A_{\mf{p}} = A_{\mf{p}}$ and thus $\mf{p}A_{\mf{p}}$ is a principal
ideal. Therefore $A_{\mf{p}}$ is a DVR, so $\mr{ht}(\mf{p}) = 1$. Thus $(a)$ is
unmixed.

    Now we will prove that $A$ is the intersection of the localizations at the
height $1$ prime ideals. One direction is obvious, so now let $\frac{b}{a} \in
\bigcap_{\mr{ht}(\mf{p})=1} A_{\mf{p}}$. Then if $\frac{b}{a} \notin A$, we
know that $A/(a) \neq 0$ and thus there exists $\mf{p}$ such that $\Ass(A/(a))
\neq \emptyset$. We know that $\mr{ht}(\mf{p}) = 1$, so $A_{\mf{p}} \neq \mf{p}
= \Ann(b)$, a contradiction with the fact that $\frac{b}{a} \in A_{\mf{p}}$.
\end{proof}

\section{Homological Theory}% \label{sec:homological_theory}

\begin{lem} Let $A$ be a ring and $M$ be an $A$-module. Then \begin{enumerate}
    \item $M$ is projective if and only if $\Ext_A^1(M,N) = 0$ for all
        $A$-modules $N$.  \item $M$ is injective if and only if $\Ext_A^1(A/I,
        M) = 0$ for all ideals $I \subset A$.  \end{enumerate} \end{lem}

\begin{proof} We only need to prove the second part. One direction follows from
    the direction, so now suppose we have $0 \to N \to N'$ injective. We need
    to show that $\Hom(N',M) \to \Hom(N,M)$ is surjective. Consider the set \[
    \mc{S} = \qty{N \subset Q \subset N' \mid \Hom(Q,M) \twoheadrightarrow
\Hom(N,M)}. \] We will show that $N' \in \mc{S}$. First, $\mc{S}$ contains a
maximal element $N_1$ by an application of Zorn's lemma. If $N_{\alpha}$ is a
chain, then $\bigcup_{\alpha} N_{\alpha} \in \mc{S}$. Now we show that $N_1 =
N'$. Otherwise, there exists $x \in N' \setminus N_1$ and set $N_2 = N_1 + Ax
\subset N'$. This gives an exact sequence \[ 0 \to N_1 \to N_2 \to A/I \to 0.
\] Now we apply the functor $\Hom(-,M)$ and we see that $\Hom(N_2, M)
\twoheadrightarrow \Hom(N_1, M)$ because $\Ext^1(A/I, M) = 0$. But this implies
that $N_2 \in \mc{S}$ and this contradicts maximality of $N_1$.  \end{proof}

\begin{defn} Let $A$ be a ring and $M$ an $A$-module. Then we define the
    \textit{projective dimension} $\operatorname{proj.dim} M$ to be the length
    of the shortest projective resolution of $M$. Equivalently, we define the
    \textit{injective dimension} $\operatorname{inj.dim} M$ to be the length of
    the shortest injective resolution of $M$.  \end{defn}

\begin{lem} Let $n$ be a positive integer. Then the following are equivalent:
    \begin{enumerate} \item All $A$-modules $M$ have projective dimension at
        most $n$; \item All $A$-modules $M$ of finite type have projective
        dimension at most $n$; \item All $A$-modules $M$ have injective
dimension at most $n$; \item $\Ext_A^{n+1}(M,N) = 0$ for all $A$-modules $M,N$.
\end{enumerate} \end{lem}

\begin{proof} Clearly \textbf{1} implies \textbf{2}.  \begin{description}
    \item[2 implies 3:] Consider \[ 0 \to M \to U_0 \to U_1 \to \cdots \to
        U_{n-1} \to C \to 0 \] such that $U_0, \ldots, U_{n-1}$ are injective.
        Now we have an exact sequence \[ 0 \to M \to U_0 \to V_0 \to 0 \] where
        $V_0$ is the kernel of $U_1 \to U_2$. For any ideal $I$, we know that
        $\Ext^{n+1}(A/I, M) = \Ext^n(A/I, V_0)$ because $U_0$ is injective. Now
        we obtain that $\Ext^n(A/I, V_0) = \Ext^{n-1}(A/I, V_1)$ where $V_1 =
        \ker (U_2 \to U_3)$. This implies that \[ \Ext_A^{n+1}(A/I, M) =
        \Ext_A^1(A/I, C). \] By \textbf{2}, we know that $\Ext^{n+1}(A/I, M) =
        0$ and therefore $C$ is injective. Thus there is an injective
        resolution of $M$ of length $n$.  \item[3 implies 4:] This is trivial.
        \item[4 implies 1:] The proof is similar to the proof that \textbf{2}
    implies \textbf{3}. \qedhere \end{description} \end{proof}

\begin{cor} The two numbers \[ \sup_M( \operatorname{proj.dim} M ) = \sup_M
(\operatorname{inj.dim} M) \] are equal and are called the \textit{global
homological dimension} of $A$.  \end{cor}

\begin{lem} Suppose $A$ is Noetherian and $M$ is an $A$-module of finite type.
Then $M$ is projective if and only if $\Ext^1(M,N) = 0$ for all $N$ of finite
type.  \end{lem}

\begin{proof} Because $M$ is finite type, we can write \[ 0 \to K
\xrightarrow{\psi} A^n \to M \to 0 \] where $K$ is finite type. Then if we
apply $\Ext_A(-,K)$, we see that $\Ext^1(M, K) = 0$ and thus $\Hom(A^n,K) \to
\Hom(K,K)$ is surjective. Thus there exists $\phi \colon A^n \to K$ such that
$\phi \circ \psi = \mr{id}_K$ and thus the exact sequence is split. Thus $M$ is
a direct factor of a free module and is thus projective.  \end{proof}

\begin{lem} Let $A$ be a local Noetherian ring and $M$ be a finite $A$-module.
Then $\operatorname{proj.dim} M \leq n$ if and only if $\Tor_{n+1}^A(M, k) =
0$.  \end{lem}

\begin{proof} Let $P_{\bullet} \to M \to 0$ be a projective resolution of
    length at most $n$. Then using this resolution, we compute $\Tor_{n+1}(M,k)
    = 0$. 

    In the other direction, first assume that $n=0$. If $\Tor_1^A(M,k) = 0$, we
    want to show that $M$ is projective. Because $M$ is finite type, then there
    exists $A^r \twoheadrightarrow M$ such that $A^r \otimes k \to M \otimes k$
    is an isomorphism. Applying the $- \otimes k$ functor to \[ 0 \to N \to A^r
    \to M \to 0, \] we obtain the exact sequence \[ 0 \to \Tor_1^A(M,k) \to N
\otimes k \to k^r \to M \otimes k \to 0 \] and this implies $N \otimes k = 0$.
By Nakayama's lemma, we see that $N = 0$ and thus $M$ is free. The general case
is similar. We construct \[ 0 \to C \to F_n \to \cdots \to F_0 \to M \to 0 \]
where each $F_i$ is projective, and then we see that $\Tor_0(C,k) = \cdots =
\Tor_{n+1}(M,k) = 0$. This implies $C \otimes k = 0$, so $C = 0$ and thus we
have a projective resolution of length $n$.  \end{proof}

\begin{cor} Let $A$ be Noetherian and $M$ be a finite $A$-module. Then
    \begin{enumerate} \item The projective dimension of $M$ is the supremum of
        the projective dimensions of $M_{\mf{m}}$ for all maximal ideals
        $\mf{m}$ of $A$.  \item $\pdim M \leq n$ if and only if $\Tor_{n+1}(M,
        A/\mf{m}) = 0$ for all maximal ideals $\mf{m}$ of $A$.  \end{enumerate}
    \end{cor}

\begin{proof} \begin{enumerate} \item This follows from the fact that
    $\Ext_A^i(M,N)_{\mf{m}} = \Ext_{A_{\mf{m}}}^i(M_{\mf{m}}, N_{\mf{m}})$ and
    the fact that for an $A$-module $M$ we have $M = 0$ if and only if
    $M_{\mf{m}} = 0$ for all $\mf{m}$ maximal.  \item This follows from
    \textbf{1}, the previous lemma, and the fact that $\Tor$ commutes with
    localization. \qedhere \end{enumerate} \end{proof}

\begin{lem} Let $A$ be noetherian. The following are equivalent:
    \begin{enumerate} \item $\gdim A \leq n$.  \item $\pdim M \leq n$ for all
        finite $A$-modules $M$.  \item $\idim M \leq n$ for all finite
        $A$-modules $M$.  \item $\Ext_A^{n+1}(M,N) = 0$ for all finite
        $A$-modules $M,N$.  \item $\Tor_{n+1}^A(M, N) = 0$ for all finite
$A$-modules $M,N$.  \end{enumerate} \end{lem}

\begin{proof} We know that \textbf{1} is equivalent to \textbf{2}, which
    implies \textbf{3}. We know that \textbf{3} implies \textbf{4} and
    \textbf{2} implies \textbf{5} are trivial. The previous corollary gives us
    \textbf{5} implies \textbf{2}, so we need to prove that \textbf{4} implies
    \textbf{2}. Let \[ 0 \to C \to F_{n-1} \to \cdots \to F_0 \to M \to 0 \]
    such that $F_i$ are free of finite type. Then we know that \[ \Ext^1(C, N)
    = \cdots = \Ext_A^{n+1}(M,N) = 0. \] This implies that $C$ is projective
    and thus $\pdim M \leq n$.  \end{proof}

\begin{cor} We have the local-global identity \[ \gdim A = \sup_{\mf{m} \text{
maximal}} \gdim A_{\mf{m}}. \] \end{cor}

\begin{thm} Let $A$ be local Noetherian with residue field $k$. Then $\gdim A
\leq n$ if and only if $\Tor_{n+1}^A(k,k) = 0$. In particular, $\gdim A =
\pdim_A k$.  \end{thm}

\begin{proof} Suppose that $\Tor_{n+1}(k,k) = 0$. Thus $\pdim k \leq n$, and
thus $\Tor_{n+1}(M, k) = 0$ for all $M$ of finite type. This implies that
$\pdim M \leq n$ for all finite $A$-modules $M$. Thus $\gdim A \leq n$.
\end{proof}

\begin{lem} Let $(A, \mf{m}, k)$ be local Noetherian and $M$ a finite
$A$-module. Let $x \in A$ be an $M$-regular element. Then if $\pdim M \leq
\infty$, so is $\pdim M/xM$ and $\pdim M/xM = \pdim M + 1$.  \end{lem}

\begin{proof} Let $r = \pdim M < \infty$. Then we have the long exact sequence
    \[ 0 \to M \to M \to M/xM \to 0, \] so after applying the functor $-
    \otimes k$, we know that $\Tor_i(M,k) = 0$ if $i > r$. Thus if $i > r+1$,
    then $\Tor_i(M/xM, k) = 0$. For $i = r+1$, we have an exact sequence \[ 0
    \to \Tor_{r+1}(M/xM, k) \to \Tor_r(M,k) \to \Tor_r(M,k) \] but $x \in M$
    annihlates $k$ and thus $\Tor_{r+1}(M/xM, k) \simeq \Tor_r(M,k) \neq 0$, so
    $\pdim(M/xM) = r+1$.  \end{proof}

\begin{thm} Let $(A,\mf{m}, k)$ be a regular local ring of dimension $n$. Then
$\gdim A = n$.  \end{thm}

\begin{proof} Let $x_1, \ldots, x_n$ be a regular system of parameters. Then we
    know $k = A/(x_1, \ldots, x_n)$. This implies that \[ \pdim k = n + \pdim A
    = n \] and thus $\gdim A = n$.  \end{proof}

\begin{cor} Let $k$ be a field. Then $\gdim k[x_1, \ldots, x_n] = n$.
\end{cor}

\begin{proof} Let $A = k[x_1, \ldots, x_n]$. Then $A_{\mf{m}} \cong k'[T_1,
    \ldots, T_n]$ for some other field $k'$ and is thus regular of dimension
    $n$. Thus $\gdim A_{\mf{m}} = n$ for all maximal ideals $\mf{m}$, and thus
    $\gdim A = n$.  \end{proof}

\section{Koszul Complex}% \label{sec:koszul_complex}

Our goal is to prove the following result: \begin{thm}[Serre] \label{thm:rlrgd}
A local Noetherian ring is regular if and only if its global dimension is
finite.  \end{thm}

To do this, we will introduce the \textit{Koszul complex}. First, recall that
if $L_{\bullet}, M_{\bullet}$ are two complexes, $(L_{\bullet} \otimes
M_{\bullet})_n = \bigoplus_{i+j = n} L_i \otimes M_j$ and the differential
satisfies the graded Leibniz rule \[ \dd(a \otimes b) = \dd{a} \otimes b +
(-1)^{\abs{a}} a \otimes \dd{b}. \] For $x \in A$, define $K(x) = \to 0 \to A
\xrightarrow{\times x} A \to 0 \to  \cdots$. We can treat any $A$-module $M$ as
a complex by putting it in degree $0$. Now for $x_1, \ldots, x_r \in A$ and $M$
an $A$-module, define \[ K(x_1, \ldots, x_r, M) = K(x_1) \otimes K(x_2) \otimes
    \cdots \otimes K(x_r) \otimes M. \] Then for an integer $p$, we have \[
K_p(x_1, \ldots, x_r, M) = \bigoplus_{1 \leq i_1 \leq \cdots \leq i_p \leq r} M
e_{i_1, \dots, i_p} \cong M^{\binom{r}{p}}. \] This satisfies \[ \dd{(m e_{i_1,
\ldots, r_p})} = \sum_{j=1}^p (-1)^{j-1} x_{i_j} m e_{i_1, \ldots, \wh{i}_j,
\ldots, i_p}. \] Now consider the exact sequence \[ 0 \to A \to K(x) \to A[-1]
\to 0 \] Now if $C$ is a complex, write $C_{\bullet}(x) = C_{\bullet} \otimes
K(x)$, so we have a short exact sequence \[ 0 \to C_{\bullet} \to
    C_{\bullet}(x) \to C_{\bullet}[-1] \to 0. \] This gives us a long exact
    sequence of modules \[ \cdots \to H_{p+1}(C) \to H_{p+1}(C(x)) \to H_p(C)
    \xrightarrow{\delta} H_p(C) \to \cdots \] By definition of $K(x)$ it is
    easy to see that $\delta_p = (-1)^p x$. As a consequence, if
    $H_p(C_{\bullet})$ is trivial, then so is $H_p(C_{\bullet}(x))$ for $p >
    1$. In addition, we have \[ 0 \to H_1(C(x)) \to H_0(C) \xrightarrow{x}
    H_0(C) \to H_0(C(x)) \to 0. \] Thus if $x$ is $H_0(C)$-regular, then
    $H_1(C(x)) = 0$ and $H_0(C(x)) = H_0(C) / x H_0(C)$. If we iterate this, we
    have now proven the following result:

\begin{thm} Let $M$ be an $A$-module and $x_1, \ldots, x_r$ an $M$-regular
    sequence. Then \[ H_p(\ul{x}, M) \coloneqq H_p(K(x_1, \ldots, x_r, M)) = 0
    \] for all $p > 0$ and $H_0(\ul{x}, M) = M/(x_1, \ldots, x_r)M$.  \end{thm}

\begin{cor} If $x_1, \ldots, x_r$ is an $A$-regular sequence, then $K(x_1,
\ldots, x_r, A)$ is a finite free resolution of $A/(x_1, \ldots, x_r) A$.
\end{cor}

Now we will discuss minimal resolutions. Suppose $(A, \mc{m}, k)$ is a local
ring. Then recall from the homework that $u \colon L \to M$ is called
\textit{minimal} when $u \otimes 1_k$ is an isomorphism. When $M$ is finite,
this is equivalent to $u$ surjective and $\ker(u) \subset \mf{m} L$. We now say
that a \textbf{resolution} $L_{\bullet} \to M$ is \textit{minimal} if $L_i
\otimes k \simeq \ker (\dd{i}) \otimes k$, where $L_i \xrightarrow{\dd{i}}
L_{i-1}$. In particular, writing $\ol{L}_i = L_i \otimes k$, then the complex
\[ \ol{L}_i \to \ol{L}_{i-1} \to \cdots \] satisfies $d = 0$, so $\Tor_i(M, k)
= H_i(\ol{L}_{\bullet}) = \ol{L}_i$. Therefore if $L_{\bullet}$ is a minimal
resolution of $M$, then $\mr{rk}(L_i) = \dim_k \Tor_i(M, k)$.

\begin{prop} Now assume $A$ is local Noetherian and $M$ is a finite $A$-module.
Then there exists a minimal free resolution of $M$ and any two such resolutions
are isomorphic.  \end{prop}

\begin{proof} To prove existence, begin with $\ol{M}$. Choose a basis $e_1,
    \ldots, e_{n_0} \in M$ of $M \otimes k = \ol{M}$. This gives us a morphism
    $0 \to K_0 \to A^{n_0} \to M \to 0$ that gives an isomorphism $k^n \simeq
    \ol{M}$. Now $K_0$ is a finite $A$-module, so we can construct a minimal
    $A^{n_1} \to K_0$ with kernel $K_1$. Iterating this procedure, we have a
    minimal resolution.

    To prove uniqueness, if $L_{\bullet}, L'_{\bullet}$ are minimal free
    resolutions, we will construct a map \begin{equation*} \begin{tikzcd}
    L_{\bullet} \arrow{r} \arrow{d} & M \arrow{d}{\mr{id}} \\ L'_{\bullet}
\arrow{r} & M \end{tikzcd} \end{equation*} that makes this diagram commute. In
degree zero, we have $f_0 \colon L_0 \to L_0'$ where $\ol{f}_0$ is an
isomorphism, so $L_0 \cong L_0' = A^n$. Thus $\det f_0 \in A^{\times}$, so
$f_0$ is an isomorphism. The rest follows by iterating this procedure.
\end{proof}

\begin{lem} Let $F_{\bullet} \to M$ be a complex such that \begin{enumerate}
    \item $F_p$ is a finite free $A$-module for all $p$; \item $\ol{F}_0 \to
        \ol{M}$ is surjective.  \item $\dd{(F_i)} \subset \mf{m} F_{i-1}$ for
        all $i > 0$ and induces a surjection $\ol{F}_i \hookrightarrow \mf{m} /
        \mf{m}^2 \otimes \ol{F}_{i-1} = \mf{m} \otimes \ol{F}_{i-1}$.
\end{enumerate} Then $\mr{rk}_A(F_i) \leq \dim_k \Tor_i(M, k)$.  \end{lem}

\begin{proof} Consider a minimal free resolution $L_{\bullet} \to M$ of $M$.
    Because free implies projective, we can find a map \begin{equation*}
        \begin{tikzcd} F_{\bullet} \arrow{r} \arrow{d} & M \arrow{d}{\mr{id}}
            \\ L_{\bullet} \arrow{r} & M.  \end{tikzcd} \end{equation*} Then
            the map $\ol{F}_0 \to \ol{L}_0$ must be injective because $\ol{F}_0
            \hookrightarrow \ol{M}$ and $\ol{L}_0 \simeq \ol{M}$. Therefore, we
            can split $L_0 \cong F_0 \oplus F_0'$. In particular, this means
            that $\mr{rk}_A(F_0) \leq \mr{rk}_A(L_0)$.

    Now we have a diagram \begin{equation*} \begin{tikzcd} F_1 \arrow{r}
    \arrow{d} & \mf{m} F_0 \arrow{d} \\ L_1 \arrow{r} & \mf{m} L_0 \end{tikzcd}
        \end{equation*} where we have $\mf{m}L_0 = \mf{m} F_0 \oplus \mf{m}
        F_0'$, so downstairs we now have \begin{equation*} \begin{tikzcd}
        \ol{F}_1 \arrow[hookrightarrow]{r} \arrow{d} & \ol{\mf{m}} \otimes F_0
    \arrow{d} \\ \ol{L}_1 \arrow{r} & \ol{\mf{m}} \otimes L_0 \end{tikzcd}
\end{equation*} and therefore $\ol{F}_1 \to \ol{L}_1$ is surjective. This gives
us a splitting $L_1 = F_1 \oplus F_1'$ and thus iterating this, we obtain the
desired result.  \end{proof}

\begin{thm} Let $(A, \mf{m}, k)$ be a local Noetherian ring and let $s = \dim
    \mf{m} / \mf{m}^2$. Then \[ \operatorname{rk} \Tor_i(k,k) \geq
    \binom{s}{i}. \] In particular, $\Tor_i(k, k) \neq 0$.  \end{thm}

\begin{proof} Write $F_i \coloneqq K_i(x_1, \ldots, x_s; A)$ where $x_1,
    \ldots, x_s \in \mf{m}$ is a basis for $\mf{m} / \mf{m}^2$. Thus we have
    $F_0 = A \to H_0(\ul{x}, A) = A/(x_1, \ldots, x_s) = k$. Thus the
    $F_{\bullet}$ satisfies the first two hypotheses of the previous lemma, so
    we check the third one. Here, we have \[ \dd \qty(\sum \lambda_{i_1,
    \ldots, i_p} e_{i_1, \ldots, i_p}) = \sum_{j=1}^s (-1)^j x_j \qty(\sum)
\lambda_{i_1, \ldots, \wh{i}_j, \ldots, i_p} e_{i_1, \ldots, \wh{i}_j, \ldots,
i_p}. \] Reducing modulo $\mf{m}$, because $\ol{x}_1, \ldots, \ol{x}_s$ is a
basis of $\mf{m} = \mf{m} / \mf{m}/\mf{m}^2$, we see that this differential is
injective. Now we can apply the previous lemma to obtain the desired result.
\end{proof}

We are now ready to prove the theorem of Serre.

\begin{proof}[Proof of Theorem~\ref{thm:rlrgd}] We proved one direction last
    time. Now set $s = \dim \mf{m} / \mf{m}^2$. We know that $\Tor_s(k,k) \neq
    0$ and thus $s \leq \mr{gl.dim} A = \mr{proj.dim}_A(k)$. On the other hand,
    by the Auslander-Buchsbaum formula, we have \[ \mr{proj.dim}_A(k) + \dpth k
    = \dpth A \] and because $\dpth k = 0$, we see that $\dim A \geq \dpth A
    \geq s$. However, we know that $s \geq \dim A$, and thus $s = \dim A =
    \dpth A$, so $A$ is regular.  \end{proof}

\begin{cor} Let $A$ be a regular local ring and choose $\mf{p} \in \Spec A$.
Then $A_{\mf{p}}$ is regular.  \end{cor}

\begin{proof} This requires showing that $A_{\mf{p}}$ has finite global
    dimension. If $M$ is an $A_{\mf{p}}$-module, then $M$ is an $A$-module.
    Then if $F_{\bullet} \to M$ is a projective resolution of $M$ of length at
    most $\dim A$, exactness of localization tells us that
    $(F_{\bullet})_{A_{\mf{p}}} \to M$ is an $A_{\mf{p}}$-projective resolution
    of $M$. Thus $\mr{proj.dim}_{A_{\mf{p}}}(M) < \dim A$, so $A_{\mf{p}}$ has
    finite global dimension.  \end{proof}

\section{Unique Factorization}% \label{sec:unique_factorization}

Recall that a \textit{unique factorization domain}, or a UFD, is a domain $A$
such that every element has a unique factorization into irreducibles (up to
multiplication of irreducibles by units). Here, $\pi \neq 0$ is irreducible if
$\pi = ab$ implies $a$ or $b$ is a unit. For example, when $A = \Z$, the
irreducibles are precisely the prime numbers.

\begin{exer} If $A$ is Noetherian, then any element can be written as a product
of irreducibles.  \end{exer}

We say that two irreducibles $\pi, \pi'$ are \textit{equivalent} if $(\pi) =
(\pi')$. Then for each equivalence class, we can choose one representative, and
the unique factorization means that if \[ u' \pi_1^{n_1'} \cdots \pi_r^{n_r'} =
u \pi_1^{n_1} \cdots \pi_r^{n_r}, \] then $u = u'$ and $n_i = n_i'$.

\begin{lem} A Noetherian domain is a UFD if and only if $(\pi)$ is a prime
ideal for all irreducibles $\pi$.  \end{lem}

\begin{proof} Suppose $ab \in (\pi)$. Then if $a = u \prod_i \pi_i^{n_i}, b = v
    \prod_i \pi_i^{n_i}$ where $\pi = \pi_1$, we know that $n_1 + m_1 \geq 1$,
    so at least one of them is at least $1$, and thus either $a \in (\pi)$ or
    $b \in (\pi)$.

    In the other direction, we will prove that Gauss's lemma is satisfied. This
says that if $\pi \mid ab$ and $\pi \nmid a$, then $\pi \mid b$. This implies
that our domain is a UFD, and is equivalent to $(\pi)$ being a prime ideal for
all units.  \end{proof}

\begin{lem} Let $A$ be a Noetherian domain. Then $A$ is a UFD if and only if
every prime ideal of height $1$ is principal.  \end{lem}

\begin{proof} Suppose $A$ is a unique factorization domain. Then let $\mf{p}$
    be a prime ideal of height $1$. Because $p \neq 0$, there exists $a = u
    \pi_1^{n_1} \cdots \pi_r^{n_r}$. Thus there exists an irreducible $\pi$
    such that $\pi \in \mf{p}$, and thus because $0 \subsetneq (\pi) \subset
    \mf{p}$ and $\mr{ht}(\mf{p}) = 1$, we see that $(\pi) = \mf{p}$.

    In the other direction, we will show that for any irreducible $\pi \in A$,
the ideal $(\pi)$ is prime. Let $\mf{p}$ be a minimal prime ideal containing
$(\pi)$. We know that $\mr{ht}(\mf{p}) \leq 1$, and thus $\mr{ht}(\mf{p}) = 1$.
But then we know that $\mf{p} = (\alpha)$, so $\pi = \alpha \beta$ for some
$\beta \in A$. But because $\pi$ is irreducible, this means that $\alpha$ or
$\beta$ is a unit, and we know $\alpha$ is not a unit, so $\beta$ is a unit and
thus $(\pi) = (\alpha) = \mf{p}$.  \end{proof}

\begin{lem} Let $A$ be Noetherian and $x \neq 0$ such that $(x)$ is a prime
ideal. Then $A_x = A[x^{-1}]$ is a UFD if and only if $A$ is.  \end{lem}

\begin{proof} Suppose $A$ is a UFD. Clearly $A_x$ us a UFD. In the other
    direction, Then if \[ a = u x^n \pi_1^{n_1} \cdots \pi_r^{n_r} = u_x
    \pi_1^{n_1} \cdots \pi_r^{n_r} = u' x^{n'} \pi_1^{n_1'} \cdots \pi_r^{n_r'}
= u_x' \pi_1^{n_1'} \cdots \pi_r^{n_r'}, \] we must have $u_x = u x^n, u'_x =
u' x^{n'}$, then we obtain $u = u', n = n'$.  \end{proof}

\begin{thm} Let $A$ be a local Noetherian regular ring. Then $A$ is a UFD.
\end{thm}

\begin{proof} We induct on $\dim A$. If $\dim A = 0$, then $A$ is a field. If
    $\dim A = 1$, then $A$ is a DVR and the result is clear.

    For the inductive step, choose $x \in \mf{m} \setminus \mf{m}^2$. Then
    $(x)$ is a prime ideal, and we want to show that $A_x$ is a UFD.\@ Choose a
    prime ideal $\mf{p}'$ of height $1$ in $A_x$. We will show that $\mf{p}'$
    is principal. Write $\mf{p} = \mf{p}' \cap A$, so $\mf{p}' = \mf{p} A_x$.
    Then because $A$ is regular, it has finite global dimension, and thus
    $\mf{p}$ has a finite projective resolution \[ 0 \to F_n \to F_{n-1} \to
    \cdots \to F_0 \to \mf{p} \to 0, \] where $F_i = A^{m_i}$. Now for any
    prime ideal $\ul{\mf{p}} \subset A_x$, we have ${(A_x)}_{\ul{\mf{p}}} =
    A_{\ul{\mf{p}} \cap A}$. This implies that if $x \notin \ul{\mf{p}} \cap
    A$, then $\mf{m} \neq \ul{\mf{p}} \cap A$ and thus $\mr{ht}(\ul{\mf{p}}
    \cap A) < \dim A$. Then we know that $A_{\ul{\mf{p}} \cap A}$ is regular,
    so we can apply the inductive hypothesis and it is thus a UFD. Then we know
    that \[ \pdim_{A_x}(\mf{p}') = \sup_{\ul{\mf{p}} \subset A_x}
    (\pdim_{{(A_x)}_{\ul{\mf{p}}}} \mf{p}_{\ul{\mf{p}}}') \] and so because
    $\mf{p}'_{\ul{\mf{p}}}$ is principal, it is free of rank $1$ and thus has
    projective dimension $0$. Thus $\mf{p}'$ is a projective $A_x$-module, and
    so we have an exact sequence \[ 0 \to F_n' \to  \cdots \to F_1' \to F_0'
        \to \mf{p}' \to 0 \] where $F_i' = A_x^{m_i}$. Then we decompose this
        into short exact sequences \begin{align*} 0 \to K_0' \to &\ F_0' \to
            \mf{p}' \to 0 \\ 0 \to K_1' \to &\ F_1' \to K_0' \to 0 \\ &\ \
        \vdots \\ 0 \to F_n' \to &\ F_{n-1}' \to K_{n-2}' \to 0.  \end{align*}
        Because $\mf{p}'$ is projective, the exact sequences split and all
        $K_i'$ are projective. Therefore \[ \bigoplus_{i \text{ even}} F_i' =
        \bigoplus_{i \text{ odd}} \oplus \mf{p}'. \] Writing the first direct
        sum as $F$ and the second as $G$, we have $F = G \oplus \mf{p}'$. Thus
        $F,G$ are free $A_x$-modules, and $\rank F = \rank G + 1$. Write $\rank
        G = r$. Now we have $\bigwedge^{r+1} F = \bigwedge^r G \otimes \mf{p}'$
        because $\bigwedge^i \mf{p}' = 0$ if $i > 1$ (after localization, we
        obtain a free module of rank $1$). In particular, we have $A_x \cong
        \bigwedge^{r+1} F = \mf{p}'$ and thus $\mf{p}'$ is free of rank $1$ as
        an $A_x$-module, so it is a principal ideal.  \end{proof}

\begin{defn} Let $A$ be a Noetherian ring. We say that $A$ is \textit{regular}
if for all $\mf{p} \in \Spec A$, $A_{\mf{p}}$ is a local Noetherian regular
ring.  \end{defn}

\section{Criterion for Normality}% \label{sec:criterion_for_normality}

Let $A$ be a ring, $k \geq 0$ an integer, and consider the conditions
\begin{description} \item[$(R_k)$:] For all $\mf{p} \in \Spec A$ such that
    $\mr{ht}(\mf{p}) \leq k$, then $A_{\mf{p}}$ is regular.  \item[$(S_k)$:]
    For all $\mf{p} \in \Spec A$, $\dpth(A_{\mf{p}}) \geq \inf (k,
    \mr{ht}(\mf{p}))$.  \end{description}

\begin{thm}\label{thm:normalcrit} Let $A$ be a Noetherian ring. Then $A$ is
normal if and only if $(S_2)$ and $(R_1)$ hold.  \end{thm}

Before we prove this, we will prove several lemmas.

\begin{lem} The ideal $(0)$ is unmixed if and only if $(S_1)$ holds. In other
words, for all $\mf{p} \in \Spec A$, we have $\dpth(A_{\mf{p}}) \geq \inf(1,
\mr{ht}(\mf{p}))$.  \end{lem}

\begin{proof} Assume that $(0)$ is unmixed and let $\mf{p} \in \Spec A$ with
    $\mr{ht}(\mf{p}) \geq 1$. Then $(0)$ is also unmixed in $A_{\mf{p}}$, so
    $\Ass(A_{\mf{p}}) \cong \qty{\mf{q} \subsetneq \mf{p} \mid \mf{q} \text{
    minimal}}$. Therefore there exists $x \in \mf{p} A_{\mf{p}} \setminus
    \bigcup_{\mf{q}} \mf{q} A_{\mf{p}}$. Thus $x$ is not a zero divisor of
    $A_{\mf{p}}$, so it must be $A_{\mf{p}}$-regular, and thus
    $\dpth(A_{\mf{p}}) \geq 1$.

    Conversely, suppose $(0)$ is not unmixed. Let $\mf{p} \in \Ass(A)$ with
$\mr{ht}(\mf{p}) \geq 1$. But then $\mf{p} A_{\mf{p}} \in \Ass(A_{\mf{p}})$,
and thus $\mf{p} A_{\mf{p}}$ is contained in the set of zero-divisors. Thus
$\dpth(A_{\mf{p}}) = 0$.  \end{proof}

\begin{cor} The property $(S_k)$ holds if and only if for all $i < k$, any
$A$-regular sequence $(a_1, \ldots, a_i)$ is unmixed. For $i = 0$, this is
saying that $(0)$ is unmixed.  \end{cor}

\begin{proof} Use the previous lemma and an induction argument. Also use the
fact that $\dpth(A_{\mf{p}} / a A_{\mf{p}}) = \dpth(A_{\mf{p}}) - 1$ and the
same for dimension.  \end{proof}

Recall that a domain $A$ is normal if it is integrally closed. Also, $A =
\bigcap_{\mf{m}} A_{\mf{m}}$, so $A$ is a normal domain if and only if all
localizations at maximal ideals are normal domains.

\begin{defn} A ring $A$ is called \textit{normal} if and only if for all
$\mf{p} \in \Spec A$, $A_{\mf{p}}$ is a normal domain.  \end{defn}

It is not hard to see that it suffices to check this condition for maximal
ideals. 

\begin{cor} Assume that $A$ is reduced and that $(S_2)$ holds. Then \[ A =
\bigcap_{\substack{\mf{p} \in \Spec A \\ \mr{ht}(\mf{p}) = 1}} A_{\mf{p}}. \]
Here, the intersection is well-defined because the intersection of all minimal
prime ideals is $\qty{0}$.  \end{cor}

\begin{proof} Assume that $(S_2)$ holds. Then consider $\frac{\alpha}{\beta}
    \in \bigcap_{\mf{ht}(\mf{p}) = 1} A_{\mf{p}}$. We will consider the ideal
    $I = \qty{x \in A \mid x \frac{\alpha}{\beta} \in A} = \Ann (\ol{\alpha}
    \in A/\beta A)$. Therefore we have an embedding $A/I \hookrightarrow
    A/\beta A$ given by $x \mapsto x \ol{\alpha}$. Assume $I \neq A$, so $A/I
    \neq 0$. Then if $Q \in \Ass(A/I)$, $Q$ must be of height $1$. Also,
    because $I \subset Q$, then $I_Q \subset Q A_Q$. However, we know that
    $\frac{\alpha}{\beta} \in A_Q$. But then this implies that $I_Q = A_Q$,
    which is a contradiction.  \end{proof}

\begin{lem} Assumes that $A$ satisfies $(S_1)$ and $(R_0)$. Then $A$ is
reduced.  \end{lem}

\begin{proof} Choose $x \in A$ be nilpotent, so $x^n = 0$ for some $n \geq 1$.
    We will show that $x = 0$. If $\mf{p} \in \Spec A$ with $\mr{ht}(\mf{p}) =
    0$, then by $(R_0)$ we know that $A_{\mf{p}}$ is a field. This implies that
    $x_{\mf{p}} = 0$, so there exists $s \notin \mf{p}$ such that $sx = 0$.
    This implies that $\Ann(x) \not\subseteq \mf{p}$ for any minimal prime
    $\mf{p}$.

    Now assume that $x \neq 0$ and let $\mf{q}$ be a maximal element of the set
of ideals $\qty{\Ann(ax) \mid a \in A}$. Then we know that $\mf{q}$ is a
non-minimal prime ideal, so by $(S_1)$, we have $\dpth(A_{\mf{q}}) \geq 1$.
Also, $\mf{q} = \Ann(y)$ for some $y = ax$. Therefore $A/\mf{q} \hookrightarrow
A$. But this implies that $\Ext^{\bullet}(A_{\mf{q}} / \mf{q} A_{\mf{q}},
A_{\mf{q}}) \neq 0$ and thus $\dpth(A_{\mf{q}}) = 0$, which is a contradiction.
\end{proof}

\begin{proof}[Proof of Theorem~\ref{thm:normalcrit}] First assume that $A$ is
    normal and choose $\mf{p} \in \Spec A$. Then if $\mr{ht}(\mf{p}) = 0$, we
    know $A_{\mf{p}}$ is a field. If $\mr{ht}(\mf{p}) = 1$, then $A_{\mf{p}}$
    has dimension $1$ and is integrally closed, so it must be a DVRaand is thus
    regular. Thus $(R_1)$ is satisfied. Now if $\mr{ht}(\mf{p}) = 2$, then
    $A_{\mf{p}}$ is normal of dimension $2$ and is thus Cohen-Macaulay.
    Therefore, $\dpth(A_{\mf{p}}) = 2$. If $\mr{ht}(\mf{p}) > 2$, we can choose
    $\mf{q} \subset \mf{p}$ of height $2$ and then $\dpth(A_{\mf{p}}) \geq
    \dpth(A_{\mf{q}}) = 2$, so $(S_2)$ is satisfied.

    Now assume that $(R_1)$ and $(S_2)$ hold. By the previous lemma, we know
    that $A$ is reduced. This embeds $A \hookrightarrow \mr{Tot}(A) = \prod_i
    K_i$, where $K_i = \mr{Frac}(A/\mf{p}_i)$ with $\mf{p}_1, \ldots, \mf{p}_r$
    the minimal primes of $A$. It suffices to prove that $A$ is integrally
    closed in $\mr{Tot}(A)$. This is because if $\ep_i = (0, \ldots, 0, 1, 0,
    \ldots, 0)$, then $\ep_i^2 = \ep_i$ and thus $\ep_i = A$ and thus $A =
    \prod_{i=1}^r A/\mf{p}_i$ and then each $A/\mf{p}_i$ is a normal domain. 

    Let $x \in \mr{Tot}(A)$ be integral over $A$. Then $x_{\mf{p}} \in
    {\mr{Tot}(A)}_{\mf{p}} = \mr{Tot}(A)$. Then we know $x_{\mf{p}}$ is
    integral over $A_{\mf{p}}$, so if $\mr{ht}(\mf{p}) = 1$, then $A_{\mf{p}}$
    is regular, so it is a DVR and is thus integrally closed. Therefore
    $x_{\mf{p}} \in A_{\mf{p}}$ for all $\mf{p}$ of height $1$. This implies
    that \[ x \in \bigcap_{\mr{ht}(\mf{p}) = 1} A_{\mf{p}} = A. \qedhere \]
\end{proof}


\end{document}
