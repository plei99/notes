\documentclass[leqno, openany]{memoir}
\setulmarginsandblock{3.5cm}{3.5cm}{*}
\setlrmarginsandblock{3cm}{3.5cm}{*}
\checkandfixthelayout

\usepackage{amsmath}
\usepackage{amssymb}
\usepackage{amsthm}
%\usepackage{MnSymbol}
\usepackage{bm}
\usepackage{accents}
\usepackage{mathtools}
\usepackage{tikz}
\usetikzlibrary{calc}
\usetikzlibrary{automata,positioning}
\usepackage{tikz-cd}
\usepackage{forest}
\usepackage{braket} 
\usepackage{listings}
\usepackage{mdframed}
\usepackage{verbatim}
\usepackage{physics}
\usepackage{stmaryrd}
\usepackage{mathrsfs} 
\usepackage[normalem]{ulem} 
\usepackage{stackengine}
\usepackage{bbm}
\usepackage{cancel}
%\usepackage{/home/patrickl/homework/macaulay2}

%font
\usepackage[sc]{mathpazo}
\usepackage{eulervm}
\usepackage[scaled=0.86]{berasans}
\usepackage{inconsolata}
\usepackage{microtype}

%CS packages
\usepackage{algorithmicx}
\usepackage{algpseudocode}
\usepackage{algorithm}

% typeset and bib
\usepackage[english]{babel} 
\usepackage[utf8]{inputenc} 
\usepackage[T1]{fontenc}
\usepackage[bookmarks, colorlinks, breaklinks]{hyperref} 
\hypersetup{linkcolor=blue,citecolor=magenta,filecolor=black,urlcolor=blue}
\usepackage{cleveref}
\usepackage[backend=biber,style=alphabetic,maxalphanames=4,maxnames=5,hyperref,backref=true,backrefstyle=none]{biblatex}
\usepackage{xpatch}
\xpatchbibmacro{pageref}{parens}{backrefparens}{}{}
\crefname{equation}{}{}

% other formatting packages
\usepackage{float}
\usepackage{booktabs}
\usepackage[shortlabels]{enumitem}
\usepackage{csquotes}
\usepackage{titlesec}
\usepackage{titling}
\usepackage{parskip}
\usepackage{graphicx}
\graphicspath{{./}}

\usepackage{lipsum}

% delimiters
\DeclarePairedDelimiter{\gen}{\langle}{\rangle}
\DeclarePairedDelimiter{\floor}{\lfloor}{\rfloor}
\DeclarePairedDelimiter{\ceil}{\lceil}{\rceil}


\newtheorem{thm}{Theorem}[section]
\newtheorem{cor}[thm]{Corollary}
\newtheorem{prop}[thm]{Proposition}
\newtheorem{lem}[thm]{Lemma}
\newtheorem{conj}[thm]{Conjecture}
\newtheorem{quest}[thm]{Question}
\newtheorem{prob}[thm]{Problem}

\theoremstyle{definition}
\newtheorem{defn}[thm]{Definition}
\newtheorem{defns}[thm]{Definitions}
\newtheorem{con}[thm]{Construction}
\newtheorem{exm}[thm]{Example}
\newtheorem{exms}[thm]{Examples}
\newtheorem{notn}[thm]{Notation}
\newtheorem{notns}[thm]{Notations}
\newtheorem{addm}[thm]{Addendum}
\newtheorem{exer}[thm]{Exercise}

\theoremstyle{remark}
\newtheorem{rmk}[thm]{Remark}
\newtheorem{rmks}[thm]{Remarks}
\newtheorem{warn}[thm]{Warning}
\newtheorem{sch}[thm]{Scholium}


% unnumbered theorems
\theoremstyle{plain}
\newtheorem*{thm*}{Theorem}
\newtheorem*{prop*}{Proposition}
\newtheorem*{lem*}{Lemma}
\newtheorem*{cor*}{Corollary}
\newtheorem*{conj*}{Conjecture}

% unnumbered definitions
\theoremstyle{definition}
\newtheorem*{defn*}{Definition}
\newtheorem*{exer*}{Exercise}
\newtheorem*{defns*}{Definitions}
\newtheorem*{con*}{Construction}
\newtheorem*{exm*}{Example}
\newtheorem*{exms*}{Examples}
\newtheorem*{notn*}{Notation}
\newtheorem*{notns*}{Notations}
\newtheorem*{addm*}{Addendum}


\theoremstyle{remark}
\newtheorem*{rmk*}{Remark}

% shortcuts
\newcommand{\Ima}{\mathrm{Im}}
\newcommand{\A}{\mathbb{A}}
\newcommand{\F}{\mathbb{F}}
\newcommand{\E}{\mathcal{E}}
\newcommand{\G}{\mathbb{G}}
\newcommand{\N}{\mathbb{N}}
\newcommand{\R}{\mathbb{R}}
\newcommand{\C}{\mathbb{C}}
\newcommand{\Z}{\mathbb{Z}}
\newcommand{\Q}{\mathbb{Q}}
\renewcommand{\k}{\Bbbk}
\renewcommand{\L}{\mathbb{L}}
\renewcommand{\P}{\mathbb{P}}
\newcommand{\M}{\overline{M}}
\newcommand{\g}{\mathfrak{g}}
\newcommand{\h}{\mathfrak{h}}
\newcommand{\n}{\mathfrak{n}}
\renewcommand{\b}{\mathfrak{b}}
\newcommand{\ep}{\varepsilon}
\newcommand*{\dt}[1]{%
   \accentset{\mbox{\Huge\bfseries .}}{#1}}
\renewcommand{\abstractname}{Official Description}
\newcommand{\mc}[1]{\mathcal{#1}}
\newcommand{\T}{\mathbb{T}}
\newcommand{\mf}[1]{\mathfrak{#1}}
\newcommand{\mr}[1]{\mathrm{#1}}
\newcommand{\ms}[1]{\mathsf{#1}}
\newcommand{\mt}[1]{\mathtt{#1}}
\newcommand{\on}[1]{\operatorname{#1}}
\newcommand{\ol}[1]{\overline{#1}}
\newcommand{\ul}[1]{\underline{#1}}
\newcommand{\wt}[1]{\widetilde{#1}}
\newcommand{\wh}[1]{\widehat{#1}}
\renewcommand{\div}{\operatorname{div}}
\newcommand{\bir}{\sim_{\mr{bir}}}
\newcommand{\stacks}[1]{\href{https://stacks.math.columbia.edu/tag/#1}{#1}}
\newcommand{\ostar}{\stackMath\mathbin{\stackinset{c}{0ex}{c}{0ex}{\star}{\bigcirc}}}

\DeclareMathOperator{\Der}{Der}
\DeclareMathOperator{\Def}{Def}
\DeclareMathOperator{\Bl}{Bl}
\DeclareMathOperator{\NE}{NE}
\DeclareMathOperator{\Tor}{Tor}
\DeclareMathOperator{\Hom}{Hom}
\DeclareMathOperator{\Ext}{Ext}
\DeclareMathOperator{\End}{End}
\DeclareMathOperator{\ad}{ad}
\DeclareMathOperator{\Ad}{Ad}
\DeclareMathOperator{\Aut}{Aut}
\DeclareMathOperator{\Rad}{Rad}
\DeclareMathOperator{\Pic}{Pic}
\DeclareMathOperator{\supp}{supp}
\DeclareMathOperator{\Supp}{Supp}
\DeclareMathOperator{\sgn}{sgn}
\DeclareMathOperator{\spec}{Spec}
\DeclareMathOperator{\Spec}{Spec}
\DeclareMathOperator{\proj}{Proj}
\DeclareMathOperator{\Proj}{Proj}
\DeclareMathOperator{\ord}{ord}
\DeclareMathOperator{\Div}{Div}
\DeclareMathOperator{\depth}{depth}
\DeclareMathOperator{\coker}{coker}
\DeclareMathOperator{\codim}{codim}
\DeclareMathOperator{\ch}{ch}
\DeclareMathOperator{\Hilb}{Hilb}

% Section formatting
\titleformat{\section}
    {\Large\sffamily\scshape\bfseries}{\thesection}{1em}{}
\titleformat{\subsection}[runin]
    {\large\sffamily\bfseries}{\thesubsection}{1em}{}
\titleformat{\subsubsection}[runin]{\normalfont\itshape}{\thesubsubsection}{1em}{}

\title{COURSE TITLE}
\author{Lectures by INSTRUCTOR, Notes by NOTETAKER}
\date{SEMESTER}

\newcommand*{\titleSW}
    {\begingroup% Story of Writing
    \raggedleft
    \vspace*{\baselineskip}
    {\Huge\itshape The Count of Instantons \\ Fall 2023}\\[\baselineskip]
    {\large\itshape Notes by Patrick Lei}\\[0.2\textheight]
    {\Large Lectures by Nikita Nekrasov}\par
    \vfill
    {\Large \sffamily Columbia University}
    \vspace*{\baselineskip}
\endgroup}
\pagestyle{simple}

\chapterstyle{ell}


%\renewcommand{\cftchapterpagefont}{}
\renewcommand\cftchapterfont{\sffamily}
\renewcommand\cftsectionfont{\scshape}
\renewcommand*{\cftchapterleader}{}
\renewcommand*{\cftsectionleader}{}
\renewcommand*{\cftsubsectionleader}{}
\renewcommand*{\cftchapterformatpnum}[1]{~\textbullet~#1}
\renewcommand*{\cftsectionformatpnum}[1]{~\textbullet~#1}
\renewcommand*{\cftsubsectionformatpnum}[1]{~\textbullet~#1}
\renewcommand{\cftchapterafterpnum}{\cftparfillskip}
\renewcommand{\cftsectionafterpnum}{\cftparfillskip}
\renewcommand{\cftsubsectionafterpnum}{\cftparfillskip}
\setrmarg{3.55em plus 1fil}
\setsecnumdepth{subsection}
\maxsecnumdepth{subsection}
\settocdepth{subsection}

\addbibresource{../../math.bib}
\DefineBibliographyStrings{english}{
    backrefpage={$\leftarrow$},
    backrefpages={$\leftarrow$},
}

\begin{document}
    
\begin{titlingpage}
\titleSW
\end{titlingpage}

\thispagestyle{empty}
\section*{Disclaimer}%
\label{sec:disclaimer}

These notes were taken during the lectures using \texttt{emacs}. 
Any errors are mine and not the speakers'. 
In addition, my notes are picture-free (but will include commutative diagrams) and are a mix of my mathematical style and that of the lecturers. Also, notation may differe between lecturers.
If you find any errors, please contact me at \texttt{plei@math.columbia.edu}.

\section*{Description}

Graduate level introduction to modern mathematical physics with the emphasis on the geometry and physics of quantum gauge theory and its connections to string theory.  We shall zoom in on a corner of the theory especially suitable for exploring non-perturbative aspects of gauge and string theory: the instanton contributions. Using a combination of methods from algebraic geometry, topology, representation theory and probability theory we shall derive a series of identities obeyed by generating functions of integrals over instanton moduli spaces, and discuss their symplectic, quantum, isomonodromic, and, more generally, representation-theoretic significance.

Quantum and classical integrable systems, both finite and infinite-dimensional ones, will be a recurring cast of characters, along with the other (qq-) characters.

\section*{Acknowledgements}
I would like to thank Davis Lazowski for providing notes for the lecture on December 8.


\newpage

\tableofcontents

\chapter{Classical mechanics}
\label{cha:intro}

We will be discussing three types of physics in an attempt to create something mathematically interesting:
\begin{itemize}
\item Classical physics;
\item Statistical physics;
\item Quantum physics;
\end{itemize}

\section{Classical physics}
\label{sec:classical}

\subsection{Hamiltonian dynamics}
\label{subsec:hamiltonian}

We will begin with a space of classical states, which is most commonly known as a \textit{phase space}. This is a symplectic manifold $(M, \omega)$, where $\dim M = 2m$ and $\omega \in \Omega^2(M)$ satisfies
\begin{align*}
  \dd{\omega} &= 0 \\
  \omega \wedge \cdots \wedge \omega &\neq 0.
\end{align*}
This carries a function
\[ H \colon M \to \R, \]
called a \textit{Hamiltonian}. Then there is a vector field $V_H$ described by
\[ \dd{H} = \iota_{V_H} \omega, \]
which generates a $1$-parameter group $g^t$ of symplectomorphisms of $M$. The evolution law of the physical system is given by
\[ \dot{x} = V_H(x). \]
Because $g^t$ acts by symplectomorphisms, the graph
\[ \Gamma_{g^t} = \qty{(m, g^t(m))} \subset M \times M \]
is a Lagrangian submanifold. Recall that a submanifold $L \subset M$ of a symplectic manifold is called \textit{Lagrangian} if $\dim L = m$ and $\omega |_L = 0$.

\begin{exer}
  Locally any symplectic manifold is given by $M = \R^{2m}$ with the symplectic form
  \[ \omega = \sum_{i=1}^m \dd{p_i} \wedge \dd{q^i}. \]
  This coordinate system on $\R^{2m}$ is unique up to $\mr{Sp}(2m)$.
\end{exer}

Now if $V$ is a vector field such that $\on{\ms{Lie}}_V \omega = 0$, then $\dd{\iota_V \omega} = 0$. Thus there locally exists a Hamiltonian $h_V$. We also need the Hamiltonian vector field to be linear in the local coordinates, so the Hamiltonian itself must be quadratic.

\begin{exm}
  An important example of a symplectic manifold is $T^* B$ for any smooth manifold $B$. There is a $1$-form $\theta$ on $T^* B$ given by the following formula. If $v$ is a tangent vector at the point $(p,b)$, then
  \[ \theta(v) \coloneqq p(\pi_* v). \]
  Then $\omega = \dd{\theta}$ is a symplectic form.
\end{exm}

\begin{exm}
  Another large class of examples are obtained by \textit{symplectic reduction}. Here, we suppose that a symplectic manifold $M$ carries the action of a compact Lie group $G$ by exact symplectomorphisms. This defines a \textit{moment map} $M \xrightarrow{\mu} \g^*$ by the formula
  \[ \ev{\mu(m), \xi} = h_{V_{\xi}}(m). \]
  There is some ambiguity in the choice of constants, but in the end we obtain a new space
  \[ M \sslash G \coloneqq \mu^{-1}(0) / G. \]
  In practice, we want the moment map $\mu$ to be equivariant with respect to the coadjoint action on $\g^*$. Then the manifold $M \sslash G$ has a canonical symplectic form, but this requires a lot of work.

  Now consider $M = \R^{2m}$ and $G = U(1)$, where we write $M = (\R^2)^m$ and $U(1)$ acts by rotations. Then the moment map is actually
  \[ \mu = \sum_{i=1}^m \frac{1}{2} (p_i^2 + q_i^2) - r, \]
  so $\mu^{-1}(0)$ is a sphere. We then obtain
  \[ \R^{2m} \sslash U(1) = S^{2m-1}/U(1) = \P^{m-1}. \]
  We made no use of the complex numbers, so the fact that we obtain a complex manifold will be viewed as a bonus. The reduced symplectic form is simply $r$ times the Fubini-Study form.

  Note that $\P^{m-1}$ is compact, so the interpretation that phase space records position and momentum breaks down. In this case, our phase space is called the \textit{classical spin phase space}, where the motion is by rotations rather than by translation.
\end{exm}

\begin{rmk}
  We will often consider time-dependent Hamiltonians, where $\dot{x} = V_{H(t)}(x)$.
\end{rmk}

\subsection{Lagrangian mechanics}
\label{subsec:lagrangian}

There is another point of view, where dynamics on $M$ are given by an optimization problem on the space
\[ \mc{P}M \coloneqq \ms{Map}([0,1], M) \]
of paths in $M$. We will consider an \textit{action}
\[ \mathbb{S}[\gamma] \coloneqq \int_{\gamma} \theta - \int_0^1 \gamma(t)^* H(t) \dd{t}, \]
where we assume that $\omega = \dd{\theta}$. If we further require that $\gamma(0) \in L_0$ and $\gamma(1) \in L_1$, our dynamics are well-defined if $L_0, L_1$ are Lagrangian submanifolds. Of course, we are looking for paths were $\delta \mathbb{S} = 0$.

\subsection{Classical field theory}
\label{subsec:classicalft}

Classical field theory should be thought of as an infinite-dimensional version of classical mechanics, where we study loopified versions of finite-dimensinoal manifolds. We want to consider integrals
\[ \mathbb{S} \coloneqq \int_{(\Sigma, h)} \mc{L}[\phi, \partial \phi] \mr{vol}_h, \]
where $\Sigma$ is the spacetime, $h$ is a metric, and $\phi$ are the \textit{fields}. Fields could be one of several options:
\begin{itemize}
\item Scalars $f \colon \Sigma \to X$, where $X$ is a Riemannian manifold;
\item Connections $\nabla$ on principal bundles
  \begin{equation*}
    \begin{tikzcd}
      G \ar{r} & P \ar{d} \\
      & \Sigma,
    \end{tikzcd}
  \end{equation*}
  called \textit{gauge fields};
  \item The metric $h$ itself, called \textit{gravity}.
\end{itemize}

We can introduce more complexity into the problem by varying the action and looking for solutions of PDEs, introducing boundary to $\Sigma$, and other operations. Also note that Lagrangian mechanics can be interpreted as a $1$-dimensional classical field theory.

\section{Statistical physics}
\label{sec:stat}

In statistical physics, a point is replaced by a cloud of points, or a probability measure. For example, the measure could contain the term $e^{-\beta H}$, where $H$ was the classical Hamiltonian and $\beta$ is a parameter of our distribution (inverse temperature). The system often flows to a stationary distribution, which is determined by the outside world. In reality, the distribution will have the form
\[ \frac{1}{Z} e^{-\beta H} \qquad Z = \int_M e^{-\beta H} \frac{\omega \wedge \cdots \wedge \omega}{m!}. \]
This factor $Z$ is called the \textit{partition function}, and most of our energy is spent on computing this partition function.

\section{Classical physics with multiple times}
\label{sec:times}

Suppose we have Hamiltonians $H_1, \ldots, H_k$ such that $[V_{H_i}, \ldots, V_{H_j}] = 0$. Then we obtain dynamics
\[ \ul{\gamma} \colon [0, \ep]^p \to M \]
and an action
\[ \mathbb{S} = \int \theta - \sum_{k=1}^p \ul{\gamma}^* H_k \dd{t_k} \]
defined on paths inside the cube $[0,\ep]^p$. The extreme case of this is an \textit{integrable system} where the $H_i$ are functionally independent, and the maximum possible value of $p$ is $m$.

\begin{thm}[Liouville-Arnold]
  If the motion is finite (fits in a compact set), then locally $M$ is a fibration
  \begin{equation*}
    \begin{tikzcd}
      T^m \ar{r} & M \ar{d} \\
      & B
    \end{tikzcd}
  \end{equation*}
  such that the $H_i$ factor through $B$ and the $V_{H_k}$ span the rotations on each $S^1$ factor of $T^m$.
\end{thm}

A typical trajectory is a winding of the torus, where if $\theta_i$ are the angle coordinates on $T^m$, there is the formula
\[ \theta^i(t) = \theta^i(0) + \omega^i t. \]
Generically, these paths will have dense image.

In the special case of an integrable system, there are \textit{action-angle variables}, where the symplectic form is
\[ \omega = \sum_{i=1}^m \dd{I_i} \wedge \dd{\theta^i}. \]
The $\theta^i$ are defined up to $SL(m, \Z)$ affine transformations. If $C_i \in H_1(T^m, \Z)$ form a basis, then the $I_i$ are defined by
\[ I_i = \frac{1}{2\pi} \oint_{V_i} \dd^{-1} \omega. \]
Here, $C_i$ is transported to other fibers via the Gauss-Manin connection. We should note that the $I_i$ are not well-defined, but the quantities $I_i(b') - I_i(b)$ are well-defined.

Because the $H_k$ are defined on the base $B$, we can write $H_k(I_1, \ldots, I_m)$. Fixing $\tau_1, \ldots, \tau_m \in \R$, we can flow along
\[ H = \sum_{k=1}^m \tau_k H_k, \]
we obtain
\[ \omega^i = \pdv{H}{I_i}. \]

In ``reality,'' which is non-integrable, consider an approximation
\[ H(I,\theta) = H_0(I) + \ep H_1(I, \theta). \]
Then we can understand the approximate evolution with respect to $H$ by averaging $H_1$ over $T^m$.

\section{Gauge symmetry}
\label{sec:gauge}

We would now like to discuss the idea of gauge symmetries and gauging in the Lagrangian formalism. Recall that the phase space $(P, \omega)$ carries an action
\[ \mathbb{S}[\gamma] = \int_{\gamma} \dd^{-1} \omega - \int_I \gamma^* H \abs{\dd{t}}, \]
and we want to consider the effect of an action of a group $G$ on $P$. The moment map
\[ \mu \colon P \to \G^* \]
is equivariant with respect to the coadjoint representation. Note that $P\sslash G$ has a symplectic form $\wt{\omega}$, and locally $P$ looks like
\[ T^* G \times P \sslash G \xrightarrow{\pi} P\sslash G. \]
Thus $\omega$ restricted to a tubular neighborhood of $\mu = 0$ has the form
\[ \pi^* \wt{\omega} + (\text{tautological form on } T^*G). \]

Recall that we are looking for extrema of $\mathbb{S}$, and we need to find a path on the quotient space. We need to enlarge the space of variables to include $A \in \Omega^1_I(\g)$. Now we will define
\[ \wt{\mathbb{S}}[\gamma, A] = \mathbb{S}[\gamma] - \int_I \ev{\gamma^* \mu, A}. \]
The space of possible $A$ has an infinite-dimensional symmetry generated by
\[ \mc{G} = \ms{Maps}(I, G) \ni g(t) \]
The action is given by
\[ (g(t)) \cdot (\gamma(t), A) \coloneqq (g(t) \gamma(t), \on{Ad}_{g(t)}A + g^{-1} \dd{g}). \]
Note that $A$ transforms as a connection, not as a $1$-form.

\begin{rmk}
The translation by $g^{-1} \dd{g}$ compensates for the change of $\dd^{-1} \omega = \sum p_i \dd{q^i}$ under the action of $G$.
\end{rmk}

\subsection{Rational Calogero-Moser model}
\label{sub:ratlcms}

The first example is called the \textit{Calogero-Moser-Sutherland model}. The phase spaces are
\[ \wt{P} = T^* (\R^N \setminus \Delta \text{ or } (S^1)^N \setminus \Delta )\]
with the standard form
\[ \omega = \sum_{i=1}^N \dd{p_i} \wedge \dd{x_i}. \]
For any $\nu \in \R_+$, the Hamiltonian is given by
\[ H = \sum_{i=1}^N \frac{1}{2} p_i^2 + \nu^2 \sum_{1 \leq i < j \leq N} \qty[\frac{1}{(x_i-x_j)^2} \text{ or }\frac{1}{4\sin^2\qty(\frac{x_i-x_j}{2})} ]. \]
These are systems of $n$ particles where they start out very far away from each other, are brought closer together, and then repel themselves apart again. These turn out to be integrable systems, and in fact they can be obtained by reduction of a system on a higher-dimensional symplectic manifold.

Define the unreduced phase space
\[ P = T^* (\mf{u}(N)) \times \C^N. \]
This is a pair of $N \times N$ Hermitian matrices $(P,Q)$ with a vector $z \in \C^N$. The Liouville form will be written as
\[ \Tr \qty( \dd{P} \wedge \dd{Q} + \frac{\Tr \dd{z} \wedge \dd{z^{\dag}}}{2\sqrt{-1}}) = \sum_{i,j=1}^N \dd{P_{ij}} \wedge \dd{Q_{ji}} + \frac{1}{2\sqrt{-1}} \sum_{i=1}^N \dd{z_i} \wedge \dd{z_i^*}. \]
Then we may define the Hamiltonians
\[ H_k = \frac{1}{k} \Tr P^k. \]
The flows look like
\[ (P,Q; z) \mapsto (P, Q + \sum_k t_k P^{k-1}; z), \]
so they clearly commute. This system carries a $U(N) \times U(1)$ symmetry, where
\[ (u,c) \cdot (P,Q,z) \mapsto (u^{-1}P u, u^{-1} Q u, u^{-1} z). \]
Because this preserves the symplectic form, we may perform the symplectic reduction. Becuase $U(n)$ is not simple, there is a free parameter $\nu$, so the moment map is given by
\[ \mu(P,Q,z) = [P,Q] + \sqrt{-1} (zz^{\dag} - \nu \cdot 1_N). \]
We only need to solve $\mu = 0$ up to $U(N)$, so we choose a diagonal representative of $\qty{u^{-1}Qu}$. Thus, assume that $Q = \mr{diag}(x_1, \ldots, x_N)$ is diagonal with $x_1 \geq \cdots \geq x_N$. Generically, the inequalities are strict. Then
\[ \mu_{ij} = P_{ij}(x_j - x_i) + \sqrt{-1} (z_i z_j^* - \nu \delta_{ij}). \]
If $i=j$, then $\abs{z_i}^2 = \nu$, so the remaining $U(1)^N$-action can be used to set $z_i = z_i^* = \sqrt{\nu}$. We can now compute
\[ P_{ij} = - \frac{\sqrt{-1} \nu}{x_i-x_j} \]
for the non-diagonal elements. We cannot compute the diagonal elements of $P$, so we obtain
\[ P = \mr{diag}(p_1, \ldots, p_N) + \norm{\frac{\sqrt{-1} \nu}{x_i-x_j}(1-\delta_{ij})}_{i,j=1}^N. \]
In this form, the Hamiltonians become
\begin{align*}
  H_1 &= \sum_{i=1}^N p_i \\
  H_2 &= \frac{1}{2} \sum_{i=1}^N p_i^2 + \nu^2 \sum_{i<j} \frac{1}{(x_i-x_j)^2},
\end{align*}
which is what we wanted. We need to show that $\omega = \sum \dd{p_i} \wedge \dd{x_i}$, so we will compute the Poisson brackets of various functions. Recall that functions on $(P, \omega)$ form a Lie algebra with the \textit{Poisson bracket}
\[ \qty{f,g} = \omega^{-1} \llcorner \dd{f} \wedge \dd{g}. \]
We note that
\begin{align*}
  \mc{O}(P \sslash G) &= \mc{O}(\mu^{-1}(0)/G) \\
                      &= \mc{O}(\mu^{-1}(0))^G \\
                      &= \mc{O}(P)^G/(\mu = 0).
\end{align*}
The functions we will consider are the resolvents
\begin{align*}
  R(\lambda) &= \Tr \frac{1}{Q-\lambda} \\
  S(\lambda) &= \Tr P \frac{1}{Q-\lambda}.
\end{align*}
Because the trace is cyclic, we obtain
\begin{align*}
  \dd{R}(\lambda) &= -\Tr(Q-\lambda)^{-1} \dd{Q} (Q-\lambda)^{-1} \\
  &= -\Tr [(Q-\lambda)^2 \dd{Q}]
\end{align*}
Therefore
\begin{align*}
  \qty{R(\lambda), S(\mu)} &= \sum_{i,j} \pdv{R(\lambda)}{Q_{ij}} \pdv{S(\mu)}{P_{ji}} \\
                           &= -\Tr (Q-\lambda)^2 (Q-\lambda)^{-1} \\
                           &= -\pdv{\lambda} \qty(\frac{R(\lambda) - R(\mu)}{\lambda - \mu}).
\end{align*}
On the reduced space, the functions become
\begin{align*}
  R(\lambda) &= \sum_{i=1}^N \frac{1}{x_i-\lambda} \\
  S(\mu) &= \sum_{i=1}^N \frac{p_i}{x_i-\mu}.
\end{align*}
This is equivalent to $\qty{x_i, x_j} = 0 = \qty{p_i, p_j}$, so $\qty{p_i, x_j} = \delta_{ij}$.

Note that this system has an alternative presentation where we assume that $P = \mr{diag}(\wt{p}_1, \ldots, \wt{p}_N)$ and
\[ Q = \mr{diag}(\wt{x}_1, \ldots, \wt{x}_N) + \norm{\frac{\sqrt{-1}\nu}{\wt{p}_i - \wt{p}_j}}. \]
Then the Hamiltonians reduce to
\[ H_k = \frac{1}{k} \sum_{i=1}^N \wt{p}_i^k, \]
and the flows are given by
\[ \wt{x}_i(t) = \wt{x}_i(0) + \sum_k t_k \wt{p}_i^{k-1}. \]
This system is not very interesting, but we could instead consider
\[ H_k^{\vee} = \frac{1}{k} \Tr Q^k \]
and obtain a system with position and momentum exchanged.

\subsection{Trigonometric Calogero-Moser (Sutherland)}
\label{sub:trigcm}

Now consider $P = T^* U(N) \times \C^N$. Then we have a triple $(P, g;z)$ where $P$ is Hermitian and $g (= \exp(\sqrt{-1} Q))$ is unitary. The moment map is given by
\[ \mu(P,g,z) = \sqrt{-1} (P-g^{-1}Pg + z z^{\dag} - \nu \cdot 1_N). \]
We may choose to either diagonalize $P$ as $\mr{diag}(\wt{p}_1, \ldots, \wt{p}_N)$ or diagonalize $g$ as $\mr{diag}\qty(e^{\sqrt{-1}x_1}, \ldots, e^{\sqrt{-1}x_N})$. Making the latter choice, we obtain
\[ P = \mr{diag}(p_1, \ldots, p_N) + \norm{\frac{\nu}{e^{\sqrt{-1}(x_j-x_i)}-1}(1-\delta_{ij})}. \]
The Hamiltonians in this case are
\begin{align*}
  H_1 &= \sum_k p_k \\
  H_2 &= \frac{1}{2} \sum_{i=1}^N p_i^2 + \frac{\nu^2}{4} \sum_{i<j} \frac{1}{\sin^2 \qty(\frac{x_i-x_j}{2})}. 
\end{align*}

Making the former choice, we obtain another integrable system called the \textit{rational relativistic Calogero-Moser system} or the \textit{rational Ruijsenaars model}. In this model, the Hamiltonians look like
\[ H_k^{\vee} = \sum e^{\wt{x}_i} \times \qty(\text{rational functions of }\wt{p}_i). \]
Here, a relativistic particle in $1+1$ dimensions has energy and momentum given by
\begin{align*}
  E &= m\cosh \theta = \Tr(g+g^{-1}) \\
  p &= m \sinh \theta = \Tr(g-g^{-1}),
\end{align*}
so $E^2-p^2 = m^2$.

\section{Infinite-dimensional symmetries}
\label{sec:infinite}

We will now replace $\g = \on{\ms{Lie}}$ with $\wh{\g} = \wh{\ms{Maps}(S^1, \g)}$, which is a central extension of the space of maps with commutator given by
\[ [(f_1, c_1), (f_2, c_2)] = \qty([f_1, f_2], \int_{S^1} \Tr f_1 \dd{f_2}). \]
Then $\wh{\g}^*$ is not $\g$ but instead
\[ \wh{\g}^* = \qty{k \partial + A \mid A \in \Omega^1_{S^1}(\g), k \in \R} \]
with the pairing
\[ \ev{k \partial + A, (f,c)} = kc + \int_{S^1} \ev{A,f}. \]
We will also consider the Lie algebra $\ms{Maps}(\R, \g)$, but this requires us to specify some kind of boundary conditions at $\infty$. We may also consider $L^2(\R) \otimes \g$. In the case of $S^1$, note that
\[ c(f_1, f_2) \coloneqq \int_{S_1} \Tr f_1 \dd{f_2} \]
is a $2$-cocycle and that
\[ H^2(L\g, \R) \cong \R \]
is $1$-dimensional, so this is the only nontrivial cocycle.

The corresponding group is given by the following construction. Define
\[ LG = \ms{Maps}(S^1, G). \]
Then $\wh{LG}$ is a nontrivial $U(1)$-bundle
\[ 1 \to U(1) \to \wh{LG} \to LG \to 1. \]
Note that $H^2(LG, \R) \simeq \R$. The cohomology $H^3(G,\Z)$ is nontrivial with a nontrivial class given by
\[ \omega \coloneqq \frac{i}{8\pi^3} \Tr(g^{-1}\dd{g})^3 \leftrightsquigarrow \Tr \xi_1[\xi_2, \xi_3] \eqqcolon c(\xi_1, \xi_2, \xi_3). \]
Then there is an evaluation map
\[ e \colon LG \times S^1 \to G \qquad (g(t), u) \mapsto g(u), \]
and then
\[ \int_{S^1} e^* \omega \in H^2(LG, \Z) \]
represents $c_1(\wh{LG} \to LG)$. Therefore, we have an identification
\[ \wh{LG} = \wh{\ms{Maps}(D^2, G)} / \ms{Maps}((D^2, S^1), (G, 1)), \]
where $\wh{\ms{Maps}(D^2, G)} = \ms{Maps}(D^2, G) \times U(1)$ with multiplication
\[ (g_1, c_1) \times (g_2, c_2) = \qty(g_1g_2, c_1c_2 \exp \frac{i}{4\pi}\int_{D^2} \Tr g_1^{-1} \dd{g_1} \wedge \dd{g_2}). \]
To embed $\ms{Maps}((D^2, S^1), (G,1))$ as a normal subgroup, we make use of the fact that $\pi_2(G) = 0$, so any map $g$ can be extended to $\wt{g} \colon B^3 \to G$. Then we define
\[ \varphi(g) \coloneqq (g, \exp(2\pi i) \wt{g}^* \omega). \]
The fact that this construction is well-defined is the \textit{Polyakov-Wiegmann formula}.

We will now discuss the adjoint and coadjoint actions of $\wh{LG}$ on $\wh{g}, \wh{g}^*$ respectively. Infinitesimally, we have
\[ (\phi, 0) \cdot (\xi, c) = \qty([\phi, \xi], \int_{S^1} \Tr \phi \dd{\xi}). \]
The action on the dual space is given by
\begin{align*}
  \ev{\ad^*_{\phi}(A, k), (\xi, c)} &= \ev{(A,k), \qty([\phi, \xi], \int \Tr \phi \dd{\xi})} \\
                                    &= k\int_{S^1} \Tr \phi \dd{\xi} + \int_{S^1} \Tr A[\phi, \xi] \\
                                    &= \int_{S^1} \Tr \xi(-k \dd{\phi} + [A, \phi]).
\end{align*}
Therefore, we obtain
\[ \Ad^*_g(A, k) = (-k \dd{g} g^{-1} - g A g^{-1}, 0). \]
Note that $\frac{A}{k}$ is a $g$-connection $1$-form on $S^1$.

There is now a natural candidate for a symplectic form, which is
\[ \Omega_{T^* \wh{\g}} = \delta k \wedge \delta c + \int_{S^1} \Tr \delta A \wedge \delta \xi. \]
Here, $\delta$ is the differential in the space of fields. The moment map $\mu \colon T^* \wh{g} \to \wh{g}^*$ is given by
\[ \mu(k, c, A, \xi) = (G(k, c, A, \xi), 0), \]
where $G(k,c,A,\xi) = -k \dd{\xi} + [A, \xi]$. We will now compute
\[ \mc{P}^{\mr{red}} = \mu^{-1}(0) / LG = \qty{(\xi, A, k, c) \mid -k \dd{\xi} + [A,\xi] = 0}/(\xi, A) \mapsto (\Ad_g \xi, k \dd{g}g^{-1} + g A g^{-1}). \]

We will now solve the moment map equation with the assumption that $k \neq 0$. We will scale $k=1$, so the equation becomes
\[ \dd{\xi} + [A, \xi] = 0. \]
This is a first order matrix differential equation with periodic coefficients which can be studied using Floquet-Lyapunov theory. This says that there exists $g$ such that
\[ g^{-1} \dd{g} + g^{-1} A g \in \mf{t} \subset \g \]
is constant and lies in a maximal Cartan of $\g$. What this means is that we can write
\[ \xi(t) = G(t) \xi_0 G(t)^{-1} \qquad G(t) = P \exp \int_0^t A. \]
These satisfy the equations $\dot{G} G^{-1} = A$ and $G(0) = 1$. The monodromy is
\[ G_A \coloneqq G(2\pi) = P \exp \int_0^{2\pi} A. \]
This must commute with $\xi_0$, so we can bring
\[ A \mapsto g^{-1} \dd{g} + g^{-1}Ag \qquad G_A \mapsto g(0)^{-1} G_A g(0). \]
Recall that $G_A$ can be brought to $T \subset G$ and $A$ can be brought to $\alpha \in \mf{t}$ and $g_A = \exp(2\pi \alpha)$. There is still some remaining symmetry by
\[ g(u) = \exp(u\lambda), \]
where $\lambda \in \Lambda^{\vee}$ is in the lattice of coroots and $u \in S^1 = \R/2\pi \Z$ is the coordinate. This shifts $\alpha \mapsto \alpha + \lambda$ while preserving monodromy. The second kind of remaining symmetry is the Weyl group $W \coloneqq N(T)/T$. Taking their semidirect product, we obtain the \textit{affine Weyl group}.

If we consider the weight space decomposition of the moment map equation and $\beta$ is a root of $\g$, then the equation for this component is
\[ \dd{\xi_{\beta}} + \ev{\beta, \alpha} \xi_{\beta} = 0. \]
Because $\xi_{\beta}(u) = e^{-u\ev{\beta, \alpha}}$, this is generically not $1$, so $\xi_{\beta} = 0$. Therefore $\xi \in \mf{t}$, and we obtain
\[ T^* \wh{g} \sslash LG = (T^* T)/W. \]
Note that $T$ parameterizes conjugacy classes of $P \exp \oint A$ and that $T = \mf{t}/\Lambda^{\vee}$. Unfortunately, the reduced space is an orbifold, not a manifold.

We will now attempt to remedy this situation by modifying the quotient. Instead of setting the moment map to be $0$, we want to consider an orbit. We want $\mc{O} = \P^{N-1}$, and if $G = SU(N)$, $LG$ acts on $\P^{N-1}$ by evaluation at some $0 \in S^1$. We choose $z \in \C^N$ such that $z^{\dag} z = N$ up to $z \sim ze^{i\alpha}$, and the modified equation is
\[ \dd{\xi} + [A, \xi] = \delta(u) \cdot (i \nu(1_N - z \circ z^{\dag})). \]

\begin{rmk}
While most of the orbits are infinite dimensional, we are taking some limit where $A$ becomes a distribution on $S^1$ supported on finitely many points.
\end{rmk}

We first apply Floquet-Lyupanov to make $A = \mr{diag}(a_1, \ldots, a_N)$ diagonal with $\sum a_i = 0$. Then on each coordinate we obtain
\begin{align*}
  \dd{\xi_{ij}} + (a_i-a_j) \xi_{ij} &= \sqrt{-1} \delta(u)\nu(-z_i \ol{z}_j) \\
  \dd{\xi_{ii}} &= \sqrt{-1} \nu(1-\abs{z_i}^2) \delta(u).
\end{align*}
Because $\xi_{ii}(+0) = \xi_{ii}(2\pi-0) = \xi_{ii}(-0)$ for any $0 \in S^1$, $\abs{z_i}^2 = 1$. Using the maximal torus, we may force $z_i = 1$. Then we obtain
\begin{align*}
  \xi_{ij}(u) &= e^{-u(a_i-a_j)}\xi_{ij}(+0) \\
  \xi_{ij}(2\pi-0) &= e^{-2\pi(a_i-a_j)}\xi_{ij}(+0) = \xi_{ij}(+0) + \sqrt{-1} \nu.
\end{align*}
Finally, the initial value is
\[ \xi_{ij}(+0) = \frac{\sqrt{-1}\nu}{e^{-2\pi \sqrt{-1}(a_i-a_j)}-1}. \]
Note that this appeared in our study of the Sutherland system.

\chapter{Quantum mechanics}
\label{cha:complexified}

\section{Quantization}
\label{sec:quantization}

The motivation for complexification of our systems is quantization. Recall that if $S$ is the action of a Lagrangian system, many systems are not described by solving the variational equation
\[ \delta S = 0 \]
but by meditating on formal path integrals (due to Feynman)
\[ \int_{P \mc{P}} e^{\frac{i S[\gamma]}{\hslash}} [\mc{D}\gamma]. \]
The classical system is obtained via stationary phase approximation. Mathematically, this is ill-defined, but if $X$ is a finite-dimensional manifold, we can consider oscillating integrals
\[ I = \int_X e^{\frac{iS}{\hslash}} \mu. \]

Here, $X$ is one of many possible cycles in the complexification $X^{\C}$ and $\mu$ is viewed as a holomorphic top-degree form, so this is just a period. If $\dim X = n$, then we may consider other $\Gamma$ such that
\[ \int_{\Gamma} e^{\frac{iS}{\hslash}} \mu \]
converges. These will satisfy
\[ \Gamma \in H_n(X^{\C}, X^{\C}_{\ll}), \]
where
\[ X^{\C}_{\ll} = \qty{z \mid \Re\qty(\frac{iS(z)}{\hslash}) \ll 0} \]
is set to force the integral to converge. These $\Gamma$ are chosen to flow from critical points of $S$ (equivalently, of $W = iS$) in $X^{\C}$ into $X^{\C}_{\ll}$. We can construct cycles using \textit{Lefschetz thimbles}. We can choose a critical point $p$ which satsifies $\dd{W}(p) = 0$ and then consider the steepest descent flow for $\Re\qty(\frac{W}{\hslash})$, or in other words
\[ \dot{x} = -\nabla \Re\qty(\frac{W}{\hslash}), \]
where the gradient is taken with respect to some metric on $X^{\C}$, and finally take the union $\Gamma_p$ of descending trajectories. If we choose a Hermitian metric, then $\Im\qty(\frac{W}{\hslash})$ is actually constant. For a generic choice of $\hslash$, the critical points have different imaginary parts, but in the special settings we may have Stokes phenomena and wall-crossing behavior of solutions.

\begin{exer}
What happens for $W = \sum_{i=1}^n z_i^2$?
\end{exer}

\section{Holomorphic symplectic dynamics}
\label{sec:hk}

Now let $M^{\C}$ be a holomorphic symplectic manifold and consider $\mf{X}^{\C} = \ms{Maps}(S^1, M^{\C})$ and
\[ W = \int p \dd{q} - \beta H(p,q) \dd{t}. \]
We will take $S^1 = \R/\Z$ and we want to find points with $\dd{W} = 0$.

\begin{rmk}
When we quantize everything, we will obtain
\[ \Tr e^{-i\beta \wh{H}} = \int e^{\frac{iS}{\hslash}} [\mc{D}\gamma]. \]
\end{rmk}

We obtain the equations
\begin{align*}
\dv{p}{t} &= - \beta \pdv{H}{q} \\
\dv{q}{t} &=  \beta \pdv{H}{p} 
\end{align*}
If $\beta$ is real, this is the usual Hamiltonian dynamics, but there may not be real solutions if $\beta$ is not real. On the other hand, if $\beta = i\beta_E$ is purely imaginary (also known as \textit{going to Euclidaen time}), then there may be solutions where $q$ is real and $p$ is purely imaginary.

\begin{exm}
  Let $M = \R^2$ and
  \[ U(q) = \frac{\lambda}{4} (q^2-a^2)^2, \]
  where $\lambda, a$ are parameters. This is usually called the \textit{Higgs potential}. Then the energy is
  \[ E = \frac{p^2}{2} + U(q). \]
  We can see that there is a $q \mapsto -q$ symmetry, so near $\pm a$ there are two copies of the same physics. After complexifying, we obtain $M^{\C} = \C^2$, while the zero set $C_E$ of $E$ is a Riemann surface. If we compactify, we will obtain an elliptic curve with the hyperelliptic form
  \begin{align*}
    p^2 &= 2\qty(E-\frac{\lambda}{4}(q^2-a^2)^2) \\
    &= -\frac{\lambda}{2} (q^2-a_+^2)(q^2-a_-^2).
  \end{align*}
  Note that $H_1(C_E, \Z) = \Z \oplus \Z$, so there are two independent cycles corresponding to classically allowed physics. Also, it is clear that
  \[ a_{\pm}^2 - a^2 = \pm \sqrt{\frac{4E}{\lambda}}. \]
  Therefore, this family degenerates to a union of two copies of $\P^1$ when $E=0$ (and one of our distinguished cycles is the vanishing cycle) and has another critical point when $E = \frac{\lambda a^4}{4}$. Now $\omega = \frac{\dd{q}}{p}$ is a holomorphic differential on $C_E$, and the Hamilton equation tells us that any
  \[ \gamma \colon S^1 \to C_E \to M^{\C} \]
  must satisfies $\gamma^* \omega = \beta \dd{t}$. Now in the low-energy region, $[\gamma] \in H_1(C_E, \Z)$ can be specified by two integers:
  \[ [\gamma] = m[A] + n[B], \]
  where $[A]$ is the vanishing cycle and $[B]$ satisfies $A \cap B = 1$. Note $B$ is described up to multiples of $A$, so $n$ is well-defined, but $m$ is defined only up to multiples of $2n$. We can then find $\beta$ by the period integral
  \begin{align*}
    \beta &= \int_{S^1} \gamma^* \omega \\
          &= \int_{\gamma(S^1)} \omega \\
          &= m \oint_A \omega + n\oint_B \omega,
  \end{align*}
  which are functions of $E$. Therefore, the $E = E_{m,n}(\beta)$ for $n\in \Z, m \in \Z/2n\Z$ satisfy a transcendental equation
  \[ \beta = m\omega_A(E) + n\omega_B(E). \]
  We can solve this equation approximately in the $\beta \to \infty$ limit, where
  \[ E_{m,n} \approx \frac{\lambda a^4}{4} e^{-\frac{\beta \Omega}{n}}e^{\pi i \frac{m}{n}}, \]
  where $\Omega^2 = 2\lambda a^2$ is the classical period of very small oscillations around the $q=a$ critical point. This is all obtained by Picard-Lefschetz theory using the fact that
  \begin{align*}
    \omega_B(E) &\sim \frac{2\pi}{\Omega}\frac{1}{\pi i} \log E + \cdots \\
    \omega_A(E) &\sim \frac{2\pi}{\Omega} + \cdots,
  \end{align*}
  where the $+\cdots$ can be computed using knowledge of elliptic integrals. The solution where $n=0$ corresponds to classical physics, while the solution where $m=0$ corresponds to tunneling between the two critical points. Therefore, the general $(m,n)$ solution is some superposition of classical motion and tunneling.
\end{exm}

\section{Algebraic integrable systems}
\label{sec:algebraic}

We will now consider the class of systems which generalize the following feature of the previous example: our manifold $(M^{2n}_{\C}, \omega_\C)$ has a Lagrangian fibration $M^{2n} \to B^n$ to some open subset of $\C^n$ by polarized abelian varieties. These are called \textit{algebraic integrable systems}. This structure gives the structure of special K\"ahler geometry on $B^n$.

For us, a polarization is simply an integral class $t \in H^2(F, \Z)$, where $F = \pi^{-1}(b)$. We will now define action variables $(a^i, a_{D,i})$ on $B$. If we have $\gamma \in H_1(F_b, \Z)$, this path can be transported canonically over paths connecting $b$ to $b'$ that avoid the discriminant locus $\Xi$. We now obtain a $2$-chain in $M^{\C}$ covering the path, and we can obtain a number
\[ a_{\gamma}(b') - a_{\gamma}(b_*) = \int_{\text{$2$-chain}} \omega^{\C} \]
for a distinguished choice of $b_*$. Therefore, we obtain a map
\[ a \colon \wt{B^n \setminus \Xi} \to H^1(F_{b_*}, \C), \]
where $\wt{B^n \setminus \Xi}$ is the space of choices $(b, \gamma)$, where $b \in B^n \setminus \Xi$ and $\gamma$ is a path connecting $b_*$ to $b$ in $B^n \setminus \Xi$ up to homotopy. The image of $a$ is a Lagrangian with respect to $t$, where if $A_i, B^i$ is a basis of $1$-cycles in $H_1(F_{b_*}, \Z)$, then $t(A_i \cap B^j) = \delta_i^j$. Now because
\[ \sum_{i=1}^n \dd{a^i} \wedge \dd{a_{D,i}} = 0, \]
locally there exists $\mc{F}(a)$ such that
\[ a_{D,i} = \pdv{\mc{F}}{a_i}. \]

\begin{defn}
This $\mc{F}$ is called a \textit{prepotential}.
\end{defn}

\begin{exm}
  One example is the \textit{elliptic Calogero-Moser system}. Here, let $E$ be an elliptic curve parameterized by $\tau \in \mc{H}$ and consider
  \[ M^{\C} = ( T^* E^n \setminus \Delta )/S(n). \]
  Then let $p_i$ be the coordinates in the fiber directions and $z_i$ be the coordinates on the copies of $E$ and define
  \begin{align*}
    H_1 &= \sum_i p_i \\
    H_2 = \sum_i \frac{1}{2} p_i^2 + \nu^2 \sum_{i<j} \wp(z_i - z_j).
  \end{align*}
  This was proven to be an algebraic integrable system by Krichever (before algebraic integrable systems were defined). Set
  \[ L(z) \coloneqq \on{diag}(p_1, \ldots, p_n) + \nu(1-\delta_{ij}) \frac{\theta(z_i-z_j+z) \theta'(0)}{\theta(z_i-z_j) \theta(z)}, \]
  where the theta function is defined by
  \[ \theta(z) = q^{\frac{1}{8}} (e^{\pi i z} - e^{-\pi i z}) \prod_{n=1}^{\infty} (1-q^n)(1-q^n e^{2\pi i z})(1-q^n e^{-2\pi i z}). \]
  Here, we make the usual substitution $q = e^{2\pi i \tau}$. This operator satisfies the equations
  \begin{align*}
    L(z+1) &= L(z) \\
    L(z+\tau) &= g L(z) g^{-1}
  \end{align*}
  and has an expansion of the form
  \[ L(z) \sim \frac{\nu}{z} (1-\delta_{ij}). \]
  Therefore, the spectrum of $L(z)$ forms an $n$-sheeted cover over $E$, and in $T^* E$, if $g(C) = n$, then
  \[ C \sim n [E] + [F]. \]
  Then we will obtain a Lagrangian fibration where the fiber is the Jacobian $\mr{Jac}(C)$.
\end{exm}

\section{How to compute some path integrals exactly}
\label{sec:path_integrals}

Recall our setting of a symplectic manifold with Hamiltonian $(M, \omega, H)$. We will assume that there is an action of $T \cong U(1)^r$ with moment map $\mu \colon M \to \mf{t}^*$. We will also assume that $H$ is linear in the moment map, so
\[ H = \ev{\mu, \xi}. \]

\begin{exm}
  Consider $M = \R^{2n}$ with Darboux coordinates $p_i, q_i$, the standard action of $U(1)^n$, and moment map given by
  \[ \mu_j = \frac{1}{2} (p_j^2 + q_j^2). \]
  Then the Hamiltonian
  \[ H = \sum_{i=1}^n \xi_i \mu_i \]
  generates an action $\R \hookrightarrow T$ which is dense for generic $\xi$.
\end{exm}

If we now assume that $M$ is compact and the fixed points of the $T$-action are isolated, then to obtain the statistical-mechanical partition function for some inverse temperature $\beta$, we have the \textit{Duistermaat-Heckman} formula
\begin{align*}
  Z(\beta) &= \int_M \frac{\omega^n}{n!} e^{-\beta H} \\
  &= \sum_{\dd H_p = 0} \frac{e^{-\beta H(p)}}{\beta^n \prod_{i=1}^n \ev{m_i(p), \xi}},
\end{align*}
where $m_i(p)$ are the weights of the $T$-action on $T_p M$. With these weights, $H$ near $p$ behaves like
\[ H = H(p) + \frac{1}{2} \sum_{i=1}^r \xi_i \sum_{j=1}^n m_{ij}(p_j^2 + q_j^2). \]

\begin{exm}
If $M = S^2$ and $H = \cos\theta$ is the cosine of the azimuthal angle, we embed $M \subset M_{\C} \cong T^* S^2$, and the contour corresponding to the south pole goes into the cotangent directions.
\end{exm}

The Duistermaat-Heckman formula can be stated in the setting of $T$-equivariant cohomology. If $G$ is a Lie group acting on $M$, we will consider the Cartan model of equivariant cohomology.
\[ \Omega_G^*(M) \coloneqq \ms{Fun}(\g, \Omega^*(M))^G, \]
where we require that
\[ f(\Ad_{g^{-1}} \xi) = g^* f(\xi). \]
The grading on differential forms is deformed to be
\[ 2 \xi \pdv{\xi} + \deg_{\Omega^*_{\mr{DR}}}. \]
Then the equivariant differential is defined by
\[ D = \dd_{\mr{DR}} + \iota_{V(-)}, \]
where $V \colon \g \to \ms{Vect}(M)$ is a linear map. We can now compute
\begin{align*}
D^2 = \ms{Lie}_{V(-)} = 0,
\end{align*}
so we set $H_G^*(M)$ to be the cohomology of this complex. We can now compute equivariant integrals of $f \in \Omega_G^*(M)$. Note that $\int_M f \in \ms{Fun}(\g)^G$ and
\[ Z(\xi) = \int_M f_{\mr{top}}(\xi) = Z(\Ad_g \xi) \]
for all $g \in G$. Also note that
\[ \int_M (D \psi) = \int_M \dd \psi = 0. \]

We will now prove the Duistermaat-Heckman formula. If $\mu \colon M \to \g^*$ is the moment map, we claim that
\[ D(\omega + \mu(-)) = 0. \]
This follows from the definition of the moment map. We then see that
\[ D(\exp(\omega + \mu(-))) = 0, \]
and this is actually a Duistermaat-Heckman integral. If the $G$-action is free (and $G$ is compact), then every closed form is exact. Because $G$ is compact, there exists a $G$-invariant metric. Assuming that $G=T$ for now, choose some generic $\ol{\xi} \in \on{\ms{Lie}} T$. Then define
\[ \alpha = g(V(\ol{\xi}), -) \in \Omega^1(M). \]
We then obtain
\[ D \alpha = g(V(\ol{\xi}), V(\xi)) + \text{$2$-form} \]
where the function part is nonzero, and finally if
\[ \psi = \frac{\alpha}{D \alpha} f, \]
we obtain $f = D\psi$ whenever $Df = 0$. Now we can replace the integral in the Duistermaat-Heckmann formula with
\[ \int M_{\ep} + \sum_p \int_{B_{\ep}(p)}, \]
where $M_{\ep} = M \setminus \bigcup_p B_{\ep}(p)$, and finally use Stokes' theorem to obtain the desired result.

\section{Infinite-dimensional generalizations and supersymmetry}
\label{sec:infinite-supersymmetry}

Now suppose that $M = LN$, where $(N, g)$ is a Riemannian manifold. Then define for $\xi, \eta \in T_{\rho} M = \Gamma(S^1, \rho^* TN)$, we can define a $2$-form by
\[ \omega(\xi, \eta) = \int_{S^1} g(\xi, \nabla_t \eta) \dd{t}, \]
where $\nabla_t$ is the Levi-Cevita connection. The Hamiltonian that generates the standard $S^1$-action is
\[ H = \int g(\dot{\rho}, \dot{\rho}) \dd{t}. \]
Then the partition function of supersymmetric quantum mechanics on $N$ (index of the Dirac operator on $N$) is given by
\[ \int_{LN} e^{\omega - \beta H} = \int_N \wh{A}(TN), \]
where $\wh{A}$ is the $\wh{A}$-genus. The term on the right should be viewed as the contribution of non-isolated components of the fixed locus and can be thought of as Fourier modes of infinitesimal loops. Of course, there is the question of the role of $\beta$ in this formula.

\subsection{The Dirac operator}
\label{subsec:dirac}

First, associated to the metric $g |_{T_x N}$ we have the Clifford algebra, which is generated by symbols $\cancel{v}$ for $v \in T_x N = V$ with the relation
\[ \cancel{v} \cdot \cancel{u} + \cancel{u} \cdot \cancel{v} = g(u,v) 1. \]
This is the analogue of the the Heisenberg-Weyl algebra over a symplectic vector space $(W,\omega)$, which is generated by elements $\wh{w}$ for $w \in W$ with the relation
\[ \wh{w}_1 \cdot \wh{w}_2 - \wh{w}_2 \cdot \wh{w}_1 = \omega(w_1, w_2) \cdot 1. \]

The Clifford algebra $\mr{Cl}$ has irreducible representations, which are called \textit{spinors}. If we choose an orthonormal basis $\qty{e_i}$ of $V$, we can write any element as
\[ \alpha + \sum_i \beta_i \cancel{e}_i + \sum_{i < j} \gamma_{ij} \cancel{e}_i \cancel{e}_j + \cdots \]
In addition, we see that the Clifford algebra relation is a deformation of the exterior algebra, so $\mr{Cl} \cong \Lambda^* V$ as $\R$-vector spaces. The space of quadratic elements of the Clifford algebra is in fact $\on{\ms{Lie}} O(V)$. Because $O(V)$ is not simply connected, representations of its Lie algebra integrate to $\mr{Spin}(V)$, which is the universal cover of $SO(V)$ (at least if $\dim V > 2$). The group-like elements take the form
\[ g = \exp \frac{1}{2} \sum_{i<j} \gamma_{ij}\cancel{e}_i \cancel{e}_j. \]
Now we impose a complex structure compatible with the metric and define
\begin{align*}
c_a &= \cancel{e}_{2a-1} + \sqrt{-1} \cancel{e}_{2a} \\
c_a^* &= \cancel{e}_{2a-1} - \sqrt{-1} \cancel{e}_{2a} 
\end{align*}
for $a = 1, \ldots, \floor{\frac{\dim V}{2}}$.
These satisfy the relation
\[ c_a c_b + c_b c_a = 0. \]
Finally, we define the vector space
\[ S \coloneqq \C \ket{\mr{vac}} \oplus \bigoplus_a \C c_a \ket{\mr{vac}} \bigoplus_{a < b} \C c_a c_b \ket{\mr{vac}} \oplus \cdots \oplus \C c_1 \cdots c_{\floor{\frac{\dim V}{2}}} \ket{\mr{vac}}. \]
We impose that $c^* \ket{\mr{vac}} = 0$, and in general that $c^*$ moves to the left. We then define $\Gamma = e_{\dim V}$ if $\dim V$ is odd. It satisfies
\begin{align*}
\Gamma^2 = 1, \qquad \Gamma c_1 + c_a \Gamma = 0, \qquad \Gamma c_a^* + c_a^* \Gamma = 0. 
\end{align*}
From now on, we will assume that $\dim V = 2k$.

Note that $S$ is an irreducible representation of $\mc{Cl}$, but recall that $\mf{o}(V)$ is much smaller. It is spanned by elements of the form $c_a c_b, c_ac_b^*, c_a^* c_b^*$. There is a $\Z/2$-valued conserved charge called $(-)^F$ which is preserved by $\mf{o}(V)$. For $S$, it is defined by
\begin{align*}
  F \ket{\mr{vac}} &= -\frac{k}{2} \ket{\mr{vac}} \\
  F (c_{a_1} \cdots c_{a_p}\ket{\mr{vac}}) = p - \frac{k}{2}.
\end{align*}
Therefore, we can split
\[ S = S_+ \oplus S_- \]
by the parity of the $F$-charge. Then for any $u \in V$, we have
\[ \cancel{u} \colon S_{\pm} \to S_{\mp}. \]
These are called \textit{Dirac matrices}.

Unfortunately, there is not always a global spinor bundle on $N$. The obstruction is the Stiefel-Whitney class
\[ w_2(TN) \in H^2(N, \Z/2). \]
If this is nonzero, then we cannot glue spinors into a vector bundle and thus the manifold $N$ does not have a spin structure. Therefore, we will now assume that $w_2(TN) = 0$. We will now define the \textit{Dirac operator}
\[ \cancel{D} \colon \Gamma(S_+) \to \Gamma(S_-) \]
by the local formula
\[ \cancel{D} \coloneqq \sum_{i=1}^{\dim M} \gamma^i \nabla_i. \]
Here, $\nabla_i$ is a spin cover of the Levi-Civita connection and $\gamma^i$ is the physicist notation for $\cancel{e}_i$. We also define $\cancel{D}^*$ to be the same operator applied to $S_-$. The symbol of $\cancel{D}^* \cancel{D}$ or of $\cancel{D} \cancel{D}^*$ is given by the well-known formula
\begin{align*}
  \sum_{i,j} (\gamma^i \partial_ii)(\gamma^j \partial_j) &= \frac{1}{2} \sum_{i,j} (\gamma^i \gamma^j + \gamma^j \gamma^i)\partial_i \partial_j \\
                                                         &= \frac{1}{2} \sum_{i,j} g^{ij} \partial_i \partial_j + \cdots \\
  &= \Delta + \cdots
\end{align*}
Because of the analogy with the Laplacian, the spaces $H_{S_+} = \ker \cancel{D}$ and $H_{S_i} = \ker \cancel{D}^*$ are known as \textit{harmonic spinors}. There is a lucky coincidence that if $\wh{H} = \cancel{D} \cancel{D}^* + \cancel{D}^*\cancel{D}$, then
\begin{align*}
  \on{Index} \cancel{D} &= \dim \ker \cancel{D} - \dim \ker \cancel{D}^* \\
                        &= \Tr_{\mc{H}_{L^2(S)}} (-1)^F e^{-\beta \wh{H}} \\
                        &= -\Tr_{L^2(S_-)} e^{-\beta\cancel{D} \cancel{D}^*} + \Tr_{L^2(S_+)} e^{-\beta \cancel{D}^*\cancel{D}}
\end{align*}
for any $\beta > 0$. Note here that working on $\R^n$,
\[ \Tr_{L^2(S)} e^{-\beta \cancel{D} \cancel{D}^*} \approx \Tr e^{\beta \Delta} \approx\int \dd^n p e^{-\beta p^2} \sim \frac{1}{\beta^{\frac{n}{2}}}. \]

Now suppose that $\psi_{(k)} \in \Gamma(S_+)$ satisfies
\[ \cancel{D}^* \cancel{D} \psi_{(k)} = \ep_k \psi_{(k)}. \]
Then $\cancel{D} \psi_{(k)} = \chi_{(k)} \in \Gamma(S_-)$ satisfies
\[ \cancel{D} \cancel{D}^* \chi_{(k)} = \ep_k \chi_{(k)}. \]
This tells us that
\begin{align*}
  \Tr e^{-\beta\cancel{D}^* \cancel{D}} - \Tr e^{-\beta \cancel{D}\cancel{D}^*} &= \ker \cancel{D} - \ker \cancel{D}^* + \sum_{k,\ep_k > 0} e^{-\beta \ep_k} - \sum_{k,\ep_k > 0} e^{-\beta \ep_k} \\
                                                                                &= \on{Index} \cancel{D}.
\end{align*}
Note that this cancellation is our first example of \textit{supersymmetry}. Finally, we take the $\beta \to 0$ limit.

\subsection{The heat kernel}
\label{subsec:heat_kernel}

If we were considering on flat space the ordinary heat kernel
\[ K(x, x', \beta) = \mel{x}{e^{\beta \Delta}}{x'}, \]
then $\Tr e^{\beta \Delta}$ is computed by
\[ \int_{LN} [D x(t)] \exp \qty(-\frac{1}{2} \int_0^{\beta} g(\dot{x}, \dot{x}) \dd{t}). \]
Note that in flat space, the heat kernel must satisfy the PDE
\[ \pdv{\beta} K = \Delta_{x'} K = \Delta_x K. \]
As $\beta \to 0$,
\[ K(x,x',\beta) \to \delta^{(n)}(x-x'). \]
There must also be the integral formula
\[ \int \dd{x'} K(x,x';\beta_1) K(x',x'';\beta_2) = K(x,x'';\beta_1 + \beta_2), \]
so in fact, it is given by
\[ K(x,x';\beta) = \exp\qty(-\frac{(x-x')^2}{2\beta}) \frac{1}{(2\pi \beta)^{n/2}}. \]

In our curved situation, we obtain the limit
\[ \lim_{M \to \infty} \int \dd{x_1} \cdots \dd{x_M} K\qty(x_1,x_2;\frac{\beta}{M}) \cdots, \]
which can be well-approximated by the expression in flat space. Considering instead the loop space $L T^*N$, we now obtain
\[ \int_{L T^*N} [D x(s) Dp(s)] \exp \qty(i \int p \dd{x} - \beta \int_0^1 g^{-1}(p,p) \dd{s}). \]
Note that the $[D x(s) Dp(s)]$ carries the term $\prod_t \frac{1}{(2\pi\beta)^{\frac{\dim M}{2}}}$. Also, if we integrate out the $p$, we obtain the previous term
\[ \frac{1}{2\beta} \int_0^1 g(\dot{x}, \dot{x}) \dd{s}. \]

\subsection{Supersymmetric Duistermaat-Heckmann}
\label{subsec:supersymmetric}

We will now rewrite the integral we wanted to compute at the beginning of this section in new notation as
\[ \int_{LN} e^{\Omega + \ep H}. \]
Here, we rewrite
\begin{align*}
  H &= \frac{1}{2} \int_0^1 g(\dot{x}, \dot{x}) \dd{s} \\
  \Omega &= \frac{1}{2} \int_{S^1} g_{ij} \psi^i \nabla_s \psi^j \dd{s},
\end{align*}
where $(\psi^i(s)) \in \Gamma(S^1, x^* TN)$. The $\psi^i$ satisfy the formula
\[ \psi^i(s) \psi^j(s') = -\psi^j(s') \psi^i(s). \]
Finally, here $\ep = \frac{1}{\beta}$ is the equivariant parameter for the action of $U(1)$ on the cotangent directions.

Now, the Duistermaat-Heckmann normalized integral (which is not the Atiyah-Singer normalized integral) is
\begin{align*}
  \int_{LN} e^{\Omega + \ep H} &= \int_M \frac{1}{\prod_{n \neq 0} \prod_{\alpha} \qty(n \ep + \frac{1}{\ep}\alpha)} \\
  &= \int_N \prod_{\alpha} \frac{\pi \sigma_{\alpha}/\ep}{\sin(\pi \sigma_{\alpha}/\ep)},
\end{align*}
where $x^i(t)$ splits into Fourier modes as
\[ x^i(t) = x^i_0 + \sum_{n \neq 0} \xi_n^i e^{2\pi i n s} \]
and thus splits
\[ \mc{N}_{N/LN} = \bigoplus_{n \neq 0} (TN)_n, \]
and $\alpha$ are the Chern roots of $TN$. This exactly reproduces the $\hat{A}$-genus.
The Atiyah-Singer normalized integral has an extra factor of $\frac{1}{\prod_s \beta^{\frac{\dim M}{2}}}$.

\section{Duality}%
\label{cha:duality}

There are two main examples: $T$-duality in $2$d sigma-models and $S$-duality in $4$d gauge theories.

\subsection{$p$-form generalized gauge theory}

We will begin with $p$-form generalized gauge theory in $D$ spacetime dimensions. For some $A \in \Omega^p(M^D)$, define
\[ \mc{L} = \int_{M^D} \dd{A} \wedge * \dd{A}. \]
This is invariant under $A \to A + \dd{B}$. The space of fields in the $\R$-type theory is given by
\[ \mc{A}_{\R} = \Omega^P(M^D) / \dd{\Omega^{p-1}(M^D)}. \]
In the $U(1)$-type setting, the space of fields is smaller and is given by
\[ \mc{A}_{U(1)} = \Omega^p(M^D) / \Omega^p_{\Z}(M^D), \]
where $\Omega^p_{\Z}$ is the space of all $p$-forms alpha such that 
\[ \int_{Z^p} \alpha \in \Z \]
for all integral cycles $Z^p$.

\begin{exm}
    When $p=0$ and $D = 2$, then the $U(1)$-type theory describes maps $M^2 \to U(1)$. Locally, these look like $A \colon M^2 \to \R$ such that $A \sim A + n$. Of course, not every manifold is simply connected. Therefore, the true space of fields is
    \[ \mc{A}_{U(1)} = \ms{Maps}(M^2, U(1)). \]
    This has a decomposition by the topological type as follows. Let $t \in \R$ be a coordinate with
    \[ \int_{\R/\Z} \dd{t} = 1. \]
    Then the topological type is given by
    \[ [f^* \dd{t}] \in H^1(M^2, \Z). \]
\end{exm}

In general, we will assume that the curvature form $\dd{A}$ is not exact, but instead integral, so
\[ \dd{A} \in \Omega^{p+1}_{\Z}(M^D). \]
This can be achieved on an open cover $U_{\alpha}$ by choosing
\[ A_{\alpha} \in \Omega^p(U_{\alpha}) \]
such that 
\[ A_{\beta} - A_{\alpha} \in \Omega^p_{\Z}(U_{\alpha} \cap U_{\beta}). \]
Then the space of fields is the space of connections on all $U(1)$-bundles and its connected components are parameterized by
\[ L = H^{p+1}(M^D, \Z). \]

The partition function is now given by
\[ Z = \int_{\mc{A}_{U(1)}} [\mc{D}A] e^{-\frac{1}{4g^2} \int_{M^D} \dd{A} \wedge * \dd{A} + \int \theta \wedge \dd{A} \wedge \dd{A}}, \]
where
\[ \theta \in \Omega^{D-2(p+1)}(M^D) \]
is a closed form.

We may also replace $f \colon M^2 \to U(1)$ by functions
\[ f \colon M^2 \to T \cong U(1)^n = V/\Gamma. \]
We introduce tensors 
\[ B \in \Lambda^2 V^*, \qquad G \in S^2_+ V^*. \]
In this form, the partition function becomes
\[ Z(G,B,h) = \int_{\mc{A}_{U(1)}} [\mc{D}A] e^{-\int_{M^D} G_{ij} \dd{A^i} \wedge * \dd{A^j} + \int B_{ij} \wedge \dd{A^i} \wedge \dd{A^j}}, \]
where $\dd{B_{ij}} = 0$ and $B_{ij} \in \Omega^{D-2(p+1)}(M^D)$. We also have
\[ B \in \begin{cases}
    \Lambda^2 V^* & p+1 \text{ odd} \\
    S^2 V^* & p+1 \text{ even}.
\end{cases}
\]
We now want to find the critical points of $\mc{L}$, where $\delta \mc{L} = 0$. The equations for $A$ become
\[ \dd * \dd{A^i} = 0 \]
for all $i$. If we set $F^i \coloneqq \dd{A^i}$, the equation becomes 
\[ \dd{*F^i} = 0, \]
which are the \textit{generalized Maxwell equations}. Therefore, we obtain
\[ Z(G,B,h) = \mc{N}(G,h) \sum_{c \in \Lambda = H^{p+1}(M^D, \Gamma)} \exp\qty[-G_{ij} \ev{c^i,*c^j} - i \ev{B_{ij} \wedge c^i \wedge c^j}], \]
where $\mc{N}(G,h)$ is a regularized version of
\[ \det_{\Omega^p/\dd{\Omega^{p-1}}}(*\dd*\dd)^{-\frac{1}{2}} \]
which is defined as follows. Define
\[ \zeta_{\Delta^{(p)}}(s) \coloneqq \Tr(-\Delta^{(p)})^{-s}. \]
This is defined for $\Re s \gg 0$, and taking the analytic continuation, we define
\[ \mc{N}(G,h) \coloneqq \exp \frac{1}{2} \zeta'_{{\Delta}}(0). \]
We may need to assume that $D = 2(p+1)$. For example, if $D=2$, then the $\ev{c^i, * c^j}$ term knows only about the conformal structure of $M^2$. Finally, $H^{p+1}(M^D, \Gamma)$ is taken modulo torsion.

\subsection{Duality}

The duality is given by inverting 
\[ \tau \coloneqq iG + B \]
when $D = 4$ and inverting
\[ \tau = G+iB \]
when $D=2$. We may consider transformations of the form
\[ \tau \mapsto -\frac{1}{\tau} \]
or more generally
\[ \tau \mapsto (C\tau+D)^{-1}(A\tau + B), \]
where
\[ \mqty(A & B \\ C & D) \in \begin{cases}
    O(n,n,\Z) & D=2 \\
    \on{Sp}(2n,\Z) & D=4.
\end{cases} \]
Applying Poisson resummation, we obtain
\[ \int_{c \in \Lambda{\R}} \sum_{\check{c} \in \Lambda^*} e^{2\pi i \check{c}(c)} \exp \qty[-\frac{1}{2} G_{ij} \ev{c_i, *c_j} + \frac{i}{2} B_{ij} \ev{c_i, c_j}]. \]
Here, we have the identity
\[ \sum_j G_{ij} * c^j + \sqrt{-1} B_{ij} c^j + 2 \pi \sqrt{-1} \check{c}_i = 0 \]
in dimension $2$. Note that when $D=2$, $*^2 = -1$, while when $D=4$, $*^2 = +1$. In the $4$-dimensional setting, if we define
\[ c_{\pm} = \frac{c \pm *c}{2}, \]
the relation is
\[ \tau c_i - \ol{\tau} c_+ + 2 \pi \sqrt{-1} (\check{c}_+ + \check{c}_-) = 0. \]
Therefore,
\[ c_- = \frac{2 \pi \sqrt{-1}}{\tau} \check{c}_-, \qquad c_+ = -\frac{2 \pi \sqrt{-1}}{\tau} \check{c}_+. \]

The duality must give some reassignment of the degrees of freedom. Locally, if
\[ M^D = X^{D-1} \times \R, \]
which is noncompact, we must discuss the Hamiltonian system. The phase space is given by the stack
\[ T^* [ \Omega^p(X^{D-1}, V) / \Omega^p(X^{D-1},\Gamma) ]. \]
The symplectic form is given by
\[ \int_{X^{D-1}} (\delta E \wedge \delta A_{(p)}) + \int_{X^{D-1}} B_{ij} \wedge \delta(\dd{A^i}) \wedge \delta A^j_{(p)}, \]
Here,
\[ E \in \Omega^{D-1-p}(X^{D-1}, V^*) \]
is the electric field and $\dd{A}$ is the magnetic field. Now we can write the Hamiltonian as
\[ \mc{H} = \frac{1}{2} \int_X G^{ij} E_i \wedge * E_j + \frac{1}{2} \int_X G_{ij} \dd{A^i} \wedge * \dd{A}^j. \]

We return to the space of spatial fields $\mc{A}$, which has a decomposition into connected components indexed by
\[ c \in H^{p+1}(X^{D-1},\Gamma)/\mr{Tors}. \]
The component corresponding to some $c$ is 
\[ \mc{A}_c = H^p(X^{D-1},V) / H^p(X^{D-1},\Gamma), \]
which is the space of flat $p$-connections.

All of this can be understood via the quantum mechanics of a particle on $S^1$. If we define 
\[ \mc{H}_{\theta} = \qty{f(t) \mid f(t+2\pi) = e^{i\theta} f(t), \int \abs{f}^2 < \infty}, \]
then $\theta$ is the analogue of $(B_{ij})$. Then if the standard symplectic form on $T^* S^1$ is given by
\[ \dd{E} \wedge \dd{t}, \]
$E^2$ quantizes to $-\partial_t^2 = \wh{H}$, and the spectrum of this operator is given by
\[ E_n = (n+\theta)^2, \qquad n \in \Z. \]
For example, if $E_n = 0$, then $n$ and $-n$ have the same eigenvalues, if $\theta = \frac{1}{2}$, then the spectrum is doubly degenerate, and for any other value of $\theta$ the spectrum is simple. Therefore, the duality exchanges the electric and magnetic fields whenever $D = 2(p+1)$.

\chapter{Instantons}%
\label{cha:Instantons}

There are three main approaches:
\begin{itemize}
    \item Morse theory;
    \item Twisted supersymmetric $\sigma$-models;
    \item Twisted supersymmetric gauge theory.
\end{itemize}

These have a Hamiltonian and Lagrangian approach, and the latter will motivate the study of integrals on moduli spaces of instantons.

\section{Morse theory}

Note that this is not the same as what topologists call Morse theory. Let $(M,g)$ be a compact Riemannian manifold and $f \colon M \to \R$. Suppose all $x$ such that $\dd{f}_x = 0$ have nondegenerate Hessian, or in other words,
\[ \det \qty(\pdv[2]{f}{x^i}{x^j}) \neq 0. \]
We will consider $\Omega^*(M)$ with the differential
\[ D \coloneqq \dd + \dd{f} \wedge \colon \Omega^*(M) \to \Omega^{*+1}(M). \]
Using the standard scalar form
\[ (\alpha,\beta)_g = \int_M \alpha \wedge * \beta, \]
we define the adjoint
\[ D^* = \dd^* + \iota_{\nabla f}. \]
Physicists refer to the study of this package as \textit{supersymmetric quantum mechanics}. Here, the Hamiltonian is
\begin{align*}
    H &= DD^* + D^*D \\
    &= -\Delta_{\dd} + \ms{Lie}_{\nabla f} + \ms{Lie}^*_{\nabla f} + g(\nabla f, \nabla f).
\end{align*}
Note that $\ker H \cong H^*(M)$, and if $H\psi = 0$, then
\begin{align*}
    0 &= (\psi, H \psi) \\
    &= (D\psi, D\psi) + (D^* \psi, D^*\psi),
\end{align*}
so we see that $H\psi = 0$ if and only if $D \Psi = D^*\psi = 0$.

There are also excited states where $H \psi_i = E_i \psi_i$ for $E_i > 0$. Then we compute
\begin{align*}
    H D\psi_i &= DD^* D\psi_i \\
    &= D(H \psi_i - DD^* \psi_i) \\
    &= D(E_i \psi_i) \\
    &= E_i D \psi_i,
\end{align*}
so the states $D\psi_i$, $D^*\psi_i$, and $DD^* \psi_i$ are also eigenstates with eigenvalue $E_i$.

There is then the following trick. If $f \mapsto t f$ and $g$ is fixed for $t \gg 0$, then
\[ H = -\Delta_g + t^2 \norm{\nabla f}^2 + t (\cdots) \]
where $\norm{\nabla f}^2$ is a very large potential outside of the critical locus of $f$. We also have
\[ D_t = e^{-tf} \dd{e^{tf}} \]
on the ground states, which are isomorphic to $H^*(M)$. More generally, we note that
\begin{align*}
    D_t &= \dd + t \dd{f} \wedge \\
    D_t^* &= t^{-1} \dd^* + \iota_{\nabla f}.
\end{align*}
Ignoring the $t^{-1} \dd^*$ term, the Hamiltonian becomes
\[ \ms{Lie}_{\nabla f} + t \norm{\nabla f}^2 = e^{-tf}(\ms{Lie}_{\nabla f})e^{tf}. \]

In the world where $f \mapsto tf$ and $g \mapsto tg$, we obtain states
\begin{align*}
    \psi &\to e^{-tf} \psi \eqqcolon \psi_{\mr{out}} \\
    \psi &\to e^{tf} \psi \eqqcolon \psi_{\mr{in}}.
\end{align*}
As $t \to \infty$, we obtain $H_{\mr{in}}$ and $H_{\mr{out}}$, which are distributions.

\begin{exm}
    Now consider $M = \R$ and $f = \frac{\omega x^2}{2}$. Note in this case we need to use $L^2$ differential forms. Then we have basic operators
    \begin{align*}
        H_0 &\coloneqq -\partial_x^2 + \omega^2 x^2 + \omega \\
        &= (-\partial_x + \omega x)(\partial_x + \omega x) \\
        H^i &\coloneqq -\partial_x^2 + \omega^2 x^2 + \omega \\
        &= (\partial_x + \omega x)(-\partial_x + \omega x).
    \end{align*}
    Finally, we set $\alpha = \psi(x) \dd{x}$.

    The spectrum is given in the following way. If $H\psi = 0$, then the equality $(\partial_x + \omega x)\psi = 0$ implies that $\psi = e^{-f}$, which is fine if $\omega > 0$ and bad if $\omega < 0$. Then suppose there is $\psi_n$ such that $H \psi_n = n \abs{\omega} \psi_n$. Then we obtain
    \[ \psi_n = \begin{cases}
        (-\partial_x + \omega x)^n \qty(e^{-\frac{\omega x^2}{2}}) & \omega > 0 \\
        (\partial_x + \omega x)^{n-1} \qty(e^{\frac{\omega x^2}{2}}) & \omega < 0 .
    \end{cases}
    \]
    Then if $P$ is a polynomial,
    \[ P(\partial_x) e^{-t \abs{\omega}x^2} \to P(\partial_x) \delta(x) \]
    as $t \to \infty$ (here $t$ scales $\omega$). This tells us that $H_{\mr{in}}^0 \cong H^1_{\mr{out}}$ are regular functions, while eigenstates of $H_{\infty}$ are monomials. Similarly, $H^1_{\mr{in}} \cong H^0_{\mr{out}}$ are distributions supported at $x=0$, while the eigenstates of $H_{\infty}$ are those of the form $\partial_x^n \delta(x)$. Observe that
    \[ H_{\infty} = \ms{Lie}_{x \dv{x}}. \]
    Then, note that $\delta(x) \dd{x}$ is an invariant distribution-valued $1$-form. The pairing between the in and out states must be the usual pairing between functions and distributions.
\end{exm}

In higher dimensions, there will be both attracting and repelling states, so we can consider something like
\[ f = \sum_{i=1}^{d-m} \frac{x_i^2}{2} - \sum_{j=d-m+1}^d \frac{x_j^2}{2} \]
and i the end $\psi_{\mr{in}}$ is a distribution supported on the attracting manifold. This concludes the study of the local picture.

However, we must consider the compact picture. For example, consider $S^1 = \R \cup \infty$ where the Morse function $f$ is attracting at $x=0$ and repelling at $x=\infty$. Then near $\infty$, the states  corresponding to monomials will look like $\on{P.V.}x^n$. Applying $x \dv{x}$ and expanding
\[ \psi = \psi_{\infty} + \frac{1}{x}\psi_{\infty}^{(1)} + \cdots + \frac{1}{x^{n+1}} \psi_{\infty}^{(n+1)} + \cdots, \]
then the scaling $x \to tx, \ep to t\ep$ gives us correction terms of $\log \ep$ and $\log t$ to $\psi_{\infty}^{(n+1)}$. Here, the principal value
\begin{align*}
    (x^n,\psi) &= \on{P.V.} \int_{\R} x^n \psi \dd{x}
\end{align*}
is the finite part in the $\ep$-expansion of
\[ \int_{-\frac{1}{\ep}}^{\frac{1}{\ep}} x^n \psi \dd{x}. \]
Therefore, we see that critical points can talk to each other via excited states.

\section{$\sigma$-models}

We will consider maps $\Sigma \to (X,\omega)$ from a Riemann surface to a symplectic manifold. We will also choose an almost complex structure $J$ which is tame with respect to $\omega$. If we consider the functional
\[ \mathbb{S} = \frac{1}{2} \int_{\Sigma} g(\dd{\phi}, *_h \dd{\phi}) + i \int_{\Sigma} \phi^* \omega, \]
its critical points are simply harmonic maps.

In the Hammiltonian formalism, consider $M = LX$. If $\xi \in T_{\ell} LX$, then
\[ \dd{f}(\xi) = \int_{S^1} \omega(\xi, \dot{\ell}). \]
Note that
\[ g(\dd{\phi}, * \dd{\phi}) = \omega(J \dd{\phi}, *_h \dd{\phi}) \]
and
\[ \dd{\phi} = \frac{1+iJ}{2} \dd{\phi} + \frac{1-iJ}{2} \dd{\phi}. \]
Then $\mathbb{S}$ can be rewritten as
\[ \mathbb{S} = \int \phi^*(\pm \omega + iB) + \norm{\frac{1-iJ}{2} (\partial \text{ or }\ol{\partial})\phi}^2, \]
where $B$ is the $B$-field, satisfying $\dd{B} = 0$. Therefore, the absolute minima are solutions to the PDE
\[ \frac{1-iJ}{2} \ol{\partial}\phi = 0. \]

\begin{exm}
    Consider $X = \R^2$ with the standard symplectic form and complex structure. Let
    \[ f = \int_{S^1} p \dd{q}. \]
    Let $t$ be the loop variable and $s$ be the noncompact direction. Then we obtain the equations
    \begin{align*}
        \dv{p}{s} &= -\fdv{f}{p} = -\dv{q}{t} \\
        \dv{q}{s} &= -\fdv{f}{q} = -\dv{p}{t},
    \end{align*}
    which are the Cauchy-Riemann equations. If $z = s + it$ and $w = \exp(z)$, the Cauchy-Riemann equations give
    \[ \ol{\partial}_{\ol{z}} (q+ip) = 0. \]
\end{exm}


\section{$4$d supersymmetric Yang-Mills}
\textit{Notes for this section were provided by Davis Lazowski. They are unchanged except for a bit of formatting.}

The word symmetry here is actually a confusing one. There are two types of symmetries one encounters in physics, and one is \textbf{not} a symmetry. Sometimes what we mean by symmetry is a \textbf{redundancy}; mathematically what that means is that we're studying a quotient of some space $X$ by some symmetry $G$, $X/G$. This is a \textit{local symmetry}. The other type of symmetry is a \textit{global symmetry} where we really have $G$ acting on the space we are studying $X$.

In the case of a local symmetry, platonically there is a space $X$ with action of $G$, but that space is not accessible to us. Rather we only can see $X/G$.

Occasionally, it's the case that we find there exists $Y/H = X/G$. We call this a \textit{duality}.

Because the groups we deal with in physics are typically infinite dimensional groups, for example $\ms{Maps}(M,G)$ of a manifold into a finite dimensional group, these groups typically have lots of normal subgroups.  Sometimes rather than the full $\ms{ Maps }(M,G)$, our local symmetry is some sort of normal subgroup that preserves additional structure: for example, possibly the normal subgroup that sends certain marked points $\{x_{i}\}$ of $M$ into certain subgroups $\{H_{i}\}$ of $G$.

The global symmetry will then be whatever is left from $\ms{Maps}(M,G)$, for example in this case $\times_{i} G/H_{i}$.

For example, say we are studying $(M,g)$ a smooth Riemannian 4-manifold, with $G$ a compact simple Lie group, and a principal $G$-bundle $\mathcal{P}$ over $M$. Let $\mathcal{A}_{\mathcal{P}}$ the affine space of $G$-connections on $\mathcal{P}$.  There is a group $g_{\mathcal{P}} := \Gamma(M^{4},G \times_{Ad} \mathcal{P})$ acting on $\mathcal{A}_{\mathcal{P}}$; what we really want to study is the quotient,

\[
B_{\mathcal{P}} := \mathcal{A}_{\mathcal{P}}/g_{\mathcal{P}}
\]

The functional we seek to integrate over $B_{\mathcal{P}}$ is

\[
\int_{B_{\mathcal{P}}} [D\mathcal{A}] e^{-\frac{1}{4c^{2}} \int_{M} \Tr F \wedge \star_{g} F +  \frac{i \theta}{2\pi} \int_{M} \Tr F \wedge F + \dots}
\]
The measure here is induced from the $L^{2}$ metric, and $Tr$ refers to the killing form. Our integral consists of two parts.

\begin{itemize}
  \item The first is the Yang-Mills action;
\item The second is a *topological term* which does not depend on the metric and picks out some topological equivalence class of principal bundles. We normalise it so that the whole thing is an integral class.
\end{itemize}

Since the second term is topological, we know eventually that our integral is equivalent to
\[
\sum_{n \in \mathbb{Z}} e^{i n \theta} Z_{n}
\]
where $Z_{n}$ is the integral taken over bundles of Pontryagin class $n$.

  We could try to write this on a lattice, viewing $\mathcal{A}_{P}$ as the edges of a graph and $g_{\mathcal{P}}$ as the vertices. But if you did this naively you would find that the coupling has to go to zero.

  Instead, we can try a trick which we started to discuss last time. If we write
\[
F_{A}^{+} = \frac{1}{2} (F_{A} + \star F_{A})
\]
The Yang-mills term in the functional is equal to
\[
\norm{F_{A}^{+}}^{2} + \int \Tr F_{A} \wedge F_{A}
\]

Then we can rewrite the whole functional as

\[
e^{2\pi i \tau( -\frac{1}{8\pi^{2}} \int \Tr F_{A} \wedge F_{A} ) - \frac{1}{4c^{2}} \norm{F^{+}}^{2}}
\]

Wherein
\[
\tau = \frac{\vartheta}{2\pi} + \frac{4\pi i }{c^{2}}
\]
is some function in the upper half-space.

We could try to study the limit $c^{2} \to 0$, while keeping $\tau$ finite.  In that limit the functional goes to

\[
\delta(F_{A}^{+}) e^{2\pi i \tau c_{2}\mathcal{P}}
\]

Studying the moduli space of $\mathcal{M}_{\mathcal{P}} := \{ A | F_{A}^{+} = 0\}/g_{\mathcal{P}}$,  which we roughly expect to be finite dimensional because $\Lambda^{2,+} \mathbb{R}^{4} \simeq \mathbb{R}^{3}$, we can at the very least compute its virtual dimension.

To do so, assume
\begin{itemize}
  \item $A_{0}$ is so that $F_{A_{0}}^{+} = 0$;
  \item $A = A_{0} + \delta A$ so that $F_{A_{0} + \delta_{A}}^{+} = \dd_{A_{0}}^{+} \delta A$
        \item wherein $\dd^{+}_{A_{0}} : \Omega^{1}(M) \otimes \ad\mathcal{P} \to \Omega^{2,+} (M) \otimes \ad \mathcal{P}$
\end{itemize}

To the linear level, it suffices to computer $\ker \dd^{+}$. Since

\[
\dd^{+}_{A_{0}} \dd_{A_{0}} = \frac{1}{2} (1 + \star) \dd^{2}_{A_{0}}
\]
The virtual tangent bundle is
\[
T_{vir} := \ker \dd_{A_{0}}^{+}/\Im \dd_{A_{0}}
\]
which is the first cohomology of the Atiyah-Hitchin-Singer complex:
\[
\Omega^{0} \oplus \ad \mathcal{P} \to \Omega^{1} \otimes \ad \mathcal{P} \to \Omega^{2,+} \otimes \ad\mathcal{P}
\]
wherein the maps are $\dd_{A_{0}}, \dd_{A_{0}}^{+}$

If $H^{0},H^{2}$ of this complex don't vanish, we could have singularities or obstructions and therefore trouble counting dimension.



\subsection{``theory of differential forms on $\mathcal{M}_{\mathcal{P}}$''}
We could try to overcome the infinite dimensionality of the problem by seeking to develop a theory of differential forms on $\mathcal{M}_{\mathcal{P}}$, the space of anti-self dual connections, i.e. those $A$ such that $F_{A}^{+} = 0$. To do so, we would need some notion of local coordinates on $\mathcal{M}_{\mathcal{P}}$. Let's first try to do this on the space of connections without the ASD condition.

Define
\[
Q = \dd_{\mr{DR}}
\]
on the space of connections $\mathcal{A}_{\mathcal{P}}$.  Then
\[
QA = \Psi \in \Omega^{1}(M) \otimes \prod  \ad  \mc{P}
\]
This is a one-form, hence an 'odd' object, i.e. a fermion. So $Q$ meaps even to odd and vice versa.


But remember that $A := A_{\mu} dx^{\mu}$ is a *redundant* description, because connections related by gauge transformation are the same.  To deal with this, rather than $d_{DR}$ we will deal with *equivariant differential forms*, and write

\[
Q = \dd_{\mr{DR}} + \iota_{V\, \phi}
\]
On the Cartan model for equivariant cohomology, so that this differential is an operator on

\[
(\Omega^{\bullet} (\mathcal{A} \times \ms{Lie}(g_{\mathcal{P}})) \times \ms{Fun}(\ms{Lie}(g_{\mathcal{P}})))^{g_{\mathcal{P}}}
\]

Since $H^{\bullet}( (M \times EG)/G) \simeq H^{\bullet}(M/G)$, if we want this relation not only at the level of chomology but also on the actual space of differential forms, it's enough to study invariant forms, hence why we're taking invariants by $g^{\mathcal{P}}$.

This is like passing from $M/G \to M \times \mathfrak{g}/G$, which slightly improves the situation of stabilisers and therefore the space of equivariant differential forms.

Then,

\begin{align}
    QA &= \Psi
  \\
    Q\Psi &= d_{A} \phi, \phi \in \Gamma(\ad\mathcal{P}) = \ms{Lie}(g_{\mathcal{P}})
  \\
    Q\phi &= 0
\end{align}
The extra factor of the Lie algebra also defines an object $\overline{\phi}$, so that $Q \overline{\phi} = \eta$ where $Q \eta = [\phi, \overline{\phi}]$.  Here $\overline{\phi}$ lives in the $\ms{Lie}(g_{\mathcal{P}})$ inside our forms itself, whilst $\phi$ lives in the space $Fun(g_{\mathcal{P}})$ we tensor by.  Then

\[
Q = \dd_{\mr{DR}\ A} + \overline{\partial}_{\overline{\phi}} + \iota_{V(\phi)} + \iota_{[\phi,\overline{\phi}]} \partial_{\overline{\phi}}
\]
On the space of $g_{\mathcal{P}}$-invariant forms, $Q^{2} = 0$ since $Q$ is just the Lie derivative associated to transformation along $\phi$.

\subsection{Finite-dimensional model}

Suppose we are in a similar finite dimensional situation.

\begin{itemize}
  \item Suppose $G$ acts on a Riemannian manifold $(N,g)$;
  \item Suppose $f \in (\Omega^{\bullet} (N) \otimes \ms{Fun}(\ms{Lie}(G)))^{G}$;
        \item If $G$ acts freely, then there is a well-defined quotient by $G$. If $f$ is equivariantly closed,it  should correspond go to $\phi \in \Omega^{\bullet}(N,G)$.
\end{itemize}

\begin{rmk}
  Let $\pi : N \to N/G$. If $\phi$ is closed so that $\iota_{V} \pi^{\star}\phi = 0$ for all $\phi \in \mathfrak{g}$, with $\phi \in \Omega^{\bullet}(N/G)$.

    Further, characteristic classes of our bundle $N \to N/G$ are represented by invariant polynomials on our Lie algebra, $(S^{\bullet} \mathfrak{g}^{\vee})^{G}$. The functions $\phi$ should somehow correespond to these, `it looks too good to be a coincidence'. This consideration hits on the fact that we likely want to use a metric when we make this map.
\end{rmk}
Let's do what we already did a previously. Using the metric, we can build a one-form:

\[
g(V(\Phi),\bullet) := \theta \in \Omega^{1}(N)
\]
It has the following useful property: if $Df =0$, we can multiply

\[
f \to f e^{D\theta} = f(t) e^{-g(V(\phi), V(\overline{\phi})) + \mathcal{O}_{n, \overline{\phi}} + \dd_{g} \mathcal{O}_{\overline{\phi}} }
\]
This will \emph{not} change the cohomology class, but it will change the representative.  For example,
\[
f \to \phi
\]
is given by
\[
\int_{\mathfrak{g}} f(t) e^{D\theta}
\]
We can write this as

\[
\ \frac{f(\phi)}{\prod_{a}(\phi^{b}f_{ab} + \dots)}\overline{\omega}_{1} \wedge \dots \wedge \overline{\omega}_{\dim G} = \int_{\mathfrak{g}} f(t) e^{D\theta}
\]

In order to not worry about contours, we treat $\phi, \overline{\phi}$ as complex conjugates. There is a canonical measure on $\ms{Lie}(G)$,
\[
[\dd\phi/\on{Vol}(G)]
\]
Which is just the Haar measure which we have for free. Insert this into the integral $\int fe^{D\theta}$ as an additional term by which we integrate out $\phi$.

A procedure like this is `universal' in the sense that it produces a differential form on $N$. Because all the integrations I've done commute with $D$, this kind of averaging results in a form which is still equivariantly closed.


The secret of Yang-Mills theory is that it does exactly that: it takes something simple on the space of connections and does some sort of integration like that to produce a form on the space of connections

But there's a second piece, the \emph{anti self-dual condition}, which we did not consider.

If we have $M = s^{-1}(0)/G$ for some section $s$ of a vector bundle $E \to N$, how do we restrict differential forms to the vanishing locus of this section? Naively the idea is to just send
\[
f \to \delta(s) f
\]
where $\delta$ is just a $\delta$ function or $\delta$ form.  To do this properly we need to add on a \emph{Koszul complex} into our complex of forms which effectively has $s$ as a differential.

At the level of mysterious formulas, this is achieved in the following way. Extra complexes imply we have extra fields; introduce the field $\chi$ which for our purposes will be the self dual form $\chi \in \Omega^{2,+}(M) \otimes \prod \ad\mathcal{P}$. This $\chi$ represents the $(-1)$ term in the Koszul complex, so that the Koszul differential of $\chi$ is $\delta \chi := F_{A}^{+} = s$

Then we have a vector bundle $E_{P}$ over $\mathcal{A}_{\mathcal{P}}$ with fibres
\[
F = \Gamma\, (\Omega^{2,_+}(M) \otimes \ad \mathcal{P})
\]
Naively $Q^{2}$ is $d_{A}^{+}$, so not zero on the nose. This is fixed in the following way: we define \emph{another} field, $\chi \in \Omega^{2,+}(M) \otimes \prod \ad\mathcal{P}$, and a further field $H \in \Omega^{2,+}(M) \otimes \ad\mathcal{P}$, so that

\begin{align}
    Q\chi &= H
  \\
    QH &= [\chi,\phi]
\end{align}
Remember also that we had fields
\begin{itemize}
  \item $A$ a connection on $\mathcal{P}$
  \item $\overline{\phi} \in \Omega^{0} \otimes \ad \mathcal{P}$
  \item $\psi \in \Omega^{4}(M) \otimes \ad \mathcal{P}$
  \item $\eta \in \Omega^{0} \otimes \ad\mathcal{P}$
        \item $\phi \in \Omega^{0} \otimes \ad\mathcal{P}$
\end{itemize}
which is a minimal list of fields for Yang-Mills theory. $Q$ acted on these fields by

\begin{itemize}
  \item $Q\chi = H$
  \item $QH = [\chi, \phi]$
  \item $QA = \psi$
  \item $Q\psi = d_{A} \phi$
  \item $Q\overline{\phi} = \eta$
  \item $Q\eta = [\phi,\overline{\phi}]$
        \item $Q\phi = 0$
\end{itemize}
Note that for all but $\phi$ there is a canonical pairing between fields just by their type, called \emph{Berezin measure}. This gives us a measure:

\[
[DA D\psi][D\chi DH] [D\overline{\phi} D\eta] \qty[\frac{D\phi}{\mr{vol}(G)}]
\]
where we have only `cheated' in the last term, whose definition leaves something to be said in the fully infinite dimensional setting.

Consider the functional

\[
\exp Q \int_{M} \Tr X(F_{A}^{+} - \frac{1}{2} c^{2} H) + \Tr \psi \wedge \star \dd_{A} \overline{\phi} + \mr{vol}_{g}\Tr \eta[\phi,\overline{\phi}]
\]
We got these terms from looking at our metric $g= \int \Tr(\delta A \wedge \star \delta A)$ and applying $Q$ to create the most general supersymmetric or $Q$-closed functional.
This is the same as
\[
\exp  \int \Tr H F_{A}^{+} - c^{2} \int \Tr H\wedge \star H- \int \Tr \dd_{A} \phi \wedge \star \dd_{A} \overline{\phi} - \int \Tr[\phi, \overline{\phi}]^{c} \mr{vol}_{g} + \Tr(\chi \dd_{A}^{+} \psi + \dots)
\]
Since $H$ is quadratic, we could integrate it out.

Also integrating $\chi,\eta$ out represents the Atiyah-Hitchin-Singer complex, because it imposes the conditions $\dd_{A}^{+}\psi = 0, \dd_{A}^{\star} \psi = 0$, equivalent to $\psi \in \ker \dd_{A}^{+}$, $\psi$ is orthogonal to $\Im(\dd_{A})$, which gives us the first cohomology of the AHS complex.


The zero modes of $\chi$ correspond to the second cohomology of the AHS complex, and the zero modes of $\eta$ correspond to the first cohomology. It's something to keep in mind because zero modes are significant because they drop out of the exponential and so observables `have to provide the missing zero modes' for the fermionic integrations to be nonzero.


Also, we could determine that the functional implies that $\phi$ is `the curvature of the universal bundle evaluated at the point x', but we would need to talk more about this later on.

\subsection{When M is not compact}

There is one important change to the whole story when $M$ is not compact; in other words, if the metric is such that there is some `end' which is infinitely far away, the story has to be modified. We don't need to integrate over all fields, rather we want to restrict to those where the curvature goes to zero at $\infty$. Naively that's because the integral would be divergent if the curvature was not zero.

Since we want $\phi$ to approach a constant $a$, $\overline{\phi} \to \overline{a}$, such that $[a, \overline{a}]= 0$, we only use those gauge transformations which approach unity sufficiently fast at infinity. The rate at which they go to 1 is fixed by the requirement that $||\dd\phi||^{2} < \infty$. The result of that is that a finite dimensional group of those gauge transformations preserved now acts as a global symmetry! Instead of integrals over the moduli space of instantons of closed differential forms, one passes to equivariant differential forms for the moduli space of \textbf{framed} instantons, where framed means the vanishing condition above.

\chapter{Gauge-theoretic instanton counting}

The difference between mathematicians and physicists is that mathematicians like closed $4$-manifolds while physicists like $\R^4$. For mathematical precision, we will let $M^4$ be a compact Riemannian manifold and $\mc{P} \to M$ be a principal $G$-bundle for some compact Lie group $G$. These are classsified by the second Chern class
\[ k \coloneqq c_2(\mc{P}) \in H^4(M^4, \Z), \]
which physicists call the \textit{instanton charge}. We are looking for connections $\nabla = \dd + A$ whose curvature satisfies
\[ F_{\nabla} = - \star F_{\nabla}, \]
which are called \textit{instantons}. This equation in fact only depends on the conformal class of the metric.

Being anti-self-dual implies that the connection gives a minimum of the Yang-Mills action
\[ S = \int_{M^4} \Tr F_{\nabla} \wedge \star F_{\nabla}. \]
If $M^4$ is a complex surface and the metric $g$ is Hermitian, then $F_{\nabla}^{0,2} = 0$, which is equivalent to requiring that that $\ol{\partial}_A^2 = 0$. Therefore, any representation $E$ of $G$ gives rise to a holomorphic vector bundle $\mc{E}$ on $M^4$. This will come into play later when we cheat by replacing instantons with torsion-free sheaves.

The \textit{moduli space} of such instantons is
\[ \mc{M}_{\mc{P}} = \qty{\nabla \mid F_{\nabla}^+ = 0} / \mc{G}_{\mc{P}}, \]
where $\mc{G}_{\mc{P}}$ is the group of sections of the associated $\Ad_{\mc{G}}$-bundle acting by
\[ A \mapsto g^{-1} A g + g^{-1} \dd g. \]
For generic metrics $g$, this is a manifold of dimension
\[ 4 h^{\vee} k - \frac{\chi + \sigma}{2} \dim G, \]
where $h^{\vee}$ is the dual Coxeter number. We should note that even though $\mc{M}_k$ as a space depends only on the conformal class of the metric, the metric depends on $g$ itself. It also lives inside the space $\mc{A}_{\mc{P}}/\mc{G}_{\mc{P}}$ of all connections on $\mc{P}$. Our aim is to understand the infinite-dimensional integral
\[ \sum_{[\mc{P}], c_2(\mc{P}) = -k} e^{- \vartheta k} \int_{\mc{A}_{\mc{P}}/\mc{G}_{\mc{P}}} [\mc{D}A] e^{-\frac{1}{4g^2} S_{\mr{YM}}}. \]
Here, $\vartheta$ is valued on $S^1$. In the $g^2 \to 0$ limit, the asymptotics of the integral should become
\[ \sum_k q^k \int_{\mc{M}_k} \dd{\mu} (1 + O(g)), \]
where $q = e^{-\frac{8\pi^2}{g^2} + i\vartheta}$ and the measure is obtained by regularity and determinants of $\delta S_{\mr{YM}}$.

Instead of trying to understand this very complicated measure $\mu$, we can try to study simpler but still interesting integrals. For examples, Donaldson invariants are defined as integrals over $\mc{M}_k$ of cohomology classes associated to $2$-cycles $\Sigma_i$ and $0$-cycles $p$ on $M^4$. The problem is that $\mc{M}_k$ is non-compact because of the conformal invariance of the equations leading to delta-function solutions. Uhlenbeck discovered that via a complicated system of gauge transformations, these points can be filled in by ideal solutions, so there is the \textit{Uhlenbeck compactification}
\begin{align*}
    \ol{\mc{M}}_k &= \qty{(\nabla, x_1, \ldots, x_{\ell}) \mid c_2(\wt{\mc{P}}) - c_2(\mc{P}) = \ell} \\
    &= \mc{M}_k \cup \mc{M}_{k-1} \times M^4 \cup \mc{M}_{k-2} \times \on{Sym}^2 M^4 \cup  \cdots \cup \mc{M}_0 \times \on{Sym}^k M^4.
\end{align*}

\section{The case of $\R^4$}

This is the case that physicists are worried about. The problem is of course that $\R^4$ is non-compact, so we can view either $\R^4 = S^4 \setminus \infty$ or $\R^4 = \C^2 = \P^2 \setminus \P^1_{\infty}$. There are already interesting solutions in this case, but because our metric is singular at infinity, we will consider the moduli space
\[ \mc{M}_k^{\mr{framed}} = \qty{\nabla \mid F_{\nabla}^+ = 0} / \mc{G}_{\mc{P}}^{\infty}, \]
where $\mc{G}_{\mc{P}}^{\infty} \subset \mc{G}_{\mc{P}}$ is the set of those elements satisfying $g(x) \to 1$ as $x \to \infty$. This is a hyperk\"ahler manifold of dimension $4kN$ when $G = SU(N)$.

It is well-known that this (or a slight modification $\wt{\mc{M}}_k^{\mr{framed}})$ is a Nakajima quiver variety corresponding to the quiver data
\begin{equation*}
\begin{tikzcd}
    k \ar[shift left=1]{d}{J} \arrow[out=0,in=30,loop,swap,"B_2"] \arrow[out=90,in=120,loop,swap,"B_1"]\\
    N \ar[shift left=1]{u}{I}
\end{tikzcd}
\end{equation*}
Here, we need to modify the instanton equations to
\begin{align*}
    [B_1, B_2] + IJ &= 0 \\
    [B_1, B_1^+] + [B_2, B_2^+] + I I^+ - J^+ J &= \zeta \cdot 1_k,
\end{align*}
which in algebraic geometry corresponds to changing the value of the moment map.

This moduli space has an instanton interpretation if we replace $\R^4$ by a noncommutative $\R^4_{\zeta}$, which has coordinates $z_1, z_2, \ol{z}_1, \ol{z}_2$ with commutators
\[ [z_1, \ol{z}_1] = -\frac{\zeta}{2} = [z_2, \ol{z}_2]. \]
In the $N=2$ case, we have $\mc{M}_2^{\mr{framed}} = \R^4 \times (\R^4 \setminus 0)/\Z_2$ while $\ol{\mc{M}}_k^{\mr{framed}} = \R^4 \times \R^4/\Z_2$ and $\wt{\mc{M}}_k^{\mr{framed}} = \R^4 \times T^* S^2$.

There is also an action of $U(2) \times SU(N)$, where for elements
\[ a = \mqty(\dmat{a_1,\ddots,a_N}) \in \on{\ms{Lie}} U(N) \otimes \C \]
and
\[\ep = \mqty(0 & \ep_1 \\ -\ep_1 & 0 \\ & & 0 & \ep_2 \\ & & -\ep_2 & 0) \in \on{\ms{Lie}} U(2) \otimes \C, \]
the corresponding vector field is denoted by $V(a,\ep)$. Given a choice of
\[ \lambda_{\ol{a}, \ol{\ep}} = G_{\wt{\mc{M}}}(V(\ol{a}, \ol{\ep}), -), \]
the partition function is
\[ Z_k = \int_{\wt{\mc{M}}_k^{\mr{framed}}} \exp \qty(-G_{\wt{\mc{M}}}(V(a,\ep), V(\ol{a}, \ol{\ep}))) \frac{1}{(2kN)!} (\dd \lambda_{\ol{a}, \ol{\ep}})^{2kN}. \]
This is not the most general formulation, but it is a distilled version of Yang-Mills with some supersymmetry. It turns out that $Z_k$ is actually a rational function of degree $-2kN$ in $a$ and $\ep$.

One generalization of this is
\[ Z(a,\ep,\Lambda) = Z^{\mr{part}} \times \sum_{k=0}^{\infty} \Lambda^{2kN} Z_k(a,e), \]
where
\[ Z^{\mr{part}}(a,\ep) = \prod_{i \neq j} \Gamma_2(a_i-a_j; \ep_1, \ep_2). \]
Here, $\Gamma_2$ has the asymptotics
\[ \Gamma_2(x; \ep_1, \ep_2) \sim \prod_{n,m \geq 1}(x+\ep_1 n + \ep_2 m) \]
and solves the equation
\[ \frac{\Gamma_2(x+\ep_1) \Gamma_2(x+\ep_2)}{\Gamma_2(x) \Gamma_2(x+\ep_1+\ep_2)} = x. \]
In the limit as $\ep_1, \ep_2 \to 0$, we expect the asymptotics
\[ \exp\qty(\frac{1}{\ep_1 \ep_2} \mc{F}(a,\Lambda) + \cdots), \]
so the question now is to evaluate $\mc{F}(a, \Lambda)$. In the case when $G = SU(N)$, we can do it by localization.

A standard computation tells us that
\[ Z_k(a,\ep) = \sum_{\substack{(\lambda^{(1)}, \ldots, \lambda^{(k)}) \\ \abs{\lambda^{(1)}} + \cdots + \abs{\lambda^{(k)}} = k}} \frac{1}{\prod_{i,j=1}^N \qty( \prod_{\square \in \lambda^{(i)}} (a_i-a_j + f(\ep_1, \ep_2))) \qty(\prod_{\blacksquare \in \lambda^{(j)}} ( a_j - a_i + g(\ep_1, \ep_2) ))}, \]
where $f$ and $g$ are defined using the (relative) arms and legs of the two Young diagrams. For example, when $i=j$, we end up with $f = \ep_1(\mr{arm}_{\square}+1)-\ep_2 \mr{leg}_{\square}$ and $g = -\ep_1 \mr{arm}_{\square} + \ep_2(\mr{leg}_{\square}+1)$.

Treating $Z_k$ as a probability measure on the set of Young diagrams, we obtain the observables
\[ Y(x)[\lambda^{(1)}, \ldots, \lambda^{(N)}] = \prod_{\alpha =1}^N \frac{\prod_{\square \in \partial_+ \lambda^{(\alpha)}}(x-a_{\alpha}-c_{\square})}{\prod_{\blacksquare \in \partial_- \lambda^{(\alpha)}}(x-a_{\alpha}-c_{\blacksquare}-\ep_1-\ep_2)}. \]
This function knows essentially everything about the shape of the diagrams, and its vacuum expectation is
\[ \ev{Y(x)} = \frac{1}{Z} \sum_{\vec{\lambda}} Y(x)[\vec{\lambda}] \mu_{\vec{\lambda}}(a,\ep) \Lambda^{2N\abs{\vec{\lambda}}}. \]
It satisfies the property that
\[ \ev{Y(x+\ep_1 + \ep_2) + \frac{\Lambda^{2N}}{Y(x)}} \]
has no poles in $x$ and is in fact a polynomial of degree $N$. Physically, this is interpreted as an interaction between two neighboring instanton sectors.

If we send $\ep_1, \ep_2 \to 0$, then $\ev{Y(x)} = \mc{Y}(x)$ we then obtain the algebraic equation
\[ \mc{Y}(x) + \frac{\Lambda^{2N}}{\mc{Y}(x)} = T(x), \]
where the coefficients of $T(x)$ are defined by
\[ \oint_{A_i} x \frac{\dd Y}{Y} \sim a_i. \]
This is a hyperelliptic curve.



\end{document} 

%%% Local Variables:
%%% mode: latex
%%% TeX-master: t
%%% End:
