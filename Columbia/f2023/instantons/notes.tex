\documentclass[leqno, openany]{memoir}
\setulmarginsandblock{3.5cm}{3.5cm}{*}
\setlrmarginsandblock{3cm}{3.5cm}{*}
\checkandfixthelayout

\usepackage{amsmath}
\usepackage{amssymb}
\usepackage{amsthm}
%\usepackage{MnSymbol}
\usepackage{bm}
\usepackage{accents}
\usepackage{mathtools}
\usepackage{tikz}
\usetikzlibrary{calc}
\usetikzlibrary{automata,positioning}
\usepackage{tikz-cd}
\usepackage{forest}
\usepackage{braket} 
\usepackage{listings}
\usepackage{mdframed}
\usepackage{verbatim}
\usepackage{physics}
\usepackage{stmaryrd}
\usepackage{mathrsfs} 
\usepackage[normalem]{ulem} 
\usepackage{stackengine}
\usepackage{bbm}
%\usepackage{/home/patrickl/homework/macaulay2}

%font
\usepackage[sc]{mathpazo}
\usepackage{eulervm}
\usepackage[scaled=0.86]{berasans}
\usepackage{inconsolata}
\usepackage{microtype}

%CS packages
\usepackage{algorithmicx}
\usepackage{algpseudocode}
\usepackage{algorithm}

% typeset and bib
\usepackage[english]{babel} 
\usepackage[utf8]{inputenc} 
\usepackage[T1]{fontenc}
\usepackage[bookmarks, colorlinks, breaklinks]{hyperref} 
\hypersetup{linkcolor=blue,citecolor=magenta,filecolor=black,urlcolor=blue}
\usepackage{cleveref}
\usepackage[backend=biber,style=alphabetic,maxalphanames=4,maxnames=5,hyperref,backref=true,backrefstyle=none]{biblatex}
\usepackage{xpatch}
\xpatchbibmacro{pageref}{parens}{backrefparens}{}{}
\crefname{equation}{}{}

% other formatting packages
\usepackage{float}
\usepackage{booktabs}
\usepackage[shortlabels]{enumitem}
\usepackage{csquotes}
\usepackage{titlesec}
\usepackage{titling}
\usepackage{parskip}
\usepackage{graphicx}
\graphicspath{{./}}

\usepackage{lipsum}

% delimiters
\DeclarePairedDelimiter{\gen}{\langle}{\rangle}
\DeclarePairedDelimiter{\floor}{\lfloor}{\rfloor}
\DeclarePairedDelimiter{\ceil}{\lceil}{\rceil}


\newtheorem{thm}{Theorem}[section]
\newtheorem{cor}[thm]{Corollary}
\newtheorem{prop}[thm]{Proposition}
\newtheorem{lem}[thm]{Lemma}
\newtheorem{conj}[thm]{Conjecture}
\newtheorem{quest}[thm]{Question}
\newtheorem{prob}[thm]{Problem}

\theoremstyle{definition}
\newtheorem{defn}[thm]{Definition}
\newtheorem{defns}[thm]{Definitions}
\newtheorem{con}[thm]{Construction}
\newtheorem{exm}[thm]{Example}
\newtheorem{exms}[thm]{Examples}
\newtheorem{notn}[thm]{Notation}
\newtheorem{notns}[thm]{Notations}
\newtheorem{addm}[thm]{Addendum}
\newtheorem{exer}[thm]{Exercise}

\theoremstyle{remark}
\newtheorem{rmk}[thm]{Remark}
\newtheorem{rmks}[thm]{Remarks}
\newtheorem{warn}[thm]{Warning}
\newtheorem{sch}[thm]{Scholium}


% unnumbered theorems
\theoremstyle{plain}
\newtheorem*{thm*}{Theorem}
\newtheorem*{prop*}{Proposition}
\newtheorem*{lem*}{Lemma}
\newtheorem*{cor*}{Corollary}
\newtheorem*{conj*}{Conjecture}

% unnumbered definitions
\theoremstyle{definition}
\newtheorem*{defn*}{Definition}
\newtheorem*{exer*}{Exercise}
\newtheorem*{defns*}{Definitions}
\newtheorem*{con*}{Construction}
\newtheorem*{exm*}{Example}
\newtheorem*{exms*}{Examples}
\newtheorem*{notn*}{Notation}
\newtheorem*{notns*}{Notations}
\newtheorem*{addm*}{Addendum}


\theoremstyle{remark}
\newtheorem*{rmk*}{Remark}

% shortcuts
\newcommand{\Ima}{\mathrm{Im}}
\newcommand{\A}{\mathbb{A}}
\newcommand{\F}{\mathbb{F}}
\newcommand{\E}{\mathcal{E}}
\newcommand{\G}{\mathbb{G}}
\newcommand{\N}{\mathbb{N}}
\newcommand{\R}{\mathbb{R}}
\newcommand{\C}{\mathbb{C}}
\newcommand{\Z}{\mathbb{Z}}
\newcommand{\Q}{\mathbb{Q}}
\renewcommand{\k}{\Bbbk}
\renewcommand{\L}{\mathbb{L}}
\renewcommand{\P}{\mathbb{P}}
\newcommand{\M}{\overline{M}}
\newcommand{\g}{\mathfrak{g}}
\newcommand{\h}{\mathfrak{h}}
\newcommand{\n}{\mathfrak{n}}
\renewcommand{\b}{\mathfrak{b}}
\newcommand{\ep}{\varepsilon}
\newcommand*{\dt}[1]{%
   \accentset{\mbox{\Huge\bfseries .}}{#1}}
\renewcommand{\abstractname}{Official Description}
\newcommand{\mc}[1]{\mathcal{#1}}
\newcommand{\T}{\mathbb{T}}
\newcommand{\mf}[1]{\mathfrak{#1}}
\newcommand{\mr}[1]{\mathrm{#1}}
\newcommand{\ms}[1]{\mathsf{#1}}
\newcommand{\mt}[1]{\mathtt{#1}}
\newcommand{\on}[1]{\operatorname{#1}}
\newcommand{\ol}[1]{\overline{#1}}
\newcommand{\ul}[1]{\underline{#1}}
\newcommand{\wt}[1]{\widetilde{#1}}
\newcommand{\wh}[1]{\widehat{#1}}
\renewcommand{\div}{\operatorname{div}}
\newcommand{\bir}{\sim_{\mr{bir}}}
\newcommand{\stacks}[1]{\href{https://stacks.math.columbia.edu/tag/#1}{#1}}
\newcommand{\ostar}{\stackMath\mathbin{\stackinset{c}{0ex}{c}{0ex}{\star}{\bigcirc}}}

\DeclareMathOperator{\Der}{Der}
\DeclareMathOperator{\Def}{Def}
\DeclareMathOperator{\Bl}{Bl}
\DeclareMathOperator{\NE}{NE}
\DeclareMathOperator{\Tor}{Tor}
\DeclareMathOperator{\Hom}{Hom}
\DeclareMathOperator{\Ext}{Ext}
\DeclareMathOperator{\End}{End}
\DeclareMathOperator{\ad}{ad}
\DeclareMathOperator{\Ad}{Ad}
\DeclareMathOperator{\Aut}{Aut}
\DeclareMathOperator{\Rad}{Rad}
\DeclareMathOperator{\Pic}{Pic}
\DeclareMathOperator{\supp}{supp}
\DeclareMathOperator{\Supp}{Supp}
\DeclareMathOperator{\sgn}{sgn}
\DeclareMathOperator{\spec}{Spec}
\DeclareMathOperator{\Spec}{Spec}
\DeclareMathOperator{\proj}{Proj}
\DeclareMathOperator{\Proj}{Proj}
\DeclareMathOperator{\ord}{ord}
\DeclareMathOperator{\Div}{Div}
\DeclareMathOperator{\depth}{depth}
\DeclareMathOperator{\coker}{coker}
\DeclareMathOperator{\codim}{codim}
\DeclareMathOperator{\ch}{ch}
\DeclareMathOperator{\Hilb}{Hilb}

% Section formatting
\titleformat{\section}
    {\Large\sffamily\scshape\bfseries}{\thesection}{1em}{}
\titleformat{\subsection}[runin]
    {\large\sffamily\bfseries}{\thesubsection}{1em}{}
\titleformat{\subsubsection}[runin]{\normalfont\itshape}{\thesubsubsection}{1em}{}

\title{COURSE TITLE}
\author{Lectures by INSTRUCTOR, Notes by NOTETAKER}
\date{SEMESTER}

\newcommand*{\titleSW}
    {\begingroup% Story of Writing
    \raggedleft
    \vspace*{\baselineskip}
    {\Huge\itshape The Count of Instantons \\ Fall 2023}\\[\baselineskip]
    {\large\itshape Notes by Patrick Lei}\\[0.2\textheight]
    {\Large Lectures by Nikita Nekrasov}\par
    \vfill
    {\Large \sffamily Columbia University}
    \vspace*{\baselineskip}
\endgroup}
\pagestyle{simple}

\chapterstyle{ell}


%\renewcommand{\cftchapterpagefont}{}
\renewcommand\cftchapterfont{\sffamily}
\renewcommand\cftsectionfont{\scshape}
\renewcommand*{\cftchapterleader}{}
\renewcommand*{\cftsectionleader}{}
\renewcommand*{\cftsubsectionleader}{}
\renewcommand*{\cftchapterformatpnum}[1]{~\textbullet~#1}
\renewcommand*{\cftsectionformatpnum}[1]{~\textbullet~#1}
\renewcommand*{\cftsubsectionformatpnum}[1]{~\textbullet~#1}
\renewcommand{\cftchapterafterpnum}{\cftparfillskip}
\renewcommand{\cftsectionafterpnum}{\cftparfillskip}
\renewcommand{\cftsubsectionafterpnum}{\cftparfillskip}
\setrmarg{3.55em plus 1fil}
\setsecnumdepth{subsection}
\maxsecnumdepth{subsection}
\settocdepth{subsection}

\addbibresource{../../math.bib}
\DefineBibliographyStrings{english}{
    backrefpage={$\leftarrow$},
    backrefpages={$\leftarrow$},
}

\begin{document}
    
\begin{titlingpage}
\titleSW
\end{titlingpage}

\thispagestyle{empty}
\section*{Disclaimer}%
\label{sec:disclaimer}

These notes were taken during the lectures using \texttt{emacs}. 
Any errors are mine and not the speakers'. 
In addition, my notes are picture-free (but will include commutative diagrams) and are a mix of my mathematical style and that of the lecturers. Also, notation may differe between lecturers.
If you find any errors, please contact me at \texttt{plei@math.columbia.edu}.

\section*{Description}

Graduate level introduction to modern mathematical physics with the emphasis on the geometry and physics of quantum gauge theory and its connections to string theory.  We shall zoom in on a corner of the theory especially suitable for exploring non-perturbative aspects of gauge and string theory: the instanton contributions. Using a combination of methods from algebraic geometry, topology, representation theory and probability theory we shall derive a series of identities obeyed by generating functions of integrals over instanton moduli spaces, and discuss their symplectic, quantum, isomonodromic, and, more generally, representation-theoretic significance.

Quantum and classical integrable systems, both finite and infinite-dimensional ones, will be a recurring cast of characters, along with the other (qq-) characters.


\newpage

\tableofcontents

\chapter{Introduction}
\label{cha:intro}

We will be discussing three types of physics in an attempt to create something mathematically interesting:
\begin{itemize}
\item Classical physics;
\item Statistical physics;
\item Quantum physics;
\end{itemize}

\section{Classical physics}
\label{sec:classical}

\subsection{Hamiltonian dynamics}
\label{subsec:hamiltonian}

We will begin with a space of classical states, which is most commonly known as a \textit{phase space}. This is a symplectic manifold $(M, \omega)$, where $\dim M = 2m$ and $\omega \in \Omega^2(M)$ satisfies
\begin{align*}
  \dd{\omega} &= 0 \\
  \omega \wedge \cdots \wedge \omega &\neq 0.
\end{align*}
This carries a function
\[ H \colon M \to \R, \]
called a \textit{Hamiltonian}. Then there is a vector field $V_H$ described by
\[ \dd{H} = \iota_{V_H} \omega, \]
which generates a $1$-parameter group $g^t$ of symplectomorphisms of $M$. The evolution law of the physical system is given by
\[ \dot{x} = V_H(x). \]
Because $g^t$ acts by symplectomorphisms, the graph
\[ \Gamma_{g^t} = \qty{(m, g^t(m))} \subset M \times M \]
is a Lagrangian submanifold. Recall that a submanifold $L \subset M$ of a symplectic manifold is called \textit{Lagrangian} if $\dim L = m$ and $\omega |_L = 0$.

\begin{exer}
  Locally any symplectic manifold is given by $M = \R^{2m}$ with the symplectic form
  \[ \omega = \sum_{i=1}^m \dd{p_i} \wedge \dd{q^i}. \]
  This coordinate system on $\R^{2m}$ is unique up to $\mr{Sp}(2m)$.
\end{exer}

Now if $V$ is a vector field such that $\on{Lie}_V \omega = 0$, then $\dd{\iota_V \omega} = 0$. Thus there locally exists a Hamiltonian $h_V$. We also need the Hamiltonian vector field to be linear in the local coordinates, so the Hamiltonian itself must be quadratic.

\begin{exm}
  An important example of a symplectic manifold is $T^* B$ for any smooth manifold $B$. There is a $1$-form $\theta$ on $T^* B$ given by the following formula. If $v$ is a tangent vector at the point $(p,b)$, then
  \[ \theta(v) \coloneqq p(\pi_* v). \]
  Then $\omega = \dd{\theta}$ is a symplectic form.
\end{exm}

\begin{exm}
  Another large class of examples are obtained by \textit{symplectic reduction}. Here, we suppose that a symplectic manifold $M$ carries the action of a compact Lie group $G$ by exact symplectomorphisms. This defines a \textit{moment map} $M \xrightarrow{\mu} \g^*$ by the formula
  \[ \ev{\mu(m), \xi} = h_{V_{\xi}}(m). \]
  There is some ambiguity in the choice of constants, but in the end we obtain a new space
  \[ M \sslash G \coloneqq \mu^{-1}(0) / G. \]
  In practice, we want the moment map $\mu$ to be equivariant with respect to the coadjoint action on $\g^*$. Then the manifold $M \sslash G$ has a canonical symplectic form, but this requires a lot of work.

  Now consider $M = \R^{2m}$ and $G = U(1)$, where we write $M = (\R^2)^m$ and $U(1)$ acts by rotations. Then the moment map is actually
  \[ \mu = \sum_{i=1}^m \frac{1}{2} (p_i^2 + q_i^2) - r, \]
  so $\mu^{-1}(0)$ is a sphere. We then obtain
  \[ \R^{2m} \sslash U(1) = S^{2m-1}/U(1) = \P^{m-1}. \]
  We made no use of the complex numbers, so the fact that we obtain a complex manifold will be viewed as a bonus. The reduced symplectic form is simply $r$ times the Fubini-Study form.

  Note that $\P^{m-1}$ is compact, so the interpretation that phase space records position and momentum breaks down. In this case, our phase space is called the \textit{classical spin phase space}, where the motion is by rotations rather than by translation.
\end{exm}

\begin{rmk}
  We will often consider time-dependent Hamiltonians, where $\dot{x} = V_{H(t)}(x)$.
\end{rmk}

\subsection{Lagrangian mechanics}
\label{subsec:lagrangian}

There is another point of view, where dynamics on $M$ are given by an optimization problem on the space
\[ \mc{P}M \coloneqq \ms{Map}([0,1], M) \]
of paths in $M$. We will consider an \textit{action}
\[ \mathbb{S}[\gamma] \coloneqq \int_{\gamma} \theta - \int_0^1 \gamma(t)^* H(t) \dd{t}, \]
where we assume that $\omega = \dd{\theta}$. If we further require that $\gamma(0) \in L_0$ and $\gamma(1) \in L_1$, our dynamics are well-defined if $L_0, L_1$ are Lagrangian submanifolds. Of course, we are looking for paths were $\delta \mathbb{S} = 0$.

\subsection{Classical field theory}
\label{subsec:classicalft}

Classical field theory should be thought of as an infinite-dimensional version of classical mechanics, where we study loopified versions of finite-dimensinoal manifolds. We want to consider integrals
\[ \mathbb{S} \coloneqq \int_{(\Sigma, h)} \mc{L}[\phi, \partial \phi] \mr{vol}_h, \]
where $\Sigma$ is the spacetime, $h$ is a metric, and $\phi$ are the \textit{fields}. Fields could be one of several options:
\begin{itemize}
\item Scalars $f \colon \Sigma \to X$, where $X$ is a Riemannian manifold;
\item Connections $\nabla$ on principal bundles
  \begin{equation*}
    \begin{tikzcd}
      G \ar{r} & P \ar{d} \\
      & \Sigma,
    \end{tikzcd}
  \end{equation*}
  called \textit{gauge fields};
  \item The metric $h$ itself, called \textit{gravity}.
\end{itemize}

We can introduce more complexity into the problem by varying the action and looking for solutions of PDEs, introducing boundary to $\Sigma$, and other operations. Also note that Lagrangian mechanics can be interpreted as a $1$-dimensional classical field theory.

\section{Statistical physics}
\label{sec:stat}

In statistical physics, a point is replaced by a cloud of points, or a probability measure. For example, the measure could contain the term $e^{-\beta H}$, where $H$ was the classical Hamiltonian and $\beta$ is a parameter of our distribution (inverse temperature). The system often flows to a stationary distribution, which is determined by the outside world. In reality, the distribution will have the form
\[ \frac{1}{Z} e^{-\beta H} \qquad Z = \int_M e^{-\beta H} \frac{\omega \wedge \cdots \wedge \omega}{m!}. \]
This factor $Z$ is called the \textit{partition function}, and most of our energy is spent on computing this partition function.

\section{Classical physics with multiple times}
\label{sec:times}

Suppose we have Hamiltonians $H_1, \ldots, H_k$ such that $[V_{H_i}, \ldots, V_{H_j}] = 0$. Then we obtain dynamics
\[ \ul{\gamma} \colon [0, \ep]^p \to M \]
and an action
\[ \mathbb{S} = \int \theta - \sum_{k=1}^p \ul{\gamma}^* H_k \dd{t_k} \]
defined on paths inside the cube $[0,\ep]^p$. The extreme case of this is an \textit{integrable system} where the $H_i$ are functionally independent, and the maximum possible value of $p$ is $m$.

\begin{thm}[Liouville-Arnold]
  If the motion is finite (fits in a compact set), then locally $M$ is a fibration
  \begin{equation*}
    \begin{tikzcd}
      T^m \ar{r} & M \ar{d} \\
      & B
    \end{tikzcd}
  \end{equation*}
  such that the $H_i$ factor through $B$ and the $V_{H_k}$ span the rotations on each $S^1$ factor of $T^m$.
\end{thm}

A typical trajectory is a winding of the torus, where if $\theta_i$ are the angle coordinates on $T^m$, there is the formula
\[ \theta^i(t) = \theta^i(0) + \omega^i t. \]
Generically, these paths will have dense image.

In the special case of an integrable system, there are \textit{action-angle variables}, where the symplectic form is
\[ \omega = \sum_{i=1}^m \dd{I_i} \wedge \dd{\theta^i}. \]
The $\theta^i$ are defined up to $SL(m, \Z)$ affine transformations. If $C_i \in H_1(T^m, \Z)$ form a basis, then the $I_i$ are defined by
\[ I_i = \frac{1}{2\pi} \oint_{V_i} \dd^{-1} \omega. \]
Here, $C_i$ is transported to other fibers via the Gauss-Manin connection. We should note that the $I_i$ are not well-defined, but the quantities $I_i(b') - I_i(b)$ are well-defined.

Because the $H_k$ are defined on the base $B$, we can write $H_k(I_1, \ldots, I_m)$. Fixing $\tau_1, \ldots, \tau_m \in \R$, we can flow along
\[ H = \sum_{k=1}^m \tau_k H_k, \]
we obtain
\[ \omega^i = \pdv{H}{I_i}. \]

In ``reality,'' which is non-integrable, consider an approximation
\[ H(I,\theta) = H_0(I) + \ep H_1(I, \theta). \]
Then we can understand the approximate evolution with respect to $H$ by averaging $H_1$ over $T^m$.

\section{Gauge symmetry}
\label{sec:gauge}

We would now like to discuss the idea of gauge symmetries and gauging in the Lagrangian formalism. Recall that the phase space $(P, \omega)$ carries an action
\[ \mathbb{S}[\gamma] = \int_{\gamma} \dd^{-1} \omega - \int_I \gamma^* H \abs{\dd{t}}, \]
and we want to consider the effect of an action of a group $G$ on $P$. The moment map
\[ \mu \colon P \to \G^* \]
is equivariant with respect to the coadjoint representation. Note that $P\sslash G$ has a symplectic form $\wt{\omega}$, and locally $P$ looks like
\[ T^* G \times P \sslash G \xrightarrow{\pi} P\sslash G. \]
Thus $\omega$ restricted to a tubular neighborhood of $\mu = 0$ has the form
\[ \pi^* \wt{\omega} + (\text{tautological form on } T^*G). \]

Recall that we are looking for extrema of $\mathbb{S}$, and we need to find a path on the quotient space. We need to enlarge the space of variables to include $A \in \Omega^1_I(\g)$. Now we will define
\[ \wt{\mathbb{S}}[\gamma, A] = \mathbb{S}[\gamma] - \int_I \ev{\gamma^* \mu, A}. \]
The space of possible $A$ has an infinite-dimensional symmetry generated by
\[ \mc{G} = \ms{Maps}(I, G) \ni g(t) \]
The action is given by
\[ (g(t)) \cdot (\gamma(t), A) \coloneqq (g(t) \gamma(t), \on{Ad}_{g(t)}A + g^{-1} \dd{g}). \]
Note that $A$ transforms as a connection, not as a $1$-form.

\begin{rmk}
The translation by $g^{-1} \dd{g}$ compensates for the change of $\dd^{-1} \omega = \sum p_i \dd{q^i}$ under the action of $G$.
\end{rmk}

\subsection{Rational Calogero-Moser model}
\label{sub:ratlcms}

The first example is called the \textit{Calogero-Moser-Sutherland model}. The phase spaces are
\[ \wt{P} = T^* (\R^N \setminus \Delta \text{ or } (S^1)^N \setminus \Delta )\]
with the standard form
\[ \omega = \sum_{i=1}^N \dd{p_i} \wedge \dd{x_i}. \]
For any $\nu \in \R_+$, the Hamiltonian is given by
\[ H = \sum_{i=1}^N \frac{1}{2} p_i^2 + \nu^2 \sum_{1 \leq i < j \leq N} \qty[\frac{1}{(x_i-x_j)^2} \text{ or }\frac{1}{4\sin^2\qty(\frac{x_i-x_j}{2})} ]. \]
These are systems of $n$ particles where they start out very far away from each other, are brought closer together, and then repel themselves apart again. These turn out to be integrable systems, and in fact they can be obtained by reduction of a system on a higher-dimensional symplectic manifold.

Define the unreduced phase space
\[ P = T^* (\mf{u}(N)) \times \C^N. \]
This is a pair of $N \times N$ Hermitian matrices $(P,Q)$ with a vector $z \in \C^N$. The Liouville form will be written as
\[ \Tr \qty( \dd{P} \wedge \dd{Q} + \frac{\Tr \dd{z} \wedge \dd{z^{\dag}}}{2\sqrt{-1}}) = \sum_{i,j=1}^N \dd{P_{ij}} \wedge \dd{Q_{ji}} + \frac{1}{2\sqrt{-1}} \sum_{i=1}^N \dd{z_i} \wedge \dd{z_i^*}. \]
Then we may define the Hamiltonians
\[ H_k = \frac{1}{k} \Tr P^k. \]
The flows look like
\[ (P,Q; z) \mapsto (P, Q + \sum_k t_k P^{k-1}; z), \]
so they clearly commute. This system carries a $U(N) \times U(1)$ symmetry, where
\[ (u,c) \cdot (P,Q,z) \mapsto (u^{-1}P u, u^{-1} Q u, u^{-1} z). \]
Because this preserves the symplectic form, we may perform the symplectic reduction. Becuase $U(n)$ is not simple, there is a free parameter $\nu$, so the moment map is given by
\[ \mu(P,Q,z) = [P,Q] + \sqrt{-1} (zz^{\dag} - \nu \cdot 1_N). \]
We only need to solve $\mu = 0$ up to $U(N)$, so we choose a diagonal representative of $\qty{u^{-1}Qu}$. Thus, assume that $Q = \mr{diag}(x_1, \ldots, x_N)$ is diagonal with $x_1 \geq \cdots \geq x_N$. Generically, the inequalities are strict. Then
\[ \mu_{ij} = P_{ij}(x_j - x_i) + \sqrt{-1} (z_i z_j^* - \nu \delta_{ij}). \]
If $i=j$, then $\abs{z_i}^2 = \nu$, so the remaining $U(1)^N$-action can be used to set $z_i = z_i^* = \sqrt{\nu}$. We can now compute
\[ P_{ij} = - \frac{\sqrt{-1} \nu}{x_i-x_j} \]
for the non-diagonal elements. We cannot compute the diagonal elements of $P$, so we obtain
\[ P = \mr{diag}(p_1, \ldots, p_N) + \norm{\frac{\sqrt{-1} \nu}{x_i-x_j}(1-\delta_{ij})}_{i,j=1}^N. \]
In this form, the Hamiltonians become
\begin{align*}
  H_1 &= \sum_{i=1}^N p_i \\
  H_2 &= \frac{1}{2} \sum_{i=1}^N p_i^2 + \nu^2 \sum_{i<j} \frac{1}{(x_i-x_j)^2},
\end{align*}
which is what we wanted. We need to show that $\omega = \sum \dd{p_i} \wedge \dd{x_i}$, so we will compute the Poisson brackets of various functions. Recall that functions on $(P, \omega)$ form a Lie algebra with the \textit{Poisson bracket}
\[ \qty{f,g} = \omega^{-1} \llcorner \dd{f} \wedge \dd{g}. \]
We note that
\begin{align*}
  \mc{O}(P \sslash G) &= \mc{O}(\mu^{-1}(0)/G) \\
                      &= \mc{O}(\mu^{-1}(0))^G \\
                      &= \mc{O}(P)^G/(\mu = 0).
\end{align*}
The functions we will consider are the resolvents
\begin{align*}
  R(\lambda) &= \Tr \frac{1}{Q-\lambda} \\
  S(\lambda) &= \Tr P \frac{1}{Q-\lambda}.
\end{align*}
Because the trace is cyclic, we obtain
\begin{align*}
  \dd{R}(\lambda) &= -\Tr(Q-\lambda)^{-1} \dd{Q} (Q-\lambda)^{-1} \\
  &= -\Tr [(Q-\lambda)^2 \dd{Q}]
\end{align*}
Therefore
\begin{align*}
  \qty{R(\lambda), S(\mu)} &= \sum_{i,j} \pdv{R(\lambda)}{Q_{ij}} \pdv{S(\mu)}{P_{ji}} \\
                           &= -\Tr (Q-\lambda)^2 (Q-\lambda)^{-1} \\
                           &= -\pdv{\lambda} \qty(\frac{R(\lambda) - R(\mu)}{\lambda - \mu}).
\end{align*}
On the reduced space, the functions become
\begin{align*}
  R(\lambda) &= \sum_{i=1}^N \frac{1}{x_i-\lambda} \\
  S(\mu) &= \sum_{i=1}^N \frac{p_i}{x_i-\mu}.
\end{align*}
This is equivalent to $\qty{x_i, x_j} = 0 = \qty{p_i, p_j}$, so $\qty{p_i, x_j} = \delta_{ij}$.

Note that this system has an alternative presentation where we assume that $P = \mr{diag}(\wt{p}_1, \ldots, \wt{p}_N)$ and
\[ Q = \mr{diag}(\wt{x}_1, \ldots, \wt{x}_N) + \norm{\frac{\sqrt{-1}\nu}{\wt{p}_i - \wt{p}_j}}. \]
Then the Hamiltonians reduce to
\[ H_k = \frac{1}{k} \sum_{i=1}^N \wt{p}_i^k, \]
and the flows are given by
\[ \wt{x}_i(t) = \wt{x}_i(0) + \sum_k t_k \wt{p}_i^{k-1}. \]
This system is not very interesting, but we could instead consider
\[ H_k^{\vee} = \frac{1}{k} \Tr Q^k \]
and obtain a system with position and momentum exchanged.

\subsection{Trigonometric Calogero-Moser (Sutherland)}
\label{sub:trigcm}

Now consider $P = T^* U(N) \times \C^N$. Then we have a triple $(P, g;z)$ where $P$ is Hermitian and $g (= \exp(\sqrt{-1} Q))$ is unitary. The moment map is given by
\[ \mu(P,g,z) = \sqrt{-1} (P-g^{-1}Pg + z z^{\dag} - \nu \cdot 1_N). \]
We may choose to either diagonalize $P$ as $\mr{diag}(\wt{p}_1, \ldots, \wt{p}_N)$ or diagonalize $g$ as $\mr{diag}\qty(e^{\sqrt{-1}x_1}, \ldots, e^{\sqrt{-1}x_N})$. Making the latter choice, we obtain
\[ P = \mr{diag}(p_1, \ldots, p_N) + \norm{\frac{\nu}{e^{\sqrt{-1}(x_j-x_i)}-1}(1-\delta_{ij})}. \]
The Hamiltonians in this case are
\begin{align*}
  H_1 &= \sum_k p_k \\
  H_2 &= \frac{1}{2} \sum_{i=1}^N p_i^2 + \frac{\nu^2}{4} \sum_{i<j} \frac{1}{\sin^2 \qty(\frac{x_i-x_j}{2})}. 
\end{align*}

Making the former choice, we obtain another integrable system called the \textit{rational relativistic Calogero-Moser system} or the \textit{rational Ruijsenaars model}. In this model, the Hamiltonians look like
\[ H_k^{\vee} = \sum e^{\wt{x}_i} \times \qty(\text{rational functions of }\wt{p}_i). \]
Here, a relativistic particle in $1+1$ dimensions has energy and momentum given by
\begin{align*}
  E &= m\cosh \theta = \Tr(g+g^{-1}) \\
  p &= m \sinh \theta = \Tr(g-g^{-1}),
\end{align*}
so $E^2-p^2 = m^2$.

\section{Infinite-dimensional symmetries}
\label{sec:infinite}

We will now replace $\g = \on{\ms{Lie}}$ with $\wh{\g} = \wh{\ms{Maps}(S^1, \g)}$, which is a central extension of the space of maps with commutator given by
\[ [(f_1, c_1), (f_2, c_2)] = \qty([f_1, f_2], \int_{S^1} \Tr f_1 \dd{f_2}). \]
Then $\wh{\g}^*$ is not $\g$ but instead
\[ \wh{\g}^* = \qty{k \partial + A \mid A \in \Omega^1_{S^1}(\g), k \in \R} \]
with the pairing
\[ \ev{k \partial + A, (f,c)} = kc + \int_{S^1} \ev{A,f}. \]
We will also consider the Lie algebra $\ms{Maps}(\R, \g)$, but this requires us to specify some kind of boundary conditions at $\infty$. We may also consider $L^2(\R) \otimes \g$. In the case of $S^1$, note that
\[ c(f_1, f_2) \coloneqq \int_{S_1} \Tr f_1 \dd{f_2} \]
is a $2$-cocycle and that
\[ H^2(L\g, \R) \cong \R \]
is $1$-dimensional, so this is the only nontrivial cocycle.

The corresponding group is given by the following construction. Define
\[ LG = \ms{Maps}(S^1, G). \]
Then $\wh{LG}$ is a nontrivial $U(1)$-bundle
\[ 1 \to U(1) \to \wh{LG} \to LG \to 1. \]
Note that $H^2(LG, \R) \simeq \R$. The cohomology $H^3(G,\Z)$ is nontrivial with a nontrivial class given by
\[ \omega \coloneqq \frac{i}{8\pi^3} \Tr(g^{-1}\dd{g})^3 \leftrightsquigarrow \Tr \xi_1[\xi_2, \xi_3] \eqqcolon c(\xi_1, \xi_2, \xi_3). \]
Then there is an evaluation map
\[ e \colon LG \times S^1 \to G \qquad (g(t), u) \mapsto g(u), \]
and then
\[ \int_{S^1} e^* \omega \in H^2(LG, \Z) \]
represents $c_1(\wh{LG} \to LG)$. Therefore, we have an identification
\[ \wh{LG} = \wh{\ms{Maps}(D^2, G)} / \ms{Maps}((D^2, S^1), (G, 1)), \]
where $\wh{\ms{Maps}(D^2, G)} = \ms{Maps}(D^2, G) \times U(1)$ with multiplication
\[ (g_1, c_1) \times (g_2, c_2) = \qty(g_1g_2, c_1c_2 \exp \frac{i}{4\pi}\int_{D^2} \Tr g_1^{-1} \dd{g_1} \wedge \dd{g_2}). \]
To embed $\ms{Maps}((D^2, S^1), (G,1))$ as a normal subgroup, we make use of the fact that $\pi_2(G) = 0$, so any map $g$ can be extended to $\wt{g} \colon B^3 \to G$. Then we define
\[ \varphi(g) \coloneqq (g, \exp(2\pi i) \wt{g}^* \omega). \]
The fact that this construction is well-defined is the \textit{Polyakov-Wiegmann formula}.

We will now discuss the adjoint and coadjoint actions of $\wh{LG}$ on $\wh{g}, \wh{g}^*$ respectively. Infinitesimally, we have
\[ (\phi, 0) \cdot (\xi, c) = \qty([\phi, \xi], \int_{S^1} \Tr \phi \dd{\xi}). \]
The action on the dual space is given by
\begin{align*}
  \ev{\ad^*_{\phi}(A, k), (\xi, c)} &= \ev{(A,k), \qty([\phi, \xi], \int \Tr \phi \dd{\xi})} \\
                                    &= k\int_{S^1} \Tr \phi \dd{\xi} + \int_{S^1} \Tr A[\phi, \xi] \\
                                    &= \int_{S^1} \Tr \xi(-k \dd{\phi} + [A, \phi]).
\end{align*}
Therefore, we obtain
\[ \Ad^*_g(A, k) = (-k \dd{g} g^{-1} - g A g^{-1}, 0). \]
Note that $\frac{A}{k}$ is a $g$-connection $1$-form on $S^1$.

There is now a natural candidate for a symplectic form, which is
\[ \Omega_{T^* \wh{\g}} = \delta k \wedge \delta c + \int_{S^1} \Tr \delta A \wedge \delta \xi. \]
Here, $\delta$ is the differential in the space of fields. The moment map $\mu \colon T^* \wh{g} \to \wh{g}^*$ is given by
\[ \mu(k, c, A, \xi) = (G(k, c, A, \xi), 0), \]
where $G(k,c,A,\xi) = -k \dd{\xi} + [A, \xi]$. We will now compute
\[ \mc{P}^{\mr{red}} = \mu^{-1}(0) / LG = \qty{(\xi, A, k, c) \mid -k \dd{\xi} + [A,\xi] = 0}/(\xi, A) \mapsto (\Ad_g \xi, k \dd{g}g^{-1} + g A g^{-1}). \]

We will now solve the moment map equation with the assumption that $k \neq 0$. We will scale $k=1$, so the equation becomes
\[ \dd{\xi} + [A, \xi] = 0. \]
This is a first order matrix differential equation with periodic coefficients which can be studied using Floquet-Lyapunov theory. This says that there exists $g$ such that
\[ g^{-1} \dd{g} + g^{-1} A g \in \mf{t} \subset \g \]
is constant and lies in a maximal Cartan of $\g$. What this means is that we can write
\[ \xi(t) = G(t) \xi_0 G(t)^{-1} \qquad G(t) = P \exp \int_0^t A. \]
These satisfy the equations $\dot{G} G^{-1} = A$ and $G(0) = 1$. The monodromy is
\[ G_A \coloneqq G(2\pi) = P \exp \int_0^{2\pi} A. \]
This must commute with $\xi_0$, so we can bring
\[ A \mapsto g^{-1} \dd{g} + g^{-1}Ag \qquad G_A \mapsto g(0)^{-1} G_A g(0). \]
Recall that $G_A$ can be brought to $T \subset G$ and $A$ can be brought to $\alpha \in \mf{t}$ and $g_A = \exp(2\pi \alpha)$. There is still some remaining symmetry by
\[ g(u) = \exp(u\lambda), \]
where $\lambda \in \Lambda^{\vee}$ is in the lattice of coroots and $u \in S^1 = \R/2\pi \Z$ is the coordinate. This shifts $\alpha \mapsto \alpha + \lambda$ while preserving monodromy. The second kind of remaining symmetry is the Weyl group $W \coloneqq N(T)/T$. Taking their semidirect product, we obtain the \textit{affine Weyl group}.

If we consider the weight space decomposition of the moment map equation and $\beta$ is a root of $\g$, then the equation for this component is
\[ \dd{\xi_{\beta}} + \ev{\beta, \alpha} \xi_{\beta} = 0. \]
Because $\xi_{\beta}(u) = e^{-u\ev{\beta, \alpha}}$, this is generically not $1$, so $\xi_{\beta} = 0$. Therefore $\xi \in \mf{t}$, and we obtain
\[ T^* \wh{g} \sslash LG = (T^* T)/W. \]
Note that $T$ parameterizes conjugacy classes of $P \exp \oint A$ and that $T = \mf{t}/\Lambda^{\vee}$. Unfortunately, the reduced space is an orbifold, not a manifold.

We will now attempt to remedy this situation by modifying the quotient. Instead of setting the moment map to be $0$, we want to consider an orbit. We want $\mc{O} = \P^{N-1}$, and if $G = SU(N)$, $LG$ acts on $\P^{N-1}$ by evaluation at some $0 \in S^1$. We choose $z \in \C^N$ such that $z^{\dag} z = N$ up to $z \sim ze^{i\alpha}$, and the modified equation is
\[ \dd{\xi} + [A, \xi] = \delta(u) \cdot (i \nu(1_N - z \circ z^{\dag})). \]

\begin{rmk}
While most of the orbits are infinite dimensional, we are taking some limit where $A$ becomes a distribution on $S^1$ supported on finitely many points.
\end{rmk}

We first apply Floquet-Lyupanov to make $A = \mr{diag}(a_1, \ldots, a_N)$ diagonal with $\sum a_i = 0$. Then on each coordinate we obtain
\begin{align*}
  \dd{\xi_{ij}} + (a_i-a_j) \xi_{ij} &= \sqrt{-1} \delta(u)\nu(-z_i \ol{z}_j) \\
  \dd{\xi_{ii}} &= \sqrt{-1} \nu(1-\abs{z_i}^2) \delta(u).
\end{align*}
Because $\xi_{ii}(+0) = \xi_{ii}(2\pi-0) = \xi_{ii}(-0)$ for any $0 \in S^1$, $\abs{z_i}^2 = 1$. Using the maximal torus, we may force $z_i = 1$. Then we obtain
\begin{align*}
  \xi_{ij}(u) &= e^{-u(a_i-a_j)}\xi_{ij}(+0) \\
  \xi_{ij}(2\pi-0) &= e^{-2\pi(a_i-a_j)}\xi_{ij}(+0) = \xi_{ij}(+0) + \sqrt{-1} \nu.
\end{align*}
Finally, the initial value is
\[ \xi_{ij}(+0) = \frac{\sqrt{-1}\nu}{e^{-2\pi \sqrt{-1}(a_i-a_j)}-1}. \]
Note that this appeared in our study of the Sutherland system.

\chapter{Complexified integrable systems}
\label{cha:complexified}

\section{Quantization}
\label{sec:quantization}

The motivation for complexification of our systems is quantization. Recall that if $S$ is the action of a Lagrangian system, many systems are not described by solving the variational equation
\[ \delta S = 0 \]
but by meditating on formal path integrals (due to Feynman)
\[ \int_{P \mc{P}} e^{\frac{i S[\gamma]}{\hslash}} [\mc{D}\gamma]. \]
The classical system is obtained via stationary phase approximation. Mathematically, this is ill-defined, but in fact if $X$ is a finite-dimensional manifold, we can consider oscillatory integrals
\[ I = \int_X e^{\frac{iS}{\hslash}} \mu. \]
Here, $X$ is one of many possible cycles in the complexification $X^{\C}$ and $\mu$ is viewed as a holomorphic top-degree form, so this is just a period. If $\dim X = n$, then we may consider other $\Gamma$ such that
\[ \int_{\Gamma} e^{\frac{iS}{\hslash}} \mu \]
converges. These will satisfy
\[ \Gamma \in H_n(X^{\C}, X^{\C}_{\ll}), \]
where
\[ X^{\C}_{\ll} = \qty{z \mid \Re\qty(\frac{iS(z)}{\hslash}) \ll 0} \]
is set to force the integral to converge. These $\Gamma$ are chosen to flow from critical points of $S$ in $X^{\C}$ into $X^{\C}_{\ll}$.



\end{document} 

%%% Local Variables:
%%% mode: latex
%%% TeX-master: t
%%% End: