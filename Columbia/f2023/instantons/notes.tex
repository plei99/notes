\documentclass[leqno, openany]{memoir}
\setulmarginsandblock{3.5cm}{3.5cm}{*}
\setlrmarginsandblock{3cm}{3.5cm}{*}
\checkandfixthelayout

\usepackage{amsmath}
\usepackage{amssymb}
\usepackage{amsthm}
%\usepackage{MnSymbol}
\usepackage{bm}
\usepackage{accents}
\usepackage{mathtools}
\usepackage{tikz}
\usetikzlibrary{calc}
\usetikzlibrary{automata,positioning}
\usepackage{tikz-cd}
\usepackage{forest}
\usepackage{braket} 
\usepackage{listings}
\usepackage{mdframed}
\usepackage{verbatim}
\usepackage{physics}
\usepackage{stmaryrd}
\usepackage{mathrsfs} 
\usepackage[normalem]{ulem} 
\usepackage{stackengine}
\usepackage{bbm}
\usepackage{cancel}
%\usepackage{/home/patrickl/homework/macaulay2}

%font
\usepackage[sc]{mathpazo}
\usepackage{eulervm}
\usepackage[scaled=0.86]{berasans}
\usepackage{inconsolata}
\usepackage{microtype}

%CS packages
\usepackage{algorithmicx}
\usepackage{algpseudocode}
\usepackage{algorithm}

% typeset and bib
\usepackage[english]{babel} 
\usepackage[utf8]{inputenc} 
\usepackage[T1]{fontenc}
\usepackage[bookmarks, colorlinks, breaklinks]{hyperref} 
\hypersetup{linkcolor=blue,citecolor=magenta,filecolor=black,urlcolor=blue}
\usepackage{cleveref}
% \usepackage[backend=biber,style=alphabetic,maxalphanames=4,maxnames=5,hyperref,backref=true,backrefstyle=none]{biblatex}
\usepackage{xpatch}
% \xpatchbibmacro{pageref}{parens}{backrefparens}{}{}
\crefname{equation}{}{}

% other formatting packages
\usepackage{float}
\usepackage{booktabs}
\usepackage[shortlabels]{enumitem}
\usepackage{csquotes}
\usepackage{titlesec}
\usepackage{titling}
\usepackage{parskip}
\usepackage{graphicx}
\graphicspath{{./}}

\usepackage{lipsum}

% delimiters
\DeclarePairedDelimiter{\gen}{\langle}{\rangle}
\DeclarePairedDelimiter{\floor}{\lfloor}{\rfloor}
\DeclarePairedDelimiter{\ceil}{\lceil}{\rceil}


\newtheorem{thm}{Theorem}[section]
\newtheorem{cor}[thm]{Corollary}
\newtheorem{prop}[thm]{Proposition}
\newtheorem{lem}[thm]{Lemma}
\newtheorem{conj}[thm]{Conjecture}
\newtheorem{quest}[thm]{Question}
\newtheorem{prob}[thm]{Problem}
\newtheorem{clm}[thm]{Claim}

\theoremstyle{definition}
\newtheorem{defn}[thm]{Definition}
\newtheorem{defns}[thm]{Definitions}
\newtheorem{con}[thm]{Construction}
\newtheorem{exm}[thm]{Example}
\newtheorem{exms}[thm]{Examples}
\newtheorem{notn}[thm]{Notation}
\newtheorem{notns}[thm]{Notations}
\newtheorem{addm}[thm]{Addendum}
\newtheorem{exer}[thm]{Exercise}

\theoremstyle{remark}
\newtheorem{rmk}[thm]{Remark}
\newtheorem{rmks}[thm]{Remarks}
\newtheorem{warn}[thm]{Warning}
\newtheorem{sch}[thm]{Scholium}


% unnumbered theorems
\theoremstyle{plain}
\newtheorem*{thm*}{Theorem}
\newtheorem*{prop*}{Proposition}
\newtheorem*{lem*}{Lemma}
\newtheorem*{cor*}{Corollary}
\newtheorem*{conj*}{Conjecture}

% unnumbered definitions
\theoremstyle{definition}
\newtheorem*{defn*}{Definition}
\newtheorem*{exer*}{Exercise}
\newtheorem*{defns*}{Definitions}
\newtheorem*{con*}{Construction}
\newtheorem*{exm*}{Example}
\newtheorem*{exms*}{Examples}
\newtheorem*{notn*}{Notation}
\newtheorem*{notns*}{Notations}
\newtheorem*{addm*}{Addendum}


\theoremstyle{remark}
\newtheorem*{rmk*}{Remark}

% shortcuts
\newcommand{\Ima}{\mathrm{Im}}
\newcommand{\A}{\mathbb{A}}
\newcommand{\F}{\mathbb{F}}
\newcommand{\E}{\mathbb{E}}
\newcommand{\G}{\mathbb{G}}
\newcommand{\N}{\mathbb{N}}
\newcommand{\R}{\mathbb{R}}
\newcommand{\C}{\mathbb{C}}
\newcommand{\Z}{\mathbb{Z}}
\newcommand{\Q}{\mathbb{Q}}
\newcommand{\K}{\mathbb{K}}
\renewcommand{\k}{\Bbbk}
\renewcommand{\L}{\mathbb{L}}
\renewcommand{\P}{\mathbb{P}}
\newcommand{\M}{\overline{M}}
\newcommand{\g}{\mathfrak{g}}
\newcommand{\h}{\mathfrak{h}}
\newcommand{\n}{\mathfrak{n}}
\renewcommand{\b}{\mathfrak{b}}
\newcommand{\ep}{\varepsilon}
\newcommand*{\dt}[1]{%
   \accentset{\mbox{\Huge\bfseries .}}{#1}}
\renewcommand{\abstractname}{Official Description}
\newcommand{\mc}[1]{\mathcal{#1}}
\newcommand{\T}{\mathbb{T}}
\newcommand{\mf}[1]{\mathfrak{#1}}
\newcommand{\mr}[1]{\mathrm{#1}}
\newcommand{\ms}[1]{\mathsf{#1}}
\newcommand{\mt}[1]{\mathtt{#1}}
\newcommand{\on}[1]{\operatorname{#1}}
\newcommand{\ol}[1]{\overline{#1}}
\newcommand{\ul}[1]{\underline{#1}}
\newcommand{\wt}[1]{\widetilde{#1}}
\newcommand{\wh}[1]{\widehat{#1}}
\renewcommand{\div}{\operatorname{div}}
\newcommand{\bir}{\sim_{\mr{bir}}}
\newcommand{\GL}{\mr{GL}}
\newcommand{\PGL}{\mr{PGL}}
\newcommand{\stacks}[1]{\href{https://stacks.math.columbia.edu/tag/#1}{#1}}
\newcommand{\ostar}{\stackMath\mathbin{\stackinset{c}{0ex}{c}{0ex}{\star}{\bigcirc}}}

\DeclareMathOperator{\Der}{Der}
\DeclareMathOperator{\Def}{Def}
\DeclareMathOperator{\Bl}{Bl}
\DeclareMathOperator{\NE}{NE}
\DeclareMathOperator{\Tor}{Tor}
\DeclareMathOperator{\Hom}{Hom}
\DeclareMathOperator{\Ext}{Ext}
\DeclareMathOperator{\End}{End}
\DeclareMathOperator{\ad}{ad}
\DeclareMathOperator{\Ad}{Ad}
\DeclareMathOperator{\Aut}{Aut}
\DeclareMathOperator{\Rad}{Rad}
\DeclareMathOperator{\Pic}{Pic}
\DeclareMathOperator{\supp}{supp}
\DeclareMathOperator{\Supp}{Supp}
\DeclareMathOperator{\sgn}{sgn}
\DeclareMathOperator{\spec}{Spec}
\DeclareMathOperator{\Spec}{Spec}
\DeclareMathOperator{\proj}{Proj}
\DeclareMathOperator{\Proj}{Proj}
\DeclareMathOperator{\ord}{ord}
\DeclareMathOperator{\Div}{Div}
\DeclareMathOperator{\depth}{depth}
\DeclareMathOperator{\coker}{coker}
\DeclareMathOperator{\codim}{codim}
\DeclareMathOperator{\ch}{ch}
\DeclareMathOperator{\Hilb}{Hilb}
\DeclareMathOperator{\Lie}{\ms{Lie}}

% Section formatting
\titleformat{\section}
    {\Large\sffamily\scshape\bfseries}{\thesection}{1em}{}
\titleformat{\subsection}[runin]
    {\large\sffamily\bfseries}{\thesubsection}{1em}{}
\titleformat{\subsubsection}[runin]{\normalfont\itshape}{\thesubsubsection}{1em}{}

\title{COURSE TITLE}
\author{Lectures by INSTRUCTOR, Notes by NOTETAKER}
\date{SEMESTER}

\newcommand*{\titleSW}
    {\begingroup% Story of Writing
    \raggedleft
    \vspace*{\baselineskip}
    {\Huge\itshape The Count of Instantons \\ 2023-24}\\[\baselineskip]
    {\large\itshape Notes by Davis Lazowski and Patrick Lei}\\[0.2\textheight]
    {\Large Lectures by Nikita Nekrasov}\par
    \vfill
    {\Large \sffamily Columbia University}
    \vspace*{\baselineskip}
\endgroup}
\pagestyle{simple}

\chapterstyle{ell}


%\renewcommand{\cftchapterpagefont}{}
\renewcommand\cftchapterfont{\sffamily}
\renewcommand\cftsectionfont{\scshape}
\renewcommand*{\cftchapterleader}{}
\renewcommand*{\cftsectionleader}{}
\renewcommand*{\cftsubsectionleader}{}
\renewcommand*{\cftchapterformatpnum}[1]{~\textbullet~#1}
\renewcommand*{\cftsectionformatpnum}[1]{~\textbullet~#1}
\renewcommand*{\cftsubsectionformatpnum}[1]{~\textbullet~#1}
\renewcommand{\cftchapterafterpnum}{\cftparfillskip}
\renewcommand{\cftsectionafterpnum}{\cftparfillskip}
\renewcommand{\cftsubsectionafterpnum}{\cftparfillskip}
\setrmarg{3.55em plus 1fil}
\setsecnumdepth{subsection}
\maxsecnumdepth{subsection}
\settocdepth{subsection}

% \addbibresource{../../math.bib}
% \DefineBibliographyStrings{english}{
%     backrefpage={$\leftarrow$},
%     backrefpages={$\leftarrow$},
% }

\begin{document}
    
\begin{titlingpage}
\titleSW
\end{titlingpage}

\thispagestyle{empty}
\section*{Disclaimer}%
\label{sec:disclaimer}

These notes were taken during the lectures using \texttt{neovim} (Patrick) and \texttt{VSCode} (Davis). 
Any errors are mine and not the speakers'. 
In addition, my notes are picture-free (but will include commutative diagrams) and are a mix of my mathematical style and that of the lecturers. Also, notation may differ between lectures.
If you find any errors, please contact me at \texttt{plei@math.columbia.edu}.

Unfortunately, I cannot guarantee that the notation in these notes is consistent. Please do not judge this too harshly.

\section*{Description}

Graduate level introduction to modern mathematical physics with the emphasis on the geometry and physics of quantum gauge theory and its connections to string theory.  We shall zoom in on a corner of the theory especially suitable for exploring non-perturbative aspects of gauge and string theory: the instanton contributions. Using a combination of methods from algebraic geometry, topology, representation theory and probability theory we shall derive a series of identities obeyed by generating functions of integrals over instanton moduli spaces, and discuss their symplectic, quantum, isomonodromic, and, more generally, representation-theoretic significance.

Quantum and classical integrable systems, both finite and infinite-dimensional ones, will be a recurring cast of characters, along with the other ($qq$-) characters.

\section*{Acknowledgements}
Notes were taken by Patrick Lei for all lectures until March 22, 2024 except for the lecture on December 8, 2023. Davis Lazowski provided notes for the lecture on December 8, 2023 and for all lectures from March 29, 2024 onwards. Slight edits (in form, not in content) were made to the notes taken by Davis to make them fit with Patrick's stylistic conventions.


\newpage

\tableofcontents

\chapter{Classical mechanics}
\label{cha:intro}

We will be discussing three types of physics in an attempt to create something mathematically interesting:
\begin{itemize}
\item Classical physics;
\item Statistical physics;
\item Quantum physics;
\end{itemize}

\section{Classical physics}
\label{sec:classical}

\subsection{Hamiltonian dynamics}
\label{subsec:hamiltonian}

We will begin with a space of classical states, which is most commonly known as a \textit{phase space}. This is a symplectic manifold $(M, \omega)$, where $\dim M = 2m$ and $\omega \in \Omega^2(M)$ satisfies
\begin{align*}
  \dd{\omega} &= 0 \\
  \omega \wedge \cdots \wedge \omega &\neq 0.
\end{align*}
This carries a function
\[ H \colon M \to \R, \]
called a \textit{Hamiltonian}. Then there is a vector field $V_H$ described by
\[ \dd{H} = \iota_{V_H} \omega, \]
which generates a $1$-parameter group $g^t$ of symplectomorphisms of $M$. The evolution law of the physical system is given by
\[ \dot{x} = V_H(x). \]
Because $g^t$ acts by symplectomorphisms, the graph
\[ \Gamma_{g^t} = \qty{(m, g^t(m))} \subset M \times M \]
is a Lagrangian submanifold. Recall that a submanifold $L \subset M$ of a symplectic manifold is called \textit{Lagrangian} if $\dim L = m$ and $\omega |_L = 0$.

\begin{exer}
  Locally any symplectic manifold is given by $M = \R^{2m}$ with the symplectic form
  \[ \omega = \sum_{i=1}^m \dd{p_i} \wedge \dd{q^i}. \]
  This coordinate system on $\R^{2m}$ is unique up to $\mr{Sp}(2m)$.
\end{exer}

Now if $V$ is a vector field such that $\on{\ms{Lie}}_V \omega = 0$, then $\dd{\iota_V \omega} = 0$. Thus there locally exists a Hamiltonian $h_V$. We also need the Hamiltonian vector field to be linear in the local coordinates, so the Hamiltonian itself must be quadratic.

\begin{exm}
  An important example of a symplectic manifold is $T^* B$ for any smooth manifold $B$. There is a $1$-form $\theta$ on $T^* B$ given by the following formula. If $v$ is a tangent vector at the point $(p,b)$, then
  \[ \theta(v) \coloneqq p(\pi_* v). \]
  Then $\omega = \dd{\theta}$ is a symplectic form.
\end{exm}

\begin{exm}
  Another large class of examples are obtained by \textit{symplectic reduction}. Here, we suppose that a symplectic manifold $M$ carries the action of a compact Lie group $G$ by exact symplectomorphisms. This defines a \textit{moment map} $M \xrightarrow{\mu} \g^*$ by the formula
  \[ \ev{\mu(m), \xi} = h_{V_{\xi}}(m). \]
  There is some ambiguity in the choice of constants, but in the end we obtain a new space
  \[ M \sslash G \coloneqq \mu^{-1}(0) / G. \]
  In practice, we want the moment map $\mu$ to be equivariant with respect to the coadjoint action on $\g^*$. Then the manifold $M \sslash G$ has a canonical symplectic form, but this requires a lot of work.

  Now consider $M = \R^{2m}$ and $G = U(1)$, where we write $M = (\R^2)^m$ and $U(1)$ acts by rotations. Then the moment map is actually
  \[ \mu = \sum_{i=1}^m \frac{1}{2} (p_i^2 + q_i^2) - r, \]
  so $\mu^{-1}(0)$ is a sphere. We then obtain
  \[ \R^{2m} \sslash U(1) = S^{2m-1}/U(1) = \P^{m-1}. \]
  We made no use of the complex numbers, so the fact that we obtain a complex manifold will be viewed as a bonus. The reduced symplectic form is simply $r$ times the Fubini-Study form.

  Note that $\P^{m-1}$ is compact, so the interpretation that phase space records position and momentum breaks down. In this case, our phase space is called the \textit{classical spin phase space}, where the motion is by rotations rather than by translation.
\end{exm}

\begin{rmk}
  We will often consider time-dependent Hamiltonians, where $\dot{x} = V_{H(t)}(x)$.
\end{rmk}

\subsection{Lagrangian mechanics}
\label{subsec:lagrangian}

There is another point of view, where dynamics on $M$ are given by an optimization problem on the space
\[ \mc{P}M \coloneqq \ms{Map}([0,1], M) \]
of paths in $M$. We will consider an \textit{action}
\[ \mathbb{S}[\gamma] \coloneqq \int_{\gamma} \theta - \int_0^1 \gamma(t)^* H(t) \dd{t}, \]
where we assume that $\omega = \dd{\theta}$. If we further require that $\gamma(0) \in L_0$ and $\gamma(1) \in L_1$, our dynamics are well-defined if $L_0, L_1$ are Lagrangian submanifolds. Of course, we are looking for paths were $\delta \mathbb{S} = 0$.

\subsection{Classical field theory}
\label{subsec:classicalft}

Classical field theory should be thought of as an infinite-dimensional version of classical mechanics, where we study loopified versions of finite-dimensinoal manifolds. We want to consider integrals
\[ \mathbb{S} \coloneqq \int_{(\Sigma, h)} \mc{L}[\phi, \partial \phi] \mr{vol}_h, \]
where $\Sigma$ is the spacetime, $h$ is a metric, and $\phi$ are the \textit{fields}. Fields could be one of several options:
\begin{itemize}
\item Scalars $f \colon \Sigma \to X$, where $X$ is a Riemannian manifold;
\item Connections $\nabla$ on principal bundles
  \begin{equation*}
    \begin{tikzcd}
      G \ar{r} & P \ar{d} \\
      & \Sigma,
    \end{tikzcd}
  \end{equation*}
  called \textit{gauge fields};
  \item The metric $h$ itself, called \textit{gravity}.
\end{itemize}

We can introduce more complexity into the problem by varying the action and looking for solutions of PDEs, introducing boundary to $\Sigma$, and other operations. Also note that Lagrangian mechanics can be interpreted as a $1$-dimensional classical field theory.

\section{Statistical physics}
\label{sec:stat}

In statistical physics, a point is replaced by a cloud of points, or a probability measure. For example, the measure could contain the term $e^{-\beta H}$, where $H$ was the classical Hamiltonian and $\beta$ is a parameter of our distribution (inverse temperature). The system often flows to a stationary distribution, which is determined by the outside world. In reality, the distribution will have the form
\[ \frac{1}{Z} e^{-\beta H} \qquad Z = \int_M e^{-\beta H} \frac{\omega \wedge \cdots \wedge \omega}{m!}. \]
This factor $Z$ is called the \textit{partition function}, and most of our energy is spent on computing this partition function.

\section{Classical physics with multiple times}
\label{sec:times}

Suppose we have Hamiltonians $H_1, \ldots, H_k$ such that $[V_{H_i}, \ldots, V_{H_j}] = 0$. Then we obtain dynamics
\[ \ul{\gamma} \colon [0, \ep]^p \to M \]
and an action
\[ \mathbb{S} = \int \theta - \sum_{k=1}^p \ul{\gamma}^* H_k \dd{t_k} \]
defined on paths inside the cube $[0,\ep]^p$. The extreme case of this is an \textit{integrable system} where the $H_i$ are functionally independent, and the maximum possible value of $p$ is $m$.

\begin{thm}[Liouville-Arnold]
  If the motion is finite (fits in a compact set), then locally $M$ is a fibration
  \begin{equation*}
    \begin{tikzcd}
      T^m \ar{r} & M \ar{d} \\
      & B
    \end{tikzcd}
  \end{equation*}
  such that the $H_i$ factor through $B$ and the $V_{H_k}$ span the rotations on each $S^1$ factor of $T^m$.
\end{thm}

A typical trajectory is a winding of the torus, where if $\theta_i$ are the angle coordinates on $T^m$, there is the formula
\[ \theta^i(t) = \theta^i(0) + \omega^i t. \]
Generically, these paths will have dense image.

In the special case of an integrable system, there are \textit{action-angle variables}, where the symplectic form is
\[ \omega = \sum_{i=1}^m \dd{I_i} \wedge \dd{\theta^i}. \]
The $\theta^i$ are defined up to $SL(m, \Z)$ affine transformations. If $C_i \in H_1(T^m, \Z)$ form a basis, then the $I_i$ are defined by
\[ I_i = \frac{1}{2\pi} \oint_{V_i} \dd^{-1} \omega. \]
Here, $C_i$ is transported to other fibers via the Gauss-Manin connection. We should note that the $I_i$ are not well-defined, but the quantities $I_i(b') - I_i(b)$ are well-defined.

Because the $H_k$ are defined on the base $B$, we can write $H_k(I_1, \ldots, I_m)$. Fixing $\tau_1, \ldots, \tau_m \in \R$, we can flow along
\[ H = \sum_{k=1}^m \tau_k H_k, \]
we obtain
\[ \omega^i = \pdv{H}{I_i}. \]

In ``reality,'' which is non-integrable, consider an approximation
\[ H(I,\theta) = H_0(I) + \ep H_1(I, \theta). \]
Then we can understand the approximate evolution with respect to $H$ by averaging $H_1$ over $T^m$.

\section{Gauge symmetry}
\label{sec:gauge}

We would now like to discuss the idea of gauge symmetries and gauging in the Lagrangian formalism. Recall that the phase space $(P, \omega)$ carries an action
\[ \mathbb{S}[\gamma] = \int_{\gamma} \dd^{-1} \omega - \int_I \gamma^* H \abs{\dd{t}}, \]
and we want to consider the effect of an action of a group $G$ on $P$. The moment map
\[ \mu \colon P \to \G^* \]
is equivariant with respect to the coadjoint representation. Note that $P\sslash G$ has a symplectic form $\wt{\omega}$, and locally $P$ looks like
\[ T^* G \times P \sslash G \xrightarrow{\pi} P\sslash G. \]
Thus $\omega$ restricted to a tubular neighborhood of $\mu = 0$ has the form
\[ \pi^* \wt{\omega} + (\text{tautological form on } T^*G). \]

Recall that we are looking for extrema of $\mathbb{S}$, and we need to find a path on the quotient space. We need to enlarge the space of variables to include $A \in \Omega^1_I(\g)$. Now we will define
\[ \wt{\mathbb{S}}[\gamma, A] = \mathbb{S}[\gamma] - \int_I \ev{\gamma^* \mu, A}. \]
The space of possible $A$ has an infinite-dimensional symmetry generated by
\[ \mc{G} = \ms{Maps}(I, G) \ni g(t) \]
The action is given by
\[ (g(t)) \cdot (\gamma(t), A) \coloneqq (g(t) \gamma(t), \on{Ad}_{g(t)}A + g^{-1} \dd{g}). \]
Note that $A$ transforms as a connection, not as a $1$-form.

\begin{rmk}
The translation by $g^{-1} \dd{g}$ compensates for the change of $\dd^{-1} \omega = \sum p_i \dd{q^i}$ under the action of $G$.
\end{rmk}

\subsection{Rational Calogero-Moser model}
\label{sub:ratlcms}

The first example is called the \textit{Calogero-Moser-Sutherland model}. The phase spaces are
\[ \wt{P} = T^* (\R^N \setminus \Delta \text{ or } (S^1)^N \setminus \Delta )\]
with the standard form
\[ \omega = \sum_{i=1}^N \dd{p_i} \wedge \dd{x_i}. \]
For any $\nu \in \R_+$, the Hamiltonian is given by
\[ H = \sum_{i=1}^N \frac{1}{2} p_i^2 + \nu^2 \sum_{1 \leq i < j \leq N} \qty[\frac{1}{(x_i-x_j)^2} \text{ or }\frac{1}{4\sin^2\qty(\frac{x_i-x_j}{2})} ]. \]
These are systems of $n$ particles where they start out very far away from each other, are brought closer together, and then repel themselves apart again. These turn out to be integrable systems, and in fact they can be obtained by reduction of a system on a higher-dimensional symplectic manifold.

Define the unreduced phase space
\[ P = T^* (\mf{u}(N)) \times \C^N. \]
This is a pair of $N \times N$ Hermitian matrices $(P,Q)$ with a vector $z \in \C^N$. The Liouville form will be written as
\[ \Tr \qty( \dd{P} \wedge \dd{Q} + \frac{\Tr \dd{z} \wedge \dd{z^{\dag}}}{2\sqrt{-1}}) = \sum_{i,j=1}^N \dd{P_{ij}} \wedge \dd{Q_{ji}} + \frac{1}{2\sqrt{-1}} \sum_{i=1}^N \dd{z_i} \wedge \dd{z_i^*}. \]
Then we may define the Hamiltonians
\[ H_k = \frac{1}{k} \Tr P^k. \]
The flows look like
\[ (P,Q; z) \mapsto (P, Q + \sum_k t_k P^{k-1}; z), \]
so they clearly commute. This system carries a $U(N) \times U(1)$ symmetry, where
\[ (u,c) \cdot (P,Q,z) \mapsto (u^{-1}P u, u^{-1} Q u, u^{-1} z). \]
Because this preserves the symplectic form, we may perform the symplectic reduction. Becuase $U(n)$ is not simple, there is a free parameter $\nu$, so the moment map is given by
\[ \mu(P,Q,z) = [P,Q] + \sqrt{-1} (zz^{\dag} - \nu \cdot 1_N). \]
We only need to solve $\mu = 0$ up to $U(N)$, so we choose a diagonal representative of $\qty{u^{-1}Qu}$. Thus, assume that $Q = \mr{diag}(x_1, \ldots, x_N)$ is diagonal with $x_1 \geq \cdots \geq x_N$. Generically, the inequalities are strict. Then
\[ \mu_{ij} = P_{ij}(x_j - x_i) + \sqrt{-1} (z_i z_j^* - \nu \delta_{ij}). \]
If $i=j$, then $\abs{z_i}^2 = \nu$, so the remaining $U(1)^N$-action can be used to set $z_i = z_i^* = \sqrt{\nu}$. We can now compute
\[ P_{ij} = - \frac{\sqrt{-1} \nu}{x_i-x_j} \]
for the non-diagonal elements. We cannot compute the diagonal elements of $P$, so we obtain
\[ P = \mr{diag}(p_1, \ldots, p_N) + \norm{\frac{\sqrt{-1} \nu}{x_i-x_j}(1-\delta_{ij})}_{i,j=1}^N. \]
In this form, the Hamiltonians become
\begin{align*}
  H_1 &= \sum_{i=1}^N p_i \\
  H_2 &= \frac{1}{2} \sum_{i=1}^N p_i^2 + \nu^2 \sum_{i<j} \frac{1}{(x_i-x_j)^2},
\end{align*}
which is what we wanted. We need to show that $\omega = \sum \dd{p_i} \wedge \dd{x_i}$, so we will compute the Poisson brackets of various functions. Recall that functions on $(P, \omega)$ form a Lie algebra with the \textit{Poisson bracket}
\[ \qty{f,g} = \omega^{-1} \llcorner \dd{f} \wedge \dd{g}. \]
We note that
\begin{align*}
  \mc{O}(P \sslash G) &= \mc{O}(\mu^{-1}(0)/G) \\
                      &= \mc{O}(\mu^{-1}(0))^G \\
                      &= \mc{O}(P)^G/(\mu = 0).
\end{align*}
The functions we will consider are the resolvents
\begin{align*}
  R(\lambda) &= \Tr \frac{1}{Q-\lambda} \\
  S(\lambda) &= \Tr P \frac{1}{Q-\lambda}.
\end{align*}
Because the trace is cyclic, we obtain
\begin{align*}
  \dd{R}(\lambda) &= -\Tr(Q-\lambda)^{-1} \dd{Q} (Q-\lambda)^{-1} \\
  &= -\Tr [(Q-\lambda)^2 \dd{Q}]
\end{align*}
Therefore
\begin{align*}
  \qty{R(\lambda), S(\mu)} &= \sum_{i,j} \pdv{R(\lambda)}{Q_{ij}} \pdv{S(\mu)}{P_{ji}} \\
                           &= -\Tr (Q-\lambda)^2 (Q-\lambda)^{-1} \\
                           &= -\pdv{\lambda} \qty(\frac{R(\lambda) - R(\mu)}{\lambda - \mu}).
\end{align*}
On the reduced space, the functions become
\begin{align*}
  R(\lambda) &= \sum_{i=1}^N \frac{1}{x_i-\lambda} \\
  S(\mu) &= \sum_{i=1}^N \frac{p_i}{x_i-\mu}.
\end{align*}
This is equivalent to $\qty{x_i, x_j} = 0 = \qty{p_i, p_j}$, so $\qty{p_i, x_j} = \delta_{ij}$.

Note that this system has an alternative presentation where we assume that $P = \mr{diag}(\wt{p}_1, \ldots, \wt{p}_N)$ and
\[ Q = \mr{diag}(\wt{x}_1, \ldots, \wt{x}_N) + \norm{\frac{\sqrt{-1}\nu}{\wt{p}_i - \wt{p}_j}}. \]
Then the Hamiltonians reduce to
\[ H_k = \frac{1}{k} \sum_{i=1}^N \wt{p}_i^k, \]
and the flows are given by
\[ \wt{x}_i(t) = \wt{x}_i(0) + \sum_k t_k \wt{p}_i^{k-1}. \]
This system is not very interesting, but we could instead consider
\[ H_k^{\vee} = \frac{1}{k} \Tr Q^k \]
and obtain a system with position and momentum exchanged.

\subsection{Trigonometric Calogero-Moser (Sutherland)}
\label{sub:trigcm}

Now consider $P = T^* U(N) \times \C^N$. Then we have a triple $(P, g;z)$ where $P$ is Hermitian and $g (= \exp(\sqrt{-1} Q))$ is unitary. The moment map is given by
\[ \mu(P,g,z) = \sqrt{-1} (P-g^{-1}Pg + z z^{\dag} - \nu \cdot 1_N). \]
We may choose to either diagonalize $P$ as $\mr{diag}(\wt{p}_1, \ldots, \wt{p}_N)$ or diagonalize $g$ as $\mr{diag}\qty(e^{\sqrt{-1}x_1}, \ldots, e^{\sqrt{-1}x_N})$. Making the latter choice, we obtain
\[ P = \mr{diag}(p_1, \ldots, p_N) + \norm{\frac{\nu}{e^{\sqrt{-1}(x_j-x_i)}-1}(1-\delta_{ij})}. \]
The Hamiltonians in this case are
\begin{align*}
  H_1 &= \sum_k p_k \\
  H_2 &= \frac{1}{2} \sum_{i=1}^N p_i^2 + \frac{\nu^2}{4} \sum_{i<j} \frac{1}{\sin^2 \qty(\frac{x_i-x_j}{2})}. 
\end{align*}

Making the former choice, we obtain another integrable system called the \textit{rational relativistic Calogero-Moser system} or the \textit{rational Ruijsenaars model}. In this model, the Hamiltonians look like
\[ H_k^{\vee} = \sum e^{\wt{x}_i} \times \qty(\text{rational functions of }\wt{p}_i). \]
Here, a relativistic particle in $1+1$ dimensions has energy and momentum given by
\begin{align*}
  E &= m\cosh \theta = \Tr(g+g^{-1}) \\
  p &= m \sinh \theta = \Tr(g-g^{-1}),
\end{align*}
so $E^2-p^2 = m^2$.

\section{Infinite-dimensional symmetries}
\label{sec:infinite}

We will now replace $\g = \on{\ms{Lie}}$ with $\wh{\g} = \wh{\ms{Maps}(S^1, \g)}$, which is a central extension of the space of maps with commutator given by
\[ [(f_1, c_1), (f_2, c_2)] = \qty([f_1, f_2], \int_{S^1} \Tr f_1 \dd{f_2}). \]
Then $\wh{\g}^*$ is not $\g$ but instead
\[ \wh{\g}^* = \qty{k \partial + A \mid A \in \Omega^1_{S^1}(\g), k \in \R} \]
with the pairing
\[ \ev{k \partial + A, (f,c)} = kc + \int_{S^1} \ev{A,f}. \]
We will also consider the Lie algebra $\ms{Maps}(\R, \g)$, but this requires us to specify some kind of boundary conditions at $\infty$. We may also consider $L^2(\R) \otimes \g$. In the case of $S^1$, note that
\[ c(f_1, f_2) \coloneqq \int_{S_1} \Tr f_1 \dd{f_2} \]
is a $2$-cocycle and that
\[ H^2(L\g, \R) \cong \R \]
is $1$-dimensional, so this is the only nontrivial cocycle.

The corresponding group is given by the following construction. Define
\[ LG = \ms{Maps}(S^1, G). \]
Then $\wh{LG}$ is a nontrivial $U(1)$-bundle
\[ 1 \to U(1) \to \wh{LG} \to LG \to 1. \]
Note that $H^2(LG, \R) \simeq \R$. The cohomology $H^3(G,\Z)$ is nontrivial with a nontrivial class given by
\[ \omega \coloneqq \frac{i}{8\pi^3} \Tr(g^{-1}\dd{g})^3 \leftrightsquigarrow \Tr \xi_1[\xi_2, \xi_3] \eqqcolon c(\xi_1, \xi_2, \xi_3). \]
Then there is an evaluation map
\[ e \colon LG \times S^1 \to G \qquad (g(t), u) \mapsto g(u), \]
and then
\[ \int_{S^1} e^* \omega \in H^2(LG, \Z) \]
represents $c_1(\wh{LG} \to LG)$. Therefore, we have an identification
\[ \wh{LG} = \wh{\ms{Maps}(D^2, G)} / \ms{Maps}((D^2, S^1), (G, 1)), \]
where $\wh{\ms{Maps}(D^2, G)} = \ms{Maps}(D^2, G) \times U(1)$ with multiplication
\[ (g_1, c_1) \times (g_2, c_2) = \qty(g_1g_2, c_1c_2 \exp \frac{i}{4\pi}\int_{D^2} \Tr g_1^{-1} \dd{g_1} \wedge \dd{g_2}). \]
To embed $\ms{Maps}((D^2, S^1), (G,1))$ as a normal subgroup, we make use of the fact that $\pi_2(G) = 0$, so any map $g$ can be extended to $\wt{g} \colon B^3 \to G$. Then we define
\[ \varphi(g) \coloneqq (g, \exp(2\pi i) \wt{g}^* \omega). \]
The fact that this construction is well-defined is the \textit{Polyakov-Wiegmann formula}.

We will now discuss the adjoint and coadjoint actions of $\wh{LG}$ on $\wh{g}, \wh{g}^*$ respectively. Infinitesimally, we have
\[ (\phi, 0) \cdot (\xi, c) = \qty([\phi, \xi], \int_{S^1} \Tr \phi \dd{\xi}). \]
The action on the dual space is given by
\begin{align*}
  \ev{\ad^*_{\phi}(A, k), (\xi, c)} &= \ev{(A,k), \qty([\phi, \xi], \int \Tr \phi \dd{\xi})} \\
                                    &= k\int_{S^1} \Tr \phi \dd{\xi} + \int_{S^1} \Tr A[\phi, \xi] \\
                                    &= \int_{S^1} \Tr \xi(-k \dd{\phi} + [A, \phi]).
\end{align*}
Therefore, we obtain
\[ \Ad^*_g(A, k) = (-k \dd{g} g^{-1} - g A g^{-1}, 0). \]
Note that $\frac{A}{k}$ is a $g$-connection $1$-form on $S^1$.

There is now a natural candidate for a symplectic form, which is
\[ \Omega_{T^* \wh{\g}} = \delta k \wedge \delta c + \int_{S^1} \Tr \delta A \wedge \delta \xi. \]
Here, $\delta$ is the differential in the space of fields. The moment map $\mu \colon T^* \wh{g} \to \wh{g}^*$ is given by
\[ \mu(k, c, A, \xi) = (G(k, c, A, \xi), 0), \]
where $G(k,c,A,\xi) = -k \dd{\xi} + [A, \xi]$. We will now compute
\[ \mc{P}^{\mr{red}} = \mu^{-1}(0) / LG = \qty{(\xi, A, k, c) \mid -k \dd{\xi} + [A,\xi] = 0}/(\xi, A) \mapsto (\Ad_g \xi, k \dd{g}g^{-1} + g A g^{-1}). \]

We will now solve the moment map equation with the assumption that $k \neq 0$. We will scale $k=1$, so the equation becomes
\[ \dd{\xi} + [A, \xi] = 0. \]
This is a first order matrix differential equation with periodic coefficients which can be studied using Floquet-Lyapunov theory. This says that there exists $g$ such that
\[ g^{-1} \dd{g} + g^{-1} A g \in \mf{t} \subset \g \]
is constant and lies in a maximal Cartan of $\g$. What this means is that we can write
\[ \xi(t) = G(t) \xi_0 G(t)^{-1} \qquad G(t) = P \exp \int_0^t A. \]
These satisfy the equations $\dot{G} G^{-1} = A$ and $G(0) = 1$. The monodromy is
\[ G_A \coloneqq G(2\pi) = P \exp \int_0^{2\pi} A. \]
This must commute with $\xi_0$, so we can bring
\[ A \mapsto g^{-1} \dd{g} + g^{-1}Ag \qquad G_A \mapsto g(0)^{-1} G_A g(0). \]
Recall that $G_A$ can be brought to $T \subset G$ and $A$ can be brought to $\alpha \in \mf{t}$ and $g_A = \exp(2\pi \alpha)$. There is still some remaining symmetry by
\[ g(u) = \exp(u\lambda), \]
where $\lambda \in \Lambda^{\vee}$ is in the lattice of coroots and $u \in S^1 = \R/2\pi \Z$ is the coordinate. This shifts $\alpha \mapsto \alpha + \lambda$ while preserving monodromy. The second kind of remaining symmetry is the Weyl group $W \coloneqq N(T)/T$. Taking their semidirect product, we obtain the \textit{affine Weyl group}.

If we consider the weight space decomposition of the moment map equation and $\beta$ is a root of $\g$, then the equation for this component is
\[ \dd{\xi_{\beta}} + \ev{\beta, \alpha} \xi_{\beta} = 0. \]
Because $\xi_{\beta}(u) = e^{-u\ev{\beta, \alpha}}$, this is generically not $1$, so $\xi_{\beta} = 0$. Therefore $\xi \in \mf{t}$, and we obtain
\[ T^* \wh{g} \sslash LG = (T^* T)/W. \]
Note that $T$ parameterizes conjugacy classes of $P \exp \oint A$ and that $T = \mf{t}/\Lambda^{\vee}$. Unfortunately, the reduced space is an orbifold, not a manifold.

We will now attempt to remedy this situation by modifying the quotient. Instead of setting the moment map to be $0$, we want to consider an orbit. We want $\mc{O} = \P^{N-1}$, and if $G = SU(N)$, $LG$ acts on $\P^{N-1}$ by evaluation at some $0 \in S^1$. We choose $z \in \C^N$ such that $z^{\dag} z = N$ up to $z \sim ze^{i\alpha}$, and the modified equation is
\[ \dd{\xi} + [A, \xi] = \delta(u) \cdot (i \nu(1_N - z \circ z^{\dag})). \]

\begin{rmk}
While most of the orbits are infinite dimensional, we are taking some limit where $A$ becomes a distribution on $S^1$ supported on finitely many points.
\end{rmk}

We first apply Floquet-Lyupanov to make $A = \mr{diag}(a_1, \ldots, a_N)$ diagonal with $\sum a_i = 0$. Then on each coordinate we obtain
\begin{align*}
  \dd{\xi_{ij}} + (a_i-a_j) \xi_{ij} &= \sqrt{-1} \delta(u)\nu(-z_i \ol{z}_j) \\
  \dd{\xi_{ii}} &= \sqrt{-1} \nu(1-\abs{z_i}^2) \delta(u).
\end{align*}
Because $\xi_{ii}(+0) = \xi_{ii}(2\pi-0) = \xi_{ii}(-0)$ for any $0 \in S^1$, $\abs{z_i}^2 = 1$. Using the maximal torus, we may force $z_i = 1$. Then we obtain
\begin{align*}
  \xi_{ij}(u) &= e^{-u(a_i-a_j)}\xi_{ij}(+0) \\
  \xi_{ij}(2\pi-0) &= e^{-2\pi(a_i-a_j)}\xi_{ij}(+0) = \xi_{ij}(+0) + \sqrt{-1} \nu.
\end{align*}
Finally, the initial value is
\[ \xi_{ij}(+0) = \frac{\sqrt{-1}\nu}{e^{-2\pi \sqrt{-1}(a_i-a_j)}-1}. \]
Note that this appeared in our study of the Sutherland system.

\chapter{Quantum mechanics}
\label{cha:complexified}

\section{Quantization}
\label{sec:quantization}

The motivation for complexification of our systems is quantization. Recall that if $S$ is the action of a Lagrangian system, many systems are not described by solving the variational equation
\[ \delta S = 0 \]
but by meditating on formal path integrals (due to Feynman)
\[ \int_{P \mc{P}} e^{\frac{i S[\gamma]}{\hslash}} [\mc{D}\gamma]. \]
The classical system is obtained via stationary phase approximation. Mathematically, this is ill-defined, but if $X$ is a finite-dimensional manifold, we can consider oscillating integrals
\[ I = \int_X e^{\frac{iS}{\hslash}} \mu. \]

Here, $X$ is one of many possible cycles in the complexification $X^{\C}$ and $\mu$ is viewed as a holomorphic top-degree form, so this is just a period. If $\dim X = n$, then we may consider other $\Gamma$ such that
\[ \int_{\Gamma} e^{\frac{iS}{\hslash}} \mu \]
converges. These will satisfy
\[ \Gamma \in H_n(X^{\C}, X^{\C}_{\ll}), \]
where
\[ X^{\C}_{\ll} = \qty{z \mid \Re\qty(\frac{iS(z)}{\hslash}) \ll 0} \]
is set to force the integral to converge. These $\Gamma$ are chosen to flow from critical points of $S$ (equivalently, of $W = iS$) in $X^{\C}$ into $X^{\C}_{\ll}$. We can construct cycles using \textit{Lefschetz thimbles}. We can choose a critical point $p$ which satsifies $\dd{W}(p) = 0$ and then consider the steepest descent flow for $\Re\qty(\frac{W}{\hslash})$, or in other words
\[ \dot{x} = -\nabla \Re\qty(\frac{W}{\hslash}), \]
where the gradient is taken with respect to some metric on $X^{\C}$, and finally take the union $\Gamma_p$ of descending trajectories. If we choose a Hermitian metric, then $\Im\qty(\frac{W}{\hslash})$ is actually constant. For a generic choice of $\hslash$, the critical points have different imaginary parts, but in the special settings we may have Stokes phenomena and wall-crossing behavior of solutions.

\begin{exer}
What happens for $W = \sum_{i=1}^n z_i^2$?
\end{exer}

\section{Holomorphic symplectic dynamics}
\label{sec:hk}

Now let $M^{\C}$ be a holomorphic symplectic manifold and consider $\mf{X}^{\C} = \ms{Maps}(S^1, M^{\C})$ and
\[ W = \int p \dd{q} - \beta H(p,q) \dd{t}. \]
We will take $S^1 = \R/\Z$ and we want to find points with $\dd{W} = 0$.

\begin{rmk}
When we quantize everything, we will obtain
\[ \Tr e^{-i\beta \wh{H}} = \int e^{\frac{iS}{\hslash}} [\mc{D}\gamma]. \]
\end{rmk}

We obtain the equations
\begin{align*}
\dv{p}{t} &= - \beta \pdv{H}{q} \\
\dv{q}{t} &=  \beta \pdv{H}{p} 
\end{align*}
If $\beta$ is real, this is the usual Hamiltonian dynamics, but there may not be real solutions if $\beta$ is not real. On the other hand, if $\beta = i\beta_E$ is purely imaginary (also known as \textit{going to Euclidaen time}), then there may be solutions where $q$ is real and $p$ is purely imaginary.

\begin{exm}
  Let $M = \R^2$ and
  \[ U(q) = \frac{\lambda}{4} (q^2-a^2)^2, \]
  where $\lambda, a$ are parameters. This is usually called the \textit{Higgs potential}. Then the energy is
  \[ E = \frac{p^2}{2} + U(q). \]
  We can see that there is a $q \mapsto -q$ symmetry, so near $\pm a$ there are two copies of the same physics. After complexifying, we obtain $M^{\C} = \C^2$, while the zero set $C_E$ of $E$ is a Riemann surface. If we compactify, we will obtain an elliptic curve with the hyperelliptic form
  \begin{align*}
    p^2 &= 2\qty(E-\frac{\lambda}{4}(q^2-a^2)^2) \\
    &= -\frac{\lambda}{2} (q^2-a_+^2)(q^2-a_-^2).
  \end{align*}
  Note that $H_1(C_E, \Z) = \Z \oplus \Z$, so there are two independent cycles corresponding to classically allowed physics. Also, it is clear that
  \[ a_{\pm}^2 - a^2 = \pm \sqrt{\frac{4E}{\lambda}}. \]
  Therefore, this family degenerates to a union of two copies of $\P^1$ when $E=0$ (and one of our distinguished cycles is the vanishing cycle) and has another critical point when $E = \frac{\lambda a^4}{4}$. Now $\omega = \frac{\dd{q}}{p}$ is a holomorphic differential on $C_E$, and the Hamilton equation tells us that any
  \[ \gamma \colon S^1 \to C_E \to M^{\C} \]
  must satisfies $\gamma^* \omega = \beta \dd{t}$. Now in the low-energy region, $[\gamma] \in H_1(C_E, \Z)$ can be specified by two integers:
  \[ [\gamma] = m[A] + n[B], \]
  where $[A]$ is the vanishing cycle and $[B]$ satisfies $A \cap B = 1$. Note $B$ is described up to multiples of $A$, so $n$ is well-defined, but $m$ is defined only up to multiples of $2n$. We can then find $\beta$ by the period integral
  \begin{align*}
    \beta &= \int_{S^1} \gamma^* \omega \\
          &= \int_{\gamma(S^1)} \omega \\
          &= m \oint_A \omega + n\oint_B \omega,
  \end{align*}
  which are functions of $E$. Therefore, the $E = E_{m,n}(\beta)$ for $n\in \Z, m \in \Z/2n\Z$ satisfy a transcendental equation
  \[ \beta = m\omega_A(E) + n\omega_B(E). \]
  We can solve this equation approximately in the $\beta \to \infty$ limit, where
  \[ E_{m,n} \approx \frac{\lambda a^4}{4} e^{-\frac{\beta \Omega}{n}}e^{\pi i \frac{m}{n}}, \]
  where $\Omega^2 = 2\lambda a^2$ is the classical period of very small oscillations around the $q=a$ critical point. This is all obtained by Picard-Lefschetz theory using the fact that
  \begin{align*}
    \omega_B(E) &\sim \frac{2\pi}{\Omega}\frac{1}{\pi i} \log E + \cdots \\
    \omega_A(E) &\sim \frac{2\pi}{\Omega} + \cdots,
  \end{align*}
  where the $+\cdots$ can be computed using knowledge of elliptic integrals. The solution where $n=0$ corresponds to classical physics, while the solution where $m=0$ corresponds to tunneling between the two critical points. Therefore, the general $(m,n)$ solution is some superposition of classical motion and tunneling.
\end{exm}

\section{Algebraic integrable systems}
\label{sec:algebraic}

We will now consider the class of systems which generalize the following feature of the previous example: our manifold $(M^{2n}_{\C}, \omega_\C)$ has a Lagrangian fibration $M^{2n} \to B^n$ to some open subset of $\C^n$ by polarized abelian varieties. These are called \textit{algebraic integrable systems}. This structure gives the structure of special K\"ahler geometry on $B^n$.

For us, a polarization is simply an integral class $t \in H^2(F, \Z)$, where $F = \pi^{-1}(b)$. We will now define action variables $(a^i, a_{D,i})$ on $B$. If we have $\gamma \in H_1(F_b, \Z)$, this path can be transported canonically over paths connecting $b$ to $b'$ that avoid the discriminant locus $\Xi$. We now obtain a $2$-chain in $M^{\C}$ covering the path, and we can obtain a number
\[ a_{\gamma}(b') - a_{\gamma}(b_*) = \int_{\text{$2$-chain}} \omega^{\C} \]
for a distinguished choice of $b_*$. Therefore, we obtain a map
\[ a \colon \wt{B^n \setminus \Xi} \to H^1(F_{b_*}, \C), \]
where $\wt{B^n \setminus \Xi}$ is the space of choices $(b, \gamma)$, where $b \in B^n \setminus \Xi$ and $\gamma$ is a path connecting $b_*$ to $b$ in $B^n \setminus \Xi$ up to homotopy. The image of $a$ is a Lagrangian with respect to $t$, where if $A_i, B^i$ is a basis of $1$-cycles in $H_1(F_{b_*}, \Z)$, then $t(A_i \cap B^j) = \delta_i^j$. Now because
\[ \sum_{i=1}^n \dd{a^i} \wedge \dd{a_{D,i}} = 0, \]
locally there exists $\mc{F}(a)$ such that
\[ a_{D,i} = \pdv{\mc{F}}{a_i}. \]

\begin{defn}
This $\mc{F}$ is called a \textit{prepotential}.
\end{defn}

\begin{exm}
  One example is the \textit{elliptic Calogero-Moser system}. Here, let $E$ be an elliptic curve parameterized by $\tau \in \mc{H}$ and consider
  \[ M^{\C} = ( T^* E^n \setminus \Delta )/S(n). \]
  Then let $p_i$ be the coordinates in the fiber directions and $z_i$ be the coordinates on the copies of $E$ and define
  \begin{align*}
    H_1 &= \sum_i p_i \\
    H_2 = \sum_i \frac{1}{2} p_i^2 + \nu^2 \sum_{i<j} \wp(z_i - z_j).
  \end{align*}
  This was proven to be an algebraic integrable system by Krichever (before algebraic integrable systems were defined). Set
  \[ L(z) \coloneqq \on{diag}(p_1, \ldots, p_n) + \nu(1-\delta_{ij}) \frac{\theta(z_i-z_j+z) \theta'(0)}{\theta(z_i-z_j) \theta(z)}, \]
  where the theta function is defined by
  \[ \theta(z) = q^{\frac{1}{8}} (e^{\pi i z} - e^{-\pi i z}) \prod_{n=1}^{\infty} (1-q^n)(1-q^n e^{2\pi i z})(1-q^n e^{-2\pi i z}). \]
  Here, we make the usual substitution $q = e^{2\pi i \tau}$. This operator satisfies the equations
  \begin{align*}
    L(z+1) &= L(z) \\
    L(z+\tau) &= g L(z) g^{-1}
  \end{align*}
  and has an expansion of the form
  \[ L(z) \sim \frac{\nu}{z} (1-\delta_{ij}). \]
  Therefore, the spectrum of $L(z)$ forms an $n$-sheeted cover over $E$, and in $T^* E$, if $g(C) = n$, then
  \[ C \sim n [E] + [F]. \]
  Then we will obtain a Lagrangian fibration where the fiber is the Jacobian $\mr{Jac}(C)$.
\end{exm}

\section{How to compute some path integrals exactly}
\label{sec:path_integrals}

Recall our setting of a symplectic manifold with Hamiltonian $(M, \omega, H)$. We will assume that there is an action of $T \cong U(1)^r$ with moment map $\mu \colon M \to \mf{t}^*$. We will also assume that $H$ is linear in the moment map, so
\[ H = \ev{\mu, \xi}. \]

\begin{exm}
  Consider $M = \R^{2n}$ with Darboux coordinates $p_i, q_i$, the standard action of $U(1)^n$, and moment map given by
  \[ \mu_j = \frac{1}{2} (p_j^2 + q_j^2). \]
  Then the Hamiltonian
  \[ H = \sum_{i=1}^n \xi_i \mu_i \]
  generates an action $\R \hookrightarrow T$ which is dense for generic $\xi$.
\end{exm}

If we now assume that $M$ is compact and the fixed points of the $T$-action are isolated, then to obtain the statistical-mechanical partition function for some inverse temperature $\beta$, we have the \textit{Duistermaat-Heckman} formula
\begin{align*}
  Z(\beta) &= \int_M \frac{\omega^n}{n!} e^{-\beta H} \\
  &= \sum_{\dd H_p = 0} \frac{e^{-\beta H(p)}}{\beta^n \prod_{i=1}^n \ev{m_i(p), \xi}},
\end{align*}
where $m_i(p)$ are the weights of the $T$-action on $T_p M$. With these weights, $H$ near $p$ behaves like
\[ H = H(p) + \frac{1}{2} \sum_{i=1}^r \xi_i \sum_{j=1}^n m_{ij}(p_j^2 + q_j^2). \]

\begin{exm}
If $M = S^2$ and $H = \cos\theta$ is the cosine of the azimuthal angle, we embed $M \subset M_{\C} \cong T^* S^2$, and the contour corresponding to the south pole goes into the cotangent directions.
\end{exm}

The Duistermaat-Heckman formula can be stated in the setting of $T$-equivariant cohomology. If $G$ is a Lie group acting on $M$, we will consider the Cartan model of equivariant cohomology.
\[ \Omega_G^*(M) \coloneqq \ms{Fun}(\g, \Omega^*(M))^G, \]
where we require that
\[ f(\Ad_{g^{-1}} \xi) = g^* f(\xi). \]
The grading on differential forms is deformed to be
\[ 2 \xi \pdv{\xi} + \deg_{\Omega^*_{\mr{DR}}}. \]
Then the equivariant differential is defined by
\[ D = \dd_{\mr{DR}} + \iota_{V(-)}, \]
where $V \colon \g \to \ms{Vect}(M)$ is a linear map. We can now compute
\begin{align*}
D^2 = \ms{Lie}_{V(-)} = 0,
\end{align*}
so we set $H_G^*(M)$ to be the cohomology of this complex. We can now compute equivariant integrals of $f \in \Omega_G^*(M)$. Note that $\int_M f \in \ms{Fun}(\g)^G$ and
\[ Z(\xi) = \int_M f_{\mr{top}}(\xi) = Z(\Ad_g \xi) \]
for all $g \in G$. Also note that
\[ \int_M (D \psi) = \int_M \dd \psi = 0. \]

We will now prove the Duistermaat-Heckman formula. If $\mu \colon M \to \g^*$ is the moment map, we claim that
\[ D(\omega + \mu(-)) = 0. \]
This follows from the definition of the moment map. We then see that
\[ D(\exp(\omega + \mu(-))) = 0, \]
and this is actually a Duistermaat-Heckman integral. If the $G$-action is free (and $G$ is compact), then every closed form is exact. Because $G$ is compact, there exists a $G$-invariant metric. Assuming that $G=T$ for now, choose some generic $\ol{\xi} \in \on{\ms{Lie}} T$. Then define
\[ \alpha = g(V(\ol{\xi}), -) \in \Omega^1(M). \]
We then obtain
\[ D \alpha = g(V(\ol{\xi}), V(\xi)) + \text{$2$-form} \]
where the function part is nonzero, and finally if
\[ \psi = \frac{\alpha}{D \alpha} f, \]
we obtain $f = D\psi$ whenever $Df = 0$. Now we can replace the integral in the Duistermaat-Heckmann formula with
\[ \int M_{\ep} + \sum_p \int_{B_{\ep}(p)}, \]
where $M_{\ep} = M \setminus \bigcup_p B_{\ep}(p)$, and finally use Stokes' theorem to obtain the desired result.

\section{Infinite-dimensional generalizations and supersymmetry}
\label{sec:infinite-supersymmetry}

Now suppose that $M = LN$, where $(N, g)$ is a Riemannian manifold. Then define for $\xi, \eta \in T_{\rho} M = \Gamma(S^1, \rho^* TN)$, we can define a $2$-form by
\[ \omega(\xi, \eta) = \int_{S^1} g(\xi, \nabla_t \eta) \dd{t}, \]
where $\nabla_t$ is the Levi-Cevita connection. The Hamiltonian that generates the standard $S^1$-action is
\[ H = \int g(\dot{\rho}, \dot{\rho}) \dd{t}. \]
Then the partition function of supersymmetric quantum mechanics on $N$ (index of the Dirac operator on $N$) is given by
\[ \int_{LN} e^{\omega - \beta H} = \int_N \wh{A}(TN), \]
where $\wh{A}$ is the $\wh{A}$-genus. The term on the right should be viewed as the contribution of non-isolated components of the fixed locus and can be thought of as Fourier modes of infinitesimal loops. Of course, there is the question of the role of $\beta$ in this formula.

\subsection{The Dirac operator}
\label{subsec:dirac}

First, associated to the metric $g |_{T_x N}$ we have the Clifford algebra, which is generated by symbols $\cancel{v}$ for $v \in T_x N = V$ with the relation
\[ \cancel{v} \cdot \cancel{u} + \cancel{u} \cdot \cancel{v} = g(u,v) 1. \]
This is the analogue of the the Heisenberg-Weyl algebra over a symplectic vector space $(W,\omega)$, which is generated by elements $\wh{w}$ for $w \in W$ with the relation
\[ \wh{w}_1 \cdot \wh{w}_2 - \wh{w}_2 \cdot \wh{w}_1 = \omega(w_1, w_2) \cdot 1. \]

The Clifford algebra $\mr{Cl}$ has irreducible representations, which are called \textit{spinors}. If we choose an orthonormal basis $\qty{e_i}$ of $V$, we can write any element as
\[ \alpha + \sum_i \beta_i \cancel{e}_i + \sum_{i < j} \gamma_{ij} \cancel{e}_i \cancel{e}_j + \cdots \]
In addition, we see that the Clifford algebra relation is a deformation of the exterior algebra, so $\mr{Cl} \cong \Lambda^* V$ as $\R$-vector spaces. The space of quadratic elements of the Clifford algebra is in fact $\on{\ms{Lie}} O(V)$. Because $O(V)$ is not simply connected, representations of its Lie algebra integrate to $\mr{Spin}(V)$, which is the universal cover of $SO(V)$ (at least if $\dim V > 2$). The group-like elements take the form
\[ g = \exp \frac{1}{2} \sum_{i<j} \gamma_{ij}\cancel{e}_i \cancel{e}_j. \]
Now we impose a complex structure compatible with the metric and define
\begin{align*}
c_a &= \cancel{e}_{2a-1} + \sqrt{-1} \cancel{e}_{2a} \\
c_a^* &= \cancel{e}_{2a-1} - \sqrt{-1} \cancel{e}_{2a} 
\end{align*}
for $a = 1, \ldots, \floor{\frac{\dim V}{2}}$.
These satisfy the relation
\[ c_a c_b + c_b c_a = 0. \]
Finally, we define the vector space
\[ S \coloneqq \C \ket{\mr{vac}} \oplus \bigoplus_a \C c_a \ket{\mr{vac}} \bigoplus_{a < b} \C c_a c_b \ket{\mr{vac}} \oplus \cdots \oplus \C c_1 \cdots c_{\floor{\frac{\dim V}{2}}} \ket{\mr{vac}}. \]
We impose that $c^* \ket{\mr{vac}} = 0$, and in general that $c^*$ moves to the left. We then define $\Gamma = e_{\dim V}$ if $\dim V$ is odd. It satisfies
\begin{align*}
\Gamma^2 = 1, \qquad \Gamma c_1 + c_a \Gamma = 0, \qquad \Gamma c_a^* + c_a^* \Gamma = 0. 
\end{align*}
From now on, we will assume that $\dim V = 2k$.

Note that $S$ is an irreducible representation of $\mc{Cl}$, but recall that $\mf{o}(V)$ is much smaller. It is spanned by elements of the form $c_a c_b, c_ac_b^*, c_a^* c_b^*$. There is a $\Z/2$-valued conserved charge called $(-)^F$ which is preserved by $\mf{o}(V)$. For $S$, it is defined by
\begin{align*}
  F \ket{\mr{vac}} &= -\frac{k}{2} \ket{\mr{vac}} \\
  F (c_{a_1} \cdots c_{a_p}\ket{\mr{vac}}) = p - \frac{k}{2}.
\end{align*}
Therefore, we can split
\[ S = S_+ \oplus S_- \]
by the parity of the $F$-charge. Then for any $u \in V$, we have
\[ \cancel{u} \colon S_{\pm} \to S_{\mp}. \]
These are called \textit{Dirac matrices}.

Unfortunately, there is not always a global spinor bundle on $N$. The obstruction is the Stiefel-Whitney class
\[ w_2(TN) \in H^2(N, \Z/2). \]
If this is nonzero, then we cannot glue spinors into a vector bundle and thus the manifold $N$ does not have a spin structure. Therefore, we will now assume that $w_2(TN) = 0$. We will now define the \textit{Dirac operator}
\[ \cancel{D} \colon \Gamma(S_+) \to \Gamma(S_-) \]
by the local formula
\[ \cancel{D} \coloneqq \sum_{i=1}^{\dim M} \gamma^i \nabla_i. \]
Here, $\nabla_i$ is a spin cover of the Levi-Civita connection and $\gamma^i$ is the physicist notation for $\cancel{e}_i$. We also define $\cancel{D}^*$ to be the same operator applied to $S_-$. The symbol of $\cancel{D}^* \cancel{D}$ or of $\cancel{D} \cancel{D}^*$ is given by the well-known formula
\begin{align*}
  \sum_{i,j} (\gamma^i \partial_ii)(\gamma^j \partial_j) &= \frac{1}{2} \sum_{i,j} (\gamma^i \gamma^j + \gamma^j \gamma^i)\partial_i \partial_j \\
                                                         &= \frac{1}{2} \sum_{i,j} g^{ij} \partial_i \partial_j + \cdots \\
  &= \Delta + \cdots
\end{align*}
Because of the analogy with the Laplacian, the spaces $H_{S_+} = \ker \cancel{D}$ and $H_{S_i} = \ker \cancel{D}^*$ are known as \textit{harmonic spinors}. There is a lucky coincidence that if $\wh{H} = \cancel{D} \cancel{D}^* + \cancel{D}^*\cancel{D}$, then
\begin{align*}
  \on{Index} \cancel{D} &= \dim \ker \cancel{D} - \dim \ker \cancel{D}^* \\
                        &= \Tr_{\mc{H}_{L^2(S)}} (-1)^F e^{-\beta \wh{H}} \\
                        &= -\Tr_{L^2(S_-)} e^{-\beta\cancel{D} \cancel{D}^*} + \Tr_{L^2(S_+)} e^{-\beta \cancel{D}^*\cancel{D}}
\end{align*}
for any $\beta > 0$. Note here that working on $\R^n$,
\[ \Tr_{L^2(S)} e^{-\beta \cancel{D} \cancel{D}^*} \approx \Tr e^{\beta \Delta} \approx\int \dd^n p e^{-\beta p^2} \sim \frac{1}{\beta^{\frac{n}{2}}}. \]

Now suppose that $\psi_{(k)} \in \Gamma(S_+)$ satisfies
\[ \cancel{D}^* \cancel{D} \psi_{(k)} = \ep_k \psi_{(k)}. \]
Then $\cancel{D} \psi_{(k)} = \chi_{(k)} \in \Gamma(S_-)$ satisfies
\[ \cancel{D} \cancel{D}^* \chi_{(k)} = \ep_k \chi_{(k)}. \]
This tells us that
\begin{align*}
  \Tr e^{-\beta\cancel{D}^* \cancel{D}} - \Tr e^{-\beta \cancel{D}\cancel{D}^*} &= \ker \cancel{D} - \ker \cancel{D}^* + \sum_{k,\ep_k > 0} e^{-\beta \ep_k} - \sum_{k,\ep_k > 0} e^{-\beta \ep_k} \\
                                                                                &= \on{Index} \cancel{D}.
\end{align*}
Note that this cancellation is our first example of \textit{supersymmetry}. Finally, we take the $\beta \to 0$ limit.

\subsection{The heat kernel}
\label{subsec:heat_kernel}

If we were considering on flat space the ordinary heat kernel
\[ K(x, x', \beta) = \mel{x}{e^{\beta \Delta}}{x'}, \]
then $\Tr e^{\beta \Delta}$ is computed by
\[ \int_{LN} [D x(t)] \exp \qty(-\frac{1}{2} \int_0^{\beta} g(\dot{x}, \dot{x}) \dd{t}). \]
Note that in flat space, the heat kernel must satisfy the PDE
\[ \pdv{\beta} K = \Delta_{x'} K = \Delta_x K. \]
As $\beta \to 0$,
\[ K(x,x',\beta) \to \delta^{(n)}(x-x'). \]
There must also be the integral formula
\[ \int \dd{x'} K(x,x';\beta_1) K(x',x'';\beta_2) = K(x,x'';\beta_1 + \beta_2), \]
so in fact, it is given by
\[ K(x,x';\beta) = \exp\qty(-\frac{(x-x')^2}{2\beta}) \frac{1}{(2\pi \beta)^{n/2}}. \]

In our curved situation, we obtain the limit
\[ \lim_{M \to \infty} \int \dd{x_1} \cdots \dd{x_M} K\qty(x_1,x_2;\frac{\beta}{M}) \cdots, \]
which can be well-approximated by the expression in flat space. Considering instead the loop space $L T^*N$, we now obtain
\[ \int_{L T^*N} [D x(s) Dp(s)] \exp \qty(i \int p \dd{x} - \beta \int_0^1 g^{-1}(p,p) \dd{s}). \]
Note that the $[D x(s) Dp(s)]$ carries the term $\prod_t \frac{1}{(2\pi\beta)^{\frac{\dim M}{2}}}$. Also, if we integrate out the $p$, we obtain the previous term
\[ \frac{1}{2\beta} \int_0^1 g(\dot{x}, \dot{x}) \dd{s}. \]

\subsection{Supersymmetric Duistermaat-Heckmann}
\label{subsec:supersymmetric}

We will now rewrite the integral we wanted to compute at the beginning of this section in new notation as
\[ \int_{LN} e^{\Omega + \ep H}. \]
Here, we rewrite
\begin{align*}
  H &= \frac{1}{2} \int_0^1 g(\dot{x}, \dot{x}) \dd{s} \\
  \Omega &= \frac{1}{2} \int_{S^1} g_{ij} \psi^i \nabla_s \psi^j \dd{s},
\end{align*}
where $(\psi^i(s)) \in \Gamma(S^1, x^* TN)$. The $\psi^i$ satisfy the formula
\[ \psi^i(s) \psi^j(s') = -\psi^j(s') \psi^i(s). \]
Finally, here $\ep = \frac{1}{\beta}$ is the equivariant parameter for the action of $U(1)$ on the cotangent directions.

Now, the Duistermaat-Heckmann normalized integral (which is not the Atiyah-Singer normalized integral) is
\begin{align*}
  \int_{LN} e^{\Omega + \ep H} &= \int_M \frac{1}{\prod_{n \neq 0} \prod_{\alpha} \qty(n \ep + \frac{1}{\ep}\alpha)} \\
  &= \int_N \prod_{\alpha} \frac{\pi \sigma_{\alpha}/\ep}{\sin(\pi \sigma_{\alpha}/\ep)},
\end{align*}
where $x^i(t)$ splits into Fourier modes as
\[ x^i(t) = x^i_0 + \sum_{n \neq 0} \xi_n^i e^{2\pi i n s} \]
and thus splits
\[ \mc{N}_{N/LN} = \bigoplus_{n \neq 0} (TN)_n, \]
and $\alpha$ are the Chern roots of $TN$. This exactly reproduces the $\hat{A}$-genus.
The Atiyah-Singer normalized integral has an extra factor of $\frac{1}{\prod_s \beta^{\frac{\dim M}{2}}}$.

\section{Duality}%
\label{cha:duality}

There are two main examples: $T$-duality in $2$d sigma-models and $S$-duality in $4$d gauge theories.

\subsection{$p$-form generalized gauge theory}

We will begin with $p$-form generalized gauge theory in $D$ spacetime dimensions. For some $A \in \Omega^p(M^D)$, define
\[ \mc{L} = \int_{M^D} \dd{A} \wedge * \dd{A}. \]
This is invariant under $A \to A + \dd{B}$. The space of fields in the $\R$-type theory is given by
\[ \mc{A}_{\R} = \Omega^P(M^D) / \dd{\Omega^{p-1}(M^D)}. \]
In the $U(1)$-type setting, the space of fields is smaller and is given by
\[ \mc{A}_{U(1)} = \Omega^p(M^D) / \Omega^p_{\Z}(M^D), \]
where $\Omega^p_{\Z}$ is the space of all $p$-forms alpha such that 
\[ \int_{Z^p} \alpha \in \Z \]
for all integral cycles $Z^p$.

\begin{exm}
    When $p=0$ and $D = 2$, then the $U(1)$-type theory describes maps $M^2 \to U(1)$. Locally, these look like $A \colon M^2 \to \R$ such that $A \sim A + n$. Of course, not every manifold is simply connected. Therefore, the true space of fields is
    \[ \mc{A}_{U(1)} = \ms{Maps}(M^2, U(1)). \]
    This has a decomposition by the topological type as follows. Let $t \in \R$ be a coordinate with
    \[ \int_{\R/\Z} \dd{t} = 1. \]
    Then the topological type is given by
    \[ [f^* \dd{t}] \in H^1(M^2, \Z). \]
\end{exm}

In general, we will assume that the curvature form $\dd{A}$ is not exact, but instead integral, so
\[ \dd{A} \in \Omega^{p+1}_{\Z}(M^D). \]
This can be achieved on an open cover $U_{\alpha}$ by choosing
\[ A_{\alpha} \in \Omega^p(U_{\alpha}) \]
such that 
\[ A_{\beta} - A_{\alpha} \in \Omega^p_{\Z}(U_{\alpha} \cap U_{\beta}). \]
Then the space of fields is the space of connections on all $U(1)$-bundles and its connected components are parameterized by
\[ L = H^{p+1}(M^D, \Z). \]

The partition function is now given by
\[ Z = \int_{\mc{A}_{U(1)}} [\mc{D}A] e^{-\frac{1}{4g^2} \int_{M^D} \dd{A} \wedge * \dd{A} + \int \theta \wedge \dd{A} \wedge \dd{A}}, \]
where
\[ \theta \in \Omega^{D-2(p+1)}(M^D) \]
is a closed form.

We may also replace $f \colon M^2 \to U(1)$ by functions
\[ f \colon M^2 \to T \cong U(1)^n = V/\Gamma. \]
We introduce tensors 
\[ B \in \Lambda^2 V^*, \qquad G \in S^2_+ V^*. \]
In this form, the partition function becomes
\[ Z(G,B,h) = \int_{\mc{A}_{U(1)}} [\mc{D}A] e^{-\int_{M^D} G_{ij} \dd{A^i} \wedge * \dd{A^j} + \int B_{ij} \wedge \dd{A^i} \wedge \dd{A^j}}, \]
where $\dd{B_{ij}} = 0$ and $B_{ij} \in \Omega^{D-2(p+1)}(M^D)$. We also have
\[ B \in \begin{cases}
    \Lambda^2 V^* & p+1 \text{ odd} \\
    S^2 V^* & p+1 \text{ even}.
\end{cases}
\]
We now want to find the critical points of $\mc{L}$, where $\delta \mc{L} = 0$. The equations for $A$ become
\[ \dd * \dd{A^i} = 0 \]
for all $i$. If we set $F^i \coloneqq \dd{A^i}$, the equation becomes 
\[ \dd{*F^i} = 0, \]
which are the \textit{generalized Maxwell equations}. Therefore, we obtain
\[ Z(G,B,h) = \mc{N}(G,h) \sum_{c \in \Lambda = H^{p+1}(M^D, \Gamma)} \exp\qty[-G_{ij} \ev{c^i,*c^j} - i \ev{B_{ij} \wedge c^i \wedge c^j}], \]
where $\mc{N}(G,h)$ is a regularized version of
\[ \det_{\Omega^p/\dd{\Omega^{p-1}}}(*\dd*\dd)^{-\frac{1}{2}} \]
which is defined as follows. Define
\[ \zeta_{\Delta^{(p)}}(s) \coloneqq \Tr(-\Delta^{(p)})^{-s}. \]
This is defined for $\Re s \gg 0$, and taking the analytic continuation, we define
\[ \mc{N}(G,h) \coloneqq \exp \frac{1}{2} \zeta'_{{\Delta}}(0). \]
We may need to assume that $D = 2(p+1)$. For example, if $D=2$, then the $\ev{c^i, * c^j}$ term knows only about the conformal structure of $M^2$. Finally, $H^{p+1}(M^D, \Gamma)$ is taken modulo torsion.

\subsection{Duality}

The duality is given by inverting 
\[ \tau \coloneqq iG + B \]
when $D = 4$ and inverting
\[ \tau = G+iB \]
when $D=2$. We may consider transformations of the form
\[ \tau \mapsto -\frac{1}{\tau} \]
or more generally
\[ \tau \mapsto (C\tau+D)^{-1}(A\tau + B), \]
where
\[ \mqty(A & B \\ C & D) \in \begin{cases}
    O(n,n,\Z) & D=2 \\
    \on{Sp}(2n,\Z) & D=4.
\end{cases} \]
Applying Poisson resummation, we obtain
\[ \int_{c \in \Lambda{\R}} \sum_{\check{c} \in \Lambda^*} e^{2\pi i \check{c}(c)} \exp \qty[-\frac{1}{2} G_{ij} \ev{c_i, *c_j} + \frac{i}{2} B_{ij} \ev{c_i, c_j}]. \]
Here, we have the identity
\[ \sum_j G_{ij} * c^j + \sqrt{-1} B_{ij} c^j + 2 \pi \sqrt{-1} \check{c}_i = 0 \]
in dimension $2$. Note that when $D=2$, $*^2 = -1$, while when $D=4$, $*^2 = +1$. In the $4$-dimensional setting, if we define
\[ c_{\pm} = \frac{c \pm *c}{2}, \]
the relation is
\[ \tau c_i - \ol{\tau} c_+ + 2 \pi \sqrt{-1} (\check{c}_+ + \check{c}_-) = 0. \]
Therefore,
\[ c_- = \frac{2 \pi \sqrt{-1}}{\tau} \check{c}_-, \qquad c_+ = -\frac{2 \pi \sqrt{-1}}{\tau} \check{c}_+. \]

The duality must give some reassignment of the degrees of freedom. Locally, if
\[ M^D = X^{D-1} \times \R, \]
which is noncompact, we must discuss the Hamiltonian system. The phase space is given by the stack
\[ T^* [ \Omega^p(X^{D-1}, V) / \Omega^p(X^{D-1},\Gamma) ]. \]
The symplectic form is given by
\[ \int_{X^{D-1}} (\delta E \wedge \delta A_{(p)}) + \int_{X^{D-1}} B_{ij} \wedge \delta(\dd{A^i}) \wedge \delta A^j_{(p)}, \]
Here,
\[ E \in \Omega^{D-1-p}(X^{D-1}, V^*) \]
is the electric field and $\dd{A}$ is the magnetic field. Now we can write the Hamiltonian as
\[ \mc{H} = \frac{1}{2} \int_X G^{ij} E_i \wedge * E_j + \frac{1}{2} \int_X G_{ij} \dd{A^i} \wedge * \dd{A}^j. \]

We return to the space of spatial fields $\mc{A}$, which has a decomposition into connected components indexed by
\[ c \in H^{p+1}(X^{D-1},\Gamma)/\mr{Tors}. \]
The component corresponding to some $c$ is 
\[ \mc{A}_c = H^p(X^{D-1},V) / H^p(X^{D-1},\Gamma), \]
which is the space of flat $p$-connections.

All of this can be understood via the quantum mechanics of a particle on $S^1$. If we define 
\[ \mc{H}_{\theta} = \qty{f(t) \mid f(t+2\pi) = e^{i\theta} f(t), \int \abs{f}^2 < \infty}, \]
then $\theta$ is the analogue of $(B_{ij})$. Then if the standard symplectic form on $T^* S^1$ is given by
\[ \dd{E} \wedge \dd{t}, \]
$E^2$ quantizes to $-\partial_t^2 = \wh{H}$, and the spectrum of this operator is given by
\[ E_n = (n+\theta)^2, \qquad n \in \Z. \]
For example, if $E_n = 0$, then $n$ and $-n$ have the same eigenvalues, if $\theta = \frac{1}{2}$, then the spectrum is doubly degenerate, and for any other value of $\theta$ the spectrum is simple. Therefore, the duality exchanges the electric and magnetic fields whenever $D = 2(p+1)$.

\chapter{Instantons}%
\label{cha:Instantons}

There are three main approaches:
\begin{itemize}
    \item Morse theory;
    \item Twisted supersymmetric $\sigma$-models;
    \item Twisted supersymmetric gauge theory.
\end{itemize}

These have a Hamiltonian and Lagrangian approach, and the latter will motivate the study of integrals on moduli spaces of instantons.

\section{Morse theory}

Note that this is not the same as what topologists call Morse theory. Let $(M,g)$ be a compact Riemannian manifold and $f \colon M \to \R$. Suppose all $x$ such that $\dd{f}_x = 0$ have nondegenerate Hessian, or in other words,
\[ \det \qty(\pdv[2]{f}{x^i}{x^j}) \neq 0. \]
We will consider $\Omega^*(M)$ with the differential
\[ D \coloneqq \dd + \dd{f} \wedge \colon \Omega^*(M) \to \Omega^{*+1}(M). \]
Using the standard scalar form
\[ (\alpha,\beta)_g = \int_M \alpha \wedge * \beta, \]
we define the adjoint
\[ D^* = \dd^* + \iota_{\nabla f}. \]
Physicists refer to the study of this package as \textit{supersymmetric quantum mechanics}. Here, the Hamiltonian is
\begin{align*}
    H &= DD^* + D^*D \\
    &= -\Delta_{\dd} + \ms{Lie}_{\nabla f} + \ms{Lie}^*_{\nabla f} + g(\nabla f, \nabla f).
\end{align*}
Note that $\ker H \cong H^*(M)$, and if $H\psi = 0$, then
\begin{align*}
    0 &= (\psi, H \psi) \\
    &= (D\psi, D\psi) + (D^* \psi, D^*\psi),
\end{align*}
so we see that $H\psi = 0$ if and only if $D \Psi = D^*\psi = 0$.

There are also excited states where $H \psi_i = E_i \psi_i$ for $E_i > 0$. Then we compute
\begin{align*}
    H D\psi_i &= DD^* D\psi_i \\
    &= D(H \psi_i - DD^* \psi_i) \\
    &= D(E_i \psi_i) \\
    &= E_i D \psi_i,
\end{align*}
so the states $D\psi_i$, $D^*\psi_i$, and $DD^* \psi_i$ are also eigenstates with eigenvalue $E_i$.

There is then the following trick. If $f \mapsto t f$ and $g$ is fixed for $t \gg 0$, then
\[ H = -\Delta_g + t^2 \norm{\nabla f}^2 + t (\cdots) \]
where $\norm{\nabla f}^2$ is a very large potential outside of the critical locus of $f$. We also have
\[ D_t = e^{-tf} \dd{e^{tf}} \]
on the ground states, which are isomorphic to $H^*(M)$. More generally, we note that
\begin{align*}
    D_t &= \dd + t \dd{f} \wedge \\
    D_t^* &= t^{-1} \dd^* + \iota_{\nabla f}.
\end{align*}
Ignoring the $t^{-1} \dd^*$ term, the Hamiltonian becomes
\[ \ms{Lie}_{\nabla f} + t \norm{\nabla f}^2 = e^{-tf}(\ms{Lie}_{\nabla f})e^{tf}. \]

In the world where $f \mapsto tf$ and $g \mapsto tg$, we obtain states
\begin{align*}
    \psi &\to e^{-tf} \psi \eqqcolon \psi_{\mr{out}} \\
    \psi &\to e^{tf} \psi \eqqcolon \psi_{\mr{in}}.
\end{align*}
As $t \to \infty$, we obtain $H_{\mr{in}}$ and $H_{\mr{out}}$, which are distributions.

\begin{exm}
    Now consider $M = \R$ and $f = \frac{\omega x^2}{2}$. Note in this case we need to use $L^2$ differential forms. Then we have basic operators
    \begin{align*}
        H_0 &\coloneqq -\partial_x^2 + \omega^2 x^2 + \omega \\
        &= (-\partial_x + \omega x)(\partial_x + \omega x) \\
        H^i &\coloneqq -\partial_x^2 + \omega^2 x^2 + \omega \\
        &= (\partial_x + \omega x)(-\partial_x + \omega x).
    \end{align*}
    Finally, we set $\alpha = \psi(x) \dd{x}$.

    The spectrum is given in the following way. If $H\psi = 0$, then the equality $(\partial_x + \omega x)\psi = 0$ implies that $\psi = e^{-f}$, which is fine if $\omega > 0$ and bad if $\omega < 0$. Then suppose there is $\psi_n$ such that $H \psi_n = n \abs{\omega} \psi_n$. Then we obtain
    \[ \psi_n = \begin{cases}
        (-\partial_x + \omega x)^n \qty(e^{-\frac{\omega x^2}{2}}) & \omega > 0 \\
        (\partial_x + \omega x)^{n-1} \qty(e^{\frac{\omega x^2}{2}}) & \omega < 0 .
    \end{cases}
    \]
    Then if $P$ is a polynomial,
    \[ P(\partial_x) e^{-t \abs{\omega}x^2} \to P(\partial_x) \delta(x) \]
    as $t \to \infty$ (here $t$ scales $\omega$). This tells us that $H_{\mr{in}}^0 \cong H^1_{\mr{out}}$ are regular functions, while eigenstates of $H_{\infty}$ are monomials. Similarly, $H^1_{\mr{in}} \cong H^0_{\mr{out}}$ are distributions supported at $x=0$, while the eigenstates of $H_{\infty}$ are those of the form $\partial_x^n \delta(x)$. Observe that
    \[ H_{\infty} = \ms{Lie}_{x \dv{x}}. \]
    Then, note that $\delta(x) \dd{x}$ is an invariant distribution-valued $1$-form. The pairing between the in and out states must be the usual pairing between functions and distributions.
\end{exm}

In higher dimensions, there will be both attracting and repelling states, so we can consider something like
\[ f = \sum_{i=1}^{d-m} \frac{x_i^2}{2} - \sum_{j=d-m+1}^d \frac{x_j^2}{2} \]
and i the end $\psi_{\mr{in}}$ is a distribution supported on the attracting manifold. This concludes the study of the local picture.

However, we must consider the compact picture. For example, consider $S^1 = \R \cup \infty$ where the Morse function $f$ is attracting at $x=0$ and repelling at $x=\infty$. Then near $\infty$, the states  corresponding to monomials will look like $\on{P.V.}x^n$. Applying $x \dv{x}$ and expanding
\[ \psi = \psi_{\infty} + \frac{1}{x}\psi_{\infty}^{(1)} + \cdots + \frac{1}{x^{n+1}} \psi_{\infty}^{(n+1)} + \cdots, \]
then the scaling $x \to tx, \ep to t\ep$ gives us correction terms of $\log \ep$ and $\log t$ to $\psi_{\infty}^{(n+1)}$. Here, the principal value
\begin{align*}
    (x^n,\psi) &= \on{P.V.} \int_{\R} x^n \psi \dd{x}
\end{align*}
is the finite part in the $\ep$-expansion of
\[ \int_{-\frac{1}{\ep}}^{\frac{1}{\ep}} x^n \psi \dd{x}. \]
Therefore, we see that critical points can talk to each other via excited states.

\section{$\sigma$-models}

We will consider maps $\Sigma \to (X,\omega)$ from a Riemann surface to a symplectic manifold. We will also choose an almost complex structure $J$ which is tame with respect to $\omega$. If we consider the functional
\[ \mathbb{S} = \frac{1}{2} \int_{\Sigma} g(\dd{\phi}, *_h \dd{\phi}) + i \int_{\Sigma} \phi^* \omega, \]
its critical points are simply harmonic maps.

In the Hammiltonian formalism, consider $M = LX$. If $\xi \in T_{\ell} LX$, then
\[ \dd{f}(\xi) = \int_{S^1} \omega(\xi, \dot{\ell}). \]
Note that
\[ g(\dd{\phi}, * \dd{\phi}) = \omega(J \dd{\phi}, *_h \dd{\phi}) \]
and
\[ \dd{\phi} = \frac{1+iJ}{2} \dd{\phi} + \frac{1-iJ}{2} \dd{\phi}. \]
Then $\mathbb{S}$ can be rewritten as
\[ \mathbb{S} = \int \phi^*(\pm \omega + iB) + \norm{\frac{1-iJ}{2} (\partial \text{ or }\ol{\partial})\phi}^2, \]
where $B$ is the $B$-field, satisfying $\dd{B} = 0$. Therefore, the absolute minima are solutions to the PDE
\[ \frac{1-iJ}{2} \ol{\partial}\phi = 0. \]

\begin{exm}
    Consider $X = \R^2$ with the standard symplectic form and complex structure. Let
    \[ f = \int_{S^1} p \dd{q}. \]
    Let $t$ be the loop variable and $s$ be the noncompact direction. Then we obtain the equations
    \begin{align*}
        \dv{p}{s} &= -\fdv{f}{p} = -\dv{q}{t} \\
        \dv{q}{s} &= -\fdv{f}{q} = -\dv{p}{t},
    \end{align*}
    which are the Cauchy-Riemann equations. If $z = s + it$ and $w = \exp(z)$, the Cauchy-Riemann equations give
    \[ \ol{\partial}_{\ol{z}} (q+ip) = 0. \]
\end{exm}


\section{$4$d supersymmetric Yang-Mills}

The word symmetry here is actually a confusing one. There are two types of symmetries one encounters in physics, and one is \textbf{not} a symmetry. Sometimes what we mean by symmetry is a \textbf{redundancy}; mathematically what that means is that we're studying a quotient of some space $X$ by some symmetry $G$, $X/G$. This is a \textit{local symmetry}. The other type of symmetry is a \textit{global symmetry} where we really have $G$ acting on the space we are studying $X$.
In the case of a local symmetry, platonically there is a space $X$ with action of $G$, but that space is not accessible to us. Rather we only can see $X/G$.
Occasionally, it's the case that we find there exists $Y/H = X/G$. We call this a \textit{duality}.

Because the groups we deal with in physics are typically infinite dimensional groups, for example $\ms{Maps}(M,G)$ of a manifold into a finite dimensional group, these groups typically have lots of normal subgroups.  Sometimes rather than the full $\ms{ Maps }(M,G)$, our local symmetry is some sort of normal subgroup that preserves additional structure: for example, possibly the normal subgroup that sends certain marked points $\{x_{i}\}$ of $M$ into certain subgroups $\{H_{i}\}$ of $G$.
The global symmetry will then be whatever is left from $\ms{Maps}(M,G)$, for example in this case $\times_{i} G/H_{i}$.

For example, say we are studying $(M,g)$ a smooth Riemannian 4-manifold, with $G$ a compact simple Lie group, and a principal $G$-bundle $\mathcal{P}$ over $M$. Let $\mathcal{A}_{\mathcal{P}}$ the affine space of $G$-connections on $\mathcal{P}$.  There is a group $g_{\mathcal{P}} := \Gamma(M^{4},G \times_{Ad} \mathcal{P})$ acting on $\mathcal{A}_{\mathcal{P}}$; what we really want to study is the quotient
\[
B_{\mathcal{P}} := \mathcal{A}_{\mathcal{P}}/g_{\mathcal{P}}.
\]
The functional we seek to integrate over $B_{\mathcal{P}}$ is
\[
\int_{B_{\mathcal{P}}} [D\mathcal{A}] e^{-\frac{1}{4c^{2}} \int_{M} \Tr F \wedge \star_{g} F +  \frac{i \theta}{2\pi} \int_{M} \Tr F \wedge F + \dots}
\]
The measure here is induced from the $L^{2}$ metric, and $Tr$ refers to the killing form. Our integral consists of two parts.

\begin{itemize}
  \item The first is the Yang-Mills action;
\item The second is a *topological term* which does not depend on the metric and picks out some topological equivalence class of principal bundles. We normalise it so that the whole thing is an integral class.
\end{itemize}

Since the second term is topological, we know eventually that our integral is equivalent to
\[
\sum_{n \in \mathbb{Z}} e^{i n \theta} Z_{n}
\]
where $Z_{n}$ is the integral taken over bundles of Pontryagin class $n$.
  We could try to write this on a lattice, viewing $\mathcal{A}_{P}$ as the edges of a graph and $g_{\mathcal{P}}$ as the vertices. But if you did this naively you would find that the coupling has to go to zero.
  Instead, we can try a trick which we started to discuss last time. If we write
\[
F_{A}^{+} = \frac{1}{2} (F_{A} + \star F_{A}),
\]
the Yang-mills term in the functional is equal to
\[
\norm{F_{A}^{+}}^{2} + \int \Tr F_{A} \wedge F_{A}.
\]
Then we can rewrite the whole functional as
\[
e^{2\pi i \tau( -\frac{1}{8\pi^{2}} \int \Tr F_{A} \wedge F_{A} ) - \frac{1}{4c^{2}} \norm{F^{+}}^{2}},
\]
where
\[
\tau = \frac{\vartheta}{2\pi} + \frac{4\pi i }{c^{2}}
\]
is some function in the upper half-space.
We could try to study the limit $c^{2} \to 0$, while keeping $\tau$ finite.  In that limit the functional goes to
\[
\delta(F_{A}^{+}) e^{2\pi i \tau c_{2}\mathcal{P}}.
\]

Studying the moduli space of $\mathcal{M}_{\mathcal{P}} := \{ A | F_{A}^{+} = 0\}/g_{\mathcal{P}}$,  which we roughly expect to be finite dimensional because $\Lambda^{2,+} \mathbb{R}^{4} \simeq \mathbb{R}^{3}$, we can at the very least compute its virtual dimension.
To do so, assume
\begin{itemize}
  \item $A_{0}$ is so that $F_{A_{0}}^{+} = 0$;
  \item $A = A_{0} + \delta A$ so that $F_{A_{0} + \delta_{A}}^{+} = \dd_{A_{0}}^{+} \delta A$;
        \item wherein $\dd^{+}_{A_{0}} : \Omega^{1}(M) \otimes \ad\mathcal{P} \to \Omega^{2,+} (M) \otimes \ad \mathcal{P}$.
\end{itemize}

To the linear level, it suffices to computer $\ker \dd^{+}$. Since
\[
\dd^{+}_{A_{0}} \dd_{A_{0}} = \frac{1}{2} (1 + \star) \dd^{2}_{A_{0}},
\]
the virtual tangent bundle is
\[
T_{vir} := \ker \dd_{A_{0}}^{+}/\Im \dd_{A_{0}},
\]
which is the first cohomology of the Atiyah-Hitchin-Singer complex
\[
\Omega^{0} \oplus \ad \mathcal{P} \to \Omega^{1} \otimes \ad \mathcal{P} \to \Omega^{2,+} \otimes \ad\mathcal{P},
\]
wherein the maps are $\dd_{A_{0}}, \dd_{A_{0}}^{+}$.
If $H^{0},H^{2}$ of this complex don't vanish, we could have singularities or obstructions and therefore trouble counting dimension.

\subsection{``theory of differential forms on $\mathcal{M}_{\mathcal{P}}$''}
We could try to overcome the infinite dimensionality of the problem by seeking to develop a theory of differential forms on $\mathcal{M}_{\mathcal{P}}$, the space of anti-self dual connections, i.e. those $A$ such that $F_{A}^{+} = 0$. To do so, we would need some notion of local coordinates on $\mathcal{M}_{\mathcal{P}}$. Let's first try to do this on the space of connections without the ASD condition.

Define
\[
Q = \dd_{\mr{DR}}
\]
on the space of connections $\mathcal{A}_{\mathcal{P}}$.  Then
\[
QA = \Psi \in \Omega^{1}(M) \otimes \prod  \ad  \mc{P}.
\]
This is a one-form, hence an 'odd' object, i.e. a fermion. So $Q$ meaps even to odd and vice versa.


But remember that $A := A_{\mu} dx^{\mu}$ is a \textbf{redundant} description, because connections related by gauge transformation are the same.  To deal with this, rather than $d_{DR}$ we will deal with *equivariant differential forms*, and write
\[
Q = \dd_{\mr{DR}} + \iota_{V\, \phi}
\]
on the Cartan model for equivariant cohomology, so that this differential is an operator on
\[
(\Omega^{\bullet} (\mathcal{A} \times \ms{Lie}(g_{\mathcal{P}})) \times \ms{Fun}(\ms{Lie}(g_{\mathcal{P}})))^{g_{\mathcal{P}}}.
\]
Since $H^{\bullet}( (M \times EG)/G) \simeq H^{\bullet}(M/G)$, if we want this relation not only at the level of chomology but also on the actual space of differential forms, it's enough to study invariant forms, hence why we're taking invariants by $g^{\mathcal{P}}$.
This is like passing from $M/G \to M \times \mathfrak{g}/G$, which slightly improves the situation of stabilisers and therefore the space of equivariant differential forms.
Then,
\begin{align}
    QA &= \Psi
  \\
    Q\Psi &= d_{A} \phi, \phi \in \Gamma(\ad\mathcal{P}) = \ms{Lie}(g_{\mathcal{P}})
  \\
    Q\phi &= 0.
\end{align}
The extra factor of the Lie algebra also defines an object $\overline{\phi}$, so that $Q \overline{\phi} = \eta$ where $Q \eta = [\phi, \overline{\phi}]$.  Here $\overline{\phi}$ lives in the $\ms{Lie}(g_{\mathcal{P}})$ inside our forms itself, whilst $\phi$ lives in the space $Fun(g_{\mathcal{P}})$ we tensor by.  Then
\[
Q = \dd_{\mr{DR}\ A} + \overline{\partial}_{\overline{\phi}} + \iota_{V(\phi)} + \iota_{[\phi,\overline{\phi}]} \partial_{\overline{\phi}}
\]
on the space of $g_{\mathcal{P}}$-invariant forms, $Q^{2} = 0$ since $Q$ is just the Lie derivative associated to transformation along $\phi$.

\subsection{Finite-dimensional model}

Suppose we are in a similar finite dimensional situation.
\begin{itemize}
  \item Suppose $G$ acts on a Riemannian manifold $(N,g)$;
  \item Suppose $f \in (\Omega^{\bullet} (N) \otimes \ms{Fun}(\ms{Lie}(G)))^{G}$;
        \item If $G$ acts freely, then there is a well-defined quotient by $G$. If $f$ is equivariantly closed,it  should correspond go to $\phi \in \Omega^{\bullet}(N,G)$.
\end{itemize}

\begin{rmk}
  Let $\pi : N \to N/G$. If $\phi$ is closed so that $\iota_{V} \pi^{\star}\phi = 0$ for all $\phi \in \mathfrak{g}$, with $\phi \in \Omega^{\bullet}(N/G)$.

    Further, characteristic classes of our bundle $N \to N/G$ are represented by invariant polynomials on our Lie algebra, $(S^{\bullet} \mathfrak{g}^{\vee})^{G}$. The functions $\phi$ should somehow correespond to these, `it looks too good to be a coincidence'. This consideration hits on the fact that we likely want to use a metric when we make this map.
\end{rmk}
Let's do what we already did a previously. Using the metric, we can build a one-form:
\[
g(V(\Phi),\bullet) := \theta \in \Omega^{1}(N).
\]
It has the following useful property: if $Df =0$, we can multiply
\[
f \to f e^{D\theta} = f(t) e^{-g(V(\phi), V(\overline{\phi})) + \mathcal{O}_{n, \overline{\phi}} + \dd_{g} \mathcal{O}_{\overline{\phi}} }.
\]
This will \emph{not} change the cohomology class, but it will change the representative.  For example,
\[
f \to \phi
\]
is given by
\[
\int_{\mathfrak{g}} f(t) e^{D\theta}.
\]
We can write this as
\[
\ \frac{f(\phi)}{\prod_{a}(\phi^{b}f_{ab} + \dots)}\overline{\omega}_{1} \wedge \dots \wedge \overline{\omega}_{\dim G} = \int_{\mathfrak{g}} f(t) e^{D\theta}.
\]
In order to not worry about contours, we treat $\phi, \overline{\phi}$ as complex conjugates. There is a canonical measure on $\ms{Lie}(G)$,
\[
[\dd\phi/\on{Vol}(G)],
\]
which is just the Haar measure which we have for free. Insert this into the integral $\int fe^{D\theta}$ as an additional term by which we integrate out $\phi$.

A procedure like this is `universal' in the sense that it produces a differential form on $N$. Because all the integrations I've done commute with $D$, this kind of averaging results in a form which is still equivariantly closed.
The secret of Yang-Mills theory is that it does exactly that: it takes something simple on the space of connections and does some sort of integration like that to produce a form on the space of connections.

But there's a second piece, the \emph{anti self-dual condition}, which we did not consider.
If we have $M = s^{-1}(0)/G$ for some section $s$ of a vector bundle $E \to N$, how do we restrict differential forms to the vanishing locus of this section? Naively the idea is to just send
\[
f \to \delta(s) f
\]
where $\delta$ is just a $\delta$ function or $\delta$ form.  To do this properly we need to add on a \emph{Koszul complex} into our complex of forms which effectively has $s$ as a differential.

At the level of mysterious formulae, this is achieved in the following way. Extra complexes imply we have extra fields; introduce the field $\chi$ which for our purposes will be the self dual form $\chi \in \Omega^{2,+}(M) \otimes \prod \ad\mathcal{P}$. This $\chi$ represents the $(-1)$ term in the Koszul complex, so that the Koszul differential of $\chi$ is $\delta \chi := F_{A}^{+} = s$

Then we have a vector bundle $E_{P}$ over $\mathcal{A}_{\mathcal{P}}$ with fibers
\[
F = \Gamma\, (\Omega^{2,_+}(M) \otimes \ad \mathcal{P}).
\]
Naively $Q^{2}$ is $d_{A}^{+}$, so not zero on the nose. This is fixed in the following way: we define \emph{another} field, $\chi \in \Omega^{2,+}(M) \otimes \prod \ad\mathcal{P}$, and a further field $H \in \Omega^{2,+}(M) \otimes \ad\mathcal{P}$, so that
\begin{align}
    Q\chi &= H
  \\
    QH &= [\chi,\phi]
\end{align}
Remember also that we had fields
\begin{itemize}
  \item $A$ a connection on $\mathcal{P}$;
  \item $\overline{\phi} \in \Omega^{0} \otimes \ad \mathcal{P}$;
  \item $\psi \in \Omega^{4}(M) \otimes \ad \mathcal{P}$;
  \item $\eta \in \Omega^{0} \otimes \ad\mathcal{P}$;
        \item $\phi \in \Omega^{0} \otimes \ad\mathcal{P}$.
\end{itemize}
which is a minimal list of fields for Yang-Mills theory. $Q$ acted on these fields by

\begin{itemize}
  \item $Q\chi = H$;
  \item $QH = [\chi, \phi]$;
  \item $QA = \psi$;
  \item $Q\psi = d_{A} \phi$;
  \item $Q\overline{\phi} = \eta$;
  \item $Q\eta = [\phi,\overline{\phi}]$;
        \item $Q\phi = 0$.
\end{itemize}
Note that for all but $\phi$ there is a canonical pairing between fields just by their type, called \emph{Berezin measure}. This gives us a measure:
\[
[DA D\psi][D\chi DH] [D\overline{\phi} D\eta] \qty[\frac{D\phi}{\mr{vol}(G)}],
\]
where we have only `cheated' in the last term, whose definition leaves something to be said in the fully infinite dimensional setting.

Consider the functional
\[
\exp Q \int_{M} \Tr X(F_{A}^{+} - \frac{1}{2} c^{2} H) + \Tr \psi \wedge \star \dd_{A} \overline{\phi} + \mr{vol}_{g}\Tr \eta[\phi,\overline{\phi}].
\]
We got these terms from looking at our metric $g= \int \Tr(\delta A \wedge \star \delta A)$ and applying $Q$ to create the most general supersymmetric or $Q$-closed functional.
This is the same as
\[
\exp  \int \Tr H F_{A}^{+} - c^{2} \int \Tr H\wedge \star H- \int \Tr \dd_{A} \phi \wedge \star \dd_{A} \overline{\phi} - \int \Tr[\phi, \overline{\phi}]^{c} \mr{vol}_{g} + \Tr(\chi \dd_{A}^{+} \psi + \dots).
\]
Since $H$ is quadratic, we could integrate it out.

Also integrating $\chi,\eta$ out represents the Atiyah-Hitchin-Singer complex, because it imposes the conditions $\dd_{A}^{+}\psi = 0, \dd_{A}^{\star} \psi = 0$, equivalent to $\psi \in \ker \dd_{A}^{+}$, $\psi$ is orthogonal to $\Im(\dd_{A})$, which gives us the first cohomology of the AHS complex.

The zero modes of $\chi$ correspond to the second cohomology of the AHS complex, and the zero modes of $\eta$ correspond to the first cohomology. It's something to keep in mind because zero modes are significant because they drop out of the exponential and so observables `have to provide the missing zero modes' for the fermionic integrations to be nonzero.
Also, we could determine that the functional implies that $\phi$ is `the curvature of the universal bundle evaluated at the point x', but we would need to talk more about this later on.

\subsection{When M is not compact}

There is one important change to the whole story when $M$ is not compact; in other words, if the metric is such that there is some `end' which is infinitely far away, the story has to be modified. We don't need to integrate over all fields, rather we want to restrict to those where the curvature goes to zero at $\infty$. Naively that's because the integral would be divergent if the curvature was not zero.

Since we want $\phi$ to approach a constant $a$, $\overline{\phi} \to \overline{a}$, such that $[a, \overline{a}]= 0$, we only use those gauge transformations which approach unity sufficiently fast at infinity. The rate at which they go to 1 is fixed by the requirement that $||\dd\phi||^{2} < \infty$. The result of that is that a finite dimensional group of those gauge transformations preserved now acts as a global symmetry! Instead of integrals over the moduli space of instantons of closed differential forms, one passes to equivariant differential forms for the moduli space of \textbf{framed} instantons, where framed means the vanishing condition above.

\chapter{Gauge-theoretic instanton counting}

The difference between mathematicians and physicists is that mathematicians like closed $4$-manifolds while physicists like $\R^4$. For mathematical precision, we will let $M^4$ be a compact Riemannian manifold and $\mc{P} \to M$ be a principal $G$-bundle for some compact Lie group $G$. These are classsified by the second Chern class
\[ k \coloneqq c_2(\mc{P}) \in H^4(M^4, \Z), \]
which physicists call the \textit{instanton charge}. We are looking for connections $\nabla = \dd + A$ whose curvature satisfies
\[ F_{\nabla} = - \star F_{\nabla}, \]
which are called \textit{instantons}. This equation in fact only depends on the conformal class of the metric.

Being anti-self-dual implies that the connection gives a minimum of the Yang-Mills action
\[ S = \int_{M^4} \Tr F_{\nabla} \wedge \star F_{\nabla}. \]
If $M^4$ is a complex surface and the metric $g$ is Hermitian, then $F_{\nabla}^{0,2} = 0$, which is equivalent to requiring that that $\ol{\partial}_A^2 = 0$. Therefore, any representation $E$ of $G$ gives rise to a holomorphic vector bundle $\mc{E}$ on $M^4$. This will come into play later when we cheat by replacing instantons with torsion-free sheaves.

The \textit{moduli space} of such instantons is
\[ \mc{M}_{\mc{P}} = \qty{\nabla \mid F_{\nabla}^+ = 0} / \mc{G}_{\mc{P}}, \]
where $\mc{G}_{\mc{P}}$ is the group of sections of the associated $\Ad_{\mc{G}}$-bundle acting by
\[ A \mapsto g^{-1} A g + g^{-1} \dd g. \]
For generic metrics $g$, this is a manifold of dimension
\[ 4 h^{\vee} k - \frac{\chi + \sigma}{2} \dim G, \]
where $h^{\vee}$ is the dual Coxeter number. We should note that even though $\mc{M}_k$ as a space depends only on the conformal class of the metric, the metric depends on $g$ itself. It also lives inside the space $\mc{A}_{\mc{P}}/\mc{G}_{\mc{P}}$ of all connections on $\mc{P}$. Our aim is to understand the infinite-dimensional integral
\[ \sum_{[\mc{P}], c_2(\mc{P}) = -k} e^{- \vartheta k} \int_{\mc{A}_{\mc{P}}/\mc{G}_{\mc{P}}} [\mc{D}A] e^{-\frac{1}{4g^2} S_{\mr{YM}}}. \]
Here, $\vartheta$ is valued on $S^1$. In the $g^2 \to 0$ limit, the asymptotics of the integral should become
\[ \sum_k q^k \int_{\mc{M}_k} \dd{\mu} (1 + O(g)), \]
where $q = e^{-\frac{8\pi^2}{g^2} + i\vartheta}$ and the measure is obtained by regularity and determinants of $\delta S_{\mr{YM}}$.

Instead of trying to understand this very complicated measure $\mu$, we can try to study simpler but still interesting integrals. For examples, Donaldson invariants are defined as integrals over $\mc{M}_k$ of cohomology classes associated to $2$-cycles $\Sigma_i$ and $0$-cycles $p$ on $M^4$. The problem is that $\mc{M}_k$ is non-compact because of the conformal invariance of the equations leading to delta-function solutions. Uhlenbeck discovered that via a complicated system of gauge transformations, these points can be filled in by ideal solutions, so there is the \textit{Uhlenbeck compactification}
\begin{align*}
    \ol{\mc{M}}_k &= \qty{(\nabla, x_1, \ldots, x_{\ell}) \mid c_2(\wt{\mc{P}}) - c_2(\mc{P}) = \ell} \\
    &= \mc{M}_k \cup \mc{M}_{k-1} \times M^4 \cup \mc{M}_{k-2} \times \on{Sym}^2 M^4 \cup  \cdots \cup \mc{M}_0 \times \on{Sym}^k M^4.
\end{align*}

\section{The case of $\R^4$}

This is the case that physicists are worried about. The problem is of course that $\R^4$ is non-compact, so we can view either $\R^4 = S^4 \setminus \infty$ or $\R^4 = \C^2 = \P^2 \setminus \P^1_{\infty}$. There are already interesting solutions in this case, but because our metric is singular at infinity, we will consider the moduli space
\[ \mc{M}_k^{\mr{framed}} = \qty{\nabla \mid F_{\nabla}^+ = 0} / \mc{G}_{\mc{P}}^{\infty}, \]
where $\mc{G}_{\mc{P}}^{\infty} \subset \mc{G}_{\mc{P}}$ is the set of those elements satisfying $g(x) \to 1$ as $x \to \infty$. This is a hyperk\"ahler manifold of dimension $4kN$ when $G = SU(N)$.

It is well-known that this (or a slight modification $\wt{\mc{M}}_k^{\mr{framed}})$ is a Nakajima quiver variety corresponding to the quiver data
\begin{equation*}
\begin{tikzcd}
    k \ar[shift left=1]{d}{J} \arrow[out=0,in=30,loop,swap,"B_2"] \arrow[out=90,in=120,loop,swap,"B_1"]\\
    N \ar[shift left=1]{u}{I}
\end{tikzcd}
\end{equation*}
Here, we need to modify the instanton equations to
\begin{align*}
    [B_1, B_2] + IJ &= 0 \\
    [B_1, B_1^{\dag}] + [B_2, B_2^{\dag}] + I I^{\dag} - J^{\dag} J &= \zeta \cdot 1_k,
\end{align*}
which in algebraic geometry corresponds to changing the value of the moment map.

This moduli space has an instanton interpretation if we replace $\R^4$ by a noncommutative $\R^4_{\zeta}$, which has coordinates $z_1, z_2, \ol{z}_1, \ol{z}_2$ with commutators
\[ [z_1, \ol{z}_1] = -\frac{\zeta}{2} = [z_2, \ol{z}_2]. \]
In the $N=2$ case, we have $\mc{M}_2^{\mr{framed}} = \R^4 \times (\R^4 \setminus 0)/\Z_2$ while $\ol{\mc{M}}_k^{\mr{framed}} = \R^4 \times \R^4/\Z_2$ and $\wt{\mc{M}}_k^{\mr{framed}} = \R^4 \times T^* S^2$.

There is also an action of $U(2) \times SU(N)$, where for elements
\[ a = \mqty(\dmat{a_1,\ddots,a_N}) \in \on{\ms{Lie}} U(N) \otimes \C \]
and
\[\ep = \mqty(0 & \ep_1 \\ -\ep_1 & 0 \\ & & 0 & \ep_2 \\ & & -\ep_2 & 0) \in \on{\ms{Lie}} U(2) \otimes \C, \]
the corresponding vector field is denoted by $V(a,\ep)$. Given a choice of
\[ \lambda_{\ol{a}, \ol{\ep}} = G_{\wt{\mc{M}}}(V(\ol{a}, \ol{\ep}), -), \]
the partition function is
\[ Z_k = \int_{\wt{\mc{M}}_k^{\mr{framed}}} \exp \qty(-G_{\wt{\mc{M}}}(V(a,\ep), V(\ol{a}, \ol{\ep}))) \frac{1}{(2kN)!} (\dd \lambda_{\ol{a}, \ol{\ep}})^{2kN}. \]
This is not the most general formulation, but it is a distilled version of Yang-Mills with some supersymmetry. It turns out that $Z_k$ is actually a rational function of degree $-2kN$ in $a$ and $\ep$.

One generalization of this is
\[ Z(a,\ep,\Lambda) = Z^{\mr{part}} \times \sum_{k=0}^{\infty} \Lambda^{2kN} Z_k(a,e), \]
where
\[ Z^{\mr{part}}(a,\ep) = \prod_{i \neq j} \Gamma_2(a_i-a_j; \ep_1, \ep_2). \]
Here, $\Gamma_2$ has the asymptotics
\[ \Gamma_2(x; \ep_1, \ep_2) \sim \prod_{n,m \geq 1}(x+\ep_1 n + \ep_2 m) \]
and solves the equation
\[ \frac{\Gamma_2(x+\ep_1) \Gamma_2(x+\ep_2)}{\Gamma_2(x) \Gamma_2(x+\ep_1+\ep_2)} = x. \]
In the limit as $\ep_1, \ep_2 \to 0$, we expect the asymptotics
\[ \exp\qty(\frac{1}{\ep_1 \ep_2} \mc{F}(a,\Lambda) + \cdots), \]
so the question now is to evaluate $\mc{F}(a, \Lambda)$. In the case when $G = SU(N)$, we can do it by localization.

A standard computation tells us that
\[ Z_k(a,\ep) = \sum_{\substack{(\lambda^{(1)}, \ldots, \lambda^{(k)}) \\ \abs{\lambda^{(1)}} + \cdots + \abs{\lambda^{(k)}} = k}} \frac{1}{\prod_{i,j=1}^N \qty( \prod_{\square \in \lambda^{(i)}} (a_i-a_j + f(\ep_1, \ep_2))) \qty(\prod_{\blacksquare \in \lambda^{(j)}} ( a_j - a_i + g(\ep_1, \ep_2) ))}, \]
where $f$ and $g$ are defined using the (relative) arms and legs of the two Young diagrams. For example, when $i=j$, we end up with $f = \ep_1(\mr{arm}_{\square}+1)-\ep_2 \mr{leg}_{\square}$ and $g = -\ep_1 \mr{arm}_{\square} + \ep_2(\mr{leg}_{\square}+1)$.

Treating $Z_k$ as a probability measure on the set of Young diagrams, we obtain the observables
\[ Y(x)[\lambda^{(1)}, \ldots, \lambda^{(N)}] = \prod_{\alpha =1}^N \frac{\prod_{\square \in \partial_+ \lambda^{(\alpha)}}(x-a_{\alpha}-c_{\square})}{\prod_{\blacksquare \in \partial_- \lambda^{(\alpha)}}(x-a_{\alpha}-c_{\blacksquare}-\ep_1-\ep_2)}. \]
This function knows essentially everything about the shape of the diagrams, and its vacuum expectation is
\[ \ev{Y(x)} = \frac{1}{Z} \sum_{\vec{\lambda}} Y(x)[\vec{\lambda}] \mu_{\vec{\lambda}}(a,\ep) \Lambda^{2N\abs{\vec{\lambda}}}. \]
It satisfies the property that
\[ \ev{Y(x+\ep_1 + \ep_2) + \frac{\Lambda^{2N}}{Y(x)}} \]
has no poles in $x$ and is in fact a polynomial of degree $N$. Physically, this is interpreted as an interaction between two neighboring instanton sectors.

If we send $\ep_1, \ep_2 \to 0$, then $\ev{Y(x)} = \mc{Y}(x)$ we then obtain the algebraic equation
\[ \mc{Y}(x) + \frac{\Lambda^{2N}}{\mc{Y}(x)} = T(x), \]
where the coefficients of $T(x)$ are defined by
\[ \oint_{A_i} x \frac{\dd Y}{Y} \sim a_i. \]
This is a hyperelliptic curve.

\section{Mathematical realization}

From now on, we will set $G = U(N)$ and consider the moduli space
\[ \ol{\mc{M}}_k^{\mr{framed}}(N) = \qty{(B_1, B_2, I, J) \mid B_1, B_2 \in \End(\C^k), I \colon \C^N \to \C^k, J \colon \C^k \to \C^N, \text{ADHM}} / U(k). \]
Recall that this is a Nakajima quiver variety corresponding to the Jordan quiver. The ADHM equations are
\begin{align*}
    \vec{\mu} &= (\mu_{\R}, \mu_{\C}, \ol{\mu}_{\C}) = 0 \\
    \mu_{\R} &= [B_1, B_1^{\dag}] + [B_2, B_2^{\dag}] + II^{\dag} - JJ^{\dag} - \zeta_{\R} \mathbf{1}_k \\
    \mu_{\C} &= [B_1, B_2] + IJ - \zeta_{\C} \cdot \mathbf{1}_k.
\end{align*}
Here, $\vec{\zeta} = (\zeta_{\R}, \zeta_{\C}, \ol{\zeta}_{\C}) \in \R^3 = \R \oplus \C$. It can in fact be rotated to $(\zeta, 0, 0)$ by $SO(3)$, so we can assume $\zeta > 0$. Here, the action of $g \in U(k)$ is given by
\[ g(B_1, B_2, I, J) = (g^{-1}B_1 g, g^{-1} B_2 g, g^{-1} I, Jg). \]
Then $\ol{\mc{M}}_k^{\mr{framed}}(N)$ has an action of $U(N)$ given by
\[ h(B_1, B_2, I, J) = (B_1, B_2, Ih, h^{-1}J), \]
and these are symmetries. There is also a $U(2)$-action given by
\[ U(2) \ni \mqty(a & b \\ -\ol{b} & \ol{a}) e^{i\gamma} (B_1, B_2, I, J) = ((aB_11 + bB_2)e^{i\gamma}, (-\ol{b} B_1 + \ol{a} B_2)e^{i\gamma}, I, J e^{2i\gamma}). \]
In fact, if we only consider $\mu_{\C}$, it in fact has $GL(2)$-symmetry.

We are interested in the fixed points of our $U(2) \times U(N)$ action on $\ol{\mc{M}}_k^{\mr{framed}}(N)$. To be a fixed point of $(u,h) \in U(2) \times U(N)$, the tuple $\beta = (B_1, B_2, I, J)$ must obey $(u,h) \beta = g_{u,h} \beta$ for some $g_{u,h} \in U(k)$ depending on $u,h$. It is in fact enough to consider the maximal torus of $U(2) \times U(N)$, so we may assume that
\[ u = \mqty(\dmat{e^{i\ep_1}, e^{i\ep_2}}), \qquad h = \mqty(\dmat{e^{ia_1},\ddots,e^{ia_N}}). \]
Now we obtain
\[ (u \times h)(B_1, B_2, I, J) = (e^{i\ep_1} B_1, e^{i\ep_2} B_2, Ih, h^{-1} e^{i(\ep_1+\ep_2)}J). \]
To make this compatible with the gauge transformations, the assignment
\[ (e^{i\ep}, e^{ia}) \mapsto g_{\ep, a} \]
is a group homomorphism, giving a decomposition
\[ \C^k = \bigoplus_{\lambda} K_{\lambda}, \qquad g_{\ep, a}|_{K_{\lambda}} = e^{i\lambda(\ep, a)}, \]
where $\lambda(\ep, a)$ is a $\Z$-linear function.

We will now use the Kempf-Ness theorem and solve $\mu_{\C} = 0$ and then divide by $GL(k)$. Given a solution, we want $g \in GL(k)$ such that $(g^{-1} B_1 g, g^{-1} B_2 g, g^{-1} I, Jg)$ solves $\mu_{\R} = 0$. This is not always possible, so we need \textit{stable} solutions in the sense of geometric invariant theory.

\begin{defn}
    The tuple $(B_1, B_2, I, J)$ is \textit{stable} if and only if for all subspaces $K' \subset \C^k$ such that $B_1(K') \subset K'$, $B_2(K') \subset K'$, and $I(\C^N) \subset K'$, then $K' = K$. Equivalently, $\C\ev{B_1, B_2} I(\C^N) = \C^k$.
\end{defn}

\begin{thm}
    We have an isomorphism
    \[ \vec{\mu}^{-1}(0) / U(k) = \mu_{\C}^{-1}(0)^s / GL(k). \]
\end{thm}

Using this, write $\C^N = \bigoplus N_{\alpha}$ as the standard decomposition into eigenspaces of the standard maximal torus of $U(N)$. Then let $I_{\alpha} = I(N_{\alpha})$. Our equations then become
\begin{align*}
    g^{-1}_{\ep, a} I_{\alpha} &= e^{i a_{\alpha}} I_{\alpha} \\
    g^{-1}_{\ep, a} B_1 &= e^{i \ep_1} B_1 g_{\ep, a}^{-1} \\
    g^{-1}_{\ep, a} B_2 &= e^{i \ep_2} B_2 g_{\ep, a}^{-1} .
\end{align*}
Then $\C\ev{B_1, B_2} I_{\alpha} = K_{\alpha}$ is spanned by eigenvectors of $g^{-1}_{\ep, a}$ with eigenvalues $e^{i a_{\alpha}} e^{i(p \ep_1 + q \ep_2)}$ for integers $p,q \geq 0$. Note if $\psi \in \C^k$ satisfies $g^{-1}_{\ep, a} \psi = \lambda \psi$, we have
\[ g_{\ep, a}^{-1}(B_1 \psi) = e^{i \ep_1} B_1 (g_{\ep, a}^{-1} \psi) = (\lambda e^{i \ep_1})(B_1 \psi). \]
We are still interested in the fate of $J$, and in fact we will see that $J = 0$, which will imply that $[B_1, B_2] = 0$. 

It is sufficient to replace $\C^k$ by $K_{\alpha}$ and $\C^N$ by $\C$, so we reduce to the case of $N=1$. This gives us
\[ 0 = [B_1, B_2] + IJ, \]
so taking the trace gives us $\Tr IJ = 0$. Therefore, $J(I) = 0$, viewing $I \in \C^k$. Next, we see that 
\[ J(B_1 I) = \Tr B_1 IJ = \Tr (B_1 [B_2, B_1]) = \Tr (B_2[B_1, B_1]) = 0. \]
Similarly, $J(B_2 I) = 0$. This is also true for any powers of $B_1, B_2$, and in fact for any polynomial of $B_1, B_2$, so using the fact that $I$ is a cyclic vector for $B_1, B_2$, we see that $J=0$.

We now see that to each $K_{\alpha}$, we obtain an ideal $\mc{I}_{\alpha} \subset \C[x_1, x_2]$ to be the annihilator of $I_{\alpha}$ under the assignment $x_i \mapsto B_i$. Thus we can identify
\[ K_{\alpha} \cong \C[x_1, x_2] / \mc{I}_{\alpha}. \]
By a standard argument, the action of $\C^{\times} \times \C^{\times}$ tells us that $\mc{I}_{\alpha}$ must be a monomial ideal, and this corresponds to a partition of $k_{\alpha} \coloneqq \dim K_{\alpha}$, or equivalently a Young diagram of size $k_{\alpha}$. The only thing we know is that
\[ \sum_{\alpha = 1}^N k_{\alpha} = k, \]
so some of the $k_{\alpha}$ could be zero. We will call the $\alpha$-th partition $\lambda^{(\alpha)}$ and its components
\[ \lambda^{(\alpha)} = \qty(\lambda_1^{(\alpha)} \geq \cdots \geq \lambda_{\ell_{\alpha}}^{(\alpha)}). \]
We have therefore proven the following theorem:

\begin{thm}
    There is a bijection 
    \[ \ol{\mc{M}}_k^{\mr{framed}}(N)^{\T} = \bigsqcup_{\substack{\lambda^{(1)}, \ldots, \lambda^{(N)} \\ \sum_{\alpha=1}^N \abs{\lambda^{(\alpha)}} = k}} \mr{pt}. \]
\end{thm}

\begin{exm}
    When $k=1$, we have
    \begin{align*}
        \ol{\mc{M}}_1^{\mr{framed}}(N) &= \C^2 \times T^* \P^{n-1}
    \end{align*}
    and the fixed points are given by $\qty{0} \times (0,\ldots,1,\ldots,0)$.
\end{exm}

\section{Explicit localization computation}%
\label{sec:Explicit localization computation}

In order to compute our integrals using equivariant localization, we need to compute
\[ T_{\ul{\lambda}} \ol{\mc{M}}_k^{\mr{framed}}(N) = \bigoplus_w \C_w \]
as a representation of the maximal torus $\T$. Its equivariant Chern character (also the character of the representation) is given by
\begin{align*}
    \on{Ch}(T_{\ul{\lambda}} \ol{\mc{M}}_k^{\mr{framed}}(N)) &= \sum_w e^{\sqrt{-1} w(\ep_1, \ep_2, a)} \in \Z[q_1, q_2, h_1, \ldots, h_N], 
\end{align*}
where $h_{\alpha} = e^{\sqrt{-1} a_{\alpha}}$ and $q_i = e^{\sqrt{-1} \ep_i}$. Then any
\[ (\delta B_1, \delta B_2, \delta I, \delta J) \in T_{\ul{\lambda}} \ol{\mc{M}}_k^{\mr{framed}}(N) \]
must satisfy
\[ [B_1, \delta B_2] + [\delta B_1, B_2] + I \delta J = 0 \]
up to
\[ \delta B_1 = [B_1, \phi], \qquad \delta B_2 = [B_2, \phi], \qquad \delta I = - \phi I, \qquad \delta J = J \phi = 0. \]
Therefore, $T_{\ul{\lambda}} \ol{\mc{M}}_k^{\mr{framed}}(N)$ is the $H^1$ of a complex
\[ \mc{T} = \End(\C^k) \to \End(\C^k) \oplus \End(\C^k) \oplus \Hom(\C^N, \C^k) \oplus \Hom(\C^k, \C^N) \to \End(\C^k) \otimes \Lambda^2 \C^2, \]
where the two arrows are given by
\begin{align*}
    d_1(\phi) &= ([B_1, \phi], [B_2, \phi], -\phi I, J \phi) \\
    d_2(\delta B_1, \delta B_2, \delta I, \delta J) &= [B_1, \delta B_2] + [\delta B_1, B_2] + I \delta J + \delta I J.
\end{align*}

\begin{lem}
    We have $H^0(\mc{T}) = H^2(\mc{T}) = 0$.
\end{lem}

\begin{proof}
    Suppose $\gamma \in \ker d_1$. Then 
    \[ \gamma (\C[B_1, B_2] I(\C^N)) = 0. \]
    But this space is all of $\C^k$, so $\gamma = 0$.

    Now suppose there exists $\wt{\gamma} \in \End \C^k$ such that
    \[ \Tr (\wt{\gamma} d_2(\beta_1, \beta_2, i,j)) = 0 \]
    for all $(\beta_1, \beta_2, i, j)$. But this implies that
    \[ \Tr (\wt{\gamma} I)j + \Tr (J \wt{\gamma}) i + \Tr [B_1, \wt{\gamma}] \beta_2 + \Tr [B_2, \wt{\gamma}] \beta_1 = 0, \]
    which is exactly $\dd_1(\wt{\gamma}) = 0$, which implies $\wt{\gamma} = 0$.
\end{proof}

To compute the character, we note that
\begin{align*}
    \chi_{\ul{\lambda}}(q_1, q_2; w_1, \ldots, w_n) &= \on{Ch} H^1(\mc{T}) \\
    &= - \chi_{\T}(\mc{T}) \\
    &= - \sum (-1)^i \on{Ch} H^i(\mc{T}) \\
    &= - \sum (-1)^i \on{Ch} C^i(\mc{T}).
\end{align*}
We also know that
\begin{align*}
    \N \coloneqq \on{Ch} \C^N &= \sum_{\alpha = 1}^n w_{\alpha} \\
    \mathbb{K} \coloneqq \on{Ch} \C^k &= \sum_{\alpha=1}^n w_{\alpha} K_{\alpha}(q_1, q_2),
\end{align*}
where
\[ K_{\alpha}(q_1, q_2) = \sum_{(i,j) \in \lambda^{(\alpha)}} q_1^{i-1} q_2^{j-1}. \]
This also gives us
\begin{align*}
    \N^* &= \sum_{\alpha} w_{\alpha}^{-1} \\
    \mathbb{K}^* &= \sum_{\alpha} w_{\alpha}^{-1} K_{\alpha}(q_1^{-1}, q_2^{-1}).
\end{align*}

We are now able to compute
\[ \on{Ch} \End(\C^k) = \mathbb{K} \mathbb{K}^*. \]
It is also easy to see that
\[ \on{Ch} C^1(\mc{T}) = \mathbb{K} \mathbb{K}^* (q_1 + q_2) + \N \mathbb{K}^* + \mathbb{K} \N^* q_1 q_2 \]
and
\[ \on{Ch} C^2(\mc{T}) = \mathbb{K} \mathbb{K}^* q_1 q_2, \]
so
\[ \chi_{\ul{\lambda}}(q_1, q_2l w_1, \ldots, w_n) = \N \mathbb{K}^* + \N^* \mathbb{K} q_1 q_2 - (1-q_1)(1-q_2) \mathbb{K} \K^*. \]
This is not obviously a pure character, but taking its dual gives us
\begin{align*}
    \chi_{\ul{\lambda}}^* &= \N^* \K + \N \K^* q_1^{-1} q_2^{-1} - q_1^{-1}q_2^{-1}(1-q_1)(1-q_2) \K \K^* \\
    &= q_1^{-1} q_2^{-1} \chi_{\ul{\lambda}}.
\end{align*}
Rewriting
\[ \chi_{\ul{\lambda}} = \sum_{\alpha,\beta=1}^N w_{\alpha}w_{\beta}^{-1 \mc{T}_{\alpha\beta}}(q_1, q_2) \]
and noting that
\[ \mc{T}_{\alpha\beta} = q_1q_2 \mc{T}_{\beta \alpha}^*, \]
we can compute
\begin{align*}
    \mc{T}_{\alpha\beta}(q_1, q_2) &= \mc{T}_{\alpha\beta}^{q_1>0} + \mc{T}_{\alpha\beta}^{q_1 \leq 0} \\
    &= K_{\beta}^* + K_{\alpha} q_1q_2 - (1-q_1)(1-q_2)K_{\alpha}K_{\beta}^*. 
\end{align*}
We then see that
\begin{align*}
    (1-q_2) K_{\alpha} &= \sum_{i=1}^{\ell_{\lambda^{(\alpha)}}} q_1^{i-1}(1-q_2^{\lambda_i^{\alpha}}) \\
    (1-q_1) K_{\beta}^* &= - q_1 \sum_{j=1}^{\lambda_1^{(\beta)}} q_2^{1-j} (1-q_1^{-\lambda_j^{(\beta) t}}).
\end{align*}
Taking only the nonpositive part, we obtain
\begin{align*}
    \mc{T}_{\alpha\beta}^{q_1^{\leq 0}} &= K_{\beta}^* - ((1-q_1)(1-q_2) K_{\alpha} K_{\beta}^*)^{q_1^{\leq 0}} \\
    &= \sum_{(i,j) \in \lambda^{(\beta)}} q_1^{i-\lambda_j^{(\beta) t}} q_2^{\lambda_i^{(\alpha)}+1-j} \\
    &= \sum_{\square \in \lambda^{(\beta)}} q_1^{-\ell_{\square}^{(\beta)}} q_2^{1+a_{\square}^{(\alpha)}}.
\end{align*}
Putting it all together, we obtain
\[ \chi_{\ul{\lambda}} = \sum_{\alpha,\beta=1}^N w_{\alpha}w_{\beta}^{-1} \qty(\sum_{\square \in \lambda^{(\beta)}} q_1^{-\ell_{\square}^{(\beta)}} q_2^{1+a_{\square}^{(\alpha)}} + \sum_{\square \in \lambda^{(\alpha)}} q_1^{\ell_{\square}^{( \alpha )}+1} q_2^{-a_{\square}^{(\beta)}}). \]

We are now able to compute the integrals we wanted as
\begin{align*}
    &\int_{\ol{\mc{M}}_k(N)} \Omega^{(4Nk)} =\\
    &=\sum_{\abs{\ul{\lambda}} = k} \frac{\Omega^{(0)}|_{\ul{\lambda}}}{\displaystyle\prod_{\alpha,\beta=1}^N \prod_{\box \in \lambda^{(\beta)}}(a_{\alpha}-a_{\beta} - \ep_1 \ell_{\square}^{(\beta)}+ \ep_2 (1+a_{\square}^{(\alpha)})) \prod_{\square \in \lambda^{(\alpha)}} (a_{\alpha}-a_{\beta}+\ep_1 (\ell_{\square}^{(\alpha)}+1) - \ep_2 a_{\square}^{(\beta)})}.
\end{align*}

We now consider the \textit{tautological complex}
\[ \mc{S}_z = [0 \to \C^k \xrightarrow{\delta_1} \C^k \otimes \C^2 \oplus \C^N \xrightarrow{\delta_2} \C^k \to 0], \]
which should be thought of as a holomorphic square root of the tangent complex. This is given by the formulae
\begin{align*}
    \delta_1(\kappa) &= ((B_1-z_1) \kappa, (B_2 - z_2) \kappa, J \kappa) \\
    \delta_2(\kappa_1, \kappa_2, \nu) &= (B_1 - z_1) \kappa_2 - (B_2 - z_2) \kappa_1 + I \nu. 
\end{align*}
In fact, the ADHM equation is equivalent to this being a complex for all $z \in \C^2$. We see what happens when $z=0$. In this case, we see that if $\wt{\kappa} \in \on{coker} \delta_2$, then
\[ \wt{\kappa}^{\dag}(B_1 \kappa_2- B_2 \kappa_1 + I \nu) = 0 \]
for all $(\kappa_1, \kappa_2, \nu)$, so it annihilates all of $B_1, B_2, I$, and thus $\wt{\kappa} = 0$. Therefore
\[ \mc{H}^1(\mc{S}_0) = 0. \]

However, in general, this complex has both $H^0$ and $H^1$. We will compute these over $\ul{\lambda}$. We can explicitly see that
\[ H^0(\mc{S}_0) \]
is the space of corners. We will compute the character of the $H^1$ using the equation
\begin{align*}
    \on{Ch} H^0(\mc{S}_0) - \on{Ch} H^1(\mc{S}_0) &= \on{Ch} C^0(\mc{S}_0) - \on{Ch} C^1(\mc{S}_0) + \on{Ch} C^2(\mc{S}_0) \\
    &= \sum_{\alpha} w_{\alpha}(K_{\alpha} - K_{\alpha}(q_1 + q_2) - 1 + K_{\alpha} q_1q_2) \\
    &= -\sum_{\alpha} w_{\alpha}(1-(1-q_1)(1-q_2) K_{\alpha}).
\end{align*}
If we note that
\[ \on{Ch} \C[q_1, q_2] = \frac{1}{(1-q_1)(1-q_2)}, \]
we see that
\[ \frac{1}{(1-q_1)(1-q_2)} - K_{\alpha} = \on{Ch} \mc{I}_{\alpha} \]
is the character of the ideal corresponding to $\lambda^{(\alpha)}$.
Therefore, the coefficient of $w_{\alpha}$ is simply
\[ \sum_{\square \in \partial_+\lambda} q_{\square} - q_{12} \sum_{\square \in \partial_- \lambda} q_{\square}, \]
and the first term yields $\on{Ch} H^1$ while the second term yields $\on{Ch} H^0$.

To see how this works in practice, fix $N=1$. Then $\mc{I}$ is a codimension $k$ ideal in $\C[z_1, z_2]$. We are looking for 
\[ f(z_1, z_2) = z_1 \kappa_2(z_1, z_2) - z_2 \kappa_1(z_1, z_2) + \nu \in \mc{I}, \]
and this is given by
\begin{align*}
    \nu &= f(0,0) \\
    \kappa_1 &=  \frac{f(0,0) - f(0,z_2)}{z_2} \\
    \kappa_2 &= \frac{f(z_1, z_2) - f(0, z_2)}{z_1}.
\end{align*}

\section{$qq$-characters}%
\label{sec:qqcharacters}

\begin{notn}
    From now on, we will denote $K = \C^k$ and $N = \C^n$.
\end{notn}


The main tool for dealing with instanton integrals (equivalently, sums over ensembles of Young diagrams) is the \textit{$qq$-character}. Here, we will have a discrete set $S$, and the densities $\mu_s \in \C$ for $s \in S$ must satisfy
\[ \sum_{s \in S} \mu_s = 1. \]
For example, we can consider $s = (\lambda^{(1)}, \ldots, \lambda^{(N)})$. It is important to note that $\mu_s$ depends on
\[ (a_1, \ldots, a_N, \ep_1, \ep_2; \Lambda), \]
but we can make choices such that all $\mu_s$ are positive real nnumbers, giving us a probability measure.

\begin{defn}
    A function $\mc{O} \colon S \to \C$ is called an \textit{observable} if 
    \[ \sum_{s \in S} \mu_s \mc{O}_s < \infty. \]
\end{defn}

Before we continue, we will consider a tautological bundle over instanton moduli spaces, which we will call $K$. In fact, it literally corresponds to the $\C^k$ in the construction of the Nakajima variety. Then we are interested in quantities
\[ \frac{\sum_{k=0}^{\infty} \Lambda^{2kN} \int_{\ol{\mc{M}}_k(N)} c_i(K) \Xi}{\sum_{k=0}^{\infty} \Lambda^{2kN} \int_{\ol{\mc{M}}_k(N)} \Xi} = \sum_{s \in S} \mu_s(a_1, \ldots, a_N, \ep_1, \ep_2, \Lambda) c_i(K)|_s, \]
where
\[ S = \bigsqcup_{k=0}^{\infty} \ol{\mc{N}}_k(N)^{\T}. \]
We can see that
\begin{align*}
    &\mu_s(\ul{a}, \ul{\ep}, \Lambda) =\\
    &=\frac{1}{Z}\frac{\Lambda^{2k_s N- n_{\Xi}(k)}\Xi|_s}{\displaystyle\prod_{\alpha,\beta=1}^N \prod_{\box \in \lambda^{(\beta)}}(a_{\alpha}-a_{\beta} - \ep_1 \ell_{\square}^{(\beta)}+ \ep_2 (1+a_{\square}^{(\alpha)})) \prod_{\square \in \lambda^{(\alpha)}} (a_{\alpha}-a_{\beta}+\ep_1 (\ell_{\square}^{(\alpha)}+1) - \ep_2 a_{\square}^{(\beta)})},
\end{align*}
where
\begin{align*}
    Z 
    =\sum_s\frac{\Lambda^{2k_s N - n_{\Xi}(k)}\Xi|_s}{\displaystyle\prod_{\alpha,\beta=1}^N \prod_{\box \in \lambda^{(\beta)}}(a_{\alpha}-a_{\beta} - \ep_1 \ell_{\square}^{(\beta)}+ \ep_2 (1+a_{\square}^{(\alpha)})) \prod_{\square \in \lambda^{(\alpha)}} (a_{\alpha}-a_{\beta}+\ep_1 (\ell_{\square}^{(\alpha)}+1) - \ep_2 a_{\square}^{(\beta)})}
\end{align*}
normalizes the sum $\sum_s \mu_s = 1$ and $n_{\Xi}(k)$ si the degree of $\Xi|_s$.

\begin{exms}\leavevmode
    \begin{enumerate}
        \item In \textit{pure super-Yang-Mills theory}, $\Xi = 1$ and $Z$ is homogeneous in $(a_1, \ldots, a_N, \ep_1, \ep_2, \Lambda)$;
        \item In \textit{super-QCD}, we set
            \[ \Xi = \prod_{f=1}^{2N} c(m_f, K) \]
            where $c$ denotes the homogeneized Chern polynomial. Here, the input is given by
            \[ (a_1, \ldots, a_N, \ep_1, \ep_2, m_1, \ldots, m_{2N}, q), \]
            where the $m_f$ are the ``flavors'' and $q$ takes the role of $\Lambda$ previously.\footnote{The $a_{\alpha}$ are called ``colors'' in physics. For example, there are six flavors and three colors of quarks in the Standard Model.}
        \item The \textit{$\mc{N}=2^*$ theory}, or adjoint super-QCD, is defined by
            \[ \Xi = c(m, T^* \ol{\mc{M}}_k(N)). \]
    \end{enumerate}
\end{exms}

\begin{thm}
    The Chern polynomials at the fixed points are given by
    \[ c(m_f, K^*)|_s = \prod_{\alpha=1}^N \prod_{(i,j) \in \lambda^{(\alpha)}} (m_f - (a_{\alpha} + \ep_1(i-1) + \ep_2(j-1))). \]
\end{thm}

\begin{exms}
For example, in super-QCD, we obtain \begin{align*}
    Z =\sum_s\frac{q^{k_s}\displaystyle \prod_{\alpha=1}^N \prod_{f=1}^{2N} \prod_{(i,j)\in \lambda^{(\alpha)}} (m_f - a_{\alpha} - \ep_1(i-1) - \ep_2(j-1))}{\displaystyle\prod_{\alpha,\beta=1}^N \prod_{\box \in \lambda^{(\beta)}}(a_{\alpha}-a_{\beta} - \ep_1 \ell_{\square}^{(\beta)}+ \ep_2 (1+a_{\square}^{(\alpha)})) \prod_{\square \in \lambda^{(\alpha)}} (a_{\alpha}-a_{\beta}+\ep_1 (\ell_{\square}^{(\alpha)}+1) - \ep_2 a_{\square}^{(\beta)})}.
\end{align*}
This series has a finite radius of convergence and is well-defined for $\abs{q} < 1$. It also admits a non-trivial analytic continuation.

In the $\mc{N}=2^*$ theory, we obtain
\begin{align*}
    Z =\sum_s q^{k_s} \prod_{\alpha,\beta=1}^N \prod_{\square \in \lambda^{(\alpha)}} \frac{\ep_1(a_{\square}^{(\alpha)}+1) - \ep_2 \ell_{\square}^{(\beta)}+m}{\ep_1(a_{\square}^{(\alpha)}+1) - \ep_2 \ell_{\square}^{\beta}} \prod_{\square \in \lambda^{(\beta)}} \frac{-\ep_1 a_{\square}^{(\beta)} + \ep_2(\ell_{\square}^{\alpha}+1)+m}{-\ep_1 a_{\square}^{(\beta)} + \ep_2 (\ell_{\square}^{(\alpha)}+1)}.
\end{align*}
\end{exms}

\begin{rmk}
    We should view $q$ as lying on $\ol{\mc{M}}_{0,4}$, and the points $0, 1, \infty$ become places where the curve degenerates.
\end{rmk}

Instead of the tautological bundle, we may also consider the tautological complex
\[ \mc{C} = K \to K \otimes \C^2 \oplus \C^N \to K \otimes \Lambda^2 \C^2, \]
which defines some $K$-theory class. Here, we set $K$ to live in degree $-1$. For any $K$-theory class
\[ \mc{E} = [E^0-E^1], \]
we may define its total Chern class by
\[ c(x, \mc{E}) \coloneqq \frac{c(x, E^0)}{c(x, E^1)}. \]

\begin{defn}
    The \textit{$Y(x)$-observable} is defined by
    \begin{align*}
        c(x, \mc{C}^*) &= c(x, N^*) \frac{c(x-\ep_1, K^*) c(x-\ep_2, K^*)}{c(x, K^*) c(x-\ep_1 - \ep_2, K^*)} \\
        &= \prod_{\alpha=1}^N (x-a_{\alpha})\frac{c(x+\ep_1, K^*) c(x+\ep_2, K^*)}{c(x, K^*) c(x+\ep_1 + \ep_2, K^*)}.
    \end{align*}
\end{defn}

We may evaluate it at a general fixed point to obtain
\begin{align*}
    Y(x)|_s &= \prod_{\alpha=1}^N \qty((x-a_{\alpha}) \prod_{(i,j) \in \lambda^{(\alpha)}} \frac{(x-a_{\alpha}-\ep_1 i - \ep_2(j-1))(x-a_{\alpha} - \ep_1(i-1) - \ep_2 j)}{(x-a_{\alpha}-\ep_1(i-1) - \ep_2(j-1))(x-a_{\alpha}-\ep_1 i - \ep_2 j)}) \\
    &= \prod_{\alpha=1}^N \frac{\prod_{\box \in \partial_+ \lambda} (x-a_{\alpha} - c_{\square})}{\prod_{\blacksquare \in \partial_- \lambda} (x-a_{\alpha} - c_{\blacksquare} - \ep_1 - \ep_2)},
\end{align*}
where $c_{\square} = \ep_1(i-1) + \ep_2(j-1)$. We can then compute the expected value
\[ \ev{Y(x)} = \frac{1}{Z} \sum_{k=0}^{\infty} q^k \int_{\ol{\mc{M}}_k(N)} Y(x) \Xi_k, \]
which is a meromorphic function of $x$ with poles at
\[ x= a_{\alpha} + \ep_1(i-1) + \ep_2(j-1), \]
where $(i, j)$ lie in a finite set with cardinality at most $k$ at order $q^k$ and satisfy $1 \leq i, j \leq k$.

Recall the gamma function
\[ \Gamma(z) = \int_0^{\infty} \frac{\dd t}{t} t^z e^{-t}, \]
which is defined for $\Re z > 0$ and has an analytic continuation which is meromorphic with first order poles at the nonpositive integers. We will need the generalization
\[ \Gamma_2(z, \ep_1, \ep_2). \]
Before this, recall that $z \Gamma(z) = \Gamma(z+1)$ and $\Gamma(1) = 1$. Similarly, if we set
\[ \Gamma_1(z, \ep) = \ep^{\frac{z}{\ep}} \Gamma\qty(\frac{z}{\ep}), \]
this satisfies the functional equation
\[ \Gamma_1(z+\ep, \ep) = z \Gamma_1(z, \ep). \]
The \textit{Barnes $2$-gamma} function will be defined the functional equation
\[ \frac{\Gamma_2(z+\ep_1, \ep_1, \ep_2)\Gamma_2(z+\ep_2, \ep_1, \ep_2)}{\Gamma_2(z, \ep_1, \ep_2) \Gamma_2(z+\ep_1+\ep_2, \ep_1, \ep_2)} = z. \]

\begin{defn}
    The \textit{fundamental $qq$-character for $A_1$-theory} is defined by
    \[ \chi(x) \coloneqq Y(x+\ep_1 + \ep_2) + q P(x) Y(x)^{-1}, \]
    where
    \[ P(x) = \prod_{f=1}^{2N}(x-m_f). \]
\end{defn}

\begin{defn}
    The \textit{$A_1$ $qq$-character} is defined by the formula
    \[ \chi_{\omega_1, \ldots, \omega_w}(x) \coloneqq \sum_{I \sqcup J = \qty{1,\ldots,w}} q^{\sharp J} \prod_{i \in I} Y(x + \omega_i + \ep_1 + \ep_2) \prod_{j \in J} \frac{P(x+\omega_j)}{Y(x+\omega_j)} \prod_{\substack{i \in I \\ j \in J}} S(\omega_i - \omega_j), \]
    where
    \[ S(x) = \frac{(x+\ep_1)(x+\ep_2)}{x(x+\ep_1+\ep_2)}. \]
\end{defn}

\begin{rmk}
    This is the beginning of quantum field theory, where we should think about this as an operator product corresponding to
    \[ \chi(x+\omega_1) \cdots \chi(x+\omega_w) \]
    when we make the $\omega_i$ collide.
\end{rmk}


We would also like to discuss the expected value more precisely.
\begin{defn}
    The \textit{expectation value} of the $Y(x)$-observable is
    \[ \ev{Y(x_1) Y(x_2) Y^{-1}(X_3)} \coloneqq \frac{1}{z} \sum_{k=0}^{\infty} q^j \int_{\ol{\mc{M}}_k(N)} Y(x_1) Y(x_2) Y(x_3) \Xi_k . \]
\end{defn}

Using equivariant localization, we see that
\begin{align*}
    \ev{Y(x_1) Y(x_2) Y(x_3)^{-1}} &= \frac{1}{Z} \sum_{\ul{\lambda}} q^{\abs{\ul{\lambda}}} \mu_{\ul{\lambda}}(\ul{a}, \ul{m}, \ul{\ep}) Y(x_1)|_{\ul{\lambda}} Y(x_2)|_{\ul{\lambda}} \frac{1}{Y(x_3)|_{\lambda}}.
\end{align*}

To speed up our calculations, we will express $Y(x)|_{\ul{\lambda}}$ as a plethystic exponential
\[ Y(x)|_{\ul{\lambda}} = \E [-e^x S_{\ul{\lambda}}^*], \]
where
\[ \E\qty[\sum_{s \in S_+} e^{\eta_s^+} - \sum_{s \in S_-} e^{\eta_s^-}] = \frac{\prod_{s \in S_-} \eta_s^-}{\prod_{s \in S_+}\eta_s^+}. \]
In addition, 
\[ S_{\ul{\lambda}} = \sum_{\alpha=1}^N e^{a_{\alpha}} - (1-q_1)(1-q_2)K_{\ul{\lambda}}, \]
where
\[ K_{\ul{\lambda}} = \sum_{\alpha=1}^N e^{a_{\alpha}} \sum_{(i,j) \in \lambda^{(\alpha)}} q_1^{i-1} q_2^{j-1} \]
is the character of $K$ at $\ul{\lambda}$. Using the plethystic exponential, we can write
\begin{align*}
    \mu_{\ul{\lambda}}(\ul{a}, \ul{m}, \ul{\ep}) &= \E \qty[- \frac{S_{\ul{\lambda}} S_{\ul{\lambda}}^* - NN^* - MS_{\ul{\lambda}}^*}{(1-q_1^{-1})(1-q_2)^{-1}}] \\
    &= \E [N K_{\ul{\lambda}}^* + q_1 q_2 N^* K_{\ul{\lambda}} - (1-q_1)(1-q_2)K_{\ul{\lambda}}K_{\ul{\lambda}}^* - M K_{\ul{\lambda}}^*],
\end{align*}
where
\[ M = \sum_{f=1}^N e^{m_f}. \]

\begin{thm}
    The expected value
    \[ \ev{\chi_{\omega_1, \ldots, \omega_w}(x)} \]
    is an entire function of $x$.
\end{thm}

\begin{proof}[Sketch of proof for fundamental $qq$-character]
    We need to check pole cancellation for the expression
    \[ \ev{Y(x+\ep_1+\ep_2)} + q P(x) \ev{Y(x)^{-1}}. \]
    Note that $x$ is a pole if there exists $\alpha \in \qty{1, \ldots, N}$ and $p, q \in \Z_{>0}$ such that
    \[ x = a_{\alpha} + \ep_1(p-1) + \ep_2(q-1) \]
    and $(p,q) \in \partial_- \lambda^{(\alpha)}$. Therefore, we have
    \[ \Res \ev{Y(x + \ep_1 + \ep_2)} = \sum_{\substack{ s = (\lambda^{(1)}, \ldots, \lambda^{(N)}) \\ (p,q) \in \partial_- \lambda^{(\alpha)} }} \mu_s \Res Y(x+\ep_1+\ep_2). \]
    Considering the other term, we need to compute the residue
    \[ \sum_s \mu_s q P(a_{\alpha} + \ep_1(p-1) + \ep_2(q-1)) \Res \frac{1}{Y(x)|_s}. \]
    The poles of this expression require $(p,q) \in \partial_+ \lambda^{(\alpha)}$. All of the combinatorics works out, so it remains to check that the measures work out, or more precisely that
    \[ \mu_{(\lambda^{(1)}, \ldots \wt{\lambda}^{(\alpha)}, \ldots, \lambda^{(N)})} \Res_{\substack{x=a_{\alpha} + c_{\blacksquare} \\ \blacksquare \in \partial_- \wt{\lambda}^{(\alpha)}}} = q P(a_{\alpha} + c_{\square}) \mu_{(\lambda^{(1)}, \ldots, \lambda^{(\alpha)}, \ldots, \lambda^{(N)})} \Res_{x = a_{\alpha} + c_{\square}} \frac{1}{Y(x)}. \]

    Using the plethystic exponential and setting $\xi = e^{a_{\alpha} + c_{\square}}$, we can write
    \begin{align*}
        \mu_{\wt{\ul{\lambda}}} &= \mu_{\ul{\lambda}} \times \E[q_1 q_2 \xi S_{\ul{\lambda}}^* + \xi^{-1}S_{\ul{\lambda}} - 1 + q_1 + q_2 - q_1 q_2 - M \xi^{-1}] \\
        &= \mu_{\ul{\lambda}} \frac{P(a_{\alpha} + c_{\square}) (-1)^N}{Y'_{\ul{\lambda}}(a_{\alpha} + c_{\square}) Y_{\ul{\lambda}}(a_{\alpha} + c_{\square} + \ep_1 + \ep_2)} \qty(\frac{\ep_1 + \ep_2}{\ep_1 \ep_2}).
    \end{align*}
    We are now able to evaluate the residue. We see that
    \begin{align*}
        \Res_{x = a_{\alpha}+ c_{\square}} & \ev{Y(x+\ep_1 + \ep_2) + q \frac{P(x)}{Y(x)}} \\ 
        =&{} \frac{1}{Z} \sum_{\square \in \partial_+ \lambda^{(\alpha)}} q^{\abs{\ul{\lambda}}+1} \mu_{\ul{\lambda}} \frac{P(a_{\alpha} + c_{\square})}{Y'(a_{\alpha} + c_{\square})} + \frac{1}{Z} \sum_{\square \in \partial_- \wt{\lambda}^{\alpha}} q^{\abs{\wt{\ul{\lambda}}}} \mu_{\wt{\ul{\lambda}}} \Res_{x = a_{\alpha} + c_{\square}} Y(x + \ep_1 + \ep_2).
    \end{align*}
    The desired result follows from more computation, namely that
    \[ \Res_{x = a_{\alpha} + c_{\square}} Y_{\wt{\ul{\lambda}}} (x+\ep_1 + \ep_2) = Y_{\ul{\lambda}}(a_{\alpha} + c_{\square} + \ep_1 + \ep_2) \frac{\ep_1 \ep_2}{\ep_1 +\ep_2} \]
    and that
    \[ \E[- \xi q_1 q_2 S_{\ul{\lambda}}^* - 1] = \Res_{x = a_{\alpha} + c_{\square}} Y_{\wt{\ul{\lambda}}}(x+\ep_1+\ep_2). \qedhere \]
\end{proof}

In the $\wh{A}_0$-theory, the partition function is defined to be
\[ Z = \sum_k q^k \int_{\ol{\mc{C}}_k(N)} c(\ep_3, T^* \ol{\mc{M}}_k(N)), \]
where we magically have new parameters $\ep_3$ and $\ep_4 = -(\ep_1 + \ep_2 + \ep_3)$. In this case, the measure is given by
\[ \mu_{\ul{\lambda}}(\ul{a}, \ul{\ep}) = \E \qty[-\frac{1-q_3}{(1-q_1^{-1})(1-q_2^{-1})}(S_{\ul{\lambda}}S_{\ul{\lambda}}^* - NN^*)]. \]
This is invariant under swapping $\ep_3, \ep_4$ and $\ep_1, \ep_2$, which corresponds to transposing $\ul{\lambda}$. In this case, the fundamental $qq$-character is given by
\begin{align*}
    &\chi(x) = Y(x+\ep_1 + \ep_2) + q \frac{Y(x-\ep_3)Y(x-\ep_4)}{Y(x)} + \cdots \\
    =& \sum_{\mu} q^{\abs{\mu}} \frac{\displaystyle\prod_{\square \in \partial_+ \mu} Y(x+\ep_1+\ep_2+\sigma_{\square})}{\displaystyle\prod_{\blacksquare \in \partial_- \mu} Y(x + \sigma_{\blacksquare})} \prod_{\square \in \mu} \qty( 1 + \frac{\ep_1 \ep_2}{(\ep_3 h_{\square} + (\ep_1 + \ep_2) a_{\square})(\ep_3 h_{\square} + (\ep_1 + \ep_2)(a_{\square}+1))}).
\end{align*}



Previously, we were integrating over generalisations of $Hilb^n\mathbb{C}^2$. Today we will generalise even further, and integrate over products of such:

\[
\overline{\mathcal{M}}_{k_1}(n_1) \times \dots \times \overline{\mathcal{M}}_{k_r}(n_r)  
\]
 
\begin{defn}
  We will call spaces like the above \emph{moduli spaces of quiver instantons}.
\end{defn}

\section{The theory for general quivers} 
\begin{defn}
Let $\Gamma = (V_\Gamma, E_\Gamma, s,t)$ be a \emph{quiver}, 
wherein $V_\Gamma$ is its set of \emph{vertices}, $E_\Gamma$ its set of \emph{edges}, 
and $s,t$ the source, target maps $E_\Gamma \to V_\Gamma$. sending an edge to where it starts or ends.
\end{defn}

\begin{defn}
Let $\Gamma$ a quiver. Let $m : V_\Gamma \to \mathbb{Z}_{\geq 0}$ be its \emph{flavours}, and 
$n: V_\Gamma \to \mathbb{Z}_{>0}$ its \emph{colors}. 
Then the \emph{moduli space of quiver instantons} is
\[
\overline{\mathcal{M}}^\Gamma(n) = \sqcup_{k : V_\Gamma \to \mathbb{Z}_{\geq 0}} \prod{v \in V_\Gamma} \overline{\mathcal{M}}_{k(r)}(n(r))
\]
\end{defn}
\begin{rmk}
Each $\mathbb{C}^2$ is acted on by a torus $\mathbb{C}^\times \times \mathbb{C}^\times$. 
Further, we can work equivariantly with respect to each $GL(n(v))$. 
Define 
\[
\GL^\Gamma(n(v)) := \prod \GL(n(v)).
\]
\end{rmk}


\begin{rmk}
  Each $\overline{\mathcal{M}}_{k(v)}(n(v))$ has a tautological bundle, $K_v$. 
We could choose a tuple of complex numbers $\vec{\mu^{(v)}}$, one for each vertex,
and a tuple of complex numbers $\vec{\mu^{(e)}}$, one for each edge, and evaluate 
\[
\sum_{k : V_\gamma \to \mathbb{Z}_{\geq 0}} \prod_{v \in V_\Gamma} q_v^{k(v)}\int \prod_{v \in V_\Gamma} \prod_{f = 1}^{m(v)} c(K_v, \mu_f^{(v)}) \prod_{e \in E_\Gamma} c(\mathcal{E}_e, \mu^{(e)}_e).
\]
\end{rmk} 

\begin{defn}
Let $\mathcal{H}\mr{om}(\mathcal{F}_1,\mathcal{F}_2)$ denote the sheaf on 
\[
\mathbb{P}^2 \times \overline{\mathcal{M}}_{k(s(e))}(n(s(e)) \times \overline{\mathcal{M}}_{k(t(e))}(n(t(e)))
\]
parametrising the data of two points in our moduli space corresponding to two torsion-free sheaves. 
Then define 
\[
\mathcal{E}_e := (R\pi)_\star \mathcal{H}\mr{om}(\mathcal{F}_1,\mathcal{F}_2)
\]
wherein $\pi$ is the projection map forgetting the $\mathbb{CP}^2$ factor. 
\end{defn} 

In principle, we could imagine integrating the above integral. However, not every $\Gamma$ leads to a nice story. 
One possible restriction would be to hope that our integrals converge. A basic condition for that would be:
\[
\dim_{\mathbb{C}} \overline{\mathcal{M}_k}^\Gamma(n) \geq \sum_v m(v) k(v) + \sum_e \dots,
\]
that is, the dimension of our moduli space is at least the degree of the chern class of the bundle we're integrating over. 
(The tautological bundle term is clear, and 
\[
\chi(\mathcal{E}_e) = \int_{\mathbb{P}^2} \on{ch}(\mathcal{F}_1^\vee) \on{ch}(\mathcal{F}_2) \on{Td}_{\mathbb{P}^2}
\]
gives a term like 
\[
\sum_e k(s(e))n(t(e)) + k(t(e)) n(s(e)).
\]
Also note 
\[
\dim_{\mathbb{C}} \overline{\mathcal{M}_k}^\Gamma(n) = \sum_v 2k(v)n(v).
\]

\begin{rmk}
The above restriction is linear in $k$ on both sides. We could for the 
restriction to be true for every $k$. 
That leads to the condition that 
\[
\sum_v (2n(v) - m(v)) - \sum_e (n(s(e)) - n(t(e)) )\geq 0.
\]
\end{rmk}

\begin{thm}
Fix $\Gamma$. There exists $\vec{n}, \vec{m}$ satisfying the 
above condition if either 
\begin{enumerate}
\item $\Gamma$ is an ADE Dynkin diagram as an unoriented graph;
\item $\Gamma$ is an affine ADE Dynkin diagram as an unoriented graph, $m = 0$, and $n$ is related to the Dynkin indices.
\end{enumerate}
\end{thm}


\begin{exm}
Choose the quiver with one vertex and $\ell$ loops. 
Then the inequality reads 
\[
2n -m \geq 2n \ell.
\]
This is satisfied if either 
\begin{enumerate}
\item $\ell = 0$, $m \geq 2n$ (the $A_1$ case);
\item $\ell = 1, m = 0$ (the $\widehat{A}_0$ case).
\end{enumerate}
\end{exm}

In more detail, the solutions are  (described here but not drawn due to the note-taker's inadequacy)
\begin{enumerate}
\item $\widehat{A}_r$: a cyclic graph with $r$ nodes, all with the mark $N$;
\item $\widehat{D}_r$: a linear graph of $r-3$ nodes, with two additional nodes attached to the first and last nodes. The marks of the linear nodes are all $2N$; the marks of the additional nodes are all $N$.
\item $\widehat{E}_r$: an exceptional series of graphs.
\begin{enumerate}
\item $\widehat{E}_6$, the graph with 7 vertices, a linear graph with three nodes with two chains of length two added at the end. Dynkin marks are $N,2N,3N$ on the linear part and then $2N,N$ on each chain.
\item $\widehat{E}_8$, the graph with 9 vertices, namely a linear graph of length 7 with one additional node attached to the sixth node. The marks on the linear component are $N,2N,3N,4N,5N,6N,4N,2N$ and the added node has mark $3N$.
\item $\widehat{E}_7$, left as an exercise. 
\end{enumerate}
\end{enumerate}

\begin{rmk}
In physics, this inequality is related to the condition of \emph{asymptotic freedom}
for the gauge theory associated to the quiver. It's hard to explain this precisely.
These integrals approximate the partition function of the gauge theory, which is a path integral over the space of all connections.
You could try to define the path integral rigorously by discretising. Asymptotic freedom, in that picture, is an equation relating to when 
the discretised path integral is a good approximation to our original path integral.
\end{rmk}

\begin{exm}
The standard model could be related to the $A_3$ quiver with $m=0$ and $n = 1,2,3$ 
(though we are of course modeling supersymmetric versions of the theory with our integrals.)
\end{exm}

\begin{rmk}  
If you've found a solution of this inequality, you could take a limit: 
\begin{align*}
\mu^{(v)}_{m(v)} \to \infty 
\\
q_v \to 0 
\end{align*}
This limit is a finite dimensional shadow of \emph{renormalisation group flow}.
Taking this limit removes one factor from our chern class; essentially it just removes one flavour.
This means that, replacing $m \to m-1$, we could find a solution to the inequality for the new $m$.
So it suffices to solve for solutions where the inequality is an equality:
together with the trick of 'freezing' one of the nodes, this gives all solutions. 
For example, we can get $A_r$ by freezing a node on $\widehat{A}_{r+1}$, sending $q_{r+1}\to0, q_0 \to 0$. 
\end{rmk}

\begin{rmk}
We have parameters 
\[
q_v \in \mathbb{C}_{1 \cdot 1 <1}
\]
and
\[
\vec{a}^{(v)} := (a_1^{(v)},\dots,a_{n(v)})^{(v)})
\]
\end{rmk}

\begin{exm}
When we seek to get $A_r$ by freezing $\widehat{A}_{r+1}$ as above, after a little bit of computation we find that 
\[
\vec{a}^{(0)} = (\mu_1,\dots,\mu_N)
\]
and
\[
\vec{a}^{(r+1)} = (\mu_1^{(r)} + \ep_1 + \ep_2,\dots)
\]
\end{exm}

Let's reiterate the idea of how we study integrals like the above, namely the use of $qq$-characters.
Since there is some ADE structure to this story, it might not be so surprising that what we called $qq$-characters 
have something to do with the characters of algebras related to the ADE series.
Again $\int_{\overline{\mathcal{M}}^\Gamma(k)_n}$ will be related to a combinatorial sum,
namely a sum over a doubly-indexed set of Young diagrams $\lambda^{v,\alpha}$, where $v \in V_\Gamma$ and $\alpha \in 1,\dots,n(v)$. 

\begin{defn}
Let 
\[
Y_v(x)_{|\lambda} = \prod_{\alpha = 1}^{n(v)} \prod_{\square \in \partial_+ \lambda^{(v,\alpha)}} \frac{\prod_{\square \in \partial_+ \lambda^{(v,\alpha)}} (x - a_\alpha^{(v)} - c_\square^{\ep_1,\ep_2})}{\prod_{\square \in \partial_- \lambda^{(v,\alpha)}} (x - a_\alpha^{(v)} - c_\square^{(\ep_1,\ep_2)})}
\]
Then start to define
\begin{align*}
    X_v(x) \coloneqq{}& Y_v(x + \ep_1 + \ep_2) \\
    &+ q_v \frac{P_v(x)\prod_{e \in s^{-1}(v)} Y_{t(e)}(x +\mu_e) \prod_{e \in t^{-1}(v)} Y_{s(e)} (x + \ep_1 + \ep_2 - \mu_e)}{Y_v(x)} + \dots 
\end{align*}
\end{defn}

\begin{thm}
There is a completion of this sum such that $<X_v(x)>$ has no poles in $x$. 
This has the same size as the sum defining characters of affine Kac-Moody groups. 
\end{thm}
\begin{exm}
Let's study the $A_2$ case with marks $N,N$ and two framing nodes $N,N$.
Redefine so $\mu_e = 0$ by rescaling. Then, 
\begin{align*}
Y_1(x + \ep_1 + \ep_2) + q_1 \frac{P_1(x) Y_2(x + \ep_1 + \ep_2)}{Y_1(x)} + q_1 q_2 \frac{P_1(x) P_2(x)}{Y_2(x)}
\end{align*}
Here, the term we needed to add was $q_1 q_2 \frac{P_1(x) P_2(x)}{Y_2(x)}$, and this 
exactly cancels all remaining pools. So it stops after 3 terms. Why 3 terms? $A_2$ is related to $sl_3$.
This looks like a character: a character of what?
The large $x$ asymptotics are like 
\[
(1 + q_1 + q_1 q_2) x^N + \dots 
\]
There's something called a \emph{conformal extension} of a simple Lie group $G$. It has to do with the fact that 
there's a centre. For any group with a centre, we can add a torus factor for every cyclic factor in the centre. 
This relates, for example, $\mr{SL}$ to $\GL$. 
\end{exm}

\begin{rmk}
The ``fundamental'' in $qq$-characters denotes that they're related to highest weight. 
\end{rmk}

\section{General $qq$-characters}%
\label{sec:General qq-characters}

\begin{defn}
A \emph{general} $qq$-character depends on 
\[
\vec{\omega} \in \oplus_{v \in V_\Gamma} \mathbb{C}^{w(v)}
\]
For $w : V_\Gamma \to \mathbb{Z}_{\geq 0}$.
We could try to define a $qq$-character $X_{\vec{\omega}}(x)$. It would look like 
\[
X_{\vec{\omega}}(x) = \prod_{v \in V_\Gamma} \prod_{a = 1}^{w(v)} Y_v(x + \omega^{(v)}_a + \ep_1 + \ep_2) + \dots 
\]
The point is that it's not \emph{just} the product of fundamental characters; there are corrections.
You're trying to encode all orders of expansion of this function.
\end{defn}

Once we define a general qq-character, we'll then be able to set the stage for the limit shape story, that is, Seiberg-Witten geometry.

\begin{rmk}
Remember that a \emph{quiver} is an oriented connected finite graph, possibly with more than one arrow between nodes. 
Then \emph{quiver gauge theory} depends on the data of $N_i, M_i$, vector spaces for each $i \in Vertices(Q)$.

Recall there is an ADE classification given by imposing the condition that 
\[
2n_i - m_i = \sum_{a(e) = i} n_{t(e)} + \sum_{t(e) = i} n_{s(e)}
\]
The $N,M$ are discrete parameters of our theory, but there are also 
continuous parameters $a^{(i)} \in \Lie(T) \subset Lie(\GL(N_i))$,
and $m^{(i)} \in \Lie(T) \subset \Lie(\GL(M_i))$. 
Further there are $q_i \in \mathbb{C}$, complex numbers of small absolute value.

Finally there are parameters $m_e \in \mathbb{C}$, for $e \in E(\Gamma)$, and there are two parameters
\[
(\ep_1,\ep_2)\in \mathbb{C}^^2 \subset \Lie(T) \subset \Lie(\GL(2)).
\]
\end{rmk}

\begin{defn}
We may then define a partition function 
\begin{align*}
Z((a^{(i)},m^{(i)}, q_i)_{i \in V_\Gamma}, (m_e)_{e \in E_\Gamma},\ep_1,\ep_2) \coloneqq \sum_{\lambda^{i,\alpha}} \mu_\lambda(a,m,\ep_1,\ep_2) \prod_{i \in V_\Gamma} q_i^{\sum_\alpha |\lambda^{(i,\alpha)}|}
\end{align*}
wherein $\alpha$ is the color parameter, and runs from $1$ to $n_i$,
and 
\[
S_{\lambda} = \{ a_\alpha^{(i)} + \ep_1(I -1) + \ep_2(J-1) | 1 \leq I \leq \lambda_J^{(i,\alpha),\vee}, 1 \leq J \leq \lambda_I^{(i,\alpha)}\}.
\]
\end{defn}

\begin{rmk}
There is some redundancy in this description. For $\mu_i \in \mathbb{C}$, $i \in V_\Gamma$, we may shift
\begin{align*}
    m_e &\to m_e + \mu_{t(e)} - \mu_{s(e)}
\\
    a^{(i)} &\to a^{(i)} + \mu_i
\\
    m^{(i)} &\to m^{(i)} + \mu_i 
\end{align*}
\end{rmk}

\begin{rmk}
There are physics expectations that as $\ep_1,\ep_2 \to 0$, or even just $\ep_2 \to 0$, the asymptotics might be computable.
\end{rmk}

To study all of this, we'll need another auxiliary object.
\begin{defn}
  A \textbf{Nakajima quiver variety} depends on the data of a quiver $Q$ and data defined by tuples of integers $\vec{v}_{i \in V_\Gamma}$, $\vec{w}_{i \in V_\Gamma}$. 
  We want at least one $w_i > 0$ and the remaining $w_i$ to be positive. 
  To this data, we assign complex vector spaces $V_i = \mathbb{C}^{v_i}, W_i = \mathbb{C}^{w_i}$. 
  Then the Nakajima quiver variety is defined as
  \[
\mathcal{M}_Q(\vec{v},\vec{w}) := (T^\star \Hom(\vec{V},\vec{W}) \oplus T^\star \Hom_{i \to j}(V_i,V_j))\sslash\!\!\!\sslash GL(\vec{v}),
  \]
  where $\sslash\!\!\sslash$ denotes the symplectic quotient. 
  We may estimate its dimension to be $2 \sum_i v_i(w_i - v_i) + 2 \sum_{i \to j} v_i v_j$.
\end{defn}

\begin{rmk}
These spaces carry obvious \emph{tautological bundles} 
\[
V_i, W_i \to \mathcal{M}_Q(\vec{v},\vec{w})
\]
and hence a \emph{tautological complex} $C_i$:
\[
0 \to V_i \to W_i \oplus \bigoplus_{i \to j} V_j \bigoplus_{k \to i} V_k \to V_i \to 0
\]
wherein all the maps are the obvious ones given by the quiver representation. 
The complex moment map condition is precisely the condition that this is a complex.
\end{rmk}

\begin{rmk}
$\mathcal{M}_Q(\vec{v},\vec{w})$ has a $GL(\vec{w})$-global symmetry. 
There is another symmetry factor, namely by permuting edges, and finally a $\mathbb{C}^\star_q$ which
acts by scaling the moment map.
There is some redundancy in this description.
\end{rmk}

\begin{defn}
Let $Y_i(x)$ the observable defined by 
\[
Y_i(x)_{|\lambda} = \prod_{\alpha = 1}^N \frac{\prod_{\square \in \partial_+ \lambda^{(i,\alpha)}}(x - a_\alpha^i - \ep_1(I- 1) - \ep_2(J-1))}{\prod_{\square \in \partial_- \lambda^{(i,\alpha)}}(x - a_\alpha^i - \ep_1 I - \ep_2 J)}
\]
\end{defn} 

We will now give a general formula for qq-characters. 
Let's fix vectors 
\[
\vec{u} \in \vec{W},
\]
which we think of as equivariant parameters for $GL(\vec{W})$, and an equivariant parameter $-\frac{\ep_1 + \ep_2}{2}$ for $\mathbb{C}^\times_q$, and define 
\begin{align*}
    &\mathcal{X}_{\vec{u}}(x) \\&= \sum_{\vec{v} \in \mathbb{Z}^{V_\Gamma}_+}\prod_{i \in V_\Gamma} q_i^{v_i}  \int_{\mathcal{M}_Q(\vec{v},\vec{w})} c(T^\star \mathcal{M}_Q(\vec{v},\vec{w}, \ep_1) \prod_{i \in V_\Gamma} \prod_{b = 1}^{v_i} P_i(x + \xi_{i, b}) \prod_{i \in V_\Gamma} \frac{\prod_a Y_i (x + \ep_1 + \ep_2 + \ep_{ia}^+)}{\prod_b Y_i(x + \ep_1 + \ep_2 + \eta_{ib}^-)},
\end{align*}
where everything is taken equivariantly with respect to all symmetries listed above and
\begin{align*}
    \ch(V_i) &\coloneqq \sum_b^{v_i} e^{\xi_{i,b}}, \\
    P_i(x) &= \prod_{f = 1}^{m_i} (x - m_f^{(i)}), \\
    \ch(C_i^+) &= \sum_a e^{\eta_i^+,a}, \\
    \ch(C_i^-) &=\sum_b e^{\eta^+_{ib} b}.
\end{align*}


\begin{defn}
$\mathcal{X}_{\vec{u}}(x)$ is a fundamental $qq$-character. 
\end{defn}
\begin{prop}
$\ev{ \mathcal{X}_{\vec{u}}(x) }$ has no poles in $x$.
\end{prop}

\begin{exm}
    Let's recover the $\vec{A}_1$-case we studied earlier. Here, $\mc{M}_{A_1}(v,w) = T^\star Gr(v,w)$. The symmetry group is $\mr{PGL}(w) \times \mathbb{C}^\times$. 
The maximal torus $T \subset \PGL(w) \times \mathbb{C}^\times$ acts with isolated fixed points, labeled by the $v$-dimensional planes spanned by the coordinate axes in the basis chosen by our choice of maximal torus. 
In other words, fixed points are labelled by partitions of $\{1,\dots,w\} = \mathcal{I} \sqcup \mathcal{J}$ into two disjoint sets, so that $|\mathcal{I}| = v$. 

In terms of the Nakajima data, let $I : W \to V, J : V \to W$, and impose the stability condition that $J$ is of maximal rank.
Hence it's the space of operators 
\[
\{ I,J \mid IJ =0, J\text{ of maximal rank}\}/\GL(V),
\]

So the tangent space is 
\[
T\mathcal{M}_{A_1} = \{ \delta I, \delta J \mid  \delta I J + J \delta I = 0\}/(\delta I , \delta J) \sim (\delta I + \zeta I, \delta J  - J \zeta, \zeta \in End(V))
\]
At the fixed point $I = 0, Je_\gamma = e_{i_\gamma}$, we can identify the tangent space as 
\[
T = T^\star \Hom(V,W/V)
\]
Therefore, 
\[
\ch(T) = \sum_{i \in \mathcal{I}, j \in \mathcal{J}} e^{u_i - u_j} + e^{\ep_1 + \ep_2 + u_j - u_i}
\]
where the first summand comes from $TGr$, and the second from $T^\star Gr$.
\end{exm}

So the qq-character takes the form 
\[
\sum_{\mathcal{I} \sqcup \mathcal{J} = \{1,\dots,w\}} q^{|\mathcal{I}|} \frac{\prod_{i \in \mathcal{I},j \in \mathcal{J}}(u_i - u_j  + \ep_1)(-\ep_2 + u_j - u_i)}{\prod_{i \in \mathcal{I},j \in \mathcal{J}}(u_i - u_j)(-\ep_1 - \ep_2 + u_j - u_i)} \prod_{i \in \mathcal{I}} P(x + u_i) \times \Centerstack{{contribution from} {canonical complex}}
\]
Of course the canonical complex takes the form 
\[
V \to W \to V.
\]
Since $I = 0$ and $J$ is injective, up to equivariant weight,
$C^+ = W.V$, and $\on{Ch}(C^+) = \sum_{j \in \mathcal{J}} e^{u_j}$, $\on{Ch}(C^-) = \sum_{i \in \mathcal{I}} e^{u_i - \ep_1 - \ep_2}$.

Hence the final factor gives an overall answer of 
\begin{align*}
    \mathcal{X}_{\vec{u}} =&{} \sum_{\mathcal{I} \sqcup \mathcal{J} = \{1,\dots,w\}} q^{|\mathcal{I}|} \frac{\prod_{i \in \mathcal{I},j \in \mathcal{J}}(u_i - u_j  + \ep_1)(-\ep_2 + u_j - u_i)}{\prod_{i \in \mathcal{I},j \in \mathcal{J}}(u_i - u_j)(-\ep_1 - \ep_2 + u_j - u_i)} \prod_{i \in \mathcal{I}} P(x + u_i) \\ &\times \frac{\prod_{j \in \mathcal{J}} Y(x + u_j + \ep_1 + \ep_2)}{\prod_{i \in \mathcal{I}} Y(x + u_i)}
\end{align*}

This is the qq character for $A_1$-quiver theory, namely SQCD. It has $2^w$ terms. This formula is already indicative that not all such formulas come as rational expressions in the $Y$ whose arguments are just shifted.
If we were to set for example $u_i = u_j$, we would have a smaller torus. The fixed point locus would not be isolated but it would still be compact. So this integral is finite. So even though there's terms like $u_i - u_j$, we should expect the limit as these two things approach to exist, but we might get some derivatives of $Y$ appearing.


\section{Limit shapes}

We've defined many partition functions 
\[
Z(\vec{a},\vec{m}, \vec{q}, \ep_1,\ep_2)
\]
dependent on many parameters. The parameters $\ep_i$ were special,
recording the weights of $GL(2)$ symmetry for $\mathfrak{g}(\mathbb{CP}^2,\mathbb{CP}^1_\infty)$.
These partition functions were essentially sums over infinite sets whose elements 
are finite subsets of $\mathbb{C}$. So these subsets had the form 
\[
\vec{\lambda} = \qty{a^{(i)}_\alpha + \ep_1(I - 1) + \ep_2(J-1) \mid i \in V_\Gamma, \alpha = 1,\dots,n, (I,J) \in \lambda^{(i,\alpha)}}.
\]
Each such set carries a certain complex weight. There is a way to specify 
all the parameters of the problem such that this weight is real and positive. In that case we could talk about the induced probability distribution.
Then we could be interested in the limiting behaviour as $\ep_1,\ep_2 \to 0$. In this probablistic case that would 
require us to find the limiting measure. 
We can draw a curve which describes the boundary of these finite sets, 
and seek its limiting shape as $\ep_1,\ep_2 \to 0$. 

Physically, $\ep_1,\ep_2$ effectively confine instantons into a polydisc of volume $\approx (\ep_1 \ep_2)^{-1}$.
Then taking the limit $\ep_1,\ep_2 \to 0$ is a \emph{thermoydynamic limit}. 
We view $Z = exp( \ep_1^{-1} \ep_2^{-1} \mathcal{F}(\vec{a},\vec{m},\vec{q}) + \dots)$,
and our goal is to extract the function $\mathcal{F}$. 
The typical size of a partition will be 
\[
|\lambda^{(i,\alpha)}| \sim (\ep_1 \ep_2)^{-1} \#(\vec{a},\vec{m},\vec{q}).
\]

\begin{rmk}
A probabilistic interpretation of $Z$ exists at all because the moduli space of instantons, 
$\overline{\mathcal{M}}_k(n)$, is a complex symplectic manifold. So each tangent space is a symplectic vector space, and so the 
weights of the symmetry group therefore split in pairs. So if $w$ appears in $T \overline{\mathcal{M}}_k(n)$, so does $\ep_1 + \ep_2 - w$. 
That defines a symplectic form on $\overline{\mathcal{M}_k(n)}$, which roughly descends from that on the space of all connections:
\[
\int_{\mathbb{R}^4} \dd z_1 \wedge \dd z_2 \wedge \Tr \delta A \wedge JA.
\]
If we specialize $\ep_1 + \ep_2 = 0$, then 
\[
\mu_\lambda = \frac{\prod(\dots)}{\prod (-w^2)}.
\]
Each $w$ is a linear function of the $\vec{a},\vec{\ep}$ terms. Choosing $\ep_1 \in i \mathbb{R}, a_\alpha^{(i)} \in i \mathbb{R}$, 
then $(-w^2) \in \mathbb{R}^+$. The numerator is trickier, but can also be made positive real by judicious specialisation of our parameters.
\end{rmk}

If we stay away from the locus wherein 
\[
a_\alpha^{(i)} - a_\beta^{(j)} \in \ep_1 \mathbb{Z}_+ + \ep_2 \mathbb{Z}_+,
\]
which is the locus wherein some of the weights could become 0 in the denominator,
we won't have any convergence issues.

\begin{rmk}
Remember our 
\[
Y_i(z) = \exp\, \int_\mathbb{R} \rho_i(x) \dd x \log(z-x) = \prod_{\alpha = 1}^{n_i} \frac{\prod_{\square \in \partial_+ \lambda^{(i,\alpha)}} (z - a_\alpha^{(i)} - c_\square)}{\prod_{\square \in \partial_- \lambda^{(i,\alpha)} (z - a_\alpha^{(i)} - c_\square)}}
\]
has $\rho_i$ with compact support near $a_\alpha^{(i)}$, and we remember that 
\[
c_\square = \ep_1(I- 1) + \ep_2(J-1).
\]
Then 
\[
\rho_{i|\lambda} = \sum \pm \delta(x - a_\alpha^{(i)} - c_\square).
\]
All these $\delta$-functions have signs.  We should think that the density between these $\pm$ will concentrate,
so we might thank that in the limit we might get a continuous function which equals zero almost everywhere except at a finite number of intervals where the $\pm$ signs concentrate.
The question is: can we express $\mu$ in terms of this limiting function?

We have 
\begin{align*}
    \mu_{\lambda}(\vec{a},\vec{m},q) =&{} \exp \hbar^{-2} \iint \dd x' \dd x'' \sum_{i \in V} \rho_1(x') \rho_2(x'') K(x' - x'') \\ 
    &- \sum_{e \in E} \rho_{t(e)} (x') \rho_{s(e)}(x'') K(x' - x'' + m_e) \\
    &+ \sum_i \rho_i(x) \qty[ \frac{x^2}{2} \log q_i + \sum_{f= 1}^{m_i} K(x - \mu_f^{(i)})],
\end{align*}
wherein 
\[
K(x) = x^2/2 \qty( \log\qty(\frac{x}{\Lambda}) - 3/2)
\]
The advantage of writing it this way is that it's a quadratic functional, so extremising it is straightforward.
\end{rmk}


Now let's extremise $\mu_\lambda$ subject to the constraints that $\int_{I_{i,\alpha}} \rho_i(x) dx = 1$, $\int_{I_{i,\alpha}} x \rho_i(x) dx = a_\alpha^{(i)}$.
Doing so, we will get linear equations on the $\rho_i$. 
The most concise way to write those linear equations is via 
\[
Y_i^+(x) Y_i^-(x) = q_i P_i(x)  \prod_{e \in f'(i)} Y_{s(e)} (x + m_e) \prod_{e \in f'(i)} Y_{t(e)} (x - m_e)
\]
wherein $P_i(x) = \prod_{f = 1}^{m_i} (x - \mu_f^{(i)})$.

After analytic continuation, the only thing we'll be able to control about the $Y_i$ are their moments
\[
(2\pi i )^{-1} \oint_{A_{i,\alpha}} \dd \log Y_i(x) = a_{i,\alpha}.
\]
Studying analytic continuation is related to the action of the Weyl group. 
So to solve these equations, it suffices to find the invariants of the Weyl group action.
So the strategy now is to find Weyl invariant functions of the $Y_i$. 

\subsection{Review of Lie theory and Weyl groups}

Let's review some Lie theory so that in the above formula for $Y^+Y^-$.
You will recognise something of classical mechanics. 

Let $Q^\vee = \text{coroot lattice} = Hom(\mathbb{C}^\star, T)$ wherein $T$ is our maximal torus. 
The weights form the dual lattice $\Lambda \subset \mathfrak{h}^\star$ (they are subsets of the dual to the Cartan). It's formed by all elements 
$\lambda$ so that $\lambda(\alpha^\vee) \in \mathbb{Z}$ for every coroot.
We let 
\[
t^\lambda = \exp (2 \pi i \lambda(x))
\]
for $t \in T$. 

\begin{rmk}
In the lattice $Q^\vee$ of coroots, there is a basis of \emph{simple coroots} 
$(\alpha_1^\vee,\dots,\alpha_r^\vee)$ such that the adjoint 
action decomposes $ad_t(e_\alpha) = t^\alpha e_\alpha$, wherein $e_\alpha$ is a \emph{root} (it's a lattice of weights, namely the weights of the adjoint representation).
This has a basis of roots, $\mathbb{Z}\alpha_1,\dots,\mathbb{Z}\alpha_r$. 
Finally there is a lattice of coweights, 
\[
\Hom(\mathbb{G}_m, T/Z),
\]
wherein $Z$ is the centre of $G$. Also, $Z = \mr{Weights}/\mr{Roots}$.
\end{rmk}

\begin{defn}
For $\xi \in \eta$, define simple reflection by $r_i(\xi) = \xi - \alpha_i(xi)\alpha_i^\vee$.
It is an interesting theorem that $Weyl = \mathbb{W} = N(T)/T$ is the action generated by the $r_i$.
This can of course also be defined in a multiplicative form. For $t \in T$, an element of the maximal torus, 
we may parametrise $t = \prod z_i^{\alpha_i^\vee}$. 
Applying $r_j$ to such an element, the coordinates $z_i$ of this torus will transform 
in the following way: 
\[
\wt{z}_i = \delta_{i\neq j} z_i + \delta_{ij} z_j \prod_{k= 1}^r z_k^{-C_{jk}},
\]
wherein $C_{ij}$ is the Cartan matrix. My point is that if you look closely at this formula, you will 
see that analytic continuation of the $Y_i$ across the cuts is 
essentially the action of the Weyl group by simple reflections. 
\end{defn}

\begin{rmk}
The more precise statement is the following. Let $g(x) \in \mathbb{C}G_\Gamma$, the group associated to the quiver somewhat extended. Specifically $\mathbb{C}G$ denotes the \emph{conformal extension} of our group, which is given by
\[
\mathbb{C}G = (\mathbb{G}_m)^{\#\text{cyclic factors in centre}} \times_{Z(G)} G 
\]
Essentially, it's where we replace all cyclic factors in the centre by $\mathbb{C}^\times$ factors.
Write $g(x) = \prod_{I \in \wt{\mr{Vertices}}_\Gamma} Y_i(x + \mu_i)^{\alpha_i^\vee} P_i(x  + \mu_i)^{-\lambda_i^\vee} $, where $\wt{\mr{Vertices}}$ denotes that we need to take the universal cover.
The claim is that under analytic continuation 
\[
g(x) \to r_i g(x) 
\]
\end{rmk}

\begin{rmk}
I don't know how well this is known in the case of matrix models, but at lesat for some matrix models, there is a very similar property where this is some combination of resolvents of densities
of matrices which, continuing across the cuts, they transform in the vector representation of the Weyl group. Therefore what I'm about to say here also applies there.
\end{rmk}

\begin{clm}
In any linear representation $V$, $\chi_V(g(x))$ is an analytic function of $x$.
With proper normalisation, it is just a polynomial in $x$.
\end{clm}

\begin{clm}
  Choose a fundamental representation of $G$. The equations 
$\chi_i(g) = T_i(x)$ define curves in $(\mathbb{C}^\times)^r \times \mathbb{C}$.
The tori give coordinates $Y_1,\dots,Y_r$; the last coordinate is $x$. 
There is a Weyl group which acts on this vector space, under which the curve is Weyl invariant.
This curve is called a \emph{Cameral curve}. Of course you have heard about spectral curves, which are curves you can assign to a matrix; its eigenvalues form a curve. That's what you would do in the case of $GL_n$. But in general these guys are subject to more intricate Weyl symmetries,
so it's not really clear what sort of spectral object to consider.
What one can do, for any $W' \subset W$, one can try to form a smaller curve by taking the quotient.
\end{clm}

\begin{exm}
  Let's take the $A_r$ quiver, and take the case wherein we have fundamental matter fields at the first and last node and all labels are $N$.
  Then $g = \on{diag}(z_1 \frac{Y_1}{P_1},\dots, z_r \frac{Y_r}{Y_{r-1}'}, z_{r+1} \frac{P_r}{Y_r}) \in GL(r+1)$.
Then invariants are given by 
\[
\frac{1}{z_1 z_i} P_1(x) \Tr_{\Lambda^i \mathbb{C}^{r+ i}}(g)
\]
the claim is this is a monic polynomial in $x$ of degree $n$. 

This is an example of a Cameral curve. It maps $\mr{Cameral} \to \mathbb{C}_x$ with degree $(r+1)!$ since any permutation gives another solution of our equations.
Now $\mr{Weyl} = S(r+1)$ acts on this curve, and inside we for example have a subgroup $S(r+1) \supset S(r)$. Now $\mr{Cameral}/S(r) = \mr{Spectral}$ is the spectral curve.
It keeps track of one eigenvalue. However, for example, we could also consider $\mr{Cameral}/S(r-1)$ to remember two eigenvalues. 
This special branched structure of course has to do with Gelfand-Tsetlin patterns and is special to $A$-type.
\end{exm}

Spaces of curves like this describe either monopoles in $\mathbb{R}^2 \times S^1$ for ADE type or instantons on $\mathbb{R}^2 \times T^2$ for affine ADE type. This curve in question is actually a quasimap of degree $n$ (in affine type at least) from $\mathbb{P}^1$ into the moduli spaces of $G$-bundles on an elliptic curve.

\section{Use of $qq$-characters}

The main idea is to use the analyticity of $\ev{X_{\vec{w}}(x)}$, together with the fact that for large $x$ the observables 
behave like polynomials, to conclude $\ev{X_{\vec{w}}(x)}$ is a polynomial in $x$.
Therefore, the moments 
\[
\ev{ x^{-k} X_{\vec{w}}(x)}
\]
are zero for $k > 0$. This will give us a set of equations on the partition function. 

\begin{rmk}
One can amputate legs with $n_i = 1$. 
Recall that we have a harmonicity condition; if there exists $i \in Vect_{\Gamma}$ for which $n_i = 1$, then it imposes the 
stringent requirement that $2 = m_i + \sum_{e \in s^{-1}(i)} n_{t(e)} + \sum_{e \in t^{-1}(i)} n_{s(e)}$. That is, it cannot be connected to 
`too many' other nodes. 
There were limited possibilities: 
\begin{enumerate}
\item $\widehat{A_0}$ theory: one node and one loop. 
\item Two more complicated graphs separated  by a node with marking $1$. 
\end{enumerate}
Since $n_i =1$, there is only one partition $\lambda^{(i)}$ at this node. Hence,  
\[
Y_i(x)|_{\lambda^{(i)}} = (x-a_i) \prod_{\square \in \lambda^{(i)}} \frac{(x - c_\square - \ep_1)(x - c_\square - \ep_2)}{(x - c_\square) (x - c_\square - \ep_1 - \ep_2)} .
\]
Now, as $x \to \infty$, to $O(x^{-1})$, we find 
\[
Y_i(x) \sim x - a_1 + \frac{\ep_1 \ep_2}{x} \abs{\lambda^{(i)}} + O(x^{-2}).
\]
The good thing about the size of the partition is that its expectation value
\[
    \ev{\abs{\lambda^{(i)}}} = q_i \frac{d}{dq_i} Z
\]
is simply the logarithmic derivative with respect to fugacity of the partition function.
\end{rmk}

\begin{exm}
Consider $A_1$-theory wherein the framing node has rank 2 and the other node has rank 1. Then 
\[
Z(q_1, \mu_1,\mu_2;\ep_1,\ep_2) = \sum_{k=0}^\infty q^k \int^{\mathbb{C}^\star_{\ep_1} \times \mathbb{C}^\star_{\ep_2}}_{\mr{Hilb}^k(\mathbb{C}^2)} c(K,\mu_1) c(K,\mu_2).
\]
We have another formula 
\[
Z = \sum_\lambda q^{|\lambda|} \prod_{\square \in \lambda} \frac{(c_\square + \mu_1)(c_\square + \mu_2)}{(\ep_1(a_\square + 1) - \ep_2 \ell_\square)(-\ep_1 a_\square + \ep_2(\ell_\square + 1)},
\]
wherein $a_\square = \lambda_i - j$ is arm length, $\ell_\square = \lambda_j^t - i$ is leg length, $c_\square = \ep_1(i-1) + \ep_2(j-1)$ is content. 
For this theory, we have a unique qq-character, so 
\[
\ev{Y(x + \ep_1 + \ep_2) + q (x + \mu_1)(x + \mu_2) Y(x)^{-1}} = ( (1 + q)x + u(\mu_1,\mu_2,\ep_1,\ep_2))
\]
since it has no poles.

Now use the computation in the above remark to expand $\ev{\cdots}$. Specifically, let's try to interact the coefficient of $x^{-1}$. 
From the first term, this is going to be 
\[
    \ep_1 \ep_2 \abs{ \lambda } + (x + \mu_1 + \mu_2 + \mu_1 \mu_2 x^{-1})(1 - \frac{\ep_1 \ep_2 \abs{\lambda}}{x^2}) = \ep_1' \ep_2 \abs{\lambda} + \mu_1 \mu_2 -\ep_1 \ep_2)\abs{q}.
\]
From this, we derive a differential equoation for $Z$:
\[
\ep_1 \ep_2 (1-q) q \dv{q} Z + \mu_1 \mu_2 q Z = 0,
\]
gotten from 
\[
\ep_1 \ep_2 (1 - q) <|\lambda|> + \mu_1 \mu_2 q <1> = 0.
\]
Define
\[
\int_{\mathbb{C}^2} 1 = (\ep_1 \ep_2)^{-1} 
\]
formally. However, we can do better. Take the Gaussian
\[
\int_{\mathbb{C}^2} e^{-k_1 \abs{ k_1 }^2 + \overline{\ep}_1 \dd z_1 \wedge \dd\overline{z}_1 + \cdots}  = (\ep_1 \ep_2)^{-1}.
\]
The main point is that you can use tricks like this to get rid of all degree 1 nodes.
\end{exm}

\begin{exm}
Consider the example of $D_4$-theory. Write the shape symmetrically, so that there's one central node with $n_i = 2$, three rank one 
gauge nodes and a framing node also of rank 1 attached.
Also give the framing node mass $m$ and give gauge node $i$ the fugacity $q_i$, where $0$ is the label of the central node. 
Then this can be amputated in the sense that 
\[
Z_{D_4}(\vec{a},\vec{q}, m) = \sum Z_{A_1}(a_1,a_2;m_1 = a_1, m_2 = a_2,\dots, q),
\]
where
\[
q = q_0 \prod \frac{1 - q_i}{1 - q_0 q_i} \frac{1 - q_0^2 q_1 q_2 q_3}{1- q_0q_1q_2 q_3}
\]
(think of $q = q_0 + \text{small correction}$).
\end{exm}

\begin{rmk}
  This is related to the phenomenon known as \emph{bubbling}. In an ideal world we would be integrating over actual moduli spaces of instantons, but 
  here we have compactified these moduli spaces by relaxing the condition that they correspond to actual vector bundles.
  The price we pay is that the rank 1 story is already nontrivial, even though Maxwell theory doesn't admit instantons on manifolds without holes. 

  Nonetheless, say that you performed a blowup, luing in a small $\mathbb{P}^1$. Then this manifold has nontrivial $b_2$, i.e. it has holes. 
  Then these solve Maxwe'll's equations, but from differential geometeric point of view you have a family of metrics that diverges as the exceptional divisor goes to 0. 
  Unfortunately it's not defined at a point unless you blow up this point.
\end{rmk}

\begin{rmk}
This is somehow related to \emph{mirror maps}. 
\end{rmk}

\begin{exm}
Finally, let's consider the example of $\widehat{A}_0$ theory. It's related to earlier work of Nekrasov-Okounkov.
In full generality, 
\[
Z_{\widehat{A}_0}(q; \ep_3:\ep_1 : \ep_2) = \sum_\lambda  q^{|\lambda|} \prod_{\square \in \lambda} \prod \frac{(\ep_1(a_\square + 1) - \ep_2 \ell_\square + \ep_3)(-\ep_1 a_\square + \ep_2(\ell_\square + 1) + \ep_3)}{(\ep_1(a_\square + 1) - \ep_2 \ell_\square)(-\ep_1 a_\square + \ep_2(\ell_\square + 1))}.
\]
Of course, 
\[
    Z = \sum q^k \int_{\mr{Hilb}^k \mathbb{C}^2} c(T^* \mr{Hilb}^k \mathbb{C}^2,\ep_3).
\]
Setting $\mu^2 = \ep_3^2/\ep_1^2$ and $\ep_1 = - \ep_2$, the sum becomes something like 
\[
\prod_\square (1 - \mu^2/h_\square^2),
\]
which vanishes if $\mu = h_\square$. 
We will prove some formula like 
\[
Z = \phi(q)^{-(\ep_1 + \ep_3)(\ep_2 + \ep_3)/(\ep_1 \ep_2)}.
\]

The fundamental qq-character of this theory is 
\begin{align*}
    X(x) ={}& Y(x + \ep_1 + \ep_2) + q\frac{Y(x + \ep_3)Y(x + \ep_4)}{Y(x)} + \dots \\ 
    ={}& \sum_\lambda q^{|\lambda|} \left( \prod_{\square \in \lambda} \frac{(\ep_3 (a_\square +1) - \ep_4 \ell_\square + \ep_1)(-\ep_3 a_\square + \ep_4(\ell_\square + 1) + \ep_1)}{(\ep_3(a_\square + 1) - \ep_4 \ell_\square)(-\ep_3 a_\square + \ep_1(\ell_\square + 1))} \right. \\
    &\times Y(x + \ep_1 + \ep_2)  \\
    &\times \left. \prod_{\square \in \lambda}\frac{Y(x + \sigma_\square + \ep_4(Y(x + \sigma_\square - \ep_1))}{Y(x + \sigma_\square + \ep_1 + \ep_2) Y(x + \sigma_\square)} \right).
\end{align*}
This is a complicated formula, but we don't want to study the full story; we just want to extract its $x^{-1}$ coefficient. 
The result is that 
\begin{align*}
    0 &= \ev{[x^{-1}] X(x) } \\
    &= \ep_1 \ep_2 Z(q, \ep_1 : \ep_4 : \ep_3)\ev{\abs{\lambda^{(1)}} }+ \ep_3 \ep_4 q \dv{q} Z(q; \ep_1,\ep_4,\ep_3) \ev{1}.
\end{align*}
This is already a differential equation for $Z$. 

But the partitions $\lambda^{(1)}$ describe certain ideals in $\mathbb{C}{[z_1,z_2]}$, i.e. they parametrise $0$-dimensional subschemes in 
a plane with coordinates $\ep_1,\ep_2$. 
However, somehow in this story a second plane emerged with equivariant parameters $\ep_3,\ep_4$ whose points were also related to $\lambda$.
This relation says there is some sort of correspondence between instantons on either plane. 
This is a transverse intersection of two planes in 4d, and there is no flat deformation of this cross into a smooth surface. 
So there is something useful coming out of this. This whole structure is called \emph{crossed instantons}.

We don't know what crossed instantons \emph{are}, even though we do know what their moduli space is. 
Potentially, $Z$ could be singular when $\ep_2/\ep_1$ is a positive rational number or 0 or $\infty$. 
However no poles appear in $\ep_3$. Moreover, if $\ep_3 = -\ep_1,\ep_3 = - \ep_2$, then this whole 
thing collapses because if the partition is nontrivial you have boxes for which the armlength is zero. 
So we may write 
\[
Z = 1 + \frac{(\ep_1 + \ep_3)(\ep_2 + \ep_3)}{\ep_1 \ep_2} Z_{\mr{regular}},
\]
where $Z_{\mr{regular}}$ is regular.  
We can more naturally write this as an equation of logarithms as
\[
\ep_1 \ep_2 + q \dv{q} \log Z (q,\ep_3, \ep_4, \ep_2) = - \ep_3 \ep_4 q \dv{q} \log Z(q, \ep_1,\ep_4,\ep_3),
\]
which means that if we define 
\[
\Phi(q,\ep_3,\ep_1,\ep_2) = \frac{\ep_1 \ep_2}{(\ep_1 + \ep_3)(\ep_2 + \ep_3)} \log Z(q, \ep_3, \ep_1,\ep_2)
\]
and note that 
\[
(\ep_1 + \ep_3)(\ep_2 + \ep_3) = (\ep_1 + \ep_4)(\ep_2 + \ep_4),
\]
the denominator is symmetric under exchanging $\ep_3,\ep_4$. Further in the small $q$ expansion we have no singularities. 
Further, the differential equation we wrote above implies we can exchange the $\ep_i$ to obtain
\[
\Phi(q,\ep_3,\ep_1,\ep_2) = \Phi(q, \ep_1, \ep_4, \ep_3).
\]
This is a meromorphic function on $\mathbb{P}^2$ with no singularities, therefore it is constant in the $\ep$. It suffices to evaluate it at any point. 
We find that it's $= - \log \phi(q)$, where $\phi$ is the Euler function. 
Many years ago, Nekrasov-Okounkov proved it by representing this formula as a trace. That proof was much more complicated 
than this qq-character method. 
\end{exm}

\begin{exm}
Consider the case of an $A_r$ quiver with all gauge nodes rank $1$ and framing nodes of rank 1 at either end. 
Add Coulomb parameters $\mu_1,\mu_r, a^{(i)}$ and sometimes write $\mu_1 = a^0, \mu_{r} = a^{r + 1}$ for more symmetric notation. 
Then, 
\[
Z[\vec{a}.\vec{q},\ep_1,\ep_2] = \sum_{\lambda^1,\dots,\lambda^r} \cdots 
\]
There are $r$ fundamental $qq$-characters in this case. The result is that 
\[
Z_{A_r} = \prod_{0 \leq j < i \leq r} (1 - z_i/z_j)^{-p_i^+ p_j^-/(\ep_1\ep_2)},
\]
where $p_i^+ = \ep_1 + \ep_2 + a_i - a_{i+1}$, $p_i^- = a_i - a_{i+1}$. 

Here, we pass to parameters $z_i$ so that $z_i/z_{i - 1} = q_i$. 
What this formula should make us think about is that a chiral block of a partition function of a Gaussian free field living on a sphere 
with vertex operators inserted at points $z_i$. 
This would, for example, be a relevant computation in string theory with strings coming from infinity somehow. 
The pieces that go off to infinity would be represented by a vertex operator.
\end{exm}

\begin{rmk}
Studying a Gaussian free field 
\[
\exp -S = e^{-\int_\Sigma \partial X \overline{\partial} X},
\]
the Green's function of the Laplacian, for $\Sigma$ a plane, is 
\[
\ev{XX} = - \log \abs{ z_1 - z_2 }^2.
\]
There's a small issue since this quantity isn't dimensionless, which is related to the fact that the constant/zero mode does not enter 
into the action and so we must make a choice. In principle, we should add 
\[
\ev{XX} = - \log \abs{ z_1 - z_2 }^2 + \text{harmonic}.
\]
Then we'd like to define vertex operators 
\[
:e^{i p X(\overline{z},z)}:,
\]
which give us 
\begin{align*}
    \ev{ \prod_{i = 0}^r : e^{i p X(\overline{z}_i,z_i)}: } &= \delta\qty(\sum p) \prod_{i \neq j} \exp(-p_i p_j G(z_i,\overline{z}_i, z_j,\overline{z_j})) \\
    &=\abs{ \Psi }^2 \\
    &= \abs{\mr{Analytic}}^2,
\end{align*}
where $G$ is the Green's function.
The analytic piece $\Psi$ will look very much like our expression for $Z$ except 
$Z$ has two types of momenta, unless $\ep_1 = -\ep_2$.

One can slightly modify the Gaussian free field action to include a term 
\[
-S = - \int_\Sigma \partial X \overline{\partial} X + a^{(2)} RX 
\]
depending on the scalar curvature. 
Having this scalar term with some coefficient modifies our computations, and can be fine-tuned to exactly reproduce our formula for $Z_{A_r}$.
\end{rmk}

\begin{rmk}
Let us list the topics which we have not covered yet:
\begin{enumerate}
\item Spiked (and the special cases of crossed and folded) instanton moduli spaces. Their study goes under the name of \emph{gauge origami}. It's a kind of model-building 
device which builds statistical models of random partitions which can also be interpreted as tools to produce natural 
observables, e.g. $qq$-characters, defects, etc. 
\item Compactness theorem (hence, regularity of expectation values of observables).
\item $N > 1$ cases where one can write closed (system) of PDEs on 
the partition function $Z$, or on some expectation values $\ev{\cdots}$. 

It turns out these equations, which often are derived more-or-less combinatorially,
are actually the same as the BPZ equations (Belavin-Polyakov-Zamolodchikov) in 2d CFT, 
or the KZ equations (Knizhnik-Zamolodchikov). This is one 
of the best incarnations of the somewhat mysterious, but 
nowadays better understood, correspondence known as \emph{the BPS/CFT correspondence}.

Since the world of CFT includes representation of things like 
Kac-Moody algebras, and the BPS world includes things like Donaldson invariants,
this is a very powerful relation. 
\end{enumerate}
\end{rmk}


\section{Gauge origami}

Previously, we explained that the basic example of the 
$qq$-character is an observable in the statistical mechanical 
model wherein we sum over $n$-tuples of partitions 
\[
\lambda^{(1)},\dots, \lambda^{(n)}
\]
which represent a subset of $\mathbb{C}$ of the form 
$\{ a_\alpha + \epsilon_1(i - 1) + \epsilon_2 (j - 1) \mid \alpha \in 1,\dots, N ; (i,j) \in \lambda^{(\alpha)}\}$.

We discovered that a $qq$-character $\chi(x)$, which is an observable 
for this model, is described by a formula like 
\[
\sum_{\mu^{(0)}, \dots,\mu^{(N)}} \cdots
\]
wherein the sum labelled a subset of $\mathbb{C}$ of the form 
\[
\{ b_\beta + \epsilon_3(i-1) + \epsilon_4(j-1) \mid \beta \in 1,\dots, N ; (i,j) \in \mu^{(\beta)}\}.
\]
However, if we tried to trace the algebrogeometric origin of these 
two subsets, it was different: 
the first subset came by doing gauge theory on the plane described by $z_1 z_2$
whilst the second subset was in fact derived 
from a partition function on a transversal plane $z_3 z_4$. 

The only connection between these two planes is that $\epsilon_1 + \epsilon_2 + \epsilon_3 + \epsilon_4 = 0$.
So there's a 3d torus acting on a 4d space relating them. 
We now aim to present a more unified story where we show that 
the qq-character is the result of projecting a moduli space 
involving both planes onto one of the planes. 
Let us proceed to define that. Unfortunately, the story is local, 
although there are many indications it can be globalised.

\subsection{Local story}

Think about the union of $\mathbb{C}^2_A = \cup_{A \in \sigma} n_A \mathbb{C}_A^2$
wherein $A \in \{12,13,14,23,24,34\}$, $|A| = 6 $ labels the 6 coordinate 
hyperplanes, and $n_A \geq 0$ are the possible integer multiplicities of the hyperplanes. 

We view 
\[
\bigcup_\sigma n_A \mathbb{C}_A^2 \subset \mathbb{C}^4_{1234}
\]
and act on it with a 3d torus. We seek to understand the $n_A$ in 
terms of certain sheaves.
As before, fix $K = \mathbb{C}^k, N_a = \mathbb{C}^{n_A}$.
Pictorially we can draw a tetrahedron with vertices $1,2,3,4$,
so that the edges label pairs of vertices, hence 2d planes. 
So attach to an edge $A$ the number $n_A$.  Within the tetrahedron
draw an interior ball labelled by $K$ with arrows $I_{A},J_A$ from and to edges to the ball. 
We also label endomorphisms of $K$, $B_i$. 

This is some homogenised version of the ADHM construction, 
however in the ADHM construction we only have one multiplit=cty 
so in that setting one doesn't see the point of all the operators 
we have introduced above.
To start, let's define an operator for $A = ab$ with $a < b$:
\[
\mu_A = [B_a, B_b] + I_A J_A .
\]
Then the equations look like 
\begin{align*}
    \mu_{12} + \mu_{34}^+ &= 0 \\
    \mu_{13} - \mu_{24}^+ &= 0 \\
    \mu_{14} + \mu_{23}^+ &= 0. 
\end{align*}
These equations are valued in $\End(K) \otimes \mathbb{C}^3$,
but we may also represent that as taking values in six copies of 
$\Lie(U(k)) \otimes \mathbb{R}^6$ by endowing $K$ with Hermition structure. 
The $\mathbb{R}^6$ here is actually an irreducible representation of $SU(4)$.

Furthermore we will impose a unifying equation: 
\[
\sum_{a \in \{1,2,3,4\}} [B_a, B_a^\dag] + \sum_{A \in \{12,13,\dots\}} I_A I_A^\dag - J_A^\dag J_A = \zeta \mr{id}_{K}
\]
Why is seven important? It's $8 -1$. There are $8$ hermition 
operators acting on $K$. Modding out by the action of $U(K)$ by 
\[
(B,I,J) \to (g^{-1} B g, g^{-1} I, J g),
\]
we seek to introduce as many equations as variables.


The intuition is the following: if we draw a 'spiked' picture where 
we connect the $z_{12}, z_{23}$ planes by another plane which bounds them, 
the $I$ somehow 'injects' this plane. Since the $B_a$ are roughly 
operators of multiplicationby the $z_a$. 
You can ask: why are we writing the $\mu_i$ in a sort of strange way, 
$mu_{12} + \mu_{34}^\dag = 0$?
There is a vaninishing theorem which says

\begin{thm}[Vanishing theorem]
The $\mu_A = 0$ all vanish individually, the $B_a I_A = 0$ if $a \not \in A$, and $J_A b_a = 0$. 
\end{thm}
The way this is proven is by squaring the equations.
To do everything algebraically, we would have to introduce derived geometry at this point.
But the differential geometry is a little easier. 

\begin{rmk}
This story is a precursor to generalising Donaldson-Thomas theory to CY4s. It has 
to do with how we really want to integrate the Pfaffian, and we somehow do this by imposing half the equations.
\end{rmk}

Ultimately, we find that 
\[
K = \sum_{a<b} \mathbb{C}[B_a,B_b] I_{ab}(N_{ab}) /GL(K).
\]
This is to be conmpared with the ordinary ADHM construction 
\[
\overline{\mathcal{M}}_k(n) = \{ (B_1,B_2,I_{12}, J_{12}) | \mu_{12} = 0, K = \mathbb{C}[B_1,B_2]I_{12}(N_{12})\}/GL(K).
\]

\begin{rmk}
The subspaces $K_{ab} = \mathbb{C}[B_a,B_b] I_{ab}(N_{ab})$
can overlap. 
For example, $B_3 K_{12} = 0$, and $B_1 K_{12} = 0$, 
so if $K_{34} \cap K_{12} \neq \emptyset$, the overlap (call it $\tilde{K}_{12,34}$) 
must satisfy $B_3 \tilde{K}_{12,34} = 0$, etc.

In a box picture, the corner boxes correspond to $\ker(B_1) \cap \ker(B_2)$. 
So elements in $\tilde{K}_{12,34}$ must come from corner boxes. 
The situation relevant for crossed instantons is when $n_{12} n_{34} \neq 0$, and rest $= 0$. 
To take a shortcut, just look at the fixed points of the torus action 
\[
K_{12}= \on{span}( B_1^{i-1} B_2^{j-1} I_{12} (N_{12,\alpha})) = e_{ij, \alpha},
\]
which is associated to the box $ij \in \lambda^{(\alpha)}$.
The vectors which conceivably could belong to $K_{12} \cap K_{34}$ here are those annihilated by $B_1,B_2$,
again the corner boxes. 

On the other hand, look at the corner boxes of $K_{34}$. 
Generically these don't intersect, but \emph{nongenerically},
these cornes can become 1 or become linearly dependent. 
They will form a 'butterfly configuration' where some corner boxes overlap. 
This will happen when we choose the equivariant torus to be a smaller subtorus, and 
hence have some relation between the equivariant parameters. 
That is, we have some equation like 
\[
a_\alpha + \epsilon_1 (i -1) + \epsilon_2(j-1) = b_\beta + \epsilon_3(i' - 1) + \epsilon_4(j' - 1),
\]
which is, notably, one of the conditions for a qq-character to have a pole. 
\end{rmk}

\begin{exm}
Set $k = 1, n_{12} = n, n_{34} = m$. Then the $B_a \in \mathbb{C}$; since they are just 
numbers, they drop out of our equations. 
Therefore we just have equations 
\begin{align*}
    I_{12} J_{12} &= 0
\\
    I_{34} J_{34} &= 0,
\end{align*}
where $I_{12} : \mathbb{C}^2 \to \mathbb{C}, J_{12} : \mathbb{C} \to \mathbb{C}^n, I_{34} : \mathbb{C}^m \to \mathbb{C}, J_{34} : \mathbb{C} \to \mathbb{C}^m$.
 Further there are lots of equations like $B_4 J_{12} = 0, \dots$. 
 There is also a stability condition 
 stating that there exist $\lambda,\mu \in \mathbb{C}$ so that $\lambda_{12} + \mu I_{34} \neq 0$. 

Let's start by assuming that at least one of $B_1,B_2 \neq 0$. 
On this branch, $J_{34}, I_{34} = 0$ by our assumption that one of the $B_i$ doesn't vanish.
Hence the stability condition implies $I_{12} \neq 0$. 
Working this out, we get a copy of $T^\star \mathbb{P}^{n-1}$. 
So one branch of the moduli space is $\mathbb{C}^2_{(B_1,B_2)} \times T^\star \mathbb{P}^{n-1}$.
The other branch of the moduli space comes when one of $(B_3,B_4) \neq 0$. On this branch, 
\[
\mathbb{C}^2_{(B_3,B_4)} \times T^\star \mathbb{P}^{m-1}.
\]
Further, there is another branch wherein all the $B_i$ are zero, and 
we get a copy of $T^\star \mathbb{P}^{n+ m- 1}$ 
which parametrises relations between the $I_{12}, I_{34}$. 
\end{exm}

\begin{rmk}
In general, let $\overline{\mathcal{M}}_k(\vec{n})$ denote \emph{the moduli space of spiked instantons}
which depends on $\vec{n} = (n_{12}, n_{13},\dots)$.

There is then a function 
\[
\vec{k} : \overline{\mathcal{M}}_k (\vec{n}) \to \mathbb{Z}^b_{\geq 0}
\]
wherein $\vec{k}(B_i, I_A, J_A) = (k_{ab} =  \dim(K_{ab}))$.

We can also map 
\[
\overline{\mathcal{M}}_k(\vec{n}) \to \prod_{\{a,b\} \in \{12,13,\dots\}} \overline{\mathcal{M}_{k_{ab}}}(n_{ab}).
\]
Then, 
\[
Z_{\widehat{A}_0} = \sum_{k = 0}^\infty q^k \int_{\overline{\mathcal{M}}_k(n)} c(T^\vee \overline{\mathcal{M}}_k(n), \epsilon_3) = \sum \int_{\overline{\mathcal{M}}_k(n,0,0,0,0,)}  q^k.
\]

Further, 
\[
\ev{\chi(x)} = \sum_{k} q^k \int_{\overline{\mathcal{M}}_k(n_{12} = n,0,0,0,0,n_{34} = 1)} 1
\]
Then what is the variable $x$? It has to do with the framing symmetry.
We have groups $\prod_A U(n_A)$ which act only on the $(I,J)$. The diagonal subgroups act trivially except $x$ acts nontrivially. 
Here $h$ in equivariant parameter.
\end{rmk}
\begin{rmk}
  Consider $U(1)^3 \subset U(K)$ which acts by multiplying the $B_a \to e^{i\phi_a} B_a$. 
Let 
\[ 
\mathbb{H} = \mathbb{P} \qty(\prod_A U(n_A)) \times G_{\mr{init}}.
    \]
\end{rmk}


Consider the orbifold classification by $\Gamma \subset H$. We 
get answers of type ADE. An interesting case is $\Gamma \subset SU(2)_{34}$. 
To be specific, we need to embed $\Gamma$ into $H$ by specifying all the decompositions
\[
N_{12} = \oplus_{\omega \in \mathbb{C}^\vee} N_{12, \omega}\otimes R_{\omega_1},
\]
and similarly for $N_{34}$, 
where $dim N{12,\omega} = \vec{n}_{12}$, for example. 
Now $\overline{\mathcal{M}}_k$ is connected, but its fixed point locus might not be. It splits, 
\[
\overline{\mathcal{M}}_k^\Gamma = \bigcup_{\vec{k}} \mathcal{M}_{\vec{k}}^\Gamma (n_{12}, n_{34})
\]
which gives us more parameters for our basic counting problem. 
That is, the generating function will now depend on many additional fugacities in addition to our other parameters: 
\[
    Z^\Gamma(q,a,\epsilon) = \sum_{\vec{k}} \prod_\omega q_\omega^{k_\omega} \int_{\mathcal{M}_{\vec{k}}^\Gamma} 1 = \ev{\text{general qq-character of quiver gauge theory}}.
\]
One can reverse the roles of $n_{12},n_{34}$. Then, the same calculation will be computing some integrals over the moduli space of instantons 
related to the work of Kronheimer-Nakajima. 

\end{document} 




%%% Local Variables:
%%% mode: latex
%%% TeX-master: t
%%% End:
