\documentclass[leqno, openany]{memoir}
\setulmarginsandblock{3.5cm}{3.5cm}{*}
\setlrmarginsandblock{3cm}{3.5cm}{*}
\checkandfixthelayout

\usepackage{amsmath}
\usepackage{amssymb}
\usepackage{amsthm}
%\usepackage{MnSymbol}
\usepackage{bm}
\usepackage{accents}
\usepackage{mathtools}
\usepackage{tikz}
\usetikzlibrary{calc}
\usetikzlibrary{automata,positioning}
\usepackage{tikz-cd}
\usepackage{forest}
\usepackage{braket} 
\usepackage{listings}
\usepackage{mdframed}
\usepackage{verbatim}
\usepackage{physics}
\usepackage{stmaryrd}
%\usepackage{/home/patrickl/homework/macaulay2}

%font
\usepackage[osf]{mathpazo}
\usepackage{microtype}

%CS packages
\usepackage{algorithmicx}
\usepackage{algpseudocode}
\usepackage{algorithm}

% typeset and bib
\usepackage[english]{babel} 
\usepackage[utf8]{inputenc} 
\usepackage[backend=biber, style=alphabetic]{biblatex}
\usepackage[bookmarks, colorlinks, breaklinks]{hyperref} 
\hypersetup{linkcolor=black,citecolor=black,filecolor=black,urlcolor=black}

% other formatting packages
\usepackage{float}
\usepackage{booktabs}
\usepackage{enumitem}
\usepackage{csquotes}
\usepackage{titlesec}
\usepackage{titling}
\usepackage{fancyhdr}
\usepackage{lastpage}
\usepackage{parskip}

\usepackage{lipsum}

% delimiters
\DeclarePairedDelimiter{\gen}{\langle}{\rangle}
\DeclarePairedDelimiter{\floor}{\lfloor}{\rfloor}
\DeclarePairedDelimiter{\ceil}{\lceil}{\rceil}


\newtheorem{thm}{Theorem}[section]
\newtheorem{cor}[thm]{Corollary}
\newtheorem{prop}[thm]{Proposition}
\newtheorem{lem}[thm]{Lemma}
\newtheorem{conj}[thm]{Conjecture}
\newtheorem{quest}[thm]{Question}

\theoremstyle{definition}
\newtheorem{defn}[thm]{Definition}
\newtheorem{defns}[thm]{Definitions}
\newtheorem{con}[thm]{Construction}
\newtheorem{exm}[thm]{Example}
\newtheorem{exms}[thm]{Examples}
\newtheorem{notn}[thm]{Notation}
\newtheorem{notns}[thm]{Notations}
\newtheorem{addm}[thm]{Addendum}
\newtheorem{exer}[thm]{Exercise}

\theoremstyle{remark}
\newtheorem{rmk}[thm]{Remark}
\newtheorem{rmks}[thm]{Remarks}
\newtheorem{warn}[thm]{Warning}
\newtheorem{sch}[thm]{Scholium}


% unnumbered theorems
\theoremstyle{plain}
\newtheorem*{thm*}{Theorem}
\newtheorem*{prop*}{Proposition}
\newtheorem*{lem*}{Lemma}
\newtheorem*{cor*}{Corollary}
\newtheorem*{conj*}{Conjecture}

% unnumbered definitions
\theoremstyle{definition}
\newtheorem*{defn*}{Definition}
\newtheorem*{exer*}{Exercise}
\newtheorem*{defns*}{Definitions}
\newtheorem*{con*}{Construction}
\newtheorem*{exm*}{Example}
\newtheorem*{exms*}{Examples}
\newtheorem*{notn*}{Notation}
\newtheorem*{notns*}{Notations}
\newtheorem*{addm*}{Addendum}


\theoremstyle{remark}
\newtheorem*{rmk*}{Remark}

% shortcuts
\newcommand{\Ima}{\mathrm{Im}}
\newcommand{\A}{\mathbb{A}}
\newcommand{\N}{\mathbb{N}}
\newcommand{\R}{\mathbb{R}}
\newcommand{\C}{\mathbb{C}}
\newcommand{\Z}{\mathbb{Z}}
\newcommand{\Q}{\mathbb{Q}}
\renewcommand{\k}{\Bbbk}
\renewcommand{\P}{\mathbb{P}}
\newcommand{\M}{\overline{M}}
\newcommand{\g}{\mathfrak{g}}
\newcommand{\h}{\mathfrak{h}}
\newcommand{\n}{\mathfrak{n}}
\renewcommand{\b}{\mathfrak{b}}
\newcommand{\ep}{\varepsilon}
\newcommand*{\dt}[1]{%
   \accentset{\mbox{\Huge\bfseries .}}{#1}}
\renewcommand{\abstractname}{Official Description}
\newcommand{\mc}[1]{\mathcal{#1}}
\newcommand{\T}{\mathbb{T}}
\newcommand{\mf}[1]{\mathfrak{#1}}
\newcommand{\mr}[1]{\mathrm{#1}}
\newcommand{\ms}[1]{\mathsf{#1}}
\newcommand{\ol}[1]{\overline{#1}}
\newcommand{\wt}[1]{\widetilde{#1}}
\newcommand{\wh}[1]{\widehat{#1}}

\DeclareMathOperator{\Der}{Der}
\DeclareMathOperator{\Hom}{Hom}
\DeclareMathOperator{\End}{End}
\DeclareMathOperator{\ad}{ad}
\DeclareMathOperator{\Aut}{Aut}
\DeclareMathOperator{\Rad}{Rad}
\DeclareMathOperator{\Pic}{Pic}
\DeclareMathOperator{\supp}{supp}
\DeclareMathOperator{\sgn}{sgn}
\DeclareMathOperator{\spec}{Spec}
\DeclareMathOperator{\Spec}{Spec}
\DeclareMathOperator{\proj}{Proj}
\DeclareMathOperator{\Proj}{Proj}

% Section formatting
\titleformat{\section}
    {\Large\sffamily\scshape\bfseries}{\thesection}{1em}{}
\titleformat{\subsection}[runin]
    {\large\sffamily\bfseries}{\thesubsection}{1em}{}
\titleformat{\subsubsection}[runin]{\normalfont\itshape}{\thesubsubsection}{1em}{}

\title{COURSE TITLE}
\author{Lectures by INSTRUCTOR, Notes by NOTETAKER}
\date{SEMESTER}

\newcommand*{\titleSW}
    {\begingroup% Story of Writing
    \raggedleft
    \vspace*{\baselineskip}
    {\Huge\itshape FGA Explained Learning Seminar \\ Fall 2020}\\[\baselineskip]
    {\large\itshape Notes by Patrick Lei}\\[0.2\textheight]
    {\Large Lectures by Various}\par
    \vfill
    {\Large \sffamily Columbia University}
    \vspace*{\baselineskip}
\endgroup}
\pagestyle{simple}

\chapterstyle{ell}


%\renewcommand{\cftchapterpagefont}{}
\renewcommand\cftchapterfont{\sffamily}
\renewcommand\cftsectionfont{\scshape}
\renewcommand*{\cftchapterleader}{}
\renewcommand*{\cftsectionleader}{}
\renewcommand*{\cftsubsectionleader}{}
\renewcommand*{\cftchapterformatpnum}[1]{~\textbullet~#1}
\renewcommand*{\cftsectionformatpnum}[1]{~\textbullet~#1}
\renewcommand*{\cftsubsectionformatpnum}[1]{~\textbullet~#1}
\renewcommand{\cftchapterafterpnum}{\cftparfillskip}
\renewcommand{\cftsectionafterpnum}{\cftparfillskip}
\renewcommand{\cftsubsectionafterpnum}{\cftparfillskip}
\setrmarg{3.55em plus 1fil}
\setsecnumdepth{subsection}
\maxsecnumdepth{subsection}
\settocdepth{subsection}

\begin{document}
    
\begin{titlingpage}
\titleSW
\end{titlingpage}

\thispagestyle{empty}
\section*{Disclaimer}%
\label{sec:disclaimer}

These notes were taken during the seminar using the \texttt{vimtex} package of the editor \texttt{neovim}. 
Any errors are mine and not the speakers'. 
In addition, my notes are picture-free (but will include commutative diagrams) and are a mix of my mathematical style and that of the lecturers.
If you find any errors, please contact me at \texttt{plei@math.columbia.edu}.

\vspace*{1cm}

\noindent\textbf{Seminar Website:}  \url{https://www.math.columbia.edu/~calebji/fga.html}
\newpage


\tableofcontents

\chapter{Caleb (Oct 16): Representable Functors and Grothendieck Topologies}%
\label{cha:caleb_oct_16_representable_functors_and_grothendieck_topologies}

\section{Representable Functors}%
\label{sec:representable_functors}

We will always denote categories by $C$.
\begin{defn}
    Given an object $x \in C$, define the functor $h_X \colon C^{\mr{op}} \to \ms{Set}$ by $h_x = \Hom(-,X)$.
\end{defn}

Any morphism $f \colon x \to y$ induces a natural transformation $h_f \colon h_x \to h_y$. By the Yoneda lemma, this correspondence is bijective.

\begin{lem}[Yoneda Lemma]
    Let $x \in C$ and $F \colon C^{\mr{op}} \to \ms{Set}$ be a functor. Then $\Hom(h_x,F) \simeq F(x)$.
\end{lem}

\begin{proof}
    Let $\theta \colon h_x \to F$. This gives a map $\theta_x \colon h_x(x) \to F(x)$, and we can consider $\mr{id} \to \theta_x(\mr{id})$. Now given $t \in F(x)$, we need $h_x(U) \to F(U)$. Given $U \to x$, then we have a map $F(x) \to F(U)$ and then $t \mapsto F_f(t)$. We can check that these are inverses.
\end{proof}

\begin{defn}
    A functor $F \colon C^{\mf{op}} \to \ms{Set}$ is \textit{representable} if it is naturally isomorphic to $h_x$ for some $x$. 
\end{defn}

\begin{defn}
    If $F$ is a presheaf, a \textit{universal object} for $F$ is a pair $(X, \xi)$ such that $\xi \in FX$ and for any $(U, \sigma)$ where $\sigma \in FU$, there exists a unique $f \colon U \to X$ such that $F_f(\xi) = \sigma$.
\end{defn}

Note that representability is equivalent to having a universal object.

\begin{exm}
    \begin{enumerate}
        \item For the first example, consider $C = \ms{Sch}/R$ for some ring $R$. Then if $F = \Gamma(\mc{O})$, then clearly this is isomorphic to $h_{\A^1}$ and the universal object is $(\A^1, x)$.
        \item Let $F(X) = \qty{\mc{L}, s_0, \ldots, s_n}$ where $\mc{L}$ is a line bundle and $s_0, \ldots, s_n$ generate $\mc{L}$, then $(\P^n, x_0, \ldots, x_n)$ is a universal object.
    \end{enumerate}
\end{exm}

\section{Grothendieck Topologies}%
\label{sec:grothendieck_topologies}

According to Wikipedia, this is supposed to be a pun on ``Riemann surface.'' We want to generalize the idea of a topology because the Zariski topology is awful. Instead of open sets, we will consider suitable maps (coverings).

\begin{defn}
    A \textit{Grothendieck topology} on a category $C$ is a specification of \textit{coverings} $\qty{U_i \to U}$ of $U$ for each $U \in C$. Here are the axioms for coverings:
    \begin{enumerate}
        \item If $V \to U$ is an isomorphism, then $\qty{V \to U}$ is a covering.
        \item If $\qty{U_i \to U}$ is a covering, for all $V \to U$, the fiber products $\qty{ U_i \times_U V \to V }$ and form a covering of $V$.
        \item If $\qty{U_i \to U}$ is a covering and $\qty{V_{ij} \to U_i}$ are coverings, then $\qty{V_{ij} \to U}$ is a covering of $U$.
    \end{enumerate}
\end{defn}

A category with a Grothendieck topology is called a \textit{site}.

\begin{exm}
    Here are some topological examples. Let $X$ be a topological space.
    \begin{enumerate}
        \item The site of $X$ is the poset category of open subsets of $X$. The fiber product is just the intersection, and a covering is a normal open covering.
        \item (Global classical topology) Let $C = \ms{Top}$. Here, the coverings are sets of open embeddings such that the union of the images covers the whole space.
        \item (Global \'etale topology) Here, $C = \ms{Top}$ and the coverings are now local homeomorphisms.
    \end{enumerate}
\end{exm}

Returning to schemes, we have several examples of Grothendieck topologies.
\begin{enumerate}
    \item (Global Zariski Topology). Let $C = \ms{Sch}$. The coverings are jointly surjective open embeddings.
    \item (Big \'etale site over $S$) The objects are schemes over $S$ and the morphisms are $S$-morphisms that are \'etale and locally of finite presentation.
    \item (Small \'etale site) This the same as the big \'etale site, but with the added requirement that $U \to S$ is also \'etale.
    \item (fppf topology) This stands for the French \textit{fid\`element plat et pr\'esentation finie}. The morphisms are $U_i \to U$ flat and locally of finite presentation. A covering is a set of jointly surjective morphisms such that the map $\bigsqcup U_i \to U$ is faithfully flat and of finite presentation. Note that flat and locally of finite presentation implies open.
    \item (fpqc topology) This stands for the French \textit{Fid\`element plat et quasi-compacte}. An \textit{fpqc} morphism is a morphism $X \to Y$ that is faithfully flat and one of the following equivalent conditions:
        \begin{enumerate}
            \item Every quasicompact open subset of $Y$ is the image of a quasicompact open subset of $X$.
            \item There exists an affine open cover $\qty{V_i}$ of $Y$ such that $V_i$ is the image of a quasicompact open subset of $X$.
            \item Given $x \in X$, there exists a neighborhood $U \ni x$ such that $f(U)$ is open in $Y$ and $U \to f(U)$ is quasicompact.
            \item Given $x \in X$, there exists a quasicompact open neighborhood $U \ni x$ such that $f(U)$ is open and affine in $Y$.
        \end{enumerate}
        The fpqc topology is given by maps $\qty{U_i \to U}$ such that $\bigsqcup U_i \to U$ is an fpqc morphism.
\end{enumerate}

To check that this is a topology, we have to do a lot of work. However, we will list some properties of fpqc morphisms and coverings.

\begin{prop}
    \begin{enumerate}
        \item The composition of fpqc morphisms is fpqc.
        \item Given $f \colon X \to Y$, if $f^{-1}(V_i) \to V_i$ is fpqc, then $f$ is fpqc.
        \item Open and faithfully flat implies fpqc. Moreover, faithfully flat and locally of finite presentation implies fpqc. This means that fppf implies fpqc.
        \item Base change preserves fpqc morphsisms.
        \item All fpqc morphsism are submersive. Thus $f^{-1}(V)$ is open if and only if $V$ is open.
    \end{enumerate}
\end{prop}

Note that Zariski is coarser than \'etale is coarser than fppf is coarser than fpqc.

\section{Sheaves on Sites}%
\label{sec:sheaves_on_sites}

Recall that a presheaf on a space is a functor $X_{\mr{cl}}^{\mr{op}} \to \ms{Set}$. Similarly, if $C$ is a site, then a presheaf is a functor $C^{\mr{op}} \to \ms{Set}$.

\begin{defn}
    A presheaf on a site $C$ is a \textit{sheaf} if
    \begin{enumerate}
        \item Given a covering $\qty{U_i \to U}$ and $a,b \in FU$ such that $p_i^*a = p_i^*b$, then $a = b$.
        \item Given a covering $\qty{U_i \to U}$ and $a_i \in FU_i$ such that $p_i^* a_j = p_j^* a_i$ (in the fiber product) for all $i,j$, there exists a unique $a \in FU$ such that $p_i^* a = a_i$.
    \end{enumerate}
\end{defn}

An alternative definition of a sheaf is that $FU \to \prod F U_i \rightrightarrows F(U_i \times_U U_j)$ is an equalizer.

\begin{thm}[Grothendieck]
    A representable functor on $\ms{Sch}/S$ is a sheaf in the fpqc topology.
\end{thm}

This means that given any fpqc cover $\qty{U_i \to U}$, then applying $h_X$, if we have $f_i \colon U_i \to X$ that glue on $U_i \times_X U_j \to X$, then the sheaf condition says we can glue to a unique $f \colon U \to X$. In the Zariski topology, this is trivial. This also means that the fpqc topology is \textit{subcanonical}, which means that $h_X$ are all sheaves.

    We will prove this result by reducing to the category of all schemes. Note that the topology on $\ms{Sch}/S$ comes from the topology on $\ms{Sch}$. Then we can show that if $C$ is subcanonical, then $C/S$ is subcanonical. Then we use the following lemma.

\begin{lem}
    Let $S$ be a scheme and $F \colon \ms{Sch}/S^{\mr{op}} \to \ms{Set}$ be a presheaf. If $F$ is a Zariski sheaf if $V \to U$ is a faithfully flat morphism of affine $S$-schemes, then $FU \to FV \rightrightarrows F(V \times_U V)$ is an equalizer, then $F$ is an fpqc sheaf.
\end{lem}

\begin{proof}
    Given $\qty{U_i \to U}$ an fpqc covering, let $V = \bigsqcup U_i$. Then consider the diagram
    \begin{equation*}
    \begin{tikzcd}
        FU \arrow{r} \arrow{d} & FV \arrow[shift left=1]{r} \arrow[shift right=1]{r} \arrow{d} & F(V \times_U V) \arrow{d} \\
        FU \arrow{r} & \prod FU_i \arrow[shift left=1]{r} \arrow[shift right=1]{r} & F(U_i \times_U U_j),
    \end{tikzcd}
    \end{equation*}
    the columns are bijective, so it suffices to check this for single coverings.   

    Now if $\qty{U_i \to U}$ are finite and all affine and the second assumption holds, we have the diagram
    \begin{equation*}
    \begin{tikzcd}
        FU \arrow{r} \arrow{d} & FV \arrow{d} \arrow[shift left=1]{r} \arrow[shift right = 1]{r} & F(V \times_U V) \\
        \prod_i F(U_i) \arrow{r} \arrow[shift left=1]{d} \arrow[shift right = 1]{d} & \prod_i \prod_a FU_{ia} \arrow[shift left=1]{r} \arrow[shift right = 1]{r}\arrow[shift left=1]{d} \arrow[shift right = 1]{d} & \prod_i \prod_{ab} F(V_{ia} \times_U V_{ib}) \\
        \prod_{ij} F(U_i \cap U_j) \arrow{r} & \prod_{ij} \prod_{ab} F(U_{ia} \cap U_{jb}).
    \end{tikzcd}
    \end{equation*}
    Then the middle row is an equalizer. 
\end{proof}

\begin{proof}[Proof of Theorem 1.3.2]
    If $X,U,V$ are affine, then we know that $\Hom(R,-)$ is left exact, so the result follows from commutative algebra. Now it suffices to check the general case for single covers. If $X = \bigcup X_i$ is a union of affines, then separatedness follows by restricting to the $X_i$ and using the affine case.

    Please read the rest of this yourself.
\end{proof}





\end{document}
