\documentclass[twoside, 10pt]{article}
\usepackage{../../../notes}
%\geometry{margin=2cm}
\newcommand{\F}{\mathbb{F}}
\renewcommand{\d}{\ \mathrm{d}}
\DeclareMathOperator{\Alt}{Alt}
\renewcommand{\H}{\mathbb{H}}
\mdfdefinestyle{default}{%
    linecolor=black,
    outerlinewidth=0.4pt,
    roundcorner=0pt,
    innertopmargin=\baselineskip,
    innerbottommargin=\baselineskip,
    innerrightmargin=\baselineskip,
    innerleftmargin=\baselineskip,
    backgroundcolor=white}


    \definecolor{darkblue}{RGB}{0,0,128}
    \definecolor{darkred}{RGB}{128,0,0}
    \definecolor{darkyellow}{RGB}{96,96,0}
    \definecolor{darkgreen}{RGB}{0,128,0}
    \definecolor{darkdarkred}{RGB}{64,0,0}
    
    

\lstset
{
    basicstyle=\ttfamily\scriptsize, 
    breaklines=true,
    postbreak=\mbox{\textcolor{darkdarkred}{$\hookrightarrow$}\space},
    showstringspaces=false
    keywordstyle = [1]\bfseries\color{darkred},
    keywordstyle = [2]\itshape\color{darkgreen},
    keywordstyle = [3]\sffamily\color{darkblue},
    keywordstyle = [4]\color{darkyellow},
}

\title{Math 703: Manifolds}
\author{Taught by Michael Sullivan; Notes by Patrick Lei}
\affil{University of Massachusetts, Amherst}
\date{Fall 2019}

\fancypagestyle{firstpage}
{
   \fancyhf{}
   \fancyfoot[R]{\itshape Page \thepage\ of \pageref{LastPage}}
   \renewcommand{\headrulewidth}{0pt}
}


\fancypagestyle{pages}
{
    \fancyhf{}
    \fancyhead[LO]{\scshape Math 718}
    \fancyhead[RO]{\scshape Lecture Notes}
    \fancyhead[CO]{\scshape Manifolds}
    \fancyhead[CE]{\scshape University of Massachusetts, Amherst}
    \fancyhead[LE]{\scshape Patrick Lei}
    \fancyhead[RE]{\scshape Fall 2019}
    \fancyfoot[RO,LE]{\itshape Page \thepage \space of \pageref{LastPage}}
    \renewcommand{\headrulewidth}{0.1pt}
}

\pagestyle{pages}

\begin{document}

    \maketitle\thispagestyle{firstpage}
    
    \begin{abstract}
        Topics to be covered: smooth manifolds, smooth maps, tangent vectors,
    vector fields, vector bundles (in particular, tangent and cotangent
bundles), submersions,immersions and embeddings, sub-manifolds, Lie groups and
Lie group actions, Whitney's theorems and transversality, tensors and tensor
fields, differential forms, orientations and integration on manifolds, The De
Rham Cohomology, integral curves and flows, Lie derivatives, The Frobenius
Theorem.  \end{abstract}

    \tableofcontents

    \section{Lecture 1 (Sep 4)}% \label{sec:lecture_1_sep_4_}

    \subsection{Overview}% \label{sub:overview_and_examples}
    
    
    We will be following Lee's book on smooth manifolds. Because this is no
    longer an exam class, there will be homework and a take home final. Prereqs
    are calculus, point-set topology and a little bit of $\pi_1$. Homework $0$
    will be to go to \url{math.stonybrook.edu/Videos/IMS/DifferentialTopology}
    or search for ``John Milnor 1965 Hedrick Lectures.''\footnote{Mike says he
    prepared this lecture and then watched these videos and thought ``I should
have taught it this way.''}\footnote{Milnor was also the person who invented
differential topology and won the Fields Medal.}

    
    We will give an overview. In topology, we have the notion of continuity
    with equivalence being either homeomorphism or homotopy equivalence. Then
    we specialize to differential topology by introducing smoothness. Inside
    differential topology, we have Riemannian geometry ($O(n)$), complex
    geometry ($GL_n(\C)$), and symplectic geometry ($Sp(2n)$). The intersection
    of these three things is K\"ahler geometry ($U(n)$).

    We will begin with some examples and nonexamples of manifolds.  \begin{exm}
    Some examples of manifolds are circles, spheres, tori, and $\R^n$.
\end{exm}

    \begin{exm} Some non-examples of manifolds are two intersecting lines, a
    sphere with a line attached, and the graph of $y = \abs{x}$.  \end{exm}

    We may also consider functions from a manifold to $\R$ and the consider the
    level sets. Then the preimages of generic values are manifolds, while at
    critical points, the preimages are not manifolds.\footnote{He just did
    Morse theory without saying it.} In addition, manifolds can be intersected
    transversally to form new manifolds.

    \subsection{Basic Notions and Examples}%
    \label{sub:basic_notions_and_examples}
    
    
    \begin{defn} A \textit{topological $n$-manifold} $M$ is a second-countable
    Hausdorff topological space $M$ that is locally Euclidean of dimension $n$.
\end{defn}

    \begin{defn} Two charts $(U, \varphi), (V, \psi)$ are smoothly compatible
    if the transition map $\psi \circ \varphi^{-1}$ is a diffeomorphism.
\end{defn}

    \begin{defn} A smooth atlas on $M$ is a collection of smoothly compatible
    charts $\mc{U} = \{(U_{\alpha}, \varphi_{\alpha})\}_{\alpha}$ that cover
$M$.  \end{defn}

    \begin{defn} A smooth structure on $M$ is a maximal smooth atlas.
    \end{defn}

    \begin{lem} Every smooth atlas on $M$ is contained in a unique maximal
    smooth atlas. Two smooth atlases determine the same smooth structure iff
their union is a smooth atlas.  \end{lem}

    \begin{exm} The simplest example of a manifold is $\R^n$ with the standard
        smooth structure.\footnote{This is misleading because there are
        uncontably many exotic $\R^4$.} We can also consider the set of
        matrices $M_{m \times n}(\R)$ identified with $R^{mn}$.  \end{exm}

    \begin{rmk} Hausdorffness and second-countability are preserved under both
    subspace and product topologies.  \end{rmk}

    \begin{exm} If $N \subseteq M$ is open, then $N$ is a smooth $n$-manifold.
    For example, we can consider $GL_n(\R) \subset M_n(\R)$.  \end{exm}

    \begin{exm} If $M_1,M_2$ are smooth of dimensions $n_1, n_2$, then $M_1
    \times M_2$ is a smooth $(n_1+n_2)$-manifold.  \end{exm}

    \begin{exm} Consider $S^2 \subset \R^3$. The standard smooth structure is
        defined using stereographic projection to $\R^2$ by \[(x_1,x_2,x_3)
        \mapsto \left(\frac{x_1}{1\pm x_3}, \frac{x_2}{ 1 \pm x_3 }\right).\]
        Homework $0.1$ is to check that the transition is smooth.  \end{exm}

    More generally, we can define the standard structure on $S^n \subset
    \R^{n+1}$. However, there may be exotic smooth structures on $S^n$. For
    example, there are exotic spheres of dimension 7, 13, 14, 15, 16, and many
    higher dimensions. This question is open for $n = 4$.

    \begin{exm} The torus $\mathrm{T}^2 \simeq S^1\times S^1$ has a unique
    smooth structure.  \end{exm}

    \begin{exm} $\R\P^2$ is the space of lines through the origin in $\R^3$. We
    will attempt to build a natural smooth structure.  \end{exm}

    \begin{lem} Given a set $M$ and a collection $\{U_{\alpha}\}$ of subsets
        with injections $\varphi_{\alpha}: U_{\alpha} \hookrightarrow \R^n$
        with \begin{enumerate} \item For all $\alpha$,
            $\varphi_{\alpha}(U_{\alpha})$ is open; \item
            $\varphi_{\alpha}(U_{\alpha} \cap U_{\beta})$ is open; \item The
    transitions $\varphi_{\alpha} \circ \varphi_{\beta}^{-1}$ are
    diffeomorphisms; \item $M$ is covered by countably many $U_{\alpha}$;
    \end{enumerate} Then $M$ has a topology with basis
    $\varphi_{\alpha}^{-1}(V)$ for all $V \subset \R^n$ open. Moreover, if the
    topology is Hausdorff, then $M$ has a unique smooth structure where
    $\{(U_{\alpha}, \varphi_{\alpha})\}$ is a smooth atlas.  \end{lem}

    Continuing Example 14, we will take the sets $U_{\alpha} = (x_{\alpha}\neq
    0)$ to be the standard Euclidean charts. Clearly the $\varphi_{\alpha}$ are
    injective (in fact they are bijective). Next, the intersections are both
    open. Third, the transitions are given by \[\varphi_2 \circ \varphi_1^{-1}:
    (x_1,x_2) \mapsto \left( \frac{1}{x_1}, \frac{x_2}{x_1} \right),\]so they
    are diffeomorphisms. Next time we will check that $\R\P^n$ is Hausdorff.

    \section{Lecture 2 (Sep 9)}% \label{sec:lecture_2_sep_9_}

    The first homework has been posted. It is due in $14$ days. The problems
    from the book are 1.1, 1.5, 1.7, 2.1, 2.4, 2.10, and 2.14. In addition,
    prove that diffeomorphism is an equivalence relation and construct a smooth
    structure on the square.
    
    Today we will complete Example 14 and show that $\R\P^2$ is Hausdorff.
    First assume that $l_1, l_2 \in \R\P^2$ are in the same Euclidean patch.
    Then $\R^2$ is Hausdorff, so there are disjoint neighborhoods $V_i \ni
    l_i$. Now suppose $l_1 \in U_1 \setminus U_2, l_2 \in U_2 \setminus U_1$.
    Thus $l_1 = [1:0:u_1], l_2 = [0:1:u_2]$. Then we can write $\varphi_i(l_i)
    = (0, u_i) \in \R^2$ for some $u_i \neq 0$. Then set $V_i = B_{\ep}(0,
    u_i)$. 
           
    Finally, we show that for small $\ep$, the $\varphi_i^{-1}(V_i)$ are
    disjoint. If they intersect, then we obtain $l = [1:y^1_1:y_2^1 + u_1] =
    [y_1^2:1:y_2^2+u_2]$, which implies that $y_1^1y_1^2 = 1$, contradicting
    our assumption on the size of $\ep$.

    \begin{defn} A topological manifold $M$ is a \textit{complex $n$-manifold}
        if $M$ admits a holomorphic atlas $\{(U_{\alpha}, \varphi_{\alpha})\}$
        to $\C^n$. Here holomorphic is taken to mean $J \cdot Df = Df \cdot J$,
        where $J$ is a matrix corresponding to multiplication by
        $I$.\footnote{Note that in general this only gives us an almost complex
        structure.} \end{defn}

    \begin{thm} Lemma 15 holds in the holomorphic setting.  \end{thm}

    \begin{exm} $\C\P^n$ is the set of (complex) lines in $\C^{n+1}$ through
    the origin. Homework 1 will show that this is a complex $n$-manifold.
\end{exm}

    \subsection{Morphisms}% \label{sub:morphisms}
    
    

    We will now construct morphisms of smooth manifolds.

    \begin{defn} Let $M$ be a smooth manifold. $f: M \to \R$ is a
    \textit{smooth function} if for all $p \in M$ there exists a chart $(U,
\varphi)$ with $p \in U$ such that $f \circ \varphi^{-1}$ is smooth.
\end{defn}

    \begin{defn} Let $M,N$ be smooth manifolds. $f:M \to N$ is a \textit{smooth
        map} if for all $p \in M$ there exist charts $(U, \varphi)$ with $p \in
        U$ and $(V, \psi)$ with $f(p) \in V$ such that $\psi \circ f \circ
        \varphi^{-1}$ is smooth.  \end{defn}

    \begin{rmk} Smoothness is independent of the choice of charts.  \end{rmk}

    \begin{defn} Let $F: M \to N$ be a smooth map. $F$ is a \textit{
    diffeomorphism  }if $F^{-1}$ exists and is smooth.  \end{defn}

    \begin{defn} $F: M \to N$ is a \textit{local diffeomorphism} if for all $p
    \in M$ there exists an open $U \ni p$ such that $F(U)$ is open and $F|_U$
is a diffeomorphism to $F(U)$.  \end{defn}
    
    \begin{prop} The following are true: \begin{enumerate} \item Smooth implies
        continuous, but the converse is false; \item Smooth maps make
        $\mathbf{Diff}$ into a category; \item The set of smooth functions
        $C^{\infty}(M)$ is a commutative ring; \item Smooth maps $M \to N$ pull
back smooth functions $C^{\infty}(N) \to C^{\infty}(M)$.\footnote{This is
entirely analogous to the case of algebraic geometry.} \end{enumerate}
\end{prop}

    \begin{exm} Consider the following basic examples: \begin{enumerate} \item
        If $N \subset M$ is open and $M$ is a smooth manifold, then the
        inclusion $\iota:N \hookrightarrow M$ is smooth; \item $\mathbf{Diff}$
        has products. In addition, the inclusions $m \mapsto (m,0)$ are smooth.
\end{enumerate} \end{exm}

    \begin{exm} We will see that the inclusion $S^2 \hookrightarrow \R^3$ is
        smooth. To do this, compute with the coordinate charts. Also, the
        inverses of the coordinate charts are given by \[ (y_1, y_2) \mapsto
        \frac{1}{y_1^2+y_2^2+1} (2y_1, 2y_2, \pm(1-y_1^2-y_2^2)).\] \end{exm}

    \begin{exm} We show that the projection $\R^3 \setminus \{0\} \to \R\P^2$
        is smooth. In the charts, we have \[(x_1, x_2, x_3) \mapsto \left[ 1:
        \frac{x_2}{x_1} : \frac{x_3}{x_1} \right], \] which are smooth.
    \end{exm}

    \begin{exm} The Hopf fibration $S^3 \to \C\P^1$ is smooth.\footnote{The
    existence of this morphism shows that $\pi_3(S^2) \neq 0$.} \end{exm}

    \begin{exm} The smooth composition of Examples 26 and 27 is a local
    diffeomorphism. In fact, this is a $2$-to-$1$ cover of $\R\P^2$ and
demonstrates that $\pi_1(\R\P^2) = \Z/2\Z$.  \end{exm}

    \section{Lecture 3 (Sep 11)}% \label{sec:lecture_3_sep_11_}

    We began class with Mike getting to know everyone. There are people
    enrolled in the class who were not here, but there was apparently a good
    mix of algebra, analysis, and geometry.

    \subsection{Example 29 Continued}% \label{sub:example_29_continued}
    
    
    We will show that the map $S^2 \to \R\P^2$ in Example 29 is a local
    diffeomorphism. Assume $l = F(p)$ is contained in the first Euclidean
    chart. Then \[p = \pm \frac{1}{\sqrt{1+x_1^2+x_2^2}}(1,x_1,x_2).\] Then for
    some small $\ep > 0$ let $W = B_{\ep}(y_1,y_2)$ and set $V =
    \varphi_1^{-1}(W)$. Then note that the preimage of any point consists of
    two antipodal points. Thus for sufficiently small $\ep$, $F_{-1}(V)$ is a
    disjoint union of two open sets. Then this is easy to see that $F|_U$ is
    injective. Next we prove that the inverse of $F|_U$ is smooth. We can do
    this by computing the inverse explicitly in charts.

    Now recall the definition of a covering map from topology. We may replace
    $\mathbf{Top}$ with $\mathbf{Diff}$, obtaining the notion of a
    \textit{smooth covering map.}

    \begin{exm} $S^2 \to \R\P^n$ is a smooth cover.  \end{exm}

    \begin{prop} If $M$ is a connected smooth $n$-manifold and $\pi:
    \widetilde{M} \to M$ is a topological covering map, then $\widetilde{M}$
has a unique smooth structure such that $\pi$ is a smooth cover.  \end{prop}

    Now note that any manifold is locally contractible, so it is locally
    connected and locally simply connected. Therefore any smooth connected
    manifold has a smooth universal cover.

    \begin{defn} A Lie group $G$ is a group object in $\mathbf{Diff}$. More
    concretely, it is a group which is also a smooth manifold such that
multiplication and inverse are smooth.  \end{defn}

    \begin{exm} Some examples of Lie groups are: \begin{enumerate} \item $G =
        \R^n$ is the simplest Lie group; \item $G = GL_n(\C)$; \item $G = S^1
        \hookrightarrow C^* = GL_1(C)$; \item $G = S^3 \hookrightarrow
        \mathbb{H} \setminus \{0\}$.  \end{enumerate} \end{exm}

    \begin{thm} Let $G$ be a connected Lie group. Then there exists a smooth
    universal cover $\pi: \widetilde{G} \to G$ which is a morphism of Lie
groups.  \end{thm}

    \begin{rmk} Lee has a chapter on Lie groups, but we will probably not get
    to them in this course.  \end{rmk}

    \subsection{Partitions of Unity}% \label{sub:partitions_of_unity}
    
    

    \begin{defn} Let $M$ be a topological space and $\mc{X} = \{U_{\alpha}\}$
        be any open cover of $M$. A \textit{partition of unity} subordinate to
        $\mc{X}$ is a collection of continuous functions $\{f_{\alpha}\}$ such
        that: \begin{enumerate} \item $0 \leq f_{\alpha} \leq 1$; \item
            $\operatorname{supp} f_{\alpha} \subset U_{\alpha}$.\footnote{These
            do not need to be compactly supported, but in practice they will
        be.} \item For all $x \in M$, there exists $U \ni x$ such that only
        finitely many $f_{\alpha}$ have support intersecting $U$.  \item
$\sum_{\alpha} f_{\alpha}(x) = 1$ for all $x \in M$.  \end{enumerate}
\end{defn}

    \begin{thm} Let $M$ be a smooth manifold and $\mc{X} = \{X_{\alpha}\}$ be
        an open cover of $M$. Then there exists a partition of unity
        $\{f_{\alpha}\}$ subordinate to $\mc{X}$.\footnote{This is like
        Seifert-Van Kampen and Mayer-Vietoris in algebraic topology, which
    allow us to compute things locally.} \end{thm}

    Before we prove Theorem 37, we need some preliminary notions.

    \begin{lem} There exists an $h \in C^{\infty}(\R)$ such that \[h(t) =
        \begin{cases} 1 & t \leq 1 \\ 0 & t \geq 2 \\ h(t) \in (0,1) & 1 < t <
        2 \\ \end{cases} \] \end{lem}

    \begin{proof}[Sketch of Proof] First note that \[f(t) = \begin{cases}
    e^{-1/t} & t > 0 \\ 0 & t \leq 0 \\ \end{cases} \] is smooth. Then we can
build $h$.  \end{proof}

    \begin{lem} The bump function $H: \R^n \to \R$ given by $x \mapsto
    h(\abs{x})$ is smooth.  \end{lem}

    \begin{defn} Suppose $X$ is a topological space. Given an open cover
        $\mc{U}$ of $X$, another open cover $\mc{V}$ of $X$ is a
        \textit{refinement} of $\mc{U}$ if for all $V \in \mc{V}$ there exists
        $U \in \mc{U}$ such that $V \subset U$.  \end{defn}

    \begin{defn} $X$ is \textit{paracompact} if every cover of $X$ admits a
    locally finite refinement.  \end{defn}

    \begin{defn} Let $X$ be a smooth manifold. Let $\{W_i\}$ be an open cover
        of $M$. Then $W$ is \textit{regular} if the following holds:
        \begin{enumerate} \item The cover $\{W_i\}$ is countable and locally
            finite; \item Each $W_i$ is the domain of a smooth coordinate map
            $\varphi_i$ such that $\varphi_i(W_i) = B_3(0) \subset
            \R^n$.\footnote{This is specifically for the proof. We are staying
            in the integers for the number theorist in the room.} \item Let
    $U_i = \varphi_i^{-1}(B_1(0))$. Then the $\{U_i\}$ cover $M$.
    \end{enumerate} \end{defn}

    \begin{prop} Let $M$ be a smooth manifold. Then every open cover of $M$ has
        a regular refinement.In particular, $M$ is paracompact.\footnote{I
        asked if we needed the condition of countable refinement, then Connor
    said that refinements of partitions add points, and Mike made an analogy to
refining flour.} \end{prop}

    \begin{proof}[Proof of Theorem 37] First we will build a partition of unity
        subordinate to a regular refinement $\{W_i\}$ of our cover $\mc{X}$.
        Let $U_i = \varphi_i^{-1}(B_1(0))$ and $V_i = \varphi_i^{-1}(B_2(0))$.
        Recall that $W_i = \varphi_i^{-1}(B_3(0))$. Now define \[f_i =
            \begin{cases} H \circ \varphi_i & x \in W_i \\ 0 & c \in M
                \setminus W_i \end{cases} \] and set \[g_i(x) =
            \frac{f_i(x)}{\sum_j f_j(x)}.\] This is well-defined because the
            $U_i$ cover $M$. Also, $0 \leq g_i(x) \leq 1$. In addition, $\sum_i
            g_i(x) = 1$ and $\operatorname{supp} g_i \subset W_i$. Thus
            $\{g_i\}$ is a partition of unity subordinate to $\{W_i\}$.

        We now need to construct $\{f_{\alpha}\}$. Because $\{W_i\}$ is a
        refinement of $\mc{X}$, then for all $i$ there exists $\alpha =
        \rho(i)$ such that $W_i \subset X_{\alpha}$. Then for all $\alpha$,
        define $f_{\alpha} = \sum_{i \in \rho^{-1}(\alpha)} g_i$. Then it is
        clear that conditions $1,4$ of being a partition of unity hold. 
        
        It is also easy to see that $\operatorname{supp} f_{\alpha} \subset
        X_{\alpha}$. Fixing $x \in \operatorname{supp} f_{\alpha}$, define
        $\{y_n\}$ such that $y_n \to x$ and $f_{\alpha}(y_n) = 0$. Because
        \[f_{\alpha} = \sum_{i\in \rho^{-1}(\alpha)},\] for each $n$ there
        exists $i_n$ such that $g_{i_n}(y_n) \neq 0$. Because $\{W_i\}$ is
        locally finite, there exists a neighborhood $U \ni x$ such that $I_x =
        \{i \in I \mid W_i \cap U \neq \emptyset\}$ is finite. Then there
        exists $i \in I_x$ such that $\{n \mid i_n = i\}$ is infinite. $y_n \in
        \operatorname{supp} g_i \subset \overline{V}_i$. 

        Finally, we show that $\{\supp f_{\alpha}\}$ is locally finite. Because
    $W_i$ is locally finite, there exists $U \ni x$ such that $I_x$ as defined
above is finite. Then for all $\alpha$ such that $U \cap \supp f_{\alpha} \neq
\emptyset$, there exists $y \in U$ such that $f_{\alpha}(y) \neq 0$. This
implies $g_i(y) \neq 0$ for some $i \in \rho^{-1}(\alpha)$. Thus $y \in U \cap
W_i$, so $i \in I_x$. Thus $\alpha = \rho(i) \subset \rho(I_x)$.  \end{proof}

    \section{Lecture 4 (Sep 16)}% \label{sec:lecture_4_sep_16_}

    We began class by finishing the proof of Theorem 37. Everything there has
    been added to the notes from last time. Today we will discuss tangent
    vectors.\footnote{There are many ways to define this. For example, we can
    define the tangent space as the stalk of the tangent sheaf. Alternatively,
we can make a construction analogous to the Zariski tangent space from
algebraic geometry. Finally, we can embed $M$ into $\R^n$ by Whitney and use
the classical notion of tangent space.}

    \subsection{Tangent Spaces}% \label{sub:tangent_spaces}
    
    
    \begin{defn} Let $M$ be a smooth manifold and $p \in M$. Then an
    $\R$-linear map $X: C^{\infty}(M) \to \R$ is called a \textit{ tangent
vector at $p$ } if for all $f,g \in C^{\infty}(M)$, $X(fg) = f(p)X(g) +
g(p)X(f)$.  \end{defn}

    \begin{notn} The set of all tangent vectors at $p$ is denoted $T_pM$. It is
    easy to see that this is an $\R$-vector space.\footnote{Later we will
define the tangent bundle.} \end{notn}

    \begin{defn} Let $F:M \to N$ be smooth. The \textit{differential of $F$ at
        $p$} $F_*: T_pM \to T_{F(p)}N$ is given by \[(F_*X)(f) = X(f \circ F)\]
        for all $f \in C^{\infty}(N)$.\footnote{This is also called the
        pushforward induced by $F$.} Usually, this is denoted by $dF_p$.
    \end{defn}

    \begin{prop} The following are true: \begin{enumerate} \item $(G \circ F)_*
    = G_* \circ F_*$; \item If $F: M \to N$ is a local diffeomorphism, then
    $F_*$ is an isomorphism.  \end{enumerate} \end{prop}

    \begin{prop}[``Localization''] Let $M$ be a smooth manifold and $U \subset
    M$ open with inclusion $i:U \to M$. Then for all $p \in U$, $i_*:T_pU \to
T_pM$ is an isomorphism.  \end{prop}
    
    \begin{lem} Fix $p \in M$ and $f,g \in C^{\infty}(M)$. If there exists a
    neighborhood $B \ni p$ such that $f|_B = g|_B$, then $X(f) = X(g)$ for all
$X \in T_pM$.  \end{lem}

    \begin{proof} Let $\mc{X} = \{B, M \setminus \{p\}\}$ and $\psi_1, \psi_2$
        be a partition of unity subordinate to $\mc{X}$. Let $h = f-g$. Then
        $\psi_2 = 1$ on $M \setminus B$. Thus $h=h\psi_2$ on $M \setminus B$.
        Because $h=0$ on $B$, then $h = h \psi_2$ on $B$. Finally, $\psi_2(p) =
        0$. Thus $X(h) = X(h \psi_2) = h(p)X(\psi_2)+\psi_2(p)X(h) = 0$.
    \end{proof}

    \begin{lem} Consider $A \subset U \subset M$, where $A$ is closed and $U$
        is open. Then there exists an extension map $C^{\infty}(U) \to
        C^{\infty}(M)$ given by $f \mapsto \widetilde{f}$ such that
        $\widetilde{f}|_A = f$ and $\supp \widetilde{f} \subset U$.  \end{lem}

    \begin{proof} Let $\psi_1, \psi_2$ be subordinate to $\{U, X \setminus
    A\}$. Define $\widetilde{f} = \psi_1 f$.  \end{proof}

    \begin{proof}[Proof of Proposition 3.7] Fix a ball $B \ni p$ such that
        $\overline{B} \subset U$. Suppose $i_*X = 0$. Let $\widetilde{f} \in
        C^{\infty}(M)$ be the extension of $f$ by Lemma 50. Then
        $X(f)=X(\widetilde{f}) = X(\widetilde{f} \circ i) = i_*X(f) = 0$.

        Now we show that $i_*$ is surjective. For $Y \in T_pM$, define $X \in
    T_pU$ by $X(f) = Y(\widetilde{f})$. Checking that $X$ is linear and
satisfies the Leibniz rule is straightforward, and it is easy to see that $i_*X
= Y$.  \end{proof}

    \begin{prop} Fix $p \in \R^n$. Define $D: \R^n \to T_p \R^n$ by $v \mapsto
        D_v$ where $D_v$ is the directional derivative. Then $D$ is an
        isomorphism. If $x^1, \ldots, x^n$ are coordinates of $\R^n$ and $v =
        (v_1, \ldots, v_n)$, then $D_v f = v \cdot ( \nabla f )$.  \end{prop}

    \begin{defn} Now consider some chart $\varphi:U \to \R^n$ and the
    projections $\pi_i:\R^n \to \R$. Let $x^i = \pi_i \circ \varphi$. Then
$\{x^i\}$ are called \textit{local coordinates} of $U$.  \end{defn}

    \section{Lecture 5 (Sep 18)}% \label{sec:lecture_5_sep_18_}
    
    Last time, we defined local coordinates on $\R^n$ at the end of class. Note
    that a chart $\varphi$ can be written as $\varphi(q) = (x^1(q), \ldots,
    x^n(q))$. We may define local coordinates on an arbitrary manifold
    analogously.

    \subsection{Working in Coordinates}% \label{sub:working_in_coordinates}
    
    Fix $p \in U \subset M$ and let $f \in C^{\infty}(U)$. Denote $\widehat{p}
    = \varphi(p) \in \R^n$ and $\widehat{f} = f \circ \varphi^{-1}$. Because
    $\varphi(U) \subset \R^n$, we can use multivariable calculus to define
    \[\frac{\partial \widehat{f}}{\partial x^i}\Bigg\vert_{\widehat{p}} \in
    C^{\infty}(U).\] This allows us to define the directional derivative on $M$
    as \[\left(\frac{\partial}{\partial x^i}\bigg\vert_p \right) \coloneqq
    \varphi^{-1}_* \left( \frac{\partial}{\partial x_i}\bigg\vert_{\widehat{p}}
\right)).\]

    \begin{prop} The set of directional derivatives with respect to the local
    coordinates is a basis for $T_pM$. In particular, $T_pM \simeq \R^n$.
\end{prop}

    This allows us to compute tangent vectors by pushforward: \begin{align*}
        \left( \frac{\partial}{\partial x^i} \bigg\vert_p \right)(f) &=
        \varphi^{-1}_* \left( \frac{\partial}{\partial x^i}
    \bigg\vert_{\widehat{p}} \right)(f) \\ &= \frac{\partial}{\partial x^i}
    \bigg\vert_{\widehat{p}} (f \circ \varphi^{-1}) \\ &= \frac{\partial
\widehat{f}}{\partial x^i} \bigg\vert_{\widehat{p}}.  \end{align*}

    Now we may define the Jacobian of a smooth map.  Let $F:M \to N$ be a
    smooth map with $p \in M, q = F(p) \in N$. Define charts $U \ni p, V \ni q$
    with charts $\varphi, \psi$ with local coordinates $x^1, \ldots, x^m$ and
    $y^1, \ldots, y^n$. We will write $F_*$ as a matrix with respect to the
    standard bases.
    
    Let $\widehat{F}^j = y^j \circ F \circ \varphi^{-1}$, so the induced map
    $\varphi(U) \to \psi(V)$ is given by $(\widehat{F}^1, \ldots,
    \widehat{F}^n)$. Then \begin{align*} \left( F_* \frac{\partial}{\partial
        x^i}\bigg\vert_p\right) (y^j) &= F_* \left( \frac{\partial}{\partial
x^i}\bigg\vert_{\widehat{p}} \right) (\pi^j \circ \psi) \\ &= \left(
\frac{\partial}{\partial x^i}\bigg\vert_{\widehat{p}} \right) (\pi^j \circ \psi
\circ F \circ \varphi^{-1}) \\ &= \left( \frac{\partial}{\partial
x^i}\bigg\vert_{\widehat{p}} \right) (\pi^j \circ \widehat{F}) \\ &= \left(
\frac{\partial}{\partial x^i}\bigg\vert_{\widehat{p}} \right) (\widehat{F}^j)
                                                               \\ &=
                                                               \frac{\partial
                                                               \widehat{F}}{\partial
                                                           x^i}\bigg\vert_{\widehat{p}}.
                                                           \end{align*}

    From multivariable calculus, we get that \[ F_* \left(
    \frac{\partial}{\partial x^i}\bigg\vert_p \right) = \sum_{j=1}^n \left(
    \frac{\partial \widehat{F}^j}{\partial x^i} \bigg\vert_{\widehat{p}}
    \right) \left( \frac{\partial}{\partial y^j} \bigg\vert_q \right).\]
    Writing this as a matrix, we get exactly the Jacobian.

    \begin{exm} We will compute the pushforward of the identity on two
        different charts. In particular, $F_* = \mr{id}$. Then we see that \[
        \frac{\partial}{\partial x^i}\bigg\vert_p = \sum_{j=1}^{n} \left(
    \frac{\partial (\psi \circ \varphi^{-1})^j}{\partial x^i} \right)
\frac{\partial}{\partial y^j} \bigg\vert_p.\] \end{exm}

    \begin{exm} Suppose $F:M \to \R$. Then $F_*:T_pM \to F_{F(p)}\R \simeq \R$.
        Thus for all $F \in C^{\infty}(U)$, we can consider $dF_p = F_* \in
        (T_pM)^*$.\footnote{This is called the cotangent space for obvious
        reasons.} Note that $dF_p(X) = X(F)$.  \end{exm}

    In local coordinates, observe that \[ \left( \frac{\partial}{\partial x^i}
    \bigg\vert_p  \right) (x^j) = \delta_{ij}.\] Thus $\{dx^i_p\}$ is the dual
    basis of $(T_pM)^*$.

    \begin{exm} Let $\gamma: (-\ep, \ep) \subset \R \to N$ such that $\gamma(0)
        = q \in N$. Then $\gamma$ is called a smooth curve through $q$. Note
        that $T_0(-\ep,\ep) = T_0\R = \R^1 = \R\{(\partial/\partial t)|_0\}$.
        We denote the tangent vector \[\gamma_* \left( \frac{\partial}{\partial
        t}\bigg\vert_0 \right) \in T_{\gamma(0)}N\] by $\gamma'(0)$ and call it
    the tangent vector of $\gamma$ at the point $q$.  \end{exm}

    \subsection{Alternative Approaches to Tangent Spaces}%
    \label{sub:alternative_approaches_to_tangent_spaces}

    We present alternative ways to define the tangent space. The first way is
    to use equivalence classes of smooth curves:

    Fix $p \in M$. Let $\mc{C}_p = \{\gamma:I \to M \mid \gamma(0) = p\}$. Then
    let $\gamma_1 \sim \gamma_2$ if they share a tangent vector (after pushing
    forward to $\R^n$ and using the calculus notion of tangent vector.)

    The second way, which will not be discussed, is as the stalk of the tangent
    sheaf.\footnote{This is closer to the algebraic point of view.}

    \subsection{Vector Fields.}% \label{sub:vector_fields_}
    
    \begin{defn} Let $M$ be a smooth manifold. The \textit{tangent bundle} of
        $M$ is \[TM = \bigsqcup_{p \in M} T_pM\] as a set.  \end{defn}

    \begin{thm} The following are true: \begin{enumerate} \item If $\dim M =
        n$, then $TM$ is naturally a smooth $(2n)$-manifold.  \item Moreover,
        the projection $\pi:TM \to M$ is a smooth map.  \item For any smooth
        $F:M \to N$, there is a smooth $F_*: TM \to TN$, defined in the obvious
        way.  \end{enumerate} \end{thm}

    \begin{proof}[Proof of (1)] Let $U \subset M$. Then consider $\pi^{-1}(U) =
        TU$. Now let $\{(U_{\alpha}, \varphi_{\alpha})$ be a smooth atlas of
            $M$. Then we will define local trivializations:
        
            Let $x^1, \ldots, x^n$ be local coordinates and consider the
        standard basis of $T_pM$. Thus we can write $X = \sum_{i=1}^n X^i
    \frac{\partial}{\partial x^i} \big\vert_p$. Then we can define
$\widetilde{\varphi}_{\alpha}(p,X) = (\varphi_{\alpha}(p), X^1, \ldots, X^n)$.
It is easy to see that the conditions of Lemma 15 hold. Applying Lemma 15, we
have a smooth structure.  \end{proof}

    \section{Lecture 6 (Sep 23)}% \label{sec:lecture_6_sep_23_}
    
    Mike cannot be here on Wednesday, December 11, so we will need to make up
    the lecture. We will begin at 2 PM on three days that are to be determined.

    \subsection{Vector Fields, Continued}% \label{sub:vector_fields_continued}
    
    \begin{defn} A (continuous, smooth) section $M \to TM$ of the tangent
    bundle is called a \textit{(continuous, smooth) vector field}.  \end{defn}
    
    \begin{rmk} An open question is to find a necessary and sufficient
    condition on manifolds $M,N$ such that $TM \simeq TN$.  \end{rmk}
    
    \begin{lem}[Smoothness Criteria] A section $X$ of the tangent bundle is
        smooth if and only if one of the following two conditions holds:
        \begin{enumerate} \item The representative on charts $\widehat{X} =
            \widetilde{\varphi} \circ X \circ \varphi^{-1}$ is smooth.  \item
    For all open $U \subset M$ and $f \in C^{\infty}(U)$, the function $Xf:p
    \to X_p(f)$ is smooth.  \end{enumerate} \end{lem}
    
    \begin{notn} We will denote the space of global sections of the tangent
    bundle by $\mc{X}(M)$.  \end{notn}

    We can check that $\mc{X}(m)$ is a nonempty module over
    $C^{\infty}(M)$.\footnote{In fact, we can extend this to define the tangent
    sheaf.}

    \begin{defn} A map $Y: C^{\infty}(M) \to C^{\infty}(M)$ is called a
    \textit{derivation} if it satisfies the Liebniz rule.  \end{defn}

    \begin{prop} $\mc{Y}$ is a derivation if and only if $\mc{Y}(f) = Yf$ for
    some $Y \in \mc{X}(M)$.  \end{prop}

    Now let $F:M \to N$ be a smooth map. If $F$ is a diffeomorphism, for $X \in
    \mc{X}(M)$ define $Y \in \mc{X}(N)$ by \[Y \coloneqq F_* \circ C \circ
    F^{-1}.\] Thus $Y \circ F = F^* \circ X$. However, if $F$ is not a
    diffeomorphism, such a $Y$ need not exist.

    \begin{exm} Consider the figure-eight $S^1 \to \R^n$. Then we cannot push
    forward a vector field onto the double point. Also, how do we push forward
the vector field onto points that are not mapped onto?  \end{exm}
    
    \begin{defn} Let $F:M \to N$ be a smooth map and $X \in \mc{X}(M), Y \in
    \mc{X}(N)$. Then $X,Y$ are \textit{$F$-related} if $F_* \circ X = Y \circ
F$, \end{defn}

    \begin{exm} Consider the vector field $\partial_t \in \mc{X}(\R)$. Then if
    $F:\R \to \R^2$ is the covering map of the circle, an $F$-related vector
field on $\R^2$ is $Y = x \partial_y - y \partial_x$.  \end{exm}

    \subsection{Lie Algebras}% \label{sub:lie_algebras}
    
    

    Note that if $X,Y$ are vector fields, then $X \cdot Y$ is not necessarily a
    vector field. However, their Lie bracket is a vector field.

    \begin{exm} Let $\varphi:U \subset M \to \R^n$ be a chart with coordinates
        $x^1, \ldots, x^n$. Then we have the relations generating the Weyl
        algebra on the Lie bracket of the standard basis vectors. Proof boils
        down to symmetry of mixed partials after symbol pushing to get us into
        $\R^n$.  \end{exm}

    Note that $\mc{X}(M)$ is a Lie algebra.\footnote{Apparently the thing we
    are supposed to study is $\infty$-algebraic structures. Mike wishes there
was a course here on higher algebra taught by Ivan or Owen.}

    \begin{prop} $[fX,gY] = fg[X,Y] + (fXg)y - (gYf)X$. Also, if $X_i$ is
    $F$-related to $Y_i$ for $i = 1,2$, then $[X_1,X_2]$ is $F$-related to
$[Y_1,Y_2]$.  \end{prop}

    We can express the Lie bracket locally. If $X = \sum x^i \partial_i$, then
    \[ [X,Y] = \sum_i \sum_j (X^j \partial_j Y^i - Y^j \partial_j X^i)
    \partial_i = \sum [X,Y]^i \partial_i.\]

    Recall the definition of a Lie algebra.\footnote{This can be internalized
    to any category enriched over $\mathbf{Ab}$.} 

    \begin{exm} Some examples of Lie algebras are: \begin{enumerate} \item
        $\mf{g} = \mc{X}(M)$ with the Lie bracket defined above.  \item $\mf{g}
        = \R^n$ with zero Lie bracket.\footnote{This is called
        \textit{abelian}.} \item $\mf{g} = M_n(\R)$ with the commutator. In
        fact, this generalizes to any associative $\R$-algebra.  \item In
        Homework $2$ we will find the $2$-dim and $3$-dim Lie algebras.  \item
        The category of Lie algebras has products.  \item Suppose the vector
        space $\mf{g}$ has basis $X_1, \ldots, X_n$. Define \[ [X_i,X_j] =
            \sum_{k=1}^n c_{ij}^k X_k.\] Then Jacobi holds if and only if for
            all $s$, \[ \sum_{\ell = 1}^{n} (c_{jk}^{\ell}c_{i\ell}^s +
            c_{ki}^{\ell}c_{j\ell}^s + c_{ij}^{\ell}c_{k\ell}^2).\]
    \end{enumerate} \end{exm}

    Let $G$ be a Lie group. Then define the left translation $L_g:G \to G$ by
    $h \mapsto gh$. This is a diffeomorphism with inverse $(L_g)^{-1} =
    L_{g^{-1}}$.

    \begin{defn} $X \in \mc{X}(G)$ is \textit{left-invariant} if $(L_g)^*(X) =
    X$ for all $g \in G$.  \end{defn}

    Define $\operatorname{Lie}(G)$ be the set of left-invariant vector fields.
    We can check that $\operatorname{Lie}(G)$ is a Lie algebra with the usual
    Lie bracket.

    \begin{thm} Let $G$ be a Lie group with $e \in G$ the identity. Define
    $\ep: \operatorname{Lie}(G) \to T_e G$ by $X \mapsto X_e$. Then $\ep$ is an
isomorphism of vector spaces. In particular, $\dim \operatorname{Lie}(G) = \dim
G$.  \end{thm}

    \begin{thm} Let $G,H$ be Lie groups with Lie algebras $\mf{g}, \mf{h}$.
        Then suppose $F: G \to H$ is a morphism of Lie groups. Then for $X \in
        \mf{g} = T_{e}G$, there exists a unique $Y \in T_eH$ which is
        $F$-related to $X$. This defines a Lie algebra homomorphism $F_*:
        \mf{g} \to \mf{h}$.  \end{thm}

    \begin{cor} If $G$ is a Lie subgroup of $H$, then $\mf{g}$ is a Lie
    subalgebra of $\mf{h}$.  \end{cor}

    \begin{cor} If $\pi:G \to H$ is a smooth cover of Lie groups, then $\mf{g}
    \simeq \mf{h}$.  \end{cor}

    \begin{exm} Let $G = GL_n(\R) \subsetneq M_n(\R)$. Then $T_{I_n}G =
    T_{I_n}M = M_n(\R)$.  \end{exm}

    \begin{prop} Under the identification $\ep: T_{I_n}G \to
    \operatorname{Lie}(G)$, the Lie bracket of $\operatorname{Lie}(G)$ is sent
to the commutator of matrices.  \end{prop}

    \section{Lecture 7 (Sep 25)}% \label{sec:lecture_7_sep_25_}
    
    Today we will begin discussion of vector bundles.

    \subsection{Vector Bundles}% \label{sub:vector_bundles}
    
    \begin{defn} Let $X$ be a topological space. A \textit{real vector bundle
        of rank $n$ over $X$} is a morphism $\pi:E \to X$ such that
        \begin{enumerate} \item For all $p \in X$, the fiber over $p$ is an
            $n$-dimensional real vector space.  \item For all $p \in X$, there
            exists $U \ni P$ and a homeomorphism $\Phi:\pi^{-1}U \to U \times
            \R^n$ such that $\pi_1 \circ \Phi = \pi$ and such that for all $q
            \in U$, $\Phi|_{E_q}$ is an isomorphism of vector spaces.
    \end{enumerate} \end{defn}

    \begin{exm} The tangent bundle of a smooth manifold is a vector bundle.
    \end{exm}

    \begin{defn} A \textit{complex vector bundle} is as in the above
    definition, but $\R$ is replaced with $\C$.  \end{defn}

    Suppose $E \to X$ is a vector bundle. Suppose $\{U_{\alpha}\}$ is an open
    cover with trivializations $\Phi_{\alpha}$. Then we have the following
    diagram: \begin{center} \begin{tikzcd} (U_{\alpha} \cap U_{\beta}) \times
        \R^n \arrow[swap]{dr}{\pi} & \pi^{-1}(U_{\alpha} \cap U_{\beta})
        \arrow{d}{\pi} \arrow[swap]{l}{\Phi_{\beta}} \arrow{r}{\Phi_{\alpha}} &
        (U_{\alpha} \cap U_{\beta}) \times \R^n \arrow{dl}{\pi} \\ & U_{\alpha}
    \cap U_{\beta} \end{tikzcd}.  \end{center}

    Then we consider the map $\Phi_{\beta}\circ\Phi_{\alpha}^{-1}$. Over every
    point $q$, we obtain some $\tau_{\beta\alpha}(q) \in GL_n(\R)$. Thus the
    transitions maps of $E$ are smooth maps $U_{\alpha} \cap U_{\beta} \to
    GL_n(\R)$.

    \begin{prop} $\tau_{\gamma\beta}\tau_{\beta\alpha} = \tau_{\gamma\alpha}$.
    \end{prop}

    \begin{exm} On the tangent bundle, the transitions are simply the
    Jacobians.  \end{exm}

    \subsection{Pullback Bundles}% \label{sub:pullback_bundles}
    
    

    \begin{lem} Let $X$ be a topological space, $E$ a set, and $\pi:E \to X$
        surjective. Suppose the fibers of $\pi$ are vector spaces of dimension
        $n$ and that there exists a cover $\{U_{\alpha}\}$ of $X$ such that:
        \begin{enumerate} \item For all $\alpha$ there exists a bijection
            $\Phi_{\alpha}:\pi^{-1}(U_{\alpha}) \to U_{\alpha}\times \R^n$
            commuting with projections and $\Phi_{\alpha}|_{E_q}$ is an
            isomorphism.  \item For all $\alpha,\beta$,
    $\tau_{\beta\alpha}:U_{\alpha}\cap U_{\beta} \to GL_n(\R)$ is continuous.
    \end{enumerate} Then $E \to X$ is a topological vector bundle.  \end{lem}

    \begin{rmk} Lemma 83 holds in the smooth category.  \end{rmk}

    \begin{exm} The trivial bundle $M \times \R^k$ is a smooth vector bundle.
    \end{exm}

    \begin{exm} Let $E = \{(x,v) \in \R\P^n \times \R^{n+1} \mid v \in x\}$.
        This is the \textit{tautological line bundle over $\R\P^n$.} We will
        choose the standard affine charts. The trivializations will be the
        obvious ones (projection of $v$ to the $i$th coordinate.) It is easy to
        see that the transitions are simply rescaling, so the transitions
        satisfy the conditions of Lemma 83.  \end{exm}

    \begin{exm} Consider $E = \{(x,v) \in \C\P^n \times \C^{n+1} \mid v \in
    x\}$. This is the tautulogical bundle over $\C\P^n$.\footnote{This
    generalizes to any field. Also, this is a holomorphic line bundle.}
\end{exm}

    Suppose $E \to M$ is a vector bundle and $F:N \to M$ is a smooth map. Then
    we will give the set $E \times_M N$ the structure of a vector bundle over
    $N$.\footnote{Compare to base change in algebraic geometry.}

    \begin{thm} Vector bundles can be pulled back.  \end{thm}

    \begin{proof} Let $\{U_{\alpha}\}$ be an open cover of $M$ with
        trivializations $\Phi_{\alpha}$. Now pull everything back to $N$ in the
        natural way. It is a mechanical exercise to verify that everything
        works and that we can use Lemma 83.  \end{proof}

    \begin{exm} Consider the embedding $\R\P^k \to \R\P^n$. Then pulling back
    the tautological bundle from $\R\P^n$ gives the tautological bundle on
$\R\P^k$.  \end{exm}

    \begin{exm} Let $M$ be the Klein bottle. Let $N = S^1$ and $F$ sends the
    circle to a loop around the cylinder part of the Klein bottle. We know that
$TM \to M$ is not trivial, but its pullback to $S^1$ is trivial.  \end{exm}

    \section{Lecture 8 (Sep 30)}% \label{sec:lecture_8_sep_30_}
    
    Last time we discussed vector bundles. Some examples are the tautological
    bundles over projective space. We also defined pullbacks of vector bundles.

    \subsection{Sections and Frames}% \label{sub:sections_and_frames}
    
    Let $E \to M$ be a vector bundle with $U \subset M$ open. Then a local
    section of $E$ over $U$ is a smooth $\sigma \in \mc{O}(E)|_U$. A global
    section is a section $\sigma \in \Gamma(E)$.

    \begin{defn} Let $E \to M$ be a smooth vector bundle of rank $n$. Then an
        ordered tuple $\{\sigma_1, \ldots, \sigma_n\}$ of local sections of $E$
        over $U \subset M$ open is called a \textit{local frame} if the values
        at each $p \in U$ form a basis of $E_p$. If $U = M$, then we have a
        \textit{global frame}.  \end{defn}

    \begin{exm} A section of the tangent bundle is the same thing as a vector
    field. In addition, every vector bundle carries a zero section.  \end{exm}

    \begin{exm} The tangent bundle $TM \to M$ carries standard local coordinate
    frames on the trivialiazations.  \end{exm}
    
    \begin{prop} Let $E \to M$ be a vector bundle. Then there exists a
    bijection between local (resp. global) frames and local (resp. global)
trivializations.  \end{prop}

    \begin{proof} Given a local frame, then we do the obvious thing to locally
    trivialize the vector bundle. In the other direction, define each frame to
be given by the coordinate functions.  \end{proof}

    Thus a vector bundle is trivial if and only if it has a global frame.

    \begin{defn} A manifold is \textit{parallelizable} if $TM \to M$ is
    trivial.  \end{defn}

    \begin{exm} $S^1$ is parallelizable. One argument is that all Lie groups
    are parallelizable. In this spirit, $S^3$ is also parallelizable. However,
$S^2$ is not parallelizable (Brouwer's fixed point theorem).  \end{exm}

    \begin{prop} Every Lie group is parallelizable.  \end{prop}

    \begin{proof} Take a frame for $T_eG$ and then apply the $g$-action to get
    a left-invariant frame.  \end{proof}

    \begin{exm} The tautological line bundle is not trivial.  \end{exm}

    \begin{exm} The Mobius strip is a nontrivial vector bundle on $S^1$.
    \end{exm}

    \begin{thm} Let $M$ be a smooth manifold with open cover $\{U_{\alpha}\}$.
        Given smooth $\tau_{\beta\alpha}$ such that
        $\tau_{\gamma\beta}\tau_{\beta\alpha} = \tau_{\gamma\alpha}$, then
        there exists a smooth rank-$n$ vector bundle over $M$ with these
        transition functions.  \end{thm}

    \subsection{Induced Bundles}% \label{sub:induced_bundles}
    
    Let $G \subset GL_n(\R)$ be a Lie subgroup and let $\rho$ be a Lie group
    homomorphism. Then suppose $E \to M$ is a smooth rank $n$ vector bundle
    with transition functions $\tau_{\beta\alpha}: U_{\alpha} \cap U_{\beta}
    \to G \subset GL_n(\R)$. Then postcomposing with $\rho:G \to GL_n(\R)$
    induces a vector bundle $\widetilde{E}$, called the \textit{induced bundle
    of $E$ by $\rho$}.

    \begin{exm} The dual bundle is induced by taking the inverse of the
    transpose of each transition function.  \end{exm}

    \section{Lecture 9 (Oct 2)}% \label{sec:lecture_9_oct_2_}
    
    \subsection{Examples of Dual Bundles}% \label{sub:examples_of_dual_bundles}
    
    \begin{exm} Let $E = TM$. Then the dual is $E^* = T^*M$, which is the
        cotangent bundle of $M$. This is the main example of a symplectic
        manifold. Also, we call a local trivialization of $TM$ a frame and a
        local trivialization of $T^*M$ a coframe.

        Physically, we can think of $M$ as position, $TM$ as position and
    velocity, and $T^*M$ as position and momentum. In particular, the laws of
mechanics can be phrased in both $TM$ and $T^*M$.  \end{exm}

    \begin{exm} Consider the projection $\R\P^m \to \R\P^{n-1}$ given by
        projection onto the first $n$ coordinates.\footnote{This is a standard
        example in algebraic geometry}. Now remove the point where $\pi$ is not
        regular. We claim $E^*$ is the tautological line bundle on
        $\R\P^n$.\footnote{In particular, $E = \mf{O}(1)$.} \end{exm}
    
    \begin{exm} Let $E_1, E_2$ be vector bundles of rank $n_1, n_2$. Then
    define $E = E_1 \oplus E_2$ is a vetor bundle of rank $n_1 + n+2$ with
transitions given by the obvious morphism $GL_{n_1} \times GL_{n_2} \to
GL_{n_1+n_2}$.  \end{exm}

    \subsection{Subbundles and Quotient Bundles}%
    \label{sub:subbundles_and_quotient_bundles}
    
    \begin{defn} Let $E \to M$ be a rank $n$ vector bundle. Then $E' \subset E$
        is a \textit{subbundle} of rank $k$ if \begin{enumerate} \item For all
            $p \in M$, $E_p'$ is a $k$-dimensional subspace of $E_p$.  \item
            For all $p \in M$, there exists a local frame of $E$ over some
            neighborhood $U \ni p$ such that for all $q \in U$, $\sigma_1(q),
            \ldots, \sigma_k(q)$ are a basis for $E_q'$.  \end{enumerate}
        \end{defn}

    \begin{prop} For any smooth vector bundle $E$ with subbundle $E'$ we can
        cover $M$ by $U_{\alpha}$ such that \begin{enumerate} \item Both $E,E'$
            are trivial $U_{\alpha}$.  \item The transition functions are block
            upper-triangular of the form $ \begin{pmatrix} A & B \\ 0 & C
        \end{pmatrix}. $ \end{enumerate} \end{prop}

    Now define \[\rho_1: \begin{pmatrix} A & B \\ 0 & C \end{pmatrix} \mapsto
    A, \rho_2: \begin{pmatrix} A & B \\0 & C \end{pmatrix} \mapsto C.\]

    \begin{defn} The bundle induced from $E$ by $\rho_2$ is called the
    \textit{quotient bundle}.  \end{defn}

    \subsection{Cotangent Bundle}% \label{sub:cotangent_bundle}
    
    We have the cotangent bundle $T^*M \to M$ with coframe $\{dx^i\}$. A
    section of the cotangent bundle is called a \textit{covector
    field}.\footnote{Later, this will be called a $1$-form.}
    
    \begin{exm} Let $M$ be a smooth manifold and $f \in C^{\infty}(M)$. The
    differential $df$ of $f$ is defined by $df_p(X_p) = X_p f$. By the
cotangent version of Lemma 61, this is a smooth covector field.  \end{exm}

    We may ask whether all smooth covector fields arise this way, and the
    answer is no, because of $H^1(M)$. We may read on our own how to integrate
    covector fields over paths.

    Differentials do satisfy the usual calculus rules. In particular, $df = 0$
    if and only if $f$ is constant.

    Pushforwards of tangent vectors are replaced by pullbacks here. If $F:M \to
    N$ is smooth, then the pullback is given by $F^*\omega(X) = \omega(F_*X)$
    pointwise. Thenwe can check that this pulls smooth covector fields to
    smooth covector fields.

    \subsection{Submersions, Immersions, Embeddings}.
    \label{sub:submersions_immersions_embeddings}

    \begin{defn} Let $F:M \to N$ be smooth. Then we define the \textit{rank} of
        $F$ at $p$ to be the rank of $(F_*)_p$.  \begin{itemize} \item $F$ is a
            \textit{submersion} if $(F_*)_p$ is surjective for all $p \in M$.
        \item $F$ is an \textit{immersion} if $(F_*)_p$ is injective for all $p
            \in M$.  \item $F$ is a \textit{smooth embedding} if $F$ is an
    immersion that is a homeomorphism onto its image.  \end{itemize}
\end{defn}%

    \begin{exm} The map $\A^1 \to (y^2 = x^3-x^2) \subset \A^2$ is an immersion
    but not an embedding.  \end{exm}

    \begin{rmk} An injective immersion must be a smooth embedding if it is
    proper; in particular when $M$ is compact.  \end{rmk}

    \begin{thm}[Inverse Function Theorem] \begin{description} \item[Euclidean
        Version:] First, we have the Euclidean version: Suppose $U,V \subset
        \R^n$ and $F:U \to V$ a smooth map. If the Jacobian $DF(p)$ is
        nonsingular, then there exists a connected neighborhood $U_0 \subset U$
        of $p$ and $V_0 \subset V \ni F(p)$ such that $F|_{U_0}$ is a
        diffeomorphism onto $V_0$.  \item[Manifold Version:] Let $F:M \to N$ be
smooth. If $(F_*)_p$ is a bijection, then $F$ is a local diffeomorphism near
$p$.  \end{description} \end{thm}

    \begin{thm}[Rank Theorem] \begin{description} \item[Euclidean Version:] Let
        $U \subset \R^m, V \subset \R^n$ and $F:U \to V$ be a smooth map of
        constant rank $k$. Then for all $p \in U$, there exist charts $(U_0,
        \varphi)$ of $p$ and $(V_0, \psi)$ of $F(p)$ such that the coordinates
        satisfy \begin{center} \begin{tikzcd} (x_1, \ldots, x_k, x_{k+1},
        \ldots, x_m) \arrow{rr}{\psi \circ F \circ \varphi^{-1}}
    \arrow{dr}{\pi} & & (x_1, \ldots, x_k, 0, \ldots, 0) \\ &  (x_1, \ldots,
x_k) \arrow[hookrightarrow]{ur} \end{tikzcd}.  \end{center} \item[Manifold
Version:] Let $F:M \to N$ be a smooth map of constant rank $k$. Then for all $p
\in M$, there exist local coordinates $x_1, \ldots, x_m$ at $p$ and $y_1,
\ldots, y_n$ at $F(p)$ such that the coordinate representation of $F$ is given
by $(x_1, \ldots, x_m) \mapsto (x_1, \ldots, x_k, 0, \ldots, 0)$.
\end{description} \end{thm}

    We may read about the Implicit Function Theorem on our own.

    \section{Lecture 10 (Oct 07)}% \label{sec:lecture_10_oct_07_}

    \begin{thm} Let $F:M \to N$ be smooth of constant rank. Then
        \begin{enumerate} \item If $F$ is surjective, then $F$ is a submersion.
            \item If $F$ is injective, then $F$ is an immersion.  \item If $F$
                is bijective, then $F$ is a diffeomorphism.  \end{enumerate}
            \end{thm}

    \begin{exm} Define $f:S^1 \times S^1 \to \R^3$ given by \[ (q_1, q_2, q_3,
    q_4) \mapsto (q_1(2+q_3), q_2(2+q_3), q_4).\]

        This embeds $S^1 \times S^1$ into $\R^3$.  \end{exm}
    
    \subsection{Submanifolds}% \label{sub:submanifolds}
    
    \begin{defn} Let $M$ be a smooth manifold of dimension $n$ and $S \subset
        M$. We say $S$ is an \textit{embedded submanifold of dimension $k \leq
        n$} if for all $p \in S$, there exists a smooth chart $U \ni p$ of $M$
        such that \[S \cap U = \{(x_1, \ldots, x_n) \in U \mid x_{k+1} = \cdots
        = x_n = 0 \}.\] \end{defn}

    Here $(U, \varphi)$ is called a \textit{slice chart} for $S$, $x_1, \ldots,
    x_n$ are the \textit{slice coordinates} for $S$, and $n-k$ is the
    \textit{codimension}.

    \begin{thm} Let $S \subset M$ be an embedded submanifold of dimension $k$.
        With the subspace topology, $S$ is a topological manifold of dimension
        $k$ and has a unique smooth structure such that $S \hookrightarrow M$
        is a smooth embedding.  \end{thm}

    \begin{thm} The image of a smooth embedding is an embedded submanifold.
    \end{thm}

    \begin{exm} The torus is a submanifold of $\R^3$.  \end{exm}

    \begin{exm} Smooth embeddings of $S^1$ in $\R^3$ are knots and their study
    is called knot theory. In general, we can study $S^k \hookrightarrow \R^n$
and the story varies by codimension.  \end{exm}

    \begin{proof}[Proof of Theorem 117] First we show that $S$ is a topological
        manifold. It is easy to check that $S$ is Hausdorff and
        second-countable in the subspace topology. To see that $S$ is locally
        Euclidean, we simply project the slice charts down to $\R^k$.

        Now we need to give $S$ a smooth structure. We will simply use the
        slice charts from $M$ with their transition functions restricted. Now
        restricting a smooth map to a coordinate subspace is smooth, so $S
        \hookrightarrow M$ is an immersion and topological embedding.

        Finally, we show that the smooth structure is unique. What we want to
    show is that $\psi, \theta$ are two charts from different atlases. Then
$\psi \theta^{-1}$ is a homeomorphism and is smooth. We can also see that $\psi
\theta^{-1}$ is an immersion. Because the domain and target have the same
dimension, this is a local diffeomorphism.  \end{proof}

    \begin{defn} Let $F:M \to N$ be a smooth map. Then a point $c \in N$ is a
    \textit{regular value} of $F$ if for all $p \in F^{-1}(C)$, $(F_*)_p$ is
surjective. Otherwise, $c$ is a \textit{critical value} of $F$.  \end{defn}

    \begin{cor} For any regular value $c$ of a smooth map $F:M \to N$, the
    preimage $F^{-1}(c)$ is an embedded submanifold of $M$ of codimension $\dim
N$. Moreover, $T_p(F^{-1}(c)) = \ker (F_*)_p$.  \end{cor}

    \begin{exm} Consider a smooth map $F: \R^n \to \R$ given by \[ (x_0,
    \ldots, x_n) \mapsto \sum_{k=0}^n x_k^2.\] Then $1 \in \R$ is a regular
    value of $F$ and $F^{-1}(1) = S^n$ is an embedded submanifold. Then $T_xS^n
    = \{y \mid y \perp x\}$.  \end{exm}

    \begin{exm} Consider the function $F: \R^3 \to \R$ given by \[ (x,y,z)
    \mapsto (\sqrt{x^2+y^2}-2)^2 + z^2.\] \end{exm}

    \begin{exm} Let $f: M \to N$ be a smooth map. Then consider the
        \textit{graph} of $f$ in $M \times N$. Then define $F: M \to M \times
        N$ given by $F(p) = (p,f(p))$. This is a smooth embedding and
        $\Gamma(f)$ is the image of $F$, so it is a submanifold of $M \times
        N$.  \end{exm}

    \begin{prop} Let $F: M \to N$ be smooth. Suppose $S \subset N$ is an
    embedded submanifold and $F(M) \subset S$. Then the range restriction
$F_S:M \to S$ is smooth.  \end{prop}

    \begin{rmk} There is a similar result for domain restriction.  \end{rmk}

    \begin{exm} Consider multiplication of quaternions on $\mathbb{H} = \R^4$.
    Then multiplication restricted to $S^3$ is smooth.  \end{exm}

    

    \section{Lecture 11 (Oct 09)}% \label{sec:lecture_11_oct_09_}
    
    \subsection{Lie Subgroups}% \label{sub:lie_subgroups}
    
    
    \begin{defn} A \textit{Lie subgroup} of a lie group $G$ is a subgroup of
    $G$ with a smooth structure making it an immersed submanifold of $G$.
\end{defn}

    \begin{prop} If $G$ is a Lie group and $H$ is any subgroup which also is an
    embedded submanifold, then $H$ is an closed Lie subgroup.  \end{prop}

    To prove this, we need to check that multiplication and inversion are
    smooth on $H$ and that $H$ is closed.

    \begin{exm} $SL_n(\R)$ is a closed Lie subgroup of $GL_n(\R)$ of
    codimension $1$, and the Lie algebra $\mf{sl}_n(\R)$ is the set of trace
$0$ matrices.  \end{exm}

    To show this, we simply show that $1$ is a regular value of the determinant
    function. The easy way to see this is that $\det$ is a Lie group
    homomorphism with derivative $\tr$.

    \begin{exm} In general, $SL_n$ is a closed algebraic subgroup of $GL_n$
    over any field $k$.  \end{exm}

    \begin{exm} $GL_n$ acts on $\P^{n-1}$ and the group of actions is called
    $PGL_n$. This has dimension $3$ over $k$.  \end{exm}

    \begin{exm} The orthogonal group $O(n)$ is a Lie group with Lie algebra
    $\mf{o}(n)$ consisting of matrices that satisfy $A^T+A = 0$.  \end{exm}

    \subsection{Lie Group Actions}% \label{sub:lie_group_actions}
    
    Let $M$ be a smooth manifold and $G$ be a Lie group.

    \begin{defn} A \textit{left-action of $G$ on $M$} is a smooth map $Theta:G
    \times M \to M$ that satisfies the usual group action axioms.  \end{defn}

    Recall that an action is transitive if for some $p$, $G.p = M$. The
    stabilizer is called the \textit{isotropy group}, and the action is free if
    every point has trivial stabilizer.

    \begin{exm} The simplest example of a Lie group action is the trivial
    action, where every element of $G$ acts by the identity.  \end{exm}

    \begin{exm} If $M$ is a vector space $V$, then a $G$-action is called a
    representation.  \end{exm}

    \begin{exm} Consider the action of $G$ on itself by conjugation. Then
    differentiate the action to obtain a Lie group homomorphism $G \to
GL(T_eG)$, which is called the adjoint representation of $G$.  \end{exm}

    \begin{defn} An action $G \times M \to M$ is called \textit{proper} if the
    map $G \times M \to M \times M$ given by $(g,p) \mapsto (g.p,p)$ is proper.
\end{defn}

    \begin{prop} Let $g.K$ be the image of $K$ under the action of $G$, and let
    $G_K = \{g \in G \mid g.K \cap K \neq \emptyset \}$. Then the action of $G$
is proper if for all compact $K \subset M$, $G_K$ is compact.  \end{prop}
    
    \begin{prop} The action of $G$ on $M$ is proper if for all convergent
    subsequences $\{p_i\}$ of $M$ and any sequence $\{g_i\}$ of $G$ such that
$\{g_i.p_i\}$ is convergent, then there exists a subsequence of $g_i$
converging in $G$.  \end{prop}

    \begin{rmk} \begin{enumerate} \item For a proper action, $G_K$ is compact
        when $K = \{p\}$.  \item Compact Lie group actions are proper.  \item
        For a discrete group, the action is proper only if $G_p$ is finite for
        any $p \in M$. Thus there exists an invariant neighborhood of $p$.
\end{enumerate} \end{rmk}

    \section{Lecture 12 (Oct 15)}% \label{sec:lecture_12_oct_15_}

    Recall that last time we discussed Lie group actions. Today we will relate
    them to manifolds.

    \subsection{Equivariance}% \label{sub:equivariance}
    
    

    \begin{thm} Suppose a Lie group $G$ has a proper free action on a manifold
        $M$. Then the orbit space is a topological manifold of dimension $\dim
        M - \dim G$ and there exists a unique smooth structure on $M/G$ such
        that the quotient map $M \to M/G$ is a submersion.  \end{thm}

    \begin{exm} Recall that the Hopf fibration is the quotient map $S^3 \to
    S^3/S^1 = \C\P^1$.  \end{exm}

    \begin{exm} Consider the smooth $\Z/2\Z$-action on $S^n$ by $\pm 1$. Then
    the quotient $S^n/(\Z/2\Z)$ is $\R\P^n$.  \end{exm}
    
    \begin{defn} Suppose $M, N$ are smooth $G$-manifolds. Then a smooth map
    $F:M \to N$ is \textit{$G$-equivariant} if it induces a natural
transformation of functors $G \to \mathbf{Diff}$.  \end{defn}    
    
    \begin{thm} Suppose $F:M \to N$ is an equivariant map between
    $G$-manifolds. Then suppose $G$ acts transitively on $M$. Then $F$ has
constant rank and $F^{-1}(c) \subset M$ is an embedded submanifold.  \end{thm}

    \begin{proof} Fix $p_0 \in M$. Then for all $p \in M$ there exists $g \in
        G$ such that $p = g \cdot p$. Then by equivariance, the diagram
        \begin{center} \begin{tikzcd} T_{p_0} M \arrow{r}{F_*} \arrow{d}{g} &
            T_{F(p_0)}N \arrow{d}{g} \\ T_pM \arrow{r}{F_*} & T_{F(p)}N
        \end{tikzcd} \end{center} commutes. Thus $F$ must be of constant rank.
        \end{proof}

    \begin{cor} Let $G$ have a smooth, free, proper action on $M$. Then the
    orbit $G.p$ is an embedded submanifold for all $p \in M$.  \end{cor}

    \begin{prop} Let $F:G \to H$ be a Lie group homomorphism. Then $\ker F$ is
    an embedded Lie subgroup of $G$.  \end{prop}

    \begin{proof} Consider the action of $G$ on itself and its induced action
    on $H$. Then use Theorem 147, \end{proof}

    \begin{defn} A smooth manifold $M$ is called a \textit{homogeneous space}
    if it admits a smooth transitive Lie group action for some $G$.  \end{defn}

    \begin{thm} Let $M$ be a homogeneous space with a transitive Lie group
    action of $G$. Fix any point $p \in M$. Then the map $G/G_p \to M$ is an
equivariant diffeomorphism. Moreover, $G_p$ is an embedded submanifold of $M$.
\end{thm}

    Alternatively, let $G$ be a Lie group and $H \subset G$ be a closed Lie
    group. Then consider the right $H$-action on $G$. Then this action is
    smooth and proper, and the quotient space $G/H$ is a smooth manifold of
    dimension $\dim G - \dim H$. Then it is easy to see that the $G$-action on
    $G/H$ is transitive, so we can define a homogeneous space to be $G/H$.

    \begin{proof}[Proof of Theorem 151] The $F$ is clearly well-defined by
        basic properties of groups. Then note that \[ F(g'gH) = (g'g)p = g'
        F(gH).\] Now set $G_p = F^{-1}(p)$, which is an embedded submanifold of
        $G$. Finally, it is easy to see that $F$ is a bijection. Because $F$
        has constant rank, then $F$ is a diffeomorphism.  \end{proof}

    \begin{exm} Consider $M = S^n$ and $G = O(n+1)$. Then $M$ has a natural
    transitive $G$-action. Let $p = (0, 0, \ldots, 0, 1)$ be the north pole.
Then $G_p = O(n)$, so $S^n = O(n+1)/O(n)$.  \end{exm}

    \begin{exm} Let $M = G_2(\R^4)$. Then consider the natural transitive
        action of $G = GL_2(\R)$. Let $p$ be the plane spanned by $e_1, e_2$.
        Then $G_p$ is the set of matrices of the block form $ \begin{pmatrix} A
        & B \\ 0 & C \end{pmatrix} $.

        Therefore $G_2(\R^4) = G/G_p$.  \end{exm}

    \subsection{Whitney's Embedding and Approximation Theorems}%
    \label{sub:whitney_s_embedding_and_approximation_theorems}

    \begin{thm}[Whitney Embedding Theorem] Let $M$ be a compact smooth
    $n$-manifold. Then there exists a smooth embedding of $M$ into $\R^{2n+1}$
and there exists a smooth immersion of $M$ into $\R^{2n}$.  \end{thm}

    \begin{thm}[Whitney's Approximation Theorem] Let $f:M \to R^k$ be a smooth
    map with $k \geq 2n+1$. Then for all $\ep > 0$ there exists a smooth
embedding $\widetilde{f}$ such that $\norm{f(p) - \widetilde{f}(p)}_{\infty} <
\ep$.  \end{thm}

    \begin{proof}[Proof of Theorem 154] First we need to embed $M$ into $\R^r$
        for some very large dimension $r$. By Proposition 43, there exists a
        finite regular cover $\{W_i \}$ of $M$. Now we build our regular
        partition of unity $\lambda_i: W_i \to \R$. Now define $f:M \to
        R^{m(n+1)}$ given by \[ x \mapsto (f_1, \ldots, f_m), \] where $f_i(x)
        = (\lambda_i \varphi_i(x) \lambda_i(x))$. We show that $f$ is a smooth
        embedding. It is easy to see that every pushforward is injective and
        that $f$ is injective. Because $M$ is compact, $f$ is a smooth
        embedding.

        Now given an embedding $M \to \R^r$ with $r>2n+1$, we will attempt to
        embed $M$ into $\R^{r-1}$. For $v \in S^{r-1}$, define $\pi_V: \R^r \to
        \R^{r-1}$ be the projection parallel to $v$. We want to choose $v \in
        S^{r-1}$ such that $\pi_V:M \to \R^{r-1}$ is an embedding. We need two
        conditions on $v$: \begin{enumerate} \item $v \neq (p-q)/\norm{p-q}$
        for all $p,q \in M$.  \item $v \neq w/\norm{w}$ for $w \in TM \subset
T\R^n$.  \end{enumerate} For the first condition, define the smooth map $F_1:M
\times M \setminus \Delta \to S^{r-1}$. We need $v \in S^{r-1} \setminus
\mr{Im}(R_1)$. For the second condition, let $T_1M = \{w \in TM \mid \abs{w} =
    1$. Then $T_1M$ is a smooth $(2n-1)$-manifold, so let $F_2: T_1M \to
    S^{r-1}$ given by projection in the second coordinate. Thus we need to
    choose $v \in S^{r-1} \setminus \mr{Im}(F_2)$.
        
        Now we need the following result:

        \begin{thm} Let $M,N$ be smooth manifolds with $\dim M < \dim N$ and
        let $F:M \to N$ be a smooth map. Then the subset $N \setminus
    \mr{Im}(F)$ is dense in $N$. In fact, the image of $F$ has measure $0$.
\end{thm}

        Therefore the desired $v$ exists. Running this procedure until $r =
    2n+1$, we have the desired embedding. Doing it again, we obtain an
immersion $M \to \R^{2n}$.  \end{proof}

    \section{Lecture 13 (Oct 16)}% \label{sec:lecture_13_oct_16_}
    
    Last time we stated Theorem 156, which says that the image of a smooth map
    $M \to N$ from a lower-dimensional manifold is measure zero. Recall the
    definition of measure from analysis. Here, we will consider the Lebesgue
    measure.

    \begin{rmk} \begin{enumerate} \item Being measure $0$ is a local property.
    \item $\Q^n \subset \R^n$ has measure $0$, \item If $k < n$, then $\R^k
\subset \R^n$ has measure $0$.  \end{enumerate} \end{rmk}

    \begin{thm}[Sard] Let $F:M \to N$ be a smooth map. Then the set of critical
    values has measure zero.  \end{thm}

    \subsection{Tensors}% \label{sub:tensors}
    
    Let $V$ be a real vector space, and define $T^k(V) = (V^*)^{\otimes k}$,
    called the space of \textit{covariant $k$-tensors}. Equivalently, this is
    the space of multilinear maps $V^{\otimes k} \to \R$.

    \begin{exm} The determinant is an alternating multilinear map $\bigwedge^n
    V \to \R$.  \end{exm}

    Define $T_{\ell}(V) = V^{\otimes \ell}$ to be the space of
    \textit{contravariant $\ell$-tnesors}. Similarly, we can define $T_{\ell}^k
    V = T^k(V) \otimes T_{\ell}(V)$, the space of \textit{mixed vectors of type
    $(k,\ell)$}.

    Now we will pass to the vector bundles. Let $E \to M$ be a real vector
    bundle of rank $n$. We may define the associated tensor bundles
    $T_{\ell}^k(E)$ analogously to the case of vector spaces.

    Let $\{e_i\}$ be a basis of $V$ with dual basis $\{ \ep_i \}$. Then we can
    form a basis $\{ \ep^{i_1} \otimes \cdots \otimes \ep^{i_k} \otimes e_{j_1}
    \otimes \cdots \otimes e_{j_{\ell}} \}$ of $T_{\ell}^k(V)$. 

    \begin{exm} Given a vector bundle $E \to M$, we can produce a smooth vector
    bundle $E \otimes E \to M$.  \end{exm}

    To do this, we use the construction lemma. To trivialize, we simply use the
    standard basis. The transition functions are simply given by the tensor
    powers of the original transitions. Finally, it is not hard to show that
    the transition functions are smooth.

    \begin{defn} Let $M$ be a smooth $n$-manifold. Then define $T^kM$ to be the
    \textit{bundle of covariant $k$-tensors}. Analogously, define $T_{\ell} M$
and $T_{\ell}^kM$.  \end{defn}

    \begin{rmk} $T^1M = T^*M$, $T_1M = TM$, ad $T_0M = T^0M = T_0^0M = M \times
    \R$.  \end{rmk}

    \begin{defn} A \textit{$(k,\ell)$-tensor field} is a smooth section of
    $T_{\ell}^k M$.  \end{defn}

    \begin{exm} $H^0(M, T_0M) = H^0(M, T^0M) = H^0(M, T_0^0M) = C^{\infty}(M)$.
    \end{exm}

    In local coordinates, we can write a tensor field as \[ \sigma = \sum
    \sigma^{J}_I dx^{\otimes I} \otimes \partial_x^{\otimes I}.\]

    \begin{prop} Consider an $\R$-linear map $\phi: \mc{X}(M)^{\otimes k} \to
        C^{\infty}(M)$. Suppose that $\phi$ is $C^{\infty}(M)$-linear. Then
        there exists a unique $\sigma \in H^0(M, T^kM)$ such that
        $\sigma_p(x_1, \ldots, x_k) = \phi(x_1, \ldots, x_k)(p)$.  \end{prop}

    \begin{defn} Let $F:M \to N$ be smooth. Then there is a pullback $F^*:
    H^0(N, T^kN) \to H^0(M, T^kM)$ defined analogously to the pullback of
$1$-forms.  \end{defn}

    \begin{rmk} If $k = 0$, then this is just the pullback of smooth functions.

        If $k = 1$, this is given by \begin{align*} (F^*\ df))_p(X) &=
        df_{F(p)}(F_{*,p}X) \\ &= F_{*,p}(X)(f) \\ &= X_p(f \circ F) \\ &= d(f
    \circ F)_p(X).  \end{align*} Therefore $F^*(df) = d(F^*f)$.  \end{rmk}

    In local coordinates, the computations become very cumbersome. Fortunately,
    in actual mathematics, people rarely work with more than four indices.

    \begin{prop} Let $\xi \in H^0(N,T^kN), \eta \in H^0(N,T^{\ell}N)$. Then
    $F^*(\xi \otimes \eta) = F^*(\xi) \otimes F^*(\eta)$. In addition, $(G
\circ F)^* = F^* \circ G^*$ and $F^*(f\sigma) = (f \circ F)F^*\sigma$.
\end{prop}
        

    \subsection{Riemannian Metrics}% \label{ssub:riemannian_metrics}

    \begin{defn} A smooth covariant $2$-tensor field $g \in H^0(M,T^2M)$ is a
        \textit{Riemannian metric} if for all $p \in M$, $g_p$ is a symmetric,
        positive definite, nondegenerate bilinear form. The pair $(M,g)$ is
        called a \textit{Riemannian manifold}.  \end{defn}
    
    In local coordinates, we write $g = \sum g_{ij} dx^i \otimes dx^j$. Then
    $g$ is a Riemannian metric if $(g_{ij})$ is a positive-definite symmetric
    matrix.

    \begin{exm} The standard example is the Euclidean metric on $\R^n$. A
    non-example from physics is the Lorentz spacetime metric, which has
signature $(3,1)$.  \end{exm}

    \section{Lecture 14 (Oct 21)}% \label{sec:lecture_14_oct_21}
    
    We continue our example of Riemannian manifolds from last time.

    \subsection{Riemannian Manifolds Continued}%
    \label{sec:riemannian_geometry_continued}
    
    \begin{defn} Let $(M,g)$ be a Riemannian manifold and let $\gamma:[a,b] \to
        M$ be a smooth curve. Define the \textit{length} of $\gamma$ to be \[
        L_g(\gamma) = \int_a^b g(\gamma'(t),\gamma'(t))\ dt. \] \end{defn}

    \begin{defn} Let $M,g$ be a connected Riemannian manifold. Define the
    \textit{distance} between $p,q$ to be the infimum over all paths $\gamma$
from $p$ to $q$ of $L(\gamma)$.  \end{defn}

    \begin{prop} Distance is well-defined.  \end{prop}

    \begin{prop} Every manifold is Riemannian.  \end{prop}

    \begin{proof} Let $(U_{\alpha},\varphi_{\alpha})$ cover $M$. Now let
        $f_{\alpha}$ be a partition of unity subordinate to $U_{\alpha}$.
        Define $g_{\alpha}$ to be the standard Riemannian metric on $\R^n$.
        Then let $g = \sum f_{\alpha}g_{\alpha}$. It is easy to see that this
        is a metric.  \end{proof}

    \begin{thm} $TM$ and $T^*M$ are isomorphic vector bundles.  \end{thm}

    \begin{proof} Choose a metric $g$ on $M$. Define $\widetilde{g}:TM \to
        T^*M$ by \[ X \mapsto g_p(X,-).\] This is the standard isomorphism $V
        \to V^*$ given by a nondegenerate bilinear form. This is a fiberwise
        isomorphism, so we simply need to check that this is smooth, which can
        be done locally, and is easy.  \end{proof}

    \begin{defn} Let $(M,g)$ be a Riemannian manifold. Then consider a local
    frame $\{e_i\}$ on $U \subset M$ is an \textit{orthonormal frame} with
respect to $g$ if $g(e_i,e_j) = \delta_{ij}$.  \end{defn}

    \begin{defn} Let $G \subset GL_n(\R)$ be a Lie subgroup. Then a (smooth)
        rank-n vector bundle $E$ \textit{admits a $G$-reduction to a
        $G$-bundle} if there exists a trivialization of $E$ such that the
        corresponding transition functions have image lying in $G$. With such a
        reduction, $E$ is called a $G$-bundle.  \end{defn}

    \begin{exm} If $E$ is a trivial bundle. Then $E$ admits a reduction to a
    $\{e\}$-bundle.  \end{exm}

    \begin{exm} If $E$ is an $O(n)$-bundle, then $E$ is self-dual. This follows
    from the definition of $O(n)$.  \end{exm}

    \subsection{Almost Complex Structures}% \label{sec:symplectic_manifolds}

    Recall that $\Hom(V,W) = V^* \otimes W$. Thus $H^0(M, T_1^1M) =
    H^0(M,\End(TM))$.

    \begin{defn} Let $J \in H^0(M, T_1^1M$. $J$ is called an \textit{almost
    complex structure} on $M$ if for all $p \in M$, $J_p: T_pM \to T_pM$
satisfies $J_p^2 = -\mr{Id}_p$.  \end{defn}

    \begin{prop} If $M$ admits an almost complex structure, then $M$ has even
    dimension. Also, $TM$ admits a reduction to a $GL_n(\C)$-bundle.
\end{prop}

    Now from linear algebra, if $J:\R^{2n} \to \R^{2n}$ satisfies $J^2 =
    -\mr{Id}$, then there exists a basis $\{x_i,y_i\}$ of $\R^{2n}$ such that
    $J$ acts like multiplication by $i \in \C$. For vector bundles, this
    applies to local trivializations and loca frames. Now consider the complex
    conjuation of $J$, allowing us to form the complex vector bundle
    $\overline{E}_J \simeq E_J^*$.

    \subsection{Differential Forms}% \label{sub:differential_forms}

    Differential forms form a basis for the link between analysis and topology,
    which culminates in the Atiyah-Singer index theorem. We begin with some
    multilinear algebra, then build forms over a manifold.

    \subsubsection{Multilinear Algebra}% \label{ssub:multilinear_algebra}

    Let $V$ be a $n$-dimensional real vector space and consider the symmetric
    group $S_k$. Recall the notion of the sign of a permutation $\sigma$. 

    \begin{defn} $T \in T^k(V)$ is alternating if for all $X_1, \ldots,X_k \in
    V$, $T(X_1, \ldots, X_n) =\ ^{\sigma}T(X_1, \ldots, X_n)$.  \end{defn}

    Denote by $\bigwedge^kV$ to be the subspace of alternating elements of
    $T^k(V)$.

    \begin{defn} Define the alternating projection by \[ T \mapsto \frac{1}{k!}
    \sum_{\sigma} \mr{sign}(\sigma)\ ^{\sigma}T.\] \end{defn}

    A basis for $\bigwedge^2V$ is simply $e^i \otimes e^j - e^j \otimes e^i$.
    This space has dimension $\binom{n}{2}$. Also, the top exterior power has
    dimension $1$ and basis element $T_0 = \sum_{\sigma} \ep^{\otimes \sigma}
    \mr{sign}(\sigma)$.

    We can verify that $T_0$ acts by the determinant and that any linear map $V
    \to V$ pulls back to $\bigwedge^kV$. On the top exterior power, the
    pullback is multiplication by the determinant.
    
    \section{Lecture 15 (Oct 23)}% \label{sec:lecture_15_oct_23_}
    
    \begin{notn} For $I = (i_1, \ldots, i_k)$, define $\ep^I = k!
    \mr{Alt}(\ep^{i_1} \otimes \cdots \otimes \ep^{i_k})$. We call this the
elementary alternating tensor.  \end{notn}

    Recall the standard basis for $\bigwedge^k V$. Also, note that \[
        \ep^I(X_1, \ldots, X_k) = \det \begin{pmatrix} \ep^{i_1}(X_1) & \cdots
        & \ep^{i_1}(X_k) \\ \vdots & & \vdots \\ \ep^{i_k}(X_1) & \cdots &
    \ep^{i_k}(X_k) \end{pmatrix}. \]

    \begin{defn} For $\omega \in \bigwedge^k(V), \eta \in \bigwedge^{\ell}(V)$,
        define the \textit{wedge product} by \[ \omega \wedge \eta =
        \frac{(k+\ell)!}{k!\ell !} \mr{Alt}(\omega \otimes ]eta).\] \end{defn}

    \begin{prop} \begin{enumerate} \item $\ep^I \otimes \ep^J = \ep^{IJ}$ if
        $I,J$ are distinct.  \item The wedge product is bilinear.  \item The
        wedge product is associative.  \item The wedge product is
        graded-commutative.\footnote{This suggests a cohomology theory.} \item
If $\omega^1, \ldots, \omega^k \in V^*$, then $(\omega_1 \wedge \cdots \wedge
\omega^k)(X_1, \ldots, X_k) = \det(\omega^j(X_i))$.  \end{enumerate} \end{prop}

    \begin{defn} The exterior algebra of $V$ is given by \[ \bigwedge^*(V) =
    \bigoplus_{k=0}^n \bigwedge^k(V).\] \end{defn} Note that the exterior
    algebra is graded-commutative.

    \begin{defn} The \textit{contraction} or interior multiplication on
        $\bigwedge^*(V)$ is defined by \[ i_X(\omega)(x_1, \ldots, X_{k-1}) =
        \omega(X,X_1, \ldots, X_{k-1}). \] \end{defn}

    \begin{lem} \begin{enumerate} \item $i_X^2 = 0$.  \item $i_{aX+bY} = ai_X +
        bi_Y$.  \item For all $\omega \in \bigwedge^k(V), \eta \in
        \bigwedge^{\ell}(V)$, $i_X(\omega \wedge \eta) = i_X(\omega) \wedge
        \eta + (-1)^{k} \omega \wedge (i_X \eta)$.  \end{enumerate} \end{lem}

    \begin{exm} For $I = (i_1, \ldots, i_k)$, then $i_{e_\ell} \ep^I = 0$ if
    $\ell \notin I$, and $(-1)^{s+\ell} \ep^{i_1} \wedge \cdots \wedge
\widehat{\ep^{i_s}} \wedge \cdots \wedge \ep^{i_k}$ if $\ell = i_s$.  \end{exm}

    \begin{exm} Let $\omega \in \bigwedge^2(V)$. Ten write $\omega = \sum
        w_{ik} \ep^i \wedge \ep^j$. Now form an antisymmetric matrix from the
        $\omega_{ij}$. Then define $\widetilde{\omega}:V \to V^*$ by $X \mapsto
        i_X \omega$. Writing $X = \sum X^ie_i$ and computing, we find that
        $\widetilde{\omega}$ is an isomorphism if $\det (w_{ij}) = 0$.
    \end{exm}

    \begin{defn} If $\omega \in \bigwedge^2(V)$ is nondegenerate, then $\omega$
    is a \textit{symplectic form} on $V$.  \end{defn}

    \begin{lem} Suppose $(V,\omega)$ is symplectic. Then there exists a basis
        $e_1, \ldots, e_m, f_1, \ldots, f_m$ such that $\omega(e_i,e_j) =
        \omega(f_i,f_j) = 0$ and $\omega(e_i,f_j) = \delta_{ij}$. This is
        called a \textit{symplectic basis}.  \end{lem}

    Note the standard symplectic form is $\omega = \sum dx^i \wedge dy_i$.

    \begin{proof}[Proof of Lemma 193] First, for $e_1$, choose $f_1$ such that
        $\omega(e_1,f_1) = 1. $Let $W$ be the span of $e_1,f_1$. Then take the
        orthogonal complement $W^{\perp}$.

        First we show taht $W \oplus W^{\perp} = V$. To see that the
        intersection is trivial. Then given $v \in V$, 

        Next we show that $\omega$ is a symplectic form restricted to
    $W^{\perp}$. This is easy assuming the above and using nondegeneracy of
$\omega$.  \end{proof}

    \begin{defn} Let $(V,\omega)$ be a symplectic space. A subspace $W \subset
        V$ is \begin{itemize} \item \textit{Symplectic} if $W \cap W^{\perp} =
            \{0\}$; \item \textit{Isotropic} if $W \subset W^{\perp}$; \item
            \textit{Coisitropic} if $W \supset W^{\perp}$; \item
    \textit{Lagrangian} if $W = W^{\perp}$.  \end{itemize} \end{defn}

    \begin{exer}[Hw4 Problem 12-9] For each $W \subset (V,\omega)$ of the above
        type, there exists a symplectic basis $e_1, \ldots, e_m, f_1, \ldots,
        f_m$ such that \begin{enumerate} \item If $W$ is symplectic, then $e_1,
            \ldots, e_k, f_1, \ldots, f_k$ is a basis for $W$.  \item If $W$ is
            isotropic, then $W$ is the span of $e_1, \ldots, e_k$.  \item If
            $W$ is coisotropic, then $W$ is the span of $e_1, \ldots, e_n, f_1,
            \ldots, f_k$.  \item If $W$ is Lagrangian, then $W$ is the span of
            $e_1, \ldots, e_n$.  \end{enumerate} \end{exer}

    There are two generalizations. First, we relax nondegeneracy of $\omega$
    and obtain vectors $g_1, \ldots, g_k$ that kill all of the $e_i,f_i$. The
    $g_i$ measure the degeneracy of $\omega$.

    If $(V,\omega)$ is a symplectic space and $g$ is a metric, then there
    exists a symplectic basis which is orthonormal with respect to $g$.

    \section{Lecture 16 (Oct 28)}% \label{sec:lecture_16_oct_28_}
    
    Today we will discuss manifold versions of the constructions we performed
    last time. 

    \subsection{Differential Forms on Manifolds}%
    \label{sub:differential_forms_on_manifolds}
    
    Define $\bigwedge^kM$ to be the $k$-th exterior power of the tangent
    bundle. This is a smooth vector bundle of rank $\binom{n}{k}$.

    \begin{defn} A smooth section of $\bigwedge^k M$ is called a
    \textit{differential $k$-form}.  \end{defn}

    We will denote the space\footnote{sheaf} of differential $k$-forms by
    $\Omega^k(M)$. This is a subspace of $H^0(M, T^kM)$. We may define the
    wedge product and contraction of differential forms as we did for vector
    spaces. This is a graded derivation. In addition, we may define local
    versions of all of these notions.

    \begin{defn} The pullback of differential $k$-forms is defined by
    $F^*\omega (X) = \omega (F_*X)$.  \end{defn}

    \begin{lem} Let $F:M \to N$ be a smooth map. Then \begin{enumerate} \item
        The pullback is linear.  \item The pullback respects the wedge
        product.\footnote{Cohomology, anyone?} \item If $X \in H^0(M,TM), Y \in
        H^0(N,TN)$ are $F$-related, then $i_X(F^*\omega) = F^*(i_Y \omega)$.
        \item In any smooth chart, $F^*(\sum_I \omega_I dy^{\wedge I}) = \sum_I
    (\omega_I \circ F) d(y^{\wedge I} \circ F)$.  \end{enumerate} \end{lem}

    \begin{prop} If $M,N$ have equal determinant, then $F^*$ is simply the
    determinant of the Jacobian.  \end{prop}

    \begin{defn} The \textit{exterior algebra}\footnote{complex} $\Omega^*(M)$
    is defined by $\bigoplus_k \Omega^k(M)$.  \end{defn}

    \subsection{Exterior Differentiation}% \label{sub:exterior_differentiation}
    
    \begin{thm} Let $M$ be a smooth manifold. Then there exists a unique
        $\R$-linear map $d: \Omega^k(M) \to \Omega^{k+1}(M)$ such that:
        \begin{enumerate} \item For $f \in C^{\infty}(M) = \Omega^0(M)$, $df$
            is the differential of $f$ as defined before.  \item $d$ is a
        graded derivation.  \item $d^2 = 0$.  \end{enumerate} Moreover, these
        imply: \begin{enumerate}[label=(\alph*)] \item In local coordinates, $d
            \left( \sum_I \omega_I dx^{\wedge I} \right) = \sum_I d\omega_I
            \wedge dx^{\wedge I}$.  \item If $\omega = \widetilde{\omega}$ on
    open $U \subset M$ then $d\omega = d\widetilde{\omega}$.  \item For $U
    \subset M$ open, $d(\omega |_U) = (d\omega)|_U$.  \end{enumerate} \end{thm}
    
    \begin{lem} Let $F:M \to N$ be a smooth map. Then $d(F^*\omega) =
    F^*(d\omega)$.  \end{lem}

    The proof involves computing in a local chart using Lemma 198.

    \begin{prop} Let $\omega \in \Omega^1, X,Y \in H^0(M,TM)$. Then $d
        \omega(X,Y) = X(\omega(Y)) - Y(\omega(X)) - \omega([X,Y])$. More
        generally, for $\omega \in \Omega^k(M)$ and $X_1, \ldots, X_{k+1}$
        vector fields, we have \[ d\omega(X_1, \ldots, X_{k+1}) = \sum_i
        (-1)^{i-1} X_i \omega(X_1, \ldots, X_{k+1}) + \sum_{i\leq j} (-1)^{i+j}
    \omega([X_i,X_j], X_1, \ldots, X_{k+1}). \] \end{prop}

    \begin{exm} Let $X,Y$ be vector fields. Then if $[X,Y] = 0$, $\alpha,\beta$
    are dual to $X,Y$, and $\alpha_p(Y_p) = 0 = \beta_p(X_p)$, then $d\alpha =
0 = d\beta$.  \end{exm}

    \begin{defn} $\omega \in \Omega^k(M)$ is \textit{closed} if $d\omega = 0$.
    It is \textit{exact} if $\omega = d\eta$ for some $\eta \in
\Omega^{k-1}(M)$.  \end{defn}

    Note that exact forms are closed. Thus we may define a cohomology theory,
    called \textit{de Rham cohomology}, and denoted $H^k_{\mr{dR}}(M)$.

    \begin{exm} All de Rham cohomology groups are trivial on $\R^n$. However,
    $H^1(S^1)$ is not trivial.  \end{exm}

    \begin{defn} $\omega \in \Omega^2(M)$ is called a \textit{symplectic form}
    if it is closed and nondegenerate.  \end{defn}
    
    The pair $(M,\omega)$ is called a \textit{symplectic manifold}.

    \begin{defn} Let $(M,\omega)$ be a symplectic manifold of dimension $2N$.
        Let $Q \subset M$ be a submanifold of dimension $N$. Let $i:Q
        \hookrightarrow M$ be the smooth embedding. Then $Q$ is a
        \textit{Lagrangian submanifold} if $i^*\omega = 0$.  \end{defn}

    \begin{exm} Let $(M,\omega)$ be $\R^{2n}$ with the standard symplectic
    form. Then let $Q$ be given by setting all $y^i$ to be constant.  \end{exm}

    \begin{defn} A diffeomorphism $F:(M,\omega) \to (M', \omega')$ is a
    \textit{symplectomorphicm} if $F^*\omega' = \omega$.  \end{defn}

    \begin{exer}[Homework] Suppose $F:M \to M$ is a symplectomorphism. Then let
        $N = M \times M$. Let $\omega_N = \pi_1^* \omega - \pi_2^* \omega$.
        Then \begin{enumerate} \item $N$ is a symplectic manifold.  \item The
            graph of $F$ is a Lagrangian submanifold of $N$.  \end{enumerate}
        \end{exer}

    \begin{exm} Suppose $M$ is smooth. Then $T^*M$ has a canonical symplectic
        form. We will define $\tau \in \Omega^1(T^*M)$ as follows:

        For all $p \in M, v\in T^*_p M$, let $\tau_{(p,v)} = \pi^*(p,v) \circ
    V$. $\tau$ is called the tautological $1$-form for $T^*M$. Then we define
the canonical symplectic form to be $\omega = -d\tau$.  \end{exm}

    \section{Lecture 17 (Oct 30)}% \label{sec:lecture_17_oct_30_}
    
    We will have our makeup class on November 11th at $1$PM.

    \begin{defn} Two submanifolds $N_1,N_2 \subset M$ are \textit{transverse}
    if for all $p \in N_1 \cap N_2$, $T_pN_1, T_pN_2$ span $T_pM$.  \end{defn}

    \begin{exer}[Homework] Let $S \subset T^*M$. Then $S$ is Lagrangian and is
    transverse to each fiber $T_p^*M$ and intersects each fiber at one point if
and only if $S$ is the image of a closed $1$-form.  \end{exer}

    \begin{exm} For $K \subset \R^3$ a knot, define $L_K = \{ \xi \in T^*\R^3
    \mid \xi(v) = 0 \text{ for all } v \in T_pK, p \in K\}$. Then $L_K$ is
Lagrangian.  \end{exm}

    \subsection{Orientations}% \label{sub:orientations}
    
    We first define orientations for vector spaces and then for manifolds.

    \begin{defn} An \textit{orientation} for an $\R$-vector space $V$ is an
    equivalence class of ordered bases of $V$. Here $(e_1, \ldots, e_n) \simeq
(f_1, \ldots, f_n)$ if the change of basis matrix $A$ has positive determinant.
\end{defn}

    \begin{lem} Alternatively, an orientation is a choice of a nonzero element
    $\Omega \in \bigwedge^n(V)$.  \end{lem}

    The proof of this is a basic fact about the top exterior power of a vector
    space.

    \begin{defn} Given a choice of orientation $[(e_1, \ldots, e_n)]$, we say
    $(e_1', \ldots, e_n')$ is \textit{positively oriented} if it is in the same
equivalence class. Otherwise, we call it \textit{negatively oriented}.
\end{defn}

    \begin{defn} Given an orientation $[\Omega]$, we say that $\Omega$ is
    \textit{positively oriented} if $\Omega(e_1, \ldots, e_n) > 0$.  \end{defn}

    Now let $M$ be a smooth manifold.

    \begin{defn} An \textit{orientation} on $M$ is an equivalence class of
        non-vanishing continuous sections of the canonical bundle $K_M$. Here
        $\Omega, \Omega' \in H^0(M, K_M)$ are equivalence if $\Omega = \lambda
        \Omega'$ where $\lambda \in C^0(M)$ is a everywhere positive continuous
        function.  \end{defn}

    \begin{defn} $M$ is \textit{orientable} if it admits an orientation. A
    global frame $(e_1, \ldots, e_n)$ is \textit{positively oriented} if
$(e_1|_p, \ldots, e_n|_p)$ is positively oriented with respect to $\Omega_p$.
\end{defn}

    \begin{defn} A choice of $\Omega \in [\Omega]$ is called a \textit{volume
    form}.\footnote{More on this later in the course.} \end{defn}

    \begin{rmk} $S^2, \mathbb{T}^2$ are orientable, but $\R\P^2$ is not. Also,
    all parallelizable manifolds are orientable. Finally, if $M$ is connected
and orientable, there are only two orientations of $M$.  \end{rmk}

    \begin{defn} A collection of charts $\{(U_{\alpha}, \varphi_{\alpha})\}$ is
    \textit{consistently oriented} if $\det J(\varphi_{\beta}
\varphi_{\alpha}^{-1}) > 0$ for all $\alpha,\beta$.  \end{defn}

    \begin{exer}[Lee, Exercise 13.3] $M$ is orientable if and only if it has a
    consistently orientable collection of charts.  \end{exer}

    \begin{defn} A local diffeomorphism $F:M \to N$ between smooth oriented
    manifolds is \textit{orientation preserving} (resp \textit{reversing}) if
$[F^*\Omega_N] = [\Omega_M]$ (resp. $-[\Omega_M]$).  \end{defn}

    \begin{prop} Let $M$ have dimension at least $2$ and $N \subset M$ be a
        hypersurface. Suppose there exists a continuous section $S: N \to
        TM|_N$ such that for all $p \in N$, $S(p) \notin T_pN$. Then for all
        orientations $\Omega_M$ on $M$, there exists an induced orientation
        $\Omega_S$ on $N$ determined by $S$.  \end{prop}

    \begin{rmk} $S$ and $-S$ induce the two orientations on $N$ if $N$ is
        connected. In addition, define $\nu_N = i^*(TM) / TN$ the
        \textit{normal bundle} of $N$ in $M$. Then $S$ descends to a section of
        $N$. If $M$ is orientable, then $\Omega_M$ induces an orientation on
        $N$ if and only if $\nu_N$ is trivial. In this case, we call $N$
        \textit{co-orientable}.  \end{rmk}

    \begin{exm} Consider $S^n \subset \R^{n+1}$. Choose the standard
    orientation on $\R^{n+1}$. If $S^n$ is embedded as the unit sphere, let
$S(p) = p \in \R^{n+1}$. then $\Omega_S$ is an orientation for $S^n$.
\end{exm}

    \begin{exm} $\R\P^2$ is not orientable. To show this, define the antipodal
        map $A: S^2 \to S^2$. This gives a smooth covering $\pi: S^2 \to
        \R\P^2$. Now let $S: S^2 \to T\R^3|_{S_2}$ be as in the previous
        example and let $\Omega_S$ be the orientation defined above. It is easy
        to see that $A$ is orientation reversing. 

        Now suppose $\R\P^2$ is orientable with volume form $\Omega$. Then let
    $\widetilde{\Omega} = \pi^*\Omega$. Then $\pi \circ A = \pi$, which implies
that $A^*\widetilde{\Omega} = - \widetilde{\Omega}$, which gives us a
contradiction.  \end{exm}

    \begin{exer}[Homework] $\R\P^n$ is orientable if and only if $n$ is odd.
    \end{exer}

    \begin{exm} Let $M$ be orientable and $f \in C^{\infty}(M)$. Then
    $f^{-1}(r)$ is an orientable hypersurface for all regular values $r$ of
$f$.  \end{exm}

    \section{Lecture 18 (Nov 4)}% \label{sec:lecture_18_nov_4_}
    
    The makeup class has been moved to November 11 at 10 AM with bagels, cream
    cheese, and lox.

    \subsection{Orientation Coverings}% \label{sub:orientation_coverings}
    
    \begin{lem} Let $M$ be a smooth manifold of dimension $n$. Then $R^{+}$
    acts on $\bigwedge^n M$ by fiberwise multiplication. This action is smooth,
free, and proper.  \end{lem}

    \begin{defn} The quotient space $\widetilde{M}$ is called the
    \textit{orientation covering of $M$}.  \end{defn}

    \begin{thm} Let $M$ be a smooth connected $n$-manifold. Then we have a
        smooth surjective map $\widetilde{M} \to M$. Then: \begin{itemize}
            \item $M$ is orientable if and only if there exists a global
                section.  \item $M$ is not orientable if and only if
                $\widetilde{M}$ is connected.  \end{itemize} \end{thm}

    \begin{thm} Let $M$ be a smooth connected $n$-manifold which is not
        oriented. Then there exists a unique double cover $\widetilde{M} \to M$
        such that $\widetilde{M}$ is orientable. Moreover, $\widetilde{M}$ is
        diffeomorphic to the orientation covering.  \end{thm}

    \begin{prop} Suppose $M$ is a connected oriented smooth manifold and
        $\Gamma$ is a discrete group acting smoothly, freely, and properly on
        $M$. We say the action is \textit{orientation-preserving} if each
        diffeomorphism $\gamma \in \Gamma$ is orientation preserving. Then
        $M/\Gamma$ is orientable if and only if $\Gamma$ is
        orientation-preserving.  \end{prop}

    \begin{cor} If $\pi_1(M)$ has no subgroups of index $2$, then $M$ is
    orientable. More generally, there is a bijection between nontrivial line
bundles over $M$ and morphisms $\pi_1(M) \to \Z/2\Z$.  \end{cor}

    The idea of the proof is to construct the first Stiefel-Whitney
    class\footnote{This is a characteristic class like the Chern class.} of
    $E$. 

    \begin{prop} Let $(M,g)$ be an orientable Riemannian manifold of dimension
        $n$. Then there exists a unique orientation form $\Omega$ such that for
        all oriented local orthonormal frames, we have $\Omega(e_1, \ldots,
        e_n) = 1$. $\Omega$ us called a \textit{volume form} for $(M,g)$.
    \end{prop}

    \begin{lem} Let $(M,g)$ be as above. Write $g = (g_{ij})$ in a local
        positively oriented coordinate chart. Then locally, the volume form is
        given by \[ \sqrt{\det(g_{ij})} dx^1 \wedge \cdots \wedge dx^n.\]
    \end{lem}

    \begin{proof} Let $\Omega = f\ dx^1 \wedge \cdots \wedge dx^n$ with $f >
        0$. Let $(e_1, \ldots, e_n)$ and $\ep^1, \ldots, \ep^n$ be a positively
        oriented orthonormal frame and its coframe. The rest of the proof is
        computing the determinant of the change of basis matrix $A$ from $e_j$
        to the $\delta_i$ coordinates.  \end{proof}

    \subsection{Manifolds with Boundary}% \label{sub:manifolds_with_boundary}
    
    Define $\mathbb{H}^n$ to be the closed upper halfspace in $\R^n$.

    \begin{defn} $M$ is an $n$-dimensional \textit{smooth manifold with
        boundary}\footnote{For some notions of stratification, this admits a
    stratification.} if \begin{enumerate} \item $M$ is Housdoeff and
        second-countable; \item $M$ has a cover $\{U_{\alpha},
        \varphi_{\alpha}\}$ where $\varphi{\alpha}$ is a homeomorphism onto its
        image in $\mathbb{H}^n$ or $\R^n$; \item The transition functions are
smooth.  \end{enumerate} \end{defn}

    The notion of a manifold with boundary extends to a manifold with corners,
    which are modeled on intersections of closed half-spaces. For simplicity,
    we will stick to boundaries.

    We may also define smooth maps between manifolds with boundaries and also
    $T_pM$ the space of derivations with local representations.\footnote{All
    tangent spaces here still have the same dimension.} We can also construct
    the cotangent bundle, tensor fields, and differential forms as before. In
    addition, we can define pushforwards and pullbacks.

    \begin{defn} We may define the interior and the boundary of a manifold with
    boundary $M$.  \end{defn}

    \begin{prop} $M$ is the disjoint union of the interior and the boundary. In
    addition, the boundary has a unique smooth structure such that $\partial M
\to M$ is a smooth embedding.  \end{prop}

    \begin{defn} Let $p \in \partial M$, $N \in T_pM$ such that $N \notin
    T_p(\partial M)$. Then $N$ is \textit{inward pointing} if there exists $\ep
> 0$ and smooth path $\gamma: [0,\ep] \to M$ such that $\gamma'(0) = N$.
\end{defn}

    \begin{lem} \begin{enumerate} \item If $N$ is inward pointing then for all
    charts, $dx^i(N) > 0$.  \item If there exists a chart such that $dx^n(N) >
    0$, then $N$ is inward-pointing.  \end{enumerate} \end{lem}

    \begin{lem} There exists a smooth outward pointing vector field $N$ along
    $\partial M$.  \end{lem}

    \begin{cor} The normal bundle of $\partial M$ is trivial.  \end{cor}

    \section{Lecture 19 (Nov 6)}% \label{sec:lecture_19_nov_6_}
    
    \begin{proof}[Proof of Lemma 246] Take charts with $\varphi_{\alpha}:
        U_{\alpha} \to \H^n$. Then let $V_{\alpha} = U_{\alpha} \setminus
        \partial M$. This is an atlas for the boundary.

        Now let $\{f_{\alpha} \} $ be a partition of unity subordinate to
    $\{V_{\alpha}\}$. Then we define $N(p) = \sum_{\alpha} f_{\alpha}(p)
N_{\alpha}(p)$, and it is easy to check that this is inward pointing.
\end{proof}

    \begin{prop} Let $M$ be an oriented smooth manifold with orientation
    $\Omega_M$ and suppose $M$ has nonempty boundary. Define $\Omega_{\partial
M} = i_{-N}\Omega_M$. Then this is a well-defined orientation for the boundary.
\end{prop}

    \begin{proof} Let $N'$ be another inward pointing normal vector field. Thus
        $N' = fN + T$, where $f$ is a positive smooth function and $T$ is a
        vector field on $\partial M$. Then we simply compute
        $i_{-N'}\Omega_M(p)$ and see that the orientation is well-defined.
    \end{proof}

    \begin{exm} Let $M = \H^n$ and take the standard orientation form $dx^1
        \wedge \cdots \wedge dx^n$. Then we have \[ i_{\partial_n}(dx^1 \wedge
        \cdots \wedge dx^n) = (-1)^n dx^1 \wedge \cdots \wedge dx^{n-1}.\]
    \end{exm}

    \begin{lem} Let $M$ be connected and let $S$ be a connected submanifold of
        codimension $1$. Suppose $M \setminus S = M_1 \sqcup M_2$ is
        disconnected. Then $\overline{M}_i$ are manifolds with boundary $S$.
        Also, $\nu_S \to S$ is trivial.  \end{lem}

    \begin{exm} Let $M = S^2, S = S^1$ Then $S^2 \setminus S^1$ is not
    connected, so the normal bundle is trivial.  \end{exm}

    \begin{exm} If $M = \mathbb{T}^2, S = S^1$, then $\nu_S$ is trivial but $M
    \setminus S$ is not trivial.  \end{exm}

    \begin{exm} If $M = \R\P^2, S = \R\P^1$, then $\nu_S$ is the tautological
    bundle. Therefore $\R\P^2 \setminus \R\P^1 = \R^2$ is connected.  \end{exm}

    \begin{thm}[Tubular Neighborhood Theorem] If $\partial M$ is compact, then
    there exists a neighborhood of $\partial M$ in $M$ diffeomorphic to
$\partial M \times [0,\ep)$.  \end{thm}

    \subsection{Integration}% \label{sub:integration}
    
    Let $U \subset \R^n$ with orientation $dx^1 \wedge \cdots \wedge dx^n$.
    Suppose $\omega \in \Omega^n(U)$ with compact support. Thus we have $\omega
    = f dx^1 \wedge \cdots \wedge dx^n$ for some compactly supported $f$. Then
    $D$ is an oriented compact manifold with boundary, which we call the
    \textit{domain of integration}.

    Then we define \[ \int_U \omega \coloneqq \int_D f dx_1 \cdots dx_n\] as
    the usual Riemann (or Lebesgue) integral. Similarly, we can define
    integrals on subsets of $\H^n$.

    \begin{prop} Let $D,E$ be domains of integration in $\R^n$ with $\omega \in
        \Omega^n(E)$. Let $F:D \to E$ be smooth such that $F$ is a
        diffeomorphism on the interior of $D$ that is orientation preserving.
        Then \[ \int_E \omega = \int_D F^*\omega.\] \end{prop}

    The proof of this fact is a change of basis calculation in multivariate
    calculus.

    To turn all of this into a global story, we use a partition of unity. Here,
    $M$ is an oriented smooth $n$-manifold. Let $\omega \in \Omega^n(M)$ with
    compact support. Choose a positively oriented atlas $U_{\alpha}$ and
    $f_{\alpha}$ be a partition of unity subordinate to $U_{\alpha}$. Then we
    define \[ \int_M \omega = \sum_{\alpha} \varphi_{\alpha}(U_{\alpha})
    (\varphi_{\alpha}^{-1})^* f_{\alpha} \omega.\]

    \begin{lem} The integral is independent of the chart and the partition of
    unity.  \end{lem}

    \begin{prop} Integration is linear and orientation-sensitive. In addition,
    integration is invariant under diffeomorphism.  \end{prop}

    For a $0$-dimensional manifold, a $0$-form is simply a function to $\R$,
    and an orientation is an assignment of a sign to each point. Then the
    integral is simply a finite sum.

    \begin{thm}[Stokes] Let $M$ be an oriented smooth manifold of positive
        dimension. Then suppose $\omega \in \Omega^{n-1}(M)$ has compact
        support. Then \[ \int_M d\omega = \int_{\partial M} \omega.\] \end{thm}

    \begin{proof} It suffices to prove this locally. We have a chart $V$ with
        $\varphi:V \to \H^n$ a positively oriented chart. Then write \[\omega =
        \sum_{i=1}^n \omega_i dx^1 \wedge \cdots \wedge \widehat{dx^i} \wedge
    \cdots \wedge dx^n.\]

        Then we have \[d\omega = \sum_{i=1}^n \sum_{j=1}^n \frac{\partial
        \omega_i}{\partial x^k} dx^j \wedge dx^1 \cdots \wedge \widehat{dx^i}
    \wedge \cdots \wedge dx^n.\]

        Therefore we have 

        \begin{align*} \int d \omega &= \int_{-R}^R \cdots \int_{-R}^R \int_0^R
            \sum_{i=1}^n (-1)^{i-1} \frac{\partial \omega_i}{\partial x_i} \d
            x^1 \cdots \d x^n \\ &= (-1)^n \int_{-R}^R \cdots \int_{-R}^R
            \omega_n(0) \d x^1 \cdots \d x^{n-1} \\ &= \int_{\varphi(V) \cap
        \partial \H^n} \omega.  \end{align*} \end{proof}

    \section{Lecture 20 (Nov 11)}% \label{sec:lecture_20_nov_11_}

    We will complete the global proof of Stokes' theorem. We simply cover the
    support of $\omega$ by finitely many positively oriented charts and let
    $f_{\alpha}$ be a partition of unity subordinate to $V_{\alpha}$. Then we
    simply have

    \begin{align*} \int \omega &= \int \sum f_{\alpha} \omega \\ &= \sum \int
    f_{\alpha} \omega \\ &= \sum \int d (f_{\alpha}) \omega \\ &= \sum \int
df_{\alpha} \wedge \omega + f_{\alpha} d\omega \\ &= \int d\omega.
\end{align*}

    Now we take a detour into geometry and analysis,

    \begin{defn} Let $(M,g)$ be an oriented Riemannian manifold with dimension
        $n$. A \textit{Riemannian volume form} is $dV_g = \xi^1 \wedge \cdots
        \wedge \xi^n$, where $\xi^i$ form a positively oriented orthonormal
        coframe. The \textit{volume} of $M$ is $\int_M dV_g$. More generally,
        if $f$ is a compactly supported continuous function on $M$, define the
        \textit{integral of $f$ over $M$} to be \[ \int_M f = \int_M f\d V_g.\]
    \end{defn}

    \begin{rmk} The volume form generalizes to densities, which in turn
    generalize to measures.  \end{rmk}

    \begin{lem} Let $(M,g)$ and $f$ be continuous and compactly supported. Then
    if $f \geq 0$, $\int_M f \geq 0$. In addition, $f = 0$ if and only if $f
\geq 0$ and $\int_M f = 0$.  \end{lem}

    Now let $\widetilde{g}$ be an induced Riemannian metric on $\partial M$ and
    let $N$ be an outward-pointing unit normal vector field along $\partial M$.
    Now define the volume form on $\partial M$ by $d \widetilde{V_g} =
    i_N(dV_g)$.

    Note that for any $X \in H^0(M,TM)$, we have $X = \gen{N,X}_g N + T$.  This
    allows to write \[ i_X dV_g = \gen{N,X}_g d \widetilde{V}_g.\]

    \begin{defn} For all vector fields $X$, define the divergence
    $\operatorname{div}(X)$ by $\operatorname{div}(X) dV_g = d(i_X dV_g)$.
\end{defn}

    \begin{thm}[Divergence Theorem] Let $(M,g)$ be a Riemannian manifold with
        outward pointing unit normal vector field $N$. Then \[ \int_M
        \operatorname{div}(X) dV_g = \int_{\partial M} \gen{X,N}_g
    d\widetilde{V}_g.\] \end{thm}

    This is simply a special case of Stokes' theorem. Also, the Divergence
    theorem is called Gauss theorem in $\R^3$.

    \begin{exm} Consider $\R^n$ with the usual metric. Then the volume form is
    simply $dx_1 \wedge \cdots \wedge dx_n$. Then define $X = \sum x^i
\partial_{x^i}$. Then we can verify that $\operatorname{div}(X) = n$.
\end{exm}

    \subsection{De Rham Cohomology}% \label{sub:de_rham_cohomology}
    
    Note here we do not need to know algebraic topology for this. Let $M$ be a
    smooth $n$-manifold. Recall the de Rham complex \[ 0 \to \Omega^0(M) \to
    \cdots \to \Omega^n(M) \to 0.\] This is known as a \textit{(co)chain
    complex}. For any such complex, we may define \textit{(co)cycles} $Z^p$ as
    $\ker(d)$ and \textit{(co)boundaries} $B^p = \operatorname{Im}(d)$. Then we
    may define the \textit{(co)homology} as $H^p = Z^p/B^p$. The cohomology of
    the de Rham complex is known as \textit{de Rham cohomology}, denoted by
    $H^*_{dR}(M)$.

    \begin{prop} Let $G:M \to N$ be smooth. Then this induces a graded map
    $G^*: H^*_{dR}(N) \to H^*_{dR}(M)$. Moreover, cohomology is a functor from
$\mathbf{Diff}^{op} \to \mathbf{Ab}$.  \end{prop}

    \begin{proof} Note that we have a map between the de Rham complexes of $M,
    N$ that commutes with $d$. This induces a map on the cohomology.
\end{proof}

    \begin{defn} $F,G:M \to N$ are \textit{smoothly homotopic} if there exists
    a smooth homotopy $H:M \times [0,1] \to N$ between them.  \end{defn}

    \begin{lem} For all $p$, there exists a linear $h: \Omega^p(M \times I) \to
    \Omega^{p-1}(M)$ such that $h \circ d + d \circ h = i_1^* - i_0^*$.
\end{lem}

    \begin{rmk} This is known as a \textit{chain homotopy}.\footnote{This is
    the wrong notion of homotopy equivalence for chain complexes. The right
    notion leads to the derived category.} \end{rmk}

    \begin{proof} Define $h \omega = \int_0^1 i_{\partial_t} \omega$.
    \end{proof}

    \begin{thm} If $F,G:M \to N$ are smoothly homotopic, then $F^* = G^*$.
    \end{thm}

    \begin{proof} Let $\omega$ be a closed form. Then \[ G^*\omega - F^*\omega
    = d(h\circ H^*\omega).\] \end{proof}

    \begin{defn} Any two spaces $M,N$ are \textit{homotopy equivalent} if there
    exist maps $F:M \to N, G:N \to M$ such that $F \circ G$ and $G \circ F$ are
homotopic to the identity.  \end{defn}

    \begin{rmk} Homotopy equivalence is the weakest equivalence relation on
    manifolds.\footnote{For more general spaces, there is weak homotopy
equivalence.} \end{rmk}

    \begin{cor} If $M,N$ are homotopy equivalent, then they have the same de
    Rham cohomology.  \end{cor}

    \begin{proof} Use the Whitney approximation theorem to construct a smooth
    homotopy from a continuous homotopy. Then use Theorem 269.  \end{proof}

    \section{Lecture 21 (Nov 13)}% \label{sec:lecture_21_nov_13_}

    We will compute some examples with de Rham cohomology. We note that:
    \begin{enumerate} \item If $M = M_1 \sqcup M_2$, then $H^*(M) = H^*(M_1)
        \times H^*(M_2)$.  \item If $M$ is connected, then $H^0(M) = \R$.
    \item If $M$ is contractible, then $H^*(M) = H^0(M) = \R$.  \item If $M$ is
simply connected, then $H^1(M) = 0$.  \end{enumerate}

    \begin{thm} Let $G$ be a finite group acting freely on $M$. Recall that the
    quotient $\pi$ to $M/G$ is smooth. Then $\pi^*$ has image the
left-invariant cohomology classes and is injective.  \end{thm}

    \begin{proof} It is easy to see that $g^*\pi^* = \pi^*$, so $\pi^*$ maps to
        the invariant elements. Now let $\omega$ be a closed form such that
        $\pi^*\omega = 0$ in cohomology. Therefore we have $\pi^*\omega =
        d\eta$. Then we consider \[\widetilde{\eta} = \frac{1}{\abs{G}} \sum_{g
        \in G}g^*\eta. \] This will be completed later.  \end{proof}

    \begin{cor} If $\pi_i(M)$ is finite, then $H^1(M) = 0$.  \end{cor}

    \subsection{Some Homological Algebra}% \label{sub:some_homological_algebra}
    
    Here we will introduce the Mayer-Vietoris sequence.

    \begin{defn} A \textit{(co)chain map} $F:A^* \to B^*$ is a graded map $A^k
    \to B^k$ that commutes with $d$.  \end{defn}

    It is easy to check that this induces a map on cohomology.

    \begin{defn} A \textit{short exact sequence} is a sequence $0 \to A^* \to
    B^* \to C^* \to 0$ that is exact at every level.  \end{defn}

    \begin{lem}[Snake Lemma] A short exact sequence of (co)chain complex
    induces a long exact sequence of cohomology.  \end{lem}

    Proof of this is given by a standard diagram chasing argument.

    \begin{rmk} The connecting morphism $\delta: H^p(C) \to H^{p-1}(A)$ is
    useful to know.  \end{rmk}

    \begin{thm}[Mayer-Vietoris Sequence] Consider a smooth manifold $M = U \cup
        V$. then the following is a short exact sequence: \[ 0 \to \Omega^*(M)
        \to \Omega^*(U) \oplus \Omega^*(V) \to \Omega^*(U \cap V) \to 0.\]
    \end{thm}

    \begin{proof} At $M$, we see that $U,V$ form an open cover of $M$. At the
        middle term, all the terms are given by restriction, so the kernel
        contains the image. Then, we note that if $(i^* - j^*)(\eta, \eta') =
        0$, then $\eta, \eta'$ agree on $U \cap V$. Then because $\Omega^p$ is
        a sheaf, we have the desired result.

        Finally, at $U \cap V$, let $\omega \in \Omega^p(U \cap V)$. Let
    $\varphi, \psi$ be a partition of unity subordinate to $U,V$. Then define
$\eta = \psi \omega$ on $U \cap V$ and $0$ elsewhere. In addition, let $\eta' =
-\varphi\omega$ on $U \cap V$ and $0$ elsewhere. Then we have $(i^*-j^*)(\eta,
\eta') = \omega$.  \end{proof}

    \begin{cor} Given $\omega \in Z^p(U \cap V)$, let $\eta = \psi\omega$ as in
    the above proof. Extend by $0$ elsewhere. Then $\delta(\omega) = d\eta$.
\end{cor}

    \begin{thm} $H^p(S^n) = \R$ if $p = 0,n$ and  $0$ otherwise.  \end{thm}

    \begin{proof} Note we have $0 \to H^0(S^n) \to H^0(\R^n) \oplus H^0(\R^n)
        \to H^0(S^{n-1}) \to H^1(S^n) \to H^1(\R^n) \oplus H^1(\R^n) = 0$. Thus
        $H^1(S^n) = 0$.  In addition, because $\R^n$ is contractible, we have
        $H^p(S^{n-1}) = H^{p-1}(S^n)$ for $0 < p < n$. Finally, we have
        $H^n(S^n) = \cdots = H^1(S^1) = \R$.  \end{proof}

    \section{Lecture 22 (Nov 18)}% \label{sec:lecture_22_nov_18_}
    
    Mike gave the proof of Theorem 273. The idea is that $\pi^*$ is injective
    and surjective at the level of differential forms by definition. Then we
    compute.

    \begin{cor} $H^p_{dR} = \begin{cases} 0 & 0 < p < n \\ \R & p = 0 \\ 0 &
    p=n=2m \\ \R & p=n=2m+1 \end{cases}$.  \end{cor}

    \begin{proof} The antipodal map on $S^n$ is orientation preserving if and
    only if $n$ is odd.  \end{proof}

    More generally, we have: \begin{thm} Let $M$ be a compact connected smooth
    manifold. Then if $M$ is not orientable, the top cohomology class vanishes,
and if $M$ is orientable, $H^n(M) \simeq \R$.  \end{thm}

    \begin{proof} Let $[\Omega_0]$ be an orientation of $M$. Let $b \coloneqq
        \int_M \Omega_0 \neq 0$. Thus we have a surjective map to $\R$ given by
        integration. Now suppose $\int_M \omega = 0$. Choose a finite cover of
        $M$ by $U_i \simeq \R^n$ and order the $U_i$ so that $U_k \cap U_{k-1}
        \neq 0$. Now we proceed by induction.

        For the base case, we note that $\int_{\R^n} \omega = 0$, so $\omega =
    d\eta$. Then for the inductive step, choose $\eta$ with support contained
in $M_k \cap U_{k+1}$ with $\int_{M_{k+1}} \eta = 1$. Then $\varphi, \psi$ is a
partition of unity subordinate to $M_k, U_{k+1}$. Let $c = \int_{M_{k+1}}
\varphi_{\omega}$. Then we have $\int_{M_{k+1}} \varphi \omega - c \eta = 0$.
By induction, there exists $\alpha$ such that $d\alpha = \varphi\omega -
c\eta$. Similarly, we can find $\beta$ such that $d\beta = \psi \omega$, and we
see that $d(\alpha+\beta) = \omega$.  \end{proof}

    Recall the notion of the connected sum of topological spaces. For smooth
    manifolds, the connected sum is a smooth manifold, and $H^p(M_1 \# M_2) =
    H^p(M_1) \times H^p(M_2)$.

    \begin{prop} Let $M$ be a compact connected orientable manifold with
        dimension at least $3$. Fix $q \in M$ and $0 \leq k < n$. Then the
        inclusion map $M \setminus \{q\} \to M$ induces an isomorphism $H^k(M)
        \to H^k(M \setminus \{q\})$. If $k=n$, the induced map is $0$.
    \end{prop}

    \begin{proof} Let $M_1$ be an open ball and $M_2 = M \setminus B$. Then for
        $0 < p < n$, because balls are contractible, we have the desired
        result. For $k=0$, $M$ without a point is connected. Finally, the case
        of $k=n$ is a simple application of Mayer-Vietoris.  \end{proof}

    \begin{exm} Consider the Klein bottle. We can cut $K$ into $U,V$, which are
        most homotopy equivalent to $S^1$. Then their intersection is a
        disjoint union of two $S^1$. By Mayer-Vietoris, we can see that the
        cohomology of the Klein bottle is $\R$ if $p = 0,1$ and $0$ if $p=2$.
    \end{exm}

    \section{Lecture 23 (Nov 20)}% \label{sec:lecture_23}
    
    Because algebraic topology is not a prerequisite for this course, today we
    will discuss the de Rham theorem. 

    \begin{thm} The de Rham cohomology is equivalent to ordinary cohomology.
    Equivalently, it satisfies the Eilenberg-Steenrod axioms.  \end{thm}

    \begin{defn} A \textit{$p$-simplex} is the convex hull $\gen{v_1, \ldots,
    v_p}$, where the $v_i$ are in general position. The \textit{standard
    $p$-simplex} is given by the basis vectors in $\R^p$.  \end{defn}

    \begin{defn} A \textit{singular $p$-simplex} for a topological space $M$ is
    a continuous map $\sigma: \Delta_o \to M$.  \end{defn}

    \begin{defn} The \textit{$p$-singular chains} on $M$ of dimension $p$ are
    $C_p(M) = \R \gen{p\text{-singular chains on }M}$.  \end{defn}

    For $0 \leq i \leq p$, define the $i$th \textit{face map} $F_{i,p}$ by
    \[(e_0, \ldots, e_{p-1}) \mapsto (e_0, e_1-e_0, \ldots, \widehat{e_i},
    \ldots, e_{p-1}-e_0).\] Now for a singular chain $\sigma:\Delta_p \to M$,
    we define the boundary \[\partial \sigma = \sum_{i=0}^p (-1)^i \sigma \circ
    F_{i,p}.\]

    By linearity, we can extend to a map $\partial: C_p(M) \to C_{p-1}(M)$.
    This defines a chain complex $C_*(M)$, so we can define the
    \textit{singular homology} $H_*(M)$. Then singular cohomology is defined by
    taking $\Hom(C_p, \R)$ and then taking the cohomology of that.

    \begin{prop} $H^p(M, \R) = H_p(M, \R)^*$.  \end{prop}

    \begin{prop} Singular homology and cohomology satisfy the same properties
    that de Rham cohomology satisfies.  \end{prop}

    \begin{thm} Singular homology and cohomology satisfy a Mayer-Vietoris
    sequence analogous to that of de Rham cohomology. The connecting
homomorphisms satisfy $\partial^* \gamma = \gamma \circ \partial_*$.  \end{thm}

    It turns out that we can take some triangulation or cell decomposition of
    our space to compute (co)homology. Now we let $M$ be a smooth manifold.
    Then we can define the notion of a \textit{smooth $p$-simplex}. Thus we can
    define the smooth chain complex $C_p^{\infty}(M)$. Because the face maps
    are smooth, we can define the smooth homology and cohomology $H_*^{\infty},
    H_{\infty}^*$.

    Note that we have a natural chain inclusion $C_p^{\infty} \to C_p(M)$.
    \begin{thm} $i_*: H_p^{\infty}(M) \to H_p(M)$ is an isomorphism.  \end{thm}
    Proof of this uses the Whitney approximation theorem to construct a
    homotopy inverse to $i$.
    
    The next part of relating analysis to topology is the de Rham theorem:
    \begin{thm}[de Rham] $H^*(M, \R) \simeq H^*_{dR}(M)$.  \end{thm}

    First, we define an integral over a smooth chain $c$ by \[ \int_c \omega =
        \sum_{i=1}^k c_i \int_{\Delta_p} \sigma_i^* \omega.\] Then we have
        \begin{thm}[Stokes] $\int_{ \partial c } \omega = \int_c d\omega$.
        \end{thm}

    \begin{proof} We simply need to consider $c = \sigma$. We can check that
    $F_{i,p}$ is orientation preserving iff $i$ is even. Then the result
follows from the ordinary Stokes theorem.  \end{proof}

    Now integration defines a homomorphism $\mc{J}: H^*_{dR}(M) \to H^p(M,
    \R)$. To see that this is well-defined, we have for homologous chains $c,
    c'$, we have \[ \int_c \omega - \int_{c'} \omega = \int_{\partial b} \omega
    = \int_b d\omega = 0.\] Then if $\omega = d\eta$, we have \[\int_c \omega =
\int_c d\eta = \int_{\partial c} \eta = 0.\]

    \begin{lem} $\mc{J}$ is natural: it commutes with pullbacks and with the
    Mayer-Vietoris connecting homomorphisms.  \end{lem}

    \begin{proof} We will prove the Mayer-Vietoris part. We will show that for
        all $[\omega] \in H^{p-1}_{dR}(U \cap V)$ and $[e] \in H_p(M)$ that
        $\mc{J}(\delta([\omega]))[e] = ( \delta^*\mc{J}[\omega] )[e]$.

        Choose $\sigma \in \Omega^p(M)$ and $c \in C_{p-1}(M)$ such that
        $[\sigma] = \delta^*[\omega]$ and $[c] = \partial_*[e]$. Then we have
        $\sigma = d\eta$ and $c = \partial d$, where $\omega = \eta - \eta',
        [d+d'] = [e]$. Therefore \begin{align*} \mf{J}[\omega](\partial_* [e])
        &= \int_c \omega = \int_{\partial d} \omega \\ &= \int_{\partial d}
            \eta - \int_{\partial d} \eta' \\ &= \int_d d\eta + \int_{d'}
            d\eta' \\ &= \int_d \sigma + \int_{d'} \sigma \\ &= \int_e \sigma
                   \\ &= \mc{J}(\delta[\omega])[e]. \qedhere \end{align*}
               \end{proof}

    \begin{proof}[Sketch of de Rham's Theorem] If $M$ is a convex open set of
        $\R^n$, then $H^p_{dR}(M) = 0$ if $p \neq 0$ and $\R$ is $p=0$. Because
        $M$ is contractible, the same applies to $H^p(M, \R)$. Then it is easy
        to check that $\mc{J}[1][\sigma] = 1$, so we have the base case.

        Then suppose $M = U_1 \cup \cdots \cup U_k$, where each $U_i$ is
        diffeomorphic to a contractible open set of $\R^n$. Assume $J$ is an
        isomorphism for unions up to $U_{k-1}$. Then if $U = U_1 \cup \cdots
        \cup U_{k-1}$ and $V = U_k$, we have the following diagram:

        \begin{center} \begin{tikzcd} H^{p-1}_{dR}(U) \oplus H_{dR}^{p-1}(V)
            \arrow{r} \arrow{d} & H^{p-1}_{dR}(U \cap V) \arrow{d} \arrow{r} &
            H^p_({dR}M) \arrow{r} \arrow{d} &H^{p-1}_{dR}(U) \oplus
            H_{dR}^{p-1}(V) \arrow{r} \arrow{d} & H^p_{dR}(U \cap V) \arrow{d}
            \\ H^{p-1}(U) \oplus H^{p-1}(V) \arrow{r} & H^{p-1}(U \cap V)
        \arrow{r} & H^pM) \arrow{r} &H^{p-1}(U) \oplus H^{p-1}(V) \arrow{r} &
    H^p(U \cap V) \\ \end{tikzcd} \end{center}

        Now we use the inductive hypothesis and the five lemma. Finally, we go
    from a finite union to an arbitrary manifold using point-set topology and a
second countable cover.  \end{proof}

    \section{Lecture 24 (Dec 4)}% \label{sec:lecture_24_dec_4_}
    
    Today we will discuss integral curves and flows. 

    \begin{defn} For $X \in H^0(M, TM)$, an \textit{integral curve} of $X$ is a
    map $\gamma: [a,b] \to M$ such that for all $t$, $\gamma'(t) =
X(\gamma(t))$.  \end{defn}

    \begin{defn} A \textit{time-dependent vector field} $X$ on $M$ is a smooth
    map $X: I \times M \to TM$ that is a section for each $T$.  \end{defn}

    \begin{defn} An \textit{integral curve} of a time-dependent vector field
    $X$ is a $\gamma:[a,b] \to M$ such that for all $t$, $\gamma'(t) =
X(t,\gamma(t))$.  \end{defn}

    \begin{exm} Let $M = \R^4$ with the standard symplectic form. Then fix
        constants $m,g > 0$. Let $H: M \to \R$ (the total energy) be given by
        \[(x_1, x_2, y_1,y_2) \mapsto mgx_2 + \frac{1}{2m} (y_1^2 + y_2^2).\]
        Define $X = \omega(X,-) = dH$. Then the claim\footnote{For proof, take
        Inanc's class.} is that an integral curve of $X$ lies on $H^{-1}(c)$
        for some $c \in \R$ (conservation of energy).  \end{exm}

    \subsection{Local Representation}% \label{sub:local_representation}

    Suppose $\gamma: (a,b) \to M$ is an integral curve of $X$ and $\gamma(t)
    \in U$ for some chart $(U, \varphi)$. Then locally, note that \[
    \gamma'(t)(x^i) = \frac{d}{dt} (x^i \circ \gamma) = \frac{d}{dt}
\gamma_i(t) = \gamma_i'(t). \] In addition, we have \begin{align*}
X(\gamma(t))(x^i) &= \sum_j X_j(\gamma(t)) \frac{\partial}{\partial x^j} (x^i)
               \\ &= X_i(\gamma(t)) = (X_i \circ \varphi^{-1})(\varphi \circ
\gamma(t)) \\ &= \widetilde{X_i}(\gamma_1(t), \ldots, \gamma_n(t)).
\end{align*} Therefore, the integral curve equation becomes \begin{equation}
\begin{cases} \gamma_1'(t) = \widetilde{X_1}(\gamma_1(t), \ldots, \gamma_n(t))
    \\ \ddots \\ \gamma_n'(t) = \widetilde{X_1}(\gamma_1(t), \ldots,
    \gamma_n(t)) \\ \end{cases}.  \end{equation}

    Next we need a large existence and uniqueness theorem for ODEs: \begin{thm}
        Fix $U \subset \R^n$ with $X = (X_1, \ldots, X_n):U \to \R^n$ smooth.
        Then for all $t_0 \in \R$ and $x_0 = (x^1, \ldots, x^n) \in \R^n$,
        consider the first order ODE (1) with initial consition $\gamma_i(t_0)
        = x^i$. Then: \begin{enumerate} \item There exists $t_0 \in (a,b)
            \subset \R$ and $x_0 \in U_0 \subset U$ such that for all $x \in
            U_0$, there exists a solution $\gamma_x$ of (1) with the initial
            condition.  \item Any two differentiable solutions agree on their
            common domain.  \item Define $\Theta: (a,b) \times U_0 \to U$ given
    by $(t,x) \mapsto \gamma_x(t)$. Then $\Theta$ is smooth.  \end{enumerate}
\end{thm}
    
    \begin{rmk} Integral curves are \textit{translation invariant}, where the
    translation is in the source interval.  \end{rmk}

    \begin{lem} For all $X \in H^0(M, TM)$, for all $p \in M$, there exists a
    unique maximal integral curve of $X$ at $p$, $\gamma: J_p \to M$ where $0
\in J_p, \gamma(0) = p$.  \end{lem}

    \begin{proof} For uniqueness, we use uniqueness from Theorem 301. For
    existence, we know that at least one $(\gamma, J)$ exists. Then we order
all $\gamma, J$ by inclusion. Then we use Zorn's lemma.  \end{proof}

    \begin{defn} $X$ is complete if for all $p \in M$, $J_p = \R$.  \end{defn}

    \begin{exm} Let $M = ( \R^2 \setminus \{(\pm 1, 0) \} ) / (5\Z)^2$ and $X =
    \partial_x$. $X$ is not complete because $J_{(0,0)} = (-1,1)$.  \end{exm}

    \begin{lem}[Escape Lemma] If $X \in H^0(M, TM)$ and $\gamma:J \to M$ an
    integral curve and $J \neq \R$, then $\gamma(J)$ does not lie in a compact
subset of $M$.  \end{lem}

    \begin{prop} Any compactly supported vector field $X$ is complete.
    \end{prop}

    \begin{proof} If $X(\gamma(t)) \neq 0$, then use the escape lemma.
    Otherwise, if there exists $t_0 \in J_p$ such that $X(\gamma(t_0)) = 0$.
Then we define the constant curve at $\gamma(t_0)$.  \end{proof}

    \begin{defn} A \textit{global flow} $\Theta:\R \times M \to M$ is a smooth
        map with the following properties: \begin{enumerate} \item For all $s
            \in \R$, the map $\Theta_s:M \to M$ is a diffeomorphism.  \item
            $\Theta_0 = \operatorname{Id}_M$.  \item For all $s, t \in \R$,
    $\Theta_s \circ \Theta_t = \Theta_{s+t}$.  \end{enumerate} \end{defn}

    \begin{thm} There is a bijection between smooth complete time independent
    vector fields and global flows on $M$.  \end{thm}

    \begin{exm} Let $M$ be a smooth compact Riemannian manifold and let $f \in
        C^{\infty}(M)$. fix $a,b \in \R$ such that there no critical values of
        $f$ in $[a,b]$. Let $M_r = f^{-1}(r)$ and $M_{[r_1, r_2]}$ similarly.
        Then $M_{[a,b]} \simeq M_a \times [0, b-1]$. In particular, $M_a \simeq
        M_b$. Informally, this means that the topology only changes when
        passing through critical values.  \end{exm}

    \begin{proof} Let $\gen{-,-}$ be a metric. Define $\operatorname{grad} f$
        by $\gen{\operatorname{grad f}, y} = df(Y)$. Now let $X =
        \frac{\operatorname{grad} f}{\gen{\operatorname{grad} f,
        \operatorname{grad} f}}$. Let $M'$ be a neighborhood of $M_{[a,b]}$
        such that $\operatorname{grad} f$ is nonzero on $M'$. Then fix $p \in
        M'$ and let $\gamma_p: J_p \to M$ be a maximal integral curve. Then we
        use the escape lemma to show that if $J_p \ni t \leq b-a$, then $b-a
        \in J_p$. Then we define a flow $\Theta:M_a \times [0, b-a] \to
        M_{a,b}$. This is well-defined. Then with work we show this is a
        diffeomorphism.  \end{proof}

    \section{Lecture 25 (Dec 09)}% \label{sec:lecture_25_dec_09_}
    
    Today we will discuss Lie derivatives. First, we mention some properties of
    flows. Recall that the ODE theorem is a \textit{local} existence,
    uniqueness, and smoothness theorem. Thus for any $X \in H^0(M, TM)$, there
    exists a unique local flow $\Theta: D \subset R \times M \to M$. We say
    that $X$ \textit{generates} $\Theta$, or is the \textit{infinitesimal
    generator} of $\Theta$. Our goal will be to measure how a vector field $Y$
    and $k$-form $\omega$ varies under flow associated to a vector field $X$.

    Let $\Theta_t$ be the local flow generated by $X$ defined on $D = (-\ep,
    \ep) \times U$ for some open $U \ni p$. We will define \begin{align*} (L_X
        \omega)_p &= \lim_{t \to 0} \frac{\Theta_t^*(\omega(\Theta_t(p))) -
        \omega(p)}{t} \\ (L_X Y)_p &= \lim_{t \to 0}
        \frac{(\Theta_{-t})_*(Y(\Theta_t(p))) - Y(p)}{t} \end{align*}
    
    \begin{lem} The assignments $p \mapsto (L_X \omega)_p$ and $p \mapsto (L_X
    Y)_p$ define smooth forms and vector fields.  \end{lem}

    \begin{prop} Let $X, Y$ be vector fields, $f$ a function, and $\omega,
        \eta$ are forms of degree $k, \ell$.  \begin{enumerate} \item $L_X f =
            X(f)$; \item $L_X(\omega \otimes \eta) = (L_X \omega) \otimes \eta
            + \omega \otimes L_X \eta$; \item $L_X (\omega \wedge \eta) = (L_X
            \omega) \wedge \eta + \omega \wedge L_X \eta$; \item $L_X(d\omega)
            = d(L_X)\omega$; \item $L_X(\alpha(Y)) = (L_X \alpha)(Y) +
            \alpha(L_X Y)$ for a $1$-form $\alpha$; \item $L_X Y = [X,Y]$;
        \item $L_X(i_Y \omega) = i_{L_X Y} \omega + i_Y(L_X \omega)$.  \item
            $L_X(\omega(Y_1, \ldots, Y_k)) = (L_X \omega)(Y_1, \ldots, Y_k) +
            \omega(L_X Y, \ldots, Y_k) + \cdots + \omega(Y_1, \ldots, L_X
            Y_k)$.  \end{enumerate} \end{prop}

    \begin{thm}[Cartan's Formula] With the same assumptions as in proposition
        312, \[L_X \omega = d(i_X \omega) + i_X(d\omega).\] \end{thm}

    \begin{cor} With the same notation as Proposition 312,
        \begin{enumerate}[label=(\alph*)] \item $L_X Y = - L_Y X$; \item
            $L_X[Y,Z] = [L_X Y, Z] + [Y, L_X Z]$; \item $L_{X,Y} Z = L_X L_Y Z
            - L_Y L_X Z$; \item $L_X(fY) = (L_X f)Y + f(L_X Y)$; \item If $X',
    Y'$ are $F$-related to $X,Y$, then $F_*(L_X Y) = L_{X'}Y'$.
    \end{enumerate} \end{cor}

    \begin{proof}[Proof of Proposition 312 (1)] Note that \begin{align*} (L_X
        f)_p &= \lim_{t \to 0} \frac{(\Theta_t^* f)(\Theta_t(p)) - f(p)}{t} \\
             &= \lim_{t \to 0} \frac{f(\Theta_t(p)) - f(p)}{t} \\ &=
             \gamma_p'(0) f = X_p(f) = (Xf)_p. \qedhere \end{align*}
         \end{proof}

    \begin{proof}[Proof of Proposition 312 (4) for functions] Let $\partial_t
        |_0$ denote the limit of the difference quotient. Then \begin{align*}
            L_X(df) &= \partial_t |_0 (\Theta_t^* df) \\ &= \partial_t |_0
            d(\Theta_t^* f) \\ &= \partial_t |_0 d(f(\Theta(t,p))) \\ &=
            \partial_t |_0 \sum_i \partial_i f(\Theta(t,p)) dx^i \\ &= \sum_i
            \partial_i \partial_t |_0 f(\Theta(t,p)) dx^i \\ &= \sum_i
            \partial_i(L_X f) dx^i \\ &= d(L_X f). \qedhere \end{align*}
        \end{proof}

    \begin{proof}[Proof of Proposition 312 (6)] Fix a smooth function $f$. Then
        first we have $L_X(df(Y)) = L_X(Yf) = X(Y(f))$. In addition, we have
        $L_X(df)(Y) = d(L_X f)(Y) = Y(L_X f) = Y(X(f))$. Thus $L_X(df(Y)) =
        (L_X(df))(Y) + df(L_X Y)$, so \[ df(L_X Y) = L_X(df(Y)) - L_X(df)(Y) =
        X(Y(F)) - Y(X(F)) = d(df(X,Y)) + df([X,Y]) = df[X,Y].\] \end{proof}

    \begin{proof}[Proof of Cartan's Formu4la] We induct. On functions, we have
        $i_X(df) + d(i_X f) = i_X(df) = df(X) = X(f) = L_X f$. For $1$-forms
        $\alpha = u dv$, we have \begin{align*} i_X(d(udv)) + d(i_X(udv)) &=
            i_X(du \wedge dv) + d(u(Xv)) \\ &= (i_X du) \wedge dv - du \wedge
        i_X dv + ud(Xv) + (Xv) du \\ &= X(u) dv - Xv du + ud(Xv) + (Xv)du \\ &=
    X(u) dv + ud(Xv).  \end{align*}

        On the other hand, \[ L_X(u dv) = L_Xu dv + u(L_X dv) = (Xu)dv + u
        d(Xv). \] For the inductive step, we write $\omega = \sum_{I \in
        \binom{[n]}{k}} \omega_I dx^{i_1} \wedge \cdots \wedge dx^{i_k}$, so it
        suffices to check Cartan holds for $\alpha \wedge \beta$. The rest is
        left to Lee.  \end{proof}

    \begin{defn} Vector fields $X,Y$ \textit{commute} if $[X,Y] = 0$.
    \end{defn}

    \begin{defn} The vector field $W$ is \textit{invariant under flow} $\Theta$
    if $(\Theta_t)_* W_p = W_{\Theta_t(p)}$.  \end{defn}

    \begin{prop} Let $X,Y$ be vector fields that generate flows $\Theta, \Phi$.
        Then the following are equivalent: \begin{enumerate} \item $X,Y$
            commute; \item $L_X Y = 0$; \item $L_Y X = 0$; \item $X$ is
            invariant under $\Phi$ and $Y$ is invariant under $\Theta$.  \item
    $\Phi_t \circ \Theta_s = \Theta_s \circ \Phi_t$.  \end{enumerate}
\end{prop}

    \begin{exm} If $M = \R^2$, we can check that $\partial_x, \partial_y$
    commute, but $x \partial_y + y \partial_x, x\partial_x - y \partial_y$ do
not commute.  \end{exm}

    \begin{thm} Let $M$ be a smooth $n$-manifold and linearly independt vector
        fields $X_1, \ldots, X_{k}$ for all $p \in U \subset M$ open. Then
        TFAE: \begin{enumerate} \item $[X_i, X_j] = 0$ for all $i,j$; \item
            There exist smooth coordinates such that $X_i = \partial_i$.
    \end{enumerate} \end{thm}

    \begin{exm} Let $(M, \omega)$ with vector field $X$. Let $\Theta$ be the
    flow generated by $X$. Then $\Theta_t$ is a symplectomorphism iff $i_X
\omega$ is closed.\footnote{This was accompanied by a plug for Math 705.}
\end{exm}

    \begin{proof} $\Theta_t^* \omega = \omega$ for all $t$ if and only if $0 =
        \frac{d}{dt} \Theta_t^*\omega = \Theta^*(L_X \omega)$ if and only if
        $L_X \omega = 0$. By Cartan, this is iff $d i_X \omega + i_X d\omega =
        0$, which happens iff $d(i_X \omega) = 0$ because $\omega$ is closed.
    \end{proof}
    

\end{document}
