\documentclass[twoside, 10pt]{article}
\usepackage{../../../notes}
%\geometry{margin=2cm}
\newcommand{\F}{\mathbb{F}}
\newcommand{\shaf}{\textbf{Shaf}}
\renewcommand{\d}{\ \mathrm{d}}
\mdfdefinestyle{default}{%
    linecolor=black,
    outerlinewidth=0.4pt,
    roundcorner=0pt,
    innertopmargin=\baselineskip,
    innerbottommargin=\baselineskip,
    innerrightmargin=\baselineskip,
    innerleftmargin=\baselineskip,
    backgroundcolor=white}


    \definecolor{darkblue}{RGB}{0,0,128}
    \definecolor{darkred}{RGB}{128,0,0}
    \definecolor{darkyellow}{RGB}{96,96,0}
    \definecolor{darkgreen}{RGB}{0,128,0}
    \definecolor{darkdarkred}{RGB}{64,0,0}
    
    

\lstset
{
    basicstyle=\ttfamily\scriptsize, 
    breaklines=true,
    postbreak=\mbox{\textcolor{darkdarkred}{$\hookrightarrow$}\space},
    showstringspaces=false
    keywordstyle = [1]\bfseries\color{darkred},
    keywordstyle = [2]\itshape\color{darkgreen},
    keywordstyle = [3]\sffamily\color{darkblue},
    keywordstyle = [4]\color{darkyellow},
}

\title{Math 797W Lecture Notes}
\author{Taught by Jenia Tevelev; Notes by Patrick Lei}
\affil{University of Massachusetts, Amherst}
\date{Spring 2019}

\fancypagestyle{firstpage}
{
   \fancyhf{}
   \fancyfoot[R]{\itshape Page \thepage\ of \pageref{LastPage}}
   \renewcommand{\headrulewidth}{0pt}
}


\fancypagestyle{pages}
{
    \fancyhf{}
    \fancyhead[LO]{\scshape Math 797W}
    \fancyhead[RO]{\scshape Lecture Notes}
    \fancyhead[CO]{\scshape Algebraic Geometry}
    \fancyhead[CE]{\scshape University of Massachusetts, Amherst}
    \fancyhead[LE]{\scshape Patrick Lei}
    \fancyhead[RE]{\scshape Spring 2019}
    \fancyfoot[RO,LE]{\itshape Page \thepage \space of \pageref{LastPage}}
    \renewcommand{\headrulewidth}{0.1pt}
}

\pagestyle{pages}

\begin{document}

    \maketitle\thispagestyle{firstpage}
    
    \begin{abstract} This will be the first course in algebraic geometry - the
        study of geometric spaces locally defined by polynomial equations. It
        is a central subject in mathematics with strong connections to
        differential geometry, number theory, and representation theory. We
        will pursue an algebraic approach to the subject, when local data is
        studied via the commutative algebra of quotients of polynomial rings in
        several variables. The emphasis will be on basic constructions and
        examples.  Topics will include projective varieties, resolution of
        singularities, divisors and differential forms. Examples will include
        algebraic curves of low genus and surfaces in projective 3-space. In
        addition to theoretical approach, we will also learn how to use
        computer algebra software, specifically the Macaulay 2 package, to help
        with basic calculations in commutative algebra and algebraic geometry.

        Forms of evaluation: biweekly homeworks (25\%), take-home midterms
    (50\%) and computer algebra project (25\%).  \end{abstract}

    \tableofcontents

    \section{Organizational/Philosophical}

    Warning: All jokes and conversations are reproduced as best as I can
    remember and my transcription is not necessarily faithful. In addition,
    footnotes and some definitions are based on my understanding of algebraic
    geometry and related topics. Also, I am a terrible person and do not
    include diagrams. Form your own geometric intuition.

    The syllabus is on Moodle. He wanted to project the syllabus on the board,
    but was unable to get the projector to work. Jenia promised to email
    everyone the syllabus today.

    Jenia considers Shafarevich (hereto referred to as \textbf{Shaf}) to be the
    best way to learn algebraic geometry.\footnote{He said something about
    starting with schemes (like Vakil's notes) being a bad idea.} It combines
    the Russian algebraic geometry approach (by Shafarevich, others, and Jenia
    himself) with the translation by Miles Reid (British, more colloquial
    approach). He has used the book to teach both undergrads and graduate
    students. Knowing projective algebraic varieties is very helpful in the
    future whether you use homological/categorical, complex (holomorphic), or
    scheme theory methods. The plan is to cover all of Shafarevich in this
    semester.
    
    There are people who attempt to read Mumford instead (it's fantastic), but
    the book is much more difficult and he attempts to develop both varieties
    and schemes at the same time.

    Jenia notes that people who do algebraic geometry use computers, and he is
    going to be teaching us some Macaulay 2 (especially because computer
    packages can do commmutative algebra). Note that this is not a course in
    computational algebraic geometry (which Jenia, Paul, and Eyal don't know
    very much about). There is David Cox at Amherst, who is retiring in June,
    who wrote a book titled \emph{Ideals, Varieties, and Algorithms}. There
    will be a computer algebra project.

    There will be two kinds of homework problems. Some are normal homework
    assignments while others will be posted on Moodle. Two homeworks will be
    designated as take-home midterms.

    Jenia struggled to turn off the projector but managed to do so.

    \section{Affine Plane Curves}

    \subsection{Lecture 1 (Jan 22)} Historically, algebraic geometry came from
    two directions: projective geometry and abelian integrals. These are the
    two big sources of algebraic geometry, and much early progress was about
    the two subjects.

    \subsubsection{Enumerative Geometry}

    \begin{thm}[Butterfly Theorem] Consider an ellipse with a chord $AC$ and
        let $B$ be the midpoint of $AC$. Draw two chords through $B$ and form a
        ``butterfly.'' We have two new points $P,Q$. Then $PB=BQ$.

        \begin{proof} Observe we have three plane curves passing through four
            points: A conic $e$ given by $f$ and two unions of two lines
            $c_1,c_2$ given by $f_1,f_2$. We have a linear system of conics
            passing through four points. 

            The space of all conics forms an $\R^6$. Every point imposes a
            linear condition on the coefficients, so conics passing through out
            four points, so the space we care about is at least
            two-dimensional. We will later prove that this is exactly two.

            We see that $f_2 = \alpha f + \beta f_1$. Now we restrict our
            polynomials to the line passing through $A,C$ and assume it is the
            $x-axis$ and that $B$ is the origin. We see that $f \mid_{y=0} =
            (x-c)(x+c)$, $f_1 \mid_{y=0} = x^2$, and $f_2 \mid_{y=0} =
        (x-p)(x-q)$. Then we must have that $p+q=0$.  \end{proof} \end{thm}

    Jenia learned this theorem in high school.\footnote{Unlike Americans, he
    seems to have gotten an actual Euclidean geometry education.} He told a
    story about Sheldon Katz and one of his interviews.

    \begin{lem} The dimension of the vector space is exactly $2$.

        \begin{proof} Assume $\dim(V) \geq 3$. Let $X$ be another point of the
            ellipse and consider the subspace $V_0 \subset V$ of conics passing
            through $X$. Then we see that $\dim(V_0) \geq 2$. Then we have two
            linearly independent conics passing $C,D$ passing through five
            points. This contradicts Bezout's Theorem which states that if $C$
        has degree $n$ and $D$ has degree $n$, then $\abs{C \cap D} \leq nm$
    unless the curves have a shared component.  \end{proof} \end{lem}

    Jenia says this is nice argument because it connects to enumerative
    geometry, the connection between algebraic and geometric objects, and
    curves on surfaces, which we will talk about later when we talk about
    divisors.

    Another famous example of a similar argument is Pascal's Theorem:

    \begin{thm}[Pascal] Given a conic with an inscribed hexagon, then opposite
    sides intersect at three collinear points.  \end{thm}

    Jenia attempted to draw the diagram on the board, but diagrams are not to
    scale. The argument is very similar to the butterfly proof and can be found
    in \shaf.

    \subsubsection{Basic Notions}

    We will give some definitions to formalize affine plane curves.
    
    \begin{defn}[Affine Plane Curve] Let $k$ be algebraically closed and denote
        by $\A^2$ the affine plane.  Let $f \in k[x,y]$ be nonconstant of
        degree $d>0$. Then the vanishing locus $C$ of $f$ is an affine plane
        curve of degree $d$. Examples are lines, conics, cubics, quartics, and
    quintics (which are only for experts).  \end{defn}

    \begin{rmk} Observe that $k[x,y]$ is a UFD (more generally, if $R$ is a
    UFD, then $R[x]$ is a UFD). This is an application of Gauss's Lemma.
\end{rmk}

    \begin{defn}[Irreducible Components] We factor $f = f_1 \cdots f_r$ into
    irreducibles. Then the vanishing locus $c_i$ of each $f_i$ is an
irreducible component of $C$.  \end{defn}
    
    Why do we consider algebraically closed fields? One reason is to have more
    points. We can see that if our field is algebraically closed, our curve has
    points.

    \begin{lem} Let $k= \overline{k}$. Then:

        \begin{enumerate} \item $C$ contains infinitely many points.  \item
            Suppose $f \in k[x,y]$ is irreducible and $g \in k[x,y]$.  Then the
            set $\{f=g=0\}$ is finite unless $f$ divides $g$. In particular, an
            irreducible polynomial is uniquely determined up to a scalar by its
    curve.  \end{enumerate}

        \begin{proof} We use the fact that $k$ is algebraically closed.
            \begin{enumerate} \item Every algebraically closed field is
                infinite, so we can write $f = \sum g_i(x)y^i$. Then take any
                $x_0$ and solve for $y$. Note each $g_i$ is a polynomial and
                has finitely many roots, so there are only finitely many $x_0$
                such that $\sum g_i(x_0)y^i$ is a nonzero constant.  \item We
                use Gauss's Lemma in a different form: Let $R$ be a UFD with
                field of fractions $K$. Then let $f \in R[y]$.  Then if $f$ is
                irreducible over $R$, it must be irreducible over $K$. We see
                that $f,g \in k[x,y] = k[x][y] \subset k(x)[y]$. By Gauss's
                Lemma, we know that $f$ does not divide $g$ in $k(x)[y]$ and
                that $f$ is irreducible. Note that $k(x)[y]$ is a PID, so $f,g$
                are coprime and using Bezout's Lemma, we have $1 = \alpha f +
                \beta g$. Let $p$ be an LCM of the denominators of the
                coefficients of $\alpha,\beta$. Then we have $p(x) = \alpha_0 f
                + \beta_0 g$. Suppose $f(x_0,y_0) = g(x_0,y_0) = 0$. Then
        $p(x_0) = 0$, which has only finitely many solutions for $x_0$. By the
same argument, there are only finitely many possible values for $y_0$.
\end{enumerate} \end{proof} \end{lem}

    During the proof of the previous lemma, Jenia said, ``Every time we run
    into a problem, we have to use a little bit more commutative algebra.
    That's how algebraic geometry works.'' In addition, Jenia's phone rang and
    made frog noises.

    We now have enough vocabulary in plane curves to state Bezout's Theorem,
    but Jenia now wants to talk about Abelian integrals.

    \subsubsection{Abelian Integrals}

    We want to be able to calculate \[\int u(x,\sqrt{1-x^2})\ \mathrm{d}x,\]
    where $u$ is a rational function in two complex variables. In Calc 2, we
    use trig substitutions, but Jenia learned something called Euler's
    substitutions. We can rewrite the integral over the unit circle, which has
    a rational parameterization \[(x,y) = \left( \frac{2t}{1+t^2},
    \frac{t^2-1}{t^2+1} \right).\] We can now express the integral in terms of
    $t$ and compute the integral using partial fractions.

    This method is harder than using trig substitution, but is more general. If
    your curve is rational, then integrating over it is amenable to this
    method.

    \begin{defn}[Rational Curve] An irreducible curve $C = (f=0)$ is rational
        if there exist $\varphi(t), \psi(t) \in k(t)$ nonconstant such that
        $f(\varphi(t),\psi(t)) = 0$.\footnote{We will see that this is
            equivalent to having function field $k(t)$, or being birational to
            the line.} \end{defn}

    \begin{cor} If $C$ is rational, then the integral over the curve $\int
    u(x,y) dx$, where $u$ is rational, can be computed using partial fractions.
\end{cor}

    \begin{rmk} Later in the semester, we will talk about differential forms,
    which are a way to talk about integration is multiple dimensions.
\end{rmk}

    \subsection{Lecture 2 (Jan 24)}% \subsubsection{Non-Mathematics}% Jenia was
    actually able to make the projector work today. He told us to install
    \texttt{Macaulay 2} which you can run with the command \texttt{M2}.  Jenia
    showed us some basic commands in \texttt{Macaulay 2}. Next, he showed us
    the Moodle and the forum with problems for us to solve.

    \subsubsection{Rational Curves Continued}% We continue with our preliminary
    discussion of plane algebraic curves to discuss some history and
    motivation. Last time we discussed Abelian integrals focusing on the
    example of the circle. From the rational parameterization $\left(
    \frac{2t}{1+t^2}, \frac{t^2-1}{t^2+1} \right)$, we can find all possible
    Pythagorean triples by choosing rational values for $t$. We saw last time
    that we can reduce an integral of the form $\int u(x,y) \d x + \int v(x,y)
    \d y$ to an integral of the form $\int \varphi(t) \d t$. 

    We introduce a new notion, the field of rational functions $k(C)$ which is
    the field of all rational functions on the curve. Observe, however, that
    rational functions are not everywhere defined. There are finitely many
    points on the curve where the denominator vanishes. We can say that $u_1 =
    u_2$ if they agree outside their bad points. Algebraically, we can say that
    $\frac{p_1}{q_1} \sim \frac{p_2}{q_2} \Leftrightarrow f \mid
    p_1q_2-p_2q_1$. 

    Alternately, we can define the coordinate algebra $k[C] = k[x,y]/(f)$ and
    then define $k(C)$ to be its field of fractions.\footnote{We will formally
    define the function field and coordinate for affine varieties later.}

    \begin{defn}[Regular Function] $u \in k(C)$ is regular at $P \in C$ if
    there exist $p,q \in k[x,y]$ such that $u=\frac{p}{q}$ such that $q(P) \neq
0$.  \end{defn}

    \begin{exm} Let $C = \Set{x^2+y^2=1}$ and $u = \frac{1-x}{y}$. Then we can
    write $u = \frac{y}{1+x}$, so it is regular at the point $(1,0)$.
\end{exm}

    We can see that $k(C)$ is finitely generated by $x,y$. We also see that
    $k(C)$ has transcendence degree\footnote{This equals its geometric
    dimension.} $1$ because $f(x,y) = 0$ is an algebraic dependence. 

    \begin{lem} Every finitely generated field of transcendence degree $1$ is
    isomorphic to $k(C)$ for some irreducible curve $C$.  \end{lem}

    \begin{rmk} Different curves can have the same function field.  \end{rmk}

    \begin{defn} We say that curves $C,D$ are birational if $k(C) \simeq
    k(D)$.\footnote{This is equivalent to the geometric dimension.} \end{defn}

    \begin{exm} Let $C$ be the circle and observe that $k(C) \simeq k(t) \simeq
    k(\A^1) = k(x)$, so $C$ is birational to $\A^1$.  \end{exm}

    Suppose $C$ is a rational curve $\Set{f=0}$. Then there exist $\varphi,
    \psi \in k(t)$ nonconstant such that $f(\varphi(t), \psi(t)) = 0$.

    \begin{lem} If $C$ is a rational curve, then $k(C) \hookrightarrow k(t)$.
        \begin{proof} Take the obvious map. Then we see that this is defined
        because if $q(\varphi(t), \psi(t)) = 0$, then $f \mid q$.  \end{proof}
    \end{lem}

    So we see that rational curves have function field a subfield of $k(t)$.
    However, we invoke a theorem from algebra.

    \begin{thm}[L\"uroth] Every subfield of $k(t)$ that is not a subfield of
    $k(t)$ is isomorphic to $k(t)$.  \end{thm}

    \begin{cor} A curve $C$ is rational if and only if $C$ is birational to a
    line.  \end{cor}

    Now we discuss nonrational curves and try to find a simple example.

    \subsubsection{Elliptic Curves}% In 1655, Wallis was interested in
    computing the arc length of an ellipse $\frac{x^2}{a^2} + \frac{y^2}{b^2} =
    1$. Then we see that $y = \frac{b}{a} \sqrt{a^2-x^2}$ and $y' =
    \frac{-bx}{a\sqrt{a^2-x^2}}$. We substitute into the formula for arc length
    $\int \sqrt{1+(y')^2} \d x$, change variables, and obtain an integral of
    the form \[ \int \frac{a-ae^2x^2}{\sqrt{(1-e^2x^2)(1-x^2)} \d x}. \] Then
    set $C = \Set{y^2 = (1-e^2x^2)(1-x^2)}$. We will see that this curve is an
    elliptic curve.

    Let's perform an invertible change of variables by moving $1$ to $\infty$.
    Then we set $t = \frac{1}{1-x}$. We now obtain the equation \[y^2 = \left(
        1-e^2 \frac{(t-1)^2}{t^2} \right) \frac{1}{t} \left( 1+ \frac{t-1}{t}
    \right).\] Now multiply by $t^4$ and obtain the equation \[ (yt^2)^2 =
t^2-e^2(t-1)^2(2t-1) \] before setting $s=yt^2$ to obtain an equation of the
form $s^2 = f_3(t)$, where $f_3(t)$ is a cubic polynomial. Then, after further
simplification,\footnote{This is only possible if the field is not of
characteristic $2,3$.} we obtain an elliptic curve $D$ given by $s^2 = w^3 +
aw+b$, which is birational to our original curve $C$.

    \begin{thm} Elliptic curves are not rational.\footnote{Jenia says Dummit
        and Foote give an algebraic proof of this fact.} Proof of this fact
        will give us motivation to understand other definitions
        \begin{proof}[Sketch] First, we projectivize our curves, so we have an
            elliptic curve $C \hookrightarrow \P^2$ and a line $L
            \hookrightarrow \P^2$. The next step is to show that the two curves
            are non-singular. Third, we show that if $C,L$ are birational and
            non-singular, then they are isomorphic. Then, we see that $C$ has a
            simple involution $\varphi$ given by $s \mapsto -s$. Then we see
            that $s$ has four fixed points. Now we use this involution to
            create an involution $f \circ \varphi \circ f^{-1}$ with $4 > 2$
            fixed points. However, every automorphism\footnote{The automorphism
                group of $\P^1$ is $\mathrm{PGL}(2,k)$} of $L$ is given by the
            images of three points,\footnote{This generalizes to higher
        dimensional projective spaces.} so this is impossible.  \end{proof}
    \end{thm}

    \subsubsection{Non-singularity of plane curves} Conider a curve $C =
    \Set{f=0}$ and let $(a,b) \in C$. Then we have a Taylor expansion $f(x,y) =
    f(a,b) + f_x(a,b)(x-a) + f_y(a,b)(y-b) + \cdots$.
   
    \begin{defn}[Non-singularity] A plane curve $C$ is nonsingular at $(a,b)$
    if at least one of $f_x(a,b), f_y(a,b) \neq 0$. In this case, $f(x,y) =
f(a,b) + f_x(a,b)(x-a) + f_y(a,b)(y-b)$ is called the tangent line.  \end{defn}

    \begin{defn}[Multiplicity] We say that $(a,b) \in C$ has multiplicity $m$
    is the smallest positive integer such that an order $m$ partial derivative
does not vanish at $m$.  \end{defn}

    We see that a point is singular if it has multiplicity at least $2$.

    \begin{exm}[Singular Conics] Take $(0,0) \in C$ and suppose the origin is a
    singularity. Then $f = \alpha x^2 + \beta y^2 + \gamma xy =
(ax+by)(cx+dy)$. Thus $C$ must be a union of two lines.  \end{exm}

    \section{Affine Zariski-Closed Sets} \subsection{Lecture 3 (Jan 29)}

    We will put plane curves away for a while and then return to them. In this
    section, we will review several facts from commutative algebra.

    \begin{thm}[Noether Normalization Lemma] Let $A$ be a finitely generated
        $k$-algebra.\footnote{Note here that $k$ does not have to be closed.}
        Then $A$ contains $x_1, \ldots, x_n$ algebraically independent over $k$
        and such that $A$ is integral over $k[x_1, \ldots, x_n]$.
        \begin{proof} We will assume that $k$ is infinite. Choose some
            generators $y_1, \ldots, y_r$. We argue by induction on $r$. If
            $r=0$, then $A=k$, so there is nothing to prove. If $y_1, \ldots,
            y_r$ are algebraically independent, then $A=k[y_1, \ldots, y_r]$. 
            
            Suppose that $f(y_1, \ldots, y_r) = 0$ for some $f \in k[Y_1,
            \ldots, Y_r]$. We can write \[ f = \sum_{i=0}^d h_i(y_1, \ldots,
            y_{r-1})y_r^i. \] If $h_d = 1$, then $y_r$ is integral over the
            subring $B$ generated by $y_1, \ldots, y_{r-1}$. By induction, $B$
            contains algebraically independent $x_1, \ldots, x_n$ such that $B$
            is integral over $k[x_1, \ldots, x_n]$. Therefore, $A$ must be
            integral over $k[x_1, \ldots, x_n]$ by transitivity of integrality.

            Noether's trick is to show that we can always reduce to the simple
            case by a linear change of variables. Introduce $y_1'=y_1-
            \lambda_1y_r, \ldots, y_{r-1}'=\lambda_{r-1}y_r, y_r' = y_r$. We
            see that $y_1', \ldots, y_r'$ generate $A$. Then we see that
            $f(y_1, \ldots, y_r) = f(y_1' + \lambda_1y_r', \ldots,
            y_{r-1}'+\lambda_{r-1}y_r', y_r') = 0$. Write $f = F +
            \mathrm{l.o.t.}$ and suppose $F$ is of degree $d$. Then we see that
            $F(y_1' + \lambda_1y_r', \ldots, y_{r-1}'+\lambda_{r-1}y_r', y_r')
            = F(\lambda_1, \ldots, \lambda_{r-1}, 1)y_r'^d$. We simply need
            $F(\lambda_1, \ldots, \lambda_{r-1}, 1) \neq 0$, which is always
        possible if $k$ is infinite.  \end{proof} \end{thm}

    Recall the fundamental theorem of algebra, which states that $\C$ is
    algebraically closed. This gives a bijection between $\C$ and maximal
    ideals $m \subset \C[x]$ by $a \leftrightarrow (x-a)$. Indeed, over any
    field, maximal ideals of $k[x]$ are principal ideals generated by
    irreducible monic polynomials. If $k = \overline{k}$, then all monic
    polynomials are linear.

    \begin{thm}[Weak Nullstellensatz] Let $k = \overline{k}$. Then there is a
        bijection between $\A^n = k^n$ and maximal ideals in $k[x_1, \ldots,
        x_n]$ given by $(a_1, \ldots, a_n) \leftrightarrow (x_1-a_1, \ldots,
        x_n-a_n)$.  \begin{proof} Given a point $(a_1, \ldots, a_n) \in \A^n$,
            consider a homomorphism $k[x_1, \ldots, x_n] \rightarrow k$ given
            by $f \mapsto f(a_1, \ldots, a_n)$. Then the kernel of this
            morphism is a maximal ideal, but it must be equal to $(a_1, \ldots,
            a_n)$

            Now take $m \subset k[x_1, \ldots, x_n]$ maximal and suppose
            $k[x_1, \ldots, x_n]/m \xrightarrow[\simeq]{\varphi} K \supset k$
            is a field. If $K=k$, define $a_i= \varphi(x_i)$. Then because
            $\varphi$ acts on the variables just like the evaluation morphism,
            we must have $m = \mathrm{ker} \varphi = (x_1-a_1, \ldots,
            x_n-a_n)$. 
            
            We need to show that $K=k$. Note that $K$ must be a finitely
            generated $k$-algebra. By Noether, $K$ is integral over a
            polynomial subring $k[x_1, \ldots, x_n]$. Recall the fact that if
            $A$ is a field and $A$ is integral over $B$, then $B$ is a field.
            To show this, we see that $b^{-1} \in A$, so $b^{-1} = (b^{-1})^r +
            a_1(b^{-1})^{r-1} + \cdots + a_r = 0$ where $a_1, \ldots, a_r \in
            B$. Multiplying through by $b^{r-1}$, we see that $b^{-1} \in B$.
        Therefore $k[x_1, \ldots, x_n]$ is a field and $n=0$ and $K$ is
    integral over $k$. Because $k$ is algebraically closed, $K=k$.  \end{proof}
\end{thm}

    \begin{defn} A subset $X \subset \A^n$ is called a closed affine
        set\footnote{Shafarevich calls these closed sets. Some people call them
            affine sets and other names. Reid uses algebraic sets.} if $X =
            \Set{\mathbf{a} \in \A^n | f_1(\mathbf{a}) = \cdots =
                f_r(\mathbf{a}) = 0 }$ for some polynomials $f_1, \ldots, f_r
            \in k[x_1, \ldots, x_n]$.  \end{defn}

    \begin{lem} Closed sets have the form $V(I) = \Set{(a_1, \ldots, a_n) \in
        \A^n | f(a_1, \ldots, a_n) = 0 \text{ for all $f \in I$}}$ for some
        ideal $I \subset k[x_1, \ldots, x_n]$.  \begin{proof} Take $X =
            \Set{f_1 = \cdots = f_r = 0}$. Then let $I = (f_1, \ldots, f_r)
            \subset k[x_1, \ldots, x_n]$. Then $V(I)$ must equal $X$. On the
            other hand, take $V(I)$. Because $k[x_1, \ldots, x_n]$ is
            Noetherian, let $f_1, \ldots, f_r$ to be generators of $I$. Then
        $V(f_1, \ldots, f_r) = V(I) = X$.  \end{proof} \end{lem}

    Now let $X \subset \A^n$ be a closed set. Define $I(X) = \Set{f \in k[x_1,
    \ldots, x_n] | f(\mathbf{a}) = 0 \text{ for all $\mathbf{a} \in X$}}$. We
    now have two maps $\Set{\text{Closed Sets}} \xleftrightarrow[I(X)]{V(I)}
    \Set{\text{Ideals}}$.

    \begin{thm}[Strong Nullstellensatz] $V(I(X))$ and $I(V(J)) = \sqrt{J}$.
        \begin{proof} Take $g = I(V(I))$. Suppose $I = (f_1, \ldots, f_r)$. We
            need to show that if $g$ vanishes at every point $(a_1, \ldots,
            a_n)$ such that $f_i(a_1, \ldots, a_n) = 0$, then $g^{\ell} =
            h_1f_1 + \cdots + h_rf_r$ for some $\ell$. We use a trick of
            Rabinowitch: Let $B = (f_1, \ldots, f_r, 1 - gx_{n+1}) \subset
            k[x_1, \ldots, x_{n+1}]$.  We show that $B = k[x_1, \ldots,
            x_{n+1}]$. If not, it is contained in some maximal ideal $B \subset
            m = (x_1-a_1, \ldots, x_{n+1}-a_{n+1})$. However, $f_i(a_1, \ldots,
            a_n) = 0$ and $1-g(a_1, \ldots, a_n)a_{n+1} = 0$. But remember $g
            \in I(X)$, so we obtain that $1=0$.

            Thus we can write $1 = \sum h_if_i + h_{n+1}(1-gx_{n+1})$. Then we
            write $x_{n+1} = \frac{1}{g(x_1, \ldots, x_n)}$. Then we obtain an
            expression of the form \[1 = \sum_{i=1}^r h_i(x_1, \ldots,
            \frac{1}{g(x_1, \ldots, x_n)})f_i.\] Then we can clear denominators
            and obtain that $g \in \sqrt{I}$.

            Finally we show that $V(I(X)) = X$. Then there exists an ideal $J$
        such that $X=V(J)$. Then $V(I(X)) = V(I(V(J))) = V(\sqrt{J}) = V(J) =
    X$.  \end{proof} \end{thm}

    \begin{rmk} $I(X)$ is always radical. In fact $V(\sqrt{I}) = V(I)$.
    \end{rmk}

    \begin{cor} The operations $I$ and $V$ give a bijection between closed sets
    in $\A^n$ and radical ideals in $k[x_1, \ldots, x_n]$.  \end{cor}

    \subsection{Lecture 4 (Jan 31)}

    Recall that the weak Nullstellensatz gives a bijection between points in
    $\A^n$ and maximal ideals in $k[x_1, \ldots, x_n]$. Also, the strong
    Nullstellensatz gives a bijection between affine closed sets and radical
    ideals. 

    \subsubsection{Zariski Topology on $\A^n$} \begin{prop} The affine closed
        sets $V(I)$ form the set of closed sets for the Zariski topology on
        $\A^n$.  \begin{proof} \begin{enumerate} \item $\A^n = V(0)$ and
            $\emptyset = V((1))$; \item $\cap_{i \in I} Y_i = V(\sum I_i)$;
\item $\cup_{i=1}^r Y_i = V(I_1\cdots I_r) = V(I_1 \cap \cdots \cap I_r)$.
\end{enumerate} \end{proof} \end{prop}

    \begin{defn}[Irreducible Closed Sets] A closed set is called irreducible if
    it is not the union of two proper closed subsets.  \end{defn}

    \begin{exm} In $\R^n$ with the Euclidean topology only points are
    irreducible.  \end{exm}

    \begin{thm} Under the correspondence between $V(I), I(X)$, we will see that
        irreducible subsets $Y \subset \A^n$ correspond to prime ideals
        $\mathfrak{p} \subset k[x_1, \ldots, x_n]$, which we denote by
        $\mathrm{Spec}\ k[x_1, \ldots, x_n]$.  \begin{proof} Suppose $Y \subset
            \A^n$ is reducible with $Y = Y_1 \cup Y_2$. Then $I(Y) \subsetneq
            I(Y_i)$ for $i=1,2$ by the Nullstellensatz. Then choose $f_1 \in
            I(Y_1) \setminus I(Y)$ and $f_2 \in I(Y_2) \setminus I(Y)$. Then
            $f_1f_2$ vanishes on $Y$ and therefore $I(Y)$ is not prime.

            Suppose $I(Y)$ is not prime. Then there exist $f,g \notin I(Y)=I$
        with $fg \in I(Y)$. Then write $Y_1 = V(f) \cap Y$ and $Y_2 = V(g) \cap
    Y$, so we see that $Y = Y_1 \cup Y_2$.  \end{proof} \end{thm}

    Now we consider this correspondence in the case of the plane.

    \begin{figure}[H] \begin{center} \begin{tabular}{cc} \toprule Irreducible
    subsets & Prime ideals \\ \midrule Points $(a,b)$ & Maximal ideals $(x-a,
y-b)$ \\ Irreducible affine plane curves & $(f)$ \\ $\A^2$ & 0 \\ \bottomrule
\end{tabular} \end{center} \caption{Irreducible subsets of $\A^2$ and prime
ideals in $k[x,y]$.} \end{figure}

    Why is there nothing else? Suppose we have a prime ideal $0 \neq
    \mathfrak{p} \subset k[x,y]$. Then write $\mathfrak{p} = (f_1, \ldots, f_r)
    = f(g_1, \ldots, g_r)$ where $f = \mathrm{gcd}(f_1, \ldots, f_r)$. Then
    either $f \in \mathfrak{p}$ or $g_i \in \mathfrak{p}$ for all $i$. If $f
    \in \mathfrak{p}$, we have that $\mathfrak{p} = (f)$. Otherwise,
    $\mathfrak{p} = (g_1, \ldots, g_r)$. However, the $g_i$ are coprime so
    $V(\mathfrak{p})$ is a finite union of points, so it must be a single
    point. Thus $\mathfrak{p}$ is maximal.

    ``That's what most arguments in algebraic geometry look like: some input
    from algebra and some input from geometry.''

    \begin{defn}[Irreducible Component] Let $Y \subset \A^n$ be a Zariski
    closed set. Then an irreducible component of $Y$ is a maximal irreducible
subset of $Y$.  \end{defn}

    \begin{lem} There exist only finitely many irreducible components $Y_1,
        \ldots, Y_s$ and $Y = Y_1 \cup \cdots \cup Y_s$.  \begin{proof} We show
            that $Y$ can be written as a finite union $z_1 \cup \cdots \cup
            z_r$ of irreducible subsets. Given that, we can also assume that
            $z_i \subsetneq z_j$ for all $i \neq j$. Take some irreducible
            subset $W \subset Y = Z_1 \cup \cdots \cup Z_r$. But in fact $W =
            \cup (W \cap Z_i)$, so $W \subset Z_i$ for some $I$. Then in
            particular, if $W$ is an irreducible component, then $W = Z_i$.
            Therefore this decomposition is a decomposition into irreducible
            components.

            To prove the claim, if $Y$ is irreducible, then clearly this is
            true. Then write $Y = Y_1 \cup Y_2$. If $Y_1, Y_2$ are irreducible,
            the we are done. Suppose $Y_1$ is reducible. Then write it as a
            union and continue to form a binary tree. But this tree must be
        finite because $k[x_1, \ldots, x_n]$ is Noetherian.  \end{proof}
    \end{lem}

    \begin{rmk} The algebraic counterpart to that is $I = \mathfrak{p}_1 \cap
        \cdots \cap \mathfrak{p}_s$.  \begin{proof} Take $f \in \mathfrak{p}_1
            \cap \cdots \cap \mathfrak{p}_s$. Then $f$ vanishes on every $y_i$,
            so it vanishes on $y$. Thus $f \in I$.  \end{proof} \end{rmk}

    \begin{thm} Every radical ideal in $l[x_1, \ldots, x_n]$ is the
    intersection of minimal prime ideals which contain $I$.\footnote{This is
true in every Noetherian ring.} \end{thm}

    We now shift to a more general perspective. So far we have considered
    $\A^n$ and $k[x_1, \ldots, x_n]$. Now we will consider $X,k[X]$ where $X$
    is an irreducible Zariski-closed subset of $\A^n$ and $k[X] = k[x_1,
    \ldots, x_n]/I(X)$ the coordinate ring of $X$. We see that $X$ is
    irreducible if and only if $k[X]$ is an integral domain.

    \begin{thm}[Generalized Nullstellensatz] There is a bijection between
        points of $X$ and maximal ideals of $k[X]$, between closed subsets of
        $X$ and radical ideals of $k[X]$, and between irreducible closed
        subsets of $X$ and prime ideals of $k[X]$.  \begin{proof} Suppose $Y
        \subset X$ is a closed subset. Then $I(Y) \supset I(X)$ as radical
    ideals. Then $I(Y)/I(X) \subset k[X]$ is a radical ideal.  \end{proof}
\end{thm}

    \subsection{Lecture 5 (Feb 5)}

    \subsubsection{Macaulay 2 Interlude} Recall that we have algebraic sets $X
    \hookrightarrow \A^n$ and the coordinate algebra $k[X] = k[x_1, \ldots,
    x_n]/I(X)$. Then for any $Y \subset X$, $I(Y) \subset k[X]$ is a radical
    ideal and $Y$ irreducible implies $I(Y)$ is prime. Then we can decompose
    $Y$ as a union of irreducible components and $I(Y)$ as an intersection of
    minimal primes.

    There are algorithms that can do these computations, but they generally use
    Gr\"obner bases, which are difficult to compute by hand. Fortunately, we
    can use computers, such as Macaulay2, to perform these calculations.

    \begin{mdframed}[style=default] \lstinputlisting{code/lec05_01.M2}
    \end{mdframed}

    We now consider the Clebsch cubic surface $x^3+y^3+z^3+1 = (x+y+z+1)^3$.
    There is a famous theorem that any smooth cubic surfaces contain exactly 27
    lines. We will calculate these 27 lines.

    \begin{mdframed}[style=default] \lstinputlisting{code/lec05_02.M2}
    \end{mdframed}

    We found an interesting phenomenon where the computation took longer and
    was the computer was unable to find the expected 25 affine lines when we
    worked over $\Z/103\Z$ because 5 is not a quadratic residue mod
    103.\footnote{We went into a digression about quadratic reciprocity.}

    \subsubsection{Morphisms of Affine Closed Sets}

    Let $f: \A^n \rightarrow \A^m$ be given by $y_i = f_i(x_1, \ldots, x_n)$,
    where $f_i \in k[x_1, \ldots, x_n]$ for $i = 1, \ldots, m$. Algebraically,
    this corresponds to a morphism $k[y_1, \ldots, y_m] \xrightarrow{\varphi}
    k[x_1, \ldots, x_n]$, which is given by $y_i \mapsto f_i(x_1, \ldots,
    x_n)$.

    We know that the kernel of this morphism is a prime ideal because the
    quotient by it is a subring of $k[x_1, \ldots, x_n]$, so is an integral
    domain. Then $V(\mathrm{Ker}\varphi) = \overline{f(\A^n)}$. To see this,
    suppose that $(b_1, \ldots, b_m) = f(a_1, \ldots, a_n)$. Then let $g \in
    \mathrm{Ker}\varphi$. We see that \begin{align*} g(b_1, \ldots, b_m) &=
        g(f_1(a_1, \ldots, a_n), \ldots, f_m(a_1, \ldots, a_n)) \\ &=
        [\varphi(g)](a_1, \ldots, a_n) \\ &= 0.  \end{align*}
    
    In fact, $\overline{\mathrm{Im} f} = V(\mathrm{Ker}\varphi)$.

    Let $X \hookrightarrow \A^n$. Now we define a morphism to be a restriction
    of a polynomial morphism from $\A^n \rightarrow \A^m$. We pull back $k[y_1,
    \ldots, y_m] \rightarrow k[X]$ in the same way.

    Finally, suppose $X \subset \A^n, Y \subset \A^m$. Then let $f:X
    \rightarrow \A^m$ and suppose $f(X) \subset Y$. Then we have a morphism $X
    \rightarrow Y$ and a pullback morphism $k[Y] \rightarrow k[X]$.

    \begin{thm} There is a bijection $\mathrm{Hom}(X,Y) \simeq
    \mathrm{Hom}(k[Y], k[X])$.\footnote{This induces an equivalence of
categories between affine algebraic sets and finite $k$-algebras with no
nilpotent elements.} \end{thm}

    \subsection{Lecture 6 (Feb 7)}

    \subsubsection{Morphisms Continued} Recall the definition of a morphism of
    affine algebraic sets. Also recall that any morphism $f : X \rightarrow Y$
    induces a pullback homomorphism $f^*: k[Y] \rightarrow k[X]$.
    
    Also recall Theorem 40. In particular, $X \simeq Y$ if and only if $k[X]
    \simeq k[Y]$.

    \begin{proof}[Proof of Theorem 40] Recover $f$ from $f^*$ by $b_i = y_i(b)
        = y_i(f(a)) = (f^*y_i)(a)$.  Then given a morphism $\alpha: k[Y]
        \rightarrow k[X]$, consider the following diagram.

        \begin{center} \begin{tikzcd} k[Y] \arrow{r}{\alpha} & k[X] \\ k[y_1,
        \ldots, y_m] \arrow[twoheadrightarrow]{u} & k[x_1, \ldots, x_n]
    \arrow[twoheadrightarrow]{u} \end{tikzcd} \end{center}

        Then there exist $f_i$ such that the image of $f_i$ under the map
        $k[x_1, \ldots, x_n] \rightarrow k[X]$ is $\alpha(y_i)$. Define $f =
        (f_1, \ldots, f_m)$. Clearly $\alpha = f^*$. 

        We check that $f(X) \subset Y$. Note $g \in O(Y)$ given by $g(f(a_1,
    \ldots, a_n)) = 0$ if and only if $g(f_1, \ldots, f_m) \in I(X)$ if and
only if $\alpha(g) \in I(X)$ if and only if $g$ restricted to $Y$ is zero.
        \end{proof}

    Recall that $X \hookrightarrow \A^n$ corresponds algebraically to $k[x_1,
    \ldots, x_n] \twoheadrightarrow k[X]$, which is choice of generators of
    $k[X]$.

    Let $G$ be a finite group acting on $X$. Then $G$ must also act on $k[X]$.
    \begin{thm} If the $\mathrm{char} k$ does not divide $\abs{G}$, then
    $k[x]^G$ is finitely generated.  \end{thm} Choose generators $g_1, \ldots,
    g_n$ of $k[X]^G$ given by $\psi: k[y_1, \ldots, y_n] \twoheadrightarrow
    k[X]^G$. Let $I = \ker \psi$.  \begin{defn}[Quotient by a finite group]
        Define $Y = V(I) \subset \A^n$. Then $k[Y] \simeq k[X]^G$. Then $Y =
        X/G$. Note that $k[X]^G \hookrightarrow k[X]$, so there is a quotient
    morphism $\pi: X \rightarrow X/G$.  \end{defn}
    

    \begin{exm} Consider the action of $\Z/2\Z$ on $\A^2$ given by $(x,y)
        \mapsto (-x,-y)$. This gives an action on $k[x,y]$, so $k[x,y]^{\Z^2} =
        k[x^2,y^2,xy] = k[u,v,w]/(uv-w^2)$. Then we see $\A^2/(\Z/2\Z) =
        V(uv-w^2) \hookrightarrow \A^3$ and the quotient morphism is given by
    $(x,y) \rightarrow (x^2,y^2,xy)$.  \end{exm}

    Now we can translate some geometry into algebra and some algebra into
    geometry.

    \begin{defn}[Dominance] A morphism $f:X \rightarrow Y$ is dominant if
    $f(X)$ is (Zariski-)dense in $Y$.  \end{defn}

    \begin{prop} $f$ is dominant if and only if $f^*$ is injective.
        \begin{proof} Suppose $f$ is dominant. Let $\varphi \in \ker \varphi$.
            We compute $\varphi(f(a)) = (f^*\varphi)(a) = 0$, so $\varphi
            |_{f(X)} = 0$.  However, $f(X) \subset Y$ is dense, so $\varphi|_Y
            = 0$.  

            Now suppose $f^*$ is injective but $f$ is not dominant. Therefore
            $f(X)$ is not dense in $Y$, which is possible if and only if there
            exists $\varphi \in K[Y]$ nonzero such that $\varphi|_{f(X)} = 0$.
            Therefore $\varphi(f(a)) = 0$ for all $a \in X$ and thus
        $f^*\varphi = 0$, so $f^*$ is not injective.  \end{proof} \end{prop}

    \begin{defn}[Finite Morphism] A dominant morphism $f: X \rightarrow Y$ is
    called finite if $k[X]$ is integral over $k[Y]$. More generally, a morphism
is finite if $k[X]$ is integral over the image of $k[Y]$.  \end{defn}

    \begin{exm}[Noether Normalization] An integral extension $k[X]
    \hookleftarrow k[y_1, \ldots, y_m]$ corresponds to a finite morphism $X
\rightarrow \A^m$.  \end{exm}

    \begin{exm}[Normalization of Cusp] Note the map $\A^1 \rightarrow
        V(y^2-x^3) \subset \A^2$ given by $t \mapsto (t^2,t^3)$. Algebraically,
        $k[x,y]/(y^2-x^3) \rightarrow k[t]$ given by $x \mapsto t^3, y \mapsto
        t^2$. Because the morphism is dominant, we have $R \simeq k[t^2, t^3]
    \hookrightarrow k[t]$ is an integral extension ($t$ is a root of
$T^2-t^2$).  \end{exm}

    \begin{exm} Consider $k[X]^G \subset k[X]$. This is always integral, so $X
        \twoheadrightarrow X/G$ is finite. Consider $\alpha \in K[X]$. It is a
        root of $\prod_{g \in G}(T-g\alpha)$. Each coefficient is invariant
        because they are elementary symmetric functions on the $g\alpha$.
    \end{exm}

    Recall the Going-up theorem:

    \begin{thm}[Going-up] Let $\mathfrak{p} \subset A$ be a prime ideal and let
    $A \hookrightarrow B$ be an integral extension. Then there exists a prime
ideal $\mathfrak{q} \subset B$ such that $\mathfrak{q} \cap A = \mathfrak{p}$.
\end{thm}

    \begin{thm} Any dominant finite morphism is surjective and has finite
        fibers.  \begin{proof} Let $\alpha$ be a finite dominant morphism. Take
            $y \in Y$, which corresponds to a maximal ideal $m \subset K[Y]$.
            We find a point $x \in X$ (a maximal ideal $n \subset k[X]$) such
            that $\alpha(x) = y$ ($n \cap k[Y] = m$). By going-up, there exists
            a prime ideal $q \subset k[X]$ such that $q \cap k[Y] = m$. Take
            any maximal ideal $n \supset q$. Then $n \cap k[Y] \supset m$, so
            by maximality of $m$, $n \cap k[Y] = m$.

            Now we prove that the fibers are finite. Let $y \in Y$ correspond
            to a maximal ideal $m \subset k[Y]$. Then we show that there are
            finitely many maximal ideals in $k[X]/(k[x]\cdot m)$. Then $k[X]$
            is integral over $k[Y]$, so $A = k[X]/(k[x]\cdot m)$ is integral
            over $k[Y]/m = k$. Therefore $A$ is a finitely-generated $k$-module
            (a finite-dimensional $k$-vector space).\footnote{We run into an
                algebra qual problem (Artinian rings have finitely many maximal
        ideals).} \end{proof} \end{thm}

    \subsection{Lecture 7 (Feb 14 $\heartsuit$)}

    We begin with a remark: Let $\alpha: X \rightarrow Y$ be a morphism of
    affine algebraic sets $X \rightarrow Y$. Then there is a pullback
    $\alpha^*: k[Y] \rightarrow k[X]$. We note that if $Z \subset Y$ is closed,
    then $\alpha^{-1}(Z) \subset X$ is also closed. Then if $I(Z) = (f_1,
    \ldots, f_r)$, the ideal of $\alpha^{-1}(Z)$ is given by $(\alpha^*f_1,
    \ldots, \alpha^*f_r)$.

    This becomes trickier when $\alpha$ is dominant (which is equivalent to
    $\alpha^*$ being injective). Then we recall the computation of the fiber
    over a point to prove that dominant morphisms are surjective with finite
    fibers.\footnote{Jenia discussed this again in class, but I will point you
    back to Theorem 51.}

    We prove that Artinian rings have finitely many maximal ideals:

    \begin{lem} Artinian rings have finitely many maximal ideals.
        \begin{proof} Consider all finite intersections of maximal ideals. This
            set contains a minimal element $m_1 \cap \cdots \cap m_r$x.
            Therefore $m_1, \cap \cdots \cap m_r \subset m$ for al maximal
            ideals $m$.  Then we see $m = m_i$ for some $i$. (Otherwise, there
        exists $x_i \in m_i \setminus m$, and then $x_1\cdots x_r \in m$, which
    contradicts maximality.) \end{proof} \end{lem}

    We recall our discussion of group actions and quotients. Suppose we have a
    finite group $G$ acting on $X$. Then we have a finitely generated algebra
    $k[X]^G \hookrightarrow k[X]$. Then we define $X/G$ to be an affine
    algebraic set such that $k[Y] \simeq k[X]^G$. Then there exists a quotient
    morphism $\pi: X \rightarrow X/G$.

    \begin{prop} The fibers of $\pi$ are $G$-orbits.  \begin{proof} First we
        show that $y = gx$ implies $\pi(X) = \pi(y)$. Suppose not.  Then there
        exists $f \in k[X/G]$ such that $f(\pi(x)) = 0$ and $f(\pi(y)) = 1$. We
        have a pullback $\pi^*: k[X]^g \hookrightarrow k[X]$. Then there exists
        a $G$-invariant function $f$ such that $f(x) = 0$ and $f(y) = 1$, which
        is impossible.

            Next we show that the fiber is precisely the orbit. Suppose that
            $G$-orbits of $x,y$ are disjoint. Then there exists $f \in k[X]$
            such that $f \vert_{G \cdot x} = 0$ and $f \vert_{G \cdot y} = 1$.
            Then we define $f^\#(x) = \frac{1}{\#G} \sum_{g \in G} f(g \cdot
            x)$. This new polynomial is now $G$-invariant and is always $0$ on
            the orbit of $x$ and $1$ on the orbit of $y$. Then we see that
            $f^{\#}(\pi(x)) = (\pi^*f^{\#})(x) = f^{\#}(x) = 0$ and
        $f^{\#}(\pi(y)) = (\pi^*f^{\#})(y) = f^{\#}(y) = 1$. Thus $\pi(x) \neq
    pi(y)$.  \end{proof} \end{prop}

    \subsubsection{Rational Maps}

    Now let $X$ be an affine variety. Then $k[X]$ is an integral domain.
    \begin{defn}[Function Field] The function field $k(X)$ of an affine variety
    $X$ is the field of fractions of $k[X]$.  \end{defn}

    \begin{defn}[Regular function] Let $f \in k(X)$. Then $f$ is regular at $x
    \in X$ if $f$ can be written as a fraction $f = \frac{p}{q}$ where $p,q \in
k[X]$ and $q(x) \neq 0$.  \end{defn}

    \begin{defn}[Domain of Definition] The domain of definition of $f$ is the
    set of all $x \in X$ such that $f$ is regular at $X$.  \end{defn}

    \begin{prop} The domain of definition is open.  \begin{proof} $D =
    \bigcup_{f = p/q} (X\setminus V(q))$.  \end{proof} \end{prop}

    \begin{prop} $f(x)$ is regular at every $x \in X$ if and only if $f \in
        k[X]$.  \begin{proof} For all $x \in X$, there exists an expression $f
            = p_x/q_x$ where $q_x(x) \neq 0$. Then take $I = (q_x)$. The
            vanishing set $V(I) = \emptyset$, so $\sqrt{I} = (1)$, which
            implies that $1 \in I$. We can write $1 = a_1q_{x_1} + a_2q_{x_2} +
            \cdots + a_rq_{x_r}$. Then \[ f = f \cdot 1 = \sum_{i=1}^r
            a_i(q_{x_i}f) = \sum_{i=1}^r a_ip_{x_i} \in k[X]. \] \end{proof}
        \end{prop}
    
    \begin{defn}[Local ring of a point.] Let $x \in X$. Then the ring
    $\mathcal{O}_x$ is the set of all functions regular at $x$. Note that this
is the same as $k[X]_{m_x}$.  This is a local ring.  \end{defn}

    \begin{defn}[Rational Maps] Let $Y \hookrightarrow \A^n$. Then a rational
        map is a map $f = (f_1, \ldots, f_n)$, where $f_i \in k(X)$ such that
        $f(x) \in Y$ whenever $f_1, \ldots, f_n$ are regular at $x$. Given
        this, we have a pullback homomorphism $f^*: k[Y] \rightarrow k(X)$.
    \end{defn}

    \begin{prop} We will see that $f^*$ is injective if and only if $f(X)$ is
        dense in $Y$. If so, we induce a field extension $k(Y) \hookrightarrow
        k(X)$.  Additionally, the other way around, given $\alpha: k(Y)
        \hookrightarrow k(X)$, we can construct a dominant rational map $f:X
        \dashrightarrow Y$ such that $\alpha = f^*$.  \begin{proof} Take $f =
        (f_1, \ldots, f_n) = (\alpha(y_1), \ldots, \alpha(y_n))$.  \end{proof}
    \end{prop}

    \begin{defn}[Birational Equivalence] $X$ and $Y$ are birationally
        equivalent if $k(X) \simeq k(Y)$, or equivalently, there exist dominant
        rational maps $f,g: X \dashleftarrow \dashrightarrow Y$ such that $f
        \circ g, g \circ f$ are the identity wherever they are defined.  In
    particular, $X$ is rational if it is birational to an affine space.
\end{defn}

    We will skip projective and quasiprojective varieties and return to them
    next week.

    \subsubsection{Dimension} Let $X \subset \A^N$. What is the dimension of
    $X$? Here are some ideas: \begin{itemize} \item ``Maximal number of
        independent parameters.'' To make this rigorous, we define the
        dimension of $X$ to be the transcendence degree of $k(X)$; \item
``Maximum possible dimension of a subspace $+1$.'' Convert this into algebra
and we define the dimension of $X$ to be the Krull dimension of $k[X]$.
\end{itemize}

    \begin{thm} If $X$ is an affine variety, then the two definitions of
        dimension agree.  \begin{proof} By Noether's Normalization Lemma,
            $k[X]$ is integral over its polynomial subalgebra $k[y_1, \ldots,
            y_n] = k[\A^n]$. Then $\mathrm{dim}\ X = \mathrm{trdeg}\ k(X) =
            \mathrm{trdeg}\ k(y_1, \ldots, y_n) = n = \mathrm{dim}\ \A^n$.

            First we show that the Krull dimension of $k[y_1, \ldots, y_n] =
            n$, and then we show that Krull dimension is preserved by integral
            extensions.

            To see the second claim, let $p_1 \subsetneq \cdots \subsetneq p_r
            \subsetneq k[X]$ be a chain of prime ideals. Then $p_1 \cap R
            \subset \cdots \subset p_r \cap R \subset R$ is a chain of prime
            ideals. As an exercise, show that $p_i \cap R \neq p_{i+1} \cap R$.
            Now let $q_1 \subset \cdots q_r \subset R$ be a chain of prime
            ideals in $R$. Then we use a stronger going-up theorem to lift the
            whole chain to $k[X]$. If the $q_i$ are different, then the lifted
        ideals are different. Thus the Krull dimensions are equal.  \end{proof}
    \end{thm}

    \section{Projective Space}
    
    \subsection{Lecture 8 (Feb 21, given by Luca Schaffler)}

    We will work over an algebraically closed field.

    \begin{defn}[Projective Space] Let $n \in \Z$ be positive. Then the
        projective space $\P^n$ is defined as $\P^n = (k^{n+1} \setminus \{0\}
        ) / G_m$, which is the set of lines in $k^{n+1}$ through the
        origin.\footnote{The mean way to define this is $\P^n =
    \mathrm{Gr}(1,n+1)$} \end{defn}

    In $\P^n$, we denote the equivalence class of $(z_0, \ldots, z_n)$ by
    $(z_0:\cdots:z_n)$ in homogeneous coordinates. Observe that at least one of
    $z_0, \ldots, z_n$ is nonzero.

    \begin{defn} We say that a polynomial $f \in k[z_0, \ldots, z_n]$ vanishes
        at a point $\xi \in \P^n$ if $f(x_0, \ldots, x_n)$ for every choice of
        homogeneous coordinates $\xi = (x_0:\cdots:x_n)$. In this case, we
        simply write $f(\xi) = 0$.  \end{defn}

    \begin{defn}[Homogeneous Polynomial] A polynomial $f \in k[z_0, \ldots,
    z_n]$ is homogeneous if every monomial term in $f$ has the same degree.
\end{defn}

    \begin{exm} $z_0^3 + z_0z_1^2 + z_1^2z_2$ is homogeneous with degree $3$.
    \end{exm}

    \begin{rmk} Every polynomial $f \in k[z_0, \ldots, z_n]$ can be decomposed
    as the sum of its homogeneous components, i.e. $k[x_1, \ldots, x_n]$ is a
graded algebra.  \end{rmk}

    \begin{prop} If $f \in k[z_0, \ldots, z_n]$ vanishes at $p \in \P^n$, then
        all of its homogeneous components vanish at $p$.  \begin{proof} Let $f
            = \sum_{d \geq 0} f_d$ and $p = (x_0:\cdots:x_n)$. Then for all
            $\lambda \in k^*$ we have \begin{align*} f(\lambda x_0, \ldots,
            \lambda x_n) &= \sum_{d \geq 0} f_d(\lambda x_0, \ldots, \lambda
        x_n) \\ &= \sum_{d \geq 0} \lambda^d f_d(x_0, \ldots, x_n) = 0.
    \end{align*} Because $k$ is infinite, then $f_d(x_0, \ldots, x_n) = 0$ for
    all $d \geq 0$. Therefore all $f_d$ vanish at $p$.  \end{proof} \end{prop}

    \begin{rmk} With the same assumptions as the previous proposition, the
    above proof also implies that $f_0 = 0$.  \end{rmk}

    \begin{defn}[Closed Set] Let $X \subset \P^n$ is called closed if $X$ is
    the vanishing set of some set of polynomials $f_1, \ldots, f_r \in k[x_0,
\ldots, z_n]$.  \end{defn}

    By the previous proposition, it is not restrictive to assume that $f_1,
    \ldots, f_r$ are homogeneous.

    \begin{defn}[Homogeneous Ideal] An ideal $I \subset k[z_0, \ldots, z_n]$ is
    called homogeneous if it is generated by homogeneous polynomials.
Equivalently, $I$ is closed under taking homogeneous parts.  \end{defn}

    \begin{defn}[Vanishing Set of Ideal] Let $U \subset k[z_0, \ldots, z_n]$ be
    an ideal. Define $V(I) = \Set{p \in \P^n | f(p) = 0 \text{ for all $f \in
I$ }}$.  \end{defn}

    \begin{rmk} If $I \subset k[z_0, \ldots, z_n]$ is a homogeneous ideal, then
    by the Hilbert Basis Theorem, $I = (f_1, \ldots, f_r)$, where each $f_i$
can be assumed to be homogeneous.  \end{rmk}

    \begin{defn}[Zariski Topology on $\P^n$] The closed subsets of $\P^n$ are
        $V(I)$, where $I \subset k[z_0, \ldots, z_n]$ is a homogeneous ideal.
        This forms a topology. If $X \subset \P^n$ is closed, then the Zariski
        topology on $X$ is the one induced by $\P^n$.  \end{defn}

    \subsubsection{Affine Constructions}

    We will now consider a useful construction.  \begin{defn}[Affine Cone] Let
    $I \subset k[z_0, \ldots, z_n]$ be a homogeneous ideal. We can conside the
vanishing set $V(I) \subset \P^n$ or the affine cone $V^a(I) \subset \A^{n+1}$.
\end{defn}

    \begin{exm} Consider $V(z_0) \subset \P^n$, which is a point. Then
    $V^a(z_0)$ is a line.  \end{exm}

    \begin{rmk} $V(I) = (V^a(I) \setminus \{0\}) / G_m$.  \end{rmk}

    \begin{rmk} $I(V(I)) = I(V^a(I))$.  \end{rmk}

    \begin{rmk} In the affine cone construction, then something happens: If $I
        = k[z_0, \ldots, z_n]$, then $V(I) = \emptyset$ and $V^a(I) =
        \emptyset$. Let $I = (z_0, \ldots, z_n)$. Then $V(I) =\emptyset $ but
        $V^a(I) = \{0\}$.  The map $V^a(I) \mapsto V(I)$ fails to be injective
        exactly in this case.  \end{rmk}

    \begin{prop} Let $I \subset k[z_0, \ldots, z_n]$ be a homogeneous ideal.
        Then the following are equivalent: \begin{enumerate} \item $V(I) =
            \emptyset$; \item $\sqrt{I} \supset (z_0, \ldots, z_n)$; \item
            There exists $s \in \Z_{>0}$ such that $I \supset I_s$ and $V(I_s)
        = \emptyset$ but $V^a(I) = \{0\}$.  \end{enumerate} \begin{proof} First
        we prove $(1) \Leftrightarrow (2)$. This is because $V(I) = \emptyset$
        if and only if $V^a(I) \subset \{0\}$ if and only if $\sqrt{I} =
        I(V^a(I)) \supset I(0) = (z_0, \ldots, z_n)$. To show that $(3)
        \Rightarrow (2)$, note that $I_s \subset I$, so $(z_0, \ldots,
        z_n)=\sqrt{I_s} \subset \sqrt{I}$.

            To see that $(2) \Rightarrow (3)$, then for all $i = 0, \ldots, n$,
        there exists $p_i \in \Z_{> 0}$ such that $z_i^{p_i} \in I$. Let $p =
    \max\{p_i\}$. Then if $s > (n+1)(p-1)$, $I_s \subset I$.  \end{proof}
\end{prop}

    \begin{defn} In $\P^n$ consider the open subset $\A_i^n =
        \Set{(z_0:\cdots:z_n) \in \P^n | z_i \neq 0}$. Note that there is an
        identification $\A_i^n \to \A^n$ given by $(z_0:\cdots:z_n) \to \left(
        \frac{z_0}{z_i}, \ldots, \frac{z_n}{z_i} \right)$. Moreover, for all $X
        \subset \P^n$ closed, we can define $X_i \coloneqq X \cap \A_i^n$.
    \end{defn}

    Explicitly, if $X = V(F_1, \ldots, F_r)$, then $X_i = V(f_1^{(i)}, \ldots,
    f_r^{(i)})$, where $f_j^{i}(x_0, \ldots, x_n) = F_j(x_0, \ldots, 1, \ldots,
    x_n)$. this is known as de-homogeneization.

    Conversely, every closed $Y \subset \A^n$ defines canonically $\overline{Y}
    \subset \P^n$ where we identify $\A^n$ with $\A_0^n$. The equations for
    $\overline{Y}$ are given as follows. If $Y = V^a(f_1, \ldots, f_r)$, define
    $F_j(z_0, \ldots, z_n) = z_0^{\mathrm{deg} f_j} f_j \left( \frac{z_1}{z_0},
    \ldots, \frac{z_n}{z_0} \right)$, which is known as homogeneization. 

    \begin{prop} $\overline{Y} = V(F_1, \ldots, F_r)$.  \end{prop}

    \begin{rmk} $X$ is not nexessarily the closure of its affine patches. For
    example, consider a line at infinity. It is lost during the procedure.
\end{rmk}

    \begin{rmk} Let $F$ be a homogeneous polynomial. Then $F =
    z_i^{\mathrm{deg} F} f^{(i)}$.  \end{rmk}

    \subsubsection{Irreducibility} \begin{defn} A closed set $X \subset \P^n$
    is irreducible if $X$ cannot be written as a union $X = X_1 \cup X_2$,
where $X_1, X_2$ are proper closed subsets.  \end{defn}

    \begin{rmk} We have a decomposition into irreducible components like in the
    affine case. Also, $X$ is irreducible if and only if $I(X)$ is prime.
\end{rmk}

    \subsubsection{Projective and Quasiprojective Varieties}
    \begin{defn}[Projective Variety] $X \subset \P^n$ closed is called a
    projective variety.  \end{defn}

    \begin{defn}[Quasiprojective Variety] A quasiprojective variety is an open
    subset of a projective variety.  \end{defn}

    \begin{exm} Affine and projective varieties are both quasiprojective.
    \end{exm}

    \subsection{Lecture 9 (Feb 26)} Let $X \subset \P^n$ be a quasiprojective
    variety. We need to define a regular function on $X$. We define $k(\P^n) =
    \set{\frac{f(z_0, \ldots, z_n)}{g(z_0, \ldots, z_n)} | f,g \in k[z_0,
        \ldots, z_n]_d \text{ for some $d$ }}$, where $z_0, \ldots, z_n$ are
        homogeneous coordinates on $\P^n$.

    \begin{rmk} In an affine chart, $\A_0^n \subset \P^n$ given by $z_0 \neq
        0$, we can write \[ \frac{f(z_0, \ldots, z_n)}{g(z_0, \ldots, z_n)} =
        \frac{z_0^df(1,z_1/z_0, \ldots, z_n/z_0)}{g(1, z_1/z_0, \ldots,
    z_n/z_0)} = \frac{f(1, x_1, \ldots, x_n)}{g(1, x_1, \ldots, x_n)}. \] We
can reverse this process, so $k(\P^n) = k(\A^n)$.  \end{rmk}

    If $f \in k(\P^n)$, $f = \frac{p}{q}$, then $f$ defines a function in some
    open neighborhood of $x \in X$.  \begin{defn}[Regular at a Point] We call
    this function $f$ regular at $x \in X$.  \end{defn}

    \begin{defn}[Regular Function] A function $f:X \to k$ is regular on $X$ if
    for all $x \in X$, $f$ can be written as $f = \frac{p}{q}$ in some open
neighborhood of $x \in X$ with $p,q \in k[z_0, \ldots, z_d]$ and $q(x) \neq 0$.
\end{defn}

    \begin{rmk} We write $k[X]$ for the set of regular functions on $X$.
    \end{rmk}

    \begin{rmk} If $X$ is affine, then $k[X] = k[x_1, \ldots, x_n]/I(X)$.
    \end{rmk}

    \begin{rmk} If $X$ is projective, then $k[X] = k$.  \end{rmk}

    \begin{rmk} There are examples of quasi-projective varieties $X$ such that
    $k[X]$ is not finitely generated.\footnote{Jenia wrote a paper about this.
The first examples were due to Rees and Nagata.} \end{rmk}

    \begin{defn} A map $X \to \A^m$ is regular if $f = (f_1, \ldots, f_m)$ and
    $f_i \in k[X]$.  \end{defn}

    \begin{defn} A map $f: X \to Y \hookrightarrow \P^m$ quasiprojective
        varieties is called regular if for all $x \in X$ we can choose an
        affine chart $\A_i^m \subset \P^m$ containing $f(x)$ and an open set $U
        \ni x$ such that $f(U) \subset \A_i^m$ and the induced map $f\vert_U$
        is regular.  \end{defn}

    \begin{rmk} This definition does not depend on the choice of affine chart
    containing $f(x)$.  \end{rmk}

    \begin{defn} Quasi-projective varieties $X,Y$ are isomorphic if there
    exists regular maps $f:X \to Y$ and $g:Y \to X$ such that $f = g^{-1}$.
\end{defn}

    \begin{lem} Let $X$ be a quasi-projective variety. Then every $x \in X$ has
        an affine neighborhood.  \begin{proof} Let $X \subset \P^n$ and $x \in
            \A_i^n$. Then $X \cap \A_i^n = \overline{X} \cap \A_i^n -
            (\overline{X}\setminus X) \cap \A_i^n$, so $x \in Y - Z \subset
            \A^n$, where $Y,Z$ are both affine. Choose $0 \neq F \in I(Z)
            \subset k[Y]$ and set $u = Y - V((F))=: D(F)$.  We will show that
            $D(F)$ is affine, called a principal open set.  Principal open sets
            form a basis of the Zariski topology.

            To prove this, note that $Y = (G_1 = \cdots = G_s= 0) \subset
            \A^n$. Define $Z \coloneqq (G_1 = \cdots G_s = Fy_{n+1} = 0)
            \subset \A^{n+1}$. Then $Z$ is affine and isomorphic to $D(F)$.
            Simply take the last coordinate to be $1/F$, which is possible
        because $F \neq 0$ on $Z$.  \end{proof} \end{lem}

    \begin{defn} A regular map $f:X \to Y$ of quasi-projective varieties is
        called finite if all $y \in Y$ have an affine neighborhood such that
        $f^{-1}(V) = U$ is also affine and the induced map $f\vert_U$ is a
        finite map of affine varieties.  \end{defn}

    \begin{prop} If $X, Y$ are both affine, then this agrees with the previous
        definition.  \begin{proof} Suppose $f:X \to Y$ is a map of affine
            varieties. Let $k[X] = B, k[Y] = A$ and the pullback be $f^*:A \to
            B$. For all $y \in Y$, there exists an affine $y \in V \subset Y$
            such that $U = f^{-1}(V) \subset X$ is affine and $f|_U$ is finite.

        Choose a principal open set $y \in D(F) \subset V \subset Y$ for some
        $f \in A$. Then $f^{-1}(D(F)) \subset U \subset X$. We see that
        $f^{-1}(D(F)) = D(f^*F)$, so it is affine.

        Now we show that $f\vert_{D(f^*F)}$ is finite. To prove this claim,
        note that $k[U]$ is a finite $k[V]$-module via $f^*$, so we show that
        $k[U][\frac{1}{f^*F}]$ is a finite $k[V][\frac{1}{F}]$-module. Simply
        take the basis to be the basis of $k[U]$ over $k[V]$.

        We need to show that $B$ is a finite $A$-module via $f^*$. We know that
        $Y$ is covered by principal open sets $D(F_{\alpha})$ such that
        $k[D(f^*F)]$ is a finite $k[D(F)]$-module via $f^*$. We see that
        $k[D(F)] = A[\frac{1}{F}]$, so $k[D(f^*F)] = B[\frac{1}{f^*F}]$. Also,
        $Y$ is covered by $D(F_{\alpha})$ if and only if $(F_{\alpha}) = A$.
        Thus we can write $1 = \sum h_{\alpha}F_{\alpha}$. In particular, we
        need only finitely many $D(F_{\alpha})$.

        We know that $B[\frac{1}{f^*F_{\alpha}}]$ is a finite
        $A[\frac{1}{F_{\alpha}}]$-module. Choose a basis of the form
        $\omega_{\alpha, i} \in B$. We will show that the $\omega_{\alpha, i}$
        form a basis of $B$ over $A$. Choose $b \in B$. Then for all $\alpha$
        we can write $b = \sum \omega_{\alpha, i} \frac{a_{i,
        \alpha}}{F_{\alpha}^{n_{\alpha}}}$. We can still write $1 = \sum
        H_{\alpha}F_{\alpha}^{n_{\alpha}}$.

        We write $b = b \sum H_{\alpha}F_{\alpha}^{n_{\alpha}} = \sum
    H_{\alpha}(\sum_i \omega_{\alpha, i}a_{i,\alpha})$.  \end{proof} \end{prop}
    
    \subsection{Lecture 10 (Feb 28)} Let $f:X \to Y$ be a morphism of
    quasiprojective varieties. Then $f$ is locally given by a polynomial map $f
    = (f_1, \ldots, f_m)$ where $f_i = \frac{p_i(x_0, \ldots, x_n)}{q_i(x_0,
    \ldots, x_n)}$ and $f_i$ is of degree $0$.

    Homogeneizing, we obtain that $f$ is given by $f(x) = [F_0:\cdots:F_m]$,
    where $F_i(x) \in k[x_0, \ldots, x_n]_d$ for some $d$. Also, for every $x
    \in X$, there exists a presentation $[f_0:\cdots:F_m]$ such that at least
    one $F_i(x) \neq 0$.

    \begin{prop} The above two notions are equivalent.  \begin{proof} Suppose
    $f$ regular at $x$ and $f(x) \in \A^m_0$. Then $f(x) =
[1:p_1/q_1:\cdots:p_m/q_m]$, so we can clear denominators to get $f =
[q_1\cdots q_m:\cdots:\cdots]$.  \end{proof} \end{prop}

    \begin{exm} Consider the projection from $\P^{\ell} \subset \P^n$. On the
        affine cones, this is just a projection, so the projection is given by
        $[x_0:\cdots:x_n] \mapsto [x_{\ell+1}:\cdots:x_n]$. In this case the
        $F_i$ are just coordinates. This gives a regular map $\P^n \setminus
    \P^{\ell} \to P^{n-\ell - 1}$.  \end{exm}
    
    \begin{exm} Consider the $d$-th Veronese embedding $\P^n \xrightarrow{v_d}
        \P^N$ given by $x \mapsto [F_i(x)]$ where $F_i$ runs through monomials
        of degree $d$ and $N+1 = \binom{n+d}{d}$. This is regular everywhere
        because one of the powers is nonzero. In fact, $v_d$ is an embedding.

        An example of this is the rational normal curve.  \end{exm}

    \begin{thm} Let $X \subset \P^n$ be an irreducible projective variety. Then
        $k[X] = k$.  \begin{proof} We will deduce this from another theorem,
            which is given below.  Using the theorem, take $f \in k[X]$. Then
            $f$ is a morphism $X \to \A^1 \hookrightarrow \P^1$. If $f(X)$ is
            dense in $\A^1$, then it is dense in $\P^1$. By the next theorem,
            $f(X) = \P^1$, which contradicts the fact that $f(X) \subset \A^1$. 

            If $f(X)$ is not dense in $\A^1$, then $f(X) = \{p_1, \ldots,
        p_r\}$. However, $X$ is irreducible, so $f(X) = p$ (consider the fiber
    above each point).  \end{proof} \end{thm}

    \begin{thm} Let $X$ be a projective variety and $f:X \to Y$. Then $f(X)
        \subset Y$ is closed.  \begin{proof} To prove this, we prove the main
            theorem of elimination theory. Now we consider the map $f:X \to Y$.
            Then consider the graph $\Gamma_f: X \to X \times Y$ given by $x
            \to (x,f(x))$. We show that $\Gamma = \Gamma_f(X)$ is closed for
            all $X,Y$. Using this, we reduce to the following: \begin{quote}
                \begin{thm} Let $X \subset X \times Y$ be closed. Then
                    $\pi_2(Z) \subset Y$ is closed.  \end{thm} This motivates
                    the following definition: \begin{defn} A variety is called
                        proper if for every variety $Y$ and closed subvariety
                        $Z \subset X \times Y$, $\pi_2(X) \subset Y$ is closed.
                    \end{defn} \end{quote} Next we cover $Y$ by affine open
                    sets $Y_i$. Then $Z = \bigcup z \cap (X \times Y_i) = Z_i$
                    and $Z_i$ is closed in $X \times Y$.  Then $\pi_2(X) = \cup
                    \pi_2(Z_i)$. It is enough to show that the projections of
                    the $z_i$ are closed.

            Now we have $X \subset \P^n$ projective and $Y \subset \A^m$
            affine. Then $Z$ is defined by equations $g_i(u,y)$ homogeneous in
            $x$ and arbitrary in $y$. The second projection of $Z$ is precisely
            the locus $T$ from the main theorem of elimination theory, which is
        closed.  \end{proof} \end{thm}

    \begin{thm}[Main Theorem of Elimination Theory] Let $g_i(u,y)$ be
        polynomials homogeneous in $u = (u_0, \ldots, u_n)$ and arbitrary in $y
        = (y_1, \ldots, y_m)$. Let $T = \Set{y_0 \in \A^m | g_i(u,y_0)=0 \text{
        has a nonzero solution }}$. Then $T$ is closed.  \begin{proof} Recall
            $g_i(u, y_0)=0$ has a nonzero solution if and only if $(g_1(u,y_0),
            \ldots, g_t(u,y_0)) \not \supseteq I_s$ for all $s$.  Then $T =
            \cap_{n \geq 1} T_s$ where $T_s = \Set{y_0 \in \A^m | (g_1(u,y_0),
            \ldots g_t(u,y_0)) \not\supseteq I_s}$. It suffices to check that
            $T_s$ is closed, or that the complement is open.

            Indeed, we see that for all monomials $N_{\alpha \in I_s}$, then
            $M_{\alpha} = \sum g_i(u,y_0)F_{i,\alpha}(u)$. Let $N_{i,\beta}$ be
            all monomials of degree $s - \mathrm{deg}g_i$. Thus $M_{\alpha}\in
            I_s$ is in the linear span of $g_i(u,y_0) \cdot N_{i,\beta}$.
            Therefore, $g_i(u,y_0) \cdot N_{i,\beta}$span the vector space of
            all degree $s$ polynomials, which means that at least one of the
            maximal minors is nonzero. Therefore the complement of $T_s$ is a
        union of open sets and is thus open.  \end{proof} \end{thm}

    \subsubsection{Products}% We discuss products of projective spaces. To do
    this, we consider the Segre embedding $\P^n \times \P^m \to \P^N$, where
    $N+1 = (n+1)(m+1)$, given by $[x_0:\cdots:x_n], [y_0:\cdots:y_m] \mapsto
    [x_iy_j]_{ij}$. Note that at least one $x_i$ and at least one $y_j$ are
    nonzero, so this is well-defined.

    Alternately, we present this as $[x],[y] \mapsto [xy^T]$, which gives a
    linear map $\psi:k^{m+1} \to k^{n+1}$ with kernel $y^{\perp}$ and image
    $x$. Therefore, the Segre embedding is injective. Matrices of this form are
    precisely the rank $1$ matrices, so the image is defined by equations the
    $2 \times 2$ minors of the matrix. In particular, $\P^n \times \P^m$ is a
    projective variety. This endows $\P^n \times \P^m$ with the structure of a
    projective variety.

    \begin{rmk} In the charts $\A^n_0, \A^m_0$, we see
        $s([1:x_1:\cdots:x_n],[1:y_1:\cdots:y_m]) =
        [1:x_1:\cdots:x_n:y_1:\cdots:y_m:c_iy_j] \subset \A_0^N$. This image is
        isomorphic to $\A^{n+m}$. Therefore this agrees on the charts with a
    product of affine spaces.  \end{rmk}

    If $X \subset \P^n, Y \subset \P^m$ are projective varieties, then $X
    \times Y \subset \P^n \times \P^m$ is a projective variety.

    To define closed subvarieties in $\P^n \times \P^m$ (or $\P^n \times \A^m$)
    without referring to the ambient projective space $\P^N$, we do the
    following:

    \begin{thm} The closed subvarieties of $\P^n \times \P^m$ are given by
        multihomogeneous (homogeneous in both variables) polynomials $g_i(u,v)
        = 0$. For $\P^n \times \A^m$, just remove the assumption that $g_i$ are
        homoegeneous in $v$.  \begin{proof} Consider the image under the Segre
            embedding. Then we see that $Z = (F_{\alpha}(w_{ij}) = 0)$ os the
            same as $F_{\alpha}(x_i,y_j) = 0$ homogeneous in $X$ and $Y$ of the
            same degree. If the $g_i$ are not of the same degree in $u,v$
        ($s>t$), we can multiply by $v_j^{s-t}$ to obtain equivalent equations.
    \end{proof} \end{thm}

    \subsection{Lecture 11 (Mar 5)} We continue our discussion of dimension
    (from Valentine's Day). Let $X \subset \P^n$ be an irreducible
    quasi-projective variety. Then we define the local ring $\mathcal{O}_X =
    \{f \in k(\P^n) \mid f \text{ is regular at $x \in X$ }\}$. Then we can
    define $k(X) = \mathcal{O}_X/m_X$ where $m_X$ is the maximal ideal of
    functions that vanish on $X$.

    Then if $U \subset X$ is open, $k(U) = k(X)$. Thus we can assume $X$ is
    affine. Then we see that $X \subset \A^n \subset \P^n$ is the vanishing $X
    = V(\mathfrak{p})$ and $\mathcal{O}_X = k[x_!, \ldots,
    x_n]_{\mathfrak{p}}$. Then $m_X = \mathfrak{p}R_{\mathfrak{p}}$, and this
    agrees with the old definition by a ``qualifying exam problem.''

    Then, if we define $\mathrm{dim} X \coloneqq \mathrm{tr.deg} k(X)$, it is
    easy to see that $\mathrm{dim} X = \mathrm{dim} U$ for an open subset $U
    \subset X$.

    \begin{thm} If $X$ is affine and irreducible, then $\mathrm{dim} X =
        \mathrm{Kr.dim} k[X]$.  \begin{proof} We use Noether normalization.
            Then there exists $k[x_1, \ldots, x_n] \subset k[X]$, which is an
            integral extension. First it is easy to see that $\mathrm{dim}X =
            \mathrm{dim}\A^n = n$ because the field extension is algebraic.

            To see that the Krull dimensions are equal, any chain $q_0 \subset
            q_1 \subsetneq \cdots \subsetneq q_r \subsetneq k[x_1, \ldots,
            x_n]$ gives a chain $p_0 \subsetneq \cdots p_r \subsetneq k[X]$ by
            going-up. However, all $p_i$ are distinct, so their intersections
            with $k[x_1, \ldots, x_n]$ are distinct (quotient by $p_i$, then
            take the field of fractions of the smaller ring).

            Now we show that the Krull dimension of $k[x_1, \ldots, x_n]$ is
            $n$. Clearly it is at least $n$  because we have the following
            chain of prime ideals: $(0) \subset (x_1) \subset \cdots \subset
            (x_1, \ldots, x_n)$. Take a chain $p_0 \subsetneq \cdots \subsetneq
        p_s$. Then $V(p_s) \subsetneq \cdots \subsetneq V(p_0) \subsetneq
    \A^n$. We prove a lemma which implies what we want.  \end{proof} \end{thm}

    \begin{lem} Let $X \subset Y$ be irreducible affine varieties. Then
        $\mathrm{dim} X \leq \mathrm{dim} Y$. If $X \neq Y$, then the
        inequality is strict.  \begin{proof} Let $X \subsetneq Y \subsetneq
            \A^N$ and let $n = \mathrm{dim}Y$.  Then given $t_1|_Y, \ldots,
            t_N|$, any $n+1$ are algebraically independent. However, $t_1|_X,
            \ldots, t_N|_X$ generate $k[X]$ and therefore $k(X)$, so
            $\mathrm{dim}X \leq n$. 
            
            Suppose $\mathrm{dim} X = \mathrm{dim} Y = n$. Then some $n$
            coordinates $t_1|_X, \ldots, t_n|_X$ are algebraically independent
            in $k(X)$. Thus $t_1|_Y, \ldots, t_1|_Y$ are also algebraically
            independent. Choose $0 \neq u \in k[Y]$ such that $u|_X = 0$ Then
            there is a relation $a_0(t_1, \ldots, t_n)u^m + \cdots + a_m(t_1,
            \ldots, t_n)$ which vanishes along $Y$. We can also assume that
            $a_m \neq 0$ along $Y$. However, it becomes $0$ when restricted to
        $X$ because $u|_X = 0$.  \end{proof} \end{lem}

    We use the Krull Principal Ideal Theorem: \begin{thm}[Krull Principal
    Ideal] If $X$ is affine, irreducible, and $f \in k[X]$, then all
irreducible components of $V(f) \subset X$ have codimension $1$.  \end{thm}

    \begin{cor} In $\A^n$, irreducible hypersurfaces correspond exactly to
        irreducible polynomials $f \in k[x_1, \ldots, x_n]$.  \begin{proof} Let
            $f \in k[x_1, \ldots, x_n]$ be irreducible. Then because $k[x_1,
            \ldots, x_n]$ is a UFD, $(f)$ is prime, so $V(f)$ is irreducible
            and has the correct dimension by PIT.

            In the other direction, let $X \subset \A^n$ be an irreducible
            hypersurface. Then choose $f \in k[x_1, \ldots, x_n]$ such that
            $f|_X = 0$. Factor $f = f_1\cdots f_r$ into irreducibles. Then $X
            \subset V(f_i)$ because $V(f_i)$ are irreducible hypersurfaces.
        Because they have the same dimension, they are equal.  \end{proof}
    \end{cor}

    \begin{cor} Let $X \subset \P^n$ be projective and $F \in k[x_0, \ldots,
        x_n]$ be homogeneous of positive degree $d$. Then $X \subset V(F)$ is
        non-empty and all irreducible components have codimension $1$.
        \begin{proof} Work in the charts. Then use the PIT on the irreducible
            affine components. To show that the intersections in the affine
            charts are nonempty, pass to the affine cone. By the PIT, the
            intersection has codimension $1$ and is non-empty because it
            contains the origin.  Then the same is true for $X$.  \end{proof}
        \end{cor}

    \begin{rmk} This implies that if $X,Y \subset \P^2$ are projective curves,
    $X \cap Y$ are nonempty. This implies that every irreducible curve of
degree at least $3$ has inflection points.  \end{rmk}

    \begin{thm} Let $f:X \to Y$ be a regular surjective map of irreducible
        varieties.  Suppose $\mathrm{dim}X = n, \mathrm{dim}Y = m$. Then:
        \begin{enumerate} \item $\mathrm{dim}\ F \geq n-m$ for every component
            $F$ of every fiber $f^{-1}(y)$.  \item There exists $U \subset Y$
    open, nonempty such that $\mathrm{dim}\ F = n-m$ for all $y \in U$.  \item
    Sets $Y_k = \{y \in Y \mid \mathrm{dim} f^{-1}(y) \geq k\}$ are closed.
    \end{enumerate} \begin{proof} To prove the first part, take $y \in Y$. Then
    we can take. We can assume $X,Y$ are affine by taking affine charts. Choose
    $f_1 \neq 0$ such that $f_1(y) = 0$. Then $\mathrm{dim}(f_1=0) = m-1$ by
    the PIT. Choose $f_2 |_D \neq 0$ ($f_2(y) = 0$) for all irreducible
    components $D$ of $V(f_1)$. Then $\mathrm{dim}(f_1=f_2=0) = m-2$.  Then $y
    \in (f_1 = \cdots = f_m = 0)$, which is a finite set of points. Now we pass
    from $Y$ to $D(F) \subset Y$ where $F(\widetilde{y}) = 0$ for all $y \in
    (f_1 = \cdots = f_m = 0) \setminus \{y\}$. With this new $Y$, $y = (f_1, =
    \cdots = f_m = 0)$ where $m = \mathrm{dim}Y$, we see that $f^{-1}(y) =
    V(f^*(f_1), \ldots, f^*(f_m))$. By the PIT, the dimension is at least
    $n-m$.  \end{proof} \end{thm}

    \subsection{Lecture 12 (Mar 7)}
    
    Last time we discussed Krull's Principal Ideal Theorem and began the proof
    of the theorem on dimension of fibers.

    \begin{cor} Let $X$ be irreducible of dimension $n$ and $f_1, \ldots, f_r
        \in k[X]$. Then every irreducible component of $V(f_1, \ldots, f_r)
        \subset X$ has dimension at least $n - r$.  \begin{proof}[Sketch] We
            induct on $r$. If $r = 1$, we use the P.I.T. The rest is left as an
        exercise.  \end{proof} \end{cor}
    
    \begin{proof}[Conclusion of proof of Theorem 121] To prove the second part,
        we do something similar to the below theorem.  We may assume $X,Y$ are
        affine and $f:X \to Y$ dominant. Then we can decompose $f^*:k[Y]
        \hookrightarrow k[Y][z_1, \ldots, z_r] \hookrightarrow k[X]$. This
        corresponds to a map $Y \overset{\pi_1}{\gets} Y \times \A^r
        \overset{g}{\gets} X$. Then $f^{-1}(y) = g^{-1}(Y \times \A^r)$.

        We will show that there exists $D(F) \subset Y$ such that for all $y
        \in D(F)$, $p_1\vert_{f^{-1}(y)}, \ldots, p_{\ell}\vert_{f^{-1}(y)}$
        are algebraic over $z_1\vert_{f^{-1}(y)}, \ldots,
        z_r\vert_{f^{-1}(y)}$, so every component of $f^{-1}(y)$ has dimension
        at most $r$. By part $1$, the dimension is at least $r$, so it must
        equal $r$.

        To prove the claim, write the equations of algebraic dependence
        $F_i(p_i, z_1, \ldots, z_r, q_1, \ldots, q_s) = 0$ with $p_i$ appearing
        in the polynomial. Then restrict to a specific $y$ and show that not
        all coefficients become zero (which is because there is an open set
        where the product of all coefficients does not vanish).

        Now we prove the third part, we induct on the dimension of $Y$. If $Y$
        is a point, there is nothing to prove. Also, by part ($1$), $Y_{n-m} =
        Y$ is closed. By part ($2$), there exists $U \subset Y$ open where
        $\mathrm{dim}\ f^{-1}(y) = n-m$ for all $y \in U$. Then $Y_j$ for $j >
    n-m$ is contained in $f^{-1}(Y \setminus U) = Y'$. Now consider $f$
restricted to $f^{-1}(Y')$. Because $\mathrm{dim}\ Y' < \mathrm{dim}\ Y$, $Y_j$
is closed.  \end{proof}

    \begin{thm} Let $f: X \to Y$ be a regular map of quasi-projective
        varieties. Then $f(X)$ contains an open subset $U \subset
        \overline{f(X)}$.  \begin{proof} We may assume that $f$ is dominant and
            that $Y$ is irreducible by considering the irreducible components
            of $Y$. Considering the irreducible components of $X$, we may
            assume that $X$ is irreducible. Passing to an affine chart $U
            \subset Y$ and its preimage, we may assume that $Y$ is affine and
            irreducible. Take any affine chart in the preimage of $U$ and we
            may assume that $X$ is affine. 
            
            Thus we have $X \overset{X}{\to} Y$ and $f^*: k[Y] \hookrightarrow
            k[X]$ a map of domains. Then we can decompose $f^*:k[Y]
            \hookrightarrow k[Y][z_1, \ldots, z_r] \hookrightarrow k[X]$. This
            corresponds to a map $Y \overset{\pi_1}{\gets} Y \times \A^r
            \overset{g}{\gets} X$. If $\alpha$ is integral, then $g$ (and thus
            $f$) is surjective. 
            
            We use a trick that there exists a principal affine $D(F) \subset Y
            \times \A^r$ such that $g\vert_{g^{-1}(D(F))} \to D(F)$ is finite.
            Let $p_1, \ldots, p_{\ell} \in k[X]$ be generators of the algebra.
            Then they are algebraic over $k[Y][z_1, \ldots, z_r]$ so we can
            write $a_0p_i^{m_i} + a_1p_i^{m_i-1} + \cdots + a_{m_i} = 0$, where
            $a_i \in k[Y][z_1, \ldots, z_r]$. If we invert $F =
            a_0^1a_0^2\cdots a_0^{\ell}$, then $p_i$ are all integral over
            $k[Y][z_1 \ldots, z_r][1/F] = k[D(F)]$.

            Then the image of $g$ contains $D(F)$ because finite maps are
            surjective. Then $D(F) \hookrightarrow Y \times \A^r \to Y$. We
            find $U \subset Y$ such that $U \subset \pi_1(D(F))$. Note that $F
            = \sum b_{i_1 \ldots i_r} z_1^{i_1} \cdots z_r^{i_r}$. Define $U =
            D(b_{i_1 \ldots i_r})$ for some coefficient. If $y \in U$, then
            $F(y, z_1, \ldots, z_r) \neq 0$ for some choice of $y, z_1, \ldots,
            z_r$. Thus $(y, z_1, \ldots, z_r) \in D(F)$.  \end{proof} \end{thm}

    \begin{thm} Let $f:X \to Y$ be a regular map between projective varieties
        such that $f(X) = Y$ and suppose $Y$ is irreducible. Suppose
        $f^{-1}(y)$ is irreducible of the same dimension ($\mathrm{dim}\ X -
        \mathrm{dim}\ Y$). Then $X$ is irreducible.  \begin{proof} Decompose $X
            = \cup X_i$. Then $f(X_1) = \cdots = f(X_s) = Y$ and $f(X_i)
            \subsetneq Y$ for all $i > s$. Define $f_i = f \vert_{x_i}$. Then
            there exists $U_i \subset Y$ open such that $f^{-1}(y)$ has
            smallest dimension $n_i$. Set $U = \cap_{i=1}^s U_i \setminus
            \cup_{j > s} f(x_j)$. Fix $y_0 \in U$. Then $f^{-1}(y_0) \subset
            \cup f_i^{-1}(y_0)$. WLOG we see that $f^{-1}(y_0) =
        f_1^{-1}(y_0)$, which implies $n = n_1$. Then for $y \in Y$, we see
    that $f_1^{-1}(y) \subset f_1^{-1}(y)$. Because $Y$ is irreducible, $X =
X_1$.  \end{proof} \end{thm}

    Jenia apologizes that this is too abstract and wants to calculate some
    actual examples.

    \begin{thm} Every cubic surface contains a line.  \begin{proof} We
        postulate the existence of the Grassmanian. One property of
        $\mathrm{Gr}(k.n)$ is that the incidence variety $\mathcal{L} = \{p \in
        \P^{n-1}, L \in \mathrm{Gr}(k,n) \mid p \in L\} \subset \P^{n-1} \times
        \mathrm{Gr}(k,n)$ is closed. It is easy to see that $\mathcal{L}$ is
        irreducible because $p_2: \mathcal{L} \to \mathrm{Gr}(k,n)$ and all
        fibers are $\P^{k-1}$.

            Define $\mathrm{Cub} = \P^{19} = \P(\mathrm{Sym}^3(k^4))$ and $W =
            \{p \in \P^3, S \in \mathrm{Cub} \mid p \in S\}$. Clearly $W$ is a
            projective variety. Define $Z = \{L \in \mathrm{Gr}(2,4), S \in
            \mathrm{Cub} \mid L \subset S\} \subset \mathrm{Gr}(2,4) \times
            \mathrm{Cub}$. This is also a closed set (construct a map such that
            $Z$ is the locus of points with $1$-dimensional fibers). Therefore
            $Z$ is projective.

            Now consider $\pi_2(Z) \subset \mathrm{Cub}$. Becuase the
            Grassmanian is projective, it is proper. Then $\pi_2(Z)$ is closed.
            This is the set of cubic surfaces containing at least one line. If
            the image is not all cubic surfaces, we may use the theorem on the
            dimension of the fibers. We see that each cubic contains a
            positive-dimensional set of lines. But the Clebsch cubic surface
            contains only $27$ lines, which is a contradiction.  \end{proof}
        \end{thm}

    \section{Local Properties}
    
    \subsection{Lecture 13 (Mar 19)} We begin our discussion of local
    properties. Let $p \in X$ be a point on a quasi-projective variety. We can
    assume $X$ is affine and $k[X] = A$. Then $m_p \subset A$ is a maximal
    ideal. The local ring is $\mathcal{O}_p = A_{m_p}$, which is independent of
    the affine chart. If $X$ is irreducible, then $\mathcal{O}_p \subset k(X)$.
    If $X$ is not irreducible, then $A$ has zero-divisors and localization is
    harder to define.

    \subsubsection{Tangent space}% Let $X \subset \A^n$. Then $X = V(I), I =
    (F_1, \ldots, F_s), p = (x_1^0, \ldots, x_n^0)$.

    \begin{defn}[Tangent Line] We say that a line $L$ passing through $p$ is
    tangent to $X$ at $p$ if the multiplicity of $L \cap X$ is at least $2$.
\end{defn}

    \begin{defn}[Tangent space, ambient version] The tangent space $T_pX$ is
    the locus of all lines tangent to $X$ at $p$.  \end{defn}

    \begin{defn}[Multiplicity] Let $L$ be defined parametrically. Then $I|_L
    \subset k[t]$ is a principal ideal $(f)$. Then $f = t^mu$, where $u(0) =
0$. The multiplicity of intersection is $m$.  \end{defn}

    \begin{exm} Let $X = (y=x^2)$ the parabola. Then if $L = \begin{cases} x =
    x_0+at \\ y = y_0+at \end{cases}$ we see that $f(t) = (b-2ax_0)t + O(t^2)$.
Thus the line is tangent if $b = 2ax_0$, or $b/a = 2x_0$.  \end{exm}

    \begin{exm} Let $X = (y^2=x^3)$ and $p = x_0$. Then $f(t) = (bt)^2 - (at)^3
    = t^2(b^2-a^3t)$, so every line is a tangent line through the origin, and
the tangent space is $\A^2$.  \end{exm}

    Again let $I = (F_1, \ldots, F_s)$ and $p = (x_1^0, \ldots, x_n^0)$. Then
    \[F_{\ell}(x) = F_{\ell}(p) + \sum \frac{\partial F_{\ell}}{\partial
    x_i}(p)(x_i - x_i^0) + \mathrm{h.o.t}, \] where $\sum \frac{\partial
F_{\ell}}{\partial x_i}(p)(x_i - x_i^0)$ is the linearization of $d_pF_{\ell}$.
Then $F_{\ell}|_L = \frac{\partial F_{\ell}}{\partial x_i}(p)a_it +
\mathrm{h.o.t}$. Thus $L$ is a tangent line if and only if \[\sum_{i=1}^n
\frac{\partial F_{\ell}}{\partial x_i}(p) a_i = 0 \] for all $\ell = 1, \ldots,
s$. Thus the defining equations for $T_pX$ are \[\sum \frac{\partial
F_{\ell}}{\partial x_i}(p)(x_i - x_i^0), \ell = 1, \ldots, s.\]

    Note that $T_pX$ is an affine subspace of $\A^n$. One can also view the
    linear subspace parallel to $T_pX$ as the tangent space.

    \begin{thm} There is a canonical isomorphism $T_p^* \simeq m_p/m_p^2$,
        where either we take $m_p \subset A = k[X]$ or $m_p \subset
        \mathcal{O}_p$. This is known as the Zariski cotangent space.
        \begin{proof} Let $g \in m_p$. Then $g = G+I$ for some $G \in k[x_1,
            \ldots, x_n]$. Then $d_pG = \sum \frac{\partial G}{\partial
            x_i}(x_i - x_i^0)$, but for all $F \in I$, $d_pF|_{T_pX} = 0$.
            Therefore we can define $d_pg = d_pG|_{T_pX}$. This gives a map
            $d_p: m_p \to T_p^*$. Then $d_p$ is clearly $k$-linear and
            surjective (linear function is its own differential). Then $m_p^2
            \subset \mathrm{Ker}d_p$ because $m_p^2$ is generated by
            $(x_i-x_i^0)(x_j-x_j^0)$, which vanish under the differential by
            Leibniz.

            Thus $d_p$ induces a surjective map $m_p/m_p^2 \to T_p^*$. We show
            that the kernel of $d_p$ is $m_p^2$. WLOG let $p = (0,\ldots, 0)$.
            Suppose $d_p g = 0$. Then $g = G|_X, G \in k[x_1, \ldots, x_n]$ and
            $d_pG|_{T_p} = 0$. Then $T_p = \{d_pF_{\ell} = 0 \mid \ell = 1,
            \ldots, s \}$, so $d_pG = \lambda_1 d_pF_1 + \cdots + \lambda_s d_p
            F_s$. Set $G_1 = G - \sum_{i=1}^s \lambda_i F_i$. Then $g_1|_X = g$
            and $d_p G_1 = 0$, so $G_1 \in (x_1, \ldots, x_n)^2$. Therefore
        $G_1|_X \in m_p^2$.  \end{proof} \end{thm}

    \begin{cor} $T_p$ and therefore its dimension is a local invariant of $p
    \in X$.  \end{cor}

    \begin{rmk} Suppose $X \subset \P^n$ is projective. Then $X = (F_1 = \cdots
        = F_s = 0)$. Choose $p \in X$, and suppose WLOG that $p \in \A^n_0$, so
        $p = (1:x_1^0:\cdots:x_n^0)$. Then $T_p X \subset \A_0^n$ is given by
        $\sum_{i=1}^n \frac{\partial F_{\ell}}{\partial x_i}(p)(x_i - x_i^0) =
        0$ where $\ell = 1, \ldots, s$.

        We can take $\overline{T_pX} \subset \P^n$ the projective tangent
        space. Recall Euler's formula $\sum_{i=0}^n \frac{\partial
        F_{\ell}}{\partial x_i}(p) x_i^0 = (\mathrm{deg}F_{\ell}) F_{\ell}(p) =
        0$. Therefore the projective tangent space is given by \[ \sum_{i=1}^n
        \frac{\partial F_{\ell}}{\partial x_i}(p) x_i = 0, \ell = 1, \ldots, s.
    \] \end{rmk}
    
    \begin{thm} Suppose $X$ is irreducible. Then $X^{\mathrm{sm}} = \{p \in X
        \mid \mathrm{dim}T_p X = \mathrm{dim} X \}$ is a non-empty open subset
        of ``smooth'' or ``non-singular'' points. The complement is known as
        the singular locus $X^{\mathrm{sing}}$, and for all $p \in
        X^{\mathrm{sing}}$, $\mathrm{dim}T_pX > \mathrm{dim} X$.  \begin{proof}
            We assume $X \subset \A^N$ is affine. Then define $TX = \{(a,p) \in
            \A^N \times X \mid a \in T_p X \}$. This is called the tangent
            bundle (or fiber space). Its equations are $F_1, \ldots, F_s$ the
            equations of $X$ and the equations of tangency $\sum \frac{\partial
            F_{\ell}}{\partial x}(x_1, \ldots, x_n) (y_i-x_i) = 0$. Thus $TX$
            is algebraic.

            Now the second projection $\pi_2: TX \to X$ has fiber the tangent
            space of $X$. By the theorem on the dimension of fibers,
            $X^{\mathrm{sm}} = \{p \in X \mid \mathrm{dim}T_pX = s\}$, where
            $s$ is the minimum possible, is an open subset. It remains to be
            seen that $s = n$.

            Note every irreducible algebraic variety is birational to a
            hypersurface. Also, we show that $X$ is birational to $Y$, if and
            only if there exist $U \subset X, V \subset Y$ open, nonempty such
            that $U \simeq V$.\footnote{The second claim is on the midterm.}

            Given both claims, we can assume $X \subset \A^{n+1}$ is a
            hypersurface. Then $X = (F=0)$ for some irreducible $F \in k[x_0,
            \ldots, x_{n+1}]$. Then $T_pX$ is given by $\sum_{i=0}^n
            \frac{\partial F}{\partial x_i}(p)(x_i - x_i^0) = 0$. Then the
            dimension of the tangent space is $n$ unless $\frac{\partial
            F}{\partial x_i}(p) = 0$ for all $i$. This implies that
            $\frac{\partial F}{\partial x_i} \in (F)$ and thus the partial
            derivatives are zero. In characteristic $0$, this implies $F = 0$,
            which is a contradiction, and in positive characteristic, $F$ is a
            $p$th power, so it is not irreducible. Thus we have a
        contradiction.  \end{proof} \end{thm}

    \begin{lem} Every irreducible variety is birational to a hypersurface.
        \begin{proof} We work in characteristic zero. Choose a transcendence
            basis $k(x_1, \ldots, x_n) \subset k(X)$ a finite extension. This
            extension is separable, so we use the primitive element theorem.
        Then $k(x_1, \ldots, x_n, y)$, where $y$ has minimal polynomial $F \in
    k[x_1, \ldots, x_n, y]$. Thus $k(X) \simeq k(H)$, where $H = V(F)$.
\end{proof} \end{lem}

    If $X$ is reducible, then $p \in X$ is nonsingular if $\mathrm{dim}T_p X =
    \max (\mathrm{dim} Y)$ where $Y \ni p$ is an irreducible component of $X$.
    In fact, there is a theorem that if $p$ lies on several irreducible
    components, then $p$ is singular. 

    \subsection{Lecture 14 (Mar 21)}
    
    Last time we considered local properties at a point. Today we will prove
    more theorems. Let $p$ be a non-singular point of $X$.

    \begin{thm} $\mc{O}_p \hookrightarrow k[[u_1, \ldots, u_n]]$.  \end{thm}

    \begin{cor} $\mc{O}_p$ is an integral domain. Thus $X$ has only one
    irreducible component passing through $p$.  \end{cor}

    \begin{rmk} This is an analogue of the result from complex analysis that
    holomorphic functions are analytic and the map $\mathcal{O}_U \to \C[[z]]$
is injective.  \end{rmk}

    \begin{defn} $u_1, \ldots, u_n \in m_p$ are called local parameters if
    $\{u_i\}$ is a basis of $m_p/m_p^2$. Equivalently, $d_p u_1, \ldots, d_p
u_n$ form a basis of $T_p^{\vee}$.  \end{defn}
    
    Fix $u_1, \ldots, u_p \in m_p$ a system of local parameters. 

    \begin{lem} $m_p = (u_1, \ldots, u_n)$.  \begin{proof} Let $U = (u_1,
    \ldots, u_n) \subset m_p$. Then $m_p (m_p/I) = m_p/I$, so by Nakayama's
lemma, $m_p/I = 0$.  \end{proof} \end{lem}

    \begin{defn} A formal power series $\phi = \sum_{k \geq 0} \phi_k(u_1,
    \ldots, u_n)$ is called a Taylor series of $f \in \mc{O}_p$ if $\left( f -
\sum_{k=0}^n \phi_k \right) \in m_p^{n+1}$.  \end{defn}

    \begin{lem} Every $f \in \mc{O}_p$ has a Taylor series.  \begin{proof} Note
        $f = f(p) + g_1$, where $g_1$ vanishes at $p$, so $g_1 \in m_p$. Then
        we can write $g_1 = \sum_{i=1}^n a_iu_i$ where $a_i \in \mc{O}_p$. Thus
        we can write $a_i = a_i(p) + \overline{a_i}$, where $\overline{a_i} \in
        m_p$. Then we can write $g_1 = \sum_{i=1}^n a_i(p)u_i + \sum
        \overline{a_i} u_i$. Set the second sum to be $g_2 \in m_p^2$.

            Now suppose $g_k \in m_p^k$. Thus $g_k = \sum_i h_1^i \cdots
        h_k^i$. Noting that $h_j^i = \sum a_{j,k}^i u_k$. Decompose this as
    above and then the claim follows by induction.  \end{proof} \end{lem}

    \begin{lem} A Taylor series is unique. Thus we have a morphism
    $\mathcal{O}_p \to k[[u_1, \ldots, u_n]]$.  \end{lem}

    \begin{lem} This morphism is injective.  \begin{proof} Take $f \in
        \mc{O}_p$ and suppose $f$ has Taylor series $0$. This happens if and
        only if $f \in m_p^n$ for every $n$. We know that by Krull's
        intersection theorem, if $(R,m)$ is a Noetherian local ring and $I
    \subset R$ is some ideal, then $\cap_{n \geq 1} I^n = 0$.  Therefore $f =
0$.  \end{proof} \end{lem}

    \begin{thm}[Krull's Intersection Theorem] If $(R,m)$ is a Noetherian local
    ring and $I \subset R$ is some ideal, then $\cap_{n \geq 1} I^n = 0$.
\end{thm}

    \begin{proof}[Proof of Krull's intersection Theorem] Consider the Rees ring
        $\widetilde{R} = R \oplus I \oplus I^2 \oplus \cdots$ with graded
        multiplication. Then $\widetilde{R}$ is Noetherian.  Consider $N =
        \cap_{n\geq 1}I^n$ and $\widetilde{N} = N \oplus N \oplus N \oplus
        \cdots \subset \widetilde{R}$, which is an ideal. Thus $\widetilde{N}$
        has a finite generating set. Thus all generators live in the first $k$
        components for some $k$. Then any $n \in N_{k+1}$ can be written $n =
        \sum \widetilde{r_i} n_i$ where $\widetilde{r_i} \in I \oplus I^2
        \oplus \cdots$. Then $N = I \cdot N$. $I$ is contained in a maximal
        ideal, so $N = mN$, and therefore $N=0$ by Nakayama.  \end{proof}

    \begin{rmk} This leads to the Artin-Rees lemma.  \end{rmk}

    \begin{proof}[Proof of Lemma 143] Suppose $f \in \mc{O}_p$ has $2$
        different Taylor series. Equivalently, suppose $0$ has a nontrivial
        Taylor series. Then $\phi_0 \in m_p$, so $\varphi_0 = 0$. Then $\phi_1
        = \sum a_i u_i \in m_p^2$. However, $u_1, \ldots, u_n$ are linearly
        independent mod $m_p^2$ by definition of local parameters. Thus $a_1 =
        \cdots = a_n = 0$. Thus $\phi_i = 0$.

        Now $\phi_2 = \sum a_{ij}u_iu_j \in m_p^3$. We show that $\phi_k(u_1,
        \ldots, u_n) \in m_p^{k+1}$ if and only if $\phi_k$ is a trivial
        polynomial. If $n = 1$, then $m_p = (u)$ and $\phi_k(u) = \alpha u^k
        \in (u)^{j+1}$. If $\alpha \neq 0$, then $(u)^k = (u)^{k+1}$, which
        implies $(u) = 0$ by Nakayama.

        From the proof of Noether's normalization, WLOG $\phi_k = u_1^k +
        \text{other monomials}$ We restrict to a hypersurface $Y = V(u_n)
        \subset X$. Then all irreducible components of $Y$ have codimension $1$
        by the PIT. We claim that $Y$ is non-singular at $p$ and $u_1|_Y,
        \ldots, u_{n-1}|_Y$ are local parameters of $Y$ at $p$.

        Given the claim, $\phi_k|_Y = \phi(u_1|_Y, \ldots, u_{n-1}|_Y, 0)$ is a
        nontrivial polynomial. It contains $(u_1|_Y)^k$. On the other hand,
        $\phi_k \in m_p^{k+1}$, so $\phi_k|Y \in (m_{p,Y})^{k+1}$. By
        induction, this is a contradiction.  \end{proof}

    \begin{thm} Let $u_1, \ldots, u_n \in m_p \subset \mc{O}_p$ be local
        parameters.  Shrink $X$ so that $u_1, \ldots, u_n \in k[X]$. Then $X_i
        = V(u_i)$ is nonsingular at $p$ and $\{u_j|_{x_i}\}_{j \neq i}$ are
        local parameters of $X_i$. Also, $\cap_{i=1}^n T_{p,X_i} = 0$.
        \begin{proof} Note $u_i \in I(x_i) \subset k[X]$. Then $T_p X_i = \{d
            f_p = 0 \text{ for all } f_p \in I(x_i) \} \subset L_i = \{ d_p u_i
        = 0\} \subset T_p X$. By the PIT, $\mathrm{dim}_p X_i = n-1$. However,
        $\mathrm{dim} T_pX_i \leq n-1$. Thus $x_i$ is smooth and $T_p X_i =
        L_i$. Also, $\{d_p u_j |_{L_i} \}_{j \neq i}$ is a basis of
        $L_i^{\vee}$. Therefore $\{u_j|_{X_i}\}_{j \neq i}$ are local
    parameters on $X_i$.  \end{proof} \end{thm}

    This completes the proof of Lemma 143.

    \subsubsection{Tangent Cone} Consider the cusp $X = \{y^2 = x^3\}, T_0 X =
    \A^2$. Rescale $x \to tx, y \to ty$, so the equation becomes $t^2y^2 =
    t^3x^3$, so $y^2 = tx^3$. Note that $X \simeq X_t$ for $t \neq 0$. Then we
    set $X_0 = \{y^2 = 0\}$.

    Let $0 \in X \subset \A^n$. Then $X = V(I)$. For all $f \in I$ we can write
    $f = \mathrm{in}(f) + \mathrm{h.o.t.}$. Then define $I_0 = \mathrm{in}(I) =
    (\mathrm{in}(f) \mid f \in I)$. For example, $\mathrm{in}(y^2-x^3) =
    (y^2)$. 

    \begin{defn} The Tangent cone is defined to be $V(I_0)$, at least if $I_0$
    is a radical ideal.  \end{defn}

    \begin{thm} The tangent cone has the same dimension as the original
    variety. There is a family of varieties $X_t \simeq X$ such that ``$\lim_{t
\to 0} X_t$''$= X_0$.  \end{thm}

    More precisely, there exists an affine variety $\mc{X} \subset \A^n \times
    \A^1_t$ such that $\pi_2^{-1}(t) \simeq X$ for $t \neq 0$, $\pi_2(1) = X$,
    and $\pi_2^{-1}(0) = X_0$.

    \subsection{Lecture 15 (Mar 26)} We can consider lines (or curves on a
    variety $X$ as maps $\P^1 \dashrightarrow X$). Also, we can consider maps
    $f: X \dashrightarrow \P^1$. Then $Y = f^{-1}(a) \subset X$ is a
    hypersurface.

    \begin{defn}[Locally Principal] A hypersurface $Y \subset X$ is called
    locally principal if it is locally given by $1$ equation.  \end{defn}

    \begin{defn}[Local Equations] $f_1, \ldots, f_r \in \mc{O}_p$ are called
        local equations of a subvariety $Y \subset X$ if there exists and
        affine neighborhood $U \subset X$ of $p$ such that $f_1, \ldots, f_r
        \in k[U]$ and $I(Y) = (f_1, \ldots, f_r)$.  \end{defn}

    \begin{lem} $f_1, \ldots, f_r$ are local equiations of $Y$ at $p$ if and
        only if $\mc{L}_{Y,p} \subset \mc{O}_p$ is generated by $f_1, \ldots,
        f_r$.  Here $\mc{L}_{Y,p} = \{f \in \mc{O}_p \mid f|_{Y \cap U} = 0
        \text{ for some neighborhood } p \in U \}$. Equivalently, if $X$ is
        already affine, $I(Y) \subset k(X)$ and $\mc{L}_{Y,p} = I(Y)_{m_p}$.
        \begin{proof} If $f_1, \ldots, f_r \in \mc{O}_p$ are local equations,
            then after shrinking $X$ to $U$, we have $I(Y) = (f_1, \ldots, f_r)
            \subset k[X]$. Then $\mathcal{L}_{Y,p} = (f_1, \ldots, f_r)$
            because localization preserves generators.

            Now suppose that $\mc{L}_{Y,p} = (f_1, \ldots, f_r) \subset
            \mc{O}_p$. By shrinking, we can assume that $f_1, \ldots, f_r \in
            k[X]$ and that $X$ is affine. We know that $I(Y) = (g_1, \ldots,
            g_s) \subset K[X]$. We can write $g_i = \sum_{j=1}^r h_{ij}f_j$ in
            $\mc{O}_p$. We can shrink $X$: Choose a principal open set $U =
            D(w) \subset X$ such that $h_{ij} \in k[U]$. We claim that in $U$,
            $I(Y \cap U) = (g_1, \ldots, g_s) \subset k[U]= k[X][1/w]$ is also
            generated by $f_1, \ldots, f_r$. We know $(g_1, \ldots, g_s)
            \subset (f_1, \ldots, f_r).$ Then $f_i|_Y = 0$, so $(f_1, \ldots,
            f_r) \subset (g_1, \ldots, g_s)$.  \end{proof} \end{lem}
    
    \begin{thm}[Criterion for Smoothness] Let $X$ be a variety of dimension
        $n$. Then suppose $p \in Y \subset X$ and $Y$ is locally principal at
        $p$, as in $\mc{L}_{Y,p} = (g) \subset \mc{O}_p$. If $Y$ is nonsingular
        at $p$, then so is $X$.  \begin{proof} Shrink $X$ to be affine. Then
            $I(Y) = (g) \subset k[X]$. We see that $I(X) = (F_1, \ldots, F_s)$
            and $\overline{I(Y)} = (F_1, \ldots, F_s, G)$. Recall that $T_pX =
            \{d_pF_1, \cdots, f_pF_s = 0 \}$ with dimension at least $n$. Then
            $T_pY = \{d_pF_1 = \cdots = d_pF_s = d_pG = 0\}$. Then we see that
            $T_pY$ has dimension $n-1$, so $T_pX$ must have dimension $n$, so
            $X$ is nonsingular at $p$.  \end{proof} \end{thm}
    
    \begin{rmk} It is not enough to assume that $Y = V(g)$. For example, take
        $X = \{z^2=xy\}$ and $Y = V(x)$. $Y$ is nonsingular at the origin but
        $X$ is singular. In fact, this implies that $Y \subset X$ is not
        locally principal, so $I(Y) = (x,z)$ needs at least $2$ generators.
    \end{rmk}

    \begin{thm} Suppose $X$ is nonsingular at $p$ and suppose $p \in Y \subset
        X$ is an irreducible hypersurface. Then $Y$ is locally principal at
        $p$.\footnote{This does not imply that $I(Y) \subset K[X]$ is
            principal. For an example, take $p \in E = \{y^2 = x^3+ax+b\}$
            where the discriminant in nonzero. Then $m_p \subset \mc{O}_p =
            (t)$ is principal, but $I(p) \subset k[E]$ is not principal. If
            $I(p) = (t) \subset k[E]$, then we have a map $f:E \to \A^1$ where
            $f^{-1}(0) = p$. This induces a map $f:\overline{E} \to \P^1$ such
    that $f(\infty) = \infty$. This implies $f$ is birational, which is a
    contradiction.} \end{thm}

    \begin{thm}[Equivalent Theorem] Let $p \in X$ be a nonsingular point. Then
        $\mc{O}_p$ is a UFD.  \begin{proof}[Proof of Equivalence] Suppose $p
            \in Y \subset X$ where $X$ is affine. Then consider $f \in
            \mc{L}_{Y,p}$. This is a UFD, so we can factor $f = f_1 \cdots
            f_r$. We can assume that $f \in k[X]$ and $f|_Y = 0$. If $Y$ is
            irreducible, then $f_1$ vanishes on $Y$ (up to renumbering). We
            will show that $\mc{L}_{Y,p} = (f_1=g)$.
            
            We know that $Y \subset V(g)$. We will sow that $Y$ is the only
            irreducible component of $V(g)$ passing through $p$. Suppose $Y'$
            is another component passing through $p$. Then there exists $h|_Y =
            h'|_{Y'} = 0$, and $hh'|_{V(g)} = 0$. Thus $g | (hh')^k$ for some
            $k$. Thus $g$ divides $h$ or $h'$ in a UFD, which leads to a
            contradiction. This implies that $Y = V(g)$ is irreducible. Take $s
            \in I(Y)$. Then $g|s^k$ for some $k$. Because $g$ is prime, it
            divides $s$. Thus $s \in (g)$. In fact, $I(Y) = (g)$.

            In the other direction, note that $\mc{O}_p$ is Noetherian, which
            implies existence of prime factorization. We prove Euclid's Lemma.
            Let $g$ be irreducible. Let $Y = V(g)$. Then $Y$ is irreducible at
            $p$. Take $Y_1 \subset Y$ an irreducible component. By Krull, $Y_1$
            has codimension $1$. Thus $Y_1$ is locally principal, so $Y_1 =
            V(g_1)$ after shrinking. We also know $I(Y_1) = (g_1)$. Then
            $g|_{Y_1} = 0$, so $g \in (g_1)$. Thus $g = g_1h_1 \in \mc{O}_p$,
        so $h_1$ is a unit. Then $(g) = (g_1)$ and thus $Y = Y_1$. Thus $Y$ is
    irreducible and $g$ is prime.  \end{proof}

        \begin{proof}[Proof of Theorem 156.] The idea is to use the injection
            $\mc{O}_p \subset k[[T_1, \ldots, T_n]]$ where $p \in X$
            nonsingular and $n = \mathrm{dim} X$. We check that Euclid's Lemma
            holds. We know that this holds in $k[[T_1, \ldots, T_n]]$. Then if
            $a,c \in \mc{O}_p$ with $a|c$ in $\widehat{\mc{O}}_p$. We also
            check that if $a,b$ are coprime in $\mc{O}_p$ then they are coprime
            in $\widehat{\mc{O}}_p$. We show that if $I \subset \mc{O}_p$, then
            $(I\widehat{\mc{O}}_p) \cap \mc{O}_p = I$.

            To prove the claim, let $I \subset \mc{O}_p$ where $I = (f_1,
            \ldots, f_s)$ and $x \in I (widehat{\mc{O}}_p) \cap \mc{O}_p$. Then
            $x = \sum f_i \alpha_i$, $\alpha_i \in \widehat{\mc{O}}_p$. Then
            for all $n \geq 1$ we see that $\alpha_i = a_i^{(n)} + \xi_i^{(n)}$
            where $a_i \in \mc{O}_p$ and $\xi_i \in \widehat{m}_p^n$. Thus $x =
            \sum a_i^{(n)} f_i + \sum \xi_i^{(n)} f_i$. We know $\xi = x-a \in
            \widehat{m}_p^n \cap \mc{O}_p$, so $\xi \in m_p^n$. Therefore $x
            \in I + m_p^n$ for all $n$, and thus $x \in I$ by Krull's
            Intersection Theorem.

            Given the claim, we see that $c \in (a)\widehat{\mc{O}}_p$.
            However, $c \in \mc{O}_p$, so $c \in (a)$. Now we show that gcd is
            preserved by the embedding. Suppose $a = \gamma \alpha, b =
            \gamma\beta$ where $(\alpha, \beta) \in \widehat{\mc{O}}_p$. Choose
            $n$ such that $\alpha, \beta \notin \widehat{m}_p^n$. Then $\alpha
            = x_n+u_n, \beta = y_n+v_n$ where $x_n, y_n \in \mc{O}_p$ and $u_n,
            v_n \in \widehat{m}_p^n$. Because $a \beta - b \alpha = 0$, we see
            that $ay_n - bx_n = a(\beta - v_n) - b (\alpha - u_n) = -av_n +
            bu_n \in (a,b)\widehat{m}_p^n$. By the claim, $ay_n - bx_n \in
            (a,b)m_p^n$, so $ay_n - bx_n = a_n + bs_n$ where $t_n,s_n \in
            m_p^n$. Bt algebraic manipulation, $\alpha(y_n-t_n) =
            \beta(x_n+s_n)$. Because $\alpha, \beta$ are coprime, then $\alpha
            \mid x_n+s_n$, so $x_n+s_n = \alpha \lambda$. Thus $\lambda = 1 +
            \mathrm{h.o.t.}$ is a unit. Therefore $x_n+s_n \mid \alpha$, so
            $x_n + s_n \mid a$ in $\mc{O}_p$. Then $a = (x_n+s_n)h$ and
            therefore $h(y_n-t_n) = b$, so $b = h(y_n-t_n)$. Therefore $h$ is a
        unit. In the power series ring, $\alpha \mid a$ and $a \mid \alpha$, so
    $\gamma$ is a unit.  \end{proof} \end{thm}

    \subsection{Lecture 16 (Mar 28)} We prove that if $p \in X$ is smooth, then
    $\mc{O}_p$ is a UFD. We will finish the proof in last time's section.
    
    \begin{cor} Let $f: X \dashrightarrow Y$ is a rational map with $X$
        nonsingular, $Y$ projective. Then the indeterminancy locus has
        codimension at least $2$. In particular, if $X$ is nonsingular, $f$ is
        regular.  \begin{proof} We work near $p \in X$ and assume $Y = \P^n$.
            Then $f = (f_0: \cdots : f_n)$ where $f_i \in k(X)$. We can assume
            that $f_i \in \mc{O}_p$ with no common factor. We show that $V(f_0,
            \ldots, f_n) \subset X$ has codimension $2$ near $p$.

            Suppose $Y \subset V(f_0, \ldots, f_n)$ and $Y$ has codimension
        $1$. Then $\mc{L}_{Y,p} = (h)$ because $\mc{O}_p$ is a UFD. Then
    $f_i|_Y = g_ih$. This contradicts the fact that the $f_i$ have no common
factor.  \end{proof} \end{cor}
    
    \subsubsection{Blowups} \begin{exm} Consider $\xi = [1:0:\cdots:0] \in
        \P^n$ and take the map $\pi: \P^n \dashrightarrow \P^{n-1}$ given by
        $[x_0: \cdots :x_n] \mapsto [x_1:\cdots:x_n]$. We know $\pi$ is regular
        at $\P^n \setminus \xi$. We want to resolve $\pi$:

        \begin{center} \begin{tikzcd} & \mathrm{Bl}_{\xi}\P^n
        \arrow{dl}{\sigma} \arrow{dr} & \\ \P^n \arrow[dashrightarrow]{rr}{\pi}
                                      & & \P^{n-1} \end{tikzcd} \end{center}

        Consider the map $\P^n \times \P^{n-1} \supset \mathrm{Bl}_{\xi}\P^n =
        \overline{\Gamma_{\pi}}$, where $\Gamma_{\pi} = \{(x,y) \mid y = \pi(x)
        \}$. By construction, $\sigma = \pi_1$ and the right arrow is $\pi_2$.
        We also know that $\mathrm{Bl}_{\xi}\P^n \setminus \sigma^{-1}(\xi)
        \simeq \P^n \setminus \xi$. Also, the blowup is irreducible. 
        
        Fix $[y_1: \cdots: y_n] \in \P^{n-1}$. We know $\pi^{-1}(y) \supset
        \{[1:t_{y_1: \cdots : ty_n}]\}$. Then the closure contains $\xi, [y_1:
        \cdots : y_n]$. Therefore $\sigma^{-1}(\xi) = \P^{n-1}$ and is called
        the exceptional divisor of the blowup.

        Now we find equations for the blowup. We know that $\P^n \times
        \P^{n-1}$ is given by $x_iy_j = x_jy_i$. Then if $x_1 = \cdots = x_n =
        0$ ($\xi$), we get $\P^{n-1}$. If $x_1 \neq 0$, then $[x_1:\cdot : x_n]
        \sim [y_1: \cdots : y_n]$ and we can use equations for $\Gamma_{\pi}$.
        \end{exm}

    \subsection{Lecture 17 (Apr 02)} We finish the discussion of blowups from
    last time. Let $f:X \to Y$ be a birational regular surjective morphism.
    Then $f$ restricts to an isomorphism $U \simeq V$ of open subsets. Using
    the theorem on dimension of the fibers, there exists an open $W \subset Y$
    such that $f$ has finite fibers over points in $W$. Away from $W$, every
    component of every fiber is positive dimensional.

    \begin{defn}[Exceptional Locus] $\mathrm{Exc}(f) = f^{-1}(Y \setminus W)$
    is the exceptional locus of $f$.  \end{defn}

    \begin{thm} Suppose $f:X \to Y$ is a surjective birational regular map.
        Suppose $f(x) = y$ and assume that $Y$ is smooth at $y$ (more
        generally, $Y$ is locally factorial at $y$, meaning that $\mc{O}_y$ is
        a UFD). Assume that $g = f^{-1}$ is not defined at $y$. Then there
        exists a subvariety $Z \subset X$ of codimension $1$ passing through
    $x$ such that $Z \subset \mathrm{Exc}(f)$.  \end{thm}
    
    \begin{cor} If $f:X \to Y$ is birational, surjective, regular, and
        $\mathrm{codim}(\mathrm{Exc}(f)) \geq 2$, then $Y$ is singular and in
        fact not even locally factorial.  \begin{proof} Pass to an affine
            neighborhood of $x$ at $X$, and assume that $X \subset \A^n$ with
            $g = f^{-1}$. Then denote $g_i = g^*(f_i) \in k(Y)$. Then $g$ is
            not defined at $y \in Y$, so one of the $g_i$, say $g_1$ is not
            regular at $y \in Y$. Thus $g_1 \notin \mc{O}_y$.
            
            If $\mc{O}_y$ is a UFD we can write $g_1 = u/v$ where $u,v$ are
            coprime and $v(y) = 0$. We know that $t_1$ is a regular function on
            $X$, where $t_1 = f^*(g^*(t_1)) = f^*(g_1) =
            \frac{f^*(u)}{f^*(v)}$.

            Define $Z = V(f^* v) \subset X$, which is a hypersurface. We know
            $f^*(u)|_Z = (t_1 f^* v)|_Z = 0$. Thus $f(Z) \subset V(u) \cap
            V(v)$. We know $V(u,v) \subset Y$ has codimension $2$ at $y$
            becuase the local ring at $y$ is a UFD.  \end{proof} \end{cor}

    We observe that the blowup is a local construction. Take $\A^n_{x_1,
    \ldots, x_n}$ and let $\xi = 0$. Then $\mathrm{Bl}_{\xi} =
    \overline{\Gamma}_{\pi} \subset \A^n \times \P^{n-1} \subset
    \mathrm{Bl}_{\xi} \P^n$. To find equations, we dehomogeneize to find
    $x_iy_j=x_j$ if we take $y_i = 1$. Also, other equations are $x_jy_k =
    x_ky_j$. Also $\mathrm{Bl}_{\xi}\A^n = \bigcup_{i=1}^n U_i$ where $u_i =
    \{y_i \neq 0\}$. Thus $U_i \simeq \A^n_{y_1, \ldots, x_i, \ldots, y_n}$ and
    there is a map $U_i \to \A^n$ given by $(y_1, \ldots, x_i, \ldots y_n)
    \mapsto (x_iy_1, \ldots, x_i, x_iy_n)$.

    \begin{cor} The blowup of $\A^n$ is smooth, covered by $\A^n$'s.  \end{cor}

    \begin{defn}[Proper Transform] Take $X \subset \A^n$. Then $\sigma^{-1}(X)
    = E \cup \overline{\sigma^{-1}(X \setminus \xi)}$. the components are $E
\simeq \P^n$ and $\widetilde{X}$, which is called a proper transform of $X$.
\end{defn}

    \begin{exm} Consider $X = (x_2=tx_1) \subset \A^2$. Then consider $U_1
        \simeq \A^2_{x_1,y_2}$ and $U_2 = \A^2_{y_1,y_2}$/ Them if $\alpha:U_1
        \to \A^2$ is the covering map, $\alpha^{-1}(x) = x_1y_2-tx_1=0$, so
        either $x_1=0$ or $y_2=t$. On $U_{21}$, we see $x_1 \mapsto x_2y_1, y_2
    \mapsto y_2$, so $\beta^{-1}(x) = x_2-tx_2y_1 = 0$, so either $x_2 = 0$ or
$1-ty_1 = 0$.  \end{exm}

    \begin{exm} Consider $X  =(x_2^2-x_1^3=0)$ in $U_1$. Then $\alpha^{-1}(X) =
    (x_1y_2)^2-x_1^3 = 0$, so either $x_1=0$ ($E$) or $x_1=y_2^2$
($\widetilde{X}$).  \end{exm}

    \subsubsection{Local Blowup II} Let $\xi \in X$ be a smooth point and the
    dimension of $X$ be $n$. We find the blowup of $X$ at $\xi$.  \begin{defn}
        Choose local parameters $u_1, \ldots, u_n \in m_{\xi}$. Near $\xi$, a
        rational map $X \dashrightarrow \P^n$ given by $x \mapsto [u_1(x):
        \cdots : u_n(x)]$ is not regular only at $\xi$. Then $\mathrm{Bl}_{\xi}
        X = \overline{\Gamma}_{\pi} \subset X \times \P^1$ which is
        irreducible, smooth and the map $\sigma: \mathrm{Bl}_{\xi}(X) \to X$ is
    an isomorphism away from $\xi$ with equations as before.  \end{defn}

    \begin{thm} Let $\zeta \in X \subset \A^n$ with $X$ smooth. Then
    $\widetilde{X} \subset \mathrm{Bl}_{\xi} \A^n$ is isomorphic to
$\mathrm{Bl}_{\xi} X$.  \end{thm}

    \begin{thm} $\mathrm{Bl}_{\xi} X$ is independent of the choice of local
    parameters.  For two choices of local parameters, there exists a unique
isomorphism $\psi$ that commutes with the $\sigma$.  \end{thm}

    \subsection{Lecture 18 (Apr 04)} Recall that local parameters are
    essentially a substitute for homolorphic local coordinates in complex
    geometry. 

    \begin{lem}[Weak Inverse Function Theorem] Let $p \in Y \subset X$ be a
        smooth point of both $X$ and $Y$. Then there exist local parameters
        $u_1, \ldots, u_n$ such that $u_1|_Y, \ldots, u_m|_Y$ are local
        parameters on $Y$ and $I(Y) = (u_{m+1}, \ldots, u_n)$ locally in some
        affine neighborhood of $p$. Equivalently, a smooth subvariety of a
        smooth variety is a local complete intersection.  \begin{proof} Observe
            the following short exact sequences: \begin{center} \begin{tikzcd}
            & 0 & 0 & 0\\ 0 \arrow{r}  & I_{Y,X}/(m_{p,X}^2 \cap I_{Y,X})
                \arrow{r} \arrow{u}& T^*_{p,X} \arrow{r} \arrow{u}& T^*_{p,Y}
                \arrow{r} \arrow{u}& 0 \\ 0 \arrow{r} & I_{Y,X}
                \arrow[twoheadrightarrow]{u} \arrow{r} & m_{p,X}
                \arrow[twoheadrightarrow]{u}\arrow{r} &
                m_{p,Y}\arrow[twoheadrightarrow]{u} \arrow{r} & 0 \\ 0
                \arrow{r} & m_{{p,X}^2} \cap I_{Y,X} \arrow[hookrightarrow]{u}
                \arrow{r} & m_{p,X}^2 \arrow[hookrightarrow]{u}\arrow{r} &
            m_{p,Y}^2 \arrow[hookrightarrow]{u} \arrow{r} & 0 \\ & 0 \arrow{u}
                                                          & 0 \arrow{u} & 0
                                                      \arrow{u} \end{tikzcd}
                                                  \end{center} Choose a basis
                                                  $u_1, \ldots, u_m$ of
                                                  $T_{p,Y}^*$ and lift to $X$.
                                                  Then choose $u_{m+1}, \ldots,
                                                  u_n$ which generate the
                                                  quotient.  This gives local
                                                  parameters $u_1, \ldots, u_n$
                                                  in $m_{p,X}$ such that
                                                  $u_1|_Y, \ldots, u_m|_Y$ are
                                                  local parameters on $Y$.
                                                  Define $Y' = V(u_{m+1},
                                                  \ldots, u+n)$. By previous
                                                  results, $Y'$ is smooth at
                                                  $p$ and has dimension $m$.
                                                  Then $Y \subset Y'$ is smooth
                                                  with the same dimension, so
                                                  they must be equal in some
                                                  neighborhood of $p$.
                                              \end{proof} \end{lem}

    \begin{rmk} $Y$ is a local complete intersection at $p$ if $\mc{L}_Y
    \subset \mc{O}_p$ is generated by $s= \mathrm{codim}_Y X$ equations.
\end{rmk}

    Now let $p \in X$ be smooth and $u_1, \ldots, u_n$ be local parameters.
    Then $\overline{X} = \mathrm{Bl}_p X \subset X \times \P^{n-1}$. We check
    that this is smooth. We know that $\mathrm{Bl}_p X \setminus E \simeq X
    \setminus \{p\}$. Therefore we check for a point $y^0 \in E$. We show that
    $m_{y^0} \overline{X}$ is generated by $n$ elements. Take the chart $U_i$
    of $\overline{X}$ given by $U_i \subset X \times \A^{n-1}_{y_2, \ldots,
    y_n}$. Then the equations are $u_j=u_1y_j$. Then \[m_{y^0} U_1 = m_p
    X|_{u_1} + m_{(y_2^0, \ldots, y_n^0)} \A^{n-1}|_{U_1} = (u_1, \ldots, u_n,
y_2-y_2^0, \ldots, y_n-y_n^0)|_{U_1} = (u_1, y_2-y_2^0, \ldots, y_n-y_n^0).\]
Thus $\overline{X}$ is smooth at $Y^0$.

    \begin{thm} Let $0 \in X \subset \A^n$ be a smooth point. The proper
        transform of $X$ in $\mathrm{Bl}_0 \A^n$ is isomorphic to
        $\mathrm{Bl}_0 X$.  \begin{proof} Choose local parameters $u_1, \ldots,
            u_n$ such that $I(X) = u_{m+1}, \ldots, u_n$ locallat at $0$. Then
            $u_1|_X, \ldots, u_m|_X$ are local parameters at $0 \in X$. We know
            the blowup is independent of the choice of local parameters.

            Therefore $\mathrm{Bl}_0 \A^n \subset \A^n \times \P^{n-1}$ with
            equations $u_iy_j = u_jy_i$ and $\mathrm{Bl}_0 X \subset X \times
            \P^{m-1}$ is given by $u_iy_j = u_jy_i$ (with appropriate indices).
            To get equations of $\mathrm{Bl}_0 X$ from equations of
            $\mathrm{Bl}_0 \A^n$, take equations $u_{m+1} = \cdots = u_n =
            y_{m+1} = \cdots = y_n = 0$ to cut out $X$ and $\A^n$.

            To see that this is the proper transform $\widetilde{X}$, note that
            $u_{m+1} = \cdots = u_n$ along $\widetilde{X}$ because they cut $X
            \times \P^{n-1}$ from $\A^n \times \P^{n-1}$. Then choose one of
            the $u_i$, say $u_1 \neq -$ at some part of $X - 0$. Then
        $u_1y_j=u_jy_1$ for all $j$ in $\mathrm{Bl}_0 \A^n$. Since $u_j=0$ for
    $j=m+1, \ldots, n$, $y_j = 0$.  \end{proof} \end{thm}
    
    \subsubsection{Normal Varieties} Let $p \in X$ be an irreducible variety.
    Then $X$ is normal at $p$ is $\mc{O}_p \subset k(X)$ is integrally closed.
    Then $X$ is normal if it is normal at every point.

    \begin{lem} If $p \in X$ is a smooth point, or factorial at $p$, then $X$
        is normal at $p$.  \begin{proof} $\mc{O}_p$ is a UFD and any UFD is
            integrally closed. To see this suppose $f \in k(X), f=
            \mathrm{a}{b}$ where $a,b \in \mc{O}_p$ and $a,b$ are coprime, is
            integral over $\mc{O}_p$. Then $f^s + a_1 f^{s-1} + \cdots + a_s =
            0$. Clearing denominators, we see that $a^s + a_1a^{s-1}b + \cdots
            + a_sb^s = 0$. Then $b \mid a^s$, so $b$ is a unit. Thus $f \in
        \mc{O}_p$.  \end{proof} \end{lem}

    \begin{lem} Let $X$ be affine. Then $X$ is normal if and only if $k[X]
        \subset k(X)$ is integrally closed.  \begin{proof} Let $R \subset
            f.f.R$ be integrally closed. Then we know that $R_{\mathfrak{p}}
            \subset f.f.R$ is integrally closed for every prime $p \subset R$.

            In the other direction, note that $R_m \subset K$ is integrally
            closed. Then note that $R = \cap R_m$ where the intersection is
            over all maximal ideals $m$. Then if $x \in K$ is integral over
            $R$, it must be integral over $R_m$ for all $m$, so it must be in
            $R_m$ for all $m$. Thus $x \in R$.

            To prove that $R = \cap_m R_m$, define $I = \{b \in R \mid xb \in R
            \}$. If $I=R$, then $1 \in I$ and thus $x \in R$. Otherwise, $I
            \subset m$ for some maximal ideal $m$. This is impossible because
            we can write $x=\frac{a}{b}$ where $b \notin m$ because $x \in
        R_m$.  \end{proof} \end{lem}

    \begin{rmk} Every irreducible surface in $\A^3$ with isolated singularities
    is normal.  \end{rmk}

    \begin{thm}[Serre's Criterion] Let $R$ be a Noetherian domain and $K$ the
        fraction field of $R$. Then $R \subset K$ is integrally closed if and
        only if \[R = \bigcap_{\substack{\mathfrak{p} \subset R \\
        \text{minimal prime}}} R_{\mathfrak{p}}\] and for all minimal primes
        $\mathfrak{p} \subset R$, $R_{\mathfrak{p}}$ is a DVR.  \end{thm}

    Now let $R=k[X]$ for some affine $X$. Then we find when $\mathfrak{p}
    \subset R$ is a minimal prime (of height $1$) if $\mathrm{dim}
    R/\mathfrak{p} = \mathrm{dim} R - 1$, or $Y = V(\mathfrak{p}) \subset X$ is
    a hypersurface.

    Note that $R_{\mathfrak{p}} = \mc{O}_Y = \{ f \in k(X) \mid f \text{ is
        regular at some point of } Y$.  Then note that $\cap R_{\mathfrak{p}} =
        \{f \in k(X) \mid f \text{ is regular at some point of every
        hypersurface}\}$. Therefore $R = \cap R_{\mathfrak{p}}$ if and only if
        every rational function regular at some point of every hypersurface is
        regular everywhere.

        \begin{thm}[Hartog's Extension Principle] Let $Z \subset X$ of
        codimension at least $2$. Then if $f \in k(x)$ is regular on $X
    \setminus Z$, $f$ is regular at some point of every hypersurface, so $f$ is
regular.  \end{thm}

        \subsection{Lecture 19 (Apr 09)} \subsubsection{Differential of a
        regular map} Let $\varphi:X \to Y$ be regular. Then we want to define
        $d_x \varphi: T_{X,x} \to T_{Y,y}$. We may assume that $X,Y$ are
        affine, so choose charts. Then recall that $T_{X,x} = (d_xf_1 = \cdots
        d_xf_r = 0)$ where $I(X) = (f_1, \ldots, f_r)$, so we define $d_x
        \varphi = (d_x \varphi_1, \ldots, d_x\varphi_m)$. 

        \begin{prop} $d_x \varphi(T_{X,x}) \subset T_{Y,y}$.  \begin{proof}
        $(d_yg_i)(d_x \varphi) = d_x (f_i \circ \varphi) = d_x 0 = 0$.
    \end{proof} \end{prop}

        More intrinsically, the dual map $T_{Y,y}^{\vee} \to T_{X,x}^{\vee}$ is
        induced by the pullback $\varphi^*: \mc{O}_{Y,y} \to \mc{O}_{X,x}$.

        \begin{exm} Consider $X = (xy=t) \subset \A^3$ and $\varphi:X \to \A^1$
            given by $(x,y,t) \mapsto t$. Note that $T_pX = \{y_0dx + x_0dy -
            dt = 0 \} \subset \A^3$. Therefore $X$ is smooth and $T_pX$ has
            coordinates $dx,dy$ (basis of $T_p X^{\vee}$) and $d\varphi = dt =
            y_0dx+xd_0y$, which is surjective unless $x_0=y_0=0$.

            To see the geometry of this map, we see that the fiber above every
        nonzero point is smooth (a torus), but the fiber at $0$ is the union of
    two coordinate axes.  \end{exm}

        \begin{lem} Let $\varphi:X \to Y$ be a regular map of smooth varieties
            with $\varphi(x) = y$. Suppose that $d_x \varphi: T_{X,x} \to
            T_{Y,y}$ is surjective. Then $varphi^{-1}(y)$ is nonsingular at
            $x$.  \begin{proof} Consider the differential $d_x \varphi: T_{X,x}
                \to T_{Y,y}$.  We show that $T_{f^{-1}(y),x} \subset
                \mathrm{Ker} d_x \varphi$. Given the claim, we see that
                $\mathrm{dim} T_{f^{-1}(y),x} \leq \mathrm{dim}\ \mathrm{Ker}
                d_x \varphi = \mathrm{dim} T_{X,x} - \mathrm{dim}T_{Y,y} =
                \mathrm{dim}X - \mathrm{dim}Y$. On the other hand,
                $\mathrm{dim}T_{f^{-1}(y),x} \geq \mathrm{dim_x} f^{-1}(y) \geq
                \mathrm{dim}X - \mathrm{dim}Y$. Thus all inequalities are
                equalities and $f^{-1}(y)$ is smooth at $x$.

                To proe the claim, consider the sequence of maps
                $T_{f^{-1}(y),x} \overset{d_x \iota}{\to} T_{X,x} \overset{d_x
                \varphi}{\to} T_{Y,y}$. Then $T_{f^{-1}(y),x} \to T_{Y,y}$ is a
                differential of the map $\varphi^{-1}(y) \to Y$. This is
            constant, so the differential is zero.  \end{proof} \end{lem}
        
        \subsubsection{Normal Varieties Continued} \begin{thm} A nonsingular
        variety is normal.  \end{thm} Proof of this theorem comes from Lemma
        173.

        \begin{defn}[Discrete Valuation] A discrete valuation on a field $k$ is
            a function $v: k^* \to \Z$ such that $v(fg) = v(f)+v(g)$ and
            $v(x+y) \geq \min\{v(x),v(y)\}$.

            Define $R = \{f \in K \mid v(f) \geq 0 \}$ and $m = \{ f \in R \mid
        v(f) > 0 \}$.  \end{defn}

        \begin{lem} $R$ is a local ring with maximal ideal $m$ (called a DVR)
            with field of fractions $k$.  \begin{proof} Clearly by definition
                of discrete valuation, $R$ is a ring and $m$ is an ideal. It is
                easy to see that everything with valuation $0$ is a unit in
            $R$, so $m$ is a maximal ideal. Then we see that for all $x \in K$,
        at least one of $v(x),v(x^{-1})$ is nonnegative, so at least one of
    $x,x^{-1}$ is in $R$.  \end{proof} \end{lem}

        \begin{lem} Every DVR is a PID and in fact, every ideal has form
            $(t^n)$ for some fixed $t \in m \setminus m^2$.  \begin{proof}
                Choose $t$ such that $v(t) = 1$. Then clearly $t \in m$. Take
                $I \subset R$ an ideal. Then let $n$ be the minimum valuation
            of any element of $I$, so $v(g/t^n) \geq 0$ and thus $g \in (t^n)$.
        Then choose a minimizing $f$, and we see that $f/t^n$ is a unit, so
    $t^n \in I$.  \end{proof} \end{lem}

        \begin{thm} Let $R$ be a ring. Then the following are equivalent:
        \begin{enumerate} \item $R$ is a DVR.  \item $R$ is a Noetherian local
        domain which is integrally closed and has Krull dimension $1$.
\end{enumerate} \end{thm}

        \begin{cor} Let $R$ be a Noetherian integrally closed domain and let
        $\mathfrak{p} \subset R$ is a height $1$ prime ideal. Then
    $R_{\mathfrak{p}}$ is a Noetherian, integrally closed, local domain of
Krull dimension $1$.  \end{cor}

        \begin{proof}[Proof of Theorem 184] First let $R$ be a DVR. Then $R$ is
            a PID, so it is Noetherian and integrally closed. Also $R$ is local
            with prime ideals $(0)$ and $m$, so it has Krull dimension $1$.

            In the other direction, let $m \subset R$ be the maximal ideal.
            Take $x \in R$ nonzero. Then there exists a unique $n \geq 0$ such
            that $x \in m^n \setminus m^{n+1}$ by Krull intersection theorem.We
            need to show that $m$ is principal. Given the claim, take $x \in R
            \setminus 0$. Then $x \in (t^n) \setminus (t^{n+1})$ for a unique
            $n$. Thus $x = ut^n$ for some unit $u$, and thus $n$ defines a
            valuation $k^* \to \Z$.

            To prove the claim, $m \neq m^2$ by Krull or by Nakayama. Take $t
            \in m \setminus m^2$. Since the only prime ideals are $0,m$, and
            every radical ideal is an intersection of primes, $m =
            \sqrt{t}$.Thus $m^n \subset (t)$ for some $n \geq 1$. Choose $n$ to
            be the smallest such $n$, so choose $b \in m^{n-1} \setminus (t)$.
            Take $x = \frac{t}{b} \in K$. Then $b \in (t)$, so $x^{-1} \notin
            R$. Because $R$ is integrally closed, $x^{-1}$ is not integral over
            $K$. This implies that $x^{-1}m \not\subseteq m$ (otherwise, choose
            $m=(e_1, \ldots, e_s)$ and write $x^{-1}e_i = \sum a_{ij} e_j$.
            Then $\sum (x^{-1} \delta_{ij} - a_{ij})e_j = 0$, so $\mathrm{det}
            \abs{ x^{-1} \delta_{ij} - a_{ij}} = 0$. Thus $x^{-1}$ is
            integral).

            However, $x^{-1}m \subset R$ because if $z \in m$ then $bz \in m^n
            \subset (t)$, so we can write $bz = rt$, so $x^{-1}z = r \in R$. We
            see that $x^{-1}m \subset R$, so it is an ideal $R$ but not
            contained in a maximal ideal. Therefore $x^{-1}m = R$, so $m =
        (x)$.  \end{proof}

        Let $X$ be a normal variety. Then we have local rings $R_{\mathfrak{p}}
        \subset k(X)$ and a DVR for every irreducible subvariety $Y \subset X$
        of codimension $1$. Thus we have a collection of discrete valuations
        $v_Y: k(X)^* \to \Z$. There are called orders of zeroes or poles of $f
        \in k(X)$ along $Y$.

        \subsection{Lecture 20 (Apr 11)} Recall that if $X$ is a normal variety
        and $Y \subset X$ is an irreducible subvariety, then $\mc{O}_{Y,X}$ is
        a DVR. Last time we defined the order of zeroes or poles as the
        valuation.

        \begin{cor} Let $X$ be normal and $Y \subset X$ be irreducible. Then
            there exists an affine chart $U \subset X$ with $U \cap Y \neq 0$
            such that $I(Y) \subset k[U]$ is principal.  \begin{proof} Assume
                that $X$ is affine. Then $\mf{p} = I(Y) = (u_1, \ldots, u_m)
                \subset k[X] = R$. Then we note that $\mf{p}_{\mf{p}} \subset
                R_{\mf{p}}$ is the maximal ideal in the DVR and it is
                principal. Then we can write $t = \frac{a}{b}$ where $b \notin
                \mf{p}$. We can write $u_i = tv_i$, where $v_i \in R_{\mf{p}}$,
                where $v_i = \frac{a_i}{b_i}$ where $b_i \notin \mf{p}$. Now we
                know $t \in k[U]$ and $u_i = tv_i$, where $v_i \in k[U]$. Then
            $I(Y \cap U) = (t) \subset k[U]$.  \end{proof} \end{cor}

        \begin{cor} The singular locus of a normal variety ha codimension at
            least $2$.  \begin{proof} Suppose there exists an irreducible
                hypersurface $Y$ in the singular locus. As a variety, $Y$ has
                an open smooth locus.  Thus we can shrink $X$ to an affine
                chart $U \subset X$ such that $U \cap Y$ is nonempty and
                smooth. By the previous corollary, we can also assume that $X$
                is affine. Then $I(Y) = (t)$. This implies $X$ is smooth along
                $Y$ by Theorem 153. This gives a contradiction.  \end{proof}
            \end{cor}

        \begin{cor} For curves, smooth is equivalent to normal.  \end{cor}

        \subsubsection{Normalization} Trying to resolve singularities is not
        very well resolved, and has led to several Fields Medals. On the other
        hand, normalization is relatively easy to understand and always
        available.  \begin{exm} Consider the curve $X = \{y^2=x^3\}$. This is
            singular, so not normal. We see that $t \in k(X)$ is a root of a
        monic equation $T^2-x$, but $t \notin k[X]$. We know that the rational
    parameterization from $\A^1$ is birational and finite, and the source is
normal.  \end{exm}

        \begin{defn} A morphism $\nu:X^{\nu} \to X$ is called a normalization
        if it is birational and finite and $X^{\nu}$ is normal.  \end{defn}

        \begin{thm} Every irreducible variety has a normalization, which is
            unique up to isomorphism.  \begin{proof}[Construction for affine
                $X$] Let $k[X] \subset k(X)$, so defien $R$ to be the integral
                closure of $k[X]$ in $k(X)$. We show that $R$ is a finitely
                generated $k[X]-module$. Given the claim, $R$ is also a
                finitely-generated $k$-algebra, so $R=k[Y]$ for some variety
                $Y$, which must be birational to $X$ because $k[X] \subset k[Y]
                \subset k(X)$. The extension $k[Y] /k[X]$ is integral, so $Y
                \to X$ is finite. Also, we know $R$ is integrally closed, so
                $Y$ is normal.

                To prove the claim, use Noether's normalization lemma. We have
                a diagram

                \begin{center} \begin{tikzcd} k[X] \arrow[hookrightarrow]{r} &
                    R \arrow[hookrightarrow]{r} & k(X) \\ k[T_1, \ldots, T_n]
                    \arrow[hookrightarrow]{rr} \arrow[hookrightarrow]{u} & &
                    k(T_1, \ldots, T_n) \arrow[hookrightarrow]{u} \end{tikzcd}
                \end{center} We know that $A$ is a UFD, so it is integrally
                closed. Thus $\A^n$ is normal, so $R$ is integral over $A$.
                This implies that $R$ is the integral closure of $A$ in $L$. It
                remains to prove the following lemma.  \end{proof} \end{thm}

        \begin{lem} Let $A$ be an integrally closed Noetherian domain in its
            field of fractions $K$. Then if $L/K$ is a finite separable
            extension and $B$ is the intgral closure of $A$ in $L$, then $B$ is
            a finitely generated $A$-module.  \begin{proof} Consider the map $L
                \to K$ given by the trace of the map given by multiplication by
                $x$. Because $L/K$ is separable, then $Tr(xy)$ is a
                nondegenerate quadratic form. It suffices to find a basis $v_1,
                \ldots, v_n$ of $L$ over $K$ such that $B \subset A v_1 +
                \cdots + Av_n$ (this is because $A$ is Noetherian).

                To find the basis, first choose some basis $u_1, \ldots, u_n$.
                After rescaling by elements of $A$, we can assume that $u_1,
                \ldots, u_n \in B$. Then the $u_i$ are algebraic over $K$, so
                it is a root of some polynomial $a_0u_i^{d_i} + \cdots +
                a_{d_i} = 0$. Then we see that $a_0u_i \in B$. Finally, let
                $v_1, \ldots, v_n$ be a dual basis via the trace form. Take $x
                \in B$ and write it as $x_1v_1 + \cdots + x_nv_n$. First note
            that $Tr(xu_j) = x_j$. On the other hand, $x \in B$, $u_i \in B$
        implies that $xu_i \in B$, so $Tr(xu_i) \in A$. Therefore $x_i \in A$
    for all $i$.  \end{proof} \end{lem}

        \begin{thm}[Universal Property of Normalization] \begin{enumerate}
            \item Suppose $g:Y \to X$ is a finite birational regular map.  Then
                there exists a unique morphism  $f:X^{\nu} \to Y$ such that
                $\nu = g \circ f$.  \item If $Y$ is a normal variety with a
                dominant regular map $g:Y \to X$, then there exists a unique
        morphism $f:Y \to X^{\nu}$ such that $g = \nu \circ f$.
\end{enumerate} \end{thm}

        \begin{proof} Assume $X$ is affine. In 1, we can also asusme $Y$ is
            affine. Then note that $k(Y) = k(X)$, so we must conclude that
            $k[Y] \subset k[X^{\nu}]$, and this inclusion is unique.

            In 2, we can also assume $Y$ is affine. Then we again know that
            $k[X] \subset k[Y]$ and $k[X]\subset k[X^{\nu}]$ and that $k(X)
            \subset k(Y)$. We know $k[Y]$ is integrally closed, so $k[Y]
            \supset k[X^{\nu}]$ and this containment is unique.  \end{proof}

        \subsection{Lecture 21 (Apr 16)} Homework is now due next Tuesday as
        opposed to this Thursday. There will be one more homework and then the
        second take-home midterm.
        
        \section{Divisors} Let $X$ be an irreducible variety.

        \begin{defn}[Prime Divisor] A prime divisor $D \subset X$ is an
        irreducible subvariety of codimension $1$ (hypersurface).  \end{defn}

        \begin{defn}[Divisor Group] The divisor group $\operatorname{Div} X$ is
        a free abelian group generated by symbols $[D]$ for each prime divisor
    $D \subset X$.  \end{defn}

        \begin{defn} A divisor $D = \sum n_i H_i$ is effective, or $D \geq 0$,
        if $n_i \geq i$ for all $i$.  \end{defn}

        Suppose $X$ is normal (or at least non-singular in codimension $1$) and
        let $H \subset X$ a prime divisor. Then $\mc{O}_{H,X} \subset k(X)$ is
        a DVR with valuation $v_H: k(X)^* \to \Z$. If $v_H(f) > 0$ then we say
        that $f$ has a zero of order $v_H(f)$ along $H$. If $v_H(f) < 0$, then
        $f$ has a pole. To every $f \in k(X)^*$ we associate a divisor $(f) =
        \sum_{H \subset X} v_H(f)[H]$ called the divisor of zeros and poles.

        \begin{lem} $v_H(f) = 0$ for all but finitely many prime divisors.
            \begin{proof} We can shrink $X$ as much as we want because if $U
                \subset X$ is open, then $X \setminus  U$ contains only
                finitely many hypersurfaces. We can also assume that $X$ is
                affine and that $f \in k[X]$. We can also remove $V(f)$ and
            assume that $f$ is invertible. Then it is easy to see that $(f) =
        0$.Indeed, $f$ is invertible on every hypersurface, so its valuation
    must be $0$.  \end{proof} \end{lem}

        \begin{rmk} Divisors of the form $(f)$ are called principal divisors.
        Principal divisors form a subgrop in $\operatorname{Div} X$ because the
    valuation is like a logarithm (takes products to sums).  \end{rmk}
        
        \begin{exm} If $X$ is normal and projective and $f \in k(X)$, $(f) \geq
            0$ if and only if $f$ is constant.  \end{exm} \begin{defn} The
            divisor class group $\operatorname{Cl} X \coloneqq
            \operatorname{Div} X / \{(f)\}$ is the quotient of the divisor
        group by the subgroup of principal divisors.  \end{defn}
        
        \begin{exm} Take $X = \A^n$. Then a prime divisor $H \subset X$ is an
            irreducible hypersurface given by an irreducible polynomial $F \in
            k[x_1, \ldots, x_n]$. We claim that $div(F) = H$. In particular,
            all divisors are principal, so $\operatorname{Cl}(\A^n) = 0$. To
        see the claim, we compute $v_H(F) = 1$ and then for any other
    hypersurface, $v_{H'}(f) = 0$.  \end{exm}

        \begin{rmk} Suppose $X$ is normal, $f \in k(X)$ and suppose $(f) \geq
            0$. Then $f$ is regular outside of that divisor and at general
            points of irreducible components of $(f)$, so $f$ is not regular
            only in codimension $2$. Therefore $f$ is regular everywhere.
        \end{rmk}

        \begin{prop}[Homework] Let $X$ be an affine variety. Then the class
        group of $X$ is trivial if and only if $k[X]$ is a UFD.  \end{prop}

        \begin{exm} Let $X = \P^n$. Then $H \subset X$ irreducible hypersurface
            is given by an irreducible homogeneous polynomial $F \in k[T_0,
            \ldots, T_n]$ of degree $k$. However, $F \notin k(\P^n)$. However,
            we know $I(H \cap \A^n_i) = \left(\frac{F}{T_i^k}\right)$. Then we
            know $v_{\widetilde{H}}(f)$ is $1$ when $\widetilde{H} = H$, $0$
            when $\widetilde{H} \neq H$ and $\widetilde{H} \cap \A^n_i \neq
            \emptyset$, and unknown when $\widetilde{H}$ is the hyperplane at
            infinity.

            To compute this, choose a different affine chart which gives $-k$.
        Now take any $f \in k(\P^n)$. We know $f = \frac{\prod F_i^{n_i}}{\prod
    G_j^{m_j}}$, so we can write $(f) = \sum n_i [F_i = 0] - \sum m_j[G_j =
0]$.  \end{exm}

        \begin{cor} $D \in \operatorname{Div}\P^n$ is principal if and only if
            $\mathrm{deg}(D) = 0$, where $\mathrm{def}(D) = \sum n_i
            \mathrm{deg}(F_i)$, where $F_i \in k[T_0, \ldots, T_n]$ is a
            homogeneous equation of $H_i$. Therefore the class group of $\P^n$
        is $\Z$.  \end{cor}

        \begin{defn} Divisors $D,D'$ on $X$ are called linearly equivalent ($D
        \sim D'$) if $D-D'$ is a principal divisor.  \end{defn}

        \begin{rmk} Every divisor on $\P^n$ is linearly equivalent to a unique
        multiple of a hyperplane.  \end{rmk}

        The notion of a divisor we just introduced is called a Weil divisor.

        \begin{defn}[Cartier Divisor] A Cartier divisor on an irreducible
            variety $X$ is the following data: \begin{enumerate} \item A cover
                $X = \cup U_i$ by open sets; \item Rational functions $f_i \in
                k(X)^*$ ``defining a divisor on $U_i$'' satisfying a
                compatibility condition: $f_i/f_j$ is invertible on $U_i \cap
        U_j$.  \end{enumerate} Two such data $(U_i,f_i)$ and $(V_j,g_j)$ are
    considered equivalent if $f_i/g_j$ is an invertible regular function on
$U_i \cap V_j$.  \end{defn}

        Now suppose $X$ is normal, or at least nonsingular in codimension $1$.
        Then to each Cartier divisor $(U_i,f_i)$ we associate a Weil divisor
        $F$ such that $D \cap U_i = (f_i) \cap U_i$. Concretely, $D = \sum
        n_jH_j$, where $n_j = v_{H_j}(f_i)$ as long as $H_j \cap U_i \neq
        \emptyset$. 

        \begin{exm} Take a prime divisor $H \subset \P^n$ and take the standard
            cover $\P^n = \cup U_i$. We see that $I(H \cap U_i) = (f_i)$ where
            $f_i \in k[x_0, \ldots, x_n]$. Thus $H$ is in fact given by a
            Cartier divisor $(U_i, f_i)$.

            Why are the $f_i$ compatible? If $H$ is given by homogeneous $F$ of
        degree $k$, then $f_i = F/T_i^k$. Then $f_i/f_j = (T_j/T_i)^k$, which
    is regular and invertible on $U_i \cap U_j$.  \end{exm}

        We can turn $\operatorname{CDiv}$, the group of Cartier divisors into a
        group $(U_i,f_i) \cdots (V_j,g_j) = (U_i \cap V_j, f_ig_j)$. By the
        above construction, we have a homomorphism $\operatorname{CDiv} \to
        \operatorname{Div}$. If $f \in k(X)^*$, there is a Cartier divisor
        $(X,f)$, so principal divisors lie in the Cartier divisors. If $X$ is
        normal, this is an inclusion.

        \begin{defn}[Picard Group] The Picard group $\operatorname{Pic} X
        \coloneqq \operatorname{CDiv}/\{(f)\}$ is the quotient of the group of
    Cartier divisors by principal divisors.  \end{defn}

        Note that the Picard group is contained in the class group if $X$ is
        normal. We want to determine when every Weil divisor is Cartier. If $X$
        is nonsingular (more generally, $\mc{O}_{x,X}$ is a UFD) for all $x \in
        X$, choose $H \subset X$ a prime Weil divisor. We need to construct an
        open cover: first take $U_0 = X \setminus H, f_0 = 1$. For all $x \in
        X$, there exists some affine neighborhood $U \supset X$, which depends
        on $x \in X$, such that $H$ is principal in this neighborhood. This
        gives an infinite open cover, but we can choose a finite subcover. Thus
        $H$ is a Cartier divisor.

        \begin{cor} If $X$ is locally factorial, every divisor is Cartier.
        \end{cor}

        \subsection{Lecture 22 (Apr 18)} We will continue our discussion of
        divisors. Here we will assume that $X$ is nonsingular and study the
        behavior of divisors under regular maps.

        Let $D = (U_{\alpha}, f_{\alpha})$ where $Y= \cup U_{\alpha}$,
        $f_{\alpha}\in k(Y)^*$. Then we define $\varphi^*D =
        (\varphi^{-1}U_{\alpha}, \varphi^* f_{\alpha})$. We need to see that
        the pullback of each $f_{\alpha}$ is well defined and nonzero. This
        happens with $\overline{\varphi(X)}$ is not contained in the divisor of
        zeroes and poles of $f_{\alpha}$, so when $\overline{\varphi(X)}
        \not\subseteq D$. Fortunately, we can pass to a linearly equivalent
        divisor.

        \begin{lem} Let $X$ be nonsingular. Then for any divisor $D \subset X$
            there exists $D' \sim D$ such that $x_1, \ldots, x_m \notin D'$.
            \begin{proof} We argue by induction on $m$. We can assume that
                $x_1, \ldots, x_{m-1} \notin D, x_m \in D$. We can also assume
                that $D$ is prime and assume that $X$ is affine. Thn $D$ has a
                local equation at $x_m$: $(\pi) = \mc{L}_D \subset
                \mc{O}_{x_m}$, where $\pi = p/q$ for some $p,q \in k[X]$. Near
                $x_m$, we see that $D = (p)$. Then $D' = D-(p) \sim D$ does not
                contain $x_m$, but it may contain $x_1, \ldots, x_{m-1}$.

                Choose $g_i \in k[X]$ such that $g_i(x_i) \neq 0$. Then for all
                $i = 1, \ldots, m-1$, $g_i|_{D \cup \{x_1, \ldots, x_m\}} = 0.$
                Then choose constants $\alpha_1, \ldots, \alpha_{m-1}$ such
                that for $f = p+\sum\alpha_i g_i^2$, $f(x_i) \neq 0$. We claim
                that $D' = D-(f)$ does nto contain $x_1, \ldots, x_m$. We know
                that $f(x_i) \neq 0$, so $x_1, \ldots, x_{m-1} \notin (f)$. Now
                we need to know that $f$ is a local equation of $D$. Then
            because $f = \sum \alpha_i g_i^2$, we know that $p|g_i$, so $f =
        p(1+\sum \alpha_i pw^2)$,so it is also a local equation of $D$.
    \end{proof} \end{lem}

        This defines a pullback on the level of Picard groups.

        \subsubsection{Divisors of rational maps to $\P^n$} Suppose $\varphi: X
        \dashrightarrow \P^n$ is a rational map defined by $(\varphi_0: \cdots
        : \varphi_n)$. Suppose know $(\varphi) = D_i - D$ where $D_0, \ldots,
        D_n,D$ are effective. Choose $D$ to be the smallest possible.

        We want to know where $\varphi$ is regular. Choose $x \in X$. Then
        $\varphi_i = p_i/q$, where $p_i$ is a local equation of $D_i$ and $q$
        is a local equation of $D$. Thus $\varphi = (p_0: \cdots : p_n).$ If
        $p_i(x) \neq 0$ for some $x$, $\varphi$ is regular at $p$. The converse
        is also true because $\mc{O}_x$ is a UFD.
        
        \begin{thm} The indeterminancy locus of $\varphi$ is $D_0 \cap \cdots
        \cap D_n$, where $D_i = (p_i=0)$.  \end{thm}

        \begin{exm} Consider the rational normal curve $\P^1 \to \P^n$. Then
            $\varphi_i = x^i$ and $(x^i) = i[0]-i[\infty]$. In another chart,
            $x = \frac{1}{y}$. Then $D = n[\infty]$, so $D_i =
            i[0]+(n-i)[\infty]$ and then $D_0 \cap \cdots \cap D_n =
            \emptyset$, so the map is indeed regular everywhere.  \end{exm}

        Suppose $X$ is a nonsingular variety and $D$ is a divisor. Then set
        $\mc{L}(D) = \{f \in k(X) \mid (f) + D \geq 0 \}$ the complete linear
        system.

        \begin{thm}[Serre] If $X$ is projective, $\mc{L}(D)$ has dimension
        $\ell(D) < \infty$.  \end{thm}

        We can define $\varphi_D: X \dashrightarrow \P^{\ell(D)-1}$. First
        choose a basis $\varphi_1, \ldots, \varphi_{\ell(D)}$ and then set
        $\varphi_D = [\varphi_1: \cdots : \varpi_{\ell(D)}]$.

        \begin{lem} If $D \sim D'$ then $\mc{L}(D) \simeq \mc{L}(D')$ and
            $\varphi_D = \varphi_{D'}$, so we have distinguished maps to
            projective spaces parameterized by the Picard group.  \begin{proof}
                If $D \sim D'$ then $D = D'+(f)$. Therefore there are
                isomorphisms between the linear systems given by multiplying
                and dividing by $f$, so they are isomorphic. Then if
                $\varphi_1, \ldots, \varphi_{\ell(D)}$ is a basis of
                $\mc{L}(D)$, $f \varphi_1, \ldots, f \varphi_{\ell(D)}$ is a
            basis of $\mc{L}(D')$. Then the two maps are the same.  \end{proof}
        \end{lem}

        \begin{rmk} If $X$ is projective then if $(f)+D = (g)+D$, we must have
        $(f) = (g)$, so $(f/g) = 0$ and thus $f/g$ is a constant.  \end{rmk}

        If $X$ is projective, then $\P^{ell(D)-1} = \P(\mc{L}(D)) = \abs{D}$,
        which is defined to be the set of effective divisors linearly
        equivalent to $D$.

        \begin{exm} If $X= \P^r$, then $\operatorname{Pic}X =
            \operatorname{Cl}X = \Z = \Z[H]$, where $H$ is a hyperplane. Then
            $\abs{dH}$ is the set of hypersurfaces of degree $d$, which is
            $\P^{\binom{r+d}{d}-1}$. We see that $\mc{L}(dH) = \mc{L}(dH_0) =
            \{f \in k(\P^r) \mid (f) + dH_0 \geq 0 \}$. Note that $\P^r
            \setminus H_0 = \A^r_{x_1, \ldots, x_r}$. Then we have that all
            poles of $f$ are at infinity. Thus $f \in k[x_1, \ldots, x_r]$, so
            $v_{H_0}(f) = - \operatorname{deg}(f)$. After homogeneizing, this
        becomes the set of degree $d$ hypersurfaces, so the map given by the
    divisor is the Veronese embedding.  \end{exm}
        
        \subsection{Lecture 23 (Apr 23)} We continue with our discussion of
        linear systems of divisors. We will always assume that $X$ is
        projective and nonsingular. Suppose $\varphi:X \dashrightarrow \P^r$ is
        a rational map.

        \begin{rmk} Given $\varphi:X \to \P^r$, it can be degenerate, say
            $\varphi(X) \subset \P^{r-1} \subset \P^r$. In this case,
            $\varphi_0, \ldots, \varphi_r$ are linearly dependent. This does
            not happen for $\varphi_D$, so we can assome that $\varphi:X
            \dashrightarrow \P^r$ is not degenerate.  \end{rmk}

        For a basis $f_0, \ldots, f_r$ of $\mc{L}(D)$, \begin{enumerate} \item
            We can assume that $D$ is effective because $\varphi_D =
            \varphi_{D'}$ for all $D \sim D'$.  \item If $D$ is effective,
            $(f_i) = D_i - D$. However, it can happen that the $D_i$ have a
            common prime divisor.  \end{enumerate}

        \begin{defn}[Fixed Part] A fixed part of a linear system $\mc{L}(D)$ is
        the largest effective divisor $F$ such that $D_i - F \geq 0$. We can
    write $D_i = F+M_i$, where $M_i$ is the moving part of the linear system.
\end{defn}

        We can write $abs{D} = \{D' \sim D \mid D' \geq 0\} = \P(\mc{L}(D))$.
        Another way to phrase this is to introduce the base locus.

        \begin{defn}[Base Locus] The base locus $BL_D$ of a divisor $D$ is
        defined by $BL_D = \cap_{D' \in \abs{D}} D' \supset F$.  \end{defn}
        
        We know that $\mc{L}(D) \simeq \mc{L}(D-F)$ and that $\varphi_D =
        \varphi_{D-F}$. Thus we can always reduce to linear systems without
        fixed part.

        \begin{defn}[Base-point-free] A divisor $D$ is base-point-free if its
        base locus is empty. Note this implies $\varphi_D$ is regular.
    \end{defn}

        \begin{defn}[Very Ample] $D$ is very ample if it is base-point-free and
        $\varphi_D$ is an embedding.  \end{defn}

        \begin{defn}[Ample] $D$ is called ample if $kD$ is very ample for some
        $k > 0$.  \end{defn}

        When we start with $\varphi:X \dashrightarrow \P^r$, $\varphi_0,
        \ldots, \varphi_r \in \mc{L}(D)$. Now we consider incomplete linear
        systems: a subspace $V \subset \mc{L}(D) \subset k(X)$. Choose a basis
        $f_0, \ldots, f_s$ of $V$. Then $\varphi_V = [f_0: \cdots : f_s]: X
        \dashrightarrow \P^s$. We can complete $f_0, \ldots, f_s$ to a basis
        $f_0, \ldots, f_r$ of $\mc{L}(D)$ and recover $\varphi_V$ as the
        composition of $\varphi_D$ and the projection.

        We can talk about the base locus, fixed part of $V$, define global
        generation, and very ampleness of $V$.

        \begin{exm} Consider $\P^2$ with coordinates $x,y,z$ and let $H =
            (Z=0)$. Then $\mc{L}(2H) = \{f \in k(\P^2) \mid (f) + 2H \geq 0 \}
            = \{f \in k[x,y] \mid \operatorname{deg} f \leq 2 \} = k[x,y,z]_2$.
            Thus $\abs{2H}$ is the set of conics in $\P^2$ and $\varphi_{2H} =
            [z^2:xz:yz:x^2:xy:y^2]$, the Veronese embedding. Thus $2H$ is very
            ample.

            Now we consider an incomplete linear system. Chooise points $p =
            [1:0:0],q = [0:1:0]$. Consider $V$ the set of conics passing
            through $a,b$. Then $\mc{L}(2H) = k^6$ and $\operatorname{dim}V
            \geq 4$. In fact, $V$ is spanned by $1,x,y,xy$. Then $\varphi_V:
            \P^2 \dashrightarrow \P^3$ is given by $[x:y:1] \mapsto
            [1:x:y:xy]$. In particular, $\varphi_V$ is birational. The equation
            of the image is actually $AD-BC = 0$, which makes $\varphi_V$ a
            birational map $\P^2 \dashrightarrow Q$ to a quadric surface in
            $\P^3$. We see that the base locus is $p,q$.

            We consider the image of $H$. For any point $[x:y:0]$, the image is
            $[0:0:0:xy] = [0:0:0:1]$. Thus $H$ collapses to the point
            $[0:0:0:1]$. In the hyperplane at infinity $Q \cap (A = 0)$ is
            given by $BC = 0$, which is a union of two lines. We have two
            points where $\varphi_V$ is not regular, so we resolve by blowing
            up at $p,q$. We blowup at $p$ and consider two charts of the local
            blowup.

            In the chart $x=1$, we see $y=u,z=uv$, so $v = z/y$. Thus the chart
            $u=0$ is exceptional. In this chart, the map is $[z^2:z:yz:y] =
            [u^2v^2:uv:u^2v:u] = [uv^2:v:uv:1]$. We see this is regular. Note
            $v$ is a coordinate along $\P^1$, so restricting to the exceptional
            divisor $E$, we get the map $[0:v:0:1]$, which maps $E$ onto a line
            $A=C=0$. The exceptional divisor over $q$ will map to another line
            $A=B=0$.

            Thus $\varphi_V$ is given by blowing up $p,q$ and then contracting
        the proper transform of $H$.  \end{exm}

        \begin{exm} We will do similar calculations without being able to do it
            too explicitly. Choose $6$ points $p_1, \ldots, p_6$ on $\P^2$ that
            are distinct, in general linear position, and not coconic. Consider
            the linear system $\mc{L}(3H)$ and $\varphi_{3H}$. Note $\abs{3H}$
            is the space of cubics in $\P^2$, so consider $V$ to be the set of
            cubics through $p_1, \ldots, p_6$. We check that the dimension of
            $V$ is $4$. If this is not the case, then every cubic that passes
            through $p_2, \ldots, p_6$ also passes through $p_1$. This is false
            because we can take a conic through $p_1, \ldots, p_6$ and add a
            line not through $p_1$. Thus we get a rational map $\varphi_V: \P^2
            \dashrightarrow \P^3$.

            First we find the base locus, which is $\{p_1, \ldots, p_6\}$. For
            any other point $q$, we can take the conic through five of the
            points and a line containing $p_6$ but not $q$. If $q$ is on the
            conic, just change the five points and then note that $q$ cannot be
        on the new conic because both conics are smooth and they already have
    four intersections points.  \end{exm}

        \subsection{Lecture 24 (Apr 25)} Last time we were building up to a
        theorem of Clebsch: \begin{thm}[Clebsch] For a smooth projective
            surface $X$, the following are equivalent: \begin{enumerate} \item
                $X \simeq Bl_6\P^2$ where the six points are in general
                position (no three are collinear and the six points are not
                coconic).  \item $X$ is a smooth cubic surface in $\P^3$.
            \item $X$ is a del Pezzo surface of degree $3$ (Fano variety of
        dimension $2$, which means the anticanonical divisor is ample).
\end{enumerate} \end{thm}

        We continue Example 226 from last time.  \begin{exm*}[Continuation of
            Example 226] We show that for all $p,q \in \P^2 \setminus \{p_1,
            \ldots, p_6 \}$, $\varphi_V(p) \neq \varphi_V(q)$. To see this,
            take the conic through $1,2,3,4,5$ and the line through $6,p$ and a
            similar combinatorial argument from the base locus calculation
            tells us that at least one choice works.

        Next we show that the induced rational map $Bl_6 \P^2 \to \P^3$ is
        regular. It suffices to show that it is regular at every point of $E
        \simeq \P^1$ over $p_1$. To do this, we find two cubics $D_1, D_2 \in
        V$ transversal at $p_1$. To do this, we take the conic $C$ through
        $p_2, \ldots, p_6$ and then two distinct lines $L_1, L_2$ through
        $p_1$. Then $D_1 = C \cup L_1, D_2 = C \cup L_2$. To see that this is
        enough, choose a basis $f_1, \ldots, f_4$ of $V$ such that $D_1 = (f_1
        = 0)$ and $D_2 = (f_2 = 0)$. Assume $p_1 = [0:0:1]$. We know
        $\varphi_V$ is given by $[f_1:f_2:f_3:f_4] = [f_1:z^3: \cdots :
        f_4/z^3]$. We can dehomogeneize in the standard chart and write
        $\varphi_V = [y + \alpha:x + \beta:g_3:g_4]$, where $\alpha,\beta \in
        m_{(0,0)}^2$ and $g_3(0,0) = g_4(0,0) = 0$. 

        Next we blow up and consider the chart $x=u,y=uv$. Then $\varphi_V =
        [uv+u^2g_1:u+u^2g_2:ug_3:ug_4] = [v+ug_1:1+ug_2:g_3:g_4]$. The equation
        of $E_1$ is $u=0$, so we substitute $u=0$ and in the second component
        we have $1$, so $\varphi_V$ is regular at every point of $E_1$.
    \end{exm*}

        \begin{rmk} We consider $\pi^* V$ where $\pi: Bl_6 \P^2 \to \P^2$. Then
            for all $D \in V$, we see $\pi^* D = E_1 + \cdots + E_6 + D'$, so
            $\pi^*V = (E_1 + \cdots + E_6) + \abs{D'}$. Thus $D' \sim \pi^*(3H)
            - E_1 - \cdots - E_6$. The $D'$ are the preimages of hyperplanes in
            $\P^3$.  Because $\varphi(E_i)$ are not points (it intersects a
            general hyperplane at one point), it must be a curve. In fact it
            must be a line because a general hyperplane will give a cubic in
            $\P^2$ smooth at $p_i$, so its proper transform intersects $E_i$ in
            one point.  \end{rmk}

        Next we will show that $\varphi(Bl_6 \P^2)$ is smooth. Indeed, if not,
        $T_x$ has dimension $3$ for some $x \in X$. (To be continued next
        lecture).

        \section{Regular Differential Forms}% Now we will attempt to perform
        calculus on algebraic varieties. Let $X$ be a variety and $f \in k[X]$.
        Then $d_xf$ is a linear function on $T_x$.

        \begin{defn} $\Phi[X] = \{ \text{ functions that assign every } x \in X
        \text{ an element of } T_x^{\vee} \}$. Note that $\Phi[X]$ is a
    $k[X]$-module. For $f \in k[X], \alpha \in \Phi[X]$, $(f \alpha)(x) = f(x)
\alpha(x)$.  \end{defn}

        \begin{defn} The regular differential forms $\Omega[X] \subset \Phi[X]$
            are $\alpha \in \Phi[X]$ such that for every $x \in X$, there
            exists a neighborhood $U \ni x$ such that $\alpha|_U - \sum_i f_i
            dg_i$. We have a map $d: k[X] \to \Omega[X]$ and the Leibniz rule
            $d(fg) = f(dg) + g(df)$.  \end{defn}

        \begin{lem} If $X$ is affine, then $\Omega[X]$ is generated by
            differentials.  \begin{proof} We use another algebraic partition of
                unity. For all $x \in X$, we can write $\alpha = \sum
                f_{i,x}dg_{i.x}$ is some neighborhood of $x$. Clearing
                denominators, $p_x \alpha = \sum_i r_{i,x} dg_i,x$ where
                $p_x,r_{i,x},g_{i,x} \in k[X]$ and $p_x(x) \neq 0$. Then the
                ideal $(p_x) = k[X]$ by the Nullstellensatz, so we can choose
                finitely many of them and write $\sum p_x q_x = 1$. Then
                multiply by $q_x$ and add, and we get $\alpha = \sum_{i.x}
            r_{i,x} q_x dg_{i,x}$.  \end{proof} \end{lem}

        \begin{cor} Let $X$ be affine and $g_1, \ldots, g_s$ generate $k[X]$.
        Then $dg_1, \ldots, dg_s$ generate $\Omega[X]$ as a $k[X]$-module (use
    the Leibniz rule to prove this).  \end{cor}

        In the simplest case, $\Omega[\A^n]$ is a free $k[x_1, \ldots,
        x_n]$-module with generators $dx_1, \ldots, dx_n$ (if $\sum f_i dx_i =
        0$ then $\exists x \in \A^n$ where one $f_i \neq 0$, and then $dx_1,
        \ldots, dx_n$ ar e linearly dependent in $T_x\A^3$, which is a
        contradiction).

        \begin{thm} Suppose $x \in X$ is a nonsingular point with local
            parameters $u_1, \ldots, u_n$. Then there exists an affine
            neighborhood $U$ of $x$ such that $\Omega[U]$ is a free
            $k[U]$-module generated by $du_1, \ldots, du_n$.  \begin{proof}
                Without loss of generality, $x \subset \A^N$. Then $I(X) =
                (F_1, \ldots, F_m)$, so $k[X]$ is generated by $f_i = T_i|_X$,
                so $\Omega[X]$ is generated by $dt_1, \ldots, dt_N$. We know
                $F_i|X = 0$, so $\sum_j \frac{\partial F_i}{\partial T_j} dt_j
                = 0$ for all $i$. The Jacobian matrix has rank $N-n$ at all
                nonsingular $x \in X$, so it has a nondegenerate minor with the
                last variables.

                Thus $t_1, \ldots, t_n$ are local parameters of $x$. From We
                get that $dt_j = \sum_{i=1}^n f^i_j dt_i$ for all $j > n$ using
                Cramer's rule, where $f_j^i \in \mc{O}_X$. Now we shrink $X$ to
                an affine neighborhood $U$ such that all $f_j^i \in k[U]$.
            After this shrinking, $\Omega[U]$ is generated by $dt_1, \ldots,
        dt_n$ and $t_1, \ldots, t_n$ are local parameters for all $y \in U$.
    \end{proof} \end{thm}

        \subsection{Lecture 25 (Apr 30)} There will be extra office hours
        tomorrow from $2$ to $4$.

        We will continue our study of multivariable calculus. Last time we
        discussed regular differential forms. 
        
        \begin{exm} We calculate $\Omega^1[\P^1]$. We know $\Omega^1[\A^1] =
            k[x] dx$.  Take another chart with coordinate $y = 1/x$, we can
            write $\omega = f(x) dx = f \left( \frac{1}{y} \right) d
            \frac{1}{y} = \frac{g(y)}{y^n} \left( -\frac{dy}{y^2} \right) =
            \frac{g(y) dy}{y^{n+2}}$ where $g(y) \neq 0$. This is not a
        polynomial multiple of $dy$, so $\Omega^1[\P^1] = 0$.  \end{exm}

        Now we talk about differential $r$-forms. We introduce $\Phi^r[X] = \{
        x \in X \mapsto \omega(x) \in \land^r T_x^*\}$ and $\Omega^r[X]$ as the
        subspace that can be written locally as $\omega = \sum f_{i_1, \ldots,
        i_r} dg_{i_1} \land \cdots \land dg_{i_r}$.

        \begin{thm} Let $X$ be nonsingular at $x$ and $U \ni x$ an affine
            neighborhood with a system of local parameters $u_1, \ldots, u_n$.
            Then $\Omega^r[U]$ is a free $k[U]$-module generated by
            $\binom{n}{r}$ forms of the form $du_{i_1} \land \cdots \land
            du_{i_r}$.  \end{thm}

        Proof of this is the same as for $\Omega^1$. Now we discuss the case of
        $\Omega^n[X]$ the space of canonical differentials. Here we have
        $\Omega^n[u] = k[U] du_1 \land \cdots \land du_n$. If $u_1', \ldots,
        u_n'$ is another system of local parameters, then $du_1 \land \cdots
        \land du_n = \mc{J}(u_1, \ldots, u_n, u_1', \ldots, u_n') du_1' \land
        \cdots \land du_n'$ where $J$ is an invertible function.

        We discuss rational canonical differentials $\Omega^n(X)$ where every
        pair is a form $(U,\omega)$ where $U \subset x$ is open and $\omega \in
        \Omega^n[U]$. They are subject to equivalence by equality on
        intersections. We can assume from now on that $X$ is nonsingular and
        $\omega \in \Omega^n(X)$ is regular on some affine open with local
        parameters $u_1, \ldots, u_n$, so $\omega = f(x) du_1 \land \cdots
        \land du_n$ where $f \in k(X)$ is regular on $U$.

        \begin{lem} $\Omega^n(X)$ is a one-dimensional vector space over $k(X)$
        generated by $du_1, \ldots, du_n$ for all choices of local parameters
    somewhere.  \end{lem}

        \begin{defn} Choose $\omega \in \Omega^n(X)^*$. We will define
            $(\omega)$ as a Cartier divisor. Cover $X = \cup U_{\alpha}$ such
            that for all alpha there exist $u_1^{\alpha}, \ldots, u_n^{\alpha}$
            are local parameters on $U_{\alpha}$. Then $\omega|_{U_{\alpha}} =
            f^{\alpha} du_1^{\alpha} \land \cdots \land du_n^{\alpha}$ for
            $f^{\alpha} \in k(X)$. This gives a Cartier divisor $(U_{\alpha},
            f_{\alpha})$. 
            
            To see this is Cartier, on the overlaps $U_{\alpha} \cap U_{\beta}$
            we see that $\omega|_{U_{\alpha} \cap U_{\beta}} = f^{\alpha}
            du_1^{\alpha} \land \cdots \land du_n^{\alpha} = f^{\alpha'}
            du_1^{\alpha'} \land \cdots \land du_n^{\alpha'} = f^{\alpha}
        \mc{J} du_1^{\alpha'} \land \cdots \land du_n^{\alpha'}$, so
    $f^{\alpha} = \mc{J} f^{\alpha'}$, where the Jacobian is invertible.
\end{defn}

        Note that if $\omega' = f\omega$ then $(\omega') = (f) + (\omega)$, so
        they are equivalent. Thus the class of $K_X$ in the Picard group only
        depends on $X$.

        \begin{exm} We calculate the canonical divisor of $\P^n$. Let $\A^n =
            \{ z_0 \neq 0 \}$ and choose $\omega = dx_1 \land \cdots \land
            dx_n$. This has no zeroes or poles on $\A^1$ and let $H = \P^n
            \setminus \A^n$ be the plane at infinity. We see that $[1:x_1:
            \cdots: x_n] = [1/x_1: 1 : x_2/x_1 : \cdots : x_n/x_1]$, so writing
            $y_1 = 1/y_1, x_2 = y_2, \ldots, x_n = y_n/y_1$, we have
            \begin{align*} \omega &= d(1/y_1) \land d(y_2/y_1) \land \cdots
                \land d(y_n/y_1) \\ &= -\frac{dy_1}{y_1^2} \land
                \frac{(dy_2)y_1 - y_2 dy_1}{y_1^2} \land \cdots \land
                \frac{y_1dy_n - y_ndy_1}{y_1^2} \\ &= \pm \frac{dy_1}{y_1^2}
                \land \frac{dy_2}{y_1} \land \cdots \land \frac{dy_n}{y_1} \\
                                                   &= \pm \frac{1}{y_1^{n+1}}
                                                   dy_1 \land \cdots \land dy_n
            \end{align*} Thus $K_{\P^n} = -(n+1)H$.  \end{exm}

        \begin{defn} $X$ is Fano is $-K_X$ is ample.  \end{defn}

        \begin{cor} $\P^n$ is Fano. $\varphi_{-K_X} = \varphi_{(n+1)H}$ is the
        $(n+1)$-th Veronese embedding of $\P^n$.  \end{cor}

        For example, $-K_{\P^2} = 3H$. and $\abs{-K_{\P^2}}$ is the set of
        cubic curves in $\P^2$. We now determine the relationship between $K_S$
        and $K_X$ if $S$ is a blowup of $X$ at a point $p$. Choose local
        parameters $u,v$ at $p$ and $\omega = du \land dv$ on $X$. Then
        $(\omega)$ is disjoint from $p$, and we know $k(S) = k(X)$, so $\omega
        \in \Omega^n(S)$. We compute its divisor on $S$.

        Working in a chart of the blowup with local parameters $x,y$ where $u =
        x, v = xy$, we have $\omega = du \land dv = dx \land (x dy + y dx) = x
        dx \land dy$. Near $E$, $(\omega)$ is precisely $E$. This proves the
        following:

        \begin{thm} Let $S \overset{\pi}{\to} X$ be the blowdown. Then $K_S =
        \pi^*K_X + E$, where $E$ is the exceptional divisor.  \end{thm}

        \begin{cor} Let $S$ be a blowup of $\P^2$ at $6$ points. Then $K_S =
        \pi^* K_{\P^2} + E_1 + \cdots + E_6 = -3H + E_1 + \cdots + E_6$ and
    thus $-K_S = 3H - E_1 - \cdots - E_6$.  \end{cor}

        Last time, we constructed the map $\varphi:S \to \P^3$. We have the
        following:

        \begin{thm} $\varphi$ is an isomorphism onto its image, which is a
            smooth cubic surface.  \begin{proof} We already know that $\varphi$
                is birational and bijective away from $E_1, \ldots, E_6$. Also,
                $\varphi|_{E+i}$ is a line in $\P^3$. Therefore $\varphi$ has
                finite fibers. We claim that $\varphi$ is finite. Instead of a
                proof of the claim, we state the Stein Factorization Theorem
                (after the end of this proof).

                Given Stein, $\varphi$ is finite and birational, so we need to
                check that $Y$ is normal. Note $X$ is smooth, so it is normal.
                We use Serre's criterion. If $X$ is a hypersurface in $\P^n$
                then $X$ is Cohen-Macaulay and is thus $S1$ by Hartog's
                principle. Now we need to show that $X = \varphi(S) \subset
                \P^3$ has isolated singularities. If not, then $X$ is singular
                along $C \subset X$, so every plane section $X \cap H$ is
                singular at $C \cap H$. Thus $X \cap H$ is a singular cubic
                curve, which means it is rational. However, on $S$, $X \cap H$
                is a member of $\abs{3H} - p_1 - \cdots - p_6$, which for some
            $H$ is a smooth cubic curve, which is not rational.  \end{proof}
        \end{thm}

        \begin{cor} $-K_S$ is very ample, so $S$ is a Fano surface, or a del
        Pezzo surface.  \end{cor}
        
        \begin{thm}[Stein Factorization Theorem] Let $f: X \to Y$ be a proper
            regular map. Then there exists a commutative diagram \begin{center}
                \begin{tikzcd} X \arrow{rr} \arrow{dr} & & Y \\ & Z \arrow{ur}
                \end{tikzcd} \end{center} where $X \to Z$ has conncted fibers
                and $Z \to Y$ is finite. In particular, if $f$ is proper and
            has finite fibers, then $f$ is finite.  \end{thm}
        
    
\end{document}
