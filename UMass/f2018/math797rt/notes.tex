\documentclass[leqno, openany]{memoir}
\setulmarginsandblock{3.5cm}{3.5cm}{*}
\setlrmarginsandblock{3cm}{3.5cm}{*}
\checkandfixthelayout

\usepackage{amsmath}
\usepackage{amssymb}
\usepackage{amsthm}
%\usepackage{MnSymbol}
\usepackage{bm}
\usepackage{accents}
\usepackage{mathtools}
\usepackage{tikz}
\usetikzlibrary{calc}
\usetikzlibrary{automata,positioning}
\usepackage{tikz-cd}
\usepackage{forest}
\usepackage{braket} 
\usepackage{listings}
\usepackage{mdframed}
\usepackage{verbatim}
\usepackage{physics}
\usepackage{ytableau} 
\usepackage{dynkin-diagrams}
%\usepackage{/home/patrickl/homework/macaulay2}

%font
\usepackage[sc]{mathpazo}
\usepackage{eulervm}
\usepackage[scaled=0.86]{berasans}
\usepackage{inconsolata}
\usepackage{microtype}

%CS packages
\usepackage{algorithmicx}
\usepackage{algpseudocode}
\usepackage{algorithm}

% typeset and bib
\usepackage[english]{babel} 
\usepackage[utf8]{inputenc} 
\usepackage[T1]{fontenc}
\usepackage[backend=biber, style=alphabetic]{biblatex}
\usepackage[bookmarks, colorlinks, breaklinks]{hyperref} 
\hypersetup{linkcolor=black,citecolor=black,filecolor=black,urlcolor=black}

% other formatting packages
\usepackage{float}
\usepackage{booktabs}
\usepackage{enumitem}
\usepackage{csquotes}
\usepackage{titlesec}
\usepackage{titling}
\usepackage{fancyhdr}
\usepackage{lastpage}
\usepackage{parskip}

\usepackage{lipsum}

% delimiters
\DeclarePairedDelimiter{\gen}{\langle}{\rangle}
\DeclarePairedDelimiter{\floor}{\lfloor}{\rfloor}
\DeclarePairedDelimiter{\ceil}{\lceil}{\rceil}


\newtheorem{thm}{Theorem}[chapter]
\newtheorem{cor}[thm]{Corollary}
\newtheorem{prop}[thm]{Proposition}
\newtheorem{lem}[thm]{Lemma}
\newtheorem{conj}[thm]{Conjecture}
\newtheorem{quest}[thm]{Question}

\theoremstyle{definition}
\newtheorem{defn}[thm]{Definition}
\newtheorem{defns}[thm]{Definitions}
\newtheorem{con}[thm]{Construction}
\newtheorem{exm}[thm]{Example}
\newtheorem{exms}[thm]{Examples}
\newtheorem{notn}[thm]{Notation}
\newtheorem{notns}[thm]{Notations}
\newtheorem{addm}[thm]{Addendum}
\newtheorem{exer}[thm]{Exercise}

\theoremstyle{remark}
\newtheorem{rmk}[thm]{Remark}
\newtheorem{rmks}[thm]{Remarks}
\newtheorem{warn}[thm]{Warning}
\newtheorem{sch}[thm]{Scholium}


% unnumbered theorems
\theoremstyle{plain}
\newtheorem*{thm*}{Theorem}
\newtheorem*{prop*}{Proposition}
\newtheorem*{lem*}{Lemma}
\newtheorem*{cor*}{Corollary}
\newtheorem*{conj*}{Conjecture}

% unnumbered definitions
\theoremstyle{definition}
\newtheorem*{defn*}{Definition}
\newtheorem*{exer*}{Exercise}
\newtheorem*{defns*}{Definitions}
\newtheorem*{con*}{Construction}
\newtheorem*{exm*}{Example}
\newtheorem*{exms*}{Examples}
\newtheorem*{notn*}{Notation}
\newtheorem*{notns*}{Notations}
\newtheorem*{addm*}{Addendum}


\theoremstyle{remark}
\newtheorem*{rmk*}{Remark}

% shortcuts
\newcommand{\Ima}{\mathrm{Im}}
\newcommand{\A}{\mathbb{A}}
\newcommand{\F}{\mathbb{F}}
\newcommand{\R}{\mathbb{R}}
\newcommand{\C}{\mathbb{C}}
\newcommand{\Z}{\mathbb{Z}}
\newcommand{\Q}{\mathbb{Q}}
\renewcommand{\k}{\Bbbk}
\renewcommand{\P}{\mathbb{P}}
\newcommand{\M}{\overline{M}}
\newcommand{\g}{\mathfrak{g}}
\newcommand{\h}{\mathfrak{h}}
\newcommand{\n}{\mathfrak{n}}
\renewcommand{\b}{\mathfrak{b}}
\newcommand{\ep}{\varepsilon}
\newcommand*{\dt}[1]{%
   \accentset{\mbox{\Huge\bfseries .}}{#1}}
\renewcommand{\abstractname}{Official Description}
\newcommand{\mc}[1]{\mathcal{#1}}
\newcommand{\ol}[1]{\overline{#1}}
\newcommand{\wt}[1]{\widetilde{#1}}
\newcommand{\wh}[1]{\widehat{#1}}
\newcommand{\T}{\mathbb{T}}
\newcommand{\mf}[1]{\mathfrak{#1}}
\newcommand{\mb}[1]{\mathbb{#1}}
\newcommand{\mr}[1]{\mathrm{#1}}
\newcommand{\Vect}{\mathsf{Vec}}


\DeclareMathOperator{\Irr}{\mathrm{Irr}}
\DeclareMathOperator{\Rep}{\mathsf{Rep}}
\DeclareMathOperator{\Der}{Der}
\DeclareMathOperator{\Hom}{Hom}
\DeclareMathOperator{\Char}{char}
\DeclareMathOperator{\End}{End}
\DeclareMathOperator{\ad}{ad}
\DeclareMathOperator{\Aut}{Aut}
\DeclareMathOperator{\Rad}{Rad}
\DeclareMathOperator{\supp}{supp}
\DeclareMathOperator{\Lie}{Lie}
\DeclareMathOperator{\sgn}{sgn}

% Section formatting
\titleformat{\section}
    {\Large\sffamily\scshape\bfseries}{\thesection}{1em}{}
\titleformat{\subsection}[runin]
    {\large\sffamily\bfseries}{\thesubsection}{1em}{}
\titleformat{\subsubsection}[runin]{\normalfont\itshape}{\thesubsubsection}{1em}{}

\title{COURSE TITLE}
\author{Lectures by INSTRUCTOR, Notes by NOTETAKER}
\date{SEMESTER}

\newcommand*{\titleSW}
    {\begingroup% Story of Writing
    \raggedleft
    \vspace*{\baselineskip}
    {\Huge\itshape Math 797RT \\ Representation Theory}\\[\baselineskip]
    {\large\itshape Notes by Patrick Lei,
                    May 2020}\\[0.2\textheight]
    {\Large Lectures by Ivan Mirkovic, Fall 2018}\par
    \vfill
    {\Large \sffamily University of Massachusetts Amherst}
    \vspace*{\baselineskip}
\endgroup}
\pagestyle{simple}

\chapterstyle{ell}


%\renewcommand{\cftchapterpagefont}{}
\renewcommand\cftchapterfont{\sffamily}
\renewcommand\cftsectionfont{\scshape}
\renewcommand*{\cftchapterleader}{}
\renewcommand*{\cftsectionleader}{}
\renewcommand*{\cftsubsectionleader}{}
\renewcommand*{\cftchapterformatpnum}[1]{~\textbullet~#1}
\renewcommand*{\cftsectionformatpnum}[1]{~\textbullet~#1}
\renewcommand*{\cftsubsectionformatpnum}[1]{~\textbullet~#1}
\renewcommand{\cftchapterafterpnum}{\cftparfillskip}
\renewcommand{\cftsectionafterpnum}{\cftparfillskip}
\renewcommand{\cftsubsectionafterpnum}{\cftparfillskip}
\setrmarg{3.55em plus 1fil}
\setsecnumdepth{subsection}
\maxsecnumdepth{subsection}
\settocdepth{subsection}

\begin{document}
    
\begin{titlingpage}
\titleSW
\end{titlingpage}

\thispagestyle{empty}
\section*{Disclaimer}%
\label{sec:disclaimer}

These notes are a May 2020 transcription of handwritten notes taken during lecture in Fall 2018.
Any errors are mine and not the instructor's. 
In addition, my notes are picture-free (but will include commutative diagrams) and are a mix of my mathematical style 
(omit lengthy computations, use category theory) and that of the instructor.
If you find any errors, please contact me at \texttt{plei@umass.edu}.

\subsection*{Additional Disclaimers}%
\label{sub:additional_disclaimers}

\begin{enumerate}
    \item Ivan was in China for a week, and I missed one of the two makeup lectures. If there are any gaps in the material, this is probably a big reason why.
    \item I did not take notes for the final lecture. I believe only the classification of semisimple Lie algebras was given.
    \item The course has been redesigned since I took it, so these notes may not be useful. In fact, I'm told that there is no set syllabus for this course, so it may continue to change in the future.
    \item Ivan has notes at \url{https://people.math.umass.edu/~mirkovic/C.Graduate/TopicsCourses/7.RepresentationTheory/8.pdf}, but his notes are much more extensive and in depth than the lectures. My notes are a more faithful representation of what actually occurred during lecture.
    \item Ivan's lectures assumed knowledge far beyond the stated prerequisites of the course, which were just graduate algebra. It may be clear that I did not understand what was going on half the time.
\end{enumerate}
\newpage



\tableofcontents

\chapter{Groups}%
\label{cha:groups}

\section{Overview}%
\label{sec:overview}

We begin with an overview of various aspects of representation theory.
\begin{enumerate}
    \item First, the principle that any subject should be organized by its symmetries (harmonic analysis). We want to spot symmetries in a problem, and this is related to physics and QFT.
    \item We want to consider symmetries related to subtle actions of space. For example, for a scheme $X$ and sufficiently nice group scheme $G$, we can consider the quotient stack $X / G$. The projection from $X$ is a principle $G$-bundle, and in fact, we can do this for $X = \operatorname{Spec} k$, which is a point. Also, we will need the idea of the fiber product $X \times_Z Y$ later.
    \item Groups are used to describe objects that are locally trivial but not globally (for example cohomology classes).
    \item Groups can be recovered from their categories of representations via Tannaka-Krein duality by considering automorphisms of the forgetful functor $\Rep G \to \Vect$.
    \item The most important class of groups is the compact Lie groups.
\end{enumerate}

Recall the notion of a group action on a set $X$. If we view $X \in \mc{C}$ for some category $\mc{C}$, then we want our group $G$ to act by automorphisms in $\mc{C}$. Therefore we can define topological group actions on topological spaces and actions of groups on vector spaces.

\begin{lem}
    For a group $G$ and vector space $V$, then an action of $G$ on $V$ is equivalent to a morphism $\pi: G \to GL(V)$.
\end{lem}

\begin{exm}
    If $G$ acts on a set $X$, then the space $\mc{O}(X) = \Hom(X,k)$ of functions on $X$ is a representation of $G$.
\end{exm}

We can generalize the above example to where $X$ has additional structure, where we can consider functions in various categories. For example, if we consider $G = SL_2(\R)$ acting on the upper half plane $\mc{H}$, then we can consider the space $\mc{O}_{\mc{H}}(\mc{H})$ of holomorphic functions on $\mc{H}$.

\section{Category of Representations}%
\label{sec:category_of_representations}

\begin{lem}
    \begin{enumerate}
        \item Representations of $G$ on vector spaces over a field $k$ form a category $\Rep_G(k) = \Rep(G)$. Note that morphisms are 
            \[ \Hom_{\Rep G}(U,V) = \qty{\varphi: U \to V \mid g_V \circ \varphi = \varphi \circ g_U \text{ for all } g \in G}. \]
        \item $\Rep_G(k)$ is an abelian category.

            Recall that an abelian category is a category with the following:
            \begin{itemize}
                \item A zero object $0$. Here, just take the zero representation.
                \item A direct sum (coproduct) operation $\oplus$. Here, take $g(u \oplus v) = gu \oplus gv$.
                \item Subobjects, which in this case are $G$-invariant subspaces.
                \item Quotients by subobjects. This is obvious.
                \item $\Rep G$ has all (co)kernels and (co)images, and images and coimages are isomorphic.
            \end{itemize}
            Some examples of abelian categories are categories of modules over a ring, but in fact $\Rep G$ is equivalent to $k[G]\text{-}\mathsf{Mod}$.
        \item $\Rep G$ is a monoidal category. For module categories, this is only true in general for commutative rings. For two representations $U,V$ of $G$, we can define an action on $U \otimes V$ by $g(u \otimes v) = gu \otimes gv$.

            Recall that a monoidal structure on a category $\mc{C}$ is an associative and unital functor $\mc{C} \times \mc{C} \to \mc{C}$. The unit in $\Rep G$ is $k$ with the trivial action.
        \item $\Rep(G)$ is a closed monoidal category. This means that there is an internal Hom functor 
            \[ - \otimes -: \underline{\Hom} \colon C^{\mr{op}} \times C \to C \] 
            that is the right adjoint to the tensor product.
    \end{enumerate}
\end{lem}

\begin{exms}
    We give some examples of closed monoidal categories. 
    \begin{enumerate}
        \item In the category of sets, we see that
            \[ \Hom(A \times B, C) \simeq \Hom(A, \Hom(B,C)) \]
            by the maps
            \[ f \coloneqq A \times B \to C \leftrightarrow (a \mapsto f(a,-)). \]
        \item In the category of vector spaces, we again note that
            \[ \Hom(A \otimes B, C) \simeq \Hom(A, \Hom(B,C)) \]
            by the same currying morphism.
    \end{enumerate}
\end{exms}

We can check that the internal Hom in the category of representations is simply the vector space Hom. To see that this is a representation, define the action $(g \varphi)(u) = g \varphi(g^{-1}u)$.

Recall that a right adjoint $G: B \to A$ to a functor $F:A \to B$ satisfies
\[ \Hom_B(Fa, b) \simeq \Hom_A(a,Gb) \]
as sets in a manner that is natural in $a,b$. Being a closed monoidal category means that $- \otimes b$ is a left adjoint to $\underline{\Hom}(b,-)$.

\section{Representations of Finite Groups}%
\label{sec:representations_of_finite_groups}

First we consider invariants of $V \in \Rep G$, which correspond to morphisms from the trivial representation.

\begin{lem}
    We have the two identities $\Hom_G(k,V) = V^g$ and $\Hom_k(U,V)^G = \Hom_G(U,V)$.
\end{lem}

\begin{defn}
    A representation $V \in \Rep G$ is \textit{irreducible} if $V \neq 0$ and has no nontrivial subrepresentations.
\end{defn}

Then define $\Irr_G$ to be the set of isomorphism classes of irreducible representations of $G$. We will use the following strategy to understand $\Rep G$:
\begin{enumerate}
    \item Find all irreducible representations;
    \item Study each irreducible representation;
    \item For any interesting representation, decompose it into irreducibles.
\end{enumerate}

For a finite group $G$, consider the set of functions $\mc{O}(G) = \Hom(G,k)$. Then we have two subrepresentations $\mc{O}_+(G) \coloneqq \mc{O}(G)^G$ and $\mc{O}_-(G) = \qty{f: G \to k \mid \sum_{g \in G} f(g) = 0}$.

\begin{lem}
    If the characteristic of $k$ does not divide $\abs{G}$, then $\mc{O}(G) \simeq \mc{O}_+(G) \oplus \mc{O}_-(G)$.
\end{lem}

\begin{proof}
    The dimensions match up, so the only constant satisfying $c\abs{G} = 0$ is $c = 0$ by assumption.
\end{proof}

\begin{exm}
    If $G = S_2 = \qty{e,a}$, then $\mc{O}_{\pm}(S_2)$ are irreducible, but they are actually the same subrepresentation of $\mc{O}(S_2)$ over $\F_2$.
\end{exm}

\begin{defn}
    A representation $V$ is \textit{semisimple} if it is a direct sum of irreducibles.
\end{defn}

\begin{thm}[Schur Lemma]
    Let $U,V$ be two irreducible finite-dimensional representations of $G$.
    \begin{enumerate}
        \item If $0 \neq \alpha \in \Hom_G(U,V)$, then $\alpha$ is an isomorphism;
        \item $\End_G(U)$ is a finite-dimensional division algebra;
        \item If $k = \ol{k}$, then $\End_G(U) = k$.
    \end{enumerate}
\end{thm}

\begin{proof} \hfill
    \begin{enumerate}
        \item Note that $\ker \alpha$ is a subrepresentation and $U$ is irreducible, so $\ker \alpha = 0$. Similarly, the image is nonzero and is a subrepresentation of $V$, so it must be all of $V$.
        \item Any nonzero endomorphism must be an isomorphism.
        \item If $\alpha \in \End_G(U)$ is nonzero, it has an eigenvalue $\lambda$ with eigenvector $v$. Then $\ker(\alpha - \lambda)$ is a nonzero subrepresentation, so it must be everything.
    \end{enumerate}
\end{proof}

\begin{cor}
    \hfill
    \begin{enumerate}
        \item $V$ is semisimple if and only if it can be written as a direct sum
            \[ V = \bigoplus_{U \in \Irr(G)} M_U \otimes U. \]
        \item Let $U$ be irreducible. Then $\Hom_G(U,V) = \Hom_G(U,U) \otimes M_U$.
    \end{enumerate}
\end{cor}

\begin{proof}
    \hfill
    \begin{enumerate}
        \item Note that if $V$ is a direct sum of irreducibles, then we have
            \[ V = \bigoplus_{U \in\Irr(G)} \qty( \bigoplus_{i} U_i ) = \bigoplus U \otimes M_u. \]
        \item Hom is an additive functor, so just use Schur's Lemma.
    \end{enumerate}
\end{proof}

\begin{cor}
    If $k$ is algebraically closed, then $\Hom_G(U,V) = M_U$ for $V$ semisimple and $U$ irreducible.
\end{cor}

\begin{thm}
    Let $G$ be abelian and $k$ be algebraically closed. Then all finite-dimensional irreducible representations of $G$ are one-dimensional.
\end{thm}

\begin{proof}
    Let $V \in \Irr(G)$ and $g \in G$. Then $g$ has an eigenvalue $\lambda$ with eigenvector $v$. Therefore, $\ker(g - \lambda) = V$ is a subrepresentation because $G$ is abelian, so all subspaces are invariant and thus $\dim V = 1$.
\end{proof}

This tells us that $\Irr(G) = \Hom(G, k^*) \eqqcolon \wh{G}$. This is known as the group of \textit{characters}\footnote{Note there is another notion of character that will appear in this course.} of $G$.

\begin{lem}
    Subrepresentations of semisimple representations are semisimple.
\end{lem}

\begin{proof}
    Let $V' \subseteq V$ be a subrepresentation of an irreducible representation. Then $V'$ has an irreducible subrepresentation $V_1'$ and thus $0 \neq \Hom(V_1', V) = \bigoplus_{i \in I} \Hom(V_1', V_i)$, so there exists $i$ such that $\Hom(V_1', V_i) \neq 0$, so $V_1' \subseteq V$. Then $V' / V_1' \subseteq V / V_1' = \bigoplus_{j \neq i} V_j$, so we have an exact sequence
    \[ 0 \to V_1' \to V' \to V'/V_1' \to 0. \]
    Therefore $V = V_i \oplus C$, so $V' = (V_i \oplus C) \cap V'$.
\end{proof}

\begin{thm}
    \hfill
    \begin{enumerate}
        \item Sums of semisimple representations are semisimple.
        \item $V$ is semisimple iff and only if for every subrepresentation $V' \subseteq V$, there is another subrepresentation $V''$ such that $V = V' \oplus V''$.
    \end{enumerate}
\end{thm}

\begin{proof}
    We will only prove the second part (the first part is obvious). Note that if the condition holds, then choose an irreducible subrepresentation $V'$, apply the condition, and use induction on the dimension.

    Now suppose $V$ is semisimple. Then apply Lemma 1.14 to $V' = \bigoplus \Hom_G(U,V') \otimes U \subseteq \bigoplus \Hom_G(U,V) \otimes U$ and we see that there is a complement.
\end{proof}

\begin{thm}
    Let $G$ be a final group and $k = \C$. Then every $V \in \Irr^{\mr{fd}}(G)$ has a $G$-invariant Hermitian inner product and is semisimple.
\end{thm}

\begin{proof}
    To construct an invariant inner product $(-,-)$, just take any Hermitian inner product $\ev{-,-}$ and define
    \[ (u,v) = \sum_{g \in G} \ev{gu,gv}. \]
    If we have an inner product, let $W \subseteq V$ be a subrepresentation. Then $W^{\perp}$ is also a subrepresentation, and clearly $W \oplus W^{\perp} = V$, so $V$ is semisimple by Theorem 1.15.
\end{proof}

\begin{thm}[Maschke]
    If $\Char k \nmid \abs{G}$, then all finite-dimensional representatons of $G$ are semisimple.
\end{thm}

We will now study in detail the characters $\Hom(G, k^*)$ of $G$.

\begin{lem}
    \hfill
    \begin{enumerate}
        \item $\Hom(G, k^*)$ is a group with operation the multiplication of functions.
        \item $\Hom(G, k^*) = \Irr^{\mr{1d}}(G)$ because $GL_1(k) = k^*$.
        \item $\Hom(G, k^*) = \Hom(G^{\mr{ab}}, k^*)$.
    \end{enumerate}
\end{lem}

If $A$ is abelian, then we call the group of characters $\wh{A} \coloneqq \Hom(A, k^*)$ the \textit{Pontryagin dual of $A$}.

\begin{exms}
    \hfill
    \begin{enumerate}
        \item Note that $\wh{\Z} = k^*$ because $\mathsf{Ab} = \Z\text{-}\mathsf{Mod}$.
        \item We can see that $\wh{\Z/n\Z} = \mu_n(k)$.
        \item If $k$ is algebraically closed, and $\Char k \nmid n$, then $\wh{\Z/n\Z} \simeq \Z/n\Z$.
    \end{enumerate}
\end{exms}

Note that $\Hom(G, k^*) \times \Hom(H, k^*) \cong \Hom(G \times H, k^*)$ by the categorical properties of the product. Consider the functor $\mathsf{Ab^{op}} \to \mathsf{Ab}$ given by $A \mapsto \wh{A} = \Hom(A, k^*)$. Note that there is a natural\footnote{Yes, a natural transformation.} morphism $\iota: A \to \wh{\wh{A}}$ given by $\iota(g)(\chi) = \chi(g)$ for $g \in A$.

\begin{lem}[Pontryagin Duality]
    If $k$ is an algebraically closed field of characteristic $0$, then $\wh{-}: \mathsf{(Ab^f)^{op}} \to \mathsf{Ab^f}$ is an equivalence of categories. In fact, the natural transformation $\iota$ defined above is an isomorphism.
\end{lem}

\begin{proof}
    This is obviously true for finite cyclic groups, and then use the structure theorem for finite abelian groups and the properties of the product.
\end{proof}

\begin{rmk}
    If $\Char k > 0$, then the result extends to finite abelian group schemes. If we take locally compact abelian groups and $\Hom(A, S^1)$, then this result still holds.
\end{rmk}

Now we will study representations of products of groups. Consider the box product
\[ -\boxtimes-: \Rep G \times \Rep H \to \Rep(G \times H) \]
with the operation the tensor product with $G$ acting from the left and $H$ acting from the right.

\begin{lem}
    Let $W \in \Irr(G \times H)$. Then there exist $V \in \Irr(G)$ and $U \in \Irr(H)$ such that $W = V \boxtimes U$.
\end{lem}

\begin{proof}
    Let $V \in \Rep G$. Then consider the morphism $ev: V \otimes \Hom_G(V, W) \to W$. We claim this is a representation of $H$. To see this, define $(h \varphi)(v) = (1,h)(\varphi v)$, which commutes with the action of $G$ on $W$. Clearly, evaluation is surjective, so we can consider $\ker(ev) = V \otimes K'$ for some representation $k'$ of $H$. Therefore we can write
    \[ W = \frac{V \otimes \Hom(V, W)}{\ker(ev)} = V \otimes \frac{\Hom(V, W)}{K'} = V \otimes H' \]
    for some $H' \in \Irr(H)$ (if $H$ is reducible, then so is the entire representation).
\end{proof}

\begin{lem}
    If $V \in \Irr G$ and $U \in \Irr H$, then $V \boxtimes U \in \Irr G \times H$.
\end{lem}

\begin{proof}
    Consider an irreducible subrepresentation $V' \boxtimes U'$. Then 
    \[ 0 \neq \Hom_{G \times H}(V' \boxtimes U', V \boxtimes U) = \Hom_G(V',V) \otimes \Hom_H(U', U), \]
    so $V' = V$ and $U' = U$.
\end{proof}

\begin{thm}
    The box product gives an isomorphism $\Irr G \times \Irr H \xrightarrow{\boxtimes} \Irr G \times H$.
\end{thm}

Now let $V$ be a representation of $G$ and consider the \textit{matrix coefficient} map
\[ c^V: V \otimes V^k \to \mc{O}(G), \hspace{1cm} c^V_{u \otimes \alpha}(g) = \gen{gu, \alpha}. \]
This is equivariant with respect to $G^2$, where $(g,h)f(x) = f(h^{-1}xg)$. Also, the matrix coefficient map $c^V: V \otimes V^* \to \mc{O}(G)$ is an embedding because $V, V^*$ are both irreducible.

\begin{cor}
    Any finite group $G$ has finitely many irreducible representations.
\end{cor}

\begin{thm}
    Let $k$ be an algebraically closed field of characteristic $0$. Then
    \[ \mc{O}(G) \cong \bigoplus_{V \in \Irr G} V^* \boxtimes V \]
    as a representation of $G^2$
\end{thm}

\begin{proof}
    We compute that as representations of $1 \times G$,
    \[ \Hom_{1 \times G}(V, \mc{O}(G)) \otimes V = \Hom_1(V, k) = V^* \]
    by Frobenius reciprocity. 
\end{proof}

It is not hard to see that an explicit isomorphism can be given by $c^V$, which is the same thing as evaluation.

\section{Characters}%
\label{sec:characters}

From now on, we will always work over the complex numbers. It turns out that each irreducible representation of $G$ gives us one conjugation-invariant function $G \to \C$, the trace. Denote the \textit{character} of a representation $V$ by $\chi_V(g) \coloneqq \Tr_V g$.

\begin{lem}
    \hfill
    \begin{enumerate}
        \item $\chi_{U \oplus V} = \chi_U + \chi_V$;
        \item $\chi_{U \otimes V} = \chi_U \cdot \chi_V$.
    \end{enumerate}
\end{lem}

These results follow from linear algebra of direct sums and tensor products.

\chapter{Representation Theory of the Symmetric Group}%
\label{cha:representation_theory_of_the_symmetric_group}

Now we will construct irreducible representations of the symmetric group. Because irreducible representations correspond to conjugacy classes, there are bijections between the following:
\begin{enumerate}
    \item Irreducible representations of $S_n$;
    \item Conjugacy classes of $S_n$;
    \item Partitions $\lambda$ of $n$;
    \item Young diagrams with $n$ squares;
    \item Conjugacy classes of nilpotent matrices of size $n$;
    \item Finite (length $n$) subschemes of $\A^2$.
\end{enumerate}

We want to construct an irreducible representation for each partition $\lambda$. For a partition $\lambda = (\lambda_1, \ldots, \lambda_m)$, write $S_{\lambda} = S_{\lambda_1} \times \cdots \times S_{\lambda_m}$. We will write $M^{\lambda} = \operatorname{Coind}_{S_{\lambda}}^{S_n} \tau_{\lambda}$, where $\tau$ is the trivial representation. Similarly, define $N_{\lambda} = \operatorname{Coind}_{S_{\lambda}}^{S_n} \sigma_{\lambda}$, where $\sigma$ is the sign representation.

\begin{thm}
    If $C, R$ are conjugate partitions, then there exists a unique irreducible representation $\pi_C$ that is common to $M^R$ and $N^C$.
\end{thm}

Now, we will consider the \textit{dominance order} on the set $\Pi_n$ of partitions of $n$. We say that $\lambda \geq \mu$ if $\lambda_1 \geq \mu_1$, $\lambda_1 + \lambda_2 \geq \mu_1 + \mu_2$, and so on.

We may also consider a more geometric notion. Let $\mc{N}_n$ be the cone of nilpotent $n \times n$ matrices. Then every nilpotent matrix has a Jordan canonical form that corresponds to a shift operator $e_{\lambda}$ (shift left on a Young diagram $\lambda$). Then $\mc{N}_n$ is a stratified space with strata given by orbits $\mc{O}_{\lambda}$ corresponding to partitions $\lambda$. Then we say that $\lambda \geq \mu$ if $\ol{\mc{O}}_{\lambda} \supseteq \mc{O}_{\mu}$.

\begin{lem}
    The dominance and closure orders coincide.
\end{lem}

\section{Geometry of $GL_n$}%
\label{sec:geometry_of_gl_n_}

The correspondence between irreducible representations of $S_n$ and orbits of nilpotent matrices under conjugation by $GL_n$, due to Springer, is the origin of \textit{geometric representation theory}. This also extends to other reductive algebraic groups. For this, we will need to do some geometry.

First recall the \textit{Grassmannian} $Gr_p(V)$ which parameterizes dimension $p$ subspaces of $V$. Then we can consider generalized flag varieties $Gr_{1 \leq p_1 \leq \cdots \leq p_k \leq n}$, which parameterizes flags of subspaces with dimension $p_1, \ldots, p_n$. Finally, we have the \textit{flag variety}
\[ \mc{F} = Gr_{1<2<\ldots<n} = \qty{U_1 \subset U_2 \subset \cdots \subset U_n = V \mid \dim U_i = i}. \]

\begin{lem}
    These spaces are all homogeneous spaces of $GL_n$ which are smooth projective varieties.
\end{lem}

Now recall that the nilpotent cone $\mc{N}_n = \bigsqcup_{\lambda \in \Pi_n} \mc{O}_{\lambda}$ is a disjoint union of smooth pieces. By definition of the dominance order, we know that
\[ \ol{\mc{O}}_{\lambda} = \bigcup_{\mu \leq \lambda} \mc{O}_{\mu}. \]
However, note that $\mc{N}_n$ itself is a singular space, so we should be able to extract information from the singularities.

\begin{defn}
    Let $X$ be a singular variety. Then a \textit{resolution} of $X$ is a proper birational map $\pi: \wt{X} \to X$ from a smooth variety $\wt{X}$.
\end{defn}

\begin{exm}
    We can resolve the variety $(xy = 0) \subseteq \C^2$ by taking the disjoint union of two lines.
\end{exm}

If we consider $\mc{N}_2$, then this is just a quadric cone in $\A^3$, so we can resolve by blowing up the origin to get $\wt{\mc{N}}_2 = \qty{(e,L) \in \mc{N}_2 \times \P^1 \mid e \in L}$.

\begin{lem}
    $\mc{F} = GL_n / B_0$, where $B_0 \subseteq GL_n$ is the subgroup of upper-triangular matrices, also called the \textit{standard Borel}. In addition, $\mc{F}$ has a natural identification with the space of Borel subgroups of $GL_n$.
\end{lem}

\begin{rmk}
    Note that $GL_n$ is a nonabelian reductive group. $G$ has a maximal Borel $B_0$, and we call the subgroup of $B_0$ with $1$s on the diagonal $N_0$. Finally, we write $T_0 = B_0 / N_0$, and this is called the \textit{standard Cartan}, or the \textit{maximal torus}, because $T_0 \simeq \mathbb{G}_m^n$.
\end{rmk}

\section{Springer Theory for $GL_n$}%
\label{sec:springer_theory_for_gl_n_}

We want to resolve $\mc{N}_n$ in general. To do this, we will consider $\mf{g} = M_n(\C) = \mf{gl}_n$. Then define
\[ \wt{\mf{g}} = \qty{(x,F) \in \mf{g} \times \mc{F} \mid x F_i \subseteq F_i}, \]
so we have a natural morphism $\pi: \wt{\mf{g}} \to \mf{g}$ that forgets the flag. Similarly, define
\[ \wt{\mc{N}} = \qty{(x,F) \in \mc{N} \times \mc{F} \mid x F_i \subseteq F_{i-1}}. \]
Again, there is a map $\pi: \wt{\mc{N}} \to \mc{N}$. Write $\mc{F}_x = \pi^{-1} x$.

\begin{lem}
    Let $s \in \mf{g}_{rs}$ be a \textit{regular semisimple} matrix. In other words, $s$ has $n$ distinct eigenvalues. Then $\abs{\mc{F}_s} = n!$ and $\mc{F}_s$ is a torsor for $S_n$.
\end{lem}

Here are some examples of torsors:
\begin{enumerate}
    \item The set of bases of $\C^n$ is a torsor for $GL_n$.
    \item The M\"obius strip with the zero section removed is a torsor for $\R^*$.
    \item Rank $n$ vector bundles over a base space $B$ are the same thing as $GL_n$-torsors over $B$.
\end{enumerate}

If $G$ acts on $X$ and $Y$, we can create a new $G$-set $X \times_G Y \coloneqq (X \times Y) / \Delta G$, where the diagonal action of $G$ is $g(x,y) = (xg^{-1}, gy)$. We can see that $X \times_G Y$ is a $Y$-bundle over $X / G$, so if $X$ is a $G$-torsor, then $X \times_G Y \simeq Y$.

Note that the set of regular semisimple matrices is a Zariski-open subset of $\mf{g}$. We now consider the special fibers:
\begin{enumerate}
    \item Above $x = 0$, the fiber is simply all of $\mc{F}$.
    \item If $x$ corresponds to the Young diagram $\ydiagram{5}$, then $\mc{F}_x$ is a point.
    \item For $n = 3$, the fiber above the Young diagram $\ydiagram{2,1}$ is two copies of $\P^1$ intersecting at a point.
\end{enumerate}

\begin{lem}
    As spaces, $\wt{\mf{g}} = G \times_B \mf{b}$. This means that $\wt{\mf{g}}$ is a vector bundle over $G/B = \mc{F}$. Because $\mf{b}$ is a representation of $B$, $H^0(G/B, \wt{\mf{g}})$ is a representation of $G$.
\end{lem}

\begin{rmk}
    These \textit{Associated Bundle} constructions allow us to construct $G$-equivariant vector bundles from representations of $B$. If we consider the character $(m,n)$ that sends $(\C^*)^2 \ni (a,b) \mapsto a^m b^n$ as a representation $\C_{m,n}$ of $B$, then we can define the vector bundle $V_{m,n} = G \times_B \C_{m,n}$. Then we can consider the representation $L_{m,n} = H^0(G/B, V_{m,n})$. Then each $L_{m,n}$ is either zero or irreducible or zero and all irreducible representations of $G$ are of this type. This is the \textit{Borel-Weil theorem}.
\end{rmk}


\begin{exm}
    For $G = GL_2$, then it is easy to see that $G / B = \P^1$.
\end{exm}

\begin{lem}
    As spaces, $\wt{\mc{N}} = G \times_B \mf{n} = T^* \mc{F}$.
\end{lem}

\begin{lem}
    If $G$ acts on $X$, then any $G$-equivariant map $X \to G/A$ identifies $X$ with an associated bundle $G \times_A X_{eA}$.
\end{lem}

Note that there is a natural symplectic structure on $T^* \mc{F}$ and that $\wt{\mc{N}} \to \mc{N}$ is a resolution. Here, over a regular nilpotent matrix, the map is an isomorphism and the map is proper because fibers $\mc{F}$ is projective.

We will call the partition $(n-1,1)$ \textit{subregular} and the partition $2, 1^{n-2}$ minimal. In general, these classes exist for all semisimple Lie algebras and we can perform the Springer construction for all of them. We now guess that $\dim \pi_{\lambda}$ is the number of irreducible components of $\mc{F}_{\lambda}$.

\begin{rmk}
    Semisimple Lie algebras are the same thing as Dynkin diagrams, which are displayed below:
    \begin{align*}
        A_n = \dynkin A{} & & E_7 = \dynkin E7 \\
        D_n = \dynkin D{} & & E_8 = \dynkin E8 \\
        E_6 = \dynkin E6  & &  \\
        B_n = \dynkin B{} & & G_2 = \dynkin G2 \\
        C_n = \dynkin C{} & & F_4 = \dynkin F4
    \end{align*}
\end{rmk}

\begin{thm}
    All irreducible components of Springer fibers are equidimensional.
\end{thm}

We now guess that $\pi_{\lambda} = \C[\Irr(\mc{F}_{\lambda})]$. However, it is not clear what the action $S_n$ is. For this, we will need a continuity argument. For this, we will need sheaf cohomology.

\begin{thm}
    $\mc{F}_{\lambda}$ is paved. This means that it is a disjoint union of affine spaces.
\end{thm}

Being paved means that it is easy to compute the cohomology with coefficients in $\Z$.

\section{Sheaves}%
\label{sec:sheaves}

A \textit{presheaf} on a space $X$ valued in a category $\mc{C}$ is a presheaf $X^{\mr{op}} \to \mc{C}$. A presheaf $\mc{F}$ is a \textit{sheaf} if for an open set $U$ and a cover $U_i$ with sections $f_i \in \mc{F}|_{U_i}$ such that $f_i |_{U_i \cap U_j} = f_j |_{U_i \cap U_j}$, then there exists a unique section $f \in \mc{F}|_{U}$ such that $f|_{U_i} = f_i$.

Let $f:X \to Y$ be a morphism of spaces. Then there is a \textit{pushforward}
\[ (f_* \mc{F}) (V) = \mc{F}(f^{-1} V) \]
and a \textit{pullback}
\[ (f^* \mc{G} (U)) = \lim_{\substack{\gets \\ f(U) \subseteq V}} \mc{G}(V). \]
There operations define functoriality for categories of sheaves.

If $\mc{C}$ is an abelian category, then so is $Sh(X, \mc{C})$, so we may consider complexes of sheaves. Then we may consider the derived category of complexes. In this setting, we have derived versions of $f_*, f^*$ as well as the compactly supported $Rf_!$ and exceptional inverse image functor $Rf^!$.

Given a sheaf $\mc{F}$ and a point $x$, the \textit{stalk} $\mc{F}_x$ is defined as the colimit
\[ \mc{F}_x = \lim_{\substack{\to \\ x \in U}} \mc{F}|_{U}. \]
Alternatively, given the inclusion $\iota: x \to X$, the stalk is simply $\iota^* \mc{F}$. Then we can define the \textit{support} of a sheaf as the closure of the set of points with nonzero stalks.

\begin{rmk}
    The derived category is not an abelian category, but it is an \textit{triangulated category}. This means we have exact triangles
    \[ A \to B \to C \to A[1]. \]
\end{rmk}

There is a \textit{Verdier duality}\footnote{Ivan says this was probably invented by Grothendieck because it required a great mathematician} functor $\mb{D}_X: D(X) \to D(X)^{\mr{op}}$ such that $\mb{D}f_* \mb{D} = f_!$ and $\mb{D} f^* \mb{D} = f^!$.

\begin{exm}
    If $a: X \to \mr{pt}$, then $a^* \mc{F} = \mc{F}_x$ and $a^! k = \omega_X$, the dualizing sheaf.
\end{exm}

Under base change, we have the standard push-pull trick, and if $f$ is proper, then $f_! = f_*$. If $f$ is a smooth morphism, then $f^! = f^* [d_y - d_x] \otimes \mr{or}_f$, where $\mc{or}$ is the orientation sheaf.

Note that cohomology is simply the derived functor of global sections, so we will treat everything as a derived functor.

\begin{defn}
    A \textit{local system} on $X$ is a locally constant sheaf, and a sheaf $\mc{F}$ is \textit{constructible} if there exists a stratification $X = \bigsqcup S_i$ such that $\mc{F}|_{S_i}$ is a local system.
\end{defn}

\begin{rmk}
    All of the derived category and functoriality constructions apply in the case of constructible sheaves.
\end{rmk}

\section{Springer Fibers Again}%
\label{sec:springer_fibers_again}

Returning to our representation theory, we want to produce $a_* a^! \C$ from our setup. Therefore, we will push the constant sheaf $\C$ forward from $\wt{\mf{g}}$ to $\mf{g}$. The resulting complex of sheaves is called the \textit{Grothendieck sheaf}.

\begin{rmk}
    Note that $S_n$ is embedded in $GL_n$ and is embedded in the normalizer of the standard Cartan. In addition, $S_n \cong N_G(T) / T$.
\end{rmk}

If $\pi: \mf{\wt{g}} \to \mf{g}$ is the projection that forgets the flag, denote the Grothendieck sheaf by $\mc{G} \coloneqq \pi_* k$. Then the stalk $\mc{G}_x$ is the same as the cohomology $H^*(\mc{F}_x)$. We know that $S_n$ acts on $\mc{G}_{\mr{rs}} = \mc{G} | _{\mf{g}_{\mr{rs}}}$, so we need to extend this action. 

\begin{prop}
    $\mc{G}$ is obtained from its restriction to the regular semisimple matrices by a procedure called ``intersection cohomology extension,'' and this works because $\pi$ is ``small,'' which means that the fibers increase slowly in some sense.
\end{prop}

\begin{defn}
    A morphism $\pi: Y \to X$ is \textit{semi-small} if $d_x = \dim \pi^{-1} x$ is generically zero and the closed set where $d_x \geq k$ has codimension at least $2k$. The morphism $\pi$ is \textit{small} if it is semi-small and the set of points with $d_x > 0$ has codimension at least $3$.
\end{defn}

\begin{thm}
    The morphism $\pi: \wt{\mf{g}} \to \mf{g}$ is small.
\end{thm}

This means that $S_n$ acts on the entire Grothendieck sheaf $\mc{G}$, so in particular, it acts on the stalks. This tells us that each stalk is a representation of $S_n$.

\begin{lem}
    There is an isomprhism $\C[S_n] \simeq \End(\mc{G}) = H_{\mr{top}}(\wt{mf{g}} \times_{\mf{g}} \wt{\mf{g}})$.
\end{lem}

We conclude from our geometric construction that
\begin{enumerate}
    \item Interesting objects have geometric realizations.
    \item Geometric methods then give construction of irreducible representations.
    \item This allows us to study $\Irr$ more efficiently.
\end{enumerate}

\chapter{Lie Algebras}%
\label{cha:lie_algebras}

\section{Basic Notions}%
\label{sec:basic_notions}

First, we will define manifolds over $k = \R, \C$. These are topological spaces $M$ locally isomorphic to $k^n$ with transition functions that are ($C^k$, analytic, holomphic). An example of a manifold is a Riemann surface. Vector fields are simply derivations of $\mc{O}_M$ and the tangent space at $a$ is simply the derivations at that point.

\begin{exm}
    If $M = k^n$, then the space of vector fields is simply $V(M) = \sum \mc{O}_M(M) \pdv{x_i}$ and the tangent spaces are $T_aM = \bigoplus k \eval{\pdv{x_i}}_a$.
\end{exm}

The tangent sheaf is locally free, so it is the sheaf of sections of the \textit{tangent bundle}. If $\mc{O}_{M,x}$ is the local ring at the point $x$ and $M_x$ is the maximal ideal, the cotangent space is isomorphic to $M_x / M_x^2$.

\begin{defn}
    A \textit{Lie Group} is a group object in the category of manifolds.
\end{defn}

\begin{exm}
    The group $GL_n$ is a Lie group, given by the equation $\det g \cdot z - 1 = 0$ in ambient space $M_n(k) \times \A^1_{z}$. Similarly, the group $SL_n$ is given by $\det g = 1$, and the group $Sp(V)$ is the set of invertible matrices that preserve the symplectic form. Finally, the group $O(V)$ is the group of matrices that preserve a nondegenerate symmetric bilinear form.
\end{exm}

\begin{rmk}
    A \textit{symplectic form} is a nondegenerate alternating bilinear form.
\end{rmk}

\begin{defn}
    A \textit{Lie algebra} over $k$ is a vector space $\mf{g}$ together with an anticommutative product $[-,-]: \mf{g} \otimes \mf{g} \to \mf{g}$ that satisfies the Jacobi identity.
\end{defn}

\begin{exm}
    Any associative algebra can be made into a Lie algebra with the commutator as Lie bracket. Similarly, if $A$ is an associative algebra, then $\Der A$ is a Lie algebra with the commutator as Lie bracket.
\end{exm}

\begin{exm}
    Let $G$ be a Lie group. Then the space $T_e G$ is a Lie algebra and is the same as the left-invariant vector fields. We will denote this Lie algebra by $\operatorname{Lie} G$.
    \begin{enumerate}
        \item It is easy to see that $\Lie GL_n = \mf{gl}_n = M_n$.
        \item The Lie algebra $\mf{sl}_n$ of $SL_n$ is simply the set of trace zero matrices.
        \item The Lie algebra $\mf{sp}_{2n}$ is the set of matrices where $\omega(xu,v) + \omega(u, xv) = 0$.
        \item The \textit{orthogonal Lie algebra} $\mf{o}_n$ is the set of matrices $x$ where $x + x^T = 0$.
    \end{enumerate}
\end{exm}

\begin{thm}
    There is an exponential map $\exp: \mf{g} \to G$ such that
    \begin{enumerate}
        \item $\exp(0) = e$;
        \item $\dd_0 \exp: \mf{g} \to \mf{g}$ is the identity. In particular, $\exp$ is an isomorphism near $0$ and $e$;
        \item $\eval{\dv{s}}_{s=0} \exp(sx) \exp(sy) \exp(sx)^{-1} \exp(sy)^{-1} = [x,y]$. This tells us that the Lie algebra commutator is an infinitesimal version of the group commutator.
    \end{enumerate}
\end{thm}

\begin{thm}
    We will see that $\mf{g}$ controls $G_e / Z(G_e)$.
    \begin{enumerate}
        \item Any morphism $\pi: G \to G'$ of Lie groups is determined on $G_e$ by $\dd_e \pi: \mf{g} \to \mf{g}'$.
        \item If $\tau: \mf{g} \to \mf{g}'$ is a map of Lie algebras and $\pi_1(G) = 0$, then $\tau = \dd_e \pi$ for a unique map of groups $\pi: G \to G'$.
    \end{enumerate}
\end{thm}

The universal cover of any connected Lie group is still a Lie group, so we can preserve most of the information by studying simply connected Lie groups.

\section{Representation Theory and Classification}%
\label{sec:representation_theory_and_classification}

A representation of a Lie group is a morphism $\pi: G \to GL_n$ and a representation of a Lie algebra is a morphism $\mf{g} \to \mf{gl}_n$.

\begin{lem}
    Representations of $G$ differentiate to representations of $\mf{g}$. In particular, if $G$ is simply connected, differentiation is an equivalence of categories.
\end{lem}

A Lie algebra is \textit{simple} if it has no nontrivial ideals $\mf{a}$ such that $[\mf{g}, \mf{a}] \subseteq \mf{a}$. A solvable Lie algebra is a Lie algebra where the series $[\mf{g}, \mf{g}], [[\mf{g}, \mf{g}], [\mf{g}, \mf{g}]], \ldots$ eventially arrives at $0$.

We will now state the classification of finite-dimensional semisimple Lie algebras over $\C$. These are simply sums of simple Lie algebras, so we simply need to classify the simple Lie algebras, which are simply given by Dynkin diagrams:

\begin{align*}
    A_n = \dynkin A{} & & E_7 = \dynkin E7 \\
    D_n = \dynkin D{} & & E_8 = \dynkin E8 \\
    E_6 = \dynkin E6  & &  \\
    B_n = \dynkin B{} & & G_2 = \dynkin G2 \\
    C_n = \dynkin C{} & & F_4 = \dynkin F4
\end{align*}

The correspondence is given by the vertices corresponding to \textit{simple roots} in $\mf{g}$.

\begin{defn}
    A connected Lie group is \textit{semisimple} if its Lie algebra is semisimple.
\end{defn}

Our slogan will be that $\mf{sl}_2 \in \mf{sl}_n \subset \text{semisimple} \subset \text{Kac-Moody}$. A Kac-Moody Lie algebra is a generalization of the semisimple or affine Lie algebras. An important example of such an object is the \textit{loop group}. For a real manifold $M$, the \textit{loop space} of $M$ is $LM = \operatorname{Map}(S^1, M)$. For a complex manifold $\mc{M}$, then $LM = \Hom(\C^*, \mf{M})$, and finally if $X$ is an algebraic variety, $LX = \Hom(\A^1 \setminus 0, X)$. 

If $G$ is a Lie group, then $LG$ is an infinite-dimensional version of a Lie group and $\Lie(LG) = L \Lie(G)$. If $G$ is an algebraic group, then $L(G)(R) = G(R((t)))$ for any ring $R$, where we think of $G$ and $LG$ as functors.

Now we will consider the representation theory of $SL_n$. Note that $GL_1(\C) = \mb{G}_m = S^1 \times \R$. The torus is a complex Lie group $T \simeq \mb{G}_m^n$. We will consider a maximal torus $T \subseteq SL_n$. There are also called \textit{Cartans}.

\begin{thm}
    All maximal tori are conjugate to the set of diagonal matrices.
\end{thm}

We will denote the set of roots of $T$ in $G = SL_n$ by $\Delta$. We will aim to describe $\mf{g}$ in terms of $\mf{t} = \Lie T$.

\begin{lem}
    Let $U$ be a compact Lie group. Then it has a left-invariant Haar measure that is unique up to constants and $\Rep^{\mr{fd}}(U, \C)$ is semisimple.
\end{lem}

\begin{proof}
    We need to find a left-invariant differential form. Fortunately, the tangent and cotangent bundles are trivial, so in particular, the canonical bundle is trivial.

    Once we have a Haar measure, choose an inner product $h_0$. Then we can write an invariant inner product by
    \[ h(x,y) = \int_U h_0(u^{-1}x, u^{-1}y) \dd{\mu(u)}. \]
    Then we finish the proof as in the case of finite groups.
\end{proof}

\begin{cor}
    Let $G$ be a connected complex Lie group with a compact real form $U$. Then let $V \in \Rep G$ and $h$ be the $U$-invariant hermitian inner product on $V$. Then $h$ is invariant under $G$.
\end{cor}

\begin{proof}
    Because $G$ is connected, we can transfer the problem to the Lie algebra. The condition that $h(gx, gy) = h(x,y)$ becomes $h(\alpha x, y) + h(x, \alpha y) = 0$. The other direction is true when $g \in \exp(\mf{g})$ because $\exp(\mf{g})$ generates $G$. Therefore, we can reduce the problem to $\mf{u}$ because $\mf{g}$ is its complexification. However, we can finally reduce from $\mf{u}$ to $U$, so we are done.
\end{proof}

\begin{exm}
    Let $T = (\C^*)^n$, Then $T_{\R} = (S^1)^n$ is compact.
\end{exm}

\begin{lem}
    For any finite dimensional representation $V$ of a torus, 
    \[ V = \bigoplus_{\chi \in X^*(T)} \chi \otimes [V: \chi]. \]
    Here $X^*(T) \coloneqq \Hom(T, \C^*)$.
\end{lem}

Note that any character $\chi: \C^* \to \C^*$ is given by $z \mapsto z^n$ for some $n \in \Z$, so we see that $\chi^*(T) \simeq \Z^n$.

Because $SL_n(\C)$ has a compact real form $SU(n)$, this means that $\Rep SL_n$ is a semisimple category. Then $\mf{g} = \mf{u} + i \mf{u}$, and there is an isomorphism $G = U \times \mf{u}$ given by $(u,x) \mapsto u \exp(x)$. Therefore $\pi_1(G) = \pi_1(U)$. In particular, for $n=2$, $SU(2) = S^3$, so it is simply connected. We also have a fiber sequence as below:
\begin{equation*}
\begin{tikzcd}
    SU(n-1) \arrow[hookrightarrow]{r} & SU(n) \arrow[twoheadrightarrow]{d} \\
                                      & S^{2n-1}
\end{tikzcd}
\end{equation*}

Given a maximal torus $T$, its Lie algebra $\mf{t}$ acts on $V \in \Rep T$. If $T$ acts by $\chi$, then $\mf{t}$ acts by $\dd_e \chi: \mf{t} \to \C$, so $\dd_e \chi \in \mf{t}^*$. We will denote these functionals by $\lambda$, so define
\[ V_{\lambda} = \qty{v \in V \mid s \cdot v = \ev{\lambda,s} v \text{ for all } s \in \mf{t}}. \]

\begin{lem}
    The eigenspaces of $\chi$ and $\dd_e \chi$ are the same. In particular, differentiation is injective.
\end{lem}

\begin{lem}
    For all $x \in \mf{g}$ there exists a unique $\Theta_x: (\R,+) \to G$ such that $d_0(\Theta_x)(1) = x$. We denote $\exp x \coloneqq \Theta_x(1)$.
\end{lem}

For $G = \C^*$, we see that $\exp \tau = e^{\tau}$, so $e^{sz} = \chi_s(e^z)$. If we set $z = 2 \pi i$, then we see that $s \in \Z$. Alternatively, we can differentiate $z \mapsto z^n$ at $z = 1$.

We will define the set of \textit{coroots} by $X_*(T) = \Hom(\C^*, T)$. Then it is easy to see that $X^*(T)$ and $X_*(T)$ are dual lattices under the pairing
\begin{align*}
    X^*(T) \otimes X_*(T) \to \Z & & \text{given by} & & \chi \otimes \eta \mapsto \chi \circ \eta \in \Hom(\C^*, \C^*) = \Z.  
\end{align*}

\begin{cor}
    \hfill
    \begin{enumerate}
        \item Differentiation at $1$ gives an injection $X_*(T) \hookrightarrow \mf{t}$;
        \item Differentiation at the identity gives an injection $X^*(T) \hookrightarrow \mf{t}^*$;
        \item The map $X_*(T) \otimes \C \to \mf{t}$ given by $\eta \otimes z \mapsto z d_1 \eta$ is an isomorphism.
    \end{enumerate}
\end{cor}

We will now study the system of roots. Let $\mb{V} = X^*(T) \otimes \R \eqqcolon \mf{t}^*_{\R}$. Then for $V \in \Rep T$, write 
\[ \mc{W}^{\mf{t}}(V) = \qty{\lambda \in \mf{t}^* \mid V_{\lambda} \neq 0} \subseteq X^*(T) \subseteq \mf{t}^*. \]
A \textit{Cartan subalgebra} of a Lie algebra $\mf{g}$ is a maximal toral subalgebra. Then for all $s \in \mf{t}$, the Lie bracket $[s,-]: \mf{g} \to \mf{g}$ is a semisimple operator.

\begin{lem}
    If $T \subseteq G$ is a Cartan, then $\mf{g} = \bigoplus_{\alpha \in \mc{W}(g)} \mf{g}_{\alpha}$
\end{lem}

\begin{exm}
    In both $GL_n$ and $SL_n$, the maximal torus is the set of diagonal matrices, and for $G = GL_n$, $\mf{t}$ is the set of diagonal matrices. In $SL_n$, the Cartan subalgebra is the set of trace zero diagonal matrices.
\end{exm}

Define the set of roots $\Delta_{\mf{t}}(\mf{g}) = \mc{W}(\mf{g}) \setminus \{0 \}$. This tells us that $\mf{g} = \bigoplus_{x \in \Delta} \mf{g}_{\alpha} \oplus Z_{\mf{g}}(\mf{t})$ and that for $V \in \Rep G$, if $\lambda \in \mc{W}(V)$ and $\alpha \in \mc{W}(\mf{g})$, then $g_{\alpha}V_{\lambda} \subseteq V_{\lambda + \alpha}$.

\section{Representation Theory of $\mf{sl}_n$}%
\label{sec:representation_theory_of_sl_n_}

\begin{lem}
    \hfill
    \begin{enumerate}
        \item For $\mf{g} = \mf{sl}_n$, $\Delta = \qty{\alpha_{ij} = \ep^i - \ep^j \mid i \neq j}$;
        \item The weight spaces are simply $\mf{g}_{\alpha_{ij}} = \C E_{ij}$;
        \item $\mf{g} = \bigoplus_{\alpha \in \Delta} \mf{g}_{\alpha} \oplus \mf{t}$;
        \item For all $\alpha \in \Delta$, $\dim \mf{g}_{\alpha} = 1$;
        \item If $\alpha, \beta, \alpha + \beta \in \Delta$, then $[\mf{g}_{\alpha}, \mf{g}_{\beta}] \subseteq \mf{g}_{\alpha + \beta}$.
    \end{enumerate}
\end{lem}

\begin{lem}
    \hfill
    \begin{enumerate}
        \item If $\alpha, \beta \in \Delta$ but $0 \neq \alpha + \beta \notin \Delta$, then $[\mf{g}_{\alpha}, \mf{g}_{\beta}] = 0$;
        \item $[\mf{g}_{\alpha}, \mf{g}_{-\alpha}] = \C (\alpha_{ij})^{\vee} \subseteq \mf{t}$.
    \end{enumerate}
\end{lem}

In the case $n = 2$, we can write a basis for $\mf{g}$:
\begin{align*}
    e = \mqty(0 & 1 \\ 0 & 0) & & h = \mqty(\dmat[0]{1,-1}) & & f = \mqty(0 & 0 \\ 1 & 0)
\end{align*}
with the identities
\begin{align*}
    [h,e] = 2e & & [h,f] = -2f & & [e,f] = h.
\end{align*}

\begin{lem}
    Any root $\alpha \in \Delta$ defines an $\mf{sl}_2$-subalgebra $S_{\alpha} = \mf{g}_{\alpha} \oplus \mf{g}_{-\alpha} \oplus [\mf{g}_{\alpha}, \mr{g}_{-\alpha}]$.
    \begin{enumerate}
        \item There exists a Lie algebra map $\varphi: \mf{sl}_2 \hookrightarrow \mf{g}$ with $\varphi e \in \mf{g}_{\alpha}$, $\varphi f \in \mf{g}_{-\alpha}$;
        \item The image is a Lie subalgebra;
        \item The image is isomorphic to $\mf{sl}_2$;
        \item $\varphi h$ is independent of $\varphi$.
    \end{enumerate}
\end{lem}

\begin{proof}
    The first three parts are easy. Suppose $\psi e = a \varphi e$ and $\psi f = b \varphi f$. Then 
    \begin{align*}
        2 \psi(e) &= \psi(2e) \\
                  &= \psi([\psi, e]) \\
                  &= [\psi h, \psi e] \\
                  &= a^2 b [\varphi h, \varphi e] \\
                  &= a^2 \cdot 2 \varphi e,
    \end{align*}
    so $ab = 1$.
\end{proof}

\begin{cor}
    We see that $\gen{\alpha, \check{\alpha}} = 2$.
\end{cor}

Now we will consider reflections. If $v \in V$ and $v^* \in V^*$, then if $\gen{v, v^*} = 2$, there is an involution $s_{v,v^*}(x) = x - \ev{v^*, x} v$. Then we see that
\[ s_{\alpha_{ij}} (\ep^k) = \begin{cases}
    \ep^k & k \neq i,j \\
    \ep^j & k = i \\
    \ep^i & k = j
\end{cases}. \]
The group $W$ generated by all reflections is called the \textit{Weyl group} of $\Delta$. For $G = SL_n$, $W = S_n$.

\begin{lem}
    For $w \in W$ and $\alpha \in \Delta$, $w(S_{\alpha}) = S_{w\alpha}$.
\end{lem}

\begin{lem}
    $V$ has a $W$-invariant inner product.
\end{lem}

The inner product is simply the standard inner product on $V_0^* = \bigoplus \R \ep^i = \R^n$. This restricts to $V^*$, so we can compute the inner product of the indices. An inner product of $0$ corresponds to an angle of $\pi/2$, $1$ corresponds to an angle of $\pi/6$, and $2$ corresponds to an angle of $0$.

\begin{prop}
    Define the coroot $\check{\alpha} = \frac{2 \alpha}{(\alpha, \alpha)} \in V \simeq V^*$.
    \begin{enumerate}
        \item For all $\alpha \in \Delta$, $s_{\alpha} \Delta = \Delta$;
        \item For all $\alpha, \beta \in \Delta$, $(\alpha, \check{\beta}) \in \Z$.
        \item $\Delta$ spans $V$.
    \end{enumerate}
\end{prop}

We will now describe the irreducible representations of $\mf{sl}_2$. Note that $\Delta = \qty{\alpha, -\alpha}$ and $X^*(T) = \Z \alpha$. Thus we will see that the irreducible representations of $\mf{sl}_2$ are simply natural numbers, where
\[ h = \mqty(\dmat{n,n-2,\ddots,2-n,-n}). \]
We see that $e$ increses the weight by $2$ and $f$ decreases the weight by $2$. In addition, if $\mqty(\dmat{a,a^{-1}}) \in T$, then 
\[ \mqty(\dmat{a,a^{-1}}) v = \exp \mqty(\dmat{A,-A}) v = e^{Ak} v = a^k v = (k\rho) \mqty(\dmat{a,a^{-1}}) v, \]
where $\rho \mqty(\dmat{a,a^{-1}}) = a$. The representation given above is the only irreducible representation of $\mf{sl}_2$ with highest weight $n$. In general, we have irreducible representations with highest weight $\lambda$.

Note that $\mb{N} \rho \subset \Z \rho$ is a direction in $X^*(T)$, which comes from a choice of Borel, so the roots come from $\mf{n}, \mf{b}$.

For $\mf{sl}_n$, define the \textit{dominant cone} $X^*(T)^+ \subset X^*(T)$ as the set of weights $\lambda \in X^*(T)$ which are positive with respect to all roots in $\mf{n}, \mf{b}$.

\begin{defn}
    In $X^*(T)$, we say that $\lambda \geq \mu$ if $\lambda - \mu$ is in the $\mb{N}$-span of $\Delta^+$.
\end{defn}

\begin{defn}
    In a representation $V \in \Rep \mf{sl}_n$, $v$ is \textit{primitive of weight $\lambda$} if $v \in V_{\lambda}$ for some maximal weight $\lambda$.
\end{defn}

\begin{thm}
    For all $\lambda \in X^*(T)^+$, there exists a unique representation $L$ of $\mf{g}$ such that
    \begin{enumerate}
        \item $L$ has highest weight $\lambda$;
        \item There exists a primitive vector $v$ of weight $\lambda$, where $\mf{n} \cdot v = 0$.
    \end{enumerate}
\end{thm}

\begin{proof}
    We know that
    \begin{align*}
        \mf{n} \cdot v &= \sum \mf{g}_{\alpha} v \\
                       &\subseteq \sum \mf{g_{\alpha}} V_{\lambda} \\
                       &\subseteq \sum V_{\lambda + \alpha}.
    \end{align*}
    Because $\lambda$ is maximal, then $V_{\lambda + \alpha} = 0$.
\end{proof}

\end{document}
