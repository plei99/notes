%!TEX TS-program = lualatex
%!TEX encoding = UTF-8 Unicode

\documentclass[leqno, openany]{memoir}
\setulmarginsandblock{3.5cm}{3.5cm}{*}
\setlrmarginsandblock{3cm}{3.5cm}{*}
\checkandfixthelayout

\usepackage{amsmath}
\usepackage{amssymb}
\usepackage{amsthm}
%\usepackage{MnSymbol}
\usepackage{bm}
\usepackage{accents}
\usepackage{mathtools}
\usepackage{tikz}
\usetikzlibrary{calc}
\usetikzlibrary{automata,positioning}
\usepackage{tikz-cd}
\usepackage{forest}
\usepackage{braket} 
\usepackage{listings}
\usepackage{mdframed}
\usepackage{verbatim}
\usepackage{physics}
%\usepackage{/home/patrickl/homework/macaulay2}

%font
\usepackage{fontspec}
\usepackage{unicode-math}
\setmainfont[Ligatures={Common}, Numbers={OldStyle}]{Libertinus Serif}
\setsansfont{Libertinus Sans}
\setmonofont{Inconsolata}
\setmathfont{Libertinus Math}
\usepackage{microtype}

%CS packages
\usepackage{algorithmicx}
\usepackage{algpseudocode}
\usepackage{algorithm}

% typeset and bib
\usepackage[english]{babel} 
\usepackage[utf8]{inputenc} 
\usepackage[backend=biber, style=alphabetic]{biblatex}
\usepackage[bookmarks, colorlinks, breaklinks]{hyperref} 
\hypersetup{linkcolor=black,citecolor=black,filecolor=black,urlcolor=black}

% other formatting packages
\usepackage{float}
\usepackage{booktabs}
\usepackage{enumitem}
\usepackage{csquotes}
\usepackage{titlesec}
\usepackage{titling}
\usepackage{fancyhdr}
\usepackage{lastpage}
\usepackage{parskip}

\usepackage{lipsum}

% delimiters
\DeclarePairedDelimiter{\gen}{\langle}{\rangle}
\DeclarePairedDelimiter{\floor}{\lfloor}{\rfloor}
\DeclarePairedDelimiter{\ceil}{\lceil}{\rceil}


\newtheorem{thm}{Theorem}[section]
\newtheorem{cor}[thm]{Corollary}
\newtheorem{prop}[thm]{Proposition}
\newtheorem{lem}[thm]{Lemma}
\newtheorem{conj}[thm]{Conjecture}
\newtheorem{quest}[thm]{Question}

\theoremstyle{definition}
\newtheorem{defn}[thm]{Definition}
\newtheorem{defns}[thm]{Definitions}
\newtheorem{con}[thm]{Construction}
\newtheorem{exm}[thm]{Example}
\newtheorem{exms}[thm]{Examples}
\newtheorem{notn}[thm]{Notation}
\newtheorem{notns}[thm]{Notations}
\newtheorem{addm}[thm]{Addendum}
\newtheorem{exer}[thm]{Exercise}

\theoremstyle{remark}
\newtheorem{rmk}[thm]{Remark}
\newtheorem{rmks}[thm]{Remarks}
\newtheorem{warn}[thm]{Warning}
\newtheorem{sch}[thm]{Scholium}


% unnumbered theorems
\theoremstyle{plain}
\newtheorem*{thm*}{Theorem}
\newtheorem*{prop*}{Proposition}
\newtheorem*{lem*}{Lemma}
\newtheorem*{cor*}{Corollary}
\newtheorem*{conj*}{Conjecture}

% unnumbered definitions
\theoremstyle{definition}
\newtheorem*{defn*}{Definition}
\newtheorem*{exer*}{Exercise}
\newtheorem*{defns*}{Definitions}
\newtheorem*{con*}{Construction}
\newtheorem*{exm*}{Example}
\newtheorem*{exms*}{Examples}
\newtheorem*{notn*}{Notation}
\newtheorem*{notns*}{Notations}
\newtheorem*{addm*}{Addendum}


\theoremstyle{remark}
\newtheorem*{rmk*}{Remark}

% shortcuts
\newcommand{\Ima}{\mathrm{Im}}
\newcommand{\A}{\mathbb{A}}
\newcommand{\R}{\mathbb{R}}
\renewcommand{\C}{\mathbb{C}}
\newcommand{\Z}{\mathbb{Z}}
\newcommand{\Q}{\mathbb{Q}}
\newcommand{\N}{\mathbb{N}}
\renewcommand{\k}{\Bbbk}
\renewcommand{\P}{\mathbb{P}}
\newcommand{\M}{\overline{M}}
\newcommand{\g}{\mathfrak{g}}
\newcommand{\h}{\mathfrak{h}}
\newcommand{\n}{\mathfrak{n}}
\renewcommand{\b}{\mathfrak{b}}
\newcommand{\ep}{\varepsilon}
\newcommand*{\dt}[1]{%
   \accentset{\mbox{\Huge\bfseries .}}{#1}}
\renewcommand{\abstractname}{Official Description}
\newcommand{\mc}[1]{\mathcal{#1}}
\newcommand{\T}{\mathbb{T}}
\newcommand{\mf}[1]{\mathfrak{#1}}
\newcommand{\mr}[1]{\mathrm{#1}}
\newcommand{\ol}[1]{\overline{#1}}
\newcommand{\wt}[1]{\widetilde{#1}}


\DeclareMathOperator{\Der}{Der}
\DeclareMathOperator{\Hom}{Hom}
\DeclareMathOperator{\End}{End}
\DeclareMathOperator{\ad}{ad}
\DeclareMathOperator{\Aut}{Aut}
\DeclareMathOperator{\Rad}{Rad}
\DeclareMathOperator{\supp}{supp}
\DeclareMathOperator{\sgn}{sgn}

% Section formatting
\titleformat{\section}
    {\Large\sffamily\scshape\bfseries}{\thesection}{1em}{}
\titleformat{\subsection}[runin]
    {\large\sffamily\bfseries}{\thesubsection}{1em}{}
\titleformat{\subsubsection}[runin]{\normalfont\itshape}{\thesubsubsection}{1em}{}

\title{COURSE TITLE}
\author{Lectures by INSTRUCTOR, Notes by NOTETAKER}
\date{SEMESTER}

\newcommand*{\titleSW}
    {\begingroup% Story of Writing
    \raggedleft
    \vspace*{\baselineskip}
    {\Huge\itshape Complex Analysis \\ Math 621}\\[\baselineskip]
    {\large\itshape Notes by Patrick Lei,
                    June 2020}\\[0.2\textheight]
    {\Large Lectures by Paul Hacking, Spring 2018}\par
    \vfill
    {\Large \sffamily University of Massachusetts Amherst}
    \vspace*{\baselineskip}
\endgroup}
\pagestyle{simple}

\chapterstyle{ell}


%\renewcommand{\cftchapterpagefont}{}
\renewcommand\cftchapterfont{\sffamily}
\renewcommand\cftsectionfont{\scshape}
\renewcommand*{\cftchapterleader}{}
\renewcommand*{\cftsectionleader}{}
\renewcommand*{\cftsubsectionleader}{}
\renewcommand*{\cftchapterformatpnum}[1]{~\textbullet~#1}
\renewcommand*{\cftsectionformatpnum}[1]{~\textbullet~#1}
\renewcommand*{\cftsubsectionformatpnum}[1]{~\textbullet~#1}
\renewcommand{\cftchapterafterpnum}{\cftparfillskip}
\renewcommand{\cftsectionafterpnum}{\cftparfillskip}
\renewcommand{\cftsubsectionafterpnum}{\cftparfillskip}
\setrmarg{3.55em plus 1fil}
\setsecnumdepth{subsection}
\maxsecnumdepth{subsection}
\settocdepth{subsection}

\begin{document}
    
\begin{titlingpage}
\titleSW
\end{titlingpage}

\thispagestyle{empty}
\section*{Disclaimer}%
\label{sec:disclaimer}

These notes are a transcription of handwritten notes that were taken during lecture. 
Any errors are mine and not the instructor's. 
In addition, my notes are picture-free (but will include commutative diagrams) and are a mix of my mathematical style 
(omit lengthy computations, use category theory) and that of the instructor.
If you find any errors, please contact me at \texttt{plei@umass.edu}.
\newpage

\tableofcontents

\chapter{Basics of Complex Analysis}%
\label{cha:basics_of_complex_analysis}

A \textit{complex number} is a sum $z = x + iy$, where $x,y \in \R$ and $i$ is a symbol satisfying the identity $i^2 = -1$. Addition and multiplication work as one would expect. The set $\C$ of complex numbers is a \textit{field}. This means that $(\C,+)$ is an abelian group, $(\C \setminus \qty{0}, \cdot)$ is an abelian group, and that multiplication distributes over addition.

\begin{rmk}
    If $F$ is a field and $f \in F[x]$ is irreducible, then we can construct a field extension $K/F$ such that $K$ has a root of $f$ by setting $K = F[x]/(f)$. In this way, we have $\C = \R[x] / (x^2 + 1)$. The Galois group $\operatorname{Gal} \C/\R$ is generated by complex conjugation.
\end{rmk}

\section{Holomorphic Functions}%
\label{sec:holomorphic_functions}

Let $\Omega \subset \C$ be an open set. Here, the topology is the Euclidean topology on $\C = \R^2$. Then $f: \Omega \to \C$ is \textit{holomorphic} at a point $z_0 \in \Omega$ if the limit
\[ \lim_{h \to 0} \frac{f(z_0+h) - f(z_0)}{h} \]
exists. If it does, we write $f'(z_0)$ for the derivative at $z_0$.

\begin{exm}
    The function $f(z) = \ol{z}$ is \textbf{not} holomorphic. To see this, the difference quotient has different limits on the real and imaginary axes.
\end{exm}

\begin{exm}
    The function $f(z) = z^n$ is holomorphic for $n \in \N$, and $f'(z) = nz^{n-1}$.
\end{exm}

\begin{rmk}
    The usual formulas for differentiation (chain rule, product rule, linearity) hold in this case.
\end{rmk}

We will now compare holomorphic and real differentiability. Rewrite $F = u + iv$. Recall that $F$ is \textit{real differentiable} at $\mathbf{a} = (x_0,y_0)$ if 
\[ \lim_{\mathbf{h} \to 0} \frac{\norm{F(\mathbf{a} + \mathbf{h}) - F(\mathbf{a}) - A \mathbf{h}}}{\norm{\mathbf{h}}} = 0 \]
for some linear map $A$. We say that $A$ is the derivative of $F$ at $\mathbf{a}$. Moreover, $A$ is given by the \textit{Jacobian matrix}
\[ J_F = \begin{pmatrix}
    \pdv{u}{x} & \pdv{u}{y} \\
    \pdv{v}{x} & \pdv{v}{y}
\end{pmatrix}. \]





\end{document}
