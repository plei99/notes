\documentclass{amsart}
\usepackage{amsmath}
\usepackage{amssymb}
\usepackage{amsthm}
%\usepackage{MnSymbol}
\usepackage{bm}
\usepackage{accents}
\usepackage{mathtools}
\usepackage{tikz}
\usetikzlibrary{calc}
\usetikzlibrary{decorations.pathmorphing,shapes}
\usetikzlibrary{automata,positioning}
\usepackage{tikz-cd}
\usepackage{forest}
\usepackage{braket} 
\usepackage{listings}
\usepackage{mdframed}
\usepackage{verbatim}
\usepackage{physics}
\usepackage{stmaryrd}
\usepackage{mathrsfs} 
\usepackage{stackengine} 
%\usepackage{/home/patrickl/homework/macaulay2}

%font
\usepackage[sc]{mathpazo}
\usepackage{eulervm}
\usepackage[scaled=0.86]{berasans}
\usepackage{inconsolata}
\usepackage{microtype}

%CS packages
\usepackage{algorithmicx}
\usepackage{algpseudocode}
\usepackage{algorithm}

% typeset and bib
\usepackage[english]{babel} 
\usepackage[utf8]{inputenc} 
\usepackage[T1]{fontenc}
%\usepackage[backend=biber, style=alphabetic]{biblatex}
\usepackage[bookmarks, colorlinks, breaklinks]{hyperref} 
\hypersetup{linkcolor=black,citecolor=black,filecolor=black,urlcolor=black}
\usepackage{graphicx}
\graphicspath{{./}}

% other formatting packages
\usepackage{float}
\usepackage{booktabs}
\usepackage[shortlabels]{enumitem}
\usepackage{csquotes}
%\usepackage{titlesec}
%\usepackage{titling}
%\usepackage{fancyhdr}
%\usepackage{lastpage}
\usepackage{parskip}

\usepackage{lipsum}

% delimiters
\DeclarePairedDelimiter{\gen}{\langle}{\rangle}
\DeclarePairedDelimiter{\floor}{\lfloor}{\rfloor}
\DeclarePairedDelimiter{\ceil}{\lceil}{\rceil}


\newtheorem{thm}{Theorem}[section]
\newtheorem{cor}[thm]{Corollary}
\newtheorem{prop}[thm]{Proposition}
\newtheorem{lem}[thm]{Lemma}
\newtheorem{conj}[thm]{Conjecture}
\newtheorem{quest}[thm]{Question}

\theoremstyle{definition}
\newtheorem{defn}[thm]{Definition}
\newtheorem{defns}[thm]{Definitions}
\newtheorem{con}[thm]{Construction}
\newtheorem{exm}[thm]{Example}
\newtheorem{exms}[thm]{Examples}
\newtheorem{notn}[thm]{Notation}
\newtheorem{notns}[thm]{Notations}
\newtheorem{addm}[thm]{Addendum}
\newtheorem{exer}[thm]{Exercise}

\theoremstyle{remark}
\newtheorem{rmk}[thm]{Remark}
\newtheorem{rmks}[thm]{Remarks}
\newtheorem{warn}[thm]{Warning}
\newtheorem{sch}[thm]{Scholium}


% unnumbered theorems
\theoremstyle{plain}
\newtheorem*{thm*}{Theorem}
\newtheorem*{prop*}{Proposition}
\newtheorem*{lem*}{Lemma}
\newtheorem*{cor*}{Corollary}
\newtheorem*{conj*}{Conjecture}

% unnumbered definitions
\theoremstyle{definition}
\newtheorem*{defn*}{Definition}
\newtheorem*{exer*}{Exercise}
\newtheorem*{defns*}{Definitions}
\newtheorem*{con*}{Construction}
\newtheorem*{exm*}{Example}
\newtheorem*{exms*}{Examples}
\newtheorem*{notn*}{Notation}
\newtheorem*{notns*}{Notations}
\newtheorem*{addm*}{Addendum}


\theoremstyle{remark}
\newtheorem*{rmk*}{Remark}

% shortcuts
\newcommand{\Ima}{\mathrm{Im}}
\newcommand{\A}{\mathbb{A}}
\newcommand{\G}{\mathbb{G}}
\newcommand{\N}{\mathbb{N}}
\newcommand{\R}{\mathbb{R}}
\newcommand{\C}{\mathbb{C}}
\newcommand{\Z}{\mathbb{Z}}
\newcommand{\Q}{\mathbb{Q}}
\renewcommand{\k}{\Bbbk}
\renewcommand{\P}{\mathbb{P}}
\newcommand{\M}{\overline{M}}
\newcommand{\g}{\mathfrak{g}}
\newcommand{\h}{\mathfrak{h}}
\newcommand{\n}{\mathfrak{n}}
\renewcommand{\b}{\mathfrak{b}}
\newcommand{\ep}{\varepsilon}
\newcommand*{\dt}[1]{%
   \accentset{\mbox{\Huge\bfseries .}}{#1}}
%\renewcommand{\abstractname}{Official Description}
\newcommand{\mc}[1]{\mathcal{#1}}
% \newcommand{\msc}[1]{\mathscr{#1}}
\newcommand{\T}{\mathbb{T}}
\newcommand{\mf}[1]{\mathfrak{#1}}
\newcommand{\mr}[1]{\mathrm{#1}}
\newcommand{\ms}[1]{\mathsf{#1}}
\newcommand{\ol}[1]{\overline{#1}}
\newcommand{\ul}[1]{\underline{#1}}
\newcommand{\wt}[1]{\widetilde{#1}}
\newcommand{\wh}[1]{\widehat{#1}}
\renewcommand{\div}{\operatorname{div}}
\newcommand{\1}{\mathbf{1}}
\newcommand{\2}{\mathbf{2}}
\newcommand{\3}{\mathbf{3}}
\newcommand{\I}{\mathrm{I}}
\newcommand{\II}{\mr{I}\hspace{-1.3pt}\mr{I}}
\newcommand{\III}{\mr{I}\hspace{-1.3pt}\mr{I}\hspace{-1.3pt}\mr{I}}

\DeclareMathOperator{\Der}{Der}
\DeclareMathOperator{\Tor}{Tor}
\DeclareMathOperator{\Hom}{Hom}
\DeclareMathOperator{\End}{End}
\DeclareMathOperator{\Ext}{Ext}
\DeclareMathOperator{\ad}{ad}
\DeclareMathOperator{\Aut}{Aut}
\DeclareMathOperator{\Rad}{Rad}
\DeclareMathOperator{\Pic}{Pic}
\DeclareMathOperator{\supp}{supp}
\DeclareMathOperator{\Supp}{Supp}
\DeclareMathOperator{\depth}{depth}
\DeclareMathOperator{\sgn}{sgn}
\DeclareMathOperator{\spec}{Spec}
\DeclareMathOperator{\Spec}{Spec}
\DeclareMathOperator{\proj}{Proj}
\DeclareMathOperator{\Proj}{Proj}
\DeclareMathOperator{\ord}{ord}
\DeclareMathOperator{\Div}{Div}
\DeclareMathOperator{\Bl}{Bl}
\DeclareMathOperator{\coker}{coker}

\title{Deformations of singularities}
\author{Patrick Lei}
\date{October 15, 2021}

\begin{document}
    
\maketitle

\begin{abstract}
    We will study explicitly the embedded deformations of singular affine schemes via explicit lifting of equations and relations. We prove that embedded deformations of codimension $2$ Cohen-Macaulay closed subschemes are unobstructed. As a corollary, Hilbert schemes of smooth surfaces are smooth. Finally, we give an example of an obstructed deformation.
\end{abstract}

We begin by fixing some notation.
Let $k$ be a field and $R = P/I$, where $P = k[x_1, \ldots, x_n]$ and $I = (f_1, \ldots, f_r)$ is an ideal. Throughout this lecture, we will denote local Artinian rings with residue field $k$ by $A,B,C,\ldots$ and rings by $R,S,T,\ldots$ Finally, denote $Z = \Spec R$.

\section{Explicit criteria for flatness}%
\label{sec:explicit_criteria_for_flatness}

We will study (embedded) deformations of singular affine schemes embedded in $\A^n$. The first thing we want to understand is to explicitly understand flatness of some $R_A$ over $A$, where $R_A \otimes_A k = R$. We will write $R_A = P_A / I_A$, where $P_A = A[x_1, \ldots, x_n] = A \otimes_k P$. Recall that over a Noetherian local ring $S$ with residue field $k$, a module $M$ is flat if and only if it is free, and this is equivalent to $\Tor_1^S(M, k) = 0$ by standard results in commutative algebra.

Now consider the exact sequence
\[ 0 \to I_A \to P_A \to R_A \to 0. \]
After tensoring with $k$, we have 
\[ 0 \to \Tor_1(R_A, k) \to I_A \otimes_A k \to P \to R \to 0. \]
Therefore, we know that $R_A$ is flat over $A$ if and only if $I_A \otimes_A k = I$. We would like to understand this statement.

Consider a presentation
\[ P^s_A \to P_A^r \to I_A \to 0 \]
of $I_A$. Then we know $R_A$ is flat over $A$ if and only if after tensoring with $k$, we obtain an exact sequence
\[ P^s \to P^r \to I \to 0. \]
Note that to give this presentation $P^s \to P^r \to I \to 0$ is the same as giving a complete set of relations among the generators of $I$.

\begin{prop}
    Suppose that 
    \begin{equation} P^s \to P^r \to P \to R \to 0 \end{equation}
    is exact and
    \begin{equation} P_A^s \to P_A^r \to P_A \to R_A \to 0 \end{equation}
    is a complex such that $P_A^r \to R_A \to R_A \to 0$ is exact and tensoring (2) with $k$ gives (1). Then $R_A$ is flat over $A$.
\end{prop}

\begin{proof}
    Note that the hypotheses are equivalent to the fact that all relations in $I$ can be lifted to $I_A$. Now given $g_1', \ldots, g_r' \in P_A$ such that 
    \[ \sum_{i=1}^r g_i' f_i' = 0, \]
    this clearly descends to a relation in $I$ by killing the maximal ideal of $A$. But now if we choose a complete set of relations for $I_A$, this descends to a complete set of relations in $I$, so we may in fact assume that (2) is exact.

    In this case, there exists some $L_A$ such that the sequence splits as
    \[ P_A^s \to L_A \to 0 \qquad 0 \to L_A \to P_A^r \to I_A \to 0 \qquad 0 \to I_A \to P_A \to R_A \to 0. \]
    By right exactness of the tensor product, we know $P_A^{s} \otimes k \to L_A \otimes k \to 0$ is exact. We also know that
    \[ L_A \otimes k \to P_A^r \otimes k \to I_A \otimes k \to 0 \]
    is exact, again by right exactness. But this means that $I_A \otimes k$ is the cokernel of $P^s \to P^r$, and therefore $I_A \otimes k = I$. This means that $R_A$ is flat.
\end{proof}

\begin{cor}
    Let $R = P/I$ and $R_A = P_A / I_A$, where $I = (f_1, \ldots, f_r)$ and $I_A = (f_1', \ldots, f_r')$ such that $f_i'$ is a lift of $f_i$. Then $R_A$ is flat over $A$ if and only if every relation among the $f_i$ lifts to a relation among the $f_i'$.
\end{cor}

\begin{rmk}
    This result essentially gives us that first-order embedded deformations of $\Spec R \subset \A^n$ are given by $\Hom(I, R)$. The first-order (not embedded) deformations of $Z$ are given by the cokernel of
    \[ 0 \to T_X \to T_{\A^n}|_X \to N_{X/\A^n}, \]
    which arises from the exact sequence
    \[ I/I^2 \to \Omega^1_{\A^n}|_X \to \Omega^1_X \to 0, \]
    and this is supported on the singular points of $X$, so when $X$ has isolated singularities, this is finite-dimensional.
\end{rmk}

Note that if $\Spec R \subset \A^n$ is a complete intersection, then $I$ is generated by a regular sequence, so in particular the Koszul complex is a free resolution of $R$ and therefore there are only trivial relations among the $f_i$ (this means the relations are generated by $f_i f_j - f_j f_i = 0$). Clearly, because we are only considering commutative rings (after all, this is normal algebraic geometry), this means that all deformations of $\Spec R$ are unobstructed.

\section{Hilbert schemes of smooth surfaces}%
\label{sec:hilbert_schemes_of_smooth_surfaces}

We will prove that deformations of finite length closed subschemes of $\A^2$ are unobstructed. In particular, this will imply that the Hilbert scheme $\mr{Hilb}(\A^2, n)$ is smooth.

Let $Z \subset \A^2$ be a closed subscheme of dimension $0$. Then because $P = k[x,y]$ has dimension $2$, there exists a free resolution
\[ 0 \to P^s \xrightarrow{(g_{ij})} P^r \to P \to R \to 0 \]
of $R$. In this case it is possible to understand the matrix $(g_{ij})$, and in fact this is the special case of a more general result. First, when we study the local behavior, we have the following result.
\begin{thm}[Hilbert, Burch]
    Let $P$ be a regular local ring of dimension $n$ and $R = P/I$ be a Cohen-Macaulay quotient of codimension $2$. Then there exists an $(r-1) \times r$ matrix $G = (g_{ij})$ whose maximal minors $f_1, \ldots, f_r$ minimally generate $I$, and there is a free resolution
    \[ 0 \to P^{r-1} \xrightarrow{(g_{ij})} P^r \xrightarrow{(f_i)} P \to R \to 0. \]
\end{thm}

\begin{proof}
    Note that the fact that the free resolution has this length is a corollary of the Auslander-Buchsbaum formula, which says that for a ring $R$ and module $M$, we have
    \[ \depth M + \operatorname{proj.dim} M = \depth R \]
    and the fact that depth equals dimension for Cohen-Macaulay things. Thus we have a free resolution
    \[ 0 \to P^{r-1} \xrightarrow{(g_{ij})} P^r \xrightarrow{(a_i)} P \to R \to 0, \]
    where $a_1, \ldots, a_r$ are a minimal set of generators for $I$. Let $f_i$ is ${(-1)}^i$ times the determinant of the $i$-th minor of $g_{ij}$. We will prove that the map $(f_i)$ is the same as the map $(a_i)$; clearly
    \[ 0 \to P^{r-1} \xrightarrow{(g_{ij})} P^r \xrightarrow{(f_i)} P \to R \to 0. \]
    is a resolution. This is because at the generic point of $P$, we know $(g_{ij})$ is injective, so at least one $f_i$ is nonzero. But then we know $\coker (g_{ij})$ is torsion-free (because $I$ is torsion-free), and so it in fact must vanish by rank reasons. Thus $(a_1, \ldots, a_{r})$ and $(f_1, \ldots, f_{r})$ are isomorphic as $P$-modules.

    At a codimension $1$ point in $\Spec P$, note that $0 \to P^{r-1} \to P^r \xrightarrow{(a_i)} P \to B \to 0$ is split exact (because $I$ has codimension $2$). This implies that at least one $f_i$ is a unit, and thus $(f_1, \ldots, f_r)$ has codimension at least $2$. But then the isomorphism $I \cong (f_1, \ldots, f_r)$ is given by multiplication by some nonzero element of $P$ which is a unit away from codimension $2$. But this means it is a unit everywhere.
\end{proof}

Considering the global picture in $\A^n$, we obtain the following result.
\begin{thm}[Hilbert, Schaps]
    Let $Z = \Spec R \subset \A^n$ be a Cohen-Macaulay closed subscheme of codimension $2$. Then $R = P/I$ has a free resolution of the form
    \[ 0 \to P^{r-1} \xrightarrow{(g_{ij})} P^r \xrightarrow{(f_i)} P \to R \to 0 \]
    where the $f_i$ are the maximal minors of the matrix $(g_{ij})$.
\end{thm}

This result in fact holds over any Artinian local ring $A$, which we will use later.

Next, we want to understand what happens if we choose some Artinian local ring with residue field $k$ and lift the $g_{ij}$ to $g_{ij}'$, where $g_{ij}' \in P_A$.

\begin{thm}[Schaps]
    If $A$ is a square zero extension of $k$, then the sequence
    \[ 0 \to P_A^{r-1} \xrightarrow{(g_{ij}')} P_A^r \xrightarrow{(f_i')} P_A \to R_A \to 0 \]
    is exact. Moreover, any lifting of $R$ over $A$ arises by lifting the matrix $(g_{ij})$.
\end{thm}

\begin{proof}
    We know that 
    \[ L_A^{\bullet} \coloneqq P_A^{r-1} \to P_A^r \to P_A \]
    is a complex. This is because composing the two maps amounts to evaluating determinants with a repeated column. Because $P_A$ is free (and therefore flat), we can tensor with the exact sequence
    \[ 0 \to \mf{m}_A \to A \to k \to 0 \]
    to obtain an exact sequence of complexes
    \[ 0 \to L_A^{\bullet} \otimes_A \mf{m}_A \to L_A^{\bullet} \to L_A^{\bullet} \otimes_A k \to 0. \]
    Note that 
    \[ L_A^{\bullet} \otimes_A k = P^{r-1} \xrightarrow{(g_{ij})} P^r \xrightarrow{(f_i)} P \eqqcolon L^{\bullet}. \]
    In particular, this term is exact by Hilbert-Schaps. In addition, clearly $L_A^{\bullet} \otimes_A \mf{m}_A = L^{\bullet} \otimes_k \mf{m}_A$ because $A \to k$ is a square zero extension, so the complex $L_A^{\bullet} \otimes_A \mf{m}_A$ is exact. By the long exact sequence in homology, we know that $L_A^{\bullet}$ is exact. Note that $L^{\bullet}$ extends to an exact sequence
    \[ 0 \to P^{r-1} \to P^r \to P \to R \to 0, \]
    and $L_A^{\bullet}$ extends to an exact sequence
    \[ 0 \to P_A^{r-1} \to P_A^r \to P_A \to R_A \to 0. \]
    However, by the homology long exact sequence, we have an exact sequence
    \[ 0 \to R \otimes_k \mf{m}_A \to R_A \to R \to 0. \]
    But this implies that $R_A \otimes_A k = R$. Finally, by the local criterion for flatness, we see that $R_A$ is flat over $A$.

    Let $R_A = P_A / I_A$ be a lifting of $R$ over $A$. Lift $f_i \in I$ to $h_i \in I_A$. By Nakayama, these generate $I_A$, so we obtain a free resolution
    \[ 0 \to P_A^{r-1} \xrightarrow{(g_{ij}')} P_A^r \xrightarrow{(h_i)} P_A \to R_A \to 0, \]
    where $g_{ij}'$ lift the $g_{ij}$. However, we already have a lift
    \[ 0 \to P_A^{r-1} \xrightarrow{(g_{ij}')} P_A^r \xrightarrow{(f_i')} P_A \to R_A' \to 0, \]
    and so we must show $R_A = R_A'$. But we know that the ideals $I_A = (h_1, \ldots, h_r)$ and $I_A' = (f_1', \ldots, f_r')$ are isomorphic as $P_A$-modules. But then if we restrict this isomorphism to $\A^n_A \setminus \supp B$, we obtain a unit in $H^0(\A^n_A \setminus \supp B, \mc{O}_{\A^n_A})$. Because functions extend over codimension $2$, we have $H^0(\A^n_A \setminus \supp B, \mc{O}_{\A^n_A}) = P_A$, so this is a global unit. This gives the desired result.
\end{proof}

This result holds if we replace $A \to k$ with any square-zero extension of Artinian local rings $B \to A$ and $P, P_A$ with flat things, and so we see that (embedded) deformations of codimension $2$ Cohen-Macaulay subschemes of $\A^n$ are unobstructed. In particular, any dimension $0$ closed subscheme $Z \subset \A^2$ is automatically Cohen-Macaulay (because it is dimension $0$), so its embedded deformations are unobstructed. By some cohomological argument, the tangent space to $\mr{Hilb}(\A^2, n)$ is isomorphic to $\Hom(R, R)$ and has dimension $2n$, so 

\section{An obstructed deformation}%
\label{sec:an_obstructed_deformation}

Let $R = k[x,y,z]/(z^2, xy, xz, yz)$. Note that this scheme has an embedded point at the origin, so in particular it is \textbf{not} Cohen-Macaulay.

\begin{figure}[H]
\begin{center}
\begin{tikzpicture}[scale=1, transform shape]
    \draw[-,thick] (0,-2) -- (0,2);
    \draw[-,thick] (-2,0) -- (2,0);
    \filldraw[black] (0,0) circle (2pt);
    \draw[->] (0,0) -- (0.5,0.5) node[above,right] {$z$};
\end{tikzpicture}
\end{center}
\caption{Drawing of $\Spec R$}%
\label{fig:}
\end{figure}

We will study embedded deformations of $\Spec R$ and see that they are obstructed. In particular, we will choose two deformations of $R$ over $k[\ep]$ that cannot be simultaneously lifted. We claim that a complete set of relations (using the ordering $(xy, xz, yz, z^2)$ for the generators of $I$) is given by the matrix
\[ G = \mqty(z & -y & 0 & 0 \\ z & 0 & -x & 0 \\ 0 & z & 0 & -x \\ 0 & 0 & z & -y). \]

Now a first-order deformation of $\Spec R$ is given by lifting $(xy,xz,yz,z^2)$ over $k[\ep]$, and the first candidate is to consider $I_{\ep_1} = (xy + \ep_1 y, xz, yz, z^2)$. Then we note that
\[ G \mqty(xy+\ep_1 y \\ xz \\ yz \\ z^2) = \ep_1 \mqty(yz \\ yz \\ 0 \\ 0), \]
and we can lift $G$ to kill this vector with the matrix
\[ G_{\ep_1} = \mqty(z & -y & -\ep_1 & 0 \\ z & 0 & -x-\ep_1 & 0 \\ 0 & z & 0 & -x \\ 0 & 0 & z & -y) = G + \mqty(0 & 0 & -\ep_1 & 0 \\ 0 & 0 & -\ep_1 & 0 \\ 0 & 0 & 0 & 0 \\ 0 & 0 & 0 & 0) \eqqcolon G + G_1. \]

Next consider the deformation given by $I_{\ep_2} = (xy, xz, yz+\ep_2 z, z^2)$. We note that
\[ G \mqty(xy \\ xz \\ yz + \ep_2 z \\ z^2) = \ep_2 \mqty(0 \\ -xz \\ 0 \\ z^2), \]
and we can lift $G$ to kill this vector with the matrix
\[ G_{\ep_2} = \mqty(z & -y & 0 & 0 \\ z & \ep_2 & -x & 0 \\ 0 & z & 0 & -x \\ 0 & 0 & z & -y - \ep_2) = G + \mqty(0 & \ep_2 & 0 & 0 \\ 0 & 0 & 0 & 0 \\ 0 & 0 & 0 & -\ep_2 \\ 0 & 0 & 0 & 0) \eqqcolon G + G_2. \]

Now we consider $I_{\ep_1^2, \ep_2^2, \ep_1\ep_2} = (xy+\ep_1 y, xz, yz + \ep_2 z, z^2)$ and attempt to lift this deformation to $k[\ep_1, \ep_2]/(\ep_1^2, \ep_2^2)$. Note that 
\begin{align*} 
    (G + G_1 + G_2) \mqty(xy+\ep_1 y \\ xz \\ yz+\ep_2 z \\ z^2) &= \mqty(z & -y & -\ep_1 & 0 \\ z & \ep_2 & -x-\ep_1 & 0 \\ 0 & z & 0 & -x \\ 0 & 0 & z & -y - \ep_2) \mqty(xy+\ep_1 y \\ xz \\ yz+\ep_2 z \\ z^2) \\
    &= \ep_1 \ep_2 \mqty(-z \\ -z \\ 0 \\ 0),
\end{align*}
and clearly $z \notin I$, so in fact we cannot lift this deformation to $k[\ep_1, \ep_2]/(\ep_1^2, \ep_2^2)$. This proves obstructedness.




\end{document}
