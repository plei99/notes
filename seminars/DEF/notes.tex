\documentclass[leqno, openany]{memoir}
\setulmarginsandblock{3.5cm}{3.5cm}{*}
\setlrmarginsandblock{3cm}{3.5cm}{*}
\checkandfixthelayout

\usepackage{amsmath}
\usepackage{amssymb}
\usepackage{amsthm}
%\usepackage{MnSymbol}
\usepackage{bm}
\usepackage{accents}
\usepackage{mathtools}
\usepackage{tikz}
\usetikzlibrary{calc}
\usetikzlibrary{automata,positioning}
\usepackage{tikz-cd}
\usepackage{forest}
\usepackage{braket} 
\usepackage{listings}
\usepackage{mdframed}
\usepackage{verbatim}
\usepackage{physics}
\usepackage{stmaryrd}
\usepackage{mathrsfs} 
%\usepackage{/home/patrickl/homework/macaulay2}

%font
\usepackage[sc]{mathpazo}
\usepackage{eulervm}
\usepackage[scaled=0.86]{berasans}
\usepackage{inconsolata}
\usepackage{microtype}

%CS packages
\usepackage{algorithmicx}
\usepackage{algpseudocode}
\usepackage{algorithm}

% typeset and bib
\usepackage[english]{babel} 
\usepackage[utf8]{inputenc} 
\usepackage[T1]{fontenc}
\usepackage[backend=biber, style=alphabetic]{biblatex}
\usepackage[bookmarks, colorlinks, breaklinks]{hyperref} 
\hypersetup{linkcolor=black,citecolor=black,filecolor=black,urlcolor=black}

% other formatting packages
\usepackage{float}
\usepackage{booktabs}
\usepackage[shortlabels]{enumitem}
\usepackage{csquotes}
\usepackage{titlesec}
\usepackage{titling}
% \usepackage{fancyhdr}
% \usepackage{lastpage}
\usepackage{parskip}

\usepackage{lipsum}

% delimiters
\DeclarePairedDelimiter{\gen}{\langle}{\rangle}
\DeclarePairedDelimiter{\floor}{\lfloor}{\rfloor}
\DeclarePairedDelimiter{\ceil}{\lceil}{\rceil}


\newtheorem{thm}{Theorem}[section]
\newtheorem{cor}[thm]{Corollary}
\newtheorem{prop}[thm]{Proposition}
\newtheorem{lem}[thm]{Lemma}
\newtheorem{conj}[thm]{Conjecture}
\newtheorem{quest}[thm]{Question}

\theoremstyle{definition}
\newtheorem{defn}[thm]{Definition}
\newtheorem{defns}[thm]{Definitions}
\newtheorem{con}[thm]{Construction}
\newtheorem{exm}[thm]{Example}
\newtheorem{exms}[thm]{Examples}
\newtheorem{notn}[thm]{Notation}
\newtheorem{notns}[thm]{Notations}
\newtheorem{addm}[thm]{Addendum}
\newtheorem{exer}[thm]{Exercise}

\theoremstyle{remark}
\newtheorem{rmk}[thm]{Remark}
\newtheorem{rmks}[thm]{Remarks}
\newtheorem{warn}[thm]{Warning}
\newtheorem{sch}[thm]{Scholium}


% unnumbered theorems
\theoremstyle{plain}
\newtheorem*{thm*}{Theorem}
\newtheorem*{prop*}{Proposition}
\newtheorem*{lem*}{Lemma}
\newtheorem*{cor*}{Corollary}
\newtheorem*{conj*}{Conjecture}

% unnumbered definitions
\theoremstyle{definition}
\newtheorem*{defn*}{Definition}
\newtheorem*{exer*}{Exercise}
\newtheorem*{defns*}{Definitions}
\newtheorem*{con*}{Construction}
\newtheorem*{exm*}{Example}
\newtheorem*{exms*}{Examples}
\newtheorem*{notn*}{Notation}
\newtheorem*{notns*}{Notations}
\newtheorem*{addm*}{Addendum}


\theoremstyle{remark}
\newtheorem*{rmk*}{Remark}

% shortcuts
\newcommand{\Ima}{\mathrm{Im}}
\newcommand{\A}{\mathbb{A}}
\newcommand{\G}{\mathbb{G}}
\newcommand{\N}{\mathbb{N}}
\newcommand{\R}{\mathbb{R}}
\newcommand{\C}{\mathbb{C}}
\newcommand{\Z}{\mathbb{Z}}
\newcommand{\Q}{\mathbb{Q}}
\renewcommand{\k}{\Bbbk}
\renewcommand{\P}{\mathbb{P}}
\newcommand{\M}{\overline{M}}
\newcommand{\g}{\mathfrak{g}}
\newcommand{\h}{\mathfrak{h}}
\newcommand{\n}{\mathfrak{n}}
\renewcommand{\b}{\mathfrak{b}}
\newcommand{\ep}{\varepsilon}
\newcommand*{\dt}[1]{%
   \accentset{\mbox{\Huge\bfseries .}}{#1}}
\renewcommand{\abstractname}{Official Description}
\newcommand{\mc}[1]{\mathcal{#1}}
\newcommand{\T}{\mathbb{T}}
\newcommand{\mf}[1]{\mathfrak{#1}}
\newcommand{\mr}[1]{\mathrm{#1}}
\newcommand{\ms}[1]{\mathsf{#1}}
\newcommand{\ol}[1]{\overline{#1}}
\newcommand{\ul}[1]{\underline{#1}}
\newcommand{\wt}[1]{\widetilde{#1}}
\newcommand{\wh}[1]{\widehat{#1}}
\renewcommand{\div}{\operatorname{div}}
\newcommand{\bir}{\sim_{\mr{bir}}}

\DeclareMathOperator{\Der}{Der}
\DeclareMathOperator{\Bl}{Bl}
\DeclareMathOperator{\NE}{NE}
\DeclareMathOperator{\Hom}{Hom}
\DeclareMathOperator{\End}{End}
\DeclareMathOperator{\ad}{ad}
\DeclareMathOperator{\Aut}{Aut}
\DeclareMathOperator{\Rad}{Rad}
\DeclareMathOperator{\Pic}{Pic}
\DeclareMathOperator{\supp}{supp}
\DeclareMathOperator{\Supp}{Supp}
\DeclareMathOperator{\sgn}{sgn}
\DeclareMathOperator{\spec}{Spec}
\DeclareMathOperator{\Spec}{Spec}
\DeclareMathOperator{\proj}{Proj}
\DeclareMathOperator{\Proj}{Proj}
\DeclareMathOperator{\ord}{ord}
\DeclareMathOperator{\Div}{Div}

% Section formatting
\titleformat{\section}
    {\Large\sffamily\scshape\bfseries}{\thesection}{1em}{}
\titleformat{\subsection}[runin]
    {\large\sffamily\bfseries}{\thesubsection}{1em}{}
\titleformat{\subsubsection}[runin]{\normalfont\itshape}{\thesubsubsection}{1em}{}

\title{COURSE TITLE}
\author{Lectures by INSTRUCTOR, Notes by NOTETAKER}
\date{SEMESTER}

\newcommand*{\titleSW}
    {\begingroup% Story of Writing
    \raggedleft
    \vspace*{\baselineskip}
    {\Huge\itshape Defomation Theory Graduate Student Seminar \\ Spring 2021}\\[\baselineskip]
    {\large\itshape Notes by Patrick Lei}\\[0.2\textheight]
    {\Large Lectures by Various}\par
    \vfill
    {\Large \sffamily Columbia University}
    \vspace*{\baselineskip}
\endgroup}
\pagestyle{simple}

\chapterstyle{ell}


%\renewcommand{\cftchapterpagefont}{}
\renewcommand\cftchapterfont{\sffamily}
\renewcommand\cftsectionfont{\scshape}
\renewcommand*{\cftchapterleader}{}
\renewcommand*{\cftsectionleader}{}
\renewcommand*{\cftsubsectionleader}{}
\renewcommand*{\cftchapterformatpnum}[1]{~\textbullet~#1}
\renewcommand*{\cftsectionformatpnum}[1]{~\textbullet~#1}
\renewcommand*{\cftsubsectionformatpnum}[1]{~\textbullet~#1}
\renewcommand{\cftchapterafterpnum}{\cftparfillskip}
\renewcommand{\cftsectionafterpnum}{\cftparfillskip}
\renewcommand{\cftsubsectionafterpnum}{\cftparfillskip}
\setrmarg{3.55em plus 1fil}
\setsecnumdepth{subsection}
\maxsecnumdepth{subsection}
\settocdepth{subsection}

\begin{document}
    
\begin{titlingpage}
\titleSW
\end{titlingpage}

\thispagestyle{empty}
\section*{Disclaimer}%
\label{sec:disclaimer}

These notes were taken during the seminar using the \texttt{vimtex} package of the editor \texttt{neovim}. 
Any errors are mine and not the speakers'. 
In addition, my notes are picture-free (but will include commutative diagrams) and are a mix of my mathematical style and that of the lecturers.
If you find any errors, please contact me at \texttt{plei@math.columbia.edu}.

\vspace*{1cm}

\noindent\textbf{Seminar Website:}  \url{https://www.math.columbia.edu/~dejong/seminar.html}
\newpage

\tableofcontents

\chapter{Johan (Sep 24): Schlessinger's Paper}%
\label{cha:johan_sep_24_schlessinger_s_paper}

The paper by Schlessinger is titled \textit{Functors of Artin Rings}. Throughout this lecture, $k$ is a field, $\mc{C}$ is the category of Artinian local $k$-algebras $A, B, C, \ldots$ with residue field $k$, and $\wh{\mc{C}}$ is the category of Noetherian complete local $k$-algebras $R, S, \ldots$ with residue field $k$.

\begin{rmk}
    Every $R \in \wh{\mc{C}}$ is of the form $k[[x_1, \ldots, x_n]] / (f_1, \ldots, f_m)$ by the Cohen structure theorem. Then $R \in \mc{C}$ if and only if and only if $(f_1, \ldots, f_m)$ contains ${(x_1, \ldots, x_n)}^N$ for some $N$.
\end{rmk}

\begin{rmk}
    In the paper, there is a more general setup, where $\Lambda$ is a complete local Noetherian ring with residue field $k$. Then $\mc{C}_{\Lambda}, \wh{\mc{C}}$ are defined analogously, which will allow things like $\Lambda = \Z_p$.
\end{rmk}

The idea of deformation theory is to look at functors $F \colon \mc{C} \to \ms{Set}$.

\begin{exm}
    Given $R \in \wh{\mc{C}}$, we set $h_R \colon \mc{C} \to \ms{Set}$ sending $A \mapsto \Hom_{\wh{\mc{C}}}(R, A)$. This is not necessarily representable because $R \notin \mc{C}$ in general, but it is pro-representable.
\end{exm}

\begin{defn}
    A functor $F$ is \textit{pro-representable} if $F \simeq h_R$ for some $R \in \wh{\mc{C}}$.
\end{defn}

\begin{exm}
    Let $M$ be a variety over $k$ and $m \in M(k)$. Then define
    \[ \mr{Def}_{M,m}(A) = \qty{ \Spec A \xrightarrow{m_A} M \mid m_A |_{\Spec k} = m}. \]
    It is easy to see that $\mr{Def}_{M,m}(A)$ is pro-representable by $\wh{\mc{O}}_{M,m}$.
\end{exm}

Observe that $h_R(k) = \qty{*}$ is a singleton. Also note that $h_R(A \times_B C) = h_R(A) \times_{h_R(B)} h_R(C)$. Here, $A \times_B C$ is the fiber product of rings and not the tensor product.

Now consider the following conditions on $F$: let $A \to B \gets C$ be a diagram in $\mc{C}$ and consider the morphism
\[ F(A \times_B C) \xrightarrow{(*)} F(A) \times_{F(B)} F(C). \]
\begin{itemize}
    \item $(H_1)$ The morphism $(*)$ is surjective if $C \twoheadrightarrow B$;
    \item $(H_2)$ The morphism $(*)$ is bijective if $C = k[\ep] \twoheadrightarrow k = B$;
    \item $(H_3)$ $\dim_k(t_F) < \infty$ (later, we will see that we need $H_2$ for formulate this). Here, $t_F$ is the tangent space to $F$;
    \item $(H_4)$ The morphism $(*)$ is bijective if $C \twoheadrightarrow B$.
\end{itemize}

\begin{exm}
    Fix a group $G$ and a representation $\rho_0 \colon G \to GL_n(k)$. Now define
    \[ \mr{Def}_{\rho_0}^{\text{naive}}(A) = \qty{\rho \colon G \curvearrowright A^{\oplus n} \mid \rho \pmod{m_A} \cong \rho_0} /\cong. \]
    Better, we will define
    \[ \mr{Def}_{\rho_0}(A) = \qty{\rho \colon G \curvearrowright A^{\oplus n} \mid \rho \pmod{m_A} = \rho_0} / \ker(GL_n(A) \to GL_n(k)). \]
    In general these functors fail $(H_4)$ and $\mr{Def}_{\rho_0}^{\text{naive}}$ even fails $(H_2)$.

    Namely, if $H = \Z$ and $\rho_0$ is the trivial representation, then for $\mr{Def}_{\rho_0}^{\text{naive}}$, we are looking at subsets of
    \[ GL_n(A \times_B C)/\text{conj} \to GL_n(A)/\text{conj} \times_{GL_n(B)/\text{conj}} GL_n(C)/\text{conj}. \]
    This morphism is always surjective, but in general it is not injective.

    For example, if $A = k[\ep_1], B = k, C = k[\ep_2]$, we can look at elements of the form $1 + \ep_1 T_1 + \ep_2 T_2$ and see that on the left we can only conjugate together, while on the right we can conjugate both $T_1, T_2$ arbitrarily. Here $A \times_B C = k[\ep_1, \ep_2] = k[x_1,x_2] / (x_1^2, x_1 x_2, x_2^2)$.
\end{exm}

\begin{defn}
    A natural transformation $t \colon F \to G$ of functors on $\mc{C}$ is \textit{smooth} if for all surjections $B \twoheadrightarrow A$ the map $F(B) \to F(A) \times_{G(A)} G(B)$ is surjective.
\end{defn}

Note that this is equivalent to the existence of a lift in the diagram below:
\begin{equation*}
\begin{tikzcd}
    \Spec A \ar{r} \ar[hookrightarrow]{d} & M \ar{d}{f} \\
    \Spec B \ar{r} \ar[dashrightarrow]{ur} & N.
\end{tikzcd}
\end{equation*}


This definition is motivated by the following example: let $f \colon M \to N$ be a morphism of varieties over $k$. Let $m \in M(k), n = f(m) \in N(k)$. Then the following are equivalent:
\begin{enumerate}
    \item $\mr{Def}_{M,m} \to \mr{Def}_{N,n}$ is smooth.
    \item $f$ is smooth at $m$.
\end{enumerate}

\begin{defn}
    We say $F$ has a \textit{hull} if and only if $F(k) = \qty{*}$ and there exists a smooth $t \colon h_R \to F$ for some $R \in \wh{C}$ which induces an isomorphism $t_R \cong t_F$.
\end{defn}

Now we will say a bit about tangent spaces.
\begin{enumerate}
    \item When $F(k) = \qty{*}$, then $t_F = F(k[\ep])$.
    \item If $F$ satisfies $(H_2)$ and $F(k) = \qty{*}$, then $t_F$ has a natural $k$-vector space structure. Here, $H_2$ gives $F(k[\ep_1, \ep_2]) \to F(k[\ep]) \times F(k[\ep])$ is a bijection, and then we take $\ep_1 \mapsto \ep, \ep_2 \mapsto \ep$, which defines addition.
    \item $t_R = \Hom_k(\mf{m}_R / \mf{m}_R^2, k) = \Hom_{\wh{C}}(R, k[\ep]) = h_R(k[\ep]) = t_{(h_R)}$.
\end{enumerate}

\begin{thm}[Schlessinger]
    Asssume that $F(k) = \qty{*}$. Then the conditions $(H_1), (H_2), (H_3)$ hold for $F$ if and only if $F$ has a hull. In addition, $(H_3)$ and $(H_4)$ hold if and only if $F$ is pro-representable.
\end{thm}

\begin{proof}[Very rough idea of proof of $\Rightarrow$ for the hull case]
    Let $n = \dim_k(t_F)$. Then $(H_2)$ and $n < \infty$ imply the following: Let $S = k[[x_1, \ldots, x_n]]$ and $\mf{m} = \mf{m}_S = (x_1, \ldots, x_n)$. We can find $\xi_1 \in F(S/\mf{m}^2)$ such that
    \[ t_S = \Hom_{\wh{\mc{C}}}(S, k[\ep]) \xrightarrow{\xi_1} t_F \]
    is an isomorphism.

    Next, we will choose $q \geq 2$ and consider pairs $(J, \xi)$ where $\mf{m}^{q+1} \subset J \subset \mf{m}^2$ and $\xi \in F(S/J)$ such that $\xi \mapsto \xi_1 \in F(S/\mf{m}^2)$. Say that $(J, \xi) \leq (J', \xi')$ if $J \subset J'$ and $\xi \mapsto \xi'$. Choose a minimal pair $(J, \xi)$ for this ordering. We can choose $J_{q}$ so that $\mf{m}^{q+1} + J_{q+1} = J_q$ and $\xi_{q+1}$ maps to $\xi_q$ for bookkeeping purposes.

    Choose $R = \lim S/J_q$, which is a quotient of $S$. Set $t \colon h_R \to F$ given by sending $\varphi \colon R \to A$ to the following: choose $q$ such that $\varphi$ factors as $R \to S/J_q \xrightarrow{\varphi_q} A$ and take $\xi_q \mapsto F(\varphi_q)(\xi_q) \in F(A)$.

    Finally, we must show that $t$ is smooth. Consider the diagram
    \begin{equation*}
    \begin{tikzcd}
        S \ar{dd} \ar{r} \ar[dashrightarrow,bend left=30]{drrr}{\varphi} & S/\mf{m}^{q+1} \ar[dashrightarrow,swap]{dr}{\ol{\varphi}'} \ar[dashrightarrow]{drr}{\ol{\varphi}} \\ 
        & & S/J_q \times_A B \ar{dl}{pr_1} \ar{r} & B \ar[twoheadrightarrow]{d} \\
        R \ar{r} & S/J_q \ar{rr} \ar[dashrightarrow,bend left=20]{ur}{\psi} & & A
    \end{tikzcd}
    \end{equation*}
    with $B \ni \wt{\xi} \mapsto \xi \in A$ and $S/J_q \ni \xi_q \mapsto \xi$. First, choose $\varphi \colon S \to B$ making the diagram commute. We may increase $q$ such that $\varphi(\mf{m}^{q+1}) = 0$, so we now have $\ol{\varphi} \colon S/\mf{m}^{q+1} \to B$. Now consider the fiber product $S/J_q \times_A B$ and $pr_1 \colon S/J_q \times_A B \to S/J_q$, so we obtain $\ol{\varphi}' \colon S/\mf{m}^{q+1} \to S/J_q \times_A B$. By $(H_1)$, we obtain some $\wt{\wt{\xi}} \in F(S/J_q \times_A B)$ mapping to $\wt{\xi}$ and $\xi_q$. We may now assume that $B \to A$ is a small extension, which means that $\dim_k \ker(B \to A) = 1$, and thus $pr_1$ is a small extension. Therefore, either $\ol{\varphi}'$ is surjective or its image maps isomorphically via $pr_1$ to $S/J_q$., so we have $\psi$ which gives $R \to B$ lifting our given $r \to A$.

    The tricky part is to show that $F(\psi)(\psi_q) = \wt{\wt{\xi}}$, and this step is deliberately omitted.
\end{proof}

A generalization of this is as follows. Consider a functor $\mc{F} \colon \mc{C} \to \ms{Grpd}$. We say that $\mc{F}$ satisfies the \textit{Rim-Schlessinger condition} (RS) if
\[ \mc{F}(A \times_B C) \to \mc{F}(A) \times_{\mc{F}(B)} \mc{F}(C) \]
is an equivalence whenever $C \twoheadrightarrow B$. Let $x_0 \in \mc{F}(k)$ and set 
\[ \ol{\mc{F}}_{x_0} \colon \mc{C} \to \ms{Set} \qquad A \mapsto \qty{(x, \alpha) \mid x \in \mc{F}(A), \alpha \colon X_0 \to x|_k} / \cong, \]
where $(x,\alpha) \cong (x', \alpha')$ means that $\varphi \colon x \to x'$ such that the diagram
\begin{equation*}
\begin{tikzcd}
    x|_k \ar{r}{\varphi} & x'|_k \\
    x_0 \ar{r}{\mr{id}} \ar{u}{\alpha} & x_0 \ar{u}{\alpha'}
\end{tikzcd}
\end{equation*}
commutes.

\begin{thm}
    If $\mc{F}$ has (RS) then $\ol{\mc{F}}_{x_0}$ has $(H_1)$ and $(H_2)$. Therefore, if $\dim t_{\ol{\mc{F}}_{x_0}} < \infty$ then $\ol{\mc{F}}_{x_0}$ has a hull.
\end{thm}

In this situation, $\ol{\mc{F}}_{x_0}$ has $(H_4)$ if and only if $\Aut_A(x) \twoheadrightarrow \Aut_B(x|_B)$ whenever $A \twoheadrightarrow B$ and $x \in \mc{F}_{x_0}(A)$.

\begin{exm}
    Let $\mc{F}(A)$ be the category of representations $G \curvearrowright A^{\oplus n}$ with morphisms being isomorphisms of representations. This has (RS).
\end{exm}

\begin{exm}
    Let $\mc{F}(A)$ be the category of smooth projective families of curves of genus $g$ over $A$ with morphisms being isomorphisms. This has (RS).
\end{exm}

Returning to the example of representations, it turns out that $t_{\mr{Def}_{\rho_0}} = H^1(G, M_{n \times n}(k))$, where $G$ acts on $M_{n \times n}(k)$ via $\rho_0$ by conjugation.

\begin{exm}
    Consider $G = \Z \oplus \Z$ and $\rho_0$ to be the trivial representation on $k^{\oplus 2}$. Then $t_{\mr{Def}_{\rho_0}} = H^1(\Z^2, M_2(k)) = M_2(k) \oplus M_2(k)$. Given two matrices $A, B$, we have the representation
    \begin{align*}
        \Z^2 \to GL_2(k[\ep]) & (1,0) \mapsto \mqty(\dmat{1,1}) + \ep A \\
        & (0,1) \mapsto \mqty(\dmat{1,1}) + \ep B.
    \end{align*}
    We get a hull $R$ with $h_R \to \mr{Def}_{\rho_0}$. We know that $R$ is a some quotient of $k[[a_{11}, \ldots, a_{22}, b_{11}, \ldots, b_{22}]]$ with $\rho$ looking like
    \[ (1,0) \mapsto \mqty(\dmat{1,1}) + A \qquad (0,1) \mapsto \mqty(\dmat{1,1}) + B, \]
    and of course $R$ is the quotient of the power series ring by the ideal generated by the coefficients of $AB - BA$.
\end{exm}

\chapter{Ivan and Cailan (Oct 1): Deformations of Schemes}%
\label{cha:ivan_and_cailan_oct_1_deformations_of_schemes}

\section{Deformations of affine schemes}%
\label{sec:deformations}

We are looking for a Cartesian diagram of schemes
\begin{equation*}
\begin{tikzcd}
    X \ar{r} \ar{d} & \mc{X} \ar{d}{\pi} \\
    \Spec k \ar{r} & S
\end{tikzcd}
\end{equation*}
where $\pi$ is flat and surjective and $S$ is surjective. This is called an \textit{deformation} of $X$ over $S$. For the beginning of this lecture (the part given by Ivan), we are interested in $S = \Spec A$, where $A \in \mc{C}^*$ (this category was defined in the previous lecture). This case is called a \textit{local deformation}, and in the face where $A$ is Artinian, it is called an \textit{infinitesimal deformation}.

For the ring theorists, we will make the following digression. Let $A$ be a ring and $I \subset A$ be an ideal with $I^2 = 0$. Suppose that $\ol{B}$ is an $A/I$-algebra, $J$ is an $\ol{B}$-module, and $h \colon I \to J$ is an $A$-module map. Then we are interested in a diagram
\begin{equation*}
\begin{tikzcd}
    0 \ar{r} & J \ar{r} & ? \ar{r} & \ol{B} \\
    0 \ar{r} & I \ar{u}{h} \ar{r} & A \ar{u} \ar{r} & A/I \ar{u} \ar{r} & 0,
\end{tikzcd}
\end{equation*}
which we will call a deformation of $A$. Here are some interesting questions:
\begin{enumerate}
    \item Is such a deformation unique?
    \item If $\ol{B}$ is flat over $A/I$, does that mean that $B$ is flat over $A$?
\end{enumerate}

Returning to the case of schemes, we will say that two deformations $\mc{X}, \mc{X}'$ of $X$ over $S$ are isomorphic if there exists an $S$-isomorphism $\phi \colon \mc{X}' \to \mc{X}$ commuting with the inclusions of the central fibers $X \to \mc{X}, \mc{X}'$.

\begin{exm}
    The most basic example of a family is the trivial deformation
    \begin{equation*}
    \begin{tikzcd}
        X \ar{r} \ar{d} & X \times_k S \ar{d} \\
        \Spec k \ar{r} & S.
    \end{tikzcd}
    \end{equation*}
\end{exm}

\begin{defn}
    A scheme $X$ is \textit{rigid} if all deformations of $X$ are isomorphic to the trivial deformation.
\end{defn}

\begin{thm}
    If $X$ is a smooth affine $k$-scheme and $S = \Spec A$ for some local Artinian ring, then $X$ is rigid.
\end{thm}

\begin{defn}
    A closed immersion $i \colon S_0 \hookrightarrow S$ of schemes is called a \textit{first (resp. nth) order thickening} if the ideal sheaf $\mc{I} = \ker(i^{\flat} \colon \mc{O}_S \to \mc{O}_{S_0})$ satisfies $I^2 = 0$ (resp. $I^{n+1} = 0$).
\end{defn}

\begin{defn}
    A morphism $f \colon X \to S$ is called \textit{formally smooth} (resp. unfamified, resp. \'etale) if for all first order thickenings $i \colon T_0 \to T$ of affine schemes and diagrams
    \begin{equation*}
    \begin{tikzcd}
        T_0 \ar{r}{u_0} \ar{d}{i} & X \ar{d}{f} \\
        T \ar{r} \ar[dashrightarrow]{ur}{\wt{u_0}} & S
    \end{tikzcd}
    \end{equation*}
    there exists a lift $\wt{u_0}$ (resp. there is at most one such $\wt{u_0}$, resp. there exists a unique $\wt{u}_0$).
\end{defn}

\begin{exm}\leavevmode
    \begin{enumerate}
        \item Open immersions are formally \'etale. This is cleaer because $T_0, T$ have the same underlying topological space.
        \item Closed immersions are formally unramified. This is clear because $X \to S$ induces an injection on $T$-points.
        \item $\A^n_S \to S$ is formally smooth. To see this, assume $S = \Spec R$ is affine and then consider the corresponding lifting problem in commutative algebra.
    \end{enumerate}
\end{exm}

\begin{prop}
    The classes of formally smooth (resp. \'etale, resp. unramified) morphisms are closed under base change, composition, and products and local on both source and target.
\end{prop}

\begin{defn}
    A $f \colon X \to S$ is \textit{smooth} if it is formally smooth and locally of finite presentation.
\end{defn}

We will now consider differentials. Let $X = \Spec A$ be an affine scheme over $k$ and choose a $k$-point and consider the diagram
\begin{equation*}
\begin{tikzcd}
    \Spec k \ar{r} \ar{d} & X \ar{d} \\
    \Spec k[\ep] \ar{r} & \Spec k.
\end{tikzcd}
\end{equation*}
If $X$ is smooth, then there exists a lift $\Spec k[\ep] \to X$. But this is given by a morphism
\[ \wt{\phi} \colon A \to k[\ep] / \ep^2 \qquad a \mapsto \phi(a) + d(a) \ep. \]
This motivates the following definition:
\begin{defn}
    Let $R \to A$ be a morphism of rings and $M$ be an $A$-module. A \textit{derivation} $d \colon A \to M$ is an $A$-linear map satisfying the Leibniz rule.
\end{defn}

\begin{prop}
    There exists an $A$-module $\Omega^1_{A/k}$ equipped with a derivation $d \colon \Omega^1_{A/k}$ that is universal among derivations from $A$. This means that all derivations $\wt{d} \colon A \to M$ factor through $d$, and formally, we have an identity
    \[ \mr{Der}_R(A,M) \simeq \Hom_A(\Omega^1_{A/k}, M). \]
\end{prop}

\begin{defn}
    For an $A$-module $M$ with derivation $d \colon A \to M$, define the ring $A[M]$ as the module $A \oplus M$ with the multiplication
    \[ (a,m) \cdot (a', m') = (aa', am' + a'm). \]
    There is a sequence $\phi \colon A \to A[M] \to A$.
\end{defn}

\begin{prop}
    Let $S \gets R \to A \to B$ be a diagram of rings. Then
    \begin{enumerate}
        \item $\Omega^1_{A \otimes S / S} \simeq \Omega^1_{A/R} \otimes_R S$;
        \item The sequence $\Omega^1_{A/R} \otimes_A B \to \Omega^1_{B/R} \to \Omega^1_{B/A} \to 0$ is exact.
        \item If $B = A/I$ for some ideal $I$, we have an exact sequence
            \[ I/I^2 \to \Omega^1_{A/R} \otimes_A B \to \Omega^1_{B/R} \to 0. \]
        \item For all $f \in A$, we havae $\Omega^1_{A[f^{-1}]/R} \simeq \Omega^1_{A/R} \otimes_A A[f^{-1}]$.
    \end{enumerate}
\end{prop}

\begin{rmk}
    If $J = \ker(A \otimes_R A \to A)$, then $\Omega^1_{A/R} = J/J^2$.
\end{rmk}

\begin{thm}
    Let $f \colon X \to S$ be locally of finite presentation. The following are equivalent:
    \begin{enumerate}
        \item $f$ is smooth;
        \item $f$ is flat with smooth fibers;
        \item $f$ is flat and has smooth geometric fibers.
    \end{enumerate}
\end{thm}

We will finally return to deformation theory.

\begin{lem}
    Let $Z_0$ be a closed subscheme of $Z$ determined by a nilpotent ideal sheaf $N$. If $Z_0$ is affine, then so is $Z$.
\end{lem}
Proof of this result can be found in EGA, Chapter I.5.9. 

\begin{proof}[Proof of Theorem 2.1.3]
    Recall that we have a diagram of the form
    \begin{equation*}
    \begin{tikzcd}
        B \ar{r} & B_0 \\
        A \ar{u} \ar{r} & k \ar{u},
    \end{tikzcd}
    \end{equation*}
    where $A \to B$ is flat and $B_0 \simeq B \otimes_A k$ is a smooth $k$-algebra. We need to prove that $B_0 \simeq B \otimes_A k$. The first step is to prove this result for first-order deformations. Suppose that $A = k[\ep]$ is a square-zero extension. 
    \begin{lem}
        For a ring $R$ with $M, N$ flat over $R$, nilpotent ideal $I \subset R$, and $f \colon M \to N$, then if $f \otimes_R R/I$ is an isomorphism, then so is $f$.
    \end{lem}
    To prove the lemma, note that the cokernel of $f$ is preserved by $I$, so it must vanish. Returning to our case, we know that $B$ is a smooth $k[\ep]$-algebra. Now we obtain a square-zero extension $B_0[\ep]$ of $B_0$ and a diagram
    \begin{equation*}
    \begin{tikzcd}
        B \ar{r} & B_0 \\
        k[\ep] \ar{r}{f} \ar{r} & B_0[\ep] \ar{u}
    \end{tikzcd}
    \end{equation*}
    with a lift $B \to B_0[\ep]$. But now by the lemma, we have $B \otimes_{k[\ep]} k = B_0[\ep] \otimes_{k[\ep]} k$. The rest of the proof follows using an inductive argument that was verbalized but now written down.
\end{proof}

\section{Deformations of schemes}%
\label{sec:deformations_of_schemes}

The main theorem of this section is
\begin{thm}
    Assume $X$ is a smooth $R$-scheme. Then there is a bijection
    \[ \mr{Def}_X^{\mr{sm}}(k[I]) \simeq H^1(X, T_{X/k} \otimes I). \]
\end{thm}

\begin{proof}
    Let $\mc{X}'$ be a smooth deformation over $k[I]$. Then the diagram
    \begin{equation*}
    \begin{tikzcd}
        X \ar{r} \ar{d} & X' \ar{d} \\
        \Spec k \ar{r} & \Spec k[I]
    \end{tikzcd}
    \end{equation*}
    is cartesian. Then if $U_k = \Spec B_k$ is an affine cover of $X$ and $U_k' = \Spec D_k$ is an affine cover of $X'$, we have a $k[I]$-linear ring isomorphism
    \[ \varphi_k \colon k[I] \otimes_k B_k \to D_k \qquad (k,i) \otimes b \mapsto s(b) + i. \]
    Modulo $I$, $\varphi_k$ is the identity on $B_k$. Without loss of generality, we may assume that $U_{kj} = U_k \cap U_j$ is a distinguished open for both $U_k$ and $U_j$, so let $U_{kj} = \Spec B_{kj}$ and $U_{kj}' = \Spec D_{kj}$. Now note that both
    \[ \varphi_k, \varphi_j \colon k[I] \otimes_k B_{kj} \to D_{kj} \]
    induce the identity on $B_{kj}$ modulo $I$. Now we have the commutative diagram
    \begin{equation*}
    \begin{tikzcd}
        0 \ar{r} & I \ar{d}{\mr{id}} \ar{r} & D_{kj} \ar{d}{\varphi_j^{-1} \varphi_k} \ar{r} & B_{kj} \ar{d}{\mr{id}} \ar{r} & 0 \\
        0 \ar{r} & I \ar{r} & D_{kj} \ar{r} & B_{kj} \ar{r} & 0.
    \end{tikzcd}
    \end{equation*}
    \begin{lem}
        The morphism $g = \varphi_j^{-1} \varphi_k$ must be of the form
        \[ g(i+b) = i + b + \delta(b), \]
        where $\delta \colon B_{kj} \to I$ is a derivation. 
    \end{lem}
    In particular, this means that $\varphi_j^{-1} \circ \varphi_k(b, b') = (b, \alpha_{kj}(b) + b')$, where $\alpha_{kj} \colon B_{kj} \to I \otimes_k B_{kj}$ is a derivation.

    By definition, we have
    \begin{align*} 
        (T_{X/k} \otimes_k I)(B_{kj}) &= {\Hom_{B_{kj}}(\Omega^1_{B_{kj}/k}, B_{kj}) \otimes_k I} \\ 
        &= {\Hom_{B_{kj}}(\Omega^1_{B_{kj}/k}, B_{kj} \otimes_k I)} \\ 
        &= \mr{Der}_k(B_{kj}, B_{kj} \otimes_k I).
    \end{align*}
    Therefore, $\alpha_k \in H^0(B_{kj}, T_X \otimes_k I)$. Note that
    \[ \varphi_{\ell}^{-1} \circ \varphi_j \circ \varphi_j^{-1} \circ \varphi_{k}^{-1} = \varphi_{\ell}^{-1} \circ \varphi_k, \]
    which implies that
    \[ (b, \alpha_{j\ell}(b) + \alpha_{kj}(b) + b') = (b, \alpha_{k\ell}(b) + b') \]
    and thus $\qty{\alpha_{kj}} \in Z^1(\qty{U_k}, T_X \otimes_k I)$.

    If two deformations are the same, note that $\varphi_k$ is defined using a ring section $s_k \colon B_k \to D_k$ of the canonical map $\pi_k \colon D_k \to B_k$. If $\varphi_k'$ is defined using another section $s_k'$, then define $\theta_k = s_k' - s_k \in \mr{Der}(B_k, I \otimes_k B_k)$. We now compute
    \[ ({(\varphi_j')}^{-1} \circ \varphi_k' - \varphi_j^{-1} \circ \varphi_k)(b, b') = (0, \theta_k(b) - \theta_j(b)), \]
    and thus the two differ by the desired coboundaries.
\end{proof}

We will now consider some obstructions. We are looking for a diagram of the form
\begin{equation*}
\begin{tikzcd}
    X' \ar{r} \ar{d}{f} & X'' \ar{d} \\
    \Spec A' \ar{r} & \Spec A''.
\end{tikzcd}
\end{equation*}
for each pair $(j, k)$, we have a isomorphism $\psi_{jk} \colon V_j' \to V_k'$ and a cocycle
\[ c_{jk\ell} = \psi_{k\ell} \circ \psi_{jk} \circ \psi_{j\ell}^{-1}. \]
This induces $B_{jk\ell} \in \mr{Der}_A(D_{jk\ell}, J \otimes_A D_{k\ell}) = Z^2(U, T_{X'/A} \otimes_A J)$.

Now we will discuss some examples.
\begin{thm}
    Let $C$ be a smooth projective curve, $T = T_C$, and $K = \Omega^1_C$. We have the following table:
    \begin{table}[H]
        \centering
        \caption{Cohomology}
        \label{tab:label}
        \begin{tabular}{ccccc}
        \toprule
            & degree & $h^0$ & $h^1$ & $h^2$ \\
            \midrule
            $K$ & $2g-2$ & $g$ & $1$ & $0$ \\
            $T$ & $2-2g$ & $\ep$ & $\ep + 3g-3$ & 0 \\
            \bottomrule
        \end{tabular}
        where $\ep = 0$ where $g \geq 2$, $\ep = 1$ if $g = 1$, and $\ep = 3$ if $g = 0$. For $g \geq 2$, $\deg T < 0$, and by Riemann-Roch and Serre duality, we have $h^1(C, T_C) = 3g-3$.
    \end{table}
\end{thm}

\begin{thm}
    $\P^n$ has no infinitesimal deformations.
\end{thm}

\begin{proof}
    Consider the Euler sequence
    \[ 0 \to \mc{O} \to {\mc{O}(1)}^{\oplus n+1} \to T_{\P^n} \to 0 \]
    and use the long exact sequence in cohomology. Because positive degree line bundles have no higher cohomology, we have $H^1(T_{\P^n}) = 0$.
\end{proof}



\end{document}
