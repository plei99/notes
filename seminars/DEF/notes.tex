\documentclass[leqno, openany]{memoir}
\setulmarginsandblock{3.5cm}{3.5cm}{*}
\setlrmarginsandblock{3cm}{3.5cm}{*}
\checkandfixthelayout

\usepackage{amsmath}
\usepackage{amssymb}
\usepackage{amsthm}
%\usepackage{MnSymbol}
\usepackage{bm}
\usepackage{accents}
\usepackage{mathtools}
\usepackage{tikz}
\usetikzlibrary{calc}
\usetikzlibrary{automata,positioning}
\usepackage{tikz-cd}
\usepackage{forest}
\usepackage{braket} 
\usepackage{listings}
\usepackage{mdframed}
\usepackage{verbatim}
\usepackage{physics}
\usepackage{stmaryrd}
\usepackage{mathrsfs} 
\usepackage{ulem} 
%\usepackage{/home/patrickl/homework/macaulay2}

%font
\usepackage[sc]{mathpazo}
\usepackage{eulervm}
\usepackage[scaled=0.86]{berasans}
\usepackage{inconsolata}
\usepackage{microtype}

%CS packages
\usepackage{algorithmicx}
\usepackage{algpseudocode}
\usepackage{algorithm}

% typeset and bib
\usepackage[english]{babel} 
\usepackage[utf8]{inputenc} 
\usepackage[T1]{fontenc}
\usepackage[backend=biber, style=alphabetic]{biblatex}
\usepackage[bookmarks, colorlinks, breaklinks]{hyperref} 
\hypersetup{linkcolor=black,citecolor=black,filecolor=black,urlcolor=blue}

% other formatting packages
\usepackage{float}
\usepackage{booktabs}
\usepackage[shortlabels]{enumitem}
\usepackage{csquotes}
\usepackage{titlesec}
\usepackage{titling}
% \usepackage{fancyhdr}
% \usepackage{lastpage}
\usepackage{parskip}

\usepackage{lipsum}

% delimiters
\DeclarePairedDelimiter{\gen}{\langle}{\rangle}
\DeclarePairedDelimiter{\floor}{\lfloor}{\rfloor}
\DeclarePairedDelimiter{\ceil}{\lceil}{\rceil}


\newtheorem{thm}{Theorem}[section]
\newtheorem{cor}[thm]{Corollary}
\newtheorem{prop}[thm]{Proposition}
\newtheorem{lem}[thm]{Lemma}
\newtheorem{conj}[thm]{Conjecture}
\newtheorem{quest}[thm]{Question}

\theoremstyle{definition}
\newtheorem{defn}[thm]{Definition}
\newtheorem{defns}[thm]{Definitions}
\newtheorem{con}[thm]{Construction}
\newtheorem{exm}[thm]{Example}
\newtheorem{exms}[thm]{Examples}
\newtheorem{notn}[thm]{Notation}
\newtheorem{notns}[thm]{Notations}
\newtheorem{addm}[thm]{Addendum}
\newtheorem{exer}[thm]{Exercise}

\theoremstyle{remark}
\newtheorem{rmk}[thm]{Remark}
\newtheorem{rmks}[thm]{Remarks}
\newtheorem{warn}[thm]{Warning}
\newtheorem{sch}[thm]{Scholium}


% unnumbered theorems
\theoremstyle{plain}
\newtheorem*{thm*}{Theorem}
\newtheorem*{prop*}{Proposition}
\newtheorem*{lem*}{Lemma}
\newtheorem*{cor*}{Corollary}
\newtheorem*{conj*}{Conjecture}

% unnumbered definitions
\theoremstyle{definition}
\newtheorem*{defn*}{Definition}
\newtheorem*{exer*}{Exercise}
\newtheorem*{defns*}{Definitions}
\newtheorem*{con*}{Construction}
\newtheorem*{exm*}{Example}
\newtheorem*{exms*}{Examples}
\newtheorem*{notn*}{Notation}
\newtheorem*{notns*}{Notations}
\newtheorem*{addm*}{Addendum}


\theoremstyle{remark}
\newtheorem*{rmk*}{Remark}

% shortcuts
\newcommand{\Ima}{\mathrm{Im}}
\newcommand{\A}{\mathbb{A}}
\newcommand{\F}{\mathbb{F}}
\newcommand{\G}{\mathbb{G}}
\newcommand{\N}{\mathbb{N}}
\newcommand{\R}{\mathbb{R}}
\newcommand{\C}{\mathbb{C}}
\newcommand{\Z}{\mathbb{Z}}
\newcommand{\Q}{\mathbb{Q}}
\renewcommand{\k}{\Bbbk}
\renewcommand{\P}{\mathbb{P}}
\newcommand{\M}{\overline{M}}
\newcommand{\g}{\mathfrak{g}}
\newcommand{\h}{\mathfrak{h}}
\newcommand{\n}{\mathfrak{n}}
\renewcommand{\b}{\mathfrak{b}}
\newcommand{\ep}{\varepsilon}
\newcommand*{\dt}[1]{%
   \accentset{\mbox{\Huge\bfseries .}}{#1}}
\renewcommand{\abstractname}{Official Description}
\newcommand{\mc}[1]{\mathcal{#1}}
\newcommand{\T}{\mathbb{T}}
\newcommand{\mf}[1]{\mathfrak{#1}}
\newcommand{\mr}[1]{\mathrm{#1}}
\newcommand{\ms}[1]{\mathsf{#1}}
\newcommand{\on}[1]{\operatorname{#1}}
\newcommand{\ol}[1]{\overline{#1}}
\newcommand{\ul}[1]{\underline{#1}}
\newcommand{\wt}[1]{\widetilde{#1}}
\newcommand{\wh}[1]{\widehat{#1}}
\renewcommand{\div}{\operatorname{div}}
\newcommand{\bir}{\sim_{\mr{bir}}}
\newcommand{\stacks}[1]{\href{https://stacks.math.columbia.edu/tag/#1}{#1}}

\DeclareMathOperator{\Der}{Der}
\DeclareMathOperator{\Def}{Def}
\DeclareMathOperator{\Bl}{Bl}
\DeclareMathOperator{\NE}{NE}
\DeclareMathOperator{\Tor}{Tor}
\DeclareMathOperator{\Hom}{Hom}
\DeclareMathOperator{\Ext}{Ext}
\DeclareMathOperator{\End}{End}
\DeclareMathOperator{\ad}{ad}
\DeclareMathOperator{\Aut}{Aut}
\DeclareMathOperator{\Rad}{Rad}
\DeclareMathOperator{\Pic}{Pic}
\DeclareMathOperator{\supp}{supp}
\DeclareMathOperator{\Supp}{Supp}
\DeclareMathOperator{\sgn}{sgn}
\DeclareMathOperator{\spec}{Spec}
\DeclareMathOperator{\Spec}{Spec}
\DeclareMathOperator{\proj}{Proj}
\DeclareMathOperator{\Proj}{Proj}
\DeclareMathOperator{\ord}{ord}
\DeclareMathOperator{\Div}{Div}
\DeclareMathOperator{\depth}{depth}
\DeclareMathOperator{\coker}{coker}

% Section formatting
\titleformat{\section}
    {\Large\sffamily\scshape\bfseries}{\thesection}{1em}{}
\titleformat{\subsection}[runin]
    {\large\sffamily\bfseries}{\thesubsection}{1em}{}
\titleformat{\subsubsection}[runin]{\normalfont\itshape}{\thesubsubsection}{1em}{}

\title{COURSE TITLE}
\author{Lectures by INSTRUCTOR, Notes by NOTETAKER}
\date{SEMESTER}

\newcommand*{\titleSW}
    {\begingroup% Story of Writing
    \raggedleft
    \vspace*{\baselineskip}
    {\Huge\itshape Defomation Theory Graduate Student Seminar \\ Fall 2021}\\[\baselineskip]
    {\large\itshape Notes by Patrick Lei}\\[0.2\textheight]
    {\Large Lectures by Various}\par
    \vfill
    {\Large \sffamily Columbia University}
    \vspace*{\baselineskip}
\endgroup}
\pagestyle{simple}

\chapterstyle{ell}


%\renewcommand{\cftchapterpagefont}{}
\renewcommand\cftchapterfont{\sffamily}
\renewcommand\cftsectionfont{\scshape}
\renewcommand*{\cftchapterleader}{}
\renewcommand*{\cftsectionleader}{}
\renewcommand*{\cftsubsectionleader}{}
\renewcommand*{\cftchapterformatpnum}[1]{~\textbullet~#1}
\renewcommand*{\cftsectionformatpnum}[1]{~\textbullet~#1}
\renewcommand*{\cftsubsectionformatpnum}[1]{~\textbullet~#1}
\renewcommand{\cftchapterafterpnum}{\cftparfillskip}
\renewcommand{\cftsectionafterpnum}{\cftparfillskip}
\renewcommand{\cftsubsectionafterpnum}{\cftparfillskip}
\setrmarg{3.55em plus 1fil}
\setsecnumdepth{subsection}
\maxsecnumdepth{subsection}
\settocdepth{subsection}

\begin{document}
    
\begin{titlingpage}
\titleSW
\end{titlingpage}

\thispagestyle{empty}
\section*{Disclaimer}%
\label{sec:disclaimer}

These notes were taken during the seminar using the \texttt{vimtex} package of the editor \texttt{neovim}. 
Any errors are mine and not the speakers'. 
In addition, my notes are picture-free (but will include commutative diagrams) and are a mix of my mathematical style and that of the lecturers.
If you find any errors, please contact me at \texttt{plei@math.columbia.edu}.

\vspace*{1cm}

\noindent\textbf{Seminar Website:}  \url{https://www.math.columbia.edu/~dejong/seminar.html}
\newpage

\tableofcontents

\chapter{Johan (Sep 24): Schlessinger's paper}%
\label{cha:johan_sep_24_schlessinger_s_paper}

The paper by Schlessinger is titled \textit{Functors of Artin Rings}. Throughout this lecture, $k$ is a field, $\mc{C}$ is the category of Artinian local $k$-algebras $A, B, C, \ldots$ with residue field $k$, and $\wh{\mc{C}}$ is the category of Noetherian complete local $k$-algebras $R, S, \ldots$ with residue field $k$.

\begin{rmk}
    Every $R \in \wh{\mc{C}}$ is of the form $k[[x_1, \ldots, x_n]] / (f_1, \ldots, f_m)$ by the Cohen structure theorem. Then $R \in \mc{C}$ if and only if and only if $(f_1, \ldots, f_m)$ contains ${(x_1, \ldots, x_n)}^N$ for some $N$.
\end{rmk}

\begin{rmk}
    In the paper, there is a more general setup, where $\Lambda$ is a complete local Noetherian ring with residue field $k$. Then $\mc{C}_{\Lambda}, \wh{\mc{C}}$ are defined analogously, which will allow things like $\Lambda = \Z_p$.
\end{rmk}

The idea of deformation theory is to look at functors $F \colon \mc{C} \to \ms{Set}$.

\begin{exm}
    Given $R \in \wh{\mc{C}}$, we set $h_R \colon \mc{C} \to \ms{Set}$ sending $A \mapsto \Hom_{\wh{\mc{C}}}(R, A)$. This is not necessarily representable because $R \notin \mc{C}$ in general, but it is pro-representable.
\end{exm}

\begin{defn}
    A functor $F$ is \textit{pro-representable} if $F \simeq h_R$ for some $R \in \wh{\mc{C}}$.
\end{defn}

\begin{exm}
    Let $M$ be a variety over $k$ and $m \in M(k)$. Then define
    \[ \mr{Def}_{M,m}(A) = \qty{ \Spec A \xrightarrow{m_A} M \mid m_A |_{\Spec k} = m}. \]
    It is easy to see that $\mr{Def}_{M,m}(A)$ is pro-representable by $\wh{\mc{O}}_{M,m}$.
\end{exm}

Observe that $h_R(k) = \qty{*}$ is a singleton. Also note that $h_R(A \times_B C) = h_R(A) \times_{h_R(B)} h_R(C)$. Here, $A \times_B C$ is the fiber product of rings and not the tensor product.

Now consider the following conditions on $F$: let $A \to B \gets C$ be a diagram in $\mc{C}$ and consider the morphism
\[ F(A \times_B C) \xrightarrow{(*)} F(A) \times_{F(B)} F(C). \]
\begin{itemize}
    \item $(H_1)$ The morphism $(*)$ is surjective if $C \twoheadrightarrow B$;
    \item $(H_2)$ The morphism $(*)$ is bijective if $C = k[\ep] \twoheadrightarrow k = B$;
    \item $(H_3)$ $\dim_k(t_F) < \infty$ (later, we will see that we need $H_2$ for formulate this). Here, $t_F$ is the tangent space to $F$;
    \item $(H_4)$ The morphism $(*)$ is bijective if $C \twoheadrightarrow B$.
\end{itemize}

\begin{exm}
    Fix a group $G$ and a representation $\rho_0 \colon G \to GL_n(k)$. Now define
    \[ \mr{Def}_{\rho_0}^{\text{naive}}(A) = \qty{\rho \colon G \curvearrowright A^{\oplus n} \mid \rho \pmod{m_A} \cong \rho_0} /\cong. \]
    Better, we will define
    \[ \mr{Def}_{\rho_0}(A) = \qty{\rho \colon G \curvearrowright A^{\oplus n} \mid \rho \pmod{m_A} = \rho_0} / \ker(GL_n(A) \to GL_n(k)). \]
    In general these functors fail $(H_4)$ and $\mr{Def}_{\rho_0}^{\text{naive}}$ even fails $(H_2)$.

    Namely, if $H = \Z$ and $\rho_0$ is the trivial representation, then for $\mr{Def}_{\rho_0}^{\text{naive}}$, we are looking at subsets of
    \[ GL_n(A \times_B C)/\text{conj} \to GL_n(A)/\text{conj} \times_{GL_n(B)/\text{conj}} GL_n(C)/\text{conj}. \]
    This morphism is always surjective, but in general it is not injective.

    For example, if $A = k[\ep_1], B = k, C = k[\ep_2]$, we can look at elements of the form $1 + \ep_1 T_1 + \ep_2 T_2$ and see that on the left we can only conjugate together, while on the right we can conjugate both $T_1, T_2$ arbitrarily. Here $A \times_B C = k[\ep_1, \ep_2] = k[x_1,x_2] / (x_1^2, x_1 x_2, x_2^2)$.
\end{exm}

\begin{defn}
    A natural transformation $t \colon F \to G$ of functors on $\mc{C}$ is \textit{smooth} if for all surjections $B \twoheadrightarrow A$ the map $F(B) \to F(A) \times_{G(A)} G(B)$ is surjective.
\end{defn}

Note that this is equivalent to the existence of a lift in the diagram below:
\begin{equation*}
\begin{tikzcd}
    \Spec A \ar{r} \ar[hookrightarrow]{d} & M \ar{d}{f} \\
    \Spec B \ar{r} \ar[dashrightarrow]{ur} & N.
\end{tikzcd}
\end{equation*}


This definition is motivated by the following example: let $f \colon M \to N$ be a morphism of varieties over $k$. Let $m \in M(k), n = f(m) \in N(k)$. Then the following are equivalent:
\begin{enumerate}
    \item $\mr{Def}_{M,m} \to \mr{Def}_{N,n}$ is smooth.
    \item $f$ is smooth at $m$.
\end{enumerate}

\begin{defn}
    We say $F$ has a \textit{hull} if and only if $F(k) = \qty{*}$ and there exists a smooth $t \colon h_R \to F$ for some $R \in \wh{C}$ which induces an isomorphism $t_R \cong t_F$.
\end{defn}

Now we will say a bit about tangent spaces.
\begin{enumerate}
    \item When $F(k) = \qty{*}$, then $t_F = F(k[\ep])$.
    \item If $F$ satisfies $(H_2)$ and $F(k) = \qty{*}$, then $t_F$ has a natural $k$-vector space structure. Here, $H_2$ gives $F(k[\ep_1, \ep_2]) \to F(k[\ep]) \times F(k[\ep])$ is a bijection, and then we take $\ep_1 \mapsto \ep, \ep_2 \mapsto \ep$, which defines addition.
    \item $t_R = \Hom_k(\mf{m}_R / \mf{m}_R^2, k) = \Hom_{\wh{C}}(R, k[\ep]) = h_R(k[\ep]) = t_{(h_R)}$.
\end{enumerate}

\begin{thm}[Schlessinger]
    Asssume that $F(k) = \qty{*}$. Then the conditions $(H_1), (H_2), (H_3)$ hold for $F$ if and only if $F$ has a hull. In addition, $(H_3)$ and $(H_4)$ hold if and only if $F$ is pro-representable.
\end{thm}

\begin{proof}[Very rough idea of proof of $\Rightarrow$ for the hull case]
    Let $n = \dim_k(t_F)$. Then $(H_2)$ and $n < \infty$ imply the following: Let $S = k[[x_1, \ldots, x_n]]$ and $\mf{m} = \mf{m}_S = (x_1, \ldots, x_n)$. We can find $\xi_1 \in F(S/\mf{m}^2)$ such that
    \[ t_S = \Hom_{\wh{\mc{C}}}(S, k[\ep]) \xrightarrow{\xi_1} t_F \]
    is an isomorphism.

    Next, we will choose $q \geq 2$ and consider pairs $(J, \xi)$ where $\mf{m}^{q+1} \subset J \subset \mf{m}^2$ and $\xi \in F(S/J)$ such that $\xi \mapsto \xi_1 \in F(S/\mf{m}^2)$. Say that $(J, \xi) \leq (J', \xi')$ if $J \subset J'$ and $\xi \mapsto \xi'$. Choose a minimal pair $(J, \xi)$ for this ordering. We can choose $J_{q}$ so that $\mf{m}^{q+1} + J_{q+1} = J_q$ and $\xi_{q+1}$ maps to $\xi_q$ for bookkeeping purposes.

    Choose $R = \lim S/J_q$, which is a quotient of $S$. Set $t \colon h_R \to F$ given by sending $\varphi \colon R \to A$ to the following: choose $q$ such that $\varphi$ factors as $R \to S/J_q \xrightarrow{\varphi_q} A$ and take $\xi_q \mapsto F(\varphi_q)(\xi_q) \in F(A)$.

    Finally, we must show that $t$ is smooth. Consider the diagram
    \begin{equation*}
    \begin{tikzcd}
        S \ar{dd} \ar{r} \ar[dashrightarrow,bend left=30]{drrr}{\varphi} & S/\mf{m}^{q+1} \ar[dashrightarrow,swap]{dr}{\ol{\varphi}'} \ar[dashrightarrow]{drr}{\ol{\varphi}} \\ 
        & & S/J_q \times_A B \ar{dl}{pr_1} \ar{r} & B \ar[twoheadrightarrow]{d} \\
        R \ar{r} & S/J_q \ar{rr} \ar[dashrightarrow,bend left=20]{ur}{\psi} & & A
    \end{tikzcd}
    \end{equation*}
    with $B \ni \wt{\xi} \mapsto \xi \in A$ and $S/J_q \ni \xi_q \mapsto \xi$. First, choose $\varphi \colon S \to B$ making the diagram commute. We may increase $q$ such that $\varphi(\mf{m}^{q+1}) = 0$, so we now have $\ol{\varphi} \colon S/\mf{m}^{q+1} \to B$. Now consider the fiber product $S/J_q \times_A B$ and $pr_1 \colon S/J_q \times_A B \to S/J_q$, so we obtain $\ol{\varphi}' \colon S/\mf{m}^{q+1} \to S/J_q \times_A B$. By $(H_1)$, we obtain some $\wt{\wt{\xi}} \in F(S/J_q \times_A B)$ mapping to $\wt{\xi}$ and $\xi_q$. We may now assume that $B \to A$ is a small extension, which means that $\dim_k \ker(B \to A) = 1$, and thus $pr_1$ is a small extension. Therefore, either $\ol{\varphi}'$ is surjective or its image maps isomorphically via $pr_1$ to $S/J_q$., so we have $\psi$ which gives $R \to B$ lifting our given $r \to A$.

    The tricky part is to show that $F(\psi)(\psi_q) = \wt{\wt{\xi}}$, and this step is deliberately omitted.
\end{proof}

A generalization of this is as follows. Consider a functor $\mc{F} \colon \mc{C} \to \ms{Grpd}$. We say that $\mc{F}$ satisfies the \textit{Rim-Schlessinger condition} (RS) if
\[ \mc{F}(A \times_B C) \to \mc{F}(A) \times_{\mc{F}(B)} \mc{F}(C) \]
is an equivalence whenever $C \twoheadrightarrow B$. Let $x_0 \in \mc{F}(k)$ and set 
\[ \ol{\mc{F}}_{x_0} \colon \mc{C} \to \ms{Set} \qquad A \mapsto \qty{(x, \alpha) \mid x \in \mc{F}(A), \alpha \colon X_0 \to x|_k} / \cong, \]
where $(x,\alpha) \cong (x', \alpha')$ means that $\varphi \colon x \to x'$ such that the diagram
\begin{equation*}
\begin{tikzcd}
    x|_k \ar{r}{\varphi} & x'|_k \\
    x_0 \ar{r}{\mr{id}} \ar{u}{\alpha} & x_0 \ar{u}{\alpha'}
\end{tikzcd}
\end{equation*}
commutes.

\begin{thm}
    If $\mc{F}$ has (RS) then $\ol{\mc{F}}_{x_0}$ has $(H_1)$ and $(H_2)$. Therefore, if $\dim t_{\ol{\mc{F}}_{x_0}} < \infty$ then $\ol{\mc{F}}_{x_0}$ has a hull.
\end{thm}

In this situation, $\ol{\mc{F}}_{x_0}$ has $(H_4)$ if and only if $\Aut_A(x) \twoheadrightarrow \Aut_B(x|_B)$ whenever $A \twoheadrightarrow B$ and $x \in \mc{F}_{x_0}(A)$.

\begin{exm}
    Let $\mc{F}(A)$ be the category of representations $G \curvearrowright A^{\oplus n}$ with morphisms being isomorphisms of representations. This has (RS).
\end{exm}

\begin{exm}
    Let $\mc{F}(A)$ be the category of smooth projective families of curves of genus $g$ over $A$ with morphisms being isomorphisms. This has (RS).
\end{exm}

Returning to the example of representations, it turns out that $t_{\mr{Def}_{\rho_0}} = H^1(G, M_{n \times n}(k))$, where $G$ acts on $M_{n \times n}(k)$ via $\rho_0$ by conjugation.

\begin{exm}
    Consider $G = \Z \oplus \Z$ and $\rho_0$ to be the trivial representation on $k^{\oplus 2}$. Then $t_{\mr{Def}_{\rho_0}} = H^1(\Z^2, M_2(k)) = M_2(k) \oplus M_2(k)$. Given two matrices $A, B$, we have the representation
    \begin{align*}
        \Z^2 \to GL_2(k[\ep]) & (1,0) \mapsto \mqty(\dmat{1,1}) + \ep A \\
        & (0,1) \mapsto \mqty(\dmat{1,1}) + \ep B.
    \end{align*}
    We get a hull $R$ with $h_R \to \mr{Def}_{\rho_0}$. We know that $R$ is a some quotient of $k[[a_{11}, \ldots, a_{22}, b_{11}, \ldots, b_{22}]]$ with $\rho$ looking like
    \[ (1,0) \mapsto \mqty(\dmat{1,1}) + A \qquad (0,1) \mapsto \mqty(\dmat{1,1}) + B, \]
    and of course $R$ is the quotient of the power series ring by the ideal generated by the coefficients of $AB - BA$.
\end{exm}

\chapter{Ivan and Cailan (Oct 1): Deformations of schemes}%
\label{cha:ivan_and_cailan_oct_1_deformations_of_schemes}

\section{Deformations of affine schemes}%
\label{sec:deformations}

We are looking for a Cartesian diagram of schemes
\begin{equation*}
\begin{tikzcd}
    X \ar{r} \ar{d} & \mc{X} \ar{d}{\pi} \\
    \Spec k \ar{r} & S
\end{tikzcd}
\end{equation*}
where $\pi$ is flat and surjective and $S$ is surjective. This is called an \textit{deformation} of $X$ over $S$. For the beginning of this lecture (the part given by Ivan), we are interested in $S = \Spec A$, where $A \in \mc{C}^*$ (this category was defined in the previous lecture). This case is called a \textit{local deformation}, and in the face where $A$ is Artinian, it is called an \textit{infinitesimal deformation}.

For the ring theorists, we will make the following digression. Let $A$ be a ring and $I \subset A$ be an ideal with $I^2 = 0$. Suppose that $\ol{B}$ is an $A/I$-algebra, $J$ is an $\ol{B}$-module, and $h \colon I \to J$ is an $A$-module map. Then we are interested in a diagram
\begin{equation*}
\begin{tikzcd}
    0 \ar{r} & J \ar{r} & ? \ar{r} & \ol{B} \\
    0 \ar{r} & I \ar{u}{h} \ar{r} & A \ar{u} \ar{r} & A/I \ar{u} \ar{r} & 0,
\end{tikzcd}
\end{equation*}
which we will call a deformation of $A$. Here are some interesting questions:
\begin{enumerate}
    \item Is such a deformation unique?
    \item If $\ol{B}$ is flat over $A/I$, does that mean that $B$ is flat over $A$?
\end{enumerate}

Returning to the case of schemes, we will say that two deformations $\mc{X}, \mc{X}'$ of $X$ over $S$ are isomorphic if there exists an $S$-isomorphism $\phi \colon \mc{X}' \to \mc{X}$ commuting with the inclusions of the central fibers $X \to \mc{X}, \mc{X}'$.

\begin{exm}
    The most basic example of a family is the trivial deformation
    \begin{equation*}
    \begin{tikzcd}
        X \ar{r} \ar{d} & X \times_k S \ar{d} \\
        \Spec k \ar{r} & S.
    \end{tikzcd}
    \end{equation*}
\end{exm}

\begin{defn}
    A scheme $X$ is \textit{rigid} if all deformations of $X$ are isomorphic to the trivial deformation.
\end{defn}

\begin{thm}
    If $X$ is a smooth affine $k$-scheme and $S = \Spec A$ for some local Artinian ring, then $X$ is rigid.
\end{thm}

\begin{defn}
    A closed immersion $i \colon S_0 \hookrightarrow S$ of schemes is called a \textit{first (resp. nth) order thickening} if the ideal sheaf $\mc{I} = \ker(i^{\flat} \colon \mc{O}_S \to \mc{O}_{S_0})$ satisfies $I^2 = 0$ (resp. $I^{n+1} = 0$).
\end{defn}

\begin{defn}
    A morphism $f \colon X \to S$ is called \textit{formally smooth} (resp. unfamified, resp. \'etale) if for all first order thickenings $i \colon T_0 \to T$ of affine schemes and diagrams
    \begin{equation*}
    \begin{tikzcd}
        T_0 \ar{r}{u_0} \ar{d}{i} & X \ar{d}{f} \\
        T \ar{r} \ar[dashrightarrow]{ur}{\wt{u_0}} & S
    \end{tikzcd}
    \end{equation*}
    there exists a lift $\wt{u_0}$ (resp. there is at most one such $\wt{u_0}$, resp. there exists a unique $\wt{u}_0$).
\end{defn}

\begin{exm}\leavevmode
    \begin{enumerate}
        \item Open immersions are formally \'etale. This is cleaer because $T_0, T$ have the same underlying topological space.
        \item Closed immersions are formally unramified. This is clear because $X \to S$ induces an injection on $T$-points.
        \item $\A^n_S \to S$ is formally smooth. To see this, assume $S = \Spec R$ is affine and then consider the corresponding lifting problem in commutative algebra.
    \end{enumerate}
\end{exm}

\begin{prop}
    The classes of formally smooth (resp. \'etale, resp. unramified) morphisms are closed under base change, composition, and products and local on both source and target.
\end{prop}

\begin{defn}
    A $f \colon X \to S$ is \textit{smooth} if it is formally smooth and locally of finite presentation.
\end{defn}

We will now consider differentials. Let $X = \Spec A$ be an affine scheme over $k$ and choose a $k$-point and consider the diagram
\begin{equation*}
\begin{tikzcd}
    \Spec k \ar{r} \ar{d} & X \ar{d} \\
    \Spec k[\ep] \ar{r} & \Spec k.
\end{tikzcd}
\end{equation*}
If $X$ is smooth, then there exists a lift $\Spec k[\ep] \to X$. But this is given by a morphism
\[ \wt{\phi} \colon A \to k[\ep] / \ep^2 \qquad a \mapsto \phi(a) + d(a) \ep. \]
This motivates the following definition:
\begin{defn}
    Let $R \to A$ be a morphism of rings and $M$ be an $A$-module. A \textit{derivation} $d \colon A \to M$ is an $A$-linear map satisfying the Leibniz rule.
\end{defn}

\begin{prop}
    There exists an $A$-module $\Omega^1_{A/k}$ equipped with a derivation $d \colon \Omega^1_{A/k}$ that is universal among derivations from $A$. This means that all derivations $\wt{d} \colon A \to M$ factor through $d$, and formally, we have an identity
    \[ \mr{Der}_R(A,M) \simeq \Hom_A(\Omega^1_{A/k}, M). \]
\end{prop}

\begin{defn}
    For an $A$-module $M$ with derivation $d \colon A \to M$, define the ring $A[M]$ as the module $A \oplus M$ with the multiplication
    \[ (a,m) \cdot (a', m') = (aa', am' + a'm). \]
    There is a sequence $\phi \colon A \to A[M] \to A$.
\end{defn}

\begin{prop}
    Let $S \gets R \to A \to B$ be a diagram of rings. Then
    \begin{enumerate}
        \item $\Omega^1_{A \otimes S / S} \simeq \Omega^1_{A/R} \otimes_R S$;
        \item The sequence $\Omega^1_{A/R} \otimes_A B \to \Omega^1_{B/R} \to \Omega^1_{B/A} \to 0$ is exact.
        \item If $B = A/I$ for some ideal $I$, we have an exact sequence
            \[ I/I^2 \to \Omega^1_{A/R} \otimes_A B \to \Omega^1_{B/R} \to 0. \]
        \item For all $f \in A$, we havae $\Omega^1_{A[f^{-1}]/R} \simeq \Omega^1_{A/R} \otimes_A A[f^{-1}]$.
    \end{enumerate}
\end{prop}

\begin{rmk}
    If $J = \ker(A \otimes_R A \to A)$, then $\Omega^1_{A/R} = J/J^2$.
\end{rmk}

\begin{thm}
    Let $f \colon X \to S$ be locally of finite presentation. The following are equivalent:
    \begin{enumerate}
        \item $f$ is smooth;
        \item $f$ is flat with smooth fibers;
        \item $f$ is flat and has smooth geometric fibers.
    \end{enumerate}
\end{thm}

We will finally return to deformation theory.

\begin{lem}
    Let $Z_0$ be a closed subscheme of $Z$ determined by a nilpotent ideal sheaf $N$. If $Z_0$ is affine, then so is $Z$.
\end{lem}
Proof of this result can be found in EGA, Chapter I.5.9. 

\begin{proof}[Proof of Theorem 2.1.3]
    Recall that we have a diagram of the form
    \begin{equation*}
    \begin{tikzcd}
        B \ar{r} & B_0 \\
        A \ar{u} \ar{r} & k \ar{u},
    \end{tikzcd}
    \end{equation*}
    where $A \to B$ is flat and $B_0 \simeq B \otimes_A k$ is a smooth $k$-algebra. We need to prove that $B_0 \simeq B \otimes_A k$. The first step is to prove this result for first-order deformations. Suppose that $A = k[\ep]$ is a square-zero extension. 
    \begin{lem}
        For a ring $R$ with $M, N$ flat over $R$, nilpotent ideal $I \subset R$, and $f \colon M \to N$, then if $f \otimes_R R/I$ is an isomorphism, then so is $f$.
    \end{lem}
    To prove the lemma, note that the cokernel of $f$ is preserved by $I$, so it must vanish. Returning to our case, we know that $B$ is a smooth $k[\ep]$-algebra. Now we obtain a square-zero extension $B_0[\ep]$ of $B_0$ and a diagram
    \begin{equation*}
    \begin{tikzcd}
        B \ar{r} & B_0 \\
        k[\ep] \ar{r}{f} \ar{r} & B_0[\ep] \ar{u}
    \end{tikzcd}
    \end{equation*}
    with a lift $B \to B_0[\ep]$. But now by the lemma, we have $B \otimes_{k[\ep]} k = B_0[\ep] \otimes_{k[\ep]} k$. The rest of the proof follows using an inductive argument that was verbalized but now written down.
\end{proof}

\section{Deformations of schemes}%
\label{sec:deformations_of_schemes}

The main theorem of this section is
\begin{thm}
    Assume $X$ is a smooth $R$-scheme. Then there is a bijection
    \[ \mr{Def}_X^{\mr{sm}}(k[I]) \simeq H^1(X, T_{X/k} \otimes I). \]
\end{thm}

\begin{proof}
    Let $\mc{X}'$ be a smooth deformation over $k[I]$. Then the diagram
    \begin{equation*}
    \begin{tikzcd}
        X \ar{r} \ar{d} & X' \ar{d} \\
        \Spec k \ar{r} & \Spec k[I]
    \end{tikzcd}
    \end{equation*}
    is cartesian. Then if $U_k = \Spec B_k$ is an affine cover of $X$ and $U_k' = \Spec D_k$ is an affine cover of $X'$, we have a $k[I]$-linear ring isomorphism
    \[ \varphi_k \colon k[I] \otimes_k B_k \to D_k \qquad (k,i) \otimes b \mapsto s(b) + i. \]
    Modulo $I$, $\varphi_k$ is the identity on $B_k$. Without loss of generality, we may assume that $U_{kj} = U_k \cap U_j$ is a distinguished open for both $U_k$ and $U_j$, so let $U_{kj} = \Spec B_{kj}$ and $U_{kj}' = \Spec D_{kj}$. Now note that both
    \[ \varphi_k, \varphi_j \colon k[I] \otimes_k B_{kj} \to D_{kj} \]
    induce the identity on $B_{kj}$ modulo $I$. Now we have the commutative diagram
    \begin{equation*}
    \begin{tikzcd}
        0 \ar{r} & I \ar{d}{\mr{id}} \ar{r} & D_{kj} \ar{d}{\varphi_j^{-1} \varphi_k} \ar{r} & B_{kj} \ar{d}{\mr{id}} \ar{r} & 0 \\
        0 \ar{r} & I \ar{r} & D_{kj} \ar{r} & B_{kj} \ar{r} & 0.
    \end{tikzcd}
    \end{equation*}
    \begin{lem}
        The morphism $g = \varphi_j^{-1} \varphi_k$ must be of the form
        \[ g(i+b) = i + b + \delta(b), \]
        where $\delta \colon B_{kj} \to I$ is a derivation. 
    \end{lem}
    In particular, this means that $\varphi_j^{-1} \circ \varphi_k(b, b') = (b, \alpha_{kj}(b) + b')$, where $\alpha_{kj} \colon B_{kj} \to I \otimes_k B_{kj}$ is a derivation.

    By definition, we have
    \begin{align*} 
        (T_{X/k} \otimes_k I)(B_{kj}) &= {\Hom_{B_{kj}}(\Omega^1_{B_{kj}/k}, B_{kj}) \otimes_k I} \\ 
        &= {\Hom_{B_{kj}}(\Omega^1_{B_{kj}/k}, B_{kj} \otimes_k I)} \\ 
        &= \mr{Der}_k(B_{kj}, B_{kj} \otimes_k I).
    \end{align*}
    Therefore, $\alpha_k \in H^0(B_{kj}, T_X \otimes_k I)$. Note that
    \[ \varphi_{\ell}^{-1} \circ \varphi_j \circ \varphi_j^{-1} \circ \varphi_{k}^{-1} = \varphi_{\ell}^{-1} \circ \varphi_k, \]
    which implies that
    \[ (b, \alpha_{j\ell}(b) + \alpha_{kj}(b) + b') = (b, \alpha_{k\ell}(b) + b') \]
    and thus $\qty{\alpha_{kj}} \in Z^1(\qty{U_k}, T_X \otimes_k I)$.

    If two deformations are the same, note that $\varphi_k$ is defined using a ring section $s_k \colon B_k \to D_k$ of the canonical map $\pi_k \colon D_k \to B_k$. If $\varphi_k'$ is defined using another section $s_k'$, then define $\theta_k = s_k' - s_k \in \mr{Der}(B_k, I \otimes_k B_k)$. We now compute
    \[ ({(\varphi_j')}^{-1} \circ \varphi_k' - \varphi_j^{-1} \circ \varphi_k)(b, b') = (0, \theta_k(b) - \theta_j(b)), \]
    and thus the two differ by the desired coboundaries.
\end{proof}

We will now consider some obstructions. We are looking for a diagram of the form
\begin{equation*}
\begin{tikzcd}
    X' \ar{r} \ar{d}{f} & X'' \ar{d} \\
    \Spec A' \ar{r} & \Spec A''.
\end{tikzcd}
\end{equation*}
for each pair $(j, k)$, we have a isomorphism $\psi_{jk} \colon V_j' \to V_k'$ and a cocycle
\[ c_{jk\ell} = \psi_{k\ell} \circ \psi_{jk} \circ \psi_{j\ell}^{-1}. \]
This induces $B_{jk\ell} \in \mr{Der}_A(D_{jk\ell}, J \otimes_A D_{k\ell}) = Z^2(U, T_{X'/A} \otimes_A J)$.

Now we will discuss some examples.
\begin{thm}
    Let $C$ be a smooth projective curve, $T = T_C$, and $K = \Omega^1_C$. We have the following table:
    \begin{table}[H]
        \centering
        \caption{Cohomology}
        \label{tab:label}
        \begin{tabular}{ccccc}
        \toprule
            & degree & $h^0$ & $h^1$ & $h^2$ \\
            \midrule
            $K$ & $2g-2$ & $g$ & $1$ & $0$ \\
            $T$ & $2-2g$ & $\ep$ & $\ep + 3g-3$ & 0 \\
            \bottomrule
        \end{tabular}
        where $\ep = 0$ where $g \geq 2$, $\ep = 1$ if $g = 1$, and $\ep = 3$ if $g = 0$. For $g \geq 2$, $\deg T < 0$, and by Riemann-Roch and Serre duality, we have $h^1(C, T_C) = 3g-3$.
    \end{table}
\end{thm}

\begin{thm}
    $\P^n$ has no infinitesimal deformations.
\end{thm}

\begin{proof}
    Consider the Euler sequence
    \[ 0 \to \mc{O} \to {\mc{O}(1)}^{\oplus n+1} \to T_{\P^n} \to 0 \]
    and use the long exact sequence in cohomology. Because positive degree line bundles have no higher cohomology, we have $H^1(T_{\P^n}) = 0$.
\end{proof}

\chapter{Kevin (Oct 08): Deformations of coherent sheaves}%
\label{cha:kevin_oct_08_deformations_of_coherent_sheaves}

There will be no mixed characteristic funny business during this lecture. Let $X$ be a projective $k$-scheme (proper might be fine, but this makes certain facts more true) and $\mc{F}$ be a coherent sheaf on $X$. Consider the deformation functor
\[ D_{\mc{F}} \colon \ms{Art}_k \to \ms{Set} \qquad A \mapsto \qty{\mc{F}_A \in \ms{Coh}(X_A) \mid \mc{F}_A |_X \cong \mc{F}, \mc{F}_A \text{ flat over }A }. \]
We want to study the properties of this functor, which means we will check Schlessinger's conditions:\footnote{Neither Kevin nor Johan knows why these conditions are called H} Let $A \to B \gets C$ be a diagram in $\mc{C}$ and consider the morphism
\[ D(B \times_A C) \xrightarrow{r} D(B) \times_{D(A)} D(C). \]
\begin{itemize}
    \item $(H_1)$ The morphism $r$ is surjective if $C \twoheadrightarrow A$;
    \item $(H_2)$ The morphism $r$ is bijective if $C = k[\ep] \twoheadrightarrow k = A$;
    \item $(H_3)$ $\dim_k(t_D) < \infty$ (later, we will see that we need $H_2$ for formulate this). Here, $t_D$ is the tangent space to $D$;
    \item $(H_4)$ The morphism $r$ is bijective if $C \twoheadrightarrow A$.
\end{itemize}
Recall from Johan's lecture that $(H_1), (H_2), (H_3)$ are equivalent to the existence of a hull and $(H_3), (H_4)$ are equivalent to $D$ being pro-representable.

We only need to check $(H_1)$ for small extensions, which are extensions by a $k$-vector spaace
\[ 0 \to I \to C \to A \to 0, \]
where $I$ is killed by the maximal ideal of $C$.

\begin{thm}
    The functor $D_{\mc{F}}$ admits a hull.
\end{thm}

\begin{lem}
    Let $(A, \mf{m})$ be a local Artinian ring.
    \begin{enumerate}
        \item If $\mf{m} M = M$, then $M \cong 0$.
        \item If $M \to N$ induces an isomorphism $M/\mf{m} M \cong N/\mf{m}N$ and $N$ is flat over $A$, then $M \cong N$.
        \item If $M$ is flat, then $M$ is free.
    \end{enumerate}
\end{lem}

\begin{proof}
    We know that $\mf{m}^d = 0$, so $\mf{m}^d M = 0$, and thus $M = \mf{m} M = \mf{m}^2 M = \cdots = \mf{m}^d M = 0$. Next, suppose $M \to N$ induces an isomorphism after killing $\mf{m}$. Then we know that the kernel and cokernel vanish because they are killed by $\mf{m}$, so $M \to N$ must be an isomorphism. The last part is left as an exercise.
\end{proof}

\begin{proof}[Proof of theorem]
    We will simply prove $(H_1), (H_2), (H_3)$:
    \begin{enumerate}
        \item Suppose that $C \twoheadrightarrow A$ is a small extension and consider a pair $(\mc{F}_B, \mc{F}_C) \in D(B) \times_{D(A)} D(C)$. We know that we have isomorphisms $\mc{F}_B |_{X_A} \cong \mc{F}_A, \mc{F}_c |_{X_A} \cong \mc{F}_A$, and so we take the fiber product
            \[ \mc{F}_{B \times_A C} \coloneqq \mc{F}_B \times_{F_A} \mc{F}_C. \]
            We only need to show that our sheaf is flat over $B \times_A C$ because it clearly restricts to $\mc{F}_B$ and $\mc{F}_C$. We can consider each sheaf as a module $M$, and so we know $M_B$ is free over $B$ by the lemma. Choose a basis $\qty{e_i}$. Also consider the diagram
            \begin{equation*}
            \begin{tikzcd}
                M_B \times_{M_A} M_C \ar{r} \ar{d} & M_C \ar{d}{v} \\
                M_B \ar{r}{u} & M_A.
            \end{tikzcd}
            \end{equation*}
            Then $M_A$ has $A$-basis $u(e_i)$. Because $M_C$ surjects onto $M_A$, we can lift the $u(e_i)$ to $f_i \in C$, and these form a $C$-basis for $M_C$. This all implies that $M_B \times_{M_A} M_C$ is free with basis $(e_i, f_i)$.
        \item It suffices to prove injectivity. Suppose $\mc{G} \in D(B \times_k k[\ep])$ maps to $(\mc{F}_B, \mc{F}_{k[\ep]}) \in D(B) \times D(k[\ep])$, and so we have morphisms
            \begin{equation*}
            \begin{tikzcd}
                \mc{G} \ar{r} \ar{d} & \mc{F}_{k[\ep]} \ar{d} \\
                \mc{F}_B \ar{r} & \mc{F}.
            \end{tikzcd}
            \end{equation*}
            We will prove that this diagram is Cartesian. By the lemma, the morphism $\mc{G} \to \mc{F}_B \times_{\mc{F}} \mc{F}_{k[\ep]}$ is an isomorphism.
        \item We will prove that $T_D = \Ext_X^1(\mc{F}, \mc{F})$. We will only prove this in the case where $\mc{F}$ is a vector bundle $\mc{E}$ of rank $r$. In this case, we have $\Ext_X^1(\mc{E}, \mc{E}) = H^1(X, \End(\mc{E}))$. Now we will associate cocycles to deformations. To each $\mc{E}_{k[\ep]}$, we will associate an open cover $(U_j)$ and 
            \[ h_{ij} \in \Aut(\mc{O}_{X_{k[\ep]}}^{\oplus r})(U_{ij}), \]
            and we write $g_{ij} + \ep f_{ij}$, where $g_{ij} \in \Aut_{\mc{O}_X^{\oplus r}}(U_{ij})$ and $f_{ij} \in \End(\mc{O}_X^{\oplus r})(U_{ij})$. The cocycle condition is that
            \[ g_{ik} + \ep f_{ik} = (g_{ij} + \ep f_{ij})(g_{jk} + \ep f_{jk}), \]
            which is the same as 
            \[ f_{ik} = g_{ij} f_{jk} + f_{ij} g_{jk}, \]
            which is exactly the \v{C}ech $1$-cocycle condition. Proving that equivalent cocycles give the same deformation is easy. \qedhere
    \end{enumerate}
\end{proof}

\begin{thm}
    The condition $(H_4)$ holds when $\mc{F}$ is simple, which means that $k \simeq \End_X(\mc{F})$.
\end{thm}

\section{Tangent-obstruction theory}%
\label{sec:tangent_obstruction_theory}

Suppose $D$ is a deformation functor. Then a \textit{tangent-obstruction theory} for $D$ is given by finite-dimensional $k$-vector spaces $(T^1, T^2)$. Suppose we have a small extension
\[ 0 \to I \to B \to A \to 0. \]
Then we have another exact sequence
\[ T^1 \otimes_k I \to D(B) \to D(A) \xrightarrow{\mr{ob}} T^2 \otimes_k I, \]
which means that
\begin{enumerate}
    \item $\xi_A \in D(A)$ lifts to $D(B)$ if and only if $\mr{ob}(\xi_A) = 0$;
    \item $T^1 \otimes I$ acts transitively on the fibers of $D(B) \to D(A)$;
    \item If $A = k$, then the action of $T^1 \otimes I$ acts simply transitively on $D(B)$.
\end{enumerate}
Note that because $T^1$ acts simply transitively on $D(k[\ep])$, we must have $T^1 = D(k[\ep])$. On the other hand, $T^2$ is not canonical.

\begin{thm}
    The deformations $D_{\mc{F}}$ admits a tangent-obstruction theory with $T^1 = \Ext^1_X(\mc{F}, \mc{F})$ and $T^2 = \Ext^2_X(\mc{F}, \mc{F})$.
\end{thm}

\begin{proof}
    We claim that if $D$ satisfies $(H_1)$ and $(H_2)$, then $D(k[\ep]) \otimes I$ naturally acts on $D(B)$ for small extensions $0 \to I \to B \to A \to 0$. To see this, note that $D(k[\ep]) \otimes_k I = D(k[I])$. We also note that by $(H_2)$, $D(k[I]) \times D(B) = D(k[I] \times_k B)$. Now define $\alpha \colon k[I] \times_k B \to B$ by $\alpha(1+i, b) = 1 + b$, and this gives us an action of $D(k[I]) \times D(B) = D(k[I] \times_k B) \xrightarrow{\alpha_*} D(B)$. To prove transitivity, apply $(H_1)$ to the diagram
    \begin{equation*}
    \begin{tikzcd}
        k[I] \times_k B \ar{r}{\alpha} \ar{d}{\pi_B} & B \ar{d} \\
        B\ar{r} & A.
    \end{tikzcd}
    \end{equation*}
    
    Now we will consider obstructions. We will assume again that $\mc{F}$ is a rank $r$ vector bundle, which we will call $\mc{E}$. Let
    \[ 0 \to I \to B \to A \to 0 \]
    be a small extension, so we will consider $H^2(X, \End(\mc{E}))$. Consider an open cover $(U_i)$ and $g_{ij} \in \Aut(\mc{O}_X^{\oplus r} \otimes_k A)(U_{ij})$. We want to lift these to $h_{ij} \in \Aut(\mc{O}_X^{\oplus r} \otimes_k B)(U_{ij})$. If this is possible, we have a cocycle
    \[ h_{ij}^{-1} h_{ij} h_{jk} \in 1 + \End(\mc{O}_X^{\oplus r} \otimes_k I)(U_{ij}), \]
    and the cocycle condition is satisfied when $h_{ij}^{-1} h_{ij} h_{ik} = 1$. If any other $h'_{ij} = h_{ij} +  s_{ij}$, then we note that
    \[ {(h'_{ij})}^{-1} h'_{ij} h'_{jk} = h_{ik}^{-1} h_{ij} h_{jk} + (- s_{ik} g_{ij} g_{jk} + g_{ik}^{-1} s_{ij} g_{ij} + g_{ik}^{-1} g_{ij} s_{jk}), \]
    and this gives us a class in $H^2(X, \End(\mc{E})) \otimes I$.
\end{proof}

\begin{rmk}
    Let $R$ be the hull of $D$, which means we have a morphism $h_R \to D$. Then we know $R = k[[t_1, \ldots, t_{d_11}]] / (f_1, \ldots, f_{d_2})$. We also know that $d_1 - d_2 \leq \dim R \leq d_1$.
\end{rmk}

\begin{exm}[Good example]
    Let $X$ be a smooth projective curve and $\mc{E}$ be a rank $r$ vector bundle. Then we know that 
    \[ T^1 = H^1(X, \End(\mc{E})), \qquad T^2 = H^2(X, \End(\mc{E})) = 0, \]
    so deformations of $\mc{E}$ are unobstructed. Now assume that $\mc{E}$ is simple. Then $H^0(X, \End(\mc{E})) = k$ by definition, and we also know that $D_{\ep}$ is pro-represented by some ring $R$ with 
    \[ \dim R = h^1(X, \End(\mc{E})) = r^2(g-1) + 1 \]
    by Riemann-Roch.
\end{exm}

\begin{exm}[Bad example]
    Let $X$ be a smoorh projective variety and $\mc{E}$ be a rank $r$ vector bundle on $X$. Let $\mc{E}_1, \mc{E}_2 = D_{\mc{E}}(k[\ep])$. Then $(\mc{E}_1, \mc{E}_2) \in D_{\mc{E}}(k[\ep_1, \ep_2] / (\ep_1^2, \ep_1 \ep_2, \ep_2^2))$, and we would like to lift to $D_{\mc{E}}(k[\ep_1, \ep_2]/(\ep_1^2, \ep_2^2))$.

    We will compute the obstruction explicitly. We know $\mc{E}_1, \mc{E}_2$ give us classes $u_1, u_2 \in H^1(X, \End(\mc{E}))$, and after some magical computation, the obstruction to lifting is given by
    \[ u_1 \smile u_2 + u_2 \smile u_1, \]
    where the cup product comes from the algebra structure on $\End(\mc{E})$.

    Now let $X = C_1 \times C_2$ be a product of curves. Then $H^1(X, \mc{O}_X) = H^1(C, \mc{O}_{C_1}) \oplus H^1(C_2, \mc{O}_{C_2})$ and $H^2(X, \mc{O}_X) = H^1(C, \mc{O}_{C_1}) \otimes_j H^2(C_2, \mc{O}_{C_2})$. Suppose that $\alpha_1 \in H^1(C_1, \mc{O}_{C_1})$ and $\alpha_2 \in H^1(C_2, \mc{O}_{C_2})$ with nonzero cup product. Then we simply set $\mc{E} = \mc{O}_X \oplus \mc{O}_X$ and 
    \[ u_1 = \mqty(0 & \alpha_1 \\ 0 & 0) \qquad u_2 = \mqty(0 & 0 \\ \alpha_2 & 0) \qquad u_1 \smile u_2 + u_2 \smile u_1 = \mqty(\dmat{\alpha_1 \smile \alpha_2, \alpha_2 \smile \alpha_1}). \]
    this gives us our obstructed deformation.
\end{exm}

\chapter{Patrick (Oct 15): Deformations of singularities}%
\label{cha:patrick_oct_15_deformations_of_singularities}

We begin by fixing some notation.
Let $k$ be a field and $R = P/I$, where $P = k[x_1, \ldots, x_n]$ and $I = (f_1, \ldots, f_r)$ is an ideal. Throughout this lecture, we will denote local Artinian rings with residue field $k$ by $A,B,C,\ldots$ and rings by $R,S,T,\ldots$ Finally, denote $Z = \Spec R$.

\section{Explicit criteria for flatness}%
\label{sec:explicit_criteria_for_flatness}

We will study (embedded) deformations of singular affine schemes embedded in $\A^n$. The first thing we want to understand is to explicitly understand flatness of some $R_A$ over $A$, where $R_A \otimes_A k = R$. We will write $R_A = P_A / I_A$, where $P_A = A[x_1, \ldots, x_n] = A \otimes_k P$. Recall that over a Noetherian local ring $S$ with residue field $k$, a module $M$ is flat if and only if it is free, and this is equivalent to $\Tor_1^S(M, k) = 0$ by standard results in commutative algebra.

Now consider the exact sequence
\[ 0 \to I_A \to P_A \to R_A \to 0. \]
After tensoring with $k$, we have 
\[ 0 \to \Tor_1(R_A, k) \to I_A \otimes_A k \to P \to R \to 0. \]
Therefore, we know that $R_A$ is flat over $A$ if and only if $I_A \otimes_A k = I$. We would like to understand this statement.

Consider a presentation
\[ P^s_A \to P_A^r \to I_A \to 0 \]
of $I_A$. Then we know $R_A$ is flat over $A$ if and only if after tensoring with $k$, we obtain an exact sequence
\[ P^s \to P^r \to I \to 0. \]
Note that to give this presentation $P^s \to P^r \to I \to 0$ is the same as giving a complete set of relations among the generators of $I$.

\begin{prop}
    Suppose that 
    \begin{equation} P^s \to P^r \to P \to R \to 0 \end{equation}
    is exact and
    \begin{equation} P_A^s \to P_A^r \to P_A \to R_A \to 0 \end{equation}
    is a complex such that $P_A^r \to R_A \to R_A \to 0$ is exact and tensoring (2) with $k$ gives (1). Then $R_A$ is flat over $A$.
\end{prop}

\begin{proof}
    Note that the hypotheses are equivalent to the fact that all relations in $I$ can be lifted to $I_A$. Now given $g_1', \ldots, g_r' \in P_A$ such that 
    \[ \sum_{i=1}^r g_i' f_i' = 0, \]
    this clearly descends to a relation in $I$ by killing the maximal ideal of $A$. But now if we choose a complete set of relations for $I_A$, this descends to a complete set of relations in $I$, so we may in fact assume that (2) is exact.

    In this case, there exists some $L_A$ such that the sequence splits as
    \[ P_A^s \to L_A \to 0 \qquad 0 \to L_A \to P_A^r \to I_A \to 0 \qquad 0 \to I_A \to P_A \to R_A \to 0. \]
    By right exactness of the tensor product, we know $P_A^{s} \otimes k \to L_A \otimes k \to 0$ is exact. We also know that
    \[ L_A \otimes k \to P_A^r \otimes k \to I_A \otimes k \to 0 \]
    is exact, again by right exactness. But this means that $I_A \otimes k$ is the cokernel of $P^s \to P^r$, and therefore $I_A \otimes k = I$. This means that $R_A$ is flat.
\end{proof}

\begin{cor}
    Let $R = P/I$ and $R_A = P_A / I_A$, where $I = (f_1, \ldots, f_r)$ and $I_A = (f_1', \ldots, f_r')$ such that $f_i'$ is a lift of $f_i$. Then $R_A$ is flat over $A$ if and only if every relation among the $f_i$ lifts to a relation among the $f_i'$.
\end{cor}

\begin{rmk}
    This result essentially gives us that first-order embedded deformations of $\Spec R \subset \A^n$ are given by $\Hom(I, R)$. The first-order (not embedded) deformations of $Z$ are given by the cokernel of
    \[ 0 \to T_X \to T_{\A^n}|_X \to N_{X/\A^n}, \]
    which arises from the exact sequence
    \[ I/I^2 \to \Omega^1_{\A^n}|_X \to \Omega^1_X \to 0, \]
    and this is supported on the singular points of $X$, so when $X$ has isolated singularities, this is finite-dimensional.
\end{rmk}

Note that if $\Spec R \subset \A^n$ is a complete intersection, then $I$ is generated by a regular sequence, so in particular the Koszul complex is a free resolution of $R$ and therefore there are only trivial relations among the $f_i$ (this means the relations are generated by $f_i f_j - f_j f_i = 0$). Clearly, because we are only considering commutative rings (after all, this is normal algebraic geometry), this means that all deformations of $\Spec R$ are unobstructed.

\section{Hilbert schemes of smooth surfaces}%
\label{sec:hilbert_schemes_of_smooth_surfaces}

We will prove that deformations of finite length closed subschemes of $\A^2$ are unobstructed. In particular, this will imply that the Hilbert scheme $\mr{Hilb}(\A^2, n)$ is smooth.

Let $Z \subset \A^2$ be a closed subscheme of dimension $0$. Then because $P = k[x,y]$ has dimension $2$, there exists a free resolution
\[ 0 \to P^s \xrightarrow{(g_{ij})} P^r \to P \to R \to 0 \]
of $R$. In this case it is possible to understand the matrix $(g_{ij})$, and in fact this is the special case of a more general result. First, when we study the local behavior, we have the following result.
\begin{thm}[Hilbert, Burch]
    Let $P$ be a regular local ring of dimension $n$ and $R = P/I$ be a Cohen-Macaulay quotient of codimension $2$. Then there exists an $(r-1) \times r$ matrix $G = (g_{ij})$ whose maximal minors $f_1, \ldots, f_r$ minimally generate $I$, and there is a free resolution
    \[ 0 \to P^{r-1} \xrightarrow{(g_{ij})} P^r \xrightarrow{(f_i)} P \to R \to 0. \]
\end{thm}

\begin{proof}
    Note that the fact that the free resolution has this length is a corollary of the Auslander-Buchsbaum formula, which says that for a ring $R$ and module $M$, we have
    \[ \depth M + \operatorname{proj.dim} M = \depth R \]
    and the fact that depth equals dimension for Cohen-Macaulay things. Thus we have a free resolution
    \[ 0 \to P^{r-1} \xrightarrow{(g_{ij})} P^r \xrightarrow{(a_i)} P \to R \to 0, \]
    where $a_1, \ldots, a_r$ are a minimal set of generators for $I$. Let $f_i$ is ${(-1)}^i$ times the determinant of the $i$-th minor of $g_{ij}$. We will prove that the map $(f_i)$ is the same as the map $(a_i)$; clearly
    \[ 0 \to P^{r-1} \xrightarrow{(g_{ij})} P^r \xrightarrow{(f_i)} P \to R \to 0. \]
    is a resolution. This is because at the generic point of $P$, we know $(g_{ij})$ is injective, so at least one $f_i$ is nonzero. But then we know $\coker (g_{ij})$ is torsion-free (because $I$ is torsion-free), and so it in fact must vanish by rank reasons. Thus $(a_1, \ldots, a_{r})$ and $(f_1, \ldots, f_{r})$ are isomorphic as $P$-modules.

    At a codimension $1$ point in $\Spec P$, note that $0 \to P^{r-1} \to P^r \xrightarrow{(a_i)} P \to B \to 0$ is split exact (because $I$ has codimension $2$). This implies that at least one $f_i$ is a unit, and thus $(f_1, \ldots, f_r)$ has codimension at least $2$. But then the isomorphism $I \cong (f_1, \ldots, f_r)$ is given by multiplication by some nonzero element of $P$ which is a unit away from codimension $2$. But this means it is a unit everywhere.
\end{proof}

Considering the global picture in $\A^n$, we obtain the following result.
\begin{thm}[Hilbert, Schaps]
    Let $Z = \Spec R \subset \A^n$ be a Cohen-Macaulay closed subscheme of codimension $2$. Then $R = P/I$ has a free resolution of the form
    \[ 0 \to P^{r-1} \xrightarrow{(g_{ij})} P^r \xrightarrow{(f_i)} P \to R \to 0 \]
    where the $f_i$ are the maximal minors of the matrix $(g_{ij})$.
\end{thm}

This result in fact holds over any Artinian local ring $A$, which we will use later.

Next, we want to understand what happens if we choose some Artinian local ring with residue field $k$ and lift the $g_{ij}$ to $g_{ij}'$, where $g_{ij}' \in P_A$.

\begin{thm}[Schaps]
    If $A$ is a square zero extension of $k$, then the sequence
    \[ 0 \to P_A^{r-1} \xrightarrow{(g_{ij}')} P_A^r \xrightarrow{(f_i')} P_A \to R_A \to 0 \]
    is exact. Moreover, any lifting of $R$ over $A$ arises by lifting the matrix $(g_{ij})$.
\end{thm}

\begin{proof}
    We know that 
    \[ L_A^{\bullet} \coloneqq P_A^{r-1} \to P_A^r \to P_A \]
    is a complex. This is because composing the two maps amounts to evaluating determinants with a repeated column. Because $P_A$ is free (and therefore flat), we can tensor with the exact sequence
    \[ 0 \to \mf{m}_A \to A \to k \to 0 \]
    to obtain an exact sequence of complexes
    \[ 0 \to L_A^{\bullet} \otimes_A \mf{m}_A \to L_A^{\bullet} \to L_A^{\bullet} \otimes_A k \to 0. \]
    Note that 
    \[ L_A^{\bullet} \otimes_A k = P^{r-1} \xrightarrow{(g_{ij})} P^r \xrightarrow{(f_i)} P \eqqcolon L^{\bullet}. \]
    In particular, this term is exact by Hilbert-Schaps. In addition, clearly $L_A^{\bullet} \otimes_A \mf{m}_A = L^{\bullet} \otimes_k \mf{m}_A$ because $A \to k$ is a square zero extension, so the complex $L_A^{\bullet} \otimes_A \mf{m}_A$ is exact. By the long exact sequence in homology, we know that $L_A^{\bullet}$ is exact. Note that $L^{\bullet}$ extends to an exact sequence
    \[ 0 \to P^{r-1} \to P^r \to P \to R \to 0, \]
    and $L_A^{\bullet}$ extends to an exact sequence
    \[ 0 \to P_A^{r-1} \to P_A^r \to P_A \to R_A \to 0. \]
    However, by the homology long exact sequence, we have an exact sequence
    \[ 0 \to R \otimes_k \mf{m}_A \to R_A \to R \to 0. \]
    But this implies that $R_A \otimes_A k = R$. Finally, by the local criterion for flatness, we see that $R_A$ is flat over $A$.

    Let $R_A = P_A / I_A$ be a lifting of $R$ over $A$. Lift $f_i \in I$ to $h_i \in I_A$. By Nakayama, these generate $I_A$, so we obtain a free resolution
    \[ 0 \to P_A^{r-1} \xrightarrow{(g_{ij}')} P_A^r \xrightarrow{(h_i)} P_A \to R_A \to 0, \]
    where $g_{ij}'$ lift the $g_{ij}$. However, we already have a lift
    \[ 0 \to P_A^{r-1} \xrightarrow{(g_{ij}')} P_A^r \xrightarrow{(f_i')} P_A \to R_A' \to 0, \]
    and so we must show $R_A = R_A'$. But we know that the ideals $I_A = (h_1, \ldots, h_r)$ and $I_A' = (f_1', \ldots, f_r')$ are isomorphic as $P_A$-modules. But then if we restrict this isomorphism to $\A^n_A \setminus \supp B$, we obtain a unit in $H^0(\A^n_A \setminus \supp B, \mc{O}_{\A^n_A})$. Because functions extend over codimension $2$, we have $H^0(\A^n_A \setminus \supp B, \mc{O}_{\A^n_A}) = P_A$, so this is a global unit. This gives the desired result.
\end{proof}

This result holds if we replace $A \to k$ with any square-zero extension of Artinian local rings $B \to A$ and $P, P_A$ with flat things, and so we see that (embedded) deformations of codimension $2$ Cohen-Macaulay subschemes of $\A^n$ are unobstructed. In particular, any dimension $0$ closed subscheme $Z \subset \A^2$ is automatically Cohen-Macaulay (because it is dimension $0$), so its embedded deformations are unobstructed. By some cohomological argument, the tangent space to $\mr{Hilb}(\A^2, n)$ is isomorphic to $\Hom(R, R)$ and has dimension $2n$, so 

\section{An obstructed deformation}%
\label{sec:an_obstructed_deformation}

Let $R = k[x,y,z]/(z^2, xy, xz, yz)$. Note that this scheme has an embedded point at the origin, so in particular it is \textbf{not} Cohen-Macaulay.

\begin{figure}[H]
\begin{center}
\begin{tikzpicture}[scale=1, transform shape]
    \draw[-,thick] (0,-2) -- (0,2);
    \draw[-,thick] (-2,0) -- (2,0);
    \filldraw[black] (0,0) circle (2pt);
    \draw[->] (0,0) -- (0.5,0.5) node[above,right] {$z$};
\end{tikzpicture}
\end{center}
\caption{Drawing of $\Spec R$}%
\label{fig:}
\end{figure}

We will study embedded deformations of $\Spec R$ and see that they are obstructed. In particular, we will choose two deformations of $R$ over $k[\ep]$ that cannot be simultaneously lifted. We claim that a complete set of relations (using the ordering $(xy, xz, yz, z^2)$ for the generators of $I$) is given by the matrix
\[ G = \mqty(z & -y & 0 & 0 \\ z & 0 & -x & 0 \\ 0 & z & 0 & -x \\ 0 & 0 & z & -y). \]

Now a first-order deformation of $\Spec R$ is given by lifting $(xy,xz,yz,z^2)$ over $k[\ep]$, and the first candidate is to consider $I_{\ep_1} = (xy + \ep_1 y, xz, yz, z^2)$. Then we note that
\[ G \mqty(xy+\ep_1 y \\ xz \\ yz \\ z^2) = \ep_1 \mqty(yz \\ yz \\ 0 \\ 0), \]
and we can lift $G$ to kill this vector with the matrix
\[ G_{\ep_1} = \mqty(z & -y & -\ep_1 & 0 \\ z & 0 & -x-\ep_1 & 0 \\ 0 & z & 0 & -x \\ 0 & 0 & z & -y) = G + \mqty(0 & 0 & -\ep_1 & 0 \\ 0 & 0 & -\ep_1 & 0 \\ 0 & 0 & 0 & 0 \\ 0 & 0 & 0 & 0) \eqqcolon G + G_1. \]

Next consider the deformation given by $I_{\ep_2} = (xy, xz, yz+\ep_2 z, z^2)$. We note that
\[ G \mqty(xy \\ xz \\ yz + \ep_2 z \\ z^2) = \ep_2 \mqty(0 \\ -xz \\ 0 \\ z^2), \]
and we can lift $G$ to kill this vector with the matrix
\[ G_{\ep_2} = \mqty(z & -y & 0 & 0 \\ z & \ep_2 & -x & 0 \\ 0 & z & 0 & -x \\ 0 & 0 & z & -y - \ep_2) = G + \mqty(0 & \ep_2 & 0 & 0 \\ 0 & 0 & 0 & 0 \\ 0 & 0 & 0 & -\ep_2 \\ 0 & 0 & 0 & 0) \eqqcolon G + G_2. \]

Now we consider $I_{\ep_1^2, \ep_2^2, \ep_1\ep_2} = (xy+\ep_1 y, xz, yz + \ep_2 z, z^2)$ and attempt to lift this deformation to $k[\ep_1, \ep_2]/(\ep_1^2, \ep_2^2)$. Note that 
\begin{align*} 
    (G + G_1 + G_2) \mqty(xy+\ep_1 y \\ xz \\ yz+\ep_2 z \\ z^2) &= \mqty(z & -y & -\ep_1 & 0 \\ z & \ep_2 & -x-\ep_1 & 0 \\ 0 & z & 0 & -x \\ 0 & 0 & z & -y - \ep_2) \mqty(xy+\ep_1 y \\ xz \\ yz+\ep_2 z \\ z^2) \\
    &= \ep_1 \ep_2 \mqty(-z \\ -z \\ 0 \\ 0),
\end{align*}
and clearly $z \notin I$, so in fact we cannot lift this deformation to $k[\ep_1, \ep_2]/(\ep_1^2, \ep_2^2)$. This proves obstructedness.

\chapter{Avi (Oct 22): Local-global methods}%
\label{cha:avi_oct_22_local_global_methods}

\section{Curves with isolated singularities}%
\label{sec:curves_with_isolated_singularities}

Here, we will consider the same lifting problems that we have consider before. Let $X$ be a curve over $k$ and consider $p \colon \mr{Def}_X \to \mc{C}$ be the deformations of $X$ as a category cofibered over $\mc{C}$. 

\begin{defn}
    A functor $F \colon D_1 \to D_2$ is \textit{smooth} if given $\varphi \colon A' \twoheadrightarrow A$ and $(Z, A') \to F(Y, Z)$ over $\varphi$ over $\varphi$ and $(Y', A') \to (Y, A)$ over $\varphi$ in $D_1$, then there exists some $F(Y', A') \to (Z, A')$ over the identity in $A'$.
\end{defn}

There is an absolute notion, for $D_1$ to be smooth, where we set $D_2 = \mr{Def}_k$ to be the category of trivial deformations. In this case, we also called $D_1$ \textit{unobstructed}, and this corresponds to being unobstructed in the tangent-obstruction theory.

\begin{thm}[Local-to-global]
    Let $X/k$ be separated of dimension at most $1$ and smooth away from finitely many points. At each singularity $p_1, \ldots, p_n$, consider the inclusions
    \[ \mc{O}_{X,p} \to \mc{O}_{X,p}^h \to \wh{\mc{O}}_{X,p} \]
    into the henselizations and completions, respectively. Then the functors
    \[ \mr{Def}_X \to \prod_i \mr{Def}_{\mc{O}_{X,p}} \xrightarrow{(2)} \prod \Def_{\mc{O}_{X,p}^h} \xrightarrow{(3)} \prod \Def_{\wh{\mc{O}}_{X,p}} \]
    are all smooth and $(2), (3)$ induce isomorphisms on tangent spaces.
\end{thm}

This means that we only have to check unobstructedness at the completions of the local rings at each singular point.

\begin{proof}
    First, we can reduce to the affine case.\footnote{Avi says that Johan sketched this proof and then said not to give it.} If $X$ is a curve, then $X = U_1 \cup U_2$ can be covered by two affine opens such that $U_1, U_1 \cap U_2$ are smooth. Because deformations of smooth affine schemes are unobstructed, we can essentially ignore $U_1$. To do this, we will prove that $\Def_X \to \Def_{U_2}$ is smooth.

    Because smoothness (roughly) respects products, the functor
    \[ \Def_X \to \Def_{U_1} \times_{\Def_{U_1 \cap U_2}} \Def_{U_2} \]
    is smooth by formal reasons. Because $U_1, U_1 \cap U_2$ are smooth, we can ignore $\Def_{U_1}, \Def_{U_1 \cap U_2}$ (in fact, the arrow is an equivalence) and project down to $\Def_{U_2}$.

    Now $U_2 = \Spec P$ is affine. Then each $p_i$ gives a maximal ideal $\mf{m}_i$, and so we set $J = \bigcap_i \mf{m}_i$. If we consider the completion $\wh{P}_J$ with respect to $J$, this decomposes as
    \[ \wh{P}_J = \prod_i \wh{P}_i. \]
    Applying \href{https://stacks.math.columbia.edu/tag/0DZ5}{Lemma 91.12.5} in the Stacks Project, the functor
    \[ \Def_P \to \Def_{\wh{P}_J} \simeq \prod_i \Def_{\wh{P}_i} = \prod_i \Def_{\wh{\mc{O}}_{X,p_i}} \]
    is smooth and an isomorphism on tangent spaces. The henselization step is similar.
\end{proof}

\begin{exm}
    Let $X = \Spec k[x,y]/(xy)$. Then $\wh{O}_{X,0} = k[[x,y]] / (xy)$. We will show that all deformations look like $k[x,y] / (xy-t)$ in some sense. Really, we will show that $\Def_X$ has a hull, which is given by $k[[t]]$ with universal deformation
    \[ X_t \coloneqq \Spec k[[t]] [x,y] / (xy-t) \to \Spec k[[t]]. \]
    We want $T$ such that $\Hom(T, -) \to \Def_X$ is smooth and induces an isomorphism on tangent spaces. Given a morphism
    \[ f \colon k[[t]] \to A \ni f(t), \]
    we obtain a deformation
    \[ A[[x,y]] / (xy - f(t)) \]
    of $k[[x,y]]/(xy)$. We can check that this is smooth, and so we need to check that we have an isomorphism of tangent spaces.

    We want to compute $\Def_X(k[\ep])$. If $S \in \Def_X(k[\ep])$, set $R = k[[x,y]]/(x,y)$, and we have a diagram
    \begin{equation*}
    \begin{tikzcd}
        k[\ep] \ar{r} \ar{d} & k \ar{d} \\
        S \ar{r} & R.
    \end{tikzcd}
    \end{equation*}
    If we consider the exact sequence
    \[ 0 \to k \to k[\ep] \to k \to 0 \]
    and tensor with $S$, by flatness we have an exact sequence
    \[ 0 \to k \otimes_{k[\ep]} S \to S \to k \otimes_{k[\ep]} S \to 0. \]
    But because $k \otimes_{k[\ep]} S = R$, we have a decomposition $S = R \oplus \ep R$. If we choose lifts $\wt{x}, \wt{y}$ of $x,y$, the product $\wt{x} \wt{y}$ lifts $xy = 0$, so $\wt{x} \wt{y} = (f(x) + g(y)) \ep$. If we choose different lifts $\wt{x}' = \wt{x} + \ep h, \wt{y}' = \wt{y} + \ep h$, we see that 
    \[ \wt{x}' \wt{y}' = \wt{x} \wt{y} + (xj + yh)\ep. \]
    But now $xj+yh \in (x,y)$, and so if we set $t = f(0) + g(0)$, then we can write $\wt{x} \wt{y} \in t \ep + (x,y) \ep$, and so there is a canonical choice of the element of $(x,y) \ep$, namely $0$. Thus we can set
    \[ S_t = k[\ep][[\wt{x}, \wt{y}]] / (\wt{x} \wt{y} - t \ep), \]
    and now we have a diagram
    \begin{equation*}
    \begin{tikzcd}
        0 \ar{r} & R \ar{r} \ar[equals]{d} & S_t \ar{r} \ar{d} & R \ar{r} \ar[equals]{d} & 0 \\
        0 \ar{r} & R \ar{r} & S \ar{r} & R \ar{r} & 0.
    \end{tikzcd}
    \end{equation*}
    By the five lemma, $S_t \simeq S$, and so we are done.
\end{exm}

\section{Smooth curves with a finite group action}%
\label{sec:smooth_curves_with_a_finite_group_action}

In this section, we will consider lifting from characteristic $p>0$ to characteristic $0$. This means we will have to redefine what lifting. Here, we will suppose $k = \ol{k}$ with characteristic $p$. Let $\Lambda$ be a Noetherian complete local ring with residue field $k$. We define $\mc{C}_{\Lambda}$ be the category of local Artinian $\Lambda$-algebras with residue field $k$.\footnote{Apparently this is interesting in \sout{infinite combinatorics} number theory.}

Forgetting about the requirement that $k$ is algebraically closed, if $k = \F_p$, then we can set $\Lambda = \Z_p$. Let $W = W(k)$ be the ring of Witt vectors in $k$. This is apparently some universal Noetherian local ring lifting $k$, and for example, $W(\F_p) = \Z_p$, and $W(\ol{\F}_p) = \mc{O}_{\check{\Q}_p}$. We will consider some extension $\mc{O}/W$ of rings and write $K$ for the fraction field of $\mc{O}$.

We are interested in deforming smooth proper curves $X/k$ with the action of some finite group $G$, conveniently denoted as a pair $(X, G)$.

\begin{defn}
    The pair $(X, G)$ \textit{lifts to characteristic $0$} if there exists $\mc{O}$ and $\mc{X} / \mc{O}$ a projective smooth curve with an action of $G$ such that $(\mc{X} \times_0 k, G) \simeq (X, G)$.
\end{defn}

Given the action of $G$ on $X$, we can construct the quotient $X/G$. To remember the data of the action, we remember the ramified Galois cover $\pi \colon X \to X/G$.\footnote{This is useful if you want to look at some of the omitted proofs.}

Consider the action of $\mr{PGL}_2(\F_q)$ on $\P^1_k$. This includes in $\P^1_{\mc{O}}$, which has an action of $\mr{PGL}_2(\mc{O}) \subset \mr{PGL}_2(K)$. We want an embedding $\mr{PGL}_2(\F_q) \subseteq \mr{PGL}_2(K)$, but generally this is not possible. For example, set $q = 9$ and consider the matrices
\[ \mqty(0 & -1 \\ 1 & -1), \qquad \mqty(a+1 & a \\ -a & 1-a), \]
which generate a $(\Z/3)^3$. But then all finite subgroups of $\mr{PGL}_2(K)$ are cyclic, dihedral, $A_4$, $A_5$, or $S_4$, so this action clearly cannot lift.

\begin{prop}
    Suppose $(X,G)$ is such that for all $x \in X$, the stabilizer $G_x$ has order $\abs{G_x}$ prime to $p$. Then $(X, G)$ lifts to characteristic $0$.
\end{prop}

Note that this is some local statement, so we need some kind of local-to-global method. Recall that $\wh{\mc{O}}_{X,x} = k[[t]]$, and this has an action of $G_x$. Thus we define a local action as some action of a finite group $G$ on $k[[t]]$, and this lifts if there exists $\mc{O}$ with an action of $G$ on $\mc{O}[[t]]$ specializing to the original action on $k[[t]]$.

\begin{thm}
    For $(X, G)$, suppose that the local action of $G_x$ on $\wh{\mc{O}}_{X,x}$ lifts to characteristic $0$. Then $(X, G)$ lifts to characteristic $0$.\footnote{Avi was lured in to a paper with French title and French body but English abstract, and in the end did not read this paper.}
\end{thm}

It is a fact that if $G$ acts faithfully on $k[[t]]$, then $G = P \rtimes C$, where $P$ is $P$-Sylow and $C$ is cyclic.

\begin{proof}[Proof of proposition]
    We will simply prove that if $\abs{G_x}$ is prime to $p$, then the local action lifts. Then we know that $G_x = G$ is cyclic of order $n$ with generator $\sigma$. If we write
    \[ \sigma^i t = \sum_{m \geq 0} a_{m,i} t^i, \]
    then of course if $a_{0,i} \neq 0$, then $\sigma^i t$ is a unit and hence $t$ is a unit, so $a_{0,i} = 0$. Our goal is to replace $t$ by some generator with a more explicit action of $\sigma$. Let $V$ be the vector space spanned by $\sigma^i t$. Then on our designated basis, we have
    \[ \sigma = \mqty( & & & & & 1 \\ 1 & & & & & \\ & 1 & & & & \\ & & 1 & & & \\ & & & \ddots & & \\ & & & & 1 &). \]
    This has eigenvalues the $n$-th roots of $1$, so we choose $z \in V$ such that $\sigma z = \zeta_n z$ and replace $k[[z]] = k[[t]]$. But now the action of $\mu_n$ clearly lifts to characteristic $0$, so we are done.
\end{proof}

We have essentially proved that local actions of cyclic groups of order prime to $p$ lift to characteristic $0$. Here is a mild generalization:

\begin{conj}[Oort\footnote{Fun fact: Oort was Johan's advisor.}]
    The local action on $k[[t]]$ by the action of any cyclic $G$ lifts to characteristic $0$.
\end{conj}

This is now a theorem due to (among others?) Obus, Wewers, and Pop. Because of this, if the stabilizers $G_x$ are all cyclic, then $(X, G)$ lifts. However, this is not a necessary condition for lifting. If we consider an action of $(\Z/2)^2$ on $\P^1_k$, this does lift (because $D_4 = (\Z/2)^2$).

\chapter{Caleb and Morena (Oct 29): \href{https://stacks.math.columbia.edu/tag/0E3X}{Lemma 0E3X} and applications to contracting curves}%
\label{cha:caleb_and_morena_oct_29_https_lemma_0e3x_and_applications_to_contracting_curves}

\section{One lemma in the Stacks Project}%
\label{sec:one_lemma_in_the_stacks_project}

\begin{lem}
    In \href{https://stacks.math.columbia.edu/tag/0DY7}{Example 0DY7} let $f \colon X \to Y$ be a morphism of schemes over $k$. If $f_* \mc{O}_X = \mc{O}_Y$ and $R^1 f_* \mc{O}_X = 0$, then the morphism of deformation categories
    \[ \mr{Def}_{X \to Y} \to \mr{Def}_X \]
    is an equivalence.
\end{lem}

Let $A$ be an Artinian local ring with residue field $k$. Remember that $\mr{Def}_X(A)$ is the set of isomorphism classes of diagrams 
\begin{equation}
\begin{tikzcd}
    X \ar{r} \ar{d} & X_A \ar{d}{\alpha} \\
    \Spec k \ar{r} & \Spec A
\end{tikzcd}
\end{equation}
with $\alpha$ flat. Also, $\mr{Def}_{X \to Y}$ consists of diagrams of the form
\begin{equation*}
\begin{tikzcd}
    & Y \ar{rr} \ar{ddl} & & Y_A \ar{ddl}{\beta} \\
    X \ar{ur}{f} \ar{rr} \ar{d} & & X_A \ar{ur}{f_A} \ar{d}{\alpha} \\
    \Spec k \ar{rr} & & \Spec A
\end{tikzcd}
\end{equation*}
with $\alpha, \beta$ flat.

\begin{lem}[\href{https://stacks.math.columbia.edu/tag/063Y}{Lemma 063Y}]
    Let $(f, f') \colon (X, X') \to (S, S')$ be a morphism of first order thickenings such that $f$ is flat. Then the following are equivalent:
    \begin{enumerate}
        \item $f'$ is flat and $X = S \times_{S'} X'$;
        \item The canonical map $f^* C_{S/S'} \to C_{X/X'}$ is an isomorphism, where $C$ is the conormal sheaf.
    \end{enumerate}
\end{lem}

\begin{proof}
    In the affine case, write $X = \Spec B, X' = \Spec B', S = \Spec A, S' = \Spec A'$. Then we are looking for a diagram of the form
    \begin{equation*}
    \begin{tikzcd}
        0 \ar{r} & I \ar{r} \ar{d} & A' \ar{d}{f'} \ar{r} & A \ar{r} \ar{d}{f} & 0 \\
        0 \ar{r} & J \ar{r} & B' \ar{r} & B \ar{r} & 0.
    \end{tikzcd}
    \end{equation*}
    The two conditions become
    \begin{enumerate}
        \item $f$ is flat and $B = B' \otimes_{A'} A$;
        \item $I/I^2 \otimes_A B = J /J^2$ and $I \otimes_A B = J$.
    \end{enumerate}
    To begin, note that $B = B' \otimes_A' A$ is equivalent to $B' / J = B' \otimes_A' A'/I$, and this implies that $J = IB'$ and thus that $I \otimes_A B' \to J$ is surjective. To prove injectivity, by flatness of $B'$, the map $0 \to I \to A'$ remains injective after tensoring with $B'$, so $I \otimes_{A'} B' \to B'$ is injective.

    In the other direction, we may cite \href{https://stacks.math.columbia.edu/tag/051C}{Lemma 051C}. Alternatively, we give the following argument. Assuming that $I \otimes_A B \to J$ is an isomorphism, we know $J = IB'$, and thus $B = B' \otimes_{A'} A$. To prove that $B'$ is flat over $A'$, we know that $B'/IB'$ is flat over $A$ because $B/A$ is flat and $J = IB'$. We will prove that if $\mf{a} \subset A'$ is an ideal, then $\mf{a} \otimes_{A'} B' \to B'$ is injective.

    By some inexplicable brilliancy, we simply need to fill in the diagram
    \begin{equation*}
    \begin{tikzcd}
        & ? \ar{r} \ar[hookrightarrow]{d} & \mf{a} \otimes_{A'} B' \ar{r} \ar{d}{?} & ? \ar{r} \ar[hookrightarrow]{d} & 0 \\
        0 \ar{r} & IB' \ar{r} & B' \ar{r} & B'/IB' \ar{r} & 0.
    \end{tikzcd}
    \end{equation*}
    By diagram chasing reasons, we will have exactness. Consider the exact sequence
    \[ 0 \to I \cap \mf{a} \to \mf{a} \to (I+\mf{a})/I \to 0. \]
    After tensoring with $B'$, we obtain a right exact sequence
    \[ (I \cap \mf{a}) \otimes_{A'} B' \to \mf{a} \otimes_{A'} B' \to (I + \mf{a})/I \otimes_{A'} B' \to 0. \]
    This gives us the desired items in the question marks.

    Now we want to prove that $(I \cap \mf{a}) \otimes_{A'} B' \to IB'$ is injective. If we consider $0 \to I \cap \mf{a} \to I$ and tensor with $B' / IB'$, we obtain
    \[ 0 \to (I \cap \mf{a}) \otimes_A B'/IB' \to I \otimes_A B' / IB', \]
    but this is clearly actually
    \[ (I \cap \mf{a}) \otimes_{A'} B' \hookrightarrow I \otimes_A B'. \]
    For the other part, we simply take
    \[ 0 \to (I+\mf{a})/I \to A \]
    and tensor with $B'/IB'$.
\end{proof}

\begin{proof}[Proof of \href{https://stacks.math.columbia.edu/tag/0E3X}{Lemma 0E3X}]
    We need to prove that $\beta \colon (Y, f_* \mc{O}_{X_A}) \to \spec A$ is flat. We can compose any thickening as a sequence
    \begin{equation*}
    \begin{tikzcd}
        X \ar{r} \ar{d}{\alpha_1} & X_2 \ar{d}{\alpha_2} \ar{r} & \cdots \ar{r} & X_{n-1} \ar{d}{\alpha_{n-1}} \ar{r} & X_A \ar{d}{\alpha_n} \\
        \Spec k \ar{r} & \Spec A/\mf{m}_A^2 \ar{r} & \cdots \ar{r} & \Spec \mf{m}_A^{n-1} \ar{r} & \Spec A.
    \end{tikzcd}
    \end{equation*}
    Now we apply \href{}{Lemma 063Y} to each square and obtain
    \[ \mc{O}_X \otimes_k \mf{m}_A^i / \mf{m}_A^{i+1} = \alpha_i^* (\mf{m}_A^i / \mf{m}_A^{i+1}) = \mf{m}_A^i \mc{O}_{X_A} / \mf{m}_A^{i+1} \mc{O}_{X_A}. \]
    Now if we consider the exact sequence
    \[ 0 \to \mf{m}_A^i \mc{O}_{X_A} / \mf{m}_A^{i+1} \mc{O}_{X_A} \to \mc{O}_{X_A} / \mf{m}_A^{i+1} \mc{O}_{X_A} \to \mc{O}_{X_A} / \mf{m}_A^i \mc{O}_{X_A} \to 0. \]
    Applying $f_*$ and the assumption that $R^1 f_* \mc{O}_X = 0$, we obtain an exact sequence
    \[ 0 \to f_* \mc{O}_X \otimes_k \mf{m}_A^i / \mf{m}_A^{i+1} \to f_* (\mc{O}_{X_A} / \mf{m}_A^{i+1} \mc{O}_{X_A}) \to f_* (\mc{O}_{X_A} / \mf{m}_A^i \mc{O}_{X_A}) \to 0. \]
    Now if we consider the diagram
    \begin{equation*}
    \begin{tikzcd}
        Y \ar{r} \ar{d}{\beta} & Y_2 \ar{d}{\beta} \ar{r} & \cdots \ar{r} & Y_{n-1} \ar{d}{\beta{n-1}} \ar{r} & Y_A \ar{d}{\beta} \\
        \Spec k \ar{r} & \Spec A/\mf{m}_A^2 \ar{r} & \cdots \ar{r} & \Spec \mf{m}_A^{n-1} \ar{r} & \Spec A,
    \end{tikzcd}
    \end{equation*}
    we want to prove that $\beta$ is flat starting with $\beta_1$ being flat. But here, we know
    \[ \mc{O}_{Y_A} = f_* \mc{O}_{X_A}, \]
    and therefore we have
    \begin{align*}
        \beta_i^* (\mf{m}_A^i / \mf{m}_A^{i+1}) &= \mc{O}_{Y_i} \otimes_{A/\mf{m}_A^i} \mf{m}_A^i / \mf{m}_A^{i+1} \\
        &= \mc{O}_Y \otimes_k \mf{m}_A^i / \mf{m}_A^{i+1} \\
        &= \mf{m}_A \mc{O}_{Y_A} / \mf{m}_A^{i+1} \mc{O}_{Y_A}.
    \end{align*}
    Now we may apply \href{https://stacks.math.columbia.edu/tag/063Y}{Lemma 063Y} repeatedly to obtain flatness of $\beta$.
\end{proof}

\section{Application to moduli of curves}%
\label{sec:application_to_moduli_of_curves}

In this part of the lecture, we wish to define a contraction map $\ol{\mc{M}}_{g, n+1} \to \ol{\mc{M}}_{g, n}$ that deletes a marked point. This was used by Knudsen to prove that $\ol{M}_{g, n}$ is a smooth and proper stack over $\Spec \Z$. Other applications include relations between $\ol{\mc{M}}_G$ and $\ol{\mc{M}}_{g+1}$ and their Chow groups. The roadmap for this section is:
\begin{enumerate}
    \item We will discuss stable curves over an algebraically closed field $k$ and define $\ol{\mc{M}}_{g, n}$.
    \item We will define $\ol{\mc{M}}_{g, n+1} \to \ol{\mc{M}}_{g, n}$ in the wrong way.
    \item We will define contraction of rational tails and bridges over an algebraically closed field correctly.
    \item Finally, we will use \href{https://stacks.math.columbia.edu/tag/0E3X}{Lemma 0E3X} to define contraction over any scheme.
\end{enumerate}

\subsection{Stable curves}%
\label{sub:stable_curves}

We will define $n$-marked genus $g$ stable curves $C$ over $k = \ol{k}$. Here, we will take $C$ to be a connected $1$-dimensional scheme of finite type over $k$. We will work only with nodal curves, which are curves where for every $x \in C(k)$ either $x$ is smooth and $\mc{O}_{C, x}$ is a regular local ring and thus a UFD by Auslander-Buchsbaum or $x$ is a node and
\[ \wh{\mc{O}}_{C, x} \simeq k [[x,y]]/(xy). \]

\begin{rmk}
    In fact, $\wh{\mc{O}}_{X,x}$ is reduced and thus $\mc{O}_{X,x}$ is reduced, so every nodal curve is reduced.
\end{rmk}

Now let 
\[ \bigcup \wt{C}_i = \wt{C} \xrightarrow{\nu} C = \bigcup_i C_i \] 
be the normalization. We know that $\nu^{-1}(\text{node}) = \Spec k \sqcup \Spec k$. To check this, recall that the normalization of a reduced scheme is constructed by gluing the local maps
\[ \Spec (\ol{A_{\mr{red}}}^{Q(A_{\mr{red}})}) \to \Spec A_{\mr{red}}. \]
But now we know that
\[ Q(A_{\mr{red}}) = \prod_{\mf{p}_i \text{ minimal}} Q(A_{\mr{red}}/\mf{p}_i). \]
However, we know that
\[ \ol{k[[x,y]]/(xy)}^{Q(k[[x,y]]/(xy)) = k((x)) \times k((y))} = k[[x]] \times k[[y]]. \]
This implies that $\ol{\mc{O}_{C,p}}^{Q(\mc{O}_{C,p})}$ is not local because $k[[x]] \times k[[y]]$ is not local.

\begin{prop}
    The arithmetic genus is given by
    \[ g = \sum_i g(\wt{C}_i) + \#(\text{nodes}) - \#(\text{components}) + 1. \]
\end{prop}

\begin{proof}
    Recall that $\nu^{-1}(\text{node}) = \Spec k \sqcup \Spec k$. Then we have the exact sequence
    \[ 0 \to \mc{O}_C \hookrightarrow \nu_* \mc{O}_{\wt{C}} \to \bigoplus_{p \text{ node}} K_p \to 0. \]
    Taking the long exact sequence in cohomology, we obtain
    \[ 0 \to H^0(C, \mc{O}_C) \to H^0(C, \nu_* \mc{O}_{\wt{C}}) \to k^{\#(\text{nodes})} \to H^1(C, \mc{O}_C) \to H^1(C, \nu_* \mc{O}_{\wt{C}}). \]
    Because $\nu$ is finite, the Leray spectral sequence computing $H^1(\wt{C}, \mc{O}_{\wt{C}})$ degenerates at the $E_2$-page, and thus
    \[ H^p(C, \nu_* \mc{O}_{\wt{C}}) = H^0(\wt{C}, \mc{O}_{\wt{C}}). \]
    But then we know $H^1(C, \nu_* \mc{O}_{\wt{C}}) = \bigoplus H^1(\mc{O}_{\wt{C}_i}) = \sum g(\wt{C}_i)$. By connectedness of $C$, we know that $H^0(C, \mc{O}_C) = k$ and $H^0(C, \nu_* \mc{O}_{\wt{C}}) = k^{\#(\text{components})}$.
\end{proof}

We now need to add marked points.
\begin{defn}
    A \textit{$n$-marked, genus $g$ nodal curve} $C$ over $k$ is a nodal curve of genus $g$ with $n$ smooth points $x_1, \ldots, x_n \in C(k)$.
\end{defn}
This definition allows too many curves, so we want to define a notion of stability.
\begin{defn}
    A genus $g$ nodal curve $C$ with $n$ marked points is \textit{stable} if for all irreducible components $\wt{C}_i$, we have
    \[ 2g-2 + \#(\text{special points}) > 0, \]
    where a special point is a node or a marked point.
\end{defn}

\begin{exm}
    Over $\C$, consider a genus $2$ curve with two marked points $x,y$, an elliptic curve with marked point $z$, and a nodal cubic with no marked points all intersecting in a triangle. Here, we have three marked points and four nodes.\footnote{Anna commented that Morena's drawing had the curves tangent to each other, but we can just pretend that they are nodes.} By observation, this curve is stable.
\end{exm}

\begin{rmk}
    All of the $\wt{C}_i$ are stable if and only if $g(\wt{C}_i) \geq 2$, $g(\wt{C}_i) = 1$ and $\wt{C}_i$ has at least one special point, or if $g(\wt{C}_i) = 0$ and there are at least three special points.
\end{rmk}

\subsection{Contraction done wrong}%
\label{sub:contraction_done_wrong}

We will denote by $\ol{\mc{M}}_{g,n}$ be the fibered category whose objects over a scheme $S$ are given by the following data:
\begin{itemize}
    \item A map $f \colon \mc{C} \to S$ proper and flat with sections $\sigma_1, \ldots, \sigma_n \colon S \to \mc{C}$ and for any geometric point $s \in S$, the map $f_s \colon \mc{C}_s \to k(s)$ and sections $\sigma_{1,s}, \ldots, \sigma_{n,s}$ form a stable curve of genus $g$.
    \item Morphisms are given by cartesian diagrams
        \begin{equation*}
        \begin{tikzcd}
            \mc{C}' \ar{d} \ar{r} & \mc{C} \ar{d} \\
            S' \ar{r} \ar[bend left=30]{u}{\sigma_i'} & S \ar[bend right=30,swap]{u}{\sigma_i}
        \end{tikzcd}
        \end{equation*}
        that respect the sections.
\end{itemize}

Clearly the forgetful functor simply forgets the last section. Unfortunately, this does not actually respect the stability condition. The problem is when the component containing $\sigma_{n+1}(s)$ satisfies $2g_i - 2 + \#(\text{special}) = 1$, because when we delete $\sigma_{n+1}$ we lose stability. This only happens when $g(\wt{C}_i) = 1$ and there is exactly one marked point (which is actually smooth and integral, so is an elliptic curve) or when $g(\wt{C}_i) = 0$ and there are exactly three special points. In the second case, we may have $\P^1$ with three marked points, two marked points and a node (rational tail), or one node and two marked points (rational bridge). Also, we may have a nodal cubic with one marked point.

\subsection{Contraction done right}%
\label{sub:contraction_done_right}

In this part, we construct from a prestable curve $\mc{C}_{k(s)}$ a stable curve $\wt{C}_{k(s)}$. In the rational tail case, we are contracting the entire $\P^1$ to the node and in the rational bridge case we identify the two nodes and collapse the $\P^1$ to that point, leaving the two other components intersecting at a node (locally; globally this can be either one or two components). Of course, we need to check that $f_* \mc{O}_{C_{k(s)}} = \mc{O}_{\wt{C}_{k(x)}}$ and $R^1 f_* \mc{O}_{C_{k(s)}} = 0$. Next, we need to check that contraction can be extended to a neighborhood in a canonical way when $S = \Spec \mc{O}_{S,s}^h$.

\chapter{Morena (Nov 05): Contraction morphisms between moduli stacks of curves}%
\label{cha:morena_nov_05_contraction_morphisms_between_moduli_stacks_of_curves}

\section{Recap of last time}%
\label{sec:recap_of_last_time}

Recall that for $f \colon X \to Y$, if $f_* \mc{O}_X = \mc{O}_Y$ and $R^1 f_* \mc{O}_X = 0$, then we have an equivalence of categories
\[ \mr{Def}_{X \to Y} \simeq \mr{Def}_X. \]
We also attempted to construct a morphism $\ol{\mc{M}}_{g, n+1} \to \ol{\mc{M}}_{g, n}$ using the forgetful functor, but this does not work. Remember the bad cases were called the rational tail and rational bridge. Also recall that the stability condition for every component was $2 g - 2 + \#(\text{special points}) > 0$, where special points are nodes and marked points.

\section{Contraction of rational tails and bridges}%
\label{sec:contraction_of_rational_tails_and_bridges}

In the rational tail case, our stable curve should be the scheme-theoretic closure (so just topological closure) $\ol{C / C_i}$, where $C_i$ is the rational tail. Because $C$ is reduced, we see that $C$ is the pushout
\begin{equation*}
\begin{tikzcd}
    \Spec k \ar[hookrightarrow]{r} \ar[hookrightarrow]{d} & C_i \ar{d} \\
    \ol{C \setminus C_i} \ar{r} & C.
\end{tikzcd}
\end{equation*}
Of course, this gives us a map $C \to \ol{C \setminus C_i}$ by contracting $C_i$.
Now we need to check that $\ol{C \setminus C_i}$ is a prestable curve of genus $g$ which is actually stable and that if $c \colon C \to \ol{C \setminus C_i}$, then $c_* \mc{O}_C \simeq \mc{O}_{\ol{C \setminus C_i}}$ and $R^1 c_* \mc{O}_C = 0$.

To check the sheafy conditions, we have an exact sequence
\[ 0 \to \mc{O}_C \to j^* \mc{O}_{\ol{C \setminus C_i}} \oplus j_{i*} \mc{O}_{C_i} \to i_{x*} k \to 0. \]
This gives us a longer exact sequence
\[ 0 \to c_* \mc{O}_C \to c_* j_* \mc{O}_{\ol{C \setminus C_i}} \oplus c_* j_{i*} \mc{O}_{C_i} \to c_* i_{x*} k \to R^1 c_* \mc{O}_C \to R^1 c_* j_* \mc{O}_{\ol{C \setminus C_i}} \oplus R^1 c_* j_{i*} \mc{O}_{C_i}. \]
To see that $R^1 c_* \mc{O}_C = 0$, we only need to check that $R^1 c_* j_* \mc{O}_{\ol{C \setminus C_i}} = 0$ because $R^1 c_* j_{i*} \mc{O}_{C_i} = H^1(C_i, \mc{O}_{C_i}) = 0$. But here we have $c_* j_* \mc{O}_{\ol{C \setminus C_i}} = 0$ because $c_* j_* = 0$, and so we need to prove surjectivity of the direct sum onto the skyscraper sheaf. This is clear.

Considering
\begin{equation*}
\begin{tikzcd}
    \Spec k \ar{r} \ar{d} & \ol{C \setminus C_i} \ar{d} \\
    \Spec k \ar{r} & \ol{C \setminus C_i},
\end{tikzcd}
\end{equation*}
we see that $c_* \mc{O}_C \simeq \mc{O}_{\wt{C}}$ and
\[ \mc{O}_{\ol{C \setminus C_i}} = c_* j_* \mc{O}_{\ol{C \setminus C_i}} \times_{i_{x*} k} c_* j_{i*} \mc{O}_{C_i}. \]

To check stability of $\ol{C \setminus C_i}$, we simply note that stability condition of the component $C_j$ attached to $C_i$ is unchanged, and everything else was untouched, so we have a stable curve.

We now check the rational bridge case. Here we consider the pushout
\begin{equation*}
\begin{tikzcd}
    \qty{x_1} \sqcup \qty{x_2} \ar{d}\ar{r} & \mc{C} \setminus \ol{C_i} \ar{d} \\
    \qty{y}\ar{r} & \ol{C}.
\end{tikzcd}
\end{equation*}
This exists and has nice properties. What this really means is that we contract $C_i$ and introduce a new self-intersection at $x_1 = x_2$. Now the contraction morphism is given by considering the total pushout of
\begin{equation*}
\begin{tikzcd}
    \qty{x_1} \sqcup \qty{x_2} \ar{d}\ar{r} & \mc{C} \setminus \ol{C_i} \ar{d} \\
    C_i \ar{r} \ar{d} & C \ar[dashrightarrow]{d} \\
    \qty{y}\ar{r} & \ol{C}.
\end{tikzcd}
\end{equation*}
Restricting to an affine piece, where $\ol{C} = \Spec A$ and $C \setminus \ol{C_i} = \Spec A'$, we obtain a fiber product diagram of rings
\begin{equation*}
\begin{tikzcd}
    A \ar{r} \ar{d} & k \ar{d} \\
    A' \ar{r} & k \times k.
\end{tikzcd}
\end{equation*}
We can now tensor with $A_y$, and we need to show that that $\wh{A}_y = k[[x,y]]/(xy)$. Because completions are exact, we can consider the fiber product diagram
\begin{equation*}
\begin{tikzcd}
    \wh{A}_y \ar{r} \ar{d} & k \ar{d} \\
    A' \otimes_A \wh{A}_y \ar{r} & k \times k.
\end{tikzcd}
\end{equation*}
Because $A' \otimes_A \wh{A}_y = \wh{A}_{x_1}' \times \wh{A}_{x_2}' = k[[t_1]] \times k[[t_2]]$, we see that
\[ \wh{A}_y = \qty{\ell + t_1 p(t_1) + t_2 p(t_2)} = k[[t_1, t_2]] / (t_1t_2), \]
and thus $y$ is a node.

\section{Contraction over any base scheme}%
\label{sec:contraction_over_any_base_scheme}

\begin{thm}
    Let $(S, f \colon \mc{C} \to S, \sigma_1, \ldots, \sigma_{n+1}) \in \ol{\mc{M}}_{g, n+1}(S)$. Then there exists a contraction such that
    \[ f = \mc{C} \xrightarrow{c} \ol{\mc{C}} \xrightarrow{g} S \]
    and the following conditions hold:
    \begin{enumerate}
        \item $(S, g \colon \ol{\mc{C}} \to S, c \circ \sigma_1, \ldots, c \circ \sigma_n) \in \ol{\mc{M}}_{g, n}(S)$.
        \item $c_* \mc{O}_{\mc{C}} \simeq \mc{O}_{\ol{\mc{C}}}$ and $R^1 c_* \mc{O}_{\mc{C}} = 0$, and this is stable under base change. In addition, for all geometric points $\ol{s} \to S$, $c_{\ol{s}}$ is either an isomorphism or a contraction of a rational tail or rational bridge.
    \end{enumerate}
    Moreover, $c \colon \mc{C} \to \ol{\mc{C}}$ is unique up to unique isomorphism.
\end{thm}

\begin{cor}
    The morphism $\ol{\mc{M}}_{g, n+1} \to \ol{\mc{M}}_{g, n}$ is defined over $\Z$.
\end{cor}

\begin{proof}
    We know what happens to objects. We know that morphisms are cartesian diagrams
    \begin{equation*}
    \begin{tikzcd}
        S' \ar{r} \ar{d}{\sigma_i'} & S \ar{d}{\sigma_i} \\
        \mc{C}' \ar{r}{b} \ar{d} & \mc{C} \ar{d} \\
        S' \ar{r}{a} & S.
    \end{tikzcd}
    \end{equation*}
    But now we have two candidate contractions of $\mc{C}'$, namely $\ol{\mc{C}'}$ and $\ol{\mc{C}} \times_S S'$. But these have a unique isomorphism. These fit into the diagram
    \begin{equation*}
    \begin{tikzcd}
        & \mc{C} \times_S S' \ar{rr} \ar{dl} \ar{dr} \ar{dd} & & \mc{C} \ar{dr} \ar{dd} \\
        \ol{\mc{C}'} \ar{dr} \ar[dashrightarrow]{rr} & & \ol{\mc{C}} \times_S S' \ar{dl} \ar{rr} & & \ol{\mc{C}} \ar{dl} \\
        & S' \ar{rr} & & S.
    \end{tikzcd}
    \end{equation*}
    This gives us compatibility with morphisms.
\end{proof}

To prove the theorem, we want to work \'etale locally over $S$, with a cover $(S_i \to S)$. Then there exists a factorization $\mc{C}_i \xrightarrow{a} \ol{\mc{C}}_i \to S_i$, and then we need to check that the cocycle condition works on the overlaps. The existence of a global factorization means that our data is effective, and this is the same as $\ol{\mc{M}}_{g, n}$ is a stack.

To prove existence of a factorization \'etale locally, over $\ol{s} \in S_i$, we have a contraction $c_s \colon C_s \to \ol{C}_s$. We want to prove that there exists an \'etale neighborhood $(U, u) \to (S, s)$ such that there exists a factorization $C_U \to \ol{C}_U$ extending $c_s$. In order to do this, we need to do some reduction. If $S = \Spec A$ is affine is of finite type over $\Z$, we want $A_i \subseteq A$ such that $A = \varinjlim A_i$, and thus $S = \Spec A = \varprojlim \Spec A_i$.

\begin{thm}[\href{https://stacks.math.columbia.edu/tag/0E6U}{0E6U}, \href{https://stacks.math.columbia.edu/tag/0E6V}{0E6V}, \href{https://stacks.math.columbia.edu/tag/0DSS}{0DSS}, \href{https://stacks.math.columbia.edu/tag/0CMV}{0CMV}]
    The fibered category $\ms{Curves}_g^{\text{prestable}}$ is limit-preserving.
\end{thm}

\begin{lem}
    The category of schemes of finite presentation over $S$ is the colimit of the categories of schemes of finite presentation over $S_i$.
\end{lem}

Now if $S = \Spec A$, then $\mc{O}_{S, s}^h = \operatorname{colim}_{(U, u) \to (S, s)} \mc{O}_U(U)$. Thus we may consider $S = \Spec (\mc{O}_{S, s}^h)$. Write $\Lambda$ for this henselization. Then if we consider $\Lambda \to \Lambda / \mf{m}^n \Lambda$, we have fiber product diagrams
\begin{equation*}
\begin{tikzcd}
    C_s \ar{r} \ar{d} & C_u \ar{d} \ar{r} & C \ar{d} \\
    \Spec k \ar{r} & \Spec \Lambda / \mf{m}^n \Lambda \ar{r} & S.
\end{tikzcd}
\end{equation*}
This gives us a formal object of $\mr{Def}_{C_s}(\wh{\Lambda})$. But now we know that
\[ \mr{Def}_{C_s \to \ol{C}_s} \simeq \mr{Def}_{C_s}, \]
and thus we actually have deformations of $C_s \to \ol{C}_s$ over $\Lambda / \mf{m}^n \Lambda$. But this gives us a formal element of
\[ \mr{Def}_{C_s \to \ol{C}_s} (\wh{\Lambda}). \]
Now using~\href{https://stacks.math.columbia.edu/tag/01W0}{01W0}, we have $\ol{\mc{C}}_n \in \ms{Curve}_g$.

\begin{thm}
    The data $\ms{Curve}_g$ is effective.
\end{thm}

For a sketch of this, note that the first stability condition is an open condition. If $p_1, \ldots, p_n$ are the marked points, then write $D = p_1 + \cdots + p_n$. Stability is equivalent to $\omega_C(D)$ being ample, and so over $C_s$ we have an ample line bundle $\omega_{C_s}(D)$. We can lift this to $(\ol{\mc{C}}_n, \mc{L}_n)$ because obstructions live in
\[ H^2(\ol{{\mc{C}}}_{n-1}, (1 + \mf{m})^n \mc{O}_{\ol{C}_n^*}) = 0. \]
Using the Grothendieck algebraization theorem, we see that if $X_i \to S_i$ is proper and $\mc{L}_i$ is ample, there exists a proper morphism $X \to S$ and $\mc{L}$ an ample line bundle such that base change to $S_n$ recovers $(X_n, \mc{L}_n)$.

Here, we know that $\ol{\mc{C}} \to \Spec \wh{\Lambda}$ is finite type and separated while $\mc{C} \to \Spec \wh{\Lambda}$ is proper, and so there exists a unique $\mc{C} \to \ol{\mc{C}}$. We now need to return back to $\Spec \Lambda$. But now we know that $\wh{\Lambda}$ is the direct limit of its finitely-generated (over $\Lambda$) subalgebras. This gives us $\Lambda \subseteq \Lambda^1 \subseteq \wh{\Lambda}$. We know that 
\[ \Lambda = \mc{O}_{S,s}^h \to \wh{\mc{O}}_{S,s} \]
is regular, it is flat, and thus $\wh{\Lambda}/\mf{m}\wh{\Lambda}$ is noetherian and geometrically regular over $k(\mf{m})$ by Popescu. We have the diagram
\begin{equation*}
\begin{tikzcd}
    \Lambda_{\mr{sm}} \ar{r} \ar{dr} & \Lambda_{\text{\'et}} \ar{d} \\
    \Lambda \ar{u}{u} \ar{r} \ar[dashrightarrow]{ur} & \Lambda / \mf{m} \Lambda.
\end{tikzcd}
\end{equation*}
Because $\Lambda$ is henselian, we have a section $\Lambda_{\text{\'et}} \to \Lambda$, and thus we have a base change to $\Lambda$.

\chapter{Baiqing (Nov 12): Deformations of group schemes}%
\label{cha:baiqing_nov_12_deformations_of_group_schemes}

\section{Deformations of abelian schemes}%
\label{sec:deformations_of_abelian_schemes}

\begin{defn}
    An \textit{abelian scheme} $p \colon X \to S$ is a proper smooth group scheme over a connected scheme $S$ with geometrically connected fibers.
\end{defn}

If $S = \Spec k$ is a field, then $X$ is projective. However, this is not true in general.

\begin{defn}
    The deformation functor $\mr{Def}_X^{\mr{AS}} \colon \Lambda_W \to \ms{Set}$ is defined for an abelian variety $X$ over $k$ and $\Lambda_W$ the category of local Artinian $W$-algebras with residue field $k$. It is given by
    \[ A \mapsto \qty{(X', \varphi) \mid X' \to \Spec A \text{ abelian scheme}, \varphi \colon X'_k \simeq X} / ~ \]
    where $(X_1', \varphi_1') \sim (X_2', \varphi_2')$ if there exists an isomorphism $\psi \colon X_1' \to X_2'$ of abelian schemes such that $\varphi_2 \circ \psi_k = \varphi_1$.
\end{defn}

\begin{thm}
    Let $X$ be an abelian variety of dimension $g$ over $k$. Then the deformation functor $\mr{Def}_X^{\mr{AS}}$ is pro-representable by $W(k)[[t_1, t_2, \ldots, t_{g^2}]]$.
\end{thm}

Recall Schlessinger's criterion from Johan's lecture.\footnote{Baiqing wrote them all down on the board, but I am too lazy to type them yet again.} Note that if $F$ is formally smooth and $F(R') \to F(R)$ is surjective for all $R' \twoheadrightarrow R$, then $F$ is pro-representable by a power series ring $W[[t_1, \ldots, t_d]]$, where $d = \dim_k F(k[\ep])$.

We will prove the conditions $(H_3), (H_4)$ and then prove that the deformation functor is formally smooth. Before this, we will consider the geometry of abelian schemes.

\begin{lem}[Rigidity lemma]
    Given a diagram
    \begin{equation*}
    \begin{tikzcd}
        X \ar{rr}{f} \ar{dr}{p} & & Y \ar{dl}{q} \\
        & S,
    \end{tikzcd}
    \end{equation*}
    suppose that $S$ is connected, $p$ is flat and proper, and $H^0(X_s, \mc{O}_{X_s}) \simeq k(s)$ for all $s \in S$. If for some point $s \in S$, $f(X_s)$ is set-theoretically a single point, then there exists a section $\eta \colon S \to Y$ such that $f = \eta \circ p$.
\end{lem}

\begin{cor}
    Let $X, Y$ be abelian schemes over $S$ and $f \colon X \to Y$. If $f \circ \ep_X = \ep_Y$, then $f$ is a homomorphism. Here, $\ep_X$ refers to the identity section.
\end{cor}

This result holds in general if we replace $Y$ be an arbitrary group scheme.

\begin{proof}[Sketch of proof]
    Consider the diagram
    \begin{equation*}
    \begin{tikzcd}
        X \times_S X \ar{rr} \ar{dr} && G_X \ar{dl} \\
        & X.
    \end{tikzcd}
    \end{equation*}
    Consider the morphisms $X \times_S X \to G \times_S X$ given by $\psi_1(x_1, x_2) = (f(x_1x_2), x_2)$ and $\psi_2(x_1, x_2) = (f(x_1), x_2)$. Now we argue that $f(g_1 g_2) = f(g_1) f(g_2)$.
\end{proof}

Also, it is clear that abelian schemes are commutative because inversion takes the identity to itself and thus is a homomorphism. If $\phi_1, \phi_2 \colon X \to Y$ are two homomorphisms and $(\phi_1)_s = (\phi_2)_2$ for some $s \in S$, then $\phi_1 = \phi_2$. Next, if $f \colon X \to Y$ is a morphism of schemes between abelian schemes, then $f - f(0) = f - f \circ \ep \circ p$ is a homomorphism.

We will now prove $(H_4)$ for $\mr{Def}_X^{\mr{AS}}$. For a small extension $R'' \to R$, we show that
\[ \Def_X^{\mr{AS}} (R'' \times_R R') \to \Def_X^{\mr{AS}}(R'') \times_{\Def_X^{\mr{AS}}(R)} \Def_X^{\mr{AS}}(R') \]
is a bijection. This is equivalent to proving that in the diagram
\begin{equation*}
\begin{tikzcd}
    \Def_X^{\mr{AS}}(R'' \times_R R') \ar{r} \ar{d}{\pi'} & \Def_X^{\mr{AS}}(R'') \ar{d}{\pi} \\
    \Def_X^{\mr{AS}}(R') \ar{r} & \Def_X^{\mr{AS}}(R),
\end{tikzcd}
\end{equation*}
if $(\mc{X}', \varphi') \mapsto (\mc{X}, \varphi)$ along the bottom arrow, then $\pi'^{-1}((\mc{X}', \varphi')) = \pi^{-1}((X, \varphi))$.

\begin{prop}
    $\pi^{-1}((\mc{X}, \varphi))$ is the set of isomorphism classes of $(\mc{X}'', \varphi_R'')$ such that $\mc{X}''$ is flat over $R''$ and $\mc{X}_R'' \to \mc{X}$ is an isomorphism as schemes over $R$.
\end{prop}

Choose $(\mc{X}'', \varphi'') \in \pi^{-1}((\mc{X}, \varphi))$. Then there exists an isomorphism $\phi \colon \mc{X}_R'' \to \mc{X}$ of abelian schemes. Now we will send $(\mc{X}'', \varphi'') \mapsto (\mc{X}'', \phi)$. Checking uniqueness is easy with the diagram
\begin{equation*}
\begin{tikzcd}
    \mc{X}_k'' \ar{rr}{\phi_k} \ar{dr}{\varphi''} && \mc{X}_k \ar{dl}{\varphi} \\
    & X
\end{tikzcd}
\end{equation*}
Now we check that if $(\mc{X}_1'', \varphi_1'') \sim (\mc{X}_2'', \varphi_2'')$, then they are sent to the same thing. This was erased from the board by Baiqing before I could process it.

    Now we prove injectivity. If $(\mc{X}_1'', \varphi_1''), (\mc{X}_2'', \varphi_2'')$ map to equivalent $(\mc{X}_1'', \phi_1), (\mc{X}_2'', \phi_2)$, then there exists $\psi \colon \mc{X}_1'' \simeq \mc{X}_2''$ such that $\phi_2 \circ \psi_R = \phi_1$. This implies that $\varphi_2'' \circ \psi_k = \varphi_1''$. Unfortunately, $\psi$ is not an isomorphism of abelian schemes, so we can replace it by $\wt{\psi} = \psi - \psi(0)$.

Finally, we prove surjectivity, which is the difficult part of the argument.
\begin{prop}
    Let $S = \Spec A$, where $A$ is an Artinian local ring, $\mf{m} \subset A$ be the maximal ideal, and $I \subset A$ such that $\mf{m} \cdot I = 0$. Let $\pi \colon X \to S$ be proper and smooth, $\ep \colon S \to X$ be a section of $\pi$, $S_0 = \Spec A/I$, and $X_0 = X \times_S S_0$. Then if $X_0$ is an abelian scheme with identity $\ep |_{S_0}$, then $X$ is an abelian scheme with identity $\ep$.
\end{prop}

\begin{thm}
    Let $X, Y/R$ are smooth schemes. Suppose that $\pi \colon R' \twoheadrightarrow R$ is a small extension. Let $X' \in \mr{Def}_X(R')$ and $Y' \in \mr{Def}_Y(R')$. For any $f \colon X \to Y$ over $R$, there is a canonically associated class $o(f) \in H^1(X_k, f_k^* T_{Y_k/k}) \otimes_k \ker \pi$. If $o(f) = 0$, then $H^0(X_k, f_k^* T_{Y_k/k}) \otimes_k \ker \pi$.
\end{thm}

\begin{proof}[Proof of proposition]
    Given $(m, \ep, \iota)$, we want $(\mu, \ep)$, where $\mu(x,y) = x-y$. We have $\mu$ on $X_0$, and so we want to deform this to $\mu'$ on $X$. We know that
    \[ o(\mu) \in H^1(X_k \times_k X_k, \mu_k^* T_{X_k/k}) \otimes_k \ker \pi, \]
    but because $X_k$ is an abelian variety, we have
    \[ T_{X_k/k} = H^0(X_k, \Omega_{X_k/k})^* \otimes_k \mc{O}_{X_k}. \]
    We want to show that $o(\mu) = 0$. Consider $g_1 = \Delta \colon X_0 \to X_0 \times S_0 X_0$ and $g_2 = (\mr{id}, \ep p) \colon X_0 \to X_0 \times_{S_0} X_0$. Then $\mu \circ g_1 = \ep_0 \circ \pi$, and thus $o(\mu \circ g_1) = 0$. Also, $\mu \circ g_2 = 1_{X_0}$ and $o(\mu \circ g_2) = 0$. We know that
    \[ o(\mu \circ g_1) = (g_1)_k^* o(\mu) = 0, \qquad o(\mu \circ g_2) = (g_2)_k^* o(\mu) = 0. \]
    Consider the morphisms $(g_1)_k^* \colon H^1(X_k \times X_k, \mu_k^* T_{X/k}) \to H^1(X_k, (\ep \circ \pi)^* T_{X/k})$ and $(g_2)_k^* \colon H^2(X_k \times X_k, \mu_k^*(T_{X/k})) \to H^1(X_k, (\ep \circ \pi)^* T_{X/k})$. But now we see that they are given by $(g_1)_k^* \colon (x, y) \otimes v \mapsto (x+y) \otimes v$ and $(x, y) \otimes v \mapsto x \otimes v$, and so $o(\mu) = 0$.

    Now consider $H^0(X_k \times_k X_k, \mu_k^* T_{X_k/k}) \otimes_k \ker \pi \cong \mf{t} \otimes_k I$ and let $\mu'$ be a deformation of $\mu$. Then
    \[ S \xrightarrow{(\ep, \ep)} X \times_S X \xrightarrow{\mu'} X \]
    is a deformation of $\ep_0$. Under the identification
    \[ H^0(\Spec k, (\ep_0)_k^* T_{X_k/k}) \otimes_k \ker \pi \simeq \mf{t} \otimes_k I \]
    and using what we have done previously, we have $\mu' \circ (\ep, \ep) = \ep$ and therefore an abelian scheme structure.
\end{proof}

Finally we prove that $\Def_X^{\mr{AS}}(R') \to \ms{Def}_X^{\mr{AS}}(R)$ is surjective for $R' \twoheadrightarrow R$. Let $(\mc{X}, \varphi) \in \Def_X^{\mr{AS}}(R)$. Then $o(\mc{X}) \in H^2(X, T_{X/k}) \otimes_k \ker \pi = \mf{t} \otimes_k (\mf{t}^* \wedge \mf{t}^*)$. We want to prove that $\pi^{-1}((\mc{X}, \varphi))$ is nonempty. We know that $\iota^*$ induces $-1$ on $\mf{t}, \mf{t}^*$, and so $o(\mc{X}) = - o(\mc{X})$, and if we are working not in characteristic $2$, $o(\mc{X}) = 0$.

To compute the dimension of the tangent space, we want to compute $\mr{Def}_X^{\mr{AS}}(k[\ep])$, but this is $H^1(X, T_{X/k}) \simeq \mf{t} \otimes \mf{t}^*$, which has dimension $g^2$.

\section{Deformations of smooth affine group schemes}%
\label{sec:deformations_of_smooth_affine_group_schemes}

We will now discuss deformations of $\G_m$ and $\G_a$. Let $G/k$ be an affine smooth algebraic group scheme. We will consider the deformation problem
\[ \Def_G(R) = \qty{(G', \phi) \mid G'/R \text{ group scheme}, G_s \xrightarrow{\phi}{\sim} G} / \mr{iso}. \]
For example, for $\G_m$, we know that $\Spec k[\ep][t, t^{-1}]$ is a deformation of $\G_m$. We want to deform the multiplication, and so we have $m' \colon T \to T_1 T_2 (1 + \ep \Delta(T_1, T_2))$. To check associativity, we obtain 
\[ \Delta(T_1, T_2) + \Delta(T_1 T_2, T_2) = \Delta(T_1, T_2 T_3) + \Delta(T_2, T_3). \]
We will also consider $\Delta, \Delta'$ equivalent if there exists $f$ such that $\Delta'(T_1, T_2) = \Delta(T_1, T_2) + f(T_1 T_2) - f(T_1) - f(T_2)$. Therefore we have
\[ \Def_{\G_m}(k[\ep]) \cong \frac{\qty{\Delta \mid \text{associative}}}{\qty{f(T_1T_2) - f(T_1) - f(T_2)}}. \]

\begin{prop}
    In fact, we have $\Def_{\G_m}(k[\ep]) = 0$.
\end{prop}

\begin{proof}
    let $f(T) = \sum a_i T^i$. Then $f(T_1 T_2) - f(T_1) - f(T_2) = \sum a_i (T_1^i T_2^i - T_1^i - T_2^i)$. If we set
    \[ \Delta(T_1, T_2) = \lambda_{ij} T_1^i T_2^j, \]
    adjust $\Delta$ by $f$ such that $\lambda_{i0} = 0$. Now we have the equation
    \[ \sum \lambda_{ij} T_1^i T_2^j = \sum_{i,j} T_1^i T_2^i T_3^j - \qty(\sum \lambda_{ij} T_1^i T_2^j T_3^j + \sum \lambda_{ij} T_2^i T_3^j), \]
    and thus $\lambda_{ij} = 0$.
\end{proof}

For the additive group, our deformed multiplication is given by $T \to T_1 + T_2 + \ep \Delta(T_1 T_2)$. Associativity is equivalent to the condition
\[ \Delta(T_1, T_2) + \Delta(T_1 + T_2, T_3) = \Delta(T_2, T_3) + \Delta(T_1, T_2 + T_3). \]
The trivial deformations are given by $f(T_1 + T_2) - f(T_1) - f(T_2)$, and thus we have
\[ \Def_{\G_a}(k[\ep]) \simeq \frac{\Delta \mid \text{associative}}{\qty{f(T_1 + T_2) - f(T_1) - f(T_2)}}. \]
This deformation space vanishes in characteristic $0$ and is infinite-dimensional in positive characteristic. If we apply $\pdv{T_3}$ to the associativity relation, we obtain
\[ \Delta_2(T_1 + T_2, T_3) = \Delta_2(T_2, T_3) + \Delta_2(T_1, T_2 + T_3), \]
and if we apply $\pdv{T_1}$, we obtain
\[ \Delta_{12}(T_1 + T_2, T_3) = \Delta_{12}(T_1, T_2 + T_3), \]
and therefore $\Delta_{12}(T_1, T_2) = f(T_1 + T_2)$. In characteristic $0$, $f$ has a primitive, and so taking the primitive twice we obtain a desired $\wt{F}$. Primitives do not exist in positive characteristic, so this proof does not work. However, it is possible to work out the following:

\begin{prop}
    In positive characteristic $p$, $\Delta$ is a linear combination of $B_p(T_1, T_2)^{p^n}$ for $n \geq 0$ and $T_1^{p^n} T_2^{p^m}$ for $m \geq n+1$, where $B_p(T_1, T_2) = \frac{(T_1 + T_2)^p - T_1^p - T_2^p}{p} \in \Z[T_1, T_2]$.
\end{prop}

\chapter{Haodong (Nov 19): Artin's axioms}%
\label{cha:haodong_nov_19_artin_s_axioms}

\section{Prestacks and stacks}%
\label{sec:prestacks_and_stacks}

First, we will generalize the Zariski topology a little bit.
\begin{defn}
    Let $S$ be a category. Then a \textit{Grothendieck topology} on $S$ is given by a set $\on{Cov}(X)$ for each $x \in X$, where each element in $\on{Cov}(X)$ is a collection of morphisms $\qty{x_i \to x}_I$ in $S$. We require that
    \begin{enumerate}
        \item All isomorphisms $x' \xrightarrow{\sim} x$ are in $\on{Cov}(x)$.
        \item If $\qty{x_i \to x}_I \in \on{Cov}(X)$ and $y \to x$, then $x_i \times_x y$ exist and $\qty{x_i \times_x y \to y}_I \in \on{Cov}(y)$.
        \item If $\qty{x_i \to x}_I$ is a covering of $X$ and $\qty{x_{ij} \to x_i}_{J_i}$ is a covering of $x_i$ for all $i$, then $\qty{x_{ij} \to x}$ is a covering of $X$.
    \end{enumerate}
    A \textit{site} is a category with a Grothendieck topology.
\end{defn}

\begin{exm}
    Let $S$ be a scheme and let $\mc{C} = \ms{Sch}/S$. For every $T \to S$, we say that $\qty{f_i \colon T_i \to T}$ is a covering if all $f_i$ are open immersions and the $f_i$ are jointly surjective. This gives the big Zariski site of $S$. If we replace open immersion with \'etale morphism, then we obtain the big \'etale site.
\end{exm}

\subsection{Prestacks}%
\label{sub:prestacks}

\begin{defn}
    Let $\mc{S}$ be a category and $p \colon \mc{X} \to \mc{S}$ be a functor. We will denote objects of $\mc{X}$ by $a,b,\ldots$ and objects in $\mc{S}$ by $S,T,\ldots$. We will denote morphisms in $\mc{X}, \mc{S}$ by $\alpha, f$ respectively. We say that $p$ is a \textit{prestack} if the following conditions are satisfied:
    \begin{enumerate}
        \item If $p(b) = T$ and $S \to T$ is a morphism, there exists $a$ and a morphism $a \to b$ such that $p(a) = S$ filling the diagram
            \begin{equation*}
            \begin{tikzcd}
                a \ar{r} \ar[mapsto]{d} & b \ar[mapsto]{d} \\
                S \ar{r}{f} & T.
            \end{tikzcd}
            \end{equation*}
        \item If we have
            \begin{equation*}
            \begin{tikzcd}
                a \ar[mapsto]{d} \ar[bend left=30]{rr} & b \ar[mapsto]{d} \ar{r} & c \ar[mapsto]{d} \\
                R \ar{r} & S \ar{r} & T,
            \end{tikzcd}
            \end{equation*}
            there exists a unique morphism $a \to b$ filling in the above arrow.
    \end{enumerate}
\end{defn}

\begin{exer}
    For every $s \in \mc{S}$, the category $\mc{X}(s)$ over $s$ has morphisms those lying over the identity and is thus a groupoid.
\end{exer}

\begin{exm}
    Let $F \colon \mc{S} \to \ms{Set}$ be a functor. We will define $\mc{X}_F$ to have objects $(a,s)$, where $s \in \mc{S}$ and $a \in F(s)$. Then morphisms $(a,s) \to (a', s')$ are morphisms $f \colon s \to s'$ such that $F(f)(a') = a$. We will define $p \colon (a,s) \mapsto s$, and this defines a prestack, which we will simply call $F$.
\end{exm}

\begin{exm}
    Let $S$ be a scheme and $\mc{C} = \ms{Sch}/S$. If $T \to S$ is an $S$-scheme, then $\Hom(-, T)$ is a presheaf on $\mc{C}$ and by the previous example defines a prestack over $\mc{C}$, which we will call $T$.
\end{exm}

\begin{exm}
    Consider the functor $\mc{M}_g \to \ms{Sch}/\C$ where the objects are morphisms $\mc{C} \to S$, where $S$ is a scheme over $\C$ and $\mc{C} \to S$ is smooth and proper with all geometric fibers connected curves of genus $g$. Morphisms are simply pairs $\alpha \colon \mc{C} \to \mc{C}', f \colon S \to S'$ such that $\mc{C} = \mc{C}' \otimes_{S'} S$.
\end{exm}

\begin{defns}
    Let $\mc{S}$ be a site.
    \begin{enumerate}
        \item A \textit{morphism of prestacks} $\mc{X} \to \mc{Y}$ is a functor $f$ such that for all $a \in \mc{X}$, $p_{\mc{X}}(a) = p_{\mc{Y}}(f(a))$. 
        \item If $f, g \colon \mc{X} \to \mc{Y}$ are morphisms of prestacks, then a \textit{$2$-morphism} $\alpha \colon f \to g$ is a natural transformation such that for all $a \in \mc{X}$, $\alpha_a \colon f(a) \to g(a)$ lies over the identity in $\mc{S}$. In particular, $\alpha$ is an isomorphism.
        \item A diagram
            \begin{equation*}
            \begin{tikzcd}
                \mc{X} \ar{r}{f'} \ar{d}{g'} & \mc{Y} \ar{d}{g} \\
                \mc{Y}' \ar{r}{f} & \mc{Z}
            \end{tikzcd}
            \end{equation*}
            is a \textit{$2$-commutative diagram} if there exists a $2$-morphism $\alpha \colon fg' \to gf'$.
        \item $f \colon \mc{X} \to \mc{Y}$ is called an \textit{equivalence} if there exists $g \colon \mc{Y} \to \mc{X}$ such that $f \circ g \xrightarrow{\sim} \mr{id}$ and $g \circ f \xrightarrow{\sim} \mr{id}$.
    \end{enumerate}
\end{defns}

\begin{lem}[2-Yoneda lemma]
    Let $\mc{X} \to \mc{S}$ be a prestack and $s \in \mc{S}$. Then $\Hom_{\mc{S}}(-, s)$ is a prestack on $\mc{S}$ and the functor $\Hom(s, \mc{X}) \to \mc{X}(s)$ given by $f \mapsto f_s(\mr{id}_s)$ is an equivalence of categories.
\end{lem}

Now we want a fiber product of prestacks for $\mc{X} \to \mc{Y}, \mc{Y}' \to \mc{Y}$, which is simply the final object in all $2$-commutative diagrams
\begin{equation*}
\begin{tikzcd}
    \mc{Z} \ar{r} \ar{d} & \mc{X} \ar{d} \\
    \mc{Y}' \ar{r} & \mr{Y}.
\end{tikzcd}
\end{equation*}
Fortuantely, these do exist.

\begin{exm}
    The product $\mc{X} \times \mc{X}$ exists and is a prestack. We also have a diagonal $\Delta \colon \mc{X} \to \mc{X} \times \mc{X}$.
\end{exm}

\begin{exm}
    Let $\mc{X}$ be a prestack and $a \colon \mc{Y} \to \mc{X}, b \colon \mc{Y}' \to \mc{X}$ be morphisms. Then there exists a $2$-cartesian diagram
    \begin{equation*}
    \begin{tikzcd}
        \mc{Y} \times_{\mc{X}} \mc{Y}' \ar{r} \ar{d} & \mc{Y} \times \mc{Y}' \ar{d}{(a,b)} \\
        \mc{X} \ar{r}{\Delta} & \mc{X} \times \mc{X}.
    \end{tikzcd}
    \end{equation*}
\end{exm}

\subsection{Stacks}%
\label{sub:stacks}

\begin{defn}
    A prestack $\mc{X}$ over a site $\mc{S}$ is called a \textit{stack} if for every $\qty{U_i \to U} \in \on{Cov}(U)$, we have
    \begin{enumerate}
        \item (morphisms glue) There exists a unique $a \to b$ filling in the diagram
            \begin{equation*}
            \begin{tikzcd}
                & a|_{U_i} \ar[bend left=20]{drr} \ar{dr} \\
                a|_{U_{ij}} \ar{ur} \ar{dr} & & a \ar[dashrightarrow]{r} & b \\
                & a|_{U_j} \ar[bend right=20]{urr} \ar{ur}
            \end{tikzcd}
            \end{equation*}
            lying over
            \begin{equation*}
            \begin{tikzcd}
                & {U_i} \ar{dr} \\
                {U_{ij}} \ar{ur} \ar{dr} & & U \\
                & {U_j}. \ar{ur}
            \end{tikzcd}
            \end{equation*}
            Precisely, this means that given $a, b \in \mc{X}(U)$ and $\phi_i \colon a |_{U_i} \to b$ such that $\phi_i |_{U_{ij}} = \phi_j |_{U_{ij}}$, then there exists a unique $\phi \colon a \to b$ such that $\phi |_{U_i} = \phi_i$ for all $i$.
        \item (objects glue) Given $a_i, a_j$ and isomorphisms $\alpha_{ij} \colon a_i |_{U_{ij}} \to a_j |_{U_{ij}}$ satisfying the cocycle condition on $U_{ij}$, there exists $a \in \mc{X}(U)$ and isomorphisms $\phi_i \colon a|_{U_i} \to a_i$ such that $\alpha_{ij} \circ \phi_i |_{U_{ij}} = \phi_j |_{U_{ij}}$.
    \end{enumerate}
\end{defn}

\begin{exm}
    Consider the prestack $\ms{Sheaves}$ over $\ms{Sch}$ with objects $(S, F)$, where $F$ is a sheaf on $S$. Then $(S,F) \to (S', F')$ is a pair $f \colon S \to S'$ and $\alpha \colon F' \to f_* F$ such that the adjoint of $\alpha$ is an isomorphism $F \simeq f^{-1} F'$. We know that sheaves and their morphisms can be glued in the Zariski topology, so the prestack $\ms{Sheaves}$ is a stack over $\ms{Sch}_{\mr{Zar}}$.
\end{exm}

\begin{exm}
    Consider the prestack $\ms{Schemes}$ over $\ms{Sch}$ with objects $(T \to S)$ and morphisms $(T \to S) \to (T' \to S')$ is a pair $f \colon T \to T'$ and $g \colon S \to S'$ such that the two compositions $T \to S'$ agree. Of course schemes can be glued in the Zariski topology, so $\ms{Sch}$ is a stack in $\ms{Sch}_{\mr{Zar}}$.
\end{exm}

\begin{prop}
    $\mc{M}_g$ is a stack over $\ms{Sch}_{\text{\'et}}$ for $g \geq 2$.
\end{prop}

\begin{rmk}
    The stackiness conditions mainly come from descent theory.
\end{rmk}

\section{Algebraic stacks and Artin's axioms}%
\label{sec:algebraic_stacks_and_artin_s_axioms}

From now on, we will work in the category $\ms{Sch}/S$ of schemes over $S$.

\subsection{Algebraic stacks}%
\label{sub:algebraic_stacks}

\begin{defn}
    A morphism $\mc{X} \to \mc{Y}$ of prestacks is \textit{representable by schemes} if for all schemes $T \to \mc{Y}$, the fiber product $\mc{X} \times_{\mc{Y}} T$ is also a scheme. A representable $\mc{X} \to \mc{Y}$ is surjective, smooth, etc if for all schemes $T \to \mc{Y}$, the morphism $\mc{X} \times_{\mc{Y}} T \to T$ is surjective, smooth, etc.
\end{defn}

\begin{defn}
    An \textit{algebraic space} is a sheaf $F$ on $(\ms{Sch}/S)_{\text{\'et}}$ such that there exists a scheme $U$ and a surjective \'etale $U \to F$ which is representable by schemes.
\end{defn}

\begin{defn}
    A morphism $\mc{X} \to \mc{Y}$ is called \textit{representable} if for all schemes $T \to \mc{Y}$, $\mc{X} \times_{\mc{Y}} T$ is an algebraic space. Moreover, a representable $f \colon X \to \mc{Y}$ is called surjective, smooth, etc if $U \twoheadrightarrow \mc{X} \times_Y T \to T$ is surjective, smooth, etc. 
\end{defn}

\begin{defn}
    A \textit{algebraic stack} over $(\ms{Sch}/S)_{\text{\'et}}$ is a stack such that there exists a scheme $U$ with a morphism $U \to \mc{X}$ that is smooth, surjective, and representable.
\end{defn}

This is equivalent to the following conditions (taken together):
\begin{enumerate}
    \item The diagonal $\mc{X} \xrightarrow{\Delta} \mc{X} \times \mc{X}$ is representable.
    \item There exists a scheme $U$ and a smooth surjective morphism $U \to \mc{X}$.
\end{enumerate}

A useful fact that the product and fiber products of algebraic stacks exist (in the category of algebraic stacks), and this is the same as their fiber products as prestacks.

\begin{defn}
    We say that $f \colon \mc{X} \to \mc{Y}$ of algebraic stacks is \textit{locally of finite type} if for all (for some) smooth presentations $V \to \mc{Y}$ and $U \to \mc{X} \times_{\mc{Y}} V$, the composition
    \[ U \to \mc{X} \times_Y V \to V \]
    is locally of finite type.
\end{defn}

\section{Artin's Axioms}%
\label{sec:artin_s_axioms}

\begin{defn}
    Let $\mc{X}$ be a stack over $(\ms{Sch}/S)_{\text{\'et}}$. Then $\mc{X}$ is \textit{limit-preserving} if
    \[ \varinjlim \mc{X}(\Spec B_i) \to \mc{X}(\Spec(\varinjlim B_i)) \]
    is an equivalence of categories. Explicitly, this means:
    \begin{enumerate}
        \item Every object on the right hand side comes from $a_i |_{\Spec B}$ for some $i$ and some $a_i \in \mc{X}(\Spec B_i)$.
        \item For $a, b \in \mc{X}(\Spec B_i)$, we have
            \[ \Hom_{\mr{RHS}}(a|_{\Spec B}, b|_{\Spec B}) = \varinjlim_{i' \geq i} \Hom_{\mc{X}(\Spec B_{i'})} (a|_{B_{i'}}, b|_{i'}). \]
    \end{enumerate}
\end{defn}

This should be viewed as a finiteness condition because if $f \colon X \to S$ is a scheme, then $\Hom(-, X)$ is limit-preserving if and only if $f$ is locally of finite presentation.

\begin{defns}
    Let $\mc{X}$ be a prestack over $\ms{Sch}/k$. A \textit{formal object} of $\mc{X}$ is $(R, \qty{\xi_n}, \qty{f_n})$, where $R$ is a complete local Noetherian $k$-algebra, $\xi_n \colon \Spec R/\mf{m}_R^n \to \mc{X}$, and $f_n \colon \xi_n \to \xi_{n+1}$ are morphisms in $\mc{X}$ lying over $\Spec R/\mf{m}_k^n \hookrightarrow \Spec R/\mf{m}_R^{n+1}$.

    A \textit{morphism of formal objects} $(R, \qty{\xi_n}, \qty{f_n}) \to (T, \qty{\eta_n}, \qty{g_n})$ is a collection of morphisms $\alpha_n \colon \xi_n \to \eta_n$ such that
    \begin{equation*}
    \begin{tikzcd}
        \xi_n \ar{r}{\alpha_n} \ar{d}{f_n} & \eta_n \ar{d}{g_n} \\
        \xi_{n+1} \ar{r}{\alpha_{n+1}} & \eta_{n+1}
    \end{tikzcd}
    \end{equation*}
    commutes. These define morphisms $\Spec R/\mf{m}_R^n \to \Spec T/\mf{m}_T^n$ that are compatible, and hence a morphism $\Spec R \to \Spec T$.
\end{defns}

There is a functor from $\mc{X}(\Spec R)$ to the category of formal objects $(R, \ldots)$. We say that a formal object is \textit{effective} if it is in the essential image of this functor.

\begin{defn}
    Let $R$ be a complete local Noetherian $k$-algebra and $\xi \in \mc{X}(\Spec R)$. We say that $\xi$ is \textit{versal} if for any diagram
    \begin{equation*}
    \begin{tikzcd}
        \Spec k \ar{r} & \Spec B \ar{r} \ar{d} & \Spec C \ar{d}{\xi} \\
        & \Spec B' \ar{r}{\eta'} & \mc{X}
    \end{tikzcd}
    \end{equation*}
    such that $B' \twoheadrightarrow B$ is a surjection of local Artinian $k$-algebras and $\alpha \colon \xi \_{\Spec B} \to \eta' |_{\Spec B}$ is an isomorphism, then there exists $\Spec B' \to \Spec R$ such that $\alpha' = \xi |_{\Spec B'} \simeq \eta'$ extending $\alpha$.

    A formal object $(R, \qty{\xi_n}, \qty{f_n})$ is \textit{versal} if in the same diagram, if we replace $\xi$ wuth $\xi_n$ and $R$ with $R/\mf{m}^n$, we can find a lift to $\Spec R/\mf{m}^m$ for some $m \geq n$.
\end{defn}

\begin{thm}[Artin's axioms]
    Let $\mc{X}$ be a stack over $(\ms{Sch}/S)_{\text{\'et}}$. Then $\mc{X}$ is an algebraic stack locally of finite type over $k$ if and only if
    \begin{enumerate}\setcounter{enumi}{-1}
        \item $\mc{X}$ is limit-preserving.
        \item The diagonal $\Delta \colon \mc{X} \to \mc{X} \times \mc{X}$ is representable.
        \item (formal deformations) for every $x \colon \Spec k \to \mc{X}$, there exists a complete local Noetherian $k$-algebra $(R, \mf{m})$ and a versal formal object $(R, \qty{\xi_n}, \qty{f_n})$ such that $\xi_1 = x$.
        \item Every formal object is effective.
        \item (openness of versality) Let $\xi_U \colon U \to \mc{X}$, where $U$ is a scheme of finite type over $k$, and $u \in U$ be a $k$-point such that $\xi_U |_{\Spec \wh{\mc{O}}_{U, u}}$ is versal. Then $\xi_U$ is versal at all $k$-points in an open neighborhood of $u$.
    \end{enumerate}
\end{thm}

\begin{rmk}
    Suppose we want to prove that Artin's axioms imply that $\mc{X}$ is an algebraic stack locally of finite type over $k$. Given a formal object $(R, \qty{\xi_n}, \qty{f_n})$, we have an actual object $\xi \in \mc{X}(\Spec R)$. By approximation and algebraization, we have $U \to \mc{X}$ finite type and versal at $X = U$. Now $U \to \mc{X}$ is smooth, so we have $\bigsqcup U \to \mc{X}$ smooth, surjective, and representable.
\end{rmk}

\begin{rmk}
    In many modular problems, the condition about formal deformations follows from the Rim-Schlessinger condition that for a diagram
    \begin{equation*}
    \begin{tikzcd}
        A \times_B C \ar{r} \ar{d} & C \ar{d} \\
        A \ar[twoheadrightarrow]{r} & B
    \end{tikzcd}
    \end{equation*}
    of Artinian local $k$-algebras, the map
    \[ \mc{X}(\Spec A \times_B C) \to \mc{X}(A) \times_{\mc{X}(B)} \mc{X}(C) \]
    is an equivalence of categories. Here, if $x_0 \in \mc{X}(\Spec k)$, define $F_{\mc{C}, x_0}$ by $A \mapsto x \in \mc{X}(\Spec A)$ such that $x_0 \to x$ lies over $\Spec k \to \Spec A$. Then $TF_{\mc{C}, x_0} = F_{\mc{C}, x_0}(k[\ep])$ is a $k$-vector space. If $\dim_k TF_{\mc{C}(x_0)} < \infty$, then we have the condition on formal deformations. Then smoothness of $U \to F_{\mc{C}, x_0}$ gives versality.
\end{rmk}

\begin{rmk}
    The third condition follows from Grothendieck's existence theorem, and a deformation-obstruction theory gives openness of versality.
\end{rmk}

\chapter{Che (Dec 03): Stack of coherent sheaves}%
\label{cha:che_dec_03_stack_of_coherent_sheaves}

Let $k$ be an algebraically closed field of characteristic $0$, $(\ms{Sch}/k)_{\text{\'et}}$ be the \'etale site of schemes over $k$, and $X$ be a projective scheme over $k$.

\begin{defn}
    Let $\ms{Coh}_X$ be the category defined as follows:
    \begin{itemize}
        \item Objects are tuples $(T, \mc{F})$ of a scheme $T$ and $\mc{F}$ is a quasicoherent sheaf on $X \times T$ of finite presentation flat over $T$.
        \item Morphisms from $(T, \mc{F})$ to $(T', \mc{F}')$ are pairs $(h, \varphi)$, where $h \colon T \to T'$ is a morphism of schemes and $\varphi \colon (h')^* \mc{F}' \to \mc{F}$ is an isomorphism of $\mc{O}_{X_T}$-modules. Here, $h' \colon X_T \to X_{T'}$ is the morphism induced from $h$.
    \end{itemize}
\end{defn}

There is a functor $p \colon \ms{Coh}_X \to (\ms{Sch}/k)_{\text{\'et}}$ given by $(T,\mc{F}) \mapsto T$, and we want to prove that $\mc{X} \coloneqq \ms{Coh}_X$ is an algebraic stack. Also, we will abuse notation and write $h$ for $h'$. We will prove that $\mc{X}$ satisfies Artin's axioms. Recall that these are
\begin{enumerate}[(1)]\setcounter{enumi}{-1}
    \item $\mc{X}$ is a stack. This means that $\mc{X}$ is a prestack and objects and morphisms glue.
    \item $\Delta \colon \mc{X} \to \mc{X} \times \mc{X}$ is representable by algebraic spaces.
    \item $\mc{X}$ is limit-preserving.
    \item $\mc{X}$ satisfies the Rim-Schlessinger condition.
    \item The tangent spaces $T \mc{F}_{\mc{X}, X_0}$ and $\mr{Inf}(\mc{F}_{\mc{X}, X_0})$ are finite-dimensional.
    \item Every formal object is effective.
    \item $\mc{X}$ satisfies openness of versality.
\end{enumerate}
These will imply that there is a smooth surjective covering of $\mc{X}$ by a scheme.

\section{$\mc{X}$ is a stack}%
\label{sec:_x_is_a_stack}

First, we will prove that $\mc{X}$ is a prestack. We note that $\mc{X}(T)$ has objects finitely presented $\mc{F}$ on $X_T$ flat over $T$ and morphisms $(\mr{id}_T, \varphi)$, where $\varphi \colon \mc{F}' \to \mc{F}$ is an isomorphism. Thus $\mc{X}(T)$ is a groupoid. We now want to prove that pullbacks exist, which is clear because given $h \colon T \to T'$ and an object $(T', \mc{F}')$, our pullback is simply $(T, h^* \mc{F}')$. To show that pullbacks are universal, consider $h_1 \colon T_1 \xrightarrow{h} T_2 \xrightarrow{h_2} T$. Now suppose that we have morphisms $(T_1, \mc{F}_1) \to (T, \mc{F})$ and $(T_2, \mc{F}_2) \to (T, \mc{F})$. Now we want to prove that there exists an isomorphism $\mc{F}_1 \simeq h^* \mc{F}_2$. But we know that 
\[ h^* \mc{F}_2 \simeq h^* h_2^* \mc{F} \simeq h_1^* \mc{F} \simeq \mc{F}_1, \]
and so we are done.

Now we want to check that objects glue. Given $T$ and a covering $\qty{a_i \colon T_i \to T}$, suppose we have $\mc{F}_i$ on $X_{T_i}$ flat over $T_i$. Suppose that the $\mc{F}_i$ are isomorphic on intersections and satisfy the cocycle condition. We want to construct a $\mc{F}$ on $X_T$ of finite presentation, flat over $T$. This follows from \'etale descent, so we are done.

We now want to prove that morphisms glue. To do this, we need to introduce some new notions.
\begin{defn}
    Given $X = (T, \mc{F})$ and $Y = (T, \mc{G})$, define the \textit{Isom presheaf} $\mr{Isom}_{\mc{X}}(X, Y)$ sending a scheme $S \to T$ to the set $\Hom(\mc{F}|_S, \mc{G}|_S)$ in $\mc{X}(S)$.
\end{defn}
It is easy to see that morphisms glue if and only if this is a sheaf, and we will omit the proof.

\section{Representability of the diagonal}%
\label{sec:representability_of_the_diagonal}

We want to prove that for all schemes $S$ over $k$, the stack $\mc{Y}$ given by pullback in the diagram
\begin{equation*}
\begin{tikzcd}
    \mc{Y} \ar{r} \ar{d} & S \ar{d} \\
    \mc{X} \ar{r}{\Delta} & \mc{X} \times \mc{X}
\end{tikzcd}
\end{equation*}
is an algebraic space. By the $2$-Yoneda lemma, a map $S \to \mc{X} \times \mc{X}$ is given by $\xi = (S, \mc{F}), \eta = (S, \mc{G})$ in $\mc{X}(S)$. If we compute the fiber product $S \times_{\mc{X} \times \mc{X}} \mc{X}$, we actually obtain the Isom presheaf $\mr{Isom}_{\mc{X}}(\xi, \eta)$. We will use without proof the fact that $\mr{Isom}_{\mc{X}}(\xi, \eta)$ is a closed subfunctor of $\ul{\Hom}(\mc{F}, \mc{G})$ (defined by $T \mapsto \Hom(\mc{F}_T, \mc{G}_T)$).

We will only prove that $\ul{\Hom}(\mc{F}, \mc{G})$ is representable by an algebraic space when $\mc{F}, \mc{G}$ are locally free (and $X$ is a point apparently).\footnote{Here, Johan intervened and came to the board to talk about cohomology and base change things.} In this case, 
\[ \ul{\Hom}(\mc{F}, \mc{G})(T) = \Hom(\mc{F}_t, \mc{G}_T) = H^0((F^{\vee} \otimes \mc{G})_T). \]
But now this is clearly represented by the total space $\Spec \on{Sym}(\mc{F} \otimes \mc{G}^{\vee})$ because\footnote{Apparently the following is wrong, but is preserved here for recordkeeping.}
\begin{align*}
    \Hom_S(T, \Spec \on{Sym}(\mc{F \otimes \mc{G}^{\vee}})) &= \Hom_{\mc{O}_S}(\on{Sym}(\mc{F} \otimes \mc{G}^{\vee}), f_* \mc{O}_T) \\
    &= \Hom_{\mc{O}_S}(\mc{F} \otimes \mc{G}^{\vee}, f_* \mc{O}_T) \\
    &= \Hom_{\mc{O}_T}(f^*(\mc{F} \otimes \mc{G}^{\vee}), \mc{O}_T) \\
    &= H^0((\mc{F}^{\vee} \otimes \mc{G})_T).
\end{align*}

\section{Preservation of limits}%
\label{sec:preservation_of_limits}

Given $T = \lim T_i$, we want to prove that $\mc{X}(T)$ is the colimit of the $\mc{X}(T_i)$. Given $\mc{F}$ on $X \times T$ of finite presentation and flat over $T$, we want to show that there exist $T_i, \mc{F}_i$ such that $\mc{F} = (X \to X_i)^* \mc{F}_i$.

\begin{prop}
    Let $R, R_i$ be rings such that $R$ is the colimit of the $R_i$. Let $M$ be an $R$-module of finite presentation. Then there exists $i$ and a finitely-presented $R_i$-module $M_i$ such that $M = M_i \otimes_{R_i} R$.
\end{prop}

\begin{proof}
    We know that $M$ is finitely presented, so we have an exact sequence
    \[ R^{\oplus m} \xrightarrow{(a_{jk})} R^{\oplus n} \to M \to 0. \]
    Because $R$ is the colimit of the $R_i$, there exists $i$ such that $a_{jk}$ lift to $R_i$. If we define $M_i$ by
    \[ R_i^{\oplus m} \xrightarrow{(a_{jk})} R_i^{\oplus n} \to M_i \to 0, \]
    this is clearly the desired $M_i$.
\end{proof}

Checking flatness is too hard, so we will not do it.

\section{Rim-Schlessinger}%
\label{sec:rim_schlessinger}

We will not prove this, but we will see that this is a natural condition to satisfy. Given a pushout diagram
\begin{equation*}
\begin{tikzcd}
    U \ar{r} \ar{d} & U' \ar{d} \\
    V \ar{r} & V'
\end{tikzcd}
\end{equation*}
where $U, U', V, V'$ are spectra of local Artinian rings of finite type over $k$ and $U \to U'$ is a closed embedding, we want the functor
\[ \mc{X}(V') \to \mc{X}(V) \times_{\mc{X}(U)} \mc{X}(U') \]
to be an equivalence of categories.

\section{Finiteness}%
\label{sec:finiteness}

We want to prove that the tangent space $T \mc{F}_{\mc{X}, X_0}$ and the infinitesimal automorphisms $\mr{Inf}(\mc{F}_{\mc{X}, X_0})$ are finite-dimensional. Given $X_0 \colon \Spec k \to \mc{X}$ (which is just a finitely presented sheaf $\mc{F}$ on $X$), the tangent space is
\[ T \mc{F}_{\mc{X}, X_0} \coloneqq \qty{\mc{F'} / X \times k[\ep] \text{ finitely-presented, flat over }k[\ep], \mc{F'}|_X \cong \mc{F}} / \sim = \Ext^1(\mc{F}, \mc{F}). \]
In addition, we know that
\[ \mr{Inf}(\mc{F}_{\mc{X}, X_0}) = \ker(\Aut(\mc{F} \otimes k[\ep] / X \times k[\ep]) \to \Aut(\mc{F}/X)) = \Ext^0(\mc{F}, \mc{F}). \]
Because $X$ is projective, these Ext groups are finite-dimensional.

\section{Formal objects are effective}%
\label{sec:formal_objects_are_effective}

We will now prove that formal objects are effective. Let $R \in \wh{\mc{C}}$ be a complete Noetherian local ring. Recall that a formal object is $\qty{\xi_n}$, where $\xi \in \mc{X}(\Spec R/\mf{m}_R^n)$ along with $f_n \colon \xi_n \to \xi_{n+1}$ living over the natural inclusions. A formal object is effective if it comes from an actual object over $R$.

In our case, a formal object is given by $\mc{F}_n$ on $X \times \Spec R/\mf{m}_R^n$ flat and finitely presented and $f_n \colon i_n^* \mc{F}_{n+1} \cong \mc{F}_n$. We want to show that there exists $\mc{F}$ over $\Spec R$ restricting to each $\mc{F}_n$.

\begin{thm}[Grothendieck existence theorem]
    Let $A$ be a Noetherian ring which is complete with respect to some ideal $I$. Let $f \colon X \to \Spec A$ be a proper morphism. Let $\mc{I} = I \mc{O}_X$. Then the functor 
    \[ \ms{Coh}(X) \to \qty{\mc{F_1} \gets \mc{F_2} \gets \cdots \mid \mc{F_n} \text{ coherent, annihilated by $\mc{I}^n$}, \mc{F}_{n+1} / \mc{I}^n \mc{F}_{n+1} \simeq \mc{F}_n} \]
    is an equivalence.
\end{thm}

In our case, take $A = R$ and $I = \mf{m}$, and now we are done.

\chapter{Johan (Dec 10): Openness of versality}%
\label{cha:johan_dec_10_openness_of_versality}

Let $k$ be an algebraically closed field. In this lecture, all schemes live over $k$ for simplicity. 

\section{Moduli of curves}%
\label{sec:moduli_of_curves}

Recall that $\ol{\mc{M}}_g$ is a stack such that a morphism $U \to \ol{\mc{M}}_g$ is a family $C \to U$ of stable curves of genus $g$. Assume that $U$ is of finite type over $k$ and let $u_0 \in U(k)$.

\begin{defn}
    We say that $U \to \ol{\mc{M}}_g$ is \textit{versal} at $u_0$ if $\wh{\mc{O}}_{U, u_0}$ and the map
    \[ h_{\wh{\mc{O}}_{U, u_0}} \to \mr{Def}_{C_{u_0}} \]
    given by $C|_{\Spec \wh{\mc{O}}_{U, u_0}}$ are a hull.
\end{defn}

\begin{lem}
    $U \to \ol{\mc{M}}_g$ is versal if and only if $U$ is smooth at $u_0$ and $T_{u_0} U \to T \mr{Def}_{C_{u_0}}$ is surjective.
\end{lem}

\begin{proof}
    Earlier, we discussed that deformations of $C_{u_0}$ are unobstructed. Therefore any hull is a power series ring over $k$. Thus $U$ must be smooth at $u_0$. If $U$ is smooth at $u_0$, then look at $\wh{\mc{O}}_{U, u_0} \gets R$, where $R$ is the deformation ring of $C_{u_0}$. But now this is a map of power series rings, and therefore defines a smooth transformation of functors if and only if the map on tangent spaces is surjective.
\end{proof}

\begin{lem}
    We have openness of versality for $\ol{\mc{M}}_g$ and $\mc{M}_g$.
\end{lem}

\begin{proof}[Proof for $\mc{M}_g$]
    By the previous lemma, we may assume that $U$ is smooth. Call $f \colon C \to U$ and consider the exact sequence
    \[ 0 \to T_{C/U} \to T_C \to f^* T_U \to 0. \]
    This gives $T_U = f_* f^* T_U \to R^1 f_* T_{C/U}$. Taking fibers aat $u_0$, we obtain
    \[ T_{u_0} U = T_U \otimes \kappa(u_0) \to R^1 f_* T_{C/U} \otimes \kappa(u_0) = H^1(C_{u_0}, T_{C_{u_0}}) = T_{u_0} \mr{Def}_{C_{u_0}}. \]
    Also, $R^1 f_* T_{C/U}$ is a vector bundle of rank $3g-3$ over $U$. By the lemma and the assumption of versality at $u_0$, we see this is surjective. Thus this is surjective in an open neighborhood.
\end{proof}

\begin{proof}[Proof for $\ol{\mc{M}}_g$\footnote{Not the same as in the Stacks project}]
    Consider the exact sequence
    \[ 0 \to f^* \Omega_{U/k} \to \Omega_{C/k} \to \Omega_{C/U} \to 0. \]
    We think of this as $\Omega_{C/U} \to f^* \Omega_{U/k}[1]$ in the derived category, and now if we tensor with the relative dualizing sheaf, we have
    \[ \Omega_{C/U} \otimes \omega_{C/U} \to f^* \Omega_{U/k} \otimes \omega_{C/U}[1] = f^!(\Omega_{U/k}), \]
    where $f^!$ is the right adjoint to $Rf_*$ in this case. This gives
    \[ Rf_* (\Omega_{C/U} \otimes \omega_C) \to \Omega_{U/k}. \]
    By positivity properties, $Rf_* (\Omega_{C/U} \otimes \omega_{C/U}) = f_* (\Omega_{C/U} \otimes \omega_{C/U})$. Taking the fiber at $u_0$, we have
    \[ H^0(\Omega_{C_{u_0}/k} \otimes \omega_{C_{u_0}})^{\vee} = \Ext^1(\Omega_{C_{u_0}/k} \otimes \omega_{C_{u_0}}, \omega_{C_{u_0}/k}) = \Ext^1_{C_{u_0}}(\Omega_{C_{u_0}}, \mc{O}_{C_{u_0}}) \cong T_{u_0} \mr{Def}_{C_{u_0}}. \]
    By the same argument as before, we are done.
\end{proof}

\section{Properties of cotangent complexes}%
\label{sec:properties_of_cotangent_complexes}

Suppose that $U$ is of finite type over $k$. Then there is a complex $L_{U/k} \in D^{\leq 0}_{\mr{coh}}(\mc{O}_U)$ such that $H^0(L_{U/k}) = \Omega_{U/k}$ called the \textit{cotangent complex}. Suppose we write
\[ \wh{\mc{O}}_{U, u_0} \cong k[[x_1, \ldots, x_n]]/ (f_1, \ldots, f_m) = k[[\ul{x}]]/I \]
with $n, m$ minimal. This implies that $f_1, \ldots, f_m \in (x_1, \ldots, x_n)^2$. Then $H^0(L_{U/k} \otimes \kappa(u_0))$ is the cotangent space of $U$ at $u_0$ and thus has dimension $n$. Next we note that
\[ H^{-1} L_{U/k} \otimes \kappa(u_0)] = I/\mf{m} I, \]
and this has dimension $m$.

Now let $g \colon U \to V$ be a morphism of schemes of finite type over $k$. Write $u_0 \mapsto v_0$. Then if we consider the distinguished triangle
\[ L g^* L_{v/k} \to L_{U/k} \to L_{U/V}, \]
we see that $g$ is smooth at $u_0$ if and only if $H^0(L_{U/k} \otimes \kappa(u_0)) \gets H^0(L_{V/k} \otimes \kappa(v_0))$ is injective and $H^{-1}(L_{U/k} \otimes \kappa(u_0)) \gets H^{-1}(L_{V/k} \otimes \kappa(v_0))$ is surjective.

\begin{rmk}
    Suppose that $f \colon X \to U$ is a proper flat morphism corresponding to $U \to \mc{X}$, where $\mc{X}$ is the prestack pramaterizing families of of flat proper schemes. Then we have $L_{X/U} \to L f^* L_{U/k}[1]$, and tensoring with the dualizing complex, we have
    \[ L_{X/U} \otimes \omega_{X/U}^{\bullet} \to L f^* L_{U/k} \otimes \omega_{X/U}^{\bullet}[1] = f^! (L_{U/k})[1]. \]
    We should consinder instead
    \[ \mr{can}_{X/U} \colon R f_* (L_{X/U} \otimes \omega_{X/U}^{\bullet})[-1] \to L_{U/k}. \]
    Versality is then related to properties of $H^i(\mr{can}_{X/U} \otimes^{\mathbb{L}} \kappa(u_0))$ for $i = 0,-1$.
\end{rmk}

\section{Coherent sheaves}%
\label{sec:coherent_sheaves}

Let $X \to \Spec k$ be proper. Then $\ms{Coh}_{X/k}$ is a stack such that for all $U/k$ of finite type, a morphism $U \to \ms{Coh}_{X/k}$ is a coherent sheaf $\mc{F}$ on $X \times U$ which is flat over $U$.

\begin{lem}
    We have openness of versality for $U \to \ms{Coh}_{X/k}$.
\end{lem}

\begin{proof}[Easy case]
    We will assume that $\mc{F}_{u_0}$ is a vector bundle and $\Ext^2_X(\mc{F}_{u_0}, \mc{F}_{u_0}) = 0$. As before, we may assume that $U$ is smooth and $\mc{F}$ is a vector bundle. We then have the Atiyah extension
    \[ 0 \to \mc{F} \otimes \Omega^1_{X \times U} \to P(\mc{F}) \to \mc{F} \to 0 \]
    and a map $\mc{F} \otimes \Omega^1_{X \times U/k} \to \mc{F} \otimes p^* \Omega_{U/k}$. Then we have the Atiyah class
    \[ \mc{{F} \to (\mc{F} \otimes \Omega^1_{X \times U/k})[1]}. \]
    Dually, we have $f^* T_{U/k} \to \Hom(\mc{F}, \mc{F})[1]$, 
    \[ T_{U/k} \to R f_* f^* T_{U/k} \to R f_* (\Hom(\mc{F}, \mc{F}))[1], \]
    etc. By similar arguments as before, we obtain the desired result.
\end{proof}

\begin{proof}[General case (terrible)]
    We have an Atiyah class
    \begin{equation*}
    \begin{tikzcd}
        \mc{F} \ar{r}{\xi} \ar{dr}{\xi'} & \mc{F} \otimes L_{X \times U/k}[1] \\
        & \mc{F} \otimes Lp^* L_{U/k}[1].
    \end{tikzcd}
    \end{equation*}
    This yields
    \[ \xi'' \colon \mc{F} \otimes R\Hom(\mc{F}, q^* \omega^{\bullet}_{X/k}) \to p^* L_{U/k} \otimes q^* \omega^{\bullet}_{X/k}[1] = p^!(L_{U/k}). \]
    The adjunction gives us
    \[ R p_* (\mc{F} \otimes R \Hom(\mc{F}, q^* \omega^{\bullet}_{X/k}))[-1] \to L_{U/k}. \]
    We need to show that formation of the left hand side commutes with base change and that 
    \[ H^i(X, \mc{F}_{u_0} \otimes \mc{F}_{u_0}, \omega_{X/k}^{\bullet}) = \Ext_X^{-i}(\mc{F}_{u_0}, \mc{F}_{u_0}), \]
    and then we can use cohomology and base change.
\end{proof}

\section{A trick}%
\label{sec:a_trick}

Sometimes we can get openness of versality for a prestack $\mc{X}$. Here, openness of versality holds for the prestack $\mc{X}$ if
\begin{enumerate}
    \item $\mc{X} \to \mc{X} \times \mc{X}$ is representable by algebraic spaces.
    \item We have the condition $(RS_*)$, which is a version of RS where the rings do not need to be Artinian.
    \item $\mc{X}$ is limit-preserving.
    \item The following effectiveness holds: If $A = \lim A_n$ where
        \[ A \to \cdots \to A_3 \to A_2 \to A_1 \]
        where each $A_n \to A-1$ is surjective with square zero kernel, then $\mc{X}(A) = \lim \mc{X}(A_n)$.
\end{enumerate}

In the example on $\ms{Coh}_{X/k}$, we have $\mc{F}_n$ on $X \otimes A_n$, and then we can attempt to take the limit $\mc{F}$ of the $\mc{F}_n$. This fails because the limit is not quasicoherent, but we can fix it.


\end{document}
