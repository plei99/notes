\documentclass[leqno, openany]{memoir}
\setulmarginsandblock{3.5cm}{3.5cm}{*}
\setlrmarginsandblock{3cm}{3.5cm}{*}
\checkandfixthelayout

\usepackage{amsmath}
\usepackage{amssymb}
\usepackage{amsthm}
%\usepackage{MnSymbol}
\usepackage{bm}
\usepackage{accents}
\usepackage{mathtools}
\usepackage{tikz}
\usetikzlibrary{decorations.pathmorphing,shapes}
\usetikzlibrary{calc}
\usetikzlibrary{automata,positioning}
\usepackage{tikz-cd}
\usepackage{forest}
\usepackage{braket} 
\usepackage{listings}
\usepackage{mdframed}
\usepackage{verbatim}
\usepackage{physics}
\usepackage{stmaryrd}
\usepackage{mathrsfs} 
\usepackage{stackengine} 
%\usepackage{/home/patrickl/homework/macaulay2}

%font
\usepackage[sc]{mathpazo}
\usepackage{eulervm}
\usepackage[scaled=0.86]{berasans}
\usepackage{inconsolata}
\usepackage{microtype}

%CS packages
\usepackage{algorithmicx}
\usepackage{algpseudocode}
\usepackage{algorithm}

% typeset and bib
\usepackage[english]{babel} 
\usepackage[utf8]{inputenc} 
\usepackage[T1]{fontenc}
\usepackage[backend=biber, style=alphabetic]{biblatex}
\usepackage[bookmarks, colorlinks, breaklinks]{hyperref} 
\hypersetup{linkcolor=black,citecolor=black,filecolor=black,urlcolor=black}

% other formatting packages
\usepackage{float}
\usepackage{booktabs}
\usepackage[shortlabels]{enumitem}
\usepackage{csquotes}
\usepackage{titlesec}
\usepackage{titling}
\usepackage{fancyhdr}
\usepackage{lastpage}
\usepackage{parskip}
\usepackage{graphicx}
\graphicspath{{./images/}}

\usepackage{lipsum}

% delimiters
\DeclarePairedDelimiter{\gen}{\langle}{\rangle}
\DeclarePairedDelimiter{\floor}{\lfloor}{\rfloor}
\DeclarePairedDelimiter{\ceil}{\lceil}{\rceil}


\newtheorem{thm}{Theorem}[section]
\newtheorem{cor}[thm]{Corollary}
\newtheorem{prop}[thm]{Proposition}
\newtheorem{lem}[thm]{Lemma}
\newtheorem{conj}[thm]{Conjecture}
\newtheorem{quest}[thm]{Question}

\theoremstyle{definition}
\newtheorem{defn}[thm]{Definition}
\newtheorem{defns}[thm]{Definitions}
\newtheorem{con}[thm]{Construction}
\newtheorem{exm}[thm]{Example}
\newtheorem{exms}[thm]{Examples}
\newtheorem{notn}[thm]{Notation}
\newtheorem{notns}[thm]{Notations}
\newtheorem{addm}[thm]{Addendum}
\newtheorem{exer}[thm]{Exercise}

\theoremstyle{remark}
\newtheorem{rmk}[thm]{Remark}
\newtheorem{rmks}[thm]{Remarks}
\newtheorem{warn}[thm]{Warning}
\newtheorem{sch}[thm]{Scholium}


% unnumbered theorems
\theoremstyle{plain}
\newtheorem*{thm*}{Theorem}
\newtheorem*{prop*}{Proposition}
\newtheorem*{lem*}{Lemma}
\newtheorem*{cor*}{Corollary}
\newtheorem*{conj*}{Conjecture}

% unnumbered definitions
\theoremstyle{definition}
\newtheorem*{defn*}{Definition}
\newtheorem*{exer*}{Exercise}
\newtheorem*{defns*}{Definitions}
\newtheorem*{con*}{Construction}
\newtheorem*{exm*}{Example}
\newtheorem*{exms*}{Examples}
\newtheorem*{notn*}{Notation}
\newtheorem*{notns*}{Notations}
\newtheorem*{addm*}{Addendum}


\theoremstyle{remark}
\newtheorem*{rmk*}{Remark}

% shortcuts
\newcommand{\Ima}{\mathrm{Im}}
\newcommand{\A}{\mathbb{A}}
\newcommand{\G}{\mathbb{G}}
\newcommand{\N}{\mathbb{N}}
\newcommand{\R}{\mathbb{R}}
\newcommand{\C}{\mathbb{C}}
\newcommand{\Z}{\mathbb{Z}}
\newcommand{\Q}{\mathbb{Q}}
\newcommand{\U}{\mathcal{U}}
\newcommand{\cO}{\mathcal{O}}
\renewcommand{\k}{\Bbbk}
\renewcommand{\P}{\mathbb{P}}
\newcommand{\M}{\overline{M}}
\newcommand{\g}{\mathfrak{g}}
\newcommand{\h}{\mathfrak{h}}
\newcommand{\n}{\mathfrak{n}}
\renewcommand{\b}{\mathfrak{b}}
\newcommand{\ep}{\varepsilon}
\newcommand*{\dt}[1]{%
   \accentset{\mbox{\Huge\bfseries .}}{#1}}
\renewcommand{\abstractname}{Official Description}
\newcommand{\mc}[1]{\mathcal{#1}}
\newcommand{\T}{\mathbb{T}}
\newcommand{\mf}[1]{\mathfrak{#1}}
\newcommand{\mr}[1]{\mathrm{#1}}
\newcommand{\ms}[1]{\mathsf{#1}}
\newcommand{\ol}[1]{\overline{#1}}
\newcommand{\ul}[1]{\underline{#1}}
\newcommand{\wt}[1]{\widetilde{#1}}
\newcommand{\wh}[1]{\widehat{#1}}
\renewcommand{\div}{\operatorname{div}}
\newcommand{\Sm}{\mathsf{Sm}}
\newcommand{\Cor}{\mathsf{Cor}}

\DeclareMathOperator{\Der}{Der}
\DeclareMathOperator{\Hom}{Hom}
\DeclareMathOperator{\End}{End}
\DeclareMathOperator{\ad}{ad}
\DeclareMathOperator{\Aut}{Aut}
\DeclareMathOperator{\Rad}{Rad}
\DeclareMathOperator{\Pic}{Pic}
\DeclareMathOperator{\supp}{supp}
\DeclareMathOperator{\Supp}{Supp}
\DeclareMathOperator{\sgn}{sgn}
\DeclareMathOperator{\spec}{Spec}
\DeclareMathOperator{\rk}{rk}
\DeclareMathOperator{\Spec}{Spec}
\DeclareMathOperator{\proj}{Proj}
\DeclareMathOperator{\Proj}{Proj}
\DeclareMathOperator{\ord}{ord}
\DeclareMathOperator{\Div}{Div}
\DeclareMathOperator{\Bl}{Bl}
\DeclareMathOperator{\ch}{ch}
\DeclareMathOperator{\td}{td}
\DeclareMathOperator{\Tor}{Tor}
\DeclareMathOperator{\depth}{depth}
\DeclareMathOperator{\CH}{CH}
\DeclareMathOperator{\Ob}{Ob}
\DeclareMathOperator{\Rat}{Rat} 
\DeclareMathOperator{\coker}{coker}
\DeclareMathOperator{\Hilb}{Hilb}
\DeclareMathOperator{\Sym}{Sym}

% Section formatting
\titleformat{\section}
    {\Large\sffamily\scshape\bfseries}{\thesection}{1em}{}
\titleformat{\subsection}[runin]
    {\large\sffamily\bfseries}{\thesubsection}{1em}{}
\titleformat{\subsubsection}[runin]{\normalfont\itshape}{\thesubsubsection}{1em}{}

\title{COURSE TITLE}
\author{Lectures by INSTRUCTOR, Notes by NOTETAKER}
\date{SEMESTER}

\newcommand*{\titleSW}
    {\begingroup% Story of Writing
    \raggedleft
    \vspace*{\baselineskip}
    {\Huge\itshape Category O Learning Seminar \\ Fall 2021}\\[\baselineskip]
    {\large\itshape Notes by Patrick Lei}\\[0.2\textheight]
    {\Large Lectures by Various}\par
    \vfill
    {\Large \sffamily Columbia University}
    \vspace*{\baselineskip}
\endgroup}
\pagestyle{simple}

\chapterstyle{ell}


%\renewcommand{\cftchapterpagefont}{}
\renewcommand\cftchapterfont{\sffamily}
\renewcommand\cftsectionfont{\scshape}
\renewcommand*{\cftchapterleader}{}
\renewcommand*{\cftsectionleader}{}
\renewcommand*{\cftsubsectionleader}{}
\renewcommand*{\cftchapterformatpnum}[1]{~\textbullet~#1}
\renewcommand*{\cftsectionformatpnum}[1]{~\textbullet~#1}
\renewcommand*{\cftsubsectionformatpnum}[1]{~\textbullet~#1}
\renewcommand{\cftchapterafterpnum}{\cftparfillskip}
\renewcommand{\cftsectionafterpnum}{\cftparfillskip}
\renewcommand{\cftsubsectionafterpnum}{\cftparfillskip}
\setrmarg{3.55em plus 1fil}
\setsecnumdepth{subsection}
\maxsecnumdepth{subsection}
\settocdepth{subsection}

\begin{document}
    
\begin{titlingpage}
\titleSW
\end{titlingpage}

\thispagestyle{empty}
\section*{Disclaimer}%
\label{sec:disclaimer}

These notes were taken during the seminar using the \texttt{vimtex} package of
the editor \texttt{neovim}.  Any errors are mine and not the speakers'.  In
addition, my notes are picture-free (but will include commutative diagrams) and
are a mix of my mathematical style and that of the lecturers.  If you find any
errors, please contact me at \texttt{plei@math.columbia.edu}.

\vspace*{1cm}

\noindent\textbf{Seminar Website:}\\
\url{https://math.columbia.edu/~plei/f21-CO.html} \newpage

\tableofcontents

\chapter{Kevin (Sep 29): Review of semisimple Lie algebras and introduction to category $\mc{O}$}%
\label{cha:kevin_sep_29_review_of_semisimple_lie_algebras_and_introduction_to_category_o}

\section{Review of semisimple Lie algebras}%
\label{sec:review_of_semisimple_lie_algebras}

Throughout this lecture, we will work over $\C$. 

\begin{defn}
    A Lie algebra $\mf{g}$ is \textit{semisimple} if any of the following equivalent conditions hold:
    \begin{enumerate}
        \item $\mf{g}$ is a direct sum of simple Lie algebras (those with no nonzero proper ideals).
        \item The Killing form $\kappa(x,y) \coloneqq \tr (\ad(x) \ad(y))$ is nondegenerate.
        \item The radical (maximal solvable ideal) of $\mf{g}$ is zero.
    \end{enumerate}
\end{defn}

Some examples of semisimple Lie algebras include $\mf{sl}_n, \mf{so}_n, \mf{sp}_{2n}$, and in some sense (the classification of simple Lie algebras), these are essentially all semisimple Lie algebras.

Now given a semisimple Lie algebra $\mf{g}$, we will fix a \textit{Cartan subalgebra} $\mf{h} \subset \mf{g}$, which is just a maximal abelian subalgebra of semisimple elements. This gives us a root decomposition
\[ \mf{g} = \mf{h} \oplus \bigoplus_{\alpha \in \mf{h}^* \setminus \qty{0}} \mf{g}_{\alpha}, \]
where $\mf{g}_{\alpha}$ is the subspace of $\mf{g}$ where $\mf{h}$ acts with weight $\alpha$. Some important facts about these root systems are the following:
\begin{itemize}
    \item For all $\alpha$, we have $\dim \mf{g}_{\alpha} = 1$.
    \item For all roots $\alpha, \beta$, we have $[\mf{g}_{\alpha}, \mf{g}_{\beta}] \subset \mf{g}_{\alpha + \beta}$.
    \item If $\alpha$ is a root, so is $-\alpha$.
\end{itemize}
In addition, the $\alpha$ are required to form a (reduced) \textit{root system} (denoted $\Phi$), the precise definition of which is deliberately omitted. Given a choice of Borel subalgebra containing $\mf{h}$, we obtain a set $\Phi^+$ of positive roots and a set $\Delta$ of simple roots. In addition, given a root system $\Phi$, there is a \textit{dual root system} $\Phi^{\vee}$, whose roots are
\[ \alpha^{\vee} = \frac{2 \alpha}{(\alpha, \alpha)}, \alpha \in \Phi. \]

Now suppose that $\mf{g}$ is a semisimple Lie algebra with root system $\Phi$. For every $\alpha \in \Phi^+$, we may choose $x_{\alpha} \in \mf{g}_{\alpha}$ and $y_{\alpha} \in \mf{g}_{-\alpha}$, and these determine some $h_{\alpha} = [x_{\alpha}, y_{\alpha}] \in \mf{h}$. This choice can be made such that $\alpha(h_{\alpha}) = 2$.

Recall that the Lie algebra $\mf{sl}_2$ is spanned by the matrices
\[ x = \mqty(0 & 1 \\ 0 & 0), \qquad y = \mqty(0 & 0 \\ 1 & 0), \qquad h = \mqty(\dmat[0]{1,-1}). \]
Then the choice of $x_{\alpha}, y_{\alpha}, h_{\alpha}$ gives an embedding $\mf{sl}_2 \to \mf{g}$. These maps, ranging over all $\alpha$, cover all of $\mf{g}$. Now a basis of $\mf{g}$ is given by $x_{\alpha}, y_{\alpha}, \alpha \in \Phi$ and $h_{\alpha_i}$ for the \textbf{simple} roots $\alpha_i$. Therefore, to specify $\mf{g}$, we only need to give commutation relations for the basis elements.

Now suppose that $\Phi$ is some root system. We would like to construct a semisimple Lie algebra $\mf{g}$ with root system $\Phi$. We want to build a semisimple Lie algebra. To do this, choose a set of simple roots $\alpha_i$, and consider the Lie algebra
\[ \ev{x_{\alpha_i}, y_{\alpha_i}, h_{\alpha_i}} / \text{relations}, \]
where the relations are as follows:
\begin{itemize}
    \item $[h_{\alpha_i}, h_{\alpha_j}] = 0$.
    \item We have $[x_{\alpha_i}, y_{\alpha_j}] = h_{\alpha_i}$ if $i = j$ and this commutator vanishes otherwise.
    \item $[h_{\alpha_i}, x_{\alpha_j}] = \ev{\alpha_j, \alpha_i^{\vee}} x_{\alpha_j}$.
    \item $[h_{\alpha_i}, y_{\alpha_j}] = - \ev{\alpha_j, \alpha_i^{\vee}} y_{\alpha_j}$.
    \item ${\ad(x_{\alpha_i})}^{1-\ev{\alpha_j, \alpha_i^{\vee}}}(x_{\alpha_j}) = 0$ if $i \neq j$.
    \item ${\ad(y_{\alpha_i})}^{1-\ev{\alpha_j, \alpha_i^{\vee}}}(y_{\alpha_j}) = 0$ if $i \neq j$.
\end{itemize}
The first four relations are called the \textit{Weyl relations} and the last two are called the \textit{Serre relations}. Given this data, we end up with a semisimple Lie algebra $\mf{g}_{\Phi}$ with root system $\Phi$. In addition, if $\mf{g}$ is any other semisimple Lie algebra with root system $\Phi$, there is an isomorphism $\mf{g}_{\Phi} \xrightarrow{\sim} \mf{g}$. Moreover, we have a bijection between semisimple Lie algebras and reduced root systems, which restricts to a bijection between simple Lie algebras and irreducible root systems.

\begin{table}[H]
    \centering
    \caption{Root systems and Lie algebras}
    \label{tab:classification}
    \begin{tabular}{cc}
    \toprule
    Irreducible root systems & simple Lie algebras \\
    \midrule
    $A_n$ & $\mf{sl}_{n+1}$ \\
    $B_n$ & $\mf{so}_{2n+1}$ \\
    $C_n$ & $\mf{sp}_{2n}$ \\
    $D_n$ & $\mf{so}_{2n}$ \\
    $E_6, E_7, E_8, F_4, G_2$ & exceptional Lie algebras \\
    \bottomrule
    \end{tabular}
\end{table}

We will now discuss the finite-dimensional representation theory of semisimple Lie algebras $\mf{g}$.

\begin{thm}[Weyl's complete reducibility theorem]
    Any finite-dimensional representation of $\mf{g}$ is decomposes as a direct sum of simple representations.
\end{thm}

Now suppose that $M$ is a finite-dimensional $\mf{g}$-representation. Then $M$ has a weight decomposition
\[ M = \bigoplus_{\lambda \in \mf{h}^*} M_{\lambda}. \]
These $\lambda$ are \textit{integral weights}, which simply means that $\ev{\lambda, \alpha^{\vee}} \in \Z$ for all roots $\alpha$. For any root $\alpha$, $x_{\alpha}(M_{\lambda}) \subset M_{\lambda + \alpha}$ and $y_{\alpha}(M_{\lambda}) \subset M_{\lambda - \alpha}$. We would like to think that the $x_{\alpha}$ raise the weights and $y_{\alpha}$ lower the weights, so we introduce a partial order. We say that $\lambda \geq \mu$ if $\lambda - \mu \in \Z_{\geq 0} \Phi^+$.

By Weyl's complete reducibility theorem, it remains to classify the irreducible representations of $\mf{g}$. These are in bijection with the \textit{dominant} integral weights, which in particular means that $\ev{\lambda, \alpha^{\vee}} \geq 0$ for all $\alpha \in \Phi^+$. For any dominant weight $\lambda$, there is a unique highest-weight representation $L(\lambda)$. Here, $L(\lambda)$ is generated by a single \textit{maximal vector} $v$ of weight $\lambda$. This means that for all positive roots $\alpha$, $x_{\alpha} v = 0$.

\section{Introduction to category $\mc{O}$}%
\label{sec:introduction_to_category_o_}

We would now like to study infinite dimensional representations of $\mf{g}$. Of course, this is impossibly complicated in general, so we will impose some finiteness conditions on our representations.

\begin{defn}
    The category $\mc{O}$ is the full subcategory of $U(\mf{g})$-modules $M$ satisfying:
    \begin{enumerate}
        \item $M$ is finitely generated as a $U(\mf{g})$-module.
        \item $M$ is $\mf{h}$-semisimple and has a weight decomposition $M = \bigoplus_{\lambda \in \mf{h}^*} M_{\lambda}$.
        \item $M$ is locally $\mf{n}$-finite, where $\mf{n} = \bigoplus_{\alpha \in \Phi^+} \mf{g_{\alpha}}$. Precisely, this means that the $U(\mf{n})$ generated by any $v \in M$ is finite-dimensional.
    \end{enumerate}
\end{defn}

Here are some facts about category $\mc{O}$, which are stated without proof.
\begin{itemize}
    \item For all $M$ in our category and weights $\lambda$, the weight space $M_{\lambda}$ is finite-dimensional.
    \item $\mc{O}$ is a Noetherian (everything satisfies the descending chain condition) abelian category.
\end{itemize}

We will now describe some infinite-dimensional objects in category $\mc{O}$.

\begin{defn}
    For any weight $\lambda$, the \textit{Verma module} $M(\lambda)$ associated to $\lambda$ is the module
    \[ M(\lambda) = U(\mf{g}) \otimes_{U(\mf{b})} \C_{\lambda}, \]
    where $\mf{b} = \mf{h} + \mf{n}$ is the Borel subalgebra associated to our choice of positive roots and $\C_{\lambda}$ is the $\mf{b}$-module associated to the $1$-dimensional representation of $\mf{h}$ with weight $\lambda$ and the identification $\mf{b} / \mf{n} = \mf{h}$.
\end{defn}

\chapter{Fan (Oct 06): Beginnings in category $\mc{O}$: Vermas, central characters, and blocks}%
\label{cha:fan_oct_06_beginnings_in_category_o_vermas_central_characters_and_blocks}

Recall the following fact ahout semisimple Lie algebras. If we have a decomposition
\[ \mf{g} = \mf{h} \oplus_{\alpha \in \Phi} \mf{g}_{\alpha}, \]
where 
\begin{enumerate}
    \item $\dim \mf{g}_{\alpha} = 1$;
    \item $\Z \Phi \subset \mf{h}^*$ is a lattice of maximal rank;
    \item $\alpha \in \Phi$ implies $-\alpha \in \Phi$;
    \item $[[\mf{g}_{\alpha}, \mf{g}_{-\alpha}], \mf{g}_{\alpha}] \neq 0$,
\end{enumerate}
then $\mf{g}$ is a semisimple Lie algebra.

Now recall that the Weyl group $W$ is the group generated by the reflections $s_{\alpha}$ in the roots. For any $w \in W$, we define the \textit{length} 
\[ \ell(w) = \# \qty{\alpha \in \Phi_+ \mid w(\alpha) \in \Phi_-}. \]
Next, there is the Bruhat order, where if $w_2 = s w_1$ and $\ell(w_2) > \ell(w_1)$, we say $w_1 < w_2$.

Finally, throught this lecture, we will denote the weight lattice by $\Lambda$ and the root lattice by $Q$. In addition, we will denote the $\lambda$-weight space of a module $M$ by $M^{\lambda}$, and $M_{\lambda}$ will be the Verma associated to $\lambda$. Also, we will need the notion of the universal enveloping algebra, which we will not write down here.

\section{Definitions}%
\label{sec:definitions}

Recall that $\mc{O}$ is the full subcategory of $\ms{Mod}(\mc{U}\mf{g})$ of modules $M$ such that
\begin{enumerate}
    \item $M$ is finitely generated over $\mc{U}\mf{g}$.
    \item $M$ is $\mf{h}$-semisimple.
    \item $M$ is locally $\mf{n}$-finite.
    \item $\dim M^{\lambda} < \infty$.
    \item The set of weights of $M$ is contained in some finite union of cones $\lambda - Q_+$.
\end{enumerate}

\begin{thm}
    The following properties hold for $\mc{O}$:
    \begin{enumerate}
        \item $\mc{O}$ is Noetherian.
        \item $\mc{O}$ is closed under submodules, quotients, finite direct sums, and is abelian.
        \item $\mc{O}$ is closed under tensoring with finite-dimensional representations (in fact, if tensoring with $\mc{O}$ is exact and lands in $\mc{O}$, then $N$ must be finite-dimensional).
        \item $\mc{M}$ is locally $Z \g$-finite.
        \item All $M \in \mc{O}$ are finitely generated over $\U \n^-$.
    \end{enumerate}
\end{thm}

\section{Highest weight modules}%
\label{sec:highest_weight_modules}

\begin{defn}
    A vector $v_+$ is a \textit{maximal vector} if $\n^+ v_+ = 0$.
\end{defn}

\begin{defn}
    A module $M$ is a \textit{highest weight module} if there exists a maximal $v_+ \in M$ generateing $M$.
\end{defn}

\begin{defn}
    Let $\lambda \in \h^*$ and consider the $\b^+$-module $\C_{\lambda}$. Then the \textit{Verma module} for $\lambda$ is the module
    \[ M_{\lambda} \coloneqq \U \g \otimes_{\U \b} \C_{\lambda}. \]
\end{defn}

Note that we have the standard adjunction $\Hom_{\g}(M_{\lambda}, -) = \Hom_{\b}(\C_{\lambda}, -)$.

\begin{thm}
    For any highest weight module $M$ with highest weight $\lambda$,
    \begin{enumerate}
        \item $M = \ev{ f_1^{n_1} \cdots f_{\abs{\Phi_+}}^{n_{\abs{\Phi}}}}$, and in particular $M$ is $\h$-semisimple.
        \item All weights of $M$ are at most $\lambda$.
        \item For any $\mu < \lambda$, $\dim M^{\mu} < \infty$, and $\dim M^{\lambda} = 1$. In addition, $M \in \mc{O}$.
        \item Any quotient of $M$ is also a highest weight module with highest weight $\lambda$.
        \item Any submodule of a highest weight module with weight $\mu < \lambda$ is a proper submodule. if $M$ is simple, all maximal vectors have weight $v_+^{\lambda}$.
        \item There exists a unique maximal submodule, and thus $M$ has a unique simple quotient and thus is indecomposable.
        \item All simple highest weight modules with highest weight $\lambda$ are isomorphic, so $\dim \End M = 1$.
    \end{enumerate}
\end{thm}

\begin{cor}
    Let $M \in \mc{O}$. Then $M$ admits a filtration whose successive quotients are highest weight modules.
\end{cor}

\section{Verma modules}%
\label{sec:verma_modules}

Let $M_{\lambda}$ be a verma module and $L_{\lambda}$ be the unique simple quotient, and $N_{\lambda}$ be the unique maximal submodule.
\begin{thm}
    Any simple $L \in \mc{O}$ is isomorphic to $L_{\lambda}$ for some $\lambda$.
\end{thm}

\begin{prop}
    Let $\Sigma$ be the set of simple roots and $\sigma \in \Sigma$. Let $\lambda \in \h^*$ such that $\sigma^*(\lambda) \in \N$. Choose $v_+^{\lambda} \in M_{\lambda}$ a maximal vector. Then
    \[ f_{\sigma}^{\sigma^* \lambda + 1} v_+^{\lambda} = v_+^{\lambda - (\sigma^* \lambda + 1) \sigma}. \]
    In particular, there exists a nonzero morphism $M_{\lambda - (\sigma^* \lambda + 1) \sigma} \hookrightarrow M_{\lambda}$.
\end{prop}

\begin{lem}
    We have the commutation relations
    \[ [e_i, f_j^{k+1}] = 0, \qquad [e_i, f_i^{k+1}] = -(k+1) f_i^{k+1} (k - h_i), \qquad [h_i, f_j^{j+1}] = - (k+1) \alpha_j(h_i)f_j^{k+1}. \]
\end{lem}

\section{Examples}%
\label{sec:examples}

We will discuss the example of $\mf{sl}_2$. Let $\phi_i$ be the operator that outputs the $i$-th diagonal of a matrix. Then let $\alpha = 2 \phi_1 = \phi_1 - \phi_2$ be the root. Let $\alpha^{\dag}$ be the matrix such that $\kappa(\alpha^{\dag}, -) = \alpha(-)$. In particular, we have $\alpha^{\dag} = \frac{1}{4} h$.

Then note that if we choose units so that $\phi_1 = 1$, then $\alpha = 2$. If $\lambda = n$, then the Verma module for $\lambda$ has weights $n, n-2, \ldots$, and the simple module has weights $n, n-2, \ldots, -n$. For non-integral weights, we just have an infinite-dimensional representation. To see this, note that hitting any non-integral weight with $\mf{n}^-$ will not reach another maximal vector.

\section{Finite-dimensional modules}%
\label{sec:finite_dimensional_modules}

\begin{thm}
    For any weight $\lambda$, $\dim L_{\lambda} < \infty$ if and only if $\lambda \in \Lambda_+$ is a dominant integral weight. This is equivalent to $\dim L_{\lambda}^{\mu} = \dim L_{\lambda}^{w(\mu)}$ for all $w \in W$.
\end{thm}

This result tells us that weights of $L_{\lambda}$ are actually symmetric under the Weyl group.

\begin{proof}
    First, if the span of $v \in M^{\lambda}$ is finite-dimensional for $\mf{sl}_2$, then all of $\mf{h}$ stabilizes $\mr{Span}_{\mf{sl}_2} v$. This is because if $v^{\mu} \in N \coloneqq \ev{v}$, then
    \[ h(e_i v^{\mu}) = e_i h v^{\mu} + \alpha_i(h) e_i v^{\mu}, \]
    and thus $h(f_i v^{\mu}) \in \C f_i v^{\mu}$.

    Next, if $\dim L_{\lambda} < \infty$, then after restriction to $\mf{sl}_2$, we have $\lambda(h_i) = \alpha_i^*(\lambda) \in \N$, and thus $\lambda \in \Lambda_+$.

    Now suppose that $\lambda \in \Lambda_+$. Then after restricting $L_{\lambda}$ to $\mf{sl}_2^i$, the span of $v_+^{\lambda}$ is isomorphic to $L_{\alpha_i^* (\lambda) \cdot \phi_1}$, and in particular it is finite-dimensional. Next, we show that $L_{\lambda}$ is a sum of finitely many $\mf{sl}_2^i$-summands. To see this, consider the sum $M$ of all $\mf{sl}_2^i$-submodules of $L_{\lambda}$. But then if we denote a summand by $N$, we note that $\g \otimes N$ is a finite-dimensional representation of $\mf{sl}_2$, so the natural morphism
    \[ \g \otimes N \to L_{\lambda} \]
    lands inside $M$. But then $M = L_{\lambda}$ because $M$ is a nonzero submodule.

    Next, recall that for $M \in \mr{Rep}(\mf{sl}_2)$, then the sets of weights are invariant under reflection across the origin, so we have isomorphisms
    \[ f^{\alpha(\lambda)} \colon M^{\lambda} \leftrightarrows M^{s(\lambda)} \colon e^{\alpha(\lambda)}. \]
    Because $L_{\lambda}$ is a sum of finite-dimensional representations of $\mf{sl}_2$, for any $v \in L_{\lambda}^{\mu}$, consider the finite-dimensional $\mf{sl}_2$-module containing it. If we add them up, we know that $L_{\lambda}^{\mu}$ is some finite-dimensional $\mf{sl}_2$-module $N^{\mu}$. But then the $s_i$ generate $W$, and thus for all $w \in W$, $L_{\lambda}^{\mu} \cong L_{\lambda}^{w(\mu)}$.

    Finally, for any orbit of $W$, there exists exactly one representative in the dominant weight lattice, and because there are only finitely many dominant integral weights less than $\lambda$, there must only be finitely many orbits, so $L_{\lambda}$ is finite-dimensional.
\end{proof}

\section{Central actions}%
\label{sec:central_actions}

Here, we will consider the action of $Z \mf{g}$ on a module $M$. Suppose that $M$ is a highest weight module for weight $\lambda$. Then we note that
\[ h(z \cdot v_+^{\lambda}) = z h v_+^{\lambda} = \lambda(h) z v_+^{\lambda}, \]
and therefore $z v_+^{\lambda} = \vartheta_{\lambda}(z) v_+^{\lambda}$. Therefore $z$ acts by $\mc{V}_{\lambda}(z)$ on any highest weight module of weight $\lambda$, and we call the function $\vartheta_{\lambda} \colon Z \g \to \C$ a \textit{central character}. In general, all algebra morphisms $Z\g \to \C$ arise in this way. Then we have the decomposition
\[ z \in \U \g = \U \n^- \otimes \U \h \otimes \U \n^+, \]
and write $\pi_{\h} \colon \U \g \to \U \h$ for the morphism killing $\n^{\pm}$. Then $\vartheta_{\lambda}(z) = \lambda(\pi_{\h}(z))$, so $\pi_{\h} \colon \Z \g \to \U \h$ is an algebra homomorphism, and we will call this $\varphi_{\mr{HC}} \eqqcolon \varpi$, the Harish-Chandra morphism. In particular, we obtain a morphism
\[ \A^{\dim \h} \to \Spec Z \g. \]
Also, we will consider $\varpi \circ w \cdot$, where $w \circ \lambda = w(\lambda + \varrho) - \rho$, where 
\[ \varrho = \frac{1}{2} \sum_{\Phi^+} \alpha. \]
If we can identify the two morphisms on a Zariski-dense subset, they must agree in general. First, note that $\varpi(\lambda) = \vartheta_{\lambda}$, and now it suffices to show that $\vartheta_{\lambda} = \vartheta_{w \circ \lambda}$ for $\lambda \in \Lambda$.

To prove this, if there exists $\sigma \in \Sigma$ such that $\sigma^* (\lambda) \in N$, then $M_{s_{\sigma} \circ \lambda} \subset M_{\lambda}$, and thus $\theta_{\lambda} = \theta_{s_{\sigma} \circ \lambda}$. In addition, if $\sigma^*(\lambda) = -1$, we have $s_{\sigma} \circ \lambda = \lambda$, and if $\sigma^* \lambda \leq -2$, we can reverse the roles of $\lambda, \sigma \circ \lambda$ because
\[ \sigma^* (s_{\sigma} \circ \lambda) = \sigma^* \lambda - 2 \sigma^* \lambda -2 \geq 0. \]

\section{More on Harish-Chandra}%
\label{sec:more_on_harish_chandra}

Consider the twisted Harish-Chandra morphism
\[ Z \g \xrightarrow{\varphi_{\mr{HC}}} S \h \xrightarrow{\lambda \mapsto \lambda - \varrho} S \h. \]
This gives us a morphism $\psi_{\mr{HC}}$. In particular, we have
\[ \vartheta_{\lambda}(z) = (\lambda + \varrho) \psi_{\mr{HC}}(z) = \lambda(\varphi_{\mr{HC}}(z)). \]

\begin{thm}
    The image of $\psi_{\mr{HC}}$ is contained in ${(S \h)}^W$.
\end{thm}

To see this, note that
\[ \vartheta_{w \circ \lambda} = \psi_{\mr{HC}}(w \circ \lambda) + \varrho = \psi_{\mr{HC}}(w(\lambda + \varrho)) = \psi_{\mr{HC}}(\lambda + \varrho). \]

\begin{thm}\leavevmode
    \begin{enumerate}
        \item $\psi_{\mr{HC}}$ is an isomorphism $Z \g \to S \h^W$;
        \item If $\lambda, \mu$ are linked, then $\vartheta_{\lambda} = \vartheta_{\mu}$;
        \item Every element of $\Hom_{\ms{Alg}}(Z \g, \C)$ arises in this way.
    \end{enumerate}
\end{thm}
There is a simple way to see the last two parts of the theorem if we assume some algebraic geometry.

For a central character $\vartheta \colon Z\g \to \C$, consider the module
\[ M^{\vartheta} \coloneqq \ker^{\infty}(\ker(\vartheta)) = \qty{ v \in M \mid {(z - \vartheta(z))}^n v = 0 \text{ for all } z}, \]
where $n$ depends on $z$. We have a decomposition
\[ M^{\mu} = \bigoplus_{\vartheta} M^{\mu} \cap M^{\vartheta}, \]
which gives us
\[ M = \bigoplus M^{\vartheta}. \]
Now we may define subcategories of $\O$ given by
\[ \O^{\vartheta} \coloneqq \qty{M \mid M = M^{\vartheta}}. \]
Some examples are that all highest weight modules of weight $\lambda$ are contained in $\mc{O}^{\vartheta_{\lambda}}$.

\begin{prop}
    We have a decomposition
    \[ \bigoplus_{\vartheta = \vartheta_{\lambda}} \mc{O}^{\vartheta}. \]
\end{prop}

We will now consider blocks of category $\mc{O}$. We say that simple modules $S_1, S_2$ are in the same block if there is a nontrivial extension of $S_2$ by $S_1$. For general $M$, we know that $M$ has a finite Jordan-H\"older decomposition because $\mc{O}$ is Artinian, so $M$ is in some block if all of its Jordan-H\"older quotients are.

\begin{prop}
    If $\lambda \in \Lambda$, then $\mc{O}^{\vartheta_{\lambda}}$ is a block of $\mc{O}$.
\end{prop}

To prove this, if $\mu < \lambda$ are linked, then we have the diagram
\[ M_{\mu} \hookrightarrow N_{\lambda} \to M_{\lambda} \]
giving us an exact sequence
\[ 0 \to L_{\mu} \hookrightarrow N_{\lambda} / \Im N_{\mu} \twoheadrightarrow L_{\lambda} \to 0. \]

\chapter{Che (Oct 13): Formal characters and applications to finite-dimensional modules}%
\label{cha:che_oct_13_formal_characters_and_applications_to_finite_dimensional_modules}

\section{Weyl character and dimension formulas}%
\label{sec:weyl_character_and_dimension_formulas}

Today we will see what Category $\cO$ tells us about finite-dimensional modules. We will fix a semisimple Lie algebra $\g = \n_+ \oplus \h \oplus \n_-$.

\begin{defn}
    Given $M \in \cO$, define the function $\operatorname{ch} M \colon \h^* \to \Z^{\geq 0}$, where $\lambda \mapsto \dim M_{\lambda}$.
\end{defn}

Also let $e_{\lambda}$ be the characteristic function of $\lambda$. Now given $f, g \colon \h^* \to \Z^{\geq 0}$, define the convolution product
\[ f * g (\lambda) = \sum_{\mu + \nu = \lambda} f(\mu) g(\nu). \]
For example, $e_{\lambda} * e_{\mu} = e_{\lambda + \mu}$. Here, we assume that $f, g$ are supported on a finite union of things of the form $\lambda - \Gamma$, where $\Gamma$ are the non-negative weights. We will call the set of such functions $\mc{X}$.

\begin{prop}\leavevmode
    \begin{enumerate}
        \item If $0 \to M' \to M \to M'' \to 0$ is an exact sequence, then $\ch(M) = \ch(M') + \ch(M'')$.
        \item If $M \in \cO$ and $L$ is finite-dimensional, then $\ch(L \otimes M) = \ch(M) * \ch(L)$.
    \end{enumerate}
\end{prop}

\begin{proof}\leavevmode
    \begin{enumerate}
        \item Note that $\dim M_{\mu} = \dim M_{\mu}' + \dim M_{\mu}''$ for any such exact sequence.
        \item Note that $\dim {(L \otimes M)}_{\lambda} = \sum_{\mu + \nu = \lambda} \dim L_{\mu} \dim M_{\nu}$.\qedhere
    \end{enumerate}
\end{proof}

Last time we considered central characters for a highest weight module of weight $\lambda$ and highest weight vector $v^+$
\[ \chi_{\lambda} \colon Z(\g) \to \C \qquad z \mapsto \frac{z \cdot v^+}{v^+} = \lambda(\mr{pr}(z)). \]
If $L(\mu)$ is a subquotient of the Verma module $M(\lambda)$, then, then $\chi_{\mu} = \chi_{\lambda}$. Equivalently, $\mu$ and $\lambda$ are linked by some element $w$ of the Weyl group. Because $\cO$ is Artinian, for all $M \in \cO$ we have a finite filtration
\[ 0 = M_0 \subset M_1 \subset \cdots \subset M_n = M, \]
where all $M_{i+1}/M_i$ is simple, so they must be isomorphic to some $L(\lambda_i)$. But then we have
\begin{align*}
    \ch M &= \sum_i \ch L(\lambda_i) \\
    &= \sum_{w \in W} a(\lambda, w) L(w \circ \lambda).
\end{align*}
Our goal is now to compute the $a(\lambda, w)$.

Let $\lambda_1, \lambda_2, \ldots, \lambda_n$ be weights linked to lambda arranged such that if $\lambda_i \geq \lambda_j$, then $i \geq j$. Then we should have some identity of the form
\[ \mqty(\ch M(\lambda_1) \\ \vdots \\ \ch M(\lambda_n)) = \mqty(1 & * & * \\ 0 & \ddots & * \\ 0 & 0 & 1) \mqty(\ch L(\lambda_1) \\ \vdots \\ \ch(\lambda_n)). \]
This implies that $\ch L(\lambda) = \sum_{w \in W} b(\lambda, w) \ch M(w \circ \lambda)$. Note that $b(\lambda, 1) = 1$.

\begin{defn}
    Define the \textit{Kostant function} 
    \[ p \colon \h^* \to \Z^{\geq 0} \qquad \nu \mapsto \# \qty{{(c_{\alpha})}_{\alpha > 0} \in \Z^{\geq 0} \mid \sum c_{\alpha} \alpha = \nu}. \]
\end{defn}

\begin{prop}
    $p = \ch M(0)$. More generally, $e_{\lambda} * 0 = \ch M(\lambda)$.
\end{prop}

\begin{proof}
    By the PBW theorem, we know that $M(0)$ is spanned by $\U(\n_-)$. This is apparently equivalent to the definition of $p$.
\end{proof}

\begin{defn}
    Define the function
    \[ q = \prod_{\alpha > 0} (e_{\alpha/2} - e_{-\alpha/2}). \]
    Also define 
    \[ f_{\lambda} = e_0 + e_{-\lambda} + \cdots = \begin{cases}
        1 * \alpha & -k \lambda, k \in \Z^{\geq 0} \\
        0 & \text{otherwise}.
    \end{cases}\]
\end{defn}
Note that $f_{\alpha} * (1- e_{-\alpha}) = 1$. Also note that
\[ q * \prod_{\alpha > 0} f_{\alpha} = e_{\rho} \prod_{\alpha > 0} (1-e_{-\alpha}) \prod_{\alpha > 0} f_{\alpha} = e_{\rho}. \]
Next, $p = \prod_{\alpha > 0} f_{\alpha}$ and if $\alpha$ is a simple root, then $s_{\alpha} \cdot q = -q$. The reason for this is that $s_{\alpha} (\alpha) = -\alpha$ but $s_\alpha$ fixes the other positive roots. This implies that $w \cdot q = {(-1)}^{\ell(w)} q$.

\begin{thm}[Weyl character formula]
    If $\lambda \in \Lambda^+$, then
    \[ q * \ch L(\lambda) = \sum_{w \in W} {(-1)}^{\ell(w)} e_{w \circ \lambda + \rho}. \]
\end{thm}

\begin{proof}
    If we apply $q * -$ to the formula
    \[ \ch L(\lambda) = \sum_{w \in W} b(\lambda, w) \ch M(w \circ \lambda), \]
    we obtain
    \begin{align*} 
        q * \ch L(\lambda) &= \sum_{w \in W} q * \ch(w \circ \lambda) \\
        &= \sum_{w \in W} b(\lambda, w) e_{w \circ \lambda + \rho}.
    \end{align*}
    Because $\lambda \in \Lambda^+$, we know that $L(\lambda)$ is finite-dimensional and all weight spaces are symmetric. If $\alpha$ is a simple root, then we apply $s_{\alpha}$ to both sides, and we obtain
    \[ -q \ch L(\lambda) = \sum_{w \in W} b(\lambda, w) s_{\alpha} (w \circ \lambda + \rho) = e_{s_{\alpha}w \circ \lambda + \rho} \]
    because $s_{\alpha}(w \circ \lambda + \rho) = s_{\alpha} w(\lambda + \rho) = s_{\alpha} w \circ \lambda + \rho$. Therefore we see that $b(\lambda, s_{\alpha} w) = -b(\lambda, w)$, so $b(\lambda, w) = {(-1)}^{\ell(w)}$.
\end{proof}

We would now like to compute $\dim L(\lambda)$ for dominant integral weights $\lambda$. We want something like
\[ \mr{sum}(q) \cdot \dim L(\lambda) = \sum_{w \in W} {(-1)}^{\ell(w)}, \]
except that both sides here vanish, so this is too na\"{\i}ve. For example, if we consider $\mf{sl}_2$, we have
\[ (e_1 - e_{-1}) \ch L(\lambda) = e_{\lambda+1} - e_{-\lambda-1}. \]
If we divide, we actually obtain $\ch L(\lambda) = e_{\lambda} + e_{\lambda - 2} + \cdots + e_{- \lambda}$.

In the general case, let $\mu \in \h^*$ and $t \in \R$. Define $F_{\mu, t} \colon \mc{X} \to \R$ by extending $e_{\lambda} \mapsto e^{t(\lambda, \mu)}$ linearly. Applying $F_{\rho, t}$ to the Weyl charaacter formula, we obtain
\begin{align*}
    e^{t(\rho, \rho)} \prod_{\alpha > 0} (1-e^{-t(\rho, \alpha)}) F_{\rho, t} \ch L(\lambda) &= \sum_{w \in W} {(-1)}^{\ell(w)} e^{t(\rho, w(\lambda + \rho))} \\
    &= \sum_{w \in W} {(-1)}^{\ell(w)} e^{t (w^{-1} \rho, \lambda + \rho)} \\
    &= \sum_{w \in W} {(-1)}^{\ell(w)} e^{t(w \rho, \lambda + \rho)} \\
    &= F_{\lambda + \rho, t} \sum_{w \in W} {(-1)}^{\ell(w)} e_{w \rho} \\
    &= F_{\lambda + \rho, t} \qty(e_{\rho} \prod_{\alpha > 0} (1-e_{-\alpha})) \\
    &= e^{t(\rho, \rho + \lambda)} \prod_{\alpha > 0} (1-e^{- t(\alpha, \lambda + \rho)}).
\end{align*}
Note here that $F_{\rho, t}(e_{\lambda} * e_{\mu}) = F_{\rho, t} (e_{\lambda}) \cdot F_{\rho, t} (e_{\mu})$. In the $t \to 0$ limit, we have $F_{\rho, t} \ch L(\lambda) \to \dim L(\lambda)$ and $e^{t(\rho, \rho)} \to 1$. Therefore we obtain
\begin{align*}
    \dim L(\lambda) &= \lim_{t \to 0} \prod_{\alpha > 0} \frac{1-e^{-t(\alpha, \lambda + \rho)}}{1-e^{-t(\alpha, \lambda)}} \\
    &= \frac{\prod_{\alpha > 0} (\alpha, \lambda + \rho)}{\prod_{\alpha > 0} (\alpha, \lambda)}.
\end{align*}
This is called the \textit{Weyl dimension formula}.

\section{Maximal submodules of Verma modules}%
\label{sec:maximal_submodules_of_verma_modules}

\begin{thm}
    Let $\lambda \in \Lambda^+$ and $\alpha_1, \ldots, \alpha_k$ be simple roots of $\g$. Then 
    \[ \sum M(s_{\alpha_i} \circ \lambda) \]
    is the maximal submodule of $M(\lambda)$.
\end{thm}

\begin{rmk}
    Last time we saw that $M(s_{\alpha_i} \circ \lambda) \subset M(\lambda)$.
\end{rmk}

\begin{lem}
    Let $a, b \in \U \g$. Then 
    \[ [a^k, b] = k [a,b] a^{k-1} + \binom{k}{2} [a[a,b]] a^{k-2} + \cdots + [a, \cdots, [a,[a,b]]]. \]
\end{lem}
This is proved by induction, so like a certain Fields medalist, we omit the proof.

If $x_{\alpha}, x_{\beta}$ correspond to roots $\alpha, \beta$, then note that eventually we will have $[x_{\alpha}, \cdots [x_{\alpha}, [x_{\alpha}, x_{\beta}]]] = 0$. In fact, four times is enough. 

\begin{lem}
    Let $\alpha$ be a simple root. Then for any $v \in M(\lambda)$, there exists $N \gg 0$ such that $y_{\alpha}^N \cdot v = 0$ in $M(\lambda) / \sum M(s_{\alpha_i} \circ \lambda)$.
\end{lem}

\begin{proof}
    We proceed by induction. Suppose that $v = y_{i_1} y_{i_2} \cdots y_{i_t} v^+$. When $t = 0$, then $y_{\alpha}^{(\alpha, \lambda) + 1} v^+ = 0$. For $t > 0$, we have
    \begin{align*}
        y_{\alpha}^N y_{i_1} \cdots y_{i_t} v^+ &= y_{i_1} y_{\alpha}^N y_{i_2} \cdots y_{i_t} v^+ + [y_{\alpha}^N, y_{i_1}] y_{i_2} \cdots y_{i_t} v^+.
    \end{align*}
    The first term on the right hand side vanishes by the inductive hypothesis, and the second term becomes $(-) \cdot y_{\alpha}^{N-3} y_{i_2} \cdots y_{i_+} = 0$ by the inductive hypothesis.
\end{proof}

\begin{proof}[Proof of Theorem]
    By the discussion last time and the second lemma, we know that $M(\lambda) / \sum M(s_{\alpha_i} \circ \lambda)$ is finite-dimensional. This implies that $M(\lambda) / \sum M(s_{\lambda_i} \circ \lambda) = L(\lambda) \oplus M'$, but we know that the left hand side is a highest weight module, so $M' = 0$.
\end{proof}

\begin{rmk}
    We have a resolution
    \[ \cdots \to \bigoplus_{\ell(w) = 1} M(w \circ \lambda) \to M(\lambda) \to L(\lambda) \to 0. \]
    This is called the \textit{BGG resolution}.\footnote{This does not imply that the terms are projective.}
\end{rmk}

\chapter{Kevin (Oct 20): Duality and projectives in category $\cO$}%
\label{cha:kevin_oct_20_duality_and_projectives_in_category_co_}

\section{Duality}%
\label{sec:duality}

Recall from the finite-dimensional story that $\g$-representation $M$ have duals $M^{\vee}$ with $\g$-action
\[ (xf)(v) = - f(xv) \]
for $x \in \g, f \in M^{*}, v \in M$. This is not well-behaved for infinite-dimensional representations (for example, $M^{**} \not\cong M$), so in this case we would like to construct a better-behaved duality functor.

Note that every semisimple Lie algebra $\g$ has a transpose $\tau \colon \g \to \g$ (if $\g$ is a matrix Lie algebra, this is literally the transpose) which is an anti-automorphism. Here, we have
\[ \tau(x_{\alpha}) = y_{\alpha}, \qquad \tau(y_{\alpha}) = x_{\alpha}, \qquad, \tau(h_{\alpha}) = h_{\alpha}. \]
This allows us to define\footnote{Kevin is unsure how he is doing on time here.}
\begin{defn}
    Let $M = \bigoplus_{\lambda} M_{\lambda} \in \cO$. Then the \textit{dual} of $M$ is defined by
    \[ M^{\vee} = \bigoplus_{\lambda} M_{\lambda}^{\vee} \qquad (xf)(v) = f(\tau(x)v). \]
\end{defn}

\begin{prop}
    $M^{\vee} \in \cO$.
\end{prop}

\begin{proof}
    To prove finite generation, note that $M^{\vee}$ has finite length (here, ${L(\lambda)}^{\vee} = L(\lambda)$ because duality preserves formal characters and exchanges quotients and submodules). Clearly the weight spaces are finite-dimensional by assumption, and the weights lie in some union $\bigcup \lambda - \Lambda$ because formal characters are preserved, so we have local $\n$-finiteness.
\end{proof}

Here are some more facts about duality.
\begin{itemize}
    \item Duality is a contravariant functor. This is obvious because everything is defined on the level of weight spaces.
    \item There is a natural isomorphism $M^{\vee \vee} \cong M$. This is clear because we are taking double duals of finite-dimensional things and adding them up, so in particular duality is an anti-equivalence of categories.
    \item We have ${L(\lambda)}^{\vee} \cong L(\lambda)$. On the other hand, duality for $M(\lambda)$ is complicated.
    \item $\tau$ fixes $Z(\g)$ by an exercise in Humphreys.\footnote{Professor Humphreys, I hope you don't descend upon us from heaven for not having done this exercise. Also please forgive me (the note taker) for never interacting with you when I was an undergrad.} In particular, this means that ${(M^{\chi})}^{\vee} = {(M^{\vee})}^{\chi}$.
\end{itemize}

\section{Projectives}%
\label{sec:projectives}

Recall that $P$ is projective if $\Hom(P,-)$ is right exact.\footnote{I (note taker) considered not putting this definition in the notes.} Our goal is to prove that $\cO$ has enough projectives (which will mean that we can do homological algebra). The first thing we will do is introduce dominance and antidominance.

Recall that for $\lambda \in \Lambda$, $W \lambda$ contains one dominant weight and one antidominant weight. This gives us two(!) good choices for representatives of $W \lambda$. Unfortunately, we care about nonintegral weights,\footnote{Apparently Kevin is speaking for us all here.} and we cannot choose representatives of $W \lambda$ for general $\lambda \in \h^*$.

\begin{rmk}
    From now on we will use the $w \circ -$ action (because this is all we care about), and therefore our new notion of (anti)dominance will not restrict to the old notion of dominance.\footnote{Said old definition has now been Stalined. Unfortunately, Humphreys just ignores the ambiguity.}
\end{rmk}

\begin{defn}
    A weight $\lambda \in \h^*$ is \textit{dominant} if $\ev{\lambda + \rho, \alpha^{\vee}} \notin \Z_{< 0}$ for all $\alpha \in \Phi^+$. A weight $\lambda \in \h^*$ is \textit{antidominant} if $\ev{\lambda + \rho, \alpha^{\vee}} \notin \Z_{>0}$ for all $\alpha \in \Phi^+$.
\end{defn}

Note that this is \textbf{not} the same as the undotted definition. For example, $-\rho$ is dominant. Also, the set $W \circ \lambda$ can have multiple dominant and/or antidominant weights.

\begin{defn}
    We define the subgroup
    \[ W_{[\lambda]} \coloneqq \qty{w \in W \mid w \circ \lambda - \lambda \in \Lambda_r}, \]
    where $\Lambda_r$ is the root lattice. We also define
    \[ \Phi_{[\lambda]} \coloneqq \qty{\alpha \in \Phi \mid \ev{\lambda, \alpha^{\vee}} \in \Z}. \]
    In fact, $W_{[\lambda]}$ is the Weyl group of $\Phi_{[\lambda]}$. We may similarly define $\Delta_{[\lambda]}$.
\end{defn}

\begin{prop}
    The following are equivalent:
    \begin{enumerate}
        \item $\lambda$ is dominant.
        \item $\ev{\lambda + \rho, \alpha^{\vee}} \geq 0$ for all $\alpha \in \Delta_{[\lambda]}$.
        \item $\lambda \geq s_{\alpha} \circ \lambda$ for all $\alpha \in \Delta_{[\lambda]}$.
        \item $\lambda \geq w \circ \lambda$ for all $w \in W_{[\lambda]}$.
    \end{enumerate}
\end{prop}

\begin{proof}
    Clearly 1 implies 2, and 2 implies 1 because positive roots are sums of simple roots with nonnegative coefficients. To prove that 2 is equivalent to 3, note that
    \[ s_{\alpha} \circ \lambda = \lambda - \ev{\lambda + \rho, \alpha^{\vee}} \alpha. \]
    Finally, to see that 3 is equivalent to 4, note that 4 implies 3 automatically. To prove that 3 implies 4, we induct on $\ell(w)$. If $w = w' s_{\alpha}$ with $\ell(w') = \ell(w) - 1$, we see that
    \[ \lambda - w \circ \lambda = (\lambda - w' \circ \lambda) + w' \circ (\lambda - s_{\alpha} \circ \lambda). \]
    It is clear that $\lambda - w' \circ \lambda \geq 0$ while $\lambda - s_{\alpha} \circ \lambda$ is a nonnegative multiple of $\alpha$. Because of the length condition, we see that $w' \circ \alpha$ is positive.
\end{proof}

\begin{cor}
    The orbit $W_{[\lambda]} \circ \lambda$ has a unique (anti)dominant weight.
\end{cor}

\begin{proof}
    This is because 1 is equivalent to 4 in the proposition.
\end{proof}

\begin{thm}\leavevmode
    \begin{enumerate}
        \item If $\lambda$ is dominant, then $M(\lambda)$ is projective.
        \item If $P \in \cO$ is projective and $L \in \cO$ is finite-dimensional, then $P \otimes L$ is projective.
        \item $\cO$ has enough projectives.
    \end{enumerate}
\end{thm}

\begin{proof}\leavevmode
    \begin{enumerate}
        \item Consider $M \twoheadrightarrow N$ and suppose $v \in N$ is a maximal weight vector with weight $\lambda$ (coming from a map $M(\lambda) \to N$). Assume that $M = M^{\chi}, N = N^{\chi}$. Our goal is to lift $v$ to a maximal weight vector in $M$, but because $M \to N$ is surjective, we can lift $v$ to $v' \in M_{\lambda}$. If $v'$ is maximal, then we are done, so suppose that $v'$ is not maximal.

            In this case, there exists $x \in \U \n$ such that $xv'$ is a maximal vector with weight greater than $\lambda$. However, this weight must be linked to $\lambda$, so by dominance of $\lambda$, it cannot exist.
        \item Here, we use the tensor-Hom adjunction
            \[ \Hom_{\cO}(P \otimes L, M) \cong \Hom_{\cO}(P, L^* \otimes M). \]
            Because $L^* \otimes -$ is exact and $\Hom_{\cO}(P, -)$ is exact, the functor $\Hom_{\cO}(P \otimes L, -)$ is exact and thus $P \otimes L$ is projective.\footnote{Apparently Stalinization is an invertible operation, although to be fair it is unclear what the history of the USSR says about this.}
        \item The first thing we want to do is to find projectives mapping onto $L(\lambda)$. For large $n$, $\lambda + n \rho$ is dominant. This implies that $M(\lambda + n \rho)$ is projective, but then $M(\lambda + r \rho) \otimes L(n \rho)$ is projective.

            In fact, there exists a surjection $M(\lambda + n \rho) \twoheadrightarrow L(n\rho)$. To see this, if $M$ is a $\U \g$-module and $L$ is a $\U \b$-module, then
            \[ (\U \g \otimes_{\U \b} L) \otimes M \cong \U \g \otimes_{\U \b} (L \otimes M). \]
            This is known as the tensor identity and is apparently not obvious unless you have the arrogance level of a certain Chinese mathematician. Because $M(\lambda + n\rho) = \U \g \otimes_{\U\b} \C_{\lambda + n\rho}$, we obtain
            \[ M(\lambda + n \rho) \otimes L(n \rho) \cong \U \g \otimes_{\U \b} (\C_{\lambda + n\rho} \otimes L(n \rho)) \twoheadrightarrow \U \g \otimes_{\U\b} \C_{\lambda} \cong M(\lambda). \]
            The surjection comes from the fact that the lowest weight of $L(n \rho)$ is $-n\rho$, so we can kill all of the higher weights.

            The rest of the proof is simply homological algebra. For a general $M \in \cO$, because $\cO$ is Artinian, we can induct on the length of $M$. First consider a short exact sequence
            \[ 0 \to L(\lambda) \to M \to N \to 0. \]
            By assumption, there exists a surjection $P \twoheadrightarrow N$, and this morphism lifts to $P \to M$. If $P$ does not surject onto $M$, then $\Im(P \to M)$ cannot intersect $L(\lambda)$ (otherwise it would contain all of $L(\lambda)$ and thus surject onto $M$). This implies that $\Im (P \to M) \cong N$, which splits the exact sequence. \qedhere
    \end{enumerate}
\end{proof}

By standard homological algebra, because $\cO$ is Artinian and has enough projectives, then $\cO$ has projective covers (i.e. unique minimal projectives surjecting onto $M$). If we define $P(\lambda)$ to be the projective cover of $L(\lambda)$, the $P(\lambda)$ are precisely the indecomposable projectives. Therefore every projective is a direct sum of $P(\lambda)$.

\begin{thm}\leavevmode
    \begin{enumerate}
        \item $P(\lambda)$ has a standard filtration, which is a filtration with subquotients that are Verma modules. 
        \item (BGG reciprocity) The multiplicity of $M(\mu)$ in the composition series for $P(\lambda)$ is given by 
            \[ (P(\lambda):M(\mu)) = [M(\mu):L(\lambda)]. \]
    \end{enumerate}
\end{thm}



\end{document}
