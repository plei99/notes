\documentclass[leqno, openany]{memoir}
\setulmarginsandblock{3.5cm}{3.5cm}{*}
\setlrmarginsandblock{3cm}{3.5cm}{*}
\checkandfixthelayout

\usepackage{amsmath}
\usepackage{amssymb}
\usepackage{amsthm}
%\usepackage{MnSymbol}
\usepackage{bm}
\usepackage{accents}
\usepackage{mathtools}
\usepackage{tikz}
\usetikzlibrary{decorations.pathmorphing,shapes}
\usetikzlibrary{calc}
\usetikzlibrary{automata,positioning}
\usepackage{tikz-cd}
\usepackage{forest}
\usepackage{braket} 
\usepackage{listings}
\usepackage{mdframed}
\usepackage{verbatim}
\usepackage{physics}
\usepackage{stmaryrd}
\usepackage{mathrsfs} 
\usepackage{stackengine} 
%\usepackage{/home/patrickl/homework/macaulay2}

%font
\usepackage[sc]{mathpazo}
\usepackage{eulervm}
\usepackage[scaled=0.86]{berasans}
\usepackage{inconsolata}
\usepackage{microtype}

%CS packages
\usepackage{algorithmicx}
\usepackage{algpseudocode}
\usepackage{algorithm}

% typeset and bib
\usepackage[english]{babel} 
\usepackage[utf8]{inputenc} 
\usepackage[T1]{fontenc}
\usepackage[backend=biber, style=alphabetic]{biblatex}
\usepackage[bookmarks, colorlinks, breaklinks]{hyperref} 
\hypersetup{linkcolor=black,citecolor=black,filecolor=black,urlcolor=black}

% other formatting packages
\usepackage{float}
\usepackage{booktabs}
\usepackage[shortlabels]{enumitem}
\usepackage{csquotes}
\usepackage{titlesec}
\usepackage{titling}
\usepackage{fancyhdr}
\usepackage{lastpage}
\usepackage{parskip}
\usepackage{graphicx}
\graphicspath{{./images/}}

\usepackage{lipsum}

% delimiters
\DeclarePairedDelimiter{\gen}{\langle}{\rangle}
\DeclarePairedDelimiter{\floor}{\lfloor}{\rfloor}
\DeclarePairedDelimiter{\ceil}{\lceil}{\rceil}


\newtheorem{thm}{Theorem}[section]
\newtheorem{cor}[thm]{Corollary}
\newtheorem{prop}[thm]{Proposition}
\newtheorem{lem}[thm]{Lemma}
\newtheorem{conj}[thm]{Conjecture}
\newtheorem{quest}[thm]{Question}

\theoremstyle{definition}
\newtheorem{defn}[thm]{Definition}
\newtheorem{defns}[thm]{Definitions}
\newtheorem{con}[thm]{Construction}
\newtheorem{exm}[thm]{Example}
\newtheorem{exms}[thm]{Examples}
\newtheorem{notn}[thm]{Notation}
\newtheorem{notns}[thm]{Notations}
\newtheorem{addm}[thm]{Addendum}
\newtheorem{exer}[thm]{Exercise}

\theoremstyle{remark}
\newtheorem{rmk}[thm]{Remark}
\newtheorem{rmks}[thm]{Remarks}
\newtheorem{warn}[thm]{Warning}
\newtheorem{sch}[thm]{Scholium}


% unnumbered theorems
\theoremstyle{plain}
\newtheorem*{thm*}{Theorem}
\newtheorem*{prop*}{Proposition}
\newtheorem*{lem*}{Lemma}
\newtheorem*{cor*}{Corollary}
\newtheorem*{conj*}{Conjecture}

% unnumbered definitions
\theoremstyle{definition}
\newtheorem*{defn*}{Definition}
\newtheorem*{exer*}{Exercise}
\newtheorem*{defns*}{Definitions}
\newtheorem*{con*}{Construction}
\newtheorem*{exm*}{Example}
\newtheorem*{exms*}{Examples}
\newtheorem*{notn*}{Notation}
\newtheorem*{notns*}{Notations}
\newtheorem*{addm*}{Addendum}


\theoremstyle{remark}
\newtheorem*{rmk*}{Remark}

% shortcuts
\newcommand{\Ima}{\mathrm{Im}}
\newcommand{\A}{\mathbb{A}}
\newcommand{\G}{\mathbb{G}}
\newcommand{\N}{\mathbb{N}}
\newcommand{\R}{\mathbb{R}}
\newcommand{\C}{\mathbb{C}}
\newcommand{\Z}{\mathbb{Z}}
\newcommand{\Q}{\mathbb{Q}}
\renewcommand{\k}{\Bbbk}
\renewcommand{\P}{\mathbb{P}}
\newcommand{\M}{\overline{M}}
\newcommand{\g}{\mathfrak{g}}
\newcommand{\h}{\mathfrak{h}}
\newcommand{\n}{\mathfrak{n}}
\renewcommand{\b}{\mathfrak{b}}
\newcommand{\ep}{\varepsilon}
\newcommand*{\dt}[1]{%
   \accentset{\mbox{\Huge\bfseries .}}{#1}}
\renewcommand{\abstractname}{Official Description}
\newcommand{\mc}[1]{\mathcal{#1}}
\newcommand{\T}{\mathbb{T}}
\newcommand{\mf}[1]{\mathfrak{#1}}
\newcommand{\mr}[1]{\mathrm{#1}}
\newcommand{\ms}[1]{\mathsf{#1}}
\newcommand{\ol}[1]{\overline{#1}}
\newcommand{\ul}[1]{\underline{#1}}
\newcommand{\wt}[1]{\widetilde{#1}}
\newcommand{\wh}[1]{\widehat{#1}}
\renewcommand{\div}{\operatorname{div}}
\newcommand{\Sm}{\mathsf{Sm}}
\newcommand{\Cor}{\mathsf{Cor}}

\DeclareMathOperator{\Der}{Der}
\DeclareMathOperator{\Hom}{Hom}
\DeclareMathOperator{\End}{End}
\DeclareMathOperator{\ad}{ad}
\DeclareMathOperator{\Aut}{Aut}
\DeclareMathOperator{\Rad}{Rad}
\DeclareMathOperator{\Pic}{Pic}
\DeclareMathOperator{\supp}{supp}
\DeclareMathOperator{\Supp}{Supp}
\DeclareMathOperator{\sgn}{sgn}
\DeclareMathOperator{\spec}{Spec}
\DeclareMathOperator{\rk}{rk}
\DeclareMathOperator{\Spec}{Spec}
\DeclareMathOperator{\proj}{Proj}
\DeclareMathOperator{\Proj}{Proj}
\DeclareMathOperator{\ord}{ord}
\DeclareMathOperator{\Div}{Div}
\DeclareMathOperator{\Bl}{Bl}
\DeclareMathOperator{\ch}{ch}
\DeclareMathOperator{\td}{td}
\DeclareMathOperator{\Tor}{Tor}
\DeclareMathOperator{\depth}{depth}
\DeclareMathOperator{\CH}{CH}
\DeclareMathOperator{\Ob}{Ob}
\DeclareMathOperator{\Rat}{Rat} 
\DeclareMathOperator{\coker}{coker}
\DeclareMathOperator{\Hilb}{Hilb}
\DeclareMathOperator{\Sym}{Sym}

% Section formatting
\titleformat{\section}
    {\Large\sffamily\scshape\bfseries}{\thesection}{1em}{}
\titleformat{\subsection}[runin]
    {\large\sffamily\bfseries}{\thesubsection}{1em}{}
\titleformat{\subsubsection}[runin]{\normalfont\itshape}{\thesubsubsection}{1em}{}

\title{COURSE TITLE}
\author{Lectures by INSTRUCTOR, Notes by NOTETAKER}
\date{SEMESTER}

\newcommand*{\titleSW}
    {\begingroup% Story of Writing
    \raggedleft
    \vspace*{\baselineskip}
    {\Huge\itshape Category O Learning Seminar \\ Fall 2021}\\[\baselineskip]
    {\large\itshape Notes by Patrick Lei}\\[0.2\textheight]
    {\Large Lectures by Various}\par
    \vfill
    {\Large \sffamily Columbia University}
    \vspace*{\baselineskip}
\endgroup}
\pagestyle{simple}

\chapterstyle{ell}


%\renewcommand{\cftchapterpagefont}{}
\renewcommand\cftchapterfont{\sffamily}
\renewcommand\cftsectionfont{\scshape}
\renewcommand*{\cftchapterleader}{}
\renewcommand*{\cftsectionleader}{}
\renewcommand*{\cftsubsectionleader}{}
\renewcommand*{\cftchapterformatpnum}[1]{~\textbullet~#1}
\renewcommand*{\cftsectionformatpnum}[1]{~\textbullet~#1}
\renewcommand*{\cftsubsectionformatpnum}[1]{~\textbullet~#1}
\renewcommand{\cftchapterafterpnum}{\cftparfillskip}
\renewcommand{\cftsectionafterpnum}{\cftparfillskip}
\renewcommand{\cftsubsectionafterpnum}{\cftparfillskip}
\setrmarg{3.55em plus 1fil}
\setsecnumdepth{subsection}
\maxsecnumdepth{subsection}
\settocdepth{subsection}

\begin{document}
    
\begin{titlingpage}
\titleSW
\end{titlingpage}

\thispagestyle{empty}
\section*{Disclaimer}%
\label{sec:disclaimer}

These notes were taken during the seminar using the \texttt{vimtex} package of
the editor \texttt{neovim}.  Any errors are mine and not the speakers'.  In
addition, my notes are picture-free (but will include commutative diagrams) and
are a mix of my mathematical style and that of the lecturers.  If you find any
errors, please contact me at \texttt{plei@math.columbia.edu}.

\vspace*{1cm}

\noindent\textbf{Seminar Website:}\\
\url{https://math.columbia.edu/~plei/f21-CO.html} \newpage

\tableofcontents

\chapter{Kevin (Sep 29): Review of semisimple Lie algebras and introduction to category $\mc{O}$}%
\label{cha:kevin_sep_29_review_of_semisimple_lie_algebras_and_introduction_to_category_o}

\section{Review of semisimple Lie algebras}%
\label{sec:review_of_semisimple_lie_algebras}

Throughout this lecture, we will work over $\C$. 

\begin{defn}
    A Lie algebra $\mf{g}$ is \textit{semisimple} if any of the following equivalent conditions hold:
    \begin{enumerate}
        \item $\mf{g}$ is a direct sum of simple Lie algebras (those with no nonzero proper ideals).
        \item The Killing form $\kappa(x,y) \coloneqq \tr (\ad(x) \ad(y))$ is nondegenerate.
        \item The radical (maximal solvable ideal) of $\mf{g}$ is zero.
    \end{enumerate}
\end{defn}

Some examples of semisimple Lie algebras include $\mf{sl}_n, \mf{so}_n, \mf{sp}_{2n}$, and in some sense (the classification of simple Lie algebras), these are essentially all semisimple Lie algebras.

Now given a semisimple Lie algebra $\mf{g}$, we will fix a \textit{Cartan subalgebra} $\mf{h} \subset \mf{g}$, which is just a maximal abelian subalgebra of semisimple elements. This gives us a root decomposition
\[ \mf{g} = \mf{h} \oplus \bigoplus_{\alpha \in \mf{h}^* \setminus \qty{0}} \mf{g}_{\alpha}, \]
where $\mf{g}_{\alpha}$ is the subspace of $\mf{g}$ where $\mf{h}$ acts with weight $\alpha$. Some important facts about these root systems are the following:
\begin{itemize}
    \item For all $\alpha$, we have $\dim \mf{g}_{\alpha} = 1$.
    \item For all roots $\alpha, \beta$, we have $[\mf{g}_{\alpha}, \mf{g}_{\beta}] \subset \mf{g}_{\alpha + \beta}$.
    \item If $\alpha$ is a root, so is $-\alpha$.
\end{itemize}
In addition, the $\alpha$ are required to form a (reduced) \textit{root system} (denoted $\Phi$), the precise definition of which is deliberately omitted. Given a choice of Borel subalgebra containing $\mf{h}$, we obtain a set $\Phi^+$ of positive roots and a set $\Delta$ of simple roots. In addition, given a root system $\Phi$, there is a \textit{dual root system} $\Phi^{\vee}$, whose roots are
\[ \alpha^{\vee} = \frac{2 \alpha}{(\alpha, \alpha)}, \alpha \in \Phi. \]

Now suppose that $\mf{g}$ is a semisimple Lie algebra with root system $\Phi$. For every $\alpha \in \Phi^+$, we may choose $x_{\alpha} \in \mf{g}_{\alpha}$ and $y_{\alpha} \in \mf{g}_{-\alpha}$, and these determine some $h_{\alpha} = [x_{\alpha}, y_{\alpha}] \in \mf{h}$. This choice can be made such that $\alpha(h_{\alpha}) = 2$.

Recall that the Lie algebra $\mf{sl}_2$ is spanned by the matrices
\[ x = \mqty(0 & 1 \\ 0 & 0), \qquad y = \mqty(0 & 0 \\ 1 & 0), \qquad h = \mqty(\dmat[0]{1,-1}). \]
Then the choice of $x_{\alpha}, y_{\alpha}, h_{\alpha}$ gives an embedding $\mf{sl}_2 \to \mf{g}$. These maps, ranging over all $\alpha$, cover all of $\mf{g}$. Now a basis of $\mf{g}$ is given by $x_{\alpha}, y_{\alpha}, \alpha \in \Phi$ and $h_{\alpha_i}$ for the \textbf{simple} roots $\alpha_i$. Therefore, to specify $\mf{g}$, we only need to give commutation relations for the basis elements.

Now suppose that $\Phi$ is some root system. We would like to construct a semisimple Lie algebra $\mf{g}$ with root system $\Phi$. We want to build a semisimple Lie algebra. To do this, choose a set of simple roots $\alpha_i$, and consider the Lie algebra
\[ \ev{x_{\alpha_i}, y_{\alpha_i}, h_{\alpha_i}} / \text{relations}, \]
where the relations are as follows:
\begin{itemize}
    \item $[h_{\alpha_i}, h_{\alpha_j}] = 0$.
    \item We have $[x_{\alpha_i}, y_{\alpha_j}] = h_{\alpha_i}$ if $i = j$ and this commutator vanishes otherwise.
    \item $[h_{\alpha_i}, x_{\alpha_j}] = \ev{\alpha_j, \alpha_i^{\vee}} x_{\alpha_j}$.
    \item $[h_{\alpha_i}, y_{\alpha_j}] = - \ev{\alpha_j, \alpha_i^{\vee}} y_{\alpha_j}$.
    \item ${\ad(x_{\alpha_i})}^{1-\ev{\alpha_j, \alpha_i^{\vee}}}(x_{\alpha_j}) = 0$ if $i \neq j$.
    \item ${\ad(y_{\alpha_i})}^{1-\ev{\alpha_j, \alpha_i^{\vee}}}(y_{\alpha_j}) = 0$ if $i \neq j$.
\end{itemize}
The first four relations are called the \textit{Weyl relations} and the last two are called the \textit{Serre relations}. Given this data, we end up with a semisimple Lie algebra $\mf{g}_{\Phi}$ with root system $\Phi$. In addition, if $\mf{g}$ is any other semisimple Lie algebra with root system $\Phi$, there is an isomorphism $\mf{g}_{\Phi} \xrightarrow{\sim} \mf{g}$. Moreover, we have a bijection between semisimple Lie algebras and reduced root systems, which restricts to a bijection between simple Lie algebras and irreducible root systems.

\begin{table}[H]
    \centering
    \caption{Root systems and Lie algebras}
    \label{tab:classification}
    \begin{tabular}{cc}
    \toprule
    Irreducible root systems & simple Lie algebras \\
    \midrule
    $A_n$ & $\mf{sl}_{n+1}$ \\
    $B_n$ & $\mf{so}_{2n+1}$ \\
    $C_n$ & $\mf{sp}_{2n}$ \\
    $D_n$ & $\mf{so}_{2n}$ \\
    $E_6, E_7, E_8, F_4, G_2$ & exceptional Lie algebras \\
    \bottomrule
    \end{tabular}
\end{table}

We will now discuss the finite-dimensional representation theory of semisimple Lie algebras $\mf{g}$.

\begin{thm}[Weyl's complete reducibility theorem]
    Any finite-dimensional representation of $\mf{g}$ is decomposes as a direct sum of simple representations.
\end{thm}

Now suppose that $M$ is a finite-dimensional $\mf{g}$-representation. Then $M$ has a weight decomposition
\[ M = \bigoplus_{\lambda \in \mf{h}^*} M_{\lambda}. \]
These $\lambda$ are \textit{integral weights}, which simply means that $\ev{\lambda, \alpha^{\vee}} \in \Z$ for all roots $\alpha$. For any root $\alpha$, $x_{\alpha}(M_{\lambda}) \subset M_{\lambda + \alpha}$ and $y_{\alpha}(M_{\lambda}) \subset M_{\lambda - \alpha}$. We would like to think that the $x_{\alpha}$ raise the weights and $y_{\alpha}$ lower the weights, so we introduce a partial order. We say that $\lambda \geq \mu$ if $\lambda - \mu \in \Z_{\geq 0} \Phi^+$.

By Weyl's complete reducibility theorem, it remains to classify the irreducible representations of $\mf{g}$. These are in bijection with the \textit{dominant} integral weights, which in particular means that $\ev{\lambda, \alpha^{\vee}} \geq 0$ for all $\alpha \in \Phi^+$. For any dominant weight $\lambda$, there is a unique highest-weight representation $L(\lambda)$. Here, $L(\lambda)$ is generated by a single \textit{maximal vector} $v$ of weight $\lambda$. This means that for all positive roots $\alpha$, $x_{\alpha} v = 0$.

\section{Introduction to category $\mc{O}$}%
\label{sec:introduction_to_category_o_}

We would now like to study infinite dimensional representations of $\mf{g}$. Of course, this is impossibly complicated in general, so we will impose some finiteness conditions on our representations.

\begin{defn}
    The category $\mc{O}$ is the full subcategory of $U(\mf{g})$-modules $M$ satisfying:
    \begin{enumerate}
        \item $M$ is finitely generated as a $U(\mf{g})$-module.
        \item $M$ is $\mf{h}$-semisimple and has a weight decomposition $M = \bigoplus_{\lambda \in \mf{h}^*} M_{\lambda}$.
        \item $M$ is locally $\mf{n}$-finite, where $\mf{n} = \bigoplus_{\alpha \in \Phi^+} \mf{g_{\alpha}}$. Precisely, this means that the $U(\mf{n})$ generated by any $v \in M$ is finite-dimensional.
    \end{enumerate}
\end{defn}

Here are some facts about category $\mc{O}$, which are stated without proof.
\begin{itemize}
    \item For all $M$ in our category and weights $\lambda$, the weight space $M_{\lambda}$ is finite-dimensional.
    \item $\mc{O}$ is a Noetherian (everything satisfies the descending chain condition) abelian category.
\end{itemize}

We will now describe some infinite-dimensional objects in category $\mc{O}$.

\begin{defn}
    For any weight $\lambda$, the \textit{Verma module} $M(\lambda)$ associated to $\lambda$ is the module
    \[ M(\lambda) = U(\mf{g}) \otimes_{U(\mf{b})} \C_{\lambda}, \]
    where $\mf{b} = \mf{h} + \mf{n}$ is the Borel subalgebra associated to our choice of positive roots and $\C_{\lambda}$ is the $\mf{b}$-module associated to the $1$-dimensional representation of $\mf{h}$ with weight $\lambda$ and the identification $\mf{b} / \mf{n} = \mf{h}$.
\end{defn}







\end{document}
