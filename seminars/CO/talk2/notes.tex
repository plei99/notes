\documentclass{amsart}
\usepackage{amsmath}
\usepackage{amssymb}
\usepackage{amsthm}
%\usepackage{MnSymbol}
\usepackage{bm}
\usepackage{accents}
\usepackage{mathtools}
\usepackage{tikz}
\usetikzlibrary{calc}
\usetikzlibrary{decorations.pathmorphing,shapes}
\usetikzlibrary{automata,positioning}
\usepackage{tikz-cd}
\usepackage{forest}
\usepackage{braket} 
\usepackage{listings}
\usepackage{mdframed}
\usepackage{verbatim}
\usepackage{physics}
\usepackage{stmaryrd}
\usepackage{mathrsfs} 
\usepackage{stackengine} 
%\usepackage{/home/patrickl/homework/macaulay2}

%font
\usepackage[sc]{mathpazo}
\usepackage{eulervm}
\usepackage[scaled=0.86]{berasans}
\usepackage{inconsolata}
\usepackage{microtype}

%CS packages
\usepackage{algorithmicx}
\usepackage{algpseudocode}
\usepackage{algorithm}

% typeset and bib
\usepackage[english]{babel} 
\usepackage[utf8]{inputenc} 
\usepackage[T1]{fontenc}
%\usepackage[backend=biber, style=alphabetic]{biblatex}
\usepackage[bookmarks, colorlinks, breaklinks]{hyperref} 
\hypersetup{linkcolor=black,citecolor=black,filecolor=black,urlcolor=black}
\usepackage{graphicx}
\graphicspath{{./}}

% other formatting packages
\usepackage{float}
\usepackage{booktabs}
\usepackage[shortlabels]{enumitem}
\usepackage{csquotes}
%\usepackage{titlesec}
%\usepackage{titling}
%\usepackage{fancyhdr}
%\usepackage{lastpage}
\usepackage{parskip}

\usepackage{lipsum}

% delimiters
\DeclarePairedDelimiter{\gen}{\langle}{\rangle}
\DeclarePairedDelimiter{\floor}{\lfloor}{\rfloor}
\DeclarePairedDelimiter{\ceil}{\lceil}{\rceil}


\newtheorem{thm}{Theorem}[section]
\newtheorem{cor}[thm]{Corollary}
\newtheorem{prop}[thm]{Proposition}
\newtheorem{lem}[thm]{Lemma}
\newtheorem{conj}[thm]{Conjecture}
\newtheorem{quest}[thm]{Question}

\theoremstyle{definition}
\newtheorem{defn}[thm]{Definition}
\newtheorem{defns}[thm]{Definitions}
\newtheorem{con}[thm]{Construction}
\newtheorem{exm}[thm]{Example}
\newtheorem{exms}[thm]{Examples}
\newtheorem{notn}[thm]{Notation}
\newtheorem{notns}[thm]{Notations}
\newtheorem{addm}[thm]{Addendum}
\newtheorem{exer}[thm]{Exercise}

\theoremstyle{remark}
\newtheorem{rmk}[thm]{Remark}
\newtheorem{rmks}[thm]{Remarks}
\newtheorem{warn}[thm]{Warning}
\newtheorem{sch}[thm]{Scholium}


% unnumbered theorems
\theoremstyle{plain}
\newtheorem*{thm*}{Theorem}
\newtheorem*{prop*}{Proposition}
\newtheorem*{lem*}{Lemma}
\newtheorem*{cor*}{Corollary}
\newtheorem*{conj*}{Conjecture}

% unnumbered definitions
\theoremstyle{definition}
\newtheorem*{defn*}{Definition}
\newtheorem*{exer*}{Exercise}
\newtheorem*{defns*}{Definitions}
\newtheorem*{con*}{Construction}
\newtheorem*{exm*}{Example}
\newtheorem*{exms*}{Examples}
\newtheorem*{notn*}{Notation}
\newtheorem*{notns*}{Notations}
\newtheorem*{addm*}{Addendum}


\theoremstyle{remark}
\newtheorem*{rmk*}{Remark}

% shortcuts
\newcommand{\Ima}{\mathrm{Im}}
\newcommand{\A}{\mathbb{A}}
\newcommand{\G}{\mathbb{G}}
\newcommand{\N}{\mathbb{N}}
\newcommand{\R}{\mathbb{R}}
\newcommand{\C}{\mathbb{C}}
\newcommand{\Z}{\mathbb{Z}}
\newcommand{\cO}{\mathcal{O}}
\newcommand{\U}{\mathcal{U}}
\newcommand{\Q}{\mathbb{Q}}
\renewcommand{\k}{\Bbbk}
\renewcommand{\P}{\mathbb{P}}
\newcommand{\M}{\overline{M}}
\newcommand{\g}{\mathfrak{g}}
\newcommand{\h}{\mathfrak{h}}
\newcommand{\n}{\mathfrak{n}}
\renewcommand{\b}{\mathfrak{b}}
\newcommand{\ep}{\varepsilon}
\newcommand*{\dt}[1]{%
   \accentset{\mbox{\Huge\bfseries .}}{#1}}
%\renewcommand{\abstractname}{Official Description}
\newcommand{\mc}[1]{\mathcal{#1}}
% \newcommand{\msc}[1]{\mathscr{#1}}
\newcommand{\T}{\mathbb{T}}
\newcommand{\mf}[1]{\mathfrak{#1}}
\newcommand{\mr}[1]{\mathrm{#1}}
\newcommand{\ms}[1]{\mathsf{#1}}
\newcommand{\ol}[1]{\overline{#1}}
\newcommand{\ul}[1]{\underline{#1}}
\newcommand{\wt}[1]{\widetilde{#1}}
\newcommand{\wh}[1]{\widehat{#1}}
\renewcommand{\div}{\operatorname{div}}

\DeclareMathOperator{\Der}{Der}
\DeclareMathOperator{\Tor}{Tor}
\DeclareMathOperator{\Hom}{Hom}
\DeclareMathOperator{\End}{End}
\DeclareMathOperator{\Ext}{Ext}
\DeclareMathOperator{\ext}{ext}
\DeclareMathOperator{\ad}{ad}
\DeclareMathOperator{\Aut}{Aut}
\DeclareMathOperator{\Rad}{Rad}
\DeclareMathOperator{\Pic}{Pic}
\DeclareMathOperator{\supp}{supp}
\DeclareMathOperator{\Supp}{Supp}
\DeclareMathOperator{\depth}{depth}
\DeclareMathOperator{\sgn}{sgn}
\DeclareMathOperator{\spec}{Spec}
\DeclareMathOperator{\Spec}{Spec}
\DeclareMathOperator{\proj}{Proj}
\DeclareMathOperator{\Proj}{Proj}
\DeclareMathOperator{\ord}{ord}
\DeclareMathOperator{\Div}{Div}
\DeclareMathOperator{\Bl}{Bl}
\DeclareMathOperator{\ch}{ch}

\title{Koszul duality for people who aren't Peter May}
\author{Patrick Lei}
\date{December 8, 2021}

\begin{document}
    
\maketitle

\begin{abstract}
    I will tell you what a Koszul algebra is and what Koszul duality is, and then I will tell you how this all relates to category $\cO$. Note: no operads were harmed in the making of this talk.
\end{abstract}

Before we begin, we will establish some notation.
\begin{notns}
    Let $A = \bigoplus A_j$ be a graded ring. We will denote by $A\text{-}\ms{Mod}$ the category of left $A$-modules and $A\text{-}\ms{mod}$ the category of graded $A$-modules. These carry functors $\Hom, \Ext, \ldots$ and $\hom, \ext, \ldots$, respectively. Also, the subcategories of finitely-generated modules will be called $A\text{-}\ms{Mof}$ and $A\text{-}\ms{mof}$. Finally, we will denote the grading shift $\ev{n}$ by $(M\ev{n})_i = M_{i-n}$. Finally, write $k \coloneqq A_0$. Here, we will assume that $k$ is a field, but after certain adjustments we may allow $k$ to be a noncommutative semisimple ring.
\end{notns}

\section{Koszul duality}%
\label{sec:koszul_duality}

\begin{defn}
    A graded ring $A = \bigoplus_{j \geq 0} A_j$ is a \textit{Koszul ring} if $A_0$ is semisimple and admits a graded projective resolution
    \[ \cdots \to P^2 \to P^1 \to P^0 \to A_0 \]
    such that $P^i$ is generated by its degree $i$ component $P^i_i$.
\end{defn}

\begin{exm}
    The Koszul resolution
    \[ S^{\bullet} V \otimes \Lambda^{\bullet} V = \cdots \to S^{\bullet} V \otimes \Lambda^2 V \to S^{\bullet} V \otimes V \to S^{\bullet} V \to k \]
    shows that $S^{\bullet} V$ is a Koszul ring.
\end{exm}

There is an alternative characterization of Koszul rings.
\begin{defn}
    Let $M$ be a graded module. Then $M$ is \textit{pure of weight $n$} if and only if $M = M_{-n}$.
\end{defn}
In the cases we will consider, any simple $A$-module is pure and any pure module is semisimple.
\begin{prop}
    Let $A$ be a graded ring with $A_0$ semisimple. Then the following are equivalent:
    \begin{enumerate}
        \item $A$ is Koszul.
        \item For any pure $A$-modules $M, N$ of weights $m, n$ we have $\ext^i_A(M, N) = 0$ whenever $i \neq m-n$.
        \item We have $\ext_A^i(A_0, A_0\ev{n}) = 0$ whenever $i \neq n$.
    \end{enumerate}
\end{prop}

We will list some more properties of Koszul rings:
\begin{prop}
    If $A$ is a Koszul ring, so is the opposite ring $A^{\mr{op}}$.
\end{prop}

\begin{proof}
    Define $M^{\odot}$ by $M^{\odot}_i = M_{-i}^*$. Then consider a projective resolution
    \[ \cdots \to P^2 \to P^1 \to P^1 \to k \]
    such that $P^i = AP_i^i$. But now we note that
    \[ k \to (P^0)^{\odot} \to (P^1)^{\odot} \to (P^2)^{\odot} \to \cdots \]
    is an injective resolution of $k$ in $\ms{mod}\text{-}A$. Now we note that
    \[ \hom_{A^{\mr{op}}}(k\ev{-n}, (P^i)^{\odot}) = (k \otimes_A P^i)_n^{\odot} = (P_i^i)_n^* = 0 \]
    whenever $i \neq n$. Thus $\ext_{A^{\mr{op}}}^i(k\ev{-n}, k) = 0$ whenever $i \neq n$, and so $A^{\mr{op}}$ is Koszul.
\end{proof}

\begin{defn}
    A graded ring $A$ is \textit{quadratic} if the natural map
    \[ T A_1 = \bigoplus_{n \geq 0} A_1^{\otimes n} \to A \]
    is surjective and the kernel $R \subseteq T A_1$ is generated by $R \cap A_1 \otimes A_1$.
\end{defn}

\begin{prop}
    Any Koszul ring is quadratic.
\end{prop}

We will now assume that all $A_i$ are finitely generated $k$-modules (I guess we will call it locally $k$-finite). We will now define a candidate dual algebra.
\begin{defn}
    Let $A = T_k V / \ev{R}$ be a locally $k$-finite quadratic ring over $k$, where $R \subseteq V \otimes V$. Then define its \textit{quadratic dual} $A^! \coloneqq T_k V^* / \ev{R^{\perp}}$, where $R^{\perp} \subseteq V^* \otimes V^*$ is the annihilator of $R$.
\end{defn}

\begin{rmk}
    Clearly $A^{!!} = A$.
\end{rmk}

\begin{prop}
    Let $A$ be a locally finite Koszul ring. Then $A^!$ is also Koszul.
\end{prop}

\begin{proof}
    We will use the fact (without proof) that the Koszul complex of $A$ is actually
    \[ \cdots \to A \otimes (A_2^!)^* \to A \otimes (A_1^!)^* \to A \]
    with differential given by $\Hom(A_{i+1}^!, A) \to \Hom(A_{i+1}^! \otimes V, A \otimes V) \to \Hom(A_i^!, A)$. 
    In coordinates if $\mr{id}_V = \sum v_{\alpha}^* \otimes v_{\alpha}$, this is given by
    \[ \dd{f}(a) = \sum f(a v_{\alpha}^*) v_{\alpha}. \]
    Therefore, we have a bigraded space
    \[ A \otimes (A^!)^{\odot} = \bigoplus_{i.j} A_i \otimes (A_{-j}^!)^* \]
    where the differential has bidegree $(1,1)$ and the cohomology appears in bidegree $(0,0)$. Taking duals, we obtain the space
    \[ A^! \otimes A^{\odot} = \bigoplus_{i,j} A_j^! \otimes A_{-i}^* \]
    where the differential has degree $(1,1)$ and the cohomology appears in degree $(0,0)$. Except this is a graded projective resolution of $k$ as an $A^!$-module, so $A^!$ is Koszul.
\end{proof}

\begin{thm}
    Let $A$ be a locally finite Koszul ring over $k$. Then $\Ext_A^{\bullet}(k, k)$ is canonically isomorphic to $(A^!)^{\mr{op}}$.
\end{thm}
We will write $\Ext_A^{\bullet}(k, k) \eqqcolon E(A)$. It is easy to see that $E(E(A)) = A$. 

We will now give a numerical criterion for a ring to be Koszul. Suppose that the $A_i$ are finite-dimensional for all $i$ and that $A_0$ is a product of copies of a field $F$. Then define the \textit{Hilbert polynomial} to be the matrix with entries
\[ P(A, t)_{x,y} = \sum t^i \dim (1_x A_i 1_y). \]

\begin{lem}
    Suppose that $A$ is Koszul. Then $P(A,t)P(A^!, -t)^T = 1$.
\end{lem}

\begin{thm}
    Let $E = E(A)$. Then $A$ is Koszul if and only if $P(A, t)P(E, -t) = 1$.
\end{thm}

Now we are finally able to define a derived category version of Koszul duality, which is what we really wanted. Write $C(A)$ for the homotopy category of complexes in $B\text{-}\ms{mod}$. We will use $M^i$ for the grading in the complex and $M^i_j$ for the grading in the module $M^i$. Now define the category $C^{\uparrow}(A)$ to be the full subcategory of $C(A)$ where $M_j^i = 0$ if $i \gg 0$ or $i + j \ll 0$. Similarly, define $C^{\downarrow}(A)$ to be the full subcategory of $C(A)$ where $M_j^i = 0$ if $i \ll 0$ or $i + j \gg 0$. Define the derived categories $D^{\uparrow}(A), D^{\downarrow}(A)$.

\begin{thm}
    Let $A$ be a locally finite Koszul ring. Then there exists an equivalence of triangulated categories $D^{\downarrow}(A) \cong D^{\uparrow}(A^!)$.
\end{thm}

We will not prove this result, but we will construct a functor. Let $M \in C(A)$. Then consider the bigraded vector space
\[ FM = A^! \otimes M = \bigoplus_{\ell,i} A_{\ell}^! \otimes M^i = \bigoplus_{\ell,i} \Hom_A(A \otimes A_{\ell}^{!*}, M^i) \]
with the differentials coming from the Koszul complex and from $M$ and given by
\begin{align*}
    \dd'(a \otimes m) &= (-1)^{i+j} \sum a v_{\alpha}^* \otimes v_{\alpha} m, \\
    \dd''(a \otimes m) &= a \otimes \partial m,
\end{align*}
where $v_{\alpha}^*, v_{\alpha}$ are as before. Now we consider the total differential $\dd = \dd' + \dd''$ and write
\[ (FM)_q^p = \bigoplus_{\substack{i+j=p \\ \ell-j = q}} A_{\ell}^! \otimes M_j^i. \]
In fact, $F$ sends $C^{\downarrow}(A)$ to $A^{\uparrow}(A^!)$ and takes acyclic complexes to acyclic complexes, so it induces a derived functor $DF \colon D^{\downarrow}(A) \to D^{\uparrow}(A^!)$. 

The inverse functor is given as follows. Let $N \in C(A^!)$. Then define
\[ (GN)_{\ell,i} = \Hom_k(A_{-\ell}, N^i). \]
This will have anticommuting differentials given by
\begin{align*}
    (\dd' f)(a) &= (-1)^i \sum v_{\alpha}^* f(v_{\alpha} a), \\
    (\dd'' f)(a) &= \partial(f(a)).
\end{align*}
Then we consider the total differential $\dd = \dd' + \dd''$ and write
\[ (GN)_q^p = \bigoplus_{\substack{p=i+j\\ q=\ell-j}} \Hom_k(A_{-\ell}, N_j^i). \]
One can check that $G$ sends $C^{\uparrow}(A^!)$ to $C_{\downarrow}(A)$ and that $G$ takes acyclic things to acyclic things. Thus there is a derived functor $DG \colon D^{\uparrow}(A^!) \to D^{\downarrow}(A)$.

We will write $K \coloneqq DF$ for the \textit{Koszul duality} functor. Here are some more properties of this functor.
\begin{thm}
    There is a canonical isomorphism $K(M\ev{n}) = (KM)[-n]\ev{-n}$.
\end{thm}

\begin{thm}
    Let $A$ be a Koszul ring over $k$ that is a finite-dimensional $k$-vector space. Also suppose that $A^!$ is left Noetherian. Then there are embeddings $D^b(A\text{-}\ms{mof}) \subseteq D^{\downarrow}(A)$ and $D^b(A^!\text{-}\ms{mof}) \subseteq D^{\uparrow}(A^!)$ such that Koszul duality induces an equivalence
    \[ K \colon D^b(A\text{-}\ms{mof}) \to D^b(A^!\text{-}\ms{mof}). \]
\end{thm}

\section{Parabolic-singular duality}%
\label{sec:parabolic_singular_duality}

Let $S \subset W$ be the set of simple reflections. Then for any subset $S_{\iota} \subset S$, let $W_{\iota} \subset W$ be the subgroup generated by $S_{\iota}$, $w_{\iota}$ be its longest element, and $W^{\iota} \subset W$ be the set of longest representatives of the cosets $W / W_{\iota}$.

Let $\lambda \in \h^*$ be integral and dominant (but possibly singular). Set $\cO_{\lambda} \subset \cO$ to be the full subcategory of objects with the same central character is $L(\lambda)$. Set
\[ S_{\lambda} \coloneqq \qty{s \in S \mid s \circ \lambda = \lambda}. \]
Then the simple objects in $\cO_{\lambda}$ are precisely the $L(x \circ \lambda)$ for all $x \in W^{\lambda}$.

Let $\b \subset \mf{q} \subset \g$ be a parabolic subalgebra. Define $\cO^{\mf{q}} \subset \cO_0$ to be the subcategory of all objects that are locally $\mf{q}$-finite. Let $S_{\mf{q}} \subset S$ be the simple reflections corresponding to $\mf{q}$. For all $x \in W^{\mf{q}}$ set $L_x^{\mf{q}} \coloneqq L(x^{-1} w_0 \circ 0) \in \cO^{\mf{q}}$. These represent the simple objects in $\cO^{\mf{q}}$. Call their projective covers $P_x^{\mf{q}}$.

The main theorem of this part of the talk is
\begin{thm}
    Suppose that $S_{\lambda} = S_{\mf{q}}$. Then there are isomorphisms of finite-dimensional algebras
    \begin{align*}
        \End_{\cO_{\lambda}} \qty(\bigoplus P(x \circ \lambda)) &\cong \Ext_{\cO^{\mf{q}}}^{\bullet}\qty(\bigoplus L_x^{\mf{q}}, \bigoplus L_x^{\mf{q}}) \\
        \End_{\cO^{\mf{q}}} \qty(\bigoplus P_x^{\mf{q}}) & \cong \Ext_{\cO_{\lambda}}^{\bullet} \qty(\bigoplus L(x \circ \lambda), \bigoplus L(x \circ \lambda)),
    \end{align*}
    where the summations are over $x \in W^{\lambda} = W^{\mf{q}}$. The algebras on the right are both Koszul rings and are Koszul dual to each other.
\end{thm}

Note that $\cO^{\b} = \cO_0$. In particular, in this case, we obtain the following:
\begin{cor}[Soergel '90]
    Let $L \in \cO$ be the direct sum of all simple modules with the same central character as $\C = L(0)$. Let $P \in \cO$ be the direct sum of their projective covers. Then there exists an isomorphism
    \[ \End_{\cO}(P) \cong \Ext_{\cO}^{\bullet}(L, L). \]
    Moreover, $\Ext_{\cO}^{\bullet}(L, L)$ is a self-dual Koszul ring.
\end{cor}

The proof of the theorem is geometric and developing the machinery would take to long, so we will prove the equality of dimensions for the first isomorphism. Using BGG reciprocity and the Kazhdan-Lusztig conjectures, we have
\begin{align*}
    \dim \End_{\cO_{\lambda}}\qty(\bigoplus P(x \circ \lambda)) &= \sum_{x, y \in W^{\mf{q}}} [P(x \circ \lambda) : L(y \circ \lambda)] \\
    &= \sum_{x, y, z \in W^{\mf{q}}} (P(x \circ \lambda) : M(z \circ \lambda)) [M(z \circ \lambda) : L(y \circ \lambda)] \\
    &= \sum_{x,y,z \in W^{\mf{q}}} [M(z \circ \lambda):L(x \circ \lambda)][M(z \circ \lambda) : L(y \circ \lambda)] \\
    &= \sum_{x,y,z \in W^{\mf{q}}} P_{z,x}(1)P_{z,y}(1).
\end{align*}
By an argument involving localization (translating the problem to geometry), this is precisely the dimennsion of $\Ext_{\cO^{\mf{q}}}\qty(\bigoplus L_x^{\mf{q}}, \bigoplus L_x^{\mf{q}})$.

Now write $A_{Q} = \Ext_{\cO}^{\bullet}(\bigoplus L_x^{\mf{q}}, \bigoplus L_x^{\mf{q}})$ and $A^{Q} = \Ext_{\cO}^{\bullet}\qty(\bigoplus L(x \circ \lambda), \bigoplus L(x \circ \lambda))$. Also, write $A^{\mf{q}} = \End_{\cO}\qty(\bigoplus P_x^{\mf{q}})$.
\begin{cor}
    There is a ring isomorphism $A^{Q} = E(A_{Q})$.
\end{cor}

\begin{proof}
    By the theorem, there is an equivalence $\cO_{\lambda} \cong A_{Q}\text{-}\ms{Mof}$. This is because if $\mc{A}$ is an abelian category whose objects have finite length and $P \in \mc{A}$ is a projective generator, then there is an equivalence $\Hom_{\mc{A}}(P,-) \colon \mc{A} \to \ms{Mof}\text{-}E$, where $E = \End_{\mc{A}} P$. But now this equivalence identifies $L(x \circ \lambda)$ with $A_{Q}^0 1_x$, and from this we deduce the desired isomorphism.
\end{proof}

\begin{prop}
    The rings $A_{Q}$ and $A^{Q}$ are Koszul.
\end{prop}

\begin{proof}
    First, note that $\cO^{\mf{q}} = \ms{Mof}\text{-}A^{\mf{q}}$. But now the simples correspond under this equivalence, and therefore
    \[ \Ext_{\cO^{\mf{q}}}^{\bullet}\qty(\bigoplus L_x^{\mf{q}}, \bigoplus L_x^{\mf{q}}) \cong \Ext_{A^{\mf{q}}, \mr{op}}^{\bullet}(A_0^{\mf{q}}, A_0^{\mf{q}}). \]
    This tells us that $A_Q = E(A^{\mf{q}, \mr{op}})$. Once we prove that $A^{\mf{q}}$ is Koszul, we are done.

    To prove this, consider that $\cO^{\mf{b}} \cong \ms{Mof}\text{-}A^B$. But now we know that $\cO^{\mf{q}}$ consists of objects in $\cO^{\b}$ such that if $[M : L_x^{\mf{b}}] \neq 0$, then $x \in W^{\mf{q}}$, and therefore $\cO^{\mf{q}} \cong \ms{Mof}\text{-}(A^B/I_Q)$, where $I_Q$ is generated by everything in $W \setminus W^{\mf{q}}$. Thus $A^{\mf{q}} \cong \cO^{\mf{q}}$ because they are both endomorphism algebras of the same progenerator. Now restricting modules over $A^B/I_Q$ to $A^B$ induces injections on Ext groups, so we now only need to prove that $A^B$ is Koszul.

    To prove that $A^B = A_B$ is Koszul, we will use the numerical criterion. We will simply use the fact (proved geometrically) that the Hilbert polynomial of $A_Q$ is given essentially by the intersection cohomology matrix
    \[ P^Q \coloneqq IC(G/Q, t)_{x,y} = P_{x,y}(t^{-2})t^{\ell(y) - \ell(x)}. \]
    In fact, we have $P(A_Q, t) = (P^Q)^T P^Q$. But now we know that $\cO^{\b} = \cO_0$ and that $L_x^{\b} = L(x^{-1}w_0 \circ 0)$, and thus we have $E(A_B) = A^B = A_B$ with $1_x$ corresponding to $1_{x^{-1} w_0}$. Therefore, we may define
    \[ (P_B)_{x,y} \coloneqq P^B_{w_0 x^{-1} w_0 y^{-1}} \]
    and we have $P(E(A_B), t) = P_B^T P_B$. We now have
    \begin{align*}
        P(A_B, t)P(E(A_B), -t) &= (P^B(t))^T P^B(t) (P_B(-t))^T P_B(-t) \\
        &= 1
    \end{align*}
    by some magic of Kazhdan-Lusztig, which says that $P^B(t) (P_B(-t))^T = 1$.
\end{proof}

\begin{rmk}
    You could also prove that $A_B$ is Koszul by using a result of Bezrukavnikov that if $A_0, A_1$ are finite-dimensional and $A \cong E(A)$, then $A$ is Koszul.
\end{rmk}


\end{document}
