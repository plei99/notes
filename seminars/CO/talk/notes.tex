\documentclass{amsart}
\usepackage{amsmath}
\usepackage{amssymb}
\usepackage{amsthm}
%\usepackage{MnSymbol}
\usepackage{bm}
\usepackage{accents}
\usepackage{mathtools}
\usepackage{tikz}
\usetikzlibrary{calc}
\usetikzlibrary{decorations.pathmorphing,shapes}
\usetikzlibrary{automata,positioning}
\usepackage{tikz-cd}
\usepackage{forest}
\usepackage{braket} 
\usepackage{listings}
\usepackage{mdframed}
\usepackage{verbatim}
\usepackage{physics}
\usepackage{stmaryrd}
\usepackage{mathrsfs} 
\usepackage{stackengine} 
%\usepackage{/home/patrickl/homework/macaulay2}

%font
\usepackage[sc]{mathpazo}
\usepackage{eulervm}
\usepackage[scaled=0.86]{berasans}
\usepackage{inconsolata}
\usepackage{microtype}

%CS packages
\usepackage{algorithmicx}
\usepackage{algpseudocode}
\usepackage{algorithm}

% typeset and bib
\usepackage[english]{babel} 
\usepackage[utf8]{inputenc} 
\usepackage[T1]{fontenc}
%\usepackage[backend=biber, style=alphabetic]{biblatex}
\usepackage[bookmarks, colorlinks, breaklinks]{hyperref} 
\hypersetup{linkcolor=black,citecolor=black,filecolor=black,urlcolor=black}
\usepackage{graphicx}
\graphicspath{{./}}

% other formatting packages
\usepackage{float}
\usepackage{booktabs}
\usepackage[shortlabels]{enumitem}
\usepackage{csquotes}
%\usepackage{titlesec}
%\usepackage{titling}
%\usepackage{fancyhdr}
%\usepackage{lastpage}
\usepackage{parskip}

\usepackage{lipsum}

% delimiters
\DeclarePairedDelimiter{\gen}{\langle}{\rangle}
\DeclarePairedDelimiter{\floor}{\lfloor}{\rfloor}
\DeclarePairedDelimiter{\ceil}{\lceil}{\rceil}


\newtheorem{thm}{Theorem}[section]
\newtheorem{cor}[thm]{Corollary}
\newtheorem{prop}[thm]{Proposition}
\newtheorem{lem}[thm]{Lemma}
\newtheorem{conj}[thm]{Conjecture}
\newtheorem{quest}[thm]{Question}

\theoremstyle{definition}
\newtheorem{defn}[thm]{Definition}
\newtheorem{defns}[thm]{Definitions}
\newtheorem{con}[thm]{Construction}
\newtheorem{exm}[thm]{Example}
\newtheorem{exms}[thm]{Examples}
\newtheorem{notn}[thm]{Notation}
\newtheorem{notns}[thm]{Notations}
\newtheorem{addm}[thm]{Addendum}
\newtheorem{exer}[thm]{Exercise}

\theoremstyle{remark}
\newtheorem{rmk}[thm]{Remark}
\newtheorem{rmks}[thm]{Remarks}
\newtheorem{warn}[thm]{Warning}
\newtheorem{sch}[thm]{Scholium}


% unnumbered theorems
\theoremstyle{plain}
\newtheorem*{thm*}{Theorem}
\newtheorem*{prop*}{Proposition}
\newtheorem*{lem*}{Lemma}
\newtheorem*{cor*}{Corollary}
\newtheorem*{conj*}{Conjecture}

% unnumbered definitions
\theoremstyle{definition}
\newtheorem*{defn*}{Definition}
\newtheorem*{exer*}{Exercise}
\newtheorem*{defns*}{Definitions}
\newtheorem*{con*}{Construction}
\newtheorem*{exm*}{Example}
\newtheorem*{exms*}{Examples}
\newtheorem*{notn*}{Notation}
\newtheorem*{notns*}{Notations}
\newtheorem*{addm*}{Addendum}


\theoremstyle{remark}
\newtheorem*{rmk*}{Remark}

% shortcuts
\newcommand{\Ima}{\mathrm{Im}}
\newcommand{\A}{\mathbb{A}}
\newcommand{\G}{\mathbb{G}}
\newcommand{\N}{\mathbb{N}}
\newcommand{\R}{\mathbb{R}}
\newcommand{\C}{\mathbb{C}}
\newcommand{\Z}{\mathbb{Z}}
\newcommand{\U}{\mathcal{U}}
\newcommand{\Q}{\mathbb{Q}}
\renewcommand{\k}{\Bbbk}
\renewcommand{\P}{\mathbb{P}}
\newcommand{\M}{\overline{M}}
\newcommand{\g}{\mathfrak{g}}
\newcommand{\h}{\mathfrak{h}}
\newcommand{\n}{\mathfrak{n}}
\renewcommand{\b}{\mathfrak{b}}
\newcommand{\ep}{\varepsilon}
\newcommand*{\dt}[1]{%
   \accentset{\mbox{\Huge\bfseries .}}{#1}}
%\renewcommand{\abstractname}{Official Description}
\newcommand{\mc}[1]{\mathcal{#1}}
% \newcommand{\msc}[1]{\mathscr{#1}}
\newcommand{\T}{\mathbb{T}}
\newcommand{\mf}[1]{\mathfrak{#1}}
\newcommand{\mr}[1]{\mathrm{#1}}
\newcommand{\ms}[1]{\mathsf{#1}}
\newcommand{\ol}[1]{\overline{#1}}
\newcommand{\ul}[1]{\underline{#1}}
\newcommand{\wt}[1]{\widetilde{#1}}
\newcommand{\wh}[1]{\widehat{#1}}
\renewcommand{\div}{\operatorname{div}}
\newcommand{\1}{\mathbf{1}}
\newcommand{\2}{\mathbf{2}}
\newcommand{\3}{\mathbf{3}}
\newcommand{\I}{\mathrm{I}}
\newcommand{\II}{\mr{I}\hspace{-1.3pt}\mr{I}}
\newcommand{\III}{\mr{I}\hspace{-1.3pt}\mr{I}\hspace{-1.3pt}\mr{I}}

\DeclareMathOperator{\Der}{Der}
\DeclareMathOperator{\Tor}{Tor}
\DeclareMathOperator{\Hom}{Hom}
\DeclareMathOperator{\End}{End}
\DeclareMathOperator{\Ext}{Ext}
\DeclareMathOperator{\ad}{ad}
\DeclareMathOperator{\Aut}{Aut}
\DeclareMathOperator{\Rad}{Rad}
\DeclareMathOperator{\Pic}{Pic}
\DeclareMathOperator{\supp}{supp}
\DeclareMathOperator{\Supp}{Supp}
\DeclareMathOperator{\depth}{depth}
\DeclareMathOperator{\sgn}{sgn}
\DeclareMathOperator{\spec}{Spec}
\DeclareMathOperator{\Spec}{Spec}
\DeclareMathOperator{\proj}{Proj}
\DeclareMathOperator{\Proj}{Proj}
\DeclareMathOperator{\ord}{ord}
\DeclareMathOperator{\Div}{Div}
\DeclareMathOperator{\Bl}{Bl}
\DeclareMathOperator{\ch}{ch}

\title{CAts, bOndage, and why you can'T do representAtioN theory without Geometry}
\author{Patrick Lei}
\date{November 3, 2021}

\begin{document}
    
\maketitle

\begin{abstract}
    We define strong linkage, and then discuss the Jantzen filtration and use it to prove a relationship between strong linkage and multiplicity in the composition series of a Verma.
\end{abstract}

\section{Introduction}
\label{sec:introduction}

Our goal is to prove the following theorem, due to Bernstein-Gelfand-Gelfand.
\begin{thm}[BGG]
  Let $\lambda, \mu \in \h^{*}$.
  \begin{enumerate}
    \item If $\mu$ is strongly linked to $\lambda$, then $M(\mu) \hookrightarrow M(\lambda)$. This implies $[M(\lambda) : L(\mu) ] \neq 0$.
    \item If $[M(\lambda) : L(\mu)] \neq 0$, then $\mu$ is strongly linked to $\lambda$.
  \end{enumerate}
\end{thm}

We first need to define what it means for weights to be strongly linked. Then we will state a result of Jantzen that we will use to prove this theorem. Next, we will do some linear algebra before proving the result of Jantzen. Finally, we will use Jantzen to prove BGG. Note that this is a different order than in which Humphreys does things.

\begin{exm}[Cursed example]
  Consider $\g = \mf{sl}_4$. Number the simple roots and fundamental weights as usual. This means that the simple roots and fundamental weights are
  \[ \alpha_1 = e_1 - e_2, \alpha_2 = e_2 - e_3, \alpha_3 = e_3 - e_4; \qquad \varpi_1 = e_1, \varpi_2 = e_1 + e_2, \varpi_3 = e_1 + e_2 + e_3. \]
  Write $c_1 \varpi_1 + c_2 \varpi_2 + c_3 \varpi_3$ as $(c_1, c_2, c_3)$. This means that
  \[\alpha_1 = (2,-1,0), \qquad \alpha_2 = (-1,2,-1), \qquad \alpha_3 = (0, -1, 2). \]
  Consider $\lambda = (1, -2, -1)$. Then we note that $\lambda + \rho = \alpha_1$ but if $w = s_2 s_3 s_2 s_1 s_2$, then $w(\lambda + \rho) = - \alpha_3$. But this implies that
  \[ \lambda - w \circ \lambda = \lambda - (w(\lambda + \rho) - \rho) = \lambda + \rho - w(\lambda + \rho) = \alpha_1 + \alpha_3, \]
  and therefore $w \circ \lambda < \lambda$. However, apparently there is no embedding of $M(w \circ \lambda)$ inside $M(\lambda)$.

  The reason for this is actually that if you inspect the Bruhat order for $W = S_4$. The weight in $\Lambda^+ - \rho$ linked to $\lambda$ is $\mu = (0, -1, 0)$. Here, we note that $\lambda = s_2 s_3 \circ \mu$ while $w \circ \lambda = s_3 s_2 s_1 \circ \mu$. However, $s_3 s_2 s_1$ and $s_2 s_3$ are incomparable.
\end{exm}

\section{Strong linkage}
\label{sec:strong_linkage}

The first thing we need to do is to define what it means for two weights $\mu, \lambda$ to be strongly linked.

\begin{defn}
  Let $\lambda, \mu \in \h^{*}$. Write $\mu \uparrow \lambda$ if $\mu = \lambda$ of there exists a root $\alpha > 0$ such that $\mu = s_{\alpha} \circ \lambda < \lambda$. Note that this is equivalent to $\ev{\lambda + \rho, \alpha^{\vee}} \in \Z^{>0}$. More generally, if $\mu = \lambda$ or there exists $\alpha_{1}, \ldots, \alpha_{r} \in \Phi^{+}$ such that
  \[ \mu = (s_{\alpha_{1}} \cdots s_{\alpha_{r}}) \circ \lambda \uparrow (s_{\alpha_{2}} \cdots s_{\alpha_{r}}) \circ \lambda \uparrow \cdots s_{\alpha_{r}} \circ \lambda \uparrow \lambda, \]
  we say that $\mu$ is \textit{strongly linked} to $\lambda$ and write $\mu \uparrow \lambda$.
\end{defn}

Note that if we repeatedly apply Theorem 4.6 (the second big theorem in Fan's lecture), we immediately obtain the first part of the theorem. The second part of the proof is more involved.

Before we proceed, we need to discuss the Bruhat order on the Weyl group (in fact, this can be defined for any Coxeter group, but we will not need this). Let $\lambda$ be a regular integral weight which is antidominant. Then suppose that $\alpha > 0$ is a positive root and that $s_{\alpha} \circ (w \circ \lambda) < w \circ \lambda$ for some $w \in W$. By the second main theorem of the previous lecture, we know that
\[ M(s_{\alpha} w \circ \lambda) \hookrightarrow M(w \circ \lambda). \]
But we know that this is equivalent to
\[ \ev{w(\lambda + \rho), \alpha^{\vee}} = \ev{\lambda + \rho, w^{-1} \alpha^{\vee}} = \ev{\lambda + \rho, (w^{-1}\alpha)^{\vee}} in \Z_{>0} \]
Because $\lambda$ is antidominant, we know $w^{-1} \alpha < 0$. This tells us that if $w' = s_{\alpha} w$, then
\[ \ell(w) = \ell(s_{\alpha} w') = \ell((w')^{-1}s_{\alpha}) > \ell((w')^{-1}) = \ell(w'). \]
All of this is reversible, so by definition we see that $w' \circ \lambda < w \circ \lambda$ if and only if $w' < w$. This implies, after we prove the main theorem, that

\begin{cor}
  Let $\lambda$ be a regular antidominant weight and let $w, w' \in W$. Then $[M(w \circ \lambda) : L(w' \circ \lambda)] \neq 0$ if and only if $w' \leq w$.
\end{cor}

\section{Jantzen filtration}
\label{sec:jantzen_filtration}

\begin{prop}[Jantzen]
  Let $\lambda \in \h^{*}$. Then $M(\lambda)$ has a filtration by submodules
  \[ M(\lambda) = M(\lambda)^{0} \supset M(\lambda)^{1} \supset M(\lambda)^{2} \supset \cdots \]
  such that $M(\lambda)^{i} \gg 0$ for $i$ large enough and the following conditions hold:
  \begin{enumerate}
    \item Each nonzero quotient $M(\lambda)^{i} / M(\lambda)^{i+1}$ has a nondegenerate contravariant form in the sense of 3.14.
    \item $M(\lambda)^{1} = N(\lambda)$ is the unique maximal submodule of $M(\lambda)$.
    \item At the level of formal characters, we have
      \[ \sum_{i > 0} \operatorname{ch} M(\lambda)^{i} = \sum_{\alpha > 0, s_{\alpha} \circ \lambda < \lambda} \operatorname{ch} M(s_{\alpha} \circ \lambda). \]
  \end{enumerate}
\end{prop}

This filtration is called the \textit{Jantzen filtration} and the formal character formula is called the \textit{Jantzen sum formula}. We will write
\[ M(\lambda)_{i} \coloneqq M(\lambda)^{i} / M(\lambda)^{i+1}. \]
Also note that the summation on the right side of the sum formula is over a set of positive roots which we will call $\Phi_{\lambda}^+$. In fact, if $M(\lambda)^n \neq 0$ but $M(\lambda)^{n+1} = 0$, the sum formula tells us that $n = \abs{\Phi_{\lambda}^+}$.

\begin{exm}
  Let $\lambda$ be regular, antidominant, and integral. Then we know that $\Phi_{w \circ \lambda}^+$ is the set of all $\alpha > 0$ satisfying $s_{\alpha} \circ (w \circ \lambda) < w \circ \lambda$. But this is the same as
  \[ \ev{w \circ \lambda + \rho, \alpha^{\vee}} = \ev{w(\lambda + \rho), \alpha^{\vee}} = \ev{\lambda + \rho, w^{-1} \alpha^{\vee}} > 0, \]
  but of course this is equivalent to $w^{-1} \alpha^{\vee} < 0$ by defintion of $\lambda$ being integral and antidominant. Therefore in fact here $n = \ell(w)$.
\end{exm}

Here are some reasonable questions about these things:
\begin{enumerate}
  \item Is the Jantzen filtration unique relative to properties such as the sum formula and existence of nondegenerate contravariant forms on the quotients?
  \item What are the composition factor multiplicities in each filtration layer $M(\lambda)_i$?
  \item Are the filtration layers semisimple? If so, does the filtration coincide with one of the standard module filtrations with semisimple quotients?
  \item How does the Jantzen filtration behave with respect to $M(\mu) \hookrightarrow M(\lambda)$?
\end{enumerate}

A more precise formulation of the last question is called the \textit{Jantzen conjecture}, and this can only be proved (as of whenever the book was published) using geometric tools like Beilinson-Bernstein localization. For completeness, it is stated here:

\begin{conj}[Jantzen]
  Suppose that $\mu \uparrow \lambda$ and set $r \coloneqq \abs{\Phi_{\lambda}^+} - \abs{\Phi_{\mu}^+}$. Then $M(\mu) \subset M(\lambda)^i$ if $i \geq r$ while $M(\mu) \cap M(\lambda)^i = M(\mu)^{i-r}$ if $i \geq r$.
\end{conj}


\begin{exm}
  Let $\g = \mf{sl}_3$. Suppose that $\lambda$ is regular, antidominant, and integral. We will write down the Jantzen filtration for $M(w \circ \lambda)$ and then deduce that all composition factors have multiplicity $1$. Let $\alpha, \beta$ be the simple roots and write $w = s_{\alpha} s_{\beta}$. The sum formula tells us that
  \begin{align*}
    \sum_{i>0} \ch M(w \circ \lambda)^i &= \ch M(s_{\alpha} \circ \lambda) + M(s_{\beta} \circ \lambda) \\
    &= \ch L(s_{\alpha} \circ \lambda) + \ch L(s_{\beta} \circ \lambda) + 2 \ch L(\lambda).
  \end{align*}
  But this implies that the Jantzen filtration is
  \[ M(w \circ \lambda)^0 = M(w \circ \lambda) \supset N(w \circ \lambda) \supset L(\lambda) \supset 0. \]
  This implies that there are four composition factors each with multiplicity $1$.
\end{exm}

\section{Proofs of results}
\label{sec:proofs_of_results}

We will now prove Jantzen's theorem and apply it to prove the main theorem. I assume that everyone is familiar with the basic theory of finitely generated modules over a principal ideal domain.\footnote{This would not be possible at certain meme-tier schools.}

Let $A$ be a principal ideal domain and $p \in A$ be a prime element. Suppose that $M$ is a free $A$-module of rank $r$ with a nondegenerate symmetric bilinear form $(-,-)$. Let $D$ be the determinant of the bilinear form -- this is well-defined up to a unit. Let $\ol{M} = M/pM$, which is a vector space over $\ol{A} \coloneqq A/pA$. For $n \in \Z$, define
\[ M(n) \coloneqq \qty{e \in M \mid (e, M) \subset p^n A}. \]
For notation, we will write $M^* = \Hom(M, A)$ and $M^{\vee} \subseteq M^*$ for the image of $M$ under the bilinear form. We will write $e_1, \ldots, e_r$ for a basis of $M$ and $f_1, \ldots, f_r$ for the dual basis.
\begin{lem}\leavevmode
  \begin{enumerate}
    \item We have the identity
          \[ v_p(D) = \sum_{n > 0} \dim_{\ol{A}} \ol{M(n)}. \]
    \item For all $n$, the modified bilinear form $(-,-)_n = p^{-n}(-,-)$ induces a nondegenerate form on $\ol{M(n)} / \ol{M(n+1)}$.
  \end{enumerate}
\end{lem}

\begin{proof}
  Write $f = \sum a_j f_j$ for some $f \in M$. Then writing $(e_i, f_j) = f_j \delta_{ij}$, we know that $(e_i, f) = a_i d_i$. On the other hand, we know that $f \in M(n)$ if and only if
  \[ n \leq v_p((e_i, f)) = v_p(a_i d_i) = v_p(a_i) + v_p(d_i) \]
  for all $i$. But if we write $n_i \coloneqq v_p(d_i)$, then we know that $M(n)$ is spanned by $f_i$ for all $i$ where $n \leq n_i$ together with $p^{n-i_i} f_i$ for the $i$ such that $n > n_i$. But this implies that
  \[ \dim \ol{M(n)} = \# \qty{i \mid n \leq n_i}, \]
  and in particular $\ol{M(n)} = 0$ for $n \gg 0$.\footnote{Humphreys refers to this last $0$ as $\ol{0}$.} This means the sum is defined, and therefore we have
  \[ \sum_{n>0} \dim \ol{M(n)} = \sum_{n > 0} \# \qty{i \mid n \leq n_i} = \sum_{i=1}^r v_p(d_i) = v_p(D). \]

  For the second part, it ic clear that this form takes values in $A$, but we need to check that there is an induced form on $\ol{M(n)}$. But here, if $e \in pM \cap M(n)$, we see that
  \[ (e, M(n))_n = p^{-n}(e, M(n)) \subset p^{1-n} (M, M(n)) \subset p^n p^{1-n} A = pA. \]
  But if $f \in M(n+1)$, the same argument gives us $(f, M(n))_n \subset pA$. This gives us a form on $\ol{M(n)} / \ol{M(n+1)}$. Finally we know that if $n \leq n_j$, then $(e_i, f_j) = \delta_{ij} p^{-n} d_j \neq 0$, and thus we conclude that the form is nondegenerate.
\end{proof}

Before we continue, we will state some facts about contravariant forms that were not covered by Kevin.

\begin{defn}
  Let $(-,-)$ be a symmetric bilinear form on a module $M$. Then $(-,-)$ is \textit{contravariant} if $(u \cdot v, w) = (v, \tau(u) \cdot w)$ for all $u \in \U \g$ and $v, w \in M$.
\end{defn}
It is not even clear that these exist, but here are some properties:
\begin{enumerate}
  \item If $M$ has a contravariant form, then $M_{\lambda} \perp M_{\mu}$ for $\lambda \neq \mu$.
  \item If $M$ is a highest weight module, then it has a unique (up to scalars) nonzero contravariant form.
  \item If $N \subset M$ is a submodule, then $N^{\perp}$ is also a submodule.
  \item Any contravariant form on a highest weight module $M$ must induce the zero form on its maximal submodule. Moreover, the maximal submodule is the radical of the form and the form is nondegenerate if and only if $M = L(\lambda)$ is simple.
\end{enumerate}

\begin{proof}[Proof of Jantzen]
  The reason we did all of that linear algebra above is because we will extend to the field $K \coloneqq \C(T)$. Also write $A = \C[T]$. Then apparently we can adapt the entire theory of Lie algebras to $\g_K \coloneqq \g \otimes_{\C} K$ and $\g_A \coloneqq \g \otimes_{\C} A$. I will not check this, and you should not ask me to do so. We will use the lemma to construct filtrations here and then kill $T$ to obtain the Jantzen filtration.

  Let $\lambda \in \h^*$ be a weight and write $\lambda_T \coloneqq \lambda + T \rho \in \h^*_K$. Clearly $\lambda_T$ is antidominant, and thus $M(\lambda_T)$ is simple. But now we know that the contravariant form on $M(\lambda_T)$ is nondegenerate by Theorem 3.15 in Humphreys. Here, recall that the contravariant form satisfies for all $v, w \in M$ and $y \in \U \g$ the identity
  \[ (y \cdot v, w) = (v, \tau(y) \cdot w). \]
  Now recall that $\U \g_A = A \otimes_{\C} \U \g \subset K \otimes_{\C} \U \g = \U \g_K$ is an $A$-form. This induces an $A$-form $M(\lambda_T)_A \subset M(\lambda_T)$. Each weight space is a free $A$-module of finite rank and the contravariant form is nondegenerate after restricting to each weight space. If $\Gamma$ is the space of positive linear combinations of simple roots, then write $M_{\lambda_T - \nu}$ for the $A$-form of $M(\lambda_T)_{\lambda_T - \nu}$. Set
  \[ M(\lambda_T)_A^i \coloneqq \sum_{\nu \in \Gamma} M_{\lambda_T - \nu}(i). \]
  These form a decreasing filtration of $M(\lambda_T)_A$.

  Now we can quotient by the ideal $(T) \subset \C[T]$ and we obtain a decreasing filtration $M(\lambda)^i$ on $M(\lambda) \cong M(\lambda_T)_A / T M(\lambda_T)_A$. By the lemma, the quotients $M(\lambda)^i / M(\lambda)^{i+1}$ have induced nondegenerate contravariant forms. Also, we know that the filtration of each individual weight space eventually ends at $0$, and thus for large enough $i$ we have $M(\lambda)^i = 0$ because only finitely many weights are linked to $\lambda$.

  Next, we note that $M(\lambda) / M(\lambda)^1$ is a highest weight module with a nondegenerate contravariant form and therefore is simple. Because $M(\lambda)$ has a unique simple quotient, we see that $M(\lambda)^1 = N(\lambda)$. First, we will express the sum
  \[ \sum_{i > 0} \ch M(\lambda)^i\]
  in terms of the determinants of the contravariant forms on the steps of the filtration. Here, by some magic involving Shapovalov elements that was discovered by Jantzen and Shapovalov independently while working on their PhD theses, the determinant of the contravariant form on the $\lambda_T - \nu$ weight space of $M(\lambda_T)$ is given by
  \[ D_{\nu}(\lambda_T) = \prod_{\alpha > 0} \prod_{r > 0} (\ev{\lambda_T + \rho, \alpha^{\vee}} - r)^{\mc{P}(\nu - r \alpha)}, \]
  where $\mc{P}$ is the Kostant partition function. Note that the Kostant partition function gives the number of ways to write a weight as a sum of positive roots with non-negative coefficients. This is actually unique only up to a unit, but this means that $v_T(D_{\nu}(\lambda_T))$ is well-defined. But now we know that
  \[ \ev{\lambda_T + \rho, \alpha^{\vee}} - r = \ev{\lambda + \rho, \alpha^{\vee}} - r + T \ev{\rho, \alpha^{\vee}}. \]
  This is not a multiple of $T$ unless $\ev{\lambda + \rho, \alpha^{\vee}} = r$. But this means that $\alpha \in \Phi_{\lambda}^+$ and $v_T$ for this term is $1$. But this implies that for fixed $\nu$ and $\alpha$, the contribution is given by
  \[ \mc{P}(\nu - \ev{\lambda + \rho, \alpha^{\vee}}) e(\lambda - \nu). \]
  Finally, we compute formally and obtain the result
  \begin{align*}
    \sum_{i > 0} \ch M(\lambda)^i &= \sum_{\nu} \sum_{\alpha} \mc{P}(\nu - \ev{\lambda + \rho, \alpha^{\vee}}) e(\lambda - \nu) \\
    &= \sum_{\alpha} \sum_{\nu} \mc{P}(\nu) e(\lambda - \ev{\lambda + \rho, \alpha^{\vee}} \alpha - \nu).
  \end{align*}
  This comes from a variable change $\nu \mapsto \nu + \ev{\lambda + \rho, \alpha^{\vee}} \alpha$ in $\nu$. But now we precisely have
  \[ s_{\alpha} \circ \lambda = \lambda - \ev{\lambda + \rho, \alpha^{\vee}} \alpha, \]
  and therefore if we fix $\alpha$ the sum becomes
  \[ \sum_{i > 0} \ch M(\lambda)^i = \sum_{\alpha} \ch M(s_{\alpha} \circ \lambda), \]
  as desired.
\end{proof}

\begin{rmk}
  Remember that $\rho$ disappears when we kill $T$. Remember we only needed the fact that $\ev{\rho, \alpha^{\vee}} \neq 0$. We may consider what ahppens if we replace $\rho$ by another weight, but a definitive answer to this is only provided by Belinson-Bernstein using a geometric approach.
\end{rmk}

\begin{proof}[Proof of BGG]
  We will finally prove the BGG theorem. We will induct on the number of linked weights $\mu \leq \lambda$. If $\lambda$ is minimal in its linkage class, then we know $M(\lambda) = L(\lambda)$, so we are done. Now suppose that $\mu < \lambda$ and $[M(\lambda) : L(\mu)] > 0$. But this implies that
  \[ [ M(\lambda)^1 : L(\mu) ] > 0. \]
  But now the sum formula for the Jantzen filtration gives us
  \[ [M(s_{\alpha} \circ \lambda) : L(\mu)] > 0 \]
  for some $\alpha \in \Phi_{\lambda}^+$. By the inductive hypothesis, there exists $\alpha_1, \ldots, \alpha_r \in \Phi^+$ such that
  \[ \mu = (s_{\alpha_1} \cdots s_{\alpha_r}) s_{\alpha} \circ \lambda \uparrow (s_{\alpha_2} \circ s_{\alpha_r}) s_{\alpha} \circ \lambda \uparrow \cdots \uparrow s_{\alpha_r} s_{\alpha} \circ \lambda \uparrow s_{\alpha} \circ \lambda. \]
  Because $s_{\alpha} \circ \lambda < \lambda$, we are done.
\end{proof}





\end{document}
