\documentclass[12pt]{article}
\usepackage{amsmath}
\usepackage{amsthm}
\usepackage{amssymb}
\usepackage{fullpage}
\usepackage{tikz-cd}
\usepackage{mathrsfs}
\usepackage{bbm}
\usepackage[shortlabels]{enumitem}

\theoremstyle{definition}
\newtheorem{defn}{Definition}[section]
\newtheorem{ex}[defn]{Example}
\theoremstyle{theorem}
\newtheorem{lem}[defn]{Lemma}
\newtheorem{thm}[defn]{Theorem}
\newtheorem{prop}[defn]{Proposition}
\newtheorem{cor}[defn]{Corollary}

\newcommand{\bl}[2]{\left \langle #1, #2 \right \rangle}
\renewcommand{\Bar}{\overline}
\newcommand{\xto}[1]{\xrightarrow{#1}}
\newcommand{\Z}{\mathbb Z}
\newcommand{\Q}{\mathbb Q}
\newcommand{\F}{\mathbb F}
\newcommand{\C}{\mathbb C}
\newcommand{\R}{\mathbb R}
\newcommand{\m}{\mathfrak m}
\renewcommand{\O}{\mathcal O}
\newcommand{\p}{\mathfrak p}
\newcommand{\q}{\mathfrak q}
\newcommand{\I}{\mathcal I}
\renewcommand{\t}{\mathfrak t}
\renewcommand{\P}{\mathbb P}
\newcommand{\sep}{\text{sep}}
\newcommand{\tensor}{\otimes}
\newcommand{\Art}[2]{\left(\frac {#1} {#2} \right)}
\newcommand{\norm}[1]{\left \Vert #1 \right \Vert}
\newcommand{\D}{\mathscr D}
\newcommand{\cc}{\mathfrak c}
\newcommand{\h}{\mathfrak h}
\newcommand{\1}{\mathbbm 1}
\newcommand{\A}{\mathbb A}
\newcommand{\Ib}{\mathbb I}
\newcommand{\ab}{\text{ab}}
\newcommand{\unr}{\text{unr}}
\renewcommand{\S}{\mathbb S}
\newcommand{\N}{\mathbb N}
\renewcommand{\H}{\mathbb H}
\newcommand{\Pet}{\text Pet}
\newcommand{\E}{\mathcal E}
\newcommand{\inject}{\hookrightarrow}
\newcommand{\op}{\text{op}}
\newcommand{\cF}{\mathcal F}
\newcommand{\cZ}{\mathcal Z}
\newcommand{\M}{\mathcal M}
\newcommand{\surject}{\twoheadrightarrow}
\newcommand{\Gm}{\mathbb G_m}
\newcommand{\Ga}{\mathbb G_a}
\newcommand{\git}{/\!\!/}
\newcommand{\g}{\mathfrak g}
\newcommand{\LieGrp}{\cat{LieGrp}}
\newcommand{\LieAlg}{\cat{LieAlg}}
\newcommand{\cat}[1]{\mathbf{#1}}
\newcommand{\Set}{\cat{Set}}
\newcommand{\Top}{\cat{Top}}
\newcommand{\Cat}{\cat{Cat}}
\newcommand{\Grp}{\cat{Grp}}
\newcommand{\Ab}{\cat{Ab}}
\newcommand{\Sch}{\cat{Sch}}
\newcommand{\CRing}{\cat{CRing}}
\newcommand{\Mod}{\cat{Mod}}
\newcommand{\Sp}{\cat{Sp}}
\newcommand{\PSp}{\cat{PSp}}
\newcommand{\PSh}{\cat{PSh}}
\newcommand{\CAlg}{\cat{CAlg}}
\newcommand{\Vect}{\cat{Vect}}
\newcommand{\Spaces}{\cat{Spaces}}
\renewcommand{\sl}{\mathfrak{sl}}
\renewcommand{\div}{\text{div}}

\DeclareMathOperator{\re}{Re}
\DeclareMathOperator{\id}{id}
\DeclareMathOperator{\St}{St}
\DeclareMathOperator{\coker}{coker}
\DeclareMathOperator{\Frac}{Frac}
\DeclareMathOperator{\Tr}{Tr}
\DeclareMathOperator{\tr}{\Tr}
\DeclareMathOperator{\im}{Im}
\DeclareMathOperator{\diag}{diag}
\DeclareMathOperator{\Hom}{Hom}
\DeclareMathOperator{\Aut}{Aut}
\DeclareMathOperator{\Gal}{Gal}
\DeclareMathOperator{\Spec}{Spec}
\DeclareMathOperator{\cl}{cl}
\DeclareMathOperator{\Char}{char}
\DeclareMathOperator{\mSpec}{mSpec}
\DeclareMathOperator{\Frob}{Frob}
\DeclareMathOperator{\disc}{disc}
\DeclareMathOperator{\covol}{covol}
\DeclareMathOperator{\Div}{Div}
\DeclareMathOperator{\Princ}{Princ}
\DeclareMathOperator{\Pic}{Pic}
\DeclareMathOperator{\Cl}{Cl}
\DeclareMathOperator{\Log}{Log}
\DeclareMathOperator{\ord}{ord}
\DeclareMathOperator{\Spl}{Spl}
\DeclareMathOperator{\cond}{cond}
\DeclareMathOperator{\Li}{Li}
\DeclareMathOperator{\SL}{SL}
\DeclareMathOperator{\SO}{SO}
\DeclareMathOperator{\SU}{SU}
\DeclareMathOperator{\GL}{GL}
\DeclareMathOperator{\U}{U}
\DeclareMathOperator{\PGL}{PGL}
\DeclareMathOperator{\PSL}{PSL}
\DeclareMathOperator{\Ind}{Ind}
\DeclareMathOperator{\Sym}{Sym}
\DeclareMathOperator{\codim}{codim}
\DeclareMathOperator{\End}{End}
\DeclareMathOperator{\Hilb}{Hilb}
\DeclareMathOperator{\supp}{supp}
\DeclareMathOperator{\ch}{ch}
\DeclareMathOperator{\cHilb}{\mathcal{H}ilb}
\DeclareMathOperator{\Tan}{Tan}
\DeclareMathOperator{\Proj}{Proj}
\DeclareMathOperator{\Lie}{Lie}
\DeclareMathOperator{\Stab}{Stab}
\DeclareMathOperator{\ad}{ad}
\DeclareMathOperator{\Ad}{Ad}
\DeclareMathOperator{\gl}{\mathfrak{gl}}
%\DeclareMathOperator{\sl}{\mathfrak{sl}}
\DeclareMathOperator{\so}{\mathfrak{so}}
\DeclareMathOperator{\Der}{Der}
\DeclareMathOperator{\Gr}{Gr}
\DeclareMathOperator{\NS}{NS}
\DeclareMathOperator{\Span}{Span}
\DeclareMathOperator{\len}{len}

\begin{document}

\section{Cartier and Weil divisors}
Let $X$ be a variety of dimension $n$ over a field $k$. We want to introduce two notions of divisors, one familiar from the last chapter.

\begin{defn}
A {\em Weil divisor} of $X$ is an $n-1$-cycle on $X$, i.e. a finite formal linear combination of codimension $1$ subvarieties of $X$. Thus the Weil divisors form a group $Z_{n-1} X$.
\end{defn}

\begin{defn}
A {\em Cartier divisor} consists of the following data:
\begin{itemize}
\item an open cover $\{U_\alpha\}$ of $X$;
\item for each $\alpha$ a nonzero rational function $f_\alpha$ on $U_\alpha$, defined up to multiplication by a {\em unit}, i.e. a function without zeros or poles, such that for any $\alpha, \beta$ we have $f_\alpha/f_\beta$ a unit on $U_\alpha \cap U_\beta$.
\end{itemize}
\end{defn}

Like the Weil divisors, the Cartier divisors form an abelian group: $(\{U_\alpha, f_\alpha\}) + (\{U_\alpha, g_\alpha\}) = (\{U_\alpha, f_\alpha g_\alpha\})$ (we can assume that the open covers are the same, since if not they refine to $\{U_\alpha \cap V_\beta\}$). We call this abelian group $\Div X$.

Given a Cartier divisor $D = (\{U_\alpha, f_\alpha\})$ and a codimension $1$ subvariety $V$ of $X$, we define \[ \ord_V D = \ord_V (f_\alpha) \] for $\alpha$ such that $U_\alpha \cap V$ is nonempty; since each $f_\alpha$ is defined up to a unit, this order is well-defined. We define the {\em associated Weil divisor} \[ [D] = \sum_V \ord_V D \cdot [V] . \] This defines a homomorphism \[ \Div X \to Z_{n-1} X. \]

For any rational function $f$ on $X$, we get a {\em principal} Cartier divisor $\div(f)$ by choosing any cover $\{U_\alpha\}$ and defining $f_\alpha = f|_{U_\alpha}$. It is immediate that the image $[\div(f)]$ of this divisor under the map to $Z_{n-1}X$ is the Weil principal divisor. Say that two Cartier divisors $D$ and $D'$ are {\em linearly equivalent} if $D - D' = \div(f)$ for some $f$; then we define $\Pic X$ to be the group of Cartier divisors modulo linear equivalence, and the above then shows that the map $\Div X \to Z_{n-1} X$ descends to a map $\Pic X \to A_{n-1} X$. This map is in general neither injective nor surjective.

Notice that the definition of a Cartier divisor yields that of a line bundle on $X$: given a divisor $D = (\{U_\alpha, f_\alpha\})$, define a line bundle $L = \O(D)$ to be trivialized on each $U_\alpha$ with transition functions $f_\alpha/f_\beta$. Two Cartier divisors $D$ and $D'$ are linearly equivalent if and only if $\O(D) = \O(D')$, and so we get the alternate description of $\Pic X$ as the abelian group of line bundles on $X$ with group operation given by the tensor product. Conversely, given a line bundle $L$, this determines a Cartier divisor $D(L)$ up to some additional data: a nonzero rational section $s$ of $L$. Therefore we can also think of Cartier divisors as the data of a line bundle together with a nonzero rational section.

We define the {\em support} $\supp D$ or $|D|$ of a Cartier divisor $D$ to be the union of codimension $1$ subvarieties $V$ of $X$ such that $f_\alpha$ is not a unit for $U_\alpha$ nontrivially intersecting $V$, i.e. $\ord_V D$ is nonzero.

We say that a Cartier divisor $D = (\{U_\alpha, f_\alpha\})$ if all of the $f_\alpha$ are regular, i.e. have no poles.

\section{Pseudo-divisors}
In general, Cartier divisors are not well-behaved under pullbacks (although line bundles are). In particular, given the data of a line bundle $L$ and a nonzero rational section $s$ and a morphism $f: Y \to X$, there is no guarantee that the pullback $f^* s$ is nonzero. Therefore we enlarge the notion to make it behave better: let $L$ be a line bundle on $X$, $Z \subset X$ be a closed subset, and $s$ be a nowhere vanishing section of $L$ restricted to $X-Z$, or equivalently a trivialization of $L|_{X-Z}$. A {\em pseudo-divisor} on $X$ consists of the data of such a triple $(L,Z,s)$, up to the following equivalence: two triples $(L,Z,s)$ and $(L',Z',s')$ define the same pseudo-divisor if $Z = Z'$ and there exists an isomorphism $\sigma: L \to L'$ such that restricted to $X-Z$ we have $\sigma \circ s = s'$. Note that this is well-behaved under pullback.

\begin{ex}
Let $D = (\{U_\alpha, f_\alpha\})$ be a Cartier divisor, with support $|D|$. Then each $f_\alpha$ away from $|D$ gives a local section of the associated line bundle $\O(D)$, and so these glue to a section $s_D$ of $\O(D)$ on $X-|D|$; this makes $(\O(D), |D|, s_D)$ a pseudo-divisor.

We say that a Cartier divisor $D$ {\em represents} a pseudo-divisor $(L,Z,s)$ when $|D| \subseteq Z$ and there exists an isomorphism $\sigma: \O(D) \to L$ such that restricted to $X-Z$ we have $\sigma \circ s_D = s$, with notation as above.
\end{ex}

\begin{lem}
If $X$ is a variety, then every pseudo-divisor $(L,Z,s)$ on $X$ is represented by a Cartier divisor $D$. If $Z \subsetneq X$, then $D$ is unique; if $Z = X$, then $D$ is unique up to linear equivalence.
\end{lem}
\begin{proof}
If $Z = X$, then $s$ is a section on $X - X = \{\}$, and so a pseudo-divisor is just a line bundle; and we saw in the previous section that the group of Cartier divisors up to linear equivalence is isomorphic to the group of line bundles, so $L$ corresponds to a unique linear equivalence class of Cartier divisors.

If $Z \neq X$, let $U = X - Z$. As above, choose a Cartier divisor $D = (\{U_\alpha, f_\alpha\})$ with $\O(D) \simeq L$. The section $s$ consists of a collection of functions $s_\alpha$ on $U \cap U_\alpha$ such that $s_\alpha = f_\alpha/f_\beta \cdot s_\beta$ on $U \cap U_\alpha \cap U_\beta$; thus $s_\alpha/f_\alpha = s_\beta/f_\beta$ on each intersection, i.e. there exists some rational function $r$ such that $s_\alpha / f_\alpha = r$ on each $U \cap U_\alpha$. Then $D' : = D + \div(r)$ is the Cartier divisor $(\{U_\alpha, f_\alpha r\})$ and by definition $f_\alpha r = s_\alpha$ on each $U \cap U_\alpha$; therefore using the definition above $s_{D'} = s$. Since $D'$ is linearly equivalent to $D$, it corresponds to the same line bundle, and since $r$ is regular on each $U_\alpha$ the support of $\div(r)$ is contained in $Z$; therefore $D'$ represents $(L,Z,s)$.

For uniqueness, suppose that two Cartier divisors $D_1 = (\{U_\alpha, f_\alpha\})$ and $D_2 = (\{V_\beta, g_\beta\})$ both represent $(L,Z,s)$. Then similarly there must exist some rational function $r$ such that $r f_\alpha = r g_\beta$ on each $U_\alpha \cap V_\beta$. But since $s_{D_1} = s_{D_2} = s$, if $Z \neq X$, i.e. $U$ is nonempty, then $s_{D_1}$ and $s_{D_2}$ must agree on every $U \cap U_\alpha \cap V_\beta$, and so $r$ restricted to $U$ must be $1$; since $f$ is rational it follows that $f = 1$ and $D_1 = D_2$.
\end{proof}

For any pseudo-divisor $D = (L,Z,s)$, as for Weil divisors we will write $\O(D) = L$, $|D| = Z$, and $s_D = s$.

If $D = (L,Z,s)$ and $D' = (L', Z', s')$ are two pseudo-divisors, we can define their sum \[ D + D' = (L \tensor L', Z \cup Z', s \tensor s') . \] This agrees with the sum on Cartier divisors, except that the supports may be larger in this case. Similarly defining \[ -D = (L^{-1}, Z, s^{-1}) \] makes the set of pseudo-divisors into an abelian group.

Given a pseudo-divisor $D$ on a variety $X$ of dimension $X$, we can define the Weil class divisor $[D]$ by taking $\tilde D$ to be the Cartier divisor which represents $D$ and setting $[D] := [\tilde D]$, the associated Weil divisor from the previous section. The above lemma shows that this yields a well-defined element of $A_{n-1} X$; this gives a homomorphism from the group of pseudo-divisors to $A_{n-1} X$.

\section{Intersecting with divisors}
Let $X$ be a variety of dimension $n$, $D$ be a pseudo-divisor on $X$, and $V$ be a subvariety of dimension $k$. Let $j: V \inject X$ be the inclusion of $V$ into $X$; then the pullback $j^* D$ is a pseudo-divisor on $V$ with support $V \cap |D|$. We define the class $D \cdot [V]$ in $A_{k-1} (V \cap |D|)$ given by the Weil class divisor of $j^* D$: \[ D \cdot [V] = [j^* D] . \] For any closed subscheme $Y \subset X$ containing $V \cap |D|$, we can also view this as an element of $A_{k-1} Y$; we will also denote this by $D \cdot [V]$.

Let $\alpha = \sum_V n_V \cdot V$ be a $k$-cycle on $X$, with support $|\alpha$ the union of the subvarieties $V$ such that $n_V$ is nonzero. For a pseudo-divisor $D$ on $X$, we define the {\em intersection class} $D \cdot \alpha$ in $A_{k-1}(V \cap |D|)$ by \[ D \cdot \alpha = \sum_V n_V \cdot (D \cdot [V]) . \] As above, we can also view this as an element of $A_{k-1} Y$ for any $Y$ containing $|\alpha| \cap |D|$.

We will apply this in two main cases. First: f $|D| = X$, then the data of $D = (L, X, s)$ is just that of a line bundle as above; in this case the action of $D$ on a $k$-cycle $\alpha$ is called that of the first Chern class, written $D \cdot \alpha = c_1(L) \cap \alpha$.

Second: if $i: |D| \inject X$ is the inclusion of $|D|$ into $X$, then $D \cdot \alpha$ is called the Gysin pullback $i^* \alpha$.

\begin{thm} \label{properties}
Let $X$ be a scheme, $D$ be a pseudo-divisor on $X$, and $\alpha$ be a $k$-cycle on $X$.
\begin{enumerate}[{{\em(\hspace{-1pt}}}a{\em)}]
\item Let $\alpha'$ be a $k$-cycle on $X$. Then \[ D \cdot (\alpha + \alpha') = D \cdot \alpha + D \cdot \alpha' \] in $A_{k-1} ((|\alpha| \cup |\alpha'|) \cap |D|)$.
\item Let $D'$ be a pseudo-divisor on $X$. Then \[ (D + D') \cdot \alpha = D \cdot \alpha + D' \cdot \alpha \] in $A_{k-1}(|\alpha| \cap (|D| \cup |D'|))$.
\item Let $f: Y \to X$ be a proper morphism, $\beta$ be a $k$-cycle on $Y$, and $g: |\beta| \cap f^{-1}(|D|) \to f(|\beta|) \cap |D|$ be the restriction of $f$ to $|\beta| \cap f^{-1}(|D|)$. Then \[ g_* (f^* D \cdot \beta) = D \cdot f_* \beta \] in $A_{k-1}(f(|\beta|) \cap |D|)$.
\item Let $f: Y \to X$ be a flat morphism of relative dimension $n$ and $g: f^{-1}(|\alpha| \cap |D|) \to |\alpha| \cap |D|$ be the restriction of $f$ to $f^{-1}(|\alpha| \cap |D|)$. Then \[ f^* D \cdot f^* \alpha = g^* (D \cdot \alpha) \] in $A_{n+k-1}(f^{-1}(|\alpha| \cap |D|))$.
\item If the line bundle $\O(D)$ is trivial, then \[ D \cdot \alpha = 0 \] in $A_{k-1}(|\alpha| \cap |D|)$.
\end{enumerate}
\end{thm}
\begin{proof}
Part (a) is immediate from the definition. Using part (a), then, we can assume by linearity that $\alpha = [V]$ for some $k$-dimensional subvariety $V \subset X$. Restricting to $V$, (b) is just the statement that taking the Weil class divisor is compatible with sums.

For part (c), we can likewise assume that $\beta = [W]$ for some $k$-dimensional subvariety $W \subset Y$; then $f^* D \cdot \beta$ is the restriction of the Cartier divisor $f^* \tilde D$ representing $f^* D$ to $W$, and so we can assume that $Y = W$. Similarly on the right-hand side $D \cdot f_* \beta = D \cdot \deg(f(W)/W) [f(W)]$ and so concerns only the restriction of $D$ to $f(W)$, and so we can assume that $f(W) = X$. In this case $g = f$ on the support of $D$ and so the statement is \[ f_* (f^* [D]) = \deg(W/f(W)) [D] \] since $D \cdot [X] = [D]$ and $f^* D \cdot [Y] = f^* [D]$. If $f$ is a map of degree $d$ and $D = \div(r)$ for some function $r$ on some open subset of $f(W)$, then from last time we know that locally \[ f_* [\div(f^* r)] = [\div(N(f^* r))] = d [\div(r)] \] where $N$ is the determinant map from functions on subsets of $W$ to functions on their images, since $N(f^* r) = dr$ since $f$ has degree $d$. But locally we can always assume that $[D]$ is principal, and so $f_* f^* [D] = d [D]$ as desired.

For (d), we can again assume that $\alpha = [V] = [X]$, so the statement similarly becomes \[ [f^* D] = f^* [D] . \] By linearity, we can assume $D = [W]$ for some subvariety $W$ of $X = V$, at which point the statement is $f^* [W] = [f^{-1}(W)]$, which is true whenever $f$ is flat.

Finally for (e) we can again assume $\alpha = [V] = [X]$, so that the statement is $[D] = 0$ in $A_{n-1} X$ whenever $\O(D)$ is trivial, where $n$ is the dimension of $V = X$. Letting $\tilde D$ be the Cartier divisor representing $D$, we know from section 1 that $\O(D)$ is trivial precisely when $\tilde D$ is linearly equivalent to the trivial Cartier divisor $0 = (\{U_\alpha, 1\})$ for which every local function is a unit; and we know that the associated Weil divisor map $\Div X \to Z_{n-1} X$ descends to a map $\Pic X \to A_{n-1} X$, i.e. $[D] = [\tilde D] = 0$ whenever $\O(D)$ is trivial.
\end{proof}

\section{Commutativity}
Suppose that we have two Cartier divisors $D, D'$ on an $n$-dimensional variety $X$. Then they both determine associated Weil divisors $[D], [D'] \in Z_{n-1} X$ (and thus in $A_{n-1} X$), and so it is natural to consider the intersections \[ D \cdot [D'], \qquad D' \cdot [D] . \]

\begin{thm} \label{commutativity}
In $A_{n-2}(|D| \cap |D'|)$, we have \[ D \cdot [D'] = D' \cdot [D] . \]
\end{thm}
%add proof later

\begin{cor} \label{preserving}
Let $D$ be a pseudo-divisor on a scheme $X$, and $\alpha$ be a $k$-cycle on $X$ rationally equivalent to $0$. Then \[ D \cdot \alpha = 0 \] in $A_{k-1} (|D|)$.
\end{cor}
\begin{proof}
We can assume without loss of generality that $\alpha = [\div (f)]$ for some rational function $f$ on a subvariety $V$ of $X$. Then letting $\tilde D$ be the Cartier divisor representing $D$ we can replace $D$ with $\tilde D$ and $X$ with $V$ without changing the result; then we can apply Theorem \ref{commutativity} to get \[ D \cdot \alpha = \tilde D \cdot [\div (f)] = \div (f) \cdot [\tilde D] . \] But by part (e) of Theorem \ref{properties}, we have $\div (f) \cdot [\tilde D] = 0$.
\end{proof}

Given a closed subscheme $Y \subset X$ and a $k$-cycle $\alpha$ on $Y$, we can construct its intersection $D \cdot \alpha \in A_{k-1}(Y \cap |D|)$ for any pseudo-divisor $D$ on $X$. This gives a map \[ Z_k Y \to A_{k-1}(Y \cap |D|) . \] The above corollary shows that in fact this map descends to a map \[ A_k Y \to A_{k-1} (Y \cap |D|) ; \] this is called {\em intersecting} with $D$.

\begin{cor}
For two pseudo-divisors $D, D'$ on a scheme $X$ and a $k$-cycle $\alpha$ on $X$, we have \[ D \cdot (D' \cdot \alpha) = D' \cdot (D \cdot \alpha) \] in $A_{k-2}(|\alpha| \cap |D| \cap |D'|)$.
\end{cor}
\begin{proof}
We can assume without loss of generality that $\alpha = [V]$ for some subvariety $V \subseteq X$ of dimension $k$. Then we can restrict $D$ and $D'$ to $V$, so that $D' \cdot [V] = [\id^* D'] = [D']$ and similarly $D \cdot [V] = [D]$; and then applying Theorem \ref{commutativity} immediately gives the result.
\end{proof}

For pseudo-divisors $D_1, \ldots, D_n$ on $X$ and a $k$-cycle $\alpha$ on $X$, we can then define inductively \[ D_1 \cdots D_n \cdot \alpha = D_1 \cdot (D_2 \cdots D_n \cdot \alpha) \] in $A_{k-n} (|\alpha| \cap (|D_1| \cup \cdots \cup |D_n|))$. Theorem \ref{commutativity} implies that the order of the $D_i$ is unimportant, and parts (a) and (b) of Theorem \ref{properties} implies that the action is linear in each $D_i$ and in $\alpha$. More generally if $p(t_1, \ldots, t_n)$ is a homogeneous polynomial of degree $d$ and $Z$ is a closed subscheme of $X$ containing $|\alpha| \cap (|D_1| \cup \cdots \cup |D_n|)$, then we can define $p(D_1, \ldots, D_n) \cdot \alpha$ in $A_{k-d}(Z)$.

\begin{defn}
We say that an algebraic variety $Y$ is {\em complete} if for any variety $Z$ the projection $Y \times Z \to Y$ is a closed map.
\end{defn}
For example, any projective variety is complete.

If $n = k$ and $Y = |\alpha| \cap (|D_1| \cup \cdots \cup |D_k|)$ is complete, then we can define the {\em intersection number} \[ (D_1 \cdots D_k \cdot \alpha)_X = \int_Y D_1 \cdots D_k \cdot \alpha . \] Similarly if $p$ is a homogeneous polynomial of degree $k$ in $k$ variables then we can define \[ (p(D_1, \ldots, D_k) \cdot \alpha)_X = \int_Y p(D_1, \ldots, D_n) \cdot \alpha . \]

For a subvariety $V$ purely of dimension $k$, we will sometimes write simply $V$ instead of $[V]$; similarly we will sometimes write $D$ instead of $[D]$.

\begin{ex}
Let $X$ be the projective completion of the affine surface $X' \subset \A^3$ defined by $z^2 = xy$. Consider the Cartier divisor $D$ on $X$ defined everywhere by the equation $x$, corresponding to the subvariety cut out by $x = 0$. Define the lines $\ell, \ell'$ by $x = z = 0$ and $y = z = 0$ respectively, and let $P$ be the origin $(0,0,0)$. Along the subvariety $x = 0$, from the defining equation we also have $z = 0$ (in affine space), and so $[D] = \ord_\ell D \cdot [\ell]$; we have \[ \ord_\ell D = \len_A A/(x) , \] where (in the affine variety) $A = \O_{X,\ell} = K[x,y,z]/(z^2 - x y)$. Thus $A/(x) = K[x,y,z]/(z^2-xy,x) = K[y,z]/(z^2)$ which has length $2$, with maximal proper subsequence of modules given by $0 \subset K[y] = K[y,z]/(z) \subset K[y,z]/(z^2)$. Therefore $[D] = 2 [\ell]$.  We can compute \[ D \cdot [\ell'] = [j^* D] = [P] \] where $j$ is the inclusion of $\ell'$ into $X$, since restricted to the line $y = z = 0$ the equation $x=0$ specifies only the point $P$ with multiplicity $1$. Therefore there cannot exist any Cartier divisor $D'$ with $[D'] = [\ell']$, since if there were we would have \[ [P] = D \cdot [\ell'] = D \cdot [D'] = D' \cdot [D] = 2 D' \cdot [\ell] \] in either $Z_1 X$ or $A_1 X$, by Theorem \ref{commutativity} and the above calculation. This proves our above claim that the maps $\Div X \to Z_{\dim X - 1}$ and $\Pic X \to A_{\dim X - 1} X$ are not in general surjective.
\end{ex}

\section{The first Chern class}
Let $X$ be a scheme, $V \subseteq X$ a subvariety of dimension $k$, and $L$ a line bundle on $X$. The restriction of $L$ to $V$ is a line bundle on $V$ and so is isomorphic to $\O(C)$ for some Cartier divisor $C$ on $V$, determined up to linear equivalence. This in turn defines a well-defined element $[C]$ of $A_{k-1} X$; we write $c_1(L) \cap [V] := [C]$. More generally, if $\alpha = \sum_V n_V \cdot [V]$ is a $k$-cycle on $X$ then define $C_V$ for each $V$ as above, and write \[ c_1(L) \cap \alpha := \sum_V n_V \cdot [C_V] . \] If $L = \O(D)$ for some pseudo-divisor $D$, then if $j: V \inject X$ is the inclusion then the Cartier divisor $\tilde D$ on $V$ representing $j^* D$ satisfies $\O(\tilde D) \simeq \O(D)$ by construction; by definition, this means that $[C_V] = [j^* D] = D \cdot [V]$ and so \[ c_1(L) \cap \alpha = D \cdot \alpha \] in $A_{k-1} X$.

\begin{thm} \label{Chern properties}
Let $X$ be a scheme, $L$ be a line bundle on $X$, and $\alpha$ be a $k$-cycle on $X$.
\begin{enumerate}[{{\em(\hspace{-1pt}}}a{\em)}]
\item If $\alpha$ is rationally equivalent to $0$, then $c_1(L) \cap \alpha = 0$. Therefore there is an induced homomorphism $c_1(L) \cap - : A_k X \to A_{k-1} X$.
\item If $L'$ is a second line bundle on $X$, then \[ c_1(L) \cap (c_1(L') \cap \alpha) = c_1(L') \cap (c_1(L) \cap \alpha) \] in $A_{k-2}X$.
\item If $f: Y \to X$ is a proper morphism and $\beta$ is a $k$-cycle on $Y$, then \[ f_* (c_1(f^* L) \cap \beta) = c_1(L) \cap f_* \beta \] in $A_{k-1} X$.
\item If $f: Y \to X$ is a flat morphism of relative dimension $n$, then \[ c_1(f^* L) \cap f^* \alpha = f^* (c_1(L) \cap \alpha) \] in $A_{n+k-1} Y$.
\item If $L'$ is a second line bundle on $X$, then \[ c_1(L \tensor L') \cap \alpha = c_1(L) \cap \alpha + c_1(L') \cap \alpha \] and \[ c_1(L^{-1}) \cap \alpha = - c_1(L) \cap \alpha \] in $A_{k-1} X$.
\end{enumerate}
\end{thm}
\begin{proof}
A line bundle on $X$ defines a pseudo-divisor with support $X$, and so the analogous properties from Theorem \ref{properties} and its corollaries immediately imply these.
\end{proof}

\section{The Gysin map}
Fix an effective Cartier divisor $D$ on a scheme $X$, with the inclusion given by $i: |D| \inject X$. Then we define the ``Gysin homomorphism" \[ i^* \alpha : = D \cdot \alpha \] for $k$-cycles $\alpha$ on $X$.

\begin{prop}
With notation as above:
\begin{enumerate}[{{\em(\hspace{-1pt}}}a{\em)}]
\item If $\alpha$ is rationally equivalent to $0$, then $i^* \alpha = 0$, and so there is an induced homomorphism $i^*: A_k X \to A_{k-1} (|D|)$.
\item We have \[ i_* i^* \alpha = c_1(\O(D)) \cap \alpha . \]
\item If $\beta$ is a $k$-cycle on $|D|$, then \[ i^* i_* \beta = c_1(i^* \O(D)) \cap \beta . \]
\item If $X$ is purely $n$-dimensional, then \[ i^* [X] = [D] \] in $A_{n-1} (|D|)$.
\item If $L$ is a line bundle on $X$, then \[ i^* (c_1(L) \cap \alpha) = c_1(i^* L) \cap i^* \alpha \] in $A_{k-2} (|D|)$.
\end{enumerate}
\end{prop}
All of these follow immediately from the definitions and the results above.

\end{document}