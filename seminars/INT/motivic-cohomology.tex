\documentclass[a4paper, 11pt]{article} % A4 paper size and default 11pt font size

\usepackage[svgnames]{xcolor} % Required for colour specification

\newcommand*{\plogo}{\fbox{$\mathcal{PL}$}} % Generic dummy publisher logo

\usepackage[utf8]{inputenc} % Required for inputting international characters
\usepackage[T1]{fontenc} % Output font encoding for international characters
\usepackage{PTSerif} % Use the Paratype Serif font

\usepackage{amsmath}
\usepackage{amssymb}
\usepackage{amsthm}
\usepackage{blindtext}
\usepackage[margin = 1 in]{geometry}
\usepackage{mathtools}
\usepackage{graphicx}
\usepackage{tikz-cd}
\usepackage{quiver}\usepackage{footnote}
\usepackage{fancyhdr}
\usepackage{hyperref}
\usepackage{stmaryrd}
\usepackage{comment}
\usepackage[normalem]{ulem}
\hypersetup{
    colorlinks,
    %citecolor=black,
    %filecolor=black,
    linkcolor=blue,
    urlcolor=blue
}

\pagestyle{fancy}
\setlength{\headheight}{25pt}

\newtheorem{theorem}{Theorem}[section]
\newtheorem{corollary}[theorem]{Corollary}
\newtheorem{proposition}[theorem]{Proposition}
\newtheorem{lemma}[theorem]{Lemma}
\newtheorem{definition}[theorem]{Definition}
\newtheorem{exercise}[theorem]{Exercise}
\newtheorem{example}[theorem]{Example}
\theoremstyle{remark}
\newtheorem*{remark}{Remark}
\newtheorem*{remarks}{Remarks}
\newtheorem*{question}{Question}
\newtheorem*{fact}{Fact}

\graphicspath{ {mot/} }

\newcommand{\?}{{\color{red} ???}} 
\newcommand{\mb}{\mathbb}
\newcommand{\mc}{\mathcal}
\newcommand{\mf}{\mathfrak}
\newcommand{\mr}[1]{\mathrm{#1}}
\newcommand{\ms}[1]{\mathsf{#1}}
\newcommand{\yog}{\text{\yogh}}
\newcommand{\OX}{\mathcal{O}_X}
\newcommand{\sep}{\text{sep}}
\newcommand{\Q}{\mb{Q}}
\newcommand{\R}{\mb{R}}
\newcommand{\Z}{\mb{Z}}
\newcommand{\C}{\mb{C}}
\newcommand{\A}{\mb{A}}
\newcommand{\F}{\mb{F}}
\newcommand{\N}{\mb{N}}
\newcommand{\G}{\mb{G}}
\renewcommand{\S}{\mb{S}}
\renewcommand{\H}{\mb{H}}
\renewcommand{\P}{\mb{P}}
\newcommand{\p}{\mb{p}}
\newcommand*{\sheafhom}{\mathcal{H}\kern -.5pt om}
\newcommand{\Ca}{\mb{C}\bf{a}}
\newcommand{\bs}{\boxslash}

\newcommand{\setting}[2]{\textit{Lecturer: #1, Date: #2 }\newline}

\newcommand{\catname}[1]{{\normalfont\textbf{#1}}}
\newcommand{\cat}{\catname}
\newcommand{\Set}{\catname{Set}}
\newcommand{\sSet}{\catname{sSet}}
\newcommand{\Rel}{\catname{Rel}}
\newcommand{\SW}{\catname{SW}}
\newcommand{\Spectra}{\catname{Spectra}}
\newcommand{\Sp}{\catname{Sp}}
\newcommand{\Topo}{\catname{Top}}
\newcommand{\Spaces}{\catname{Spaces}}
\newcommand{\Spacesp}{\catname{Spaces}_*}
\newcommand{\M}{\catname{M}}
\newcommand{\Ch}{\catname{Ch}}
\newcommand{\hTop}{\catname{hTop}}
\newcommand{\CW}{\catname{CW-complexes}}
\newcommand{\Vect}{\catname{Vect}}
\newcommand{\Fred}{\catname{Fred}}
\newcommand{\Sm}{\catname{Sm}}
\newcommand{\Cor}{\catname{Cor}}
\newcommand{\PST}{\catname{PST}}


\DeclareMathOperator{\Tr}{Tr}
\DeclareMathOperator{\Stab}{Stab}
\DeclareMathOperator{\Lie}{Lie}
\DeclareMathOperator{\image}{image}
\DeclareMathOperator{\Hom}{Hom}
\DeclareMathOperator{\Ho}{Ho}
\DeclareMathOperator{\age}{age}
\DeclareMathOperator{\codim}{codim}
\DeclareMathOperator{\Ker}{Ker}
\DeclareMathOperator{\Gr}{Gr}
\DeclareMathOperator{\Crit}{Crit}
\DeclareMathOperator{\Des}{Des}
\DeclareMathOperator{\sgn}{sgn}
\DeclareMathOperator{\Spec}{Spec}
\DeclareMathOperator{\Mor}{Mor}
\DeclareMathOperator{\id}{id}
\DeclareMathOperator{\cov}{COV}
\DeclareMathOperator{\tra}{TRA}
\DeclareMathOperator{\im}{im}
\DeclareMathOperator{\colim}{colim}
\DeclareMathOperator{\Tor}{Tor}
\DeclareMathOperator{\Ext}{Ext}
\DeclareMathOperator{\Bord}{Bord} 
\DeclareMathOperator{\Maps}{Maps}
\DeclareMathOperator{\pr}{pr} 
\DeclareMathOperator{\op}{op} 
\DeclareMathOperator{\Ab}{Ab}
\DeclareMathOperator{\Th}{Th}
\DeclareMathOperator{\st}{st} 
\DeclareMathOperator{\hocolim}{hocolim}
\DeclareMathOperator{\Cyl}{Cyl} 
\DeclareMathOperator{\cof}{cof}
\DeclareMathOperator{\fib}{fib} 
\DeclareMathOperator{\supp}{supp} 
\DeclareMathOperator{\ch}{ch}  
\DeclareMathOperator{\td}{td}  
\DeclareMathOperator{\ind}{ind}  
\DeclareMathOperator{\coker}{coker}
\DeclareMathOperator{\sign}{sign} 
\DeclareMathOperator{\vol}{vol} 
\DeclareMathOperator{\di}{div} 
\DeclareMathOperator{\grad}{grad}  
\DeclareMathOperator{\pt}{pt}  
\DeclareMathOperator{\CH}{CH}
\DeclareMathOperator{\Ob}{Ob}
\DeclareMathOperator{\Rat}{Rat} 
\DeclareMathOperator{\len}{length}  

\usepackage[framemethod=tikz]{mdframed}
\usepackage{lipsum}

\definecolor{mycolor}{rgb}{0.122, 0.435, 0.698}

\newmdenv[innerlinewidth=0.5pt, roundcorner=4pt,linecolor=mycolor,innerleftmargin=6pt,
innerrightmargin=6pt,innertopmargin=6pt,innerbottommargin=6pt]{mybox}


% Makes section headers smaller
\usepackage{sectsty}
%\sectionfont{\fontsize{12}{15}\selectfont}


\lhead{} 
\chead{Motivic cohomology}
\rhead{} 
\title{A whirlwhind tour of motivic cohomology}
\author{Notes by Caleb Ji}
\date{}

\begin{document}
%----------------------------------------------------------------------------------------
%	TITLE PAGE
%----------------------------------------------------------------------------------------



%----------------------------------------------------------------------------------------
\maketitle 

These are my notes on motivic cohomology.  Essentially everything here is based off of Voevodsky's lectures, now turned into a book by Mazza and Weibel. \cite{Voevodsky} 

\tableofcontents 

\section{Bloch's higher Chow groups}

\subsection{Some topological motivation} 
Recall the following exact sequence. 

\begin{proposition}
[Fulton, Prop. 1.8] Let $Y$ be a closed subscheme of a scheme $X$, and let $U=X-Y$.  Let $i:Y\rightarrow X, j:U\rightarrow X$ be the inclusions.  Then the sequence 
\[
\CH^kY\xrightarrow{i_*}\CH^kX\xrightarrow{j^*}\CH^kU\rightarrow 0
\]
is exact for all $k$.
\end{proposition} 

This is great, but if we want homology to continue to the left.  We certainly cannot put these sequences together.  In fact, the indexing is a bit misleading in this way -- instead, we should put the $k$s together for a grading of $\CH^*(-)$.  So, we really do need new groups if we want to continue this sequence.  To construct these groups, we first recall the following definition of rational equivalence. 

\begin{definition}
[$\Rat(X)$]
Let $Z(X)$ denote the cycles of a scheme $X$ and let $\Phi$ be any subvariety of $X\times \P^1$.  Then we define $\Rat(X)\subset Z(X)$ be the subgroup generated by differences of the form 
\[
[\Phi\cap (X\times \{0\})] - [\Phi\cap (X\times \{\infty\})].
\]
\end{definition}
Then rationally equivalent cycles are those which differ by something in $\Rat(X)$.  We see that it looks like there is a homotopy between rationally equivalent cycles. \\  

\subsection{Definition}

Motivated by algebraic topology, we define the algebraic simplex 
\[
\Delta^k = \Spec k[x_0, \ldots, x_n]/(x_0+\cdots +x_k-1).
\] 

Let $z^i(X, n)$ be the subgroup of $Z^i(X\times \Delta^n)$ that meet all faces properly.   This gives both a simplicial abelian group $z^i(X, \bullet)$ and a chain complex $z^i(X, *)$.  

\begin{definition}
The higher Chow groups $\CH^i(X, m)$ are defined 
\[
\CH^i(X, m)\coloneqq\pi_m(z^i(X, \bullet))=H_m(z^i(X, *)).
\] 
\end{definition}


\subsection{Properties}

\begin{enumerate}
\item Homotopy invariance: 

The projection $X\times \A^1\rightarrow X$ induces an isomorphism 
\[
\CH^i(X, m)\cong \CH^i(X\times \A^1, m).
\] 

\item Long exact sequence: 

There is a distinguished triangle 
\[
z_p(Y, *)\rightarrow z_p(X, *)\rightarrow z_p(U, *)\rightarrow z_p(Y, *)[1].
\]

\item Isomorphism with rational K-theory: 
\[
(K_i(X)\otimes \Q)^{(q)}\cong \CH^q(X, i)\otimes \Q.
\]
\end{enumerate}

We will see that $H^{p, q}(X; A)=\CH^q(X, 2q-p; A)$.  In particular, we have $H^{2q, q}(X, A)=\CH^q(X)\otimes A$. 



\section{The category of correspondences} 
\subsection{Correspondences}
Let $X, Y\in \Sm_k$ be smooth separated schemes of finite type over $k$.  Very informally, one can think of $\Cor(X, Y)$ as a generalization of $\Mor(X, Y)$ to multivalued morphisms. 

\begin{definition}
An \textbf{elementary correspondence} between a smooth connected scheme $X/k$ to a separated scheme $Y/k$ is an irreducible closed subset $W\subset X\times Y$ whose associated integral subscheme is finite and surjective over $X$. \\ 

If $X$ is not connected, then an elementary correspondence refers to one that is one from a connected component of $X$ to $Y$.   \\ 

The group $\Cor(X, Y)$ of \textbf{finite correspondences} is the free abelian group generated by the elementary correspondences.
\end{definition} 

Then given a closed subscheme $Z\subset X\times Y$ finite and surjective over $X$, we can associate the finite correspondence $\sum n_iW_i$ where $W_i$ are the irreducible components of the support of $Z$ surjective over a component of $X$ with generic points $\xi_i$ and $n_i =\len O_{Z, \xi_i}$. \\ 

\subsection{The category of correspondences}

We compose correspondences $V\in\Cor_k(X, Y)$ and $W\in \Cor_k(Y, Z)$ as follows.  Construct the cycle $[T] = (V\times Z)\cdot (X\times W)$ on $X\times Y\times Z$.  Then take its pushforward along the projection $p:X\times Y\times Z\rightarrow X\times Z$. \\ 

It is not difficult to check that $\Sm_k$ embeds into $\Cor_k$ as a subcategory, where $f:X\rightarrow Y$ becomes the graph $\Gamma_f\subset X\times Y$. \\ 

Furthermore, $\Cor_k$ is a symmetric monoidal category.  Indeed, the tensor product is simply $X\otimes Y=X\times Y$.  Given $V\in \Cor_k(X, X')$ and $W\in \Cor_k(Y, Y')$, we get the desired cycle $V\times W\in \Cor_k(X\otimes Y, X'\otimes Y')$. 

\subsection{Examples} 
\begin{enumerate}
\item $\Cor_k(\Spec k, X)$ is generated by the $0$-cycles of $X$. 

\item $\Cor_k(X, \Spec k)$ is generated by the irreducible components of $X$. 

\item Take $W\in \Cor_k(\A^1, X)$ and two $k$-points $s, t:\Spec k\rightarrow \A^1$.  Then the zero-cycles $W\circ \Gamma_s$ and $W\circ \Gamma_t$ are rationally equivalent. 
\end{enumerate}


\section{Presheaves with transfers} 
\subsection{Definition}
\begin{definition}
A \textbf{presheaf with transfers} is a contravariant additive functor $F:\Cor_k\rightarrow\Ab$.
\end{definition} 

Additivity gives a map 
\[
\Cor_k(X, Y)\otimes F(Y)\rightarrow F(X).
\]

Thus there are extra ``transfer maps" $F(Y)\rightarrow F(X)$ coming from $\Cor_k(X, Y)$. 

\begin{theorem}
$\PST(k)$ is an abelian category with enough injectives and projectives. 
\end{theorem}

\subsection{Examples}

\begin{example}
The constant presheaf $A$ on $\Sm_k$ can be extended to a pst.  

For $W\in \Cor(X, Y)$ with $X, Y$ connected, the corresponding homomorphism $A\rightarrow A$ is multiplication by the degree of $W$ over $X$. 
\end{example}

\begin{example}
$\mc{O}^*$ and $\mc{O}$, at least for $X$ normal.  Use the norm and trace maps. 

% https://q.uiver.app/?q=WzAsMyxbMCwwLCJcXG1je099XiooWSkiXSxbMiwwLCJcXG1je099XiooWCkiXSxbMSwxLCJcXG1je099XiooVykiXSxbMCwxXSxbMCwyXSxbMiwxLCJOIiwyXV0=
\[\begin{tikzcd}
	{\mc{O}^*(Y)} && {\mc{O}^*(X)} \\
	& {\mc{O}^*(W)}
	\arrow[from=1-1, to=1-3]
	\arrow[from=1-1, to=2-2]
	\arrow["N"', from=2-2, to=1-3]
\end{tikzcd}\]

% https://q.uiver.app/?q=WzAsMyxbMCwwLCJcXG1je099XiooWSkiXSxbMiwwLCJcXG1je099XiooWCkiXSxbMSwxLCJcXG1je099XiooVykiXSxbMCwxXSxbMCwyXSxbMiwxLCJOIiwyXV0=
\[\begin{tikzcd}
	{\mc{O}(Y)} && {\mc{O}(X)} \\
	& {\mc{O}(W)}
	\arrow[from=1-1, to=1-3]
	\arrow[from=1-1, to=2-2]
	\arrow["\Tr"', from=2-2, to=1-3]
\end{tikzcd}\]
\end{example}

\begin{example}
$\CH^i(-)$, the Chow groups.
\end{example}

\begin{example}
Representable functors: $h_X(-)$
\end{example} 

\subsection{Representable functors of $\Cor_k(X)$} 
Take $X\in \Ob(\Cor_k(X))$.  We denote
\[
\Z_{tr}(X)\coloneqq h_X(-).
\]

By Yoneda, $\Z_{tr}(X)$ is a projective object in $\PST(k)$. \\ 

Note that $\Z_{tr}(\Spec k)$ is just the constant sheaf $\Z$ on $\Sm_k$, with the transfer maps constructed in the example from the previous subsection.  Let $(X, x)$ be a pointed scheme.  We define 
\[
\Z_{tr}(X, x)\coloneqq \coker [x_*:\Z\rightarrow \Z_{tr}(X)].
\]
The structure map $X\rightarrow\Spec k$ provides a splitting, so 
\[
\Z_{tr}(X)\cong \Z\oplus \Z_{tr}(X, x).
\]

Out of laziness we screenshot the following definitions from Voevodsky's lectures.

\includegraphics[scale=0.5]{20.png}

\includegraphics[scale=0.5]{21.png} 

Consider the pointed scheme $(\G_m, 1)$.  We will be interested in the presheaf with transfers $\Z_{tr}(\G_m^{\wedge q})$. \\ 

Before continuing, we recall our construction \[
\Delta^k = \Spec k[x_0, \ldots, x_n]/(x_0+\cdots +x_k-1).
\] 

Recall that a simplicial object of a category $C$ is a functor $F:\Delta^{op}\rightarrow C$.
Then if $F$ is a presheaf of abelian groups on $\Sm_k$, then $F(U\times\Delta^\bullet)$ is a simplicial abelian group.  Then 
\[
C_\bullet F: U\mapsto F(U\times \Delta^\bullet)
\]
is a simplicial presheaf with transfers.  Similarly, $C_*F(U)$ gives the complex of abelian groups 
\[
\cdots\rightarrow F(U\times \Delta^2)\rightarrow F(U\times \Delta^1)\rightarrow F(U)\rightarrow 0.
\]

\subsection{Homotopy invariant presheaves}

\begin{definition}
A presheaf $F$ is \textbf{homotopy invariant} if for every $X$, the map $p^*:F(X)\rightarrow F(X\times \A^1)$ is an isomorphism.
\end{definition}

Note that this is equivalent to $p^*$ being surjective.  We can check that an equivalent condition is that for all $X$, we have 
\[
i_0^*=i_1^*:F(X\times \A^1)\rightarrow F(X).
\]
Furthermore, if $F$ is any presheaf, we have that $i_0^*, i_1^*:C_*F(X\times \A^1)\rightarrow C_*F(X)$ are chain homotopic.  From this we deduce that if $F$ is a presheaf, then the homology presheaves 
\[
H_nC_*F:X\mapsto H_nC_*F(X)
\]
are homotopy invariant for all $n$. 

\begin{definition}
Two finite correspondences from $X$ to $Y$ are \textbf{$\A^1$-homotopic} if they are the restrictions along $X\times 0$ and $X\times 1$ of an element of $\Cor(X\times \A^1, Y)$. 
\end{definition}

This is an equivalence relation on $\Cor(X, Y)$.  Note that it is not one if we just look at morphisms of schemes!  With this definition though, we define $f:X\rightarrow Y$ to be an $\A^1$-homotopy equivalence in the expected way.


\section{Motivic cohomology} 

\subsection{The motivic complex}

\begin{definition}
For $q\in \Z_{\ge 0}$, the \textbf{motivic complex} $\Z(q)$ is defined as the following complex of presheaves with transfers. 
\[
\Z(q)\coloneqq C_*\Z_{tr}(\G_m^{\wedge q})[-q].
\]
\end{definition}

We can change coefficients to $A\in \Ab$ by setting $A(q)=\Z(q)\otimes A$.  \\  

These are actually complexes of sheaves with respect to the Zariski topology.  In fact, they are also sheaves in the \'etale topology.  

For example when $q=0$, applying this to a scheme $Y$ we just get 
\[
\cdots\xrightarrow{0} \Z\xrightarrow{\id} \Z\xrightarrow{0}\Z\rightarrow 0 
\]
which is quasi-isomorphic to just $\Z$.  When $q=1$, the complex looks like 
\[
\cdots\xrightarrow{} \Cor(Y\times\Delta^2, \G_m) \xrightarrow{} \Cor(Y\times\Delta^1, \G_m)\xrightarrow{}\Cor(Y, \G_m)\rightarrow 0.
\]

\subsection{Motivic cohomology groups}

\includegraphics[scale=0.55]{28.png} 

\subsection{Weight 1} 
There is a quasi-isomorphism 
\[
\Z(1)\xrightarrow{\cong}\mc{O}^*[-1].
\]
Thus we have the following table.

\includegraphics[scale=0.6]{32.png} 
 

\section{Relation to other fields} 
\subsection{Algebraic K-theory}
Atiyah-Hirzebruch: 
\[E_2^{p, q}=H^p(X; K^q(*))\Rightarrow K^{p+q}(X).\]

In the algebraic setting, it is much more difficult.  Indeed, both algebraic K-theory and motivic cohomology are significantly harder to define than their topological counterparts.  In 2002, Suslin and Friedlander built upon previous work of Bloch and Lichtenbaum to show the following spectral sequence. 
\[
E_2^{p,q}=H^{p-q}(X, \Z(-q))=\CH^{-q}(X, -p-q)\Rightarrow K_{-p-q}(X).
\]
\subsection{Motives} 
\includegraphics[scale=0.22]{39.png} 

Grothendieck constructed Chow motives by replacing morphisms of schemes with correspondences (defined under rational equivalence, different from the correspondences discussed earlier), augmenting the category to look like an abelian category, and taking the opposite category.  By construction, this works in that cohomology theories factor through it.  However, to truly achieve what is desired from them, one must assume the standard conjectures on algebraic cycles (or some variants), which have been open for over 50 years! \\ 

Voevodsky used motivic cohomology to construct a triangulated category $DM(k; R)$, which for all intents and purposes acts as the derived category of the desired category of motives.  He studied mixed motives, which apply to all varieties (not just the smooth ones).  These can be thought of as extensions of pure motives, and motivic cohomology studies these $Ext$ groups.  

\subsection{Arithmetic geometry} 
There's the Bloch-Kato conjecture and the Bloch-Kato conjectures, which are different! \\ 

The Bloch-Kato conjecture is now a theorem: the norm residue isomorphism theorem, proven by Voevodsky.  Through proving it, Voevodsky developed motivic cohomology, motivic homotopy theory, motivic Steenrod algebra... \\

\includegraphics[scale=0.5]{37.png} 
(taken from wikipedia) \\ 

The Bloch-Kato conjectures are on special values of L-functions.  

\includegraphics[scale=0.7]{38.png} 
(taken from \cite{Haine})

Applied to elliptic curves, this implies (one of the two parts of) the Birch-Swinnerton Dyer conjecture!

\begin{thebibliography}{9}
\bibitem{Voevodsky} 
C. Mazza, V. Voevodsky, C. Weibel, \textit{Lecture notes on motivic cohomology}
\url{https://sites.math.rutgers.edu/~weibel/MVWnotes/prova-hyperlink.pdf}

\bibitem{Haine}
P. Haine, \textit{An overview of motivic cohomohlogy}, \url{https://math.mit.edu/~phaine/files/Motivic_Overview.pdf}
\end{thebibliography}

\end{document}



