\documentclass{amsart}
\usepackage{amsmath}
\usepackage{amssymb}
\usepackage{amsthm}
%\usepackage{MnSymbol}
\usepackage{bm}
\usepackage{accents}
\usepackage{mathtools}
\usepackage{tikz}
\usetikzlibrary{calc}
\usetikzlibrary{decorations.pathmorphing,shapes}
\usetikzlibrary{automata,positioning}
\usepackage{tikz-cd}
\usepackage{forest}
\usepackage{braket} 
\usepackage{listings}
\usepackage{mdframed}
\usepackage{verbatim}
\usepackage{physics}
\usepackage{stmaryrd}
\usepackage{mathrsfs} 
\usepackage{stackengine} 
%\usepackage{/home/patrickl/homework/macaulay2}

%font
\usepackage[osf]{mathpazo}
\usepackage{microtype}

%CS packages
\usepackage{algorithmicx}
\usepackage{algpseudocode}
\usepackage{algorithm}

% typeset and bib
\usepackage[english]{babel} 
\usepackage[utf8]{inputenc} 
%\usepackage[backend=biber, style=alphabetic]{biblatex}
\usepackage[bookmarks, colorlinks, breaklinks]{hyperref} 
\hypersetup{linkcolor=black,citecolor=black,filecolor=black,urlcolor=black}

% other formatting packages
\usepackage{float}
\usepackage{booktabs}
\usepackage[shortlabels]{enumitem}
\usepackage{csquotes}
%\usepackage{titlesec}
%\usepackage{titling}
%\usepackage{fancyhdr}
%\usepackage{lastpage}
\usepackage{parskip}

\usepackage{lipsum}

% delimiters
\DeclarePairedDelimiter{\gen}{\langle}{\rangle}
\DeclarePairedDelimiter{\floor}{\lfloor}{\rfloor}
\DeclarePairedDelimiter{\ceil}{\lceil}{\rceil}


\newtheorem{thm}{Theorem}[section]
\newtheorem{cor}[thm]{Corollary}
\newtheorem{prop}[thm]{Proposition}
\newtheorem{lem}[thm]{Lemma}
\newtheorem{conj}[thm]{Conjecture}
\newtheorem{quest}[thm]{Question}

\theoremstyle{definition}
\newtheorem{defn}[thm]{Definition}
\newtheorem{defns}[thm]{Definitions}
\newtheorem{con}[thm]{Construction}
\newtheorem{exm}[thm]{Example}
\newtheorem{exms}[thm]{Examples}
\newtheorem{notn}[thm]{Notation}
\newtheorem{notns}[thm]{Notations}
\newtheorem{addm}[thm]{Addendum}
\newtheorem{exer}[thm]{Exercise}

\theoremstyle{remark}
\newtheorem{rmk}[thm]{Remark}
\newtheorem{rmks}[thm]{Remarks}
\newtheorem{warn}[thm]{Warning}
\newtheorem{sch}[thm]{Scholium}


% unnumbered theorems
\theoremstyle{plain}
\newtheorem*{thm*}{Theorem}
\newtheorem*{prop*}{Proposition}
\newtheorem*{lem*}{Lemma}
\newtheorem*{cor*}{Corollary}
\newtheorem*{conj*}{Conjecture}

% unnumbered definitions
\theoremstyle{definition}
\newtheorem*{defn*}{Definition}
\newtheorem*{exer*}{Exercise}
\newtheorem*{defns*}{Definitions}
\newtheorem*{con*}{Construction}
\newtheorem*{exm*}{Example}
\newtheorem*{exms*}{Examples}
\newtheorem*{notn*}{Notation}
\newtheorem*{notns*}{Notations}
\newtheorem*{addm*}{Addendum}


\theoremstyle{remark}
\newtheorem*{rmk*}{Remark}

% shortcuts
\newcommand{\Ima}{\mathrm{Im}}
\newcommand{\A}{\mathbb{A}}
\newcommand{\G}{\mathbb{G}}
\newcommand{\N}{\mathbb{N}}
\newcommand{\R}{\mathbb{R}}
\newcommand{\C}{\mathbb{C}}
\newcommand{\Z}{\mathbb{Z}}
\newcommand{\Q}{\mathbb{Q}}
\renewcommand{\k}{\Bbbk}
\renewcommand{\P}{\mathbb{P}}
\newcommand{\M}{\overline{M}}
\newcommand{\g}{\mathfrak{g}}
\newcommand{\h}{\mathfrak{h}}
\newcommand{\n}{\mathfrak{n}}
\renewcommand{\b}{\mathfrak{b}}
\newcommand{\ep}{\varepsilon}
\newcommand*{\dt}[1]{%
   \accentset{\mbox{\Huge\bfseries .}}{#1}}
%\renewcommand{\abstractname}{Official Description}
\newcommand{\mc}[1]{\mathcal{#1}}
\newcommand{\msc}[1]{\mathscr{#1}}
\newcommand{\T}{\mathbb{T}}
\newcommand{\mf}[1]{\mathfrak{#1}}
\newcommand{\mr}[1]{\mathrm{#1}}
\newcommand{\ms}[1]{\mathsf{#1}}
\newcommand{\ol}[1]{\overline{#1}}
\newcommand{\ul}[1]{\underline{#1}}
\newcommand{\wt}[1]{\widetilde{#1}}
\newcommand{\wh}[1]{\widehat{#1}}
\renewcommand{\div}{\operatorname{div}}

\DeclareMathOperator{\Der}{Der}
\DeclareMathOperator{\Tor}{Tor}
\DeclareMathOperator{\Hom}{Hom}
\DeclareMathOperator{\End}{End}
\DeclareMathOperator{\ad}{ad}
\DeclareMathOperator{\Aut}{Aut}
\DeclareMathOperator{\Rad}{Rad}
\DeclareMathOperator{\Pic}{Pic}
\DeclareMathOperator{\supp}{supp}
\DeclareMathOperator{\Supp}{Supp}
\DeclareMathOperator{\depth}{depth}
\DeclareMathOperator{\sgn}{sgn}
\DeclareMathOperator{\spec}{Spec}
\DeclareMathOperator{\Spec}{Spec}
\DeclareMathOperator{\proj}{Proj}
\DeclareMathOperator{\Proj}{Proj}
\DeclareMathOperator{\ord}{ord}
\DeclareMathOperator{\Div}{Div}
\DeclareMathOperator{\Bl}{Bl}

\title{Do we even need derived categories?}
\author{Patrick Lei}
\date{April 9, 2021}

\begin{document}
    
\maketitle

\begin{abstract}
    We will discuss the moduli space of stable curves of genus $0$ with $n$ marked points and its intersection theory, following~\cite{keel}. We will give a nice presentation of its Chow ring in terms of boundary divisors.
\end{abstract}

\section{Serre formula}%
\label{sec:serre_formula}

Recall the following from my March 19 lecture (this is Proposition 7.2.9 in my notes):
\begin{prop*}
    Let $X$ be a smooth variety, $V, W \subset X$ be closed subschemes that intersect properly, and $Z$ be an irreducible component of $V \cap W$. Then
    \[ V \cdot W = \sum_Z i(Z, V \cdot W; X) \cdot [Z], \]
    where $1 \leq i(Z, V \cdot W; X) \leq \ell(\msc{O}_{V \cap W, Z})$ is the intersection number and $i(Z, V \cdot W; X) = \ell(\msc{O}_{V \cap W, Z})$ if and only if the local ring is Cohen-Macaulay.
\end{prop*}

However, most rings are \textbf{not} Cohen-Macaulay, so we would like a formula to compute the intersection multiplicities in all cases. Serre gives a formula in terms of higher Tor functors (because $\msc{O}_{V \cap W, Z} = \msc{O}_{V, Z} \otimes \msc{O}_{W, Z}$). Before we state the formula, first we will state some results about the higher Tor functors. 

First, if $X$ is a locally ringed space and $\msc{F}, \msc{G}$ are modules on $X$, then 
\[ {\Tor_i^{\msc{O}_X}(\msc{F}, \msc{G})}_x = \Tor_i^{\msc{O}_{X,x}}(\msc{F}_x, \msc{G}_x). \]
This follows from the construction of the derived tensor product $\otimes^L$ in Stacks, which exposits derived categories much better than I ever could.

\begin{lem}
    Let $X$ be a locally Noetherian scheme. If $\msc{F}, \msc{G}$ are coherent, so is $\Tor_i^{\msc{O}_X}(\msc{F}, \msc{G})$. Also, if $L, K \in D^-_{\mr{coh}}(\msc{O}_X)$ (this means bounded above complexes of quasicoherent sheaves with coherent homology), then so is $L \otimes^L K$.
\end{lem}

Proof of this fact is pure homological algebra, and again can be found in the Stacks project.

\begin{lem}
    Let $X$ be a smooth variety and $\msc{F}, \msc{G}$ be coherent sheaves. Then $\Tor_i^{\msc{O}_X}(\msc{F}, \msc{G})$ is supported on $\Supp \msc{F} \cap \Supp \msc{G}$, and is nonzero only when $0 \leq i \leq \dim X$.
\end{lem}

\begin{proof}
    The support condition is clear by looking at the stalks, so we need to consider when the stalks are nonzero. Here, we note that because $X$ is smooth, the local rings $\msc{O}_{X,x}$ are regular local rings. By a result of Serre (Theorem 4.4.1 in my commutative algebra notes), $\msc{O}_{X,x}$ has finite global dimension $\dim \msc{O}_{X,x} \leq \dim X$. Here, the global dimension and the dimension are the same by Theorem 4.3.12 of my commutative algebra notes.\footnote{Originally there was an argument that the global dimension of a Noetherian local ring is the projective dimension of the residue field, which is Theorem 4.3.10 of the commutative algebra notes, and then by the Auslander-Buchsbaum formula this is the same as the depth, and finally regular implies Cohen-Macaulay, so depth equals dimension.}
\end{proof}

Now we can compute the intersection multiplicities as (Stacks gives this as the definition of intersection multiplicity)
\[ i(Z, V \cdot W; X) = \sum {(-1)}^i \ell( \Tor_i^{\msc{O}_{X,Z}} (\msc{O}_{V,Z}, \msc{O}_{W,Z}) ). \]
This formula is due to Serre, and Stacks writes the total intersection as
\[ W \cdot V = \sum {(-1)}^i [\Tor_i^{\msc{O}_X}(\msc{O}_V, \msc{O}_W)]. \]
\begin{rmk}
    Stacks writes the intersection multiplicity as $e(X, V \cdot W, Z)$. I am using the notation in Fulton's book.
\end{rmk}

\begin{lem}
    Assume that $\ell(\msc{O}_{V \cap W, Z}) = 1$. Then $i(Z, V \cdot W; X) = 1$ and $V, W$ are smooth in a general point of $Z$.
\end{lem}

\begin{proof}
    Write $A = \msc{O}_{X,Z}$. Then $\dim A = \dim X - \dim Z$. Let $I, J$ be the ideals of $V, W$. By Proposition 7.2.9 of the notes,\footnote{This may be cheating, and a self-contained argument is given in Stacks} $I + J = \mf{m}$. Thus there exists $f_1, \ldots, f_r \in I , g_1, \ldots, f_s \in J$ Forming a basis for $\mf{m}/\mf{m}^2$. But this is a regular sequence and a system of parameters, so $A/ (f_1, \ldots, f_r)$ is a regular local ring of dimension $\dim X - \dim V$, so $I = (f_1, \ldots, f_r)$. Similarly, $J = (g_1, \ldots, g_s)$. Now by Corollary 4.4.3 of commutative algebra, the Koszul complex $K(f_1, \ldots, f_r, A)$ resolves $A/I$, so we obtain
    \begin{align*}
        \Tor_i^A(A/I, A/J) &= H_i (K(f_1, \ldots, f_r, A) \otimes A/J) \\
                           &= H_i (K(f_1, \ldots, f_r, A/J)).
    \end{align*}
    By Theorem 4.4.2 from commutative algebra, we only have $H_0 = k$.
\end{proof}

\begin{exm}
    Suppose $V, W \subset X$ are closed subvarieties, $\dim X = 4$, $\wh{\msc{O}}_{X,p} = \C[[x,y,z,w]]$ and $V = (xz,xw,yz,yw), W = (x-z,y-w)$. Then 
    \[ \ell(\C[[x,y,z,w]]/(xz,xw,yz,yw,x-z,x-w)) = 3,\] 
    but the intersection multiplicity is $2$ because $V$ is locally a union $(x=y=0) \cup (z=w=0)$.
\end{exm}

\section{Some algebra}%
\label{sec:some_algebra}

Let $(A, \mf{m}, k)$ be a Noetherian local ring. If $M$ is a module and $I$ is an ideal of definition, recall the Hilbert-Samuel polynomial $\varphi_{I,M}(n) = \ell(I^n M / I^{n+1} M)$. Similarly recall the function 
\[ \chi_{I,M}(n) = \ell(M/I^{n+1} M) = \sum_{i=0}^n \varphi_{I,M}(i). \]
Recall that $d(M) \coloneqq \deg \chi$ is independent of $I$ and equals the dimension of the support of $M$ (from the proof of Theorem 3.2.9 in my commutative algebta notes). Now write $\chi_{I,M}(n) = e_I(M, d) \frac{n^d}{d!} + O(n^{d-1})$.

\begin{defn}
    For $d = d(M)$ we write $e_I(M, d)$ as above, and for $d > d(M)$, we set $e_I(M,d) = 0$.
\end{defn}

\begin{lem}
    For all $I, M$, we have
    \[ e_I(M, d) = \sum_{\dim A/\mf{p} = d} \ell_{A_{\mf{p}}} (M_{\mf{p}}) e_{I}(A/\mf{p}, d). \]
\end{lem}

\begin{lem}
    Let $P$ be a polynomial of degree $r$ with leading coefficient $a$. Then
    \[ r! a = \sum_{i=0}^r {(-1)}^i \binom{r}{i} P(t-i) \]
    for any $t$.
\end{lem}

\begin{proof}
    Write $\Delta$ for the operator taking a polynomial $P$ to $P(t) - P(t-1)$. Then
    \begin{align*}
        \Delta^{r+1}(P) &= \sum_{i=0}^r {(-1)}^i \binom{r}{i} \Delta(P)(t-i) \\
                        &= \sum_{i=0}^r {(-1)}^i \binom{r}{i} (P(t-i) - P(t-i-1)).
    \end{align*}
    The desired claim follows from Pascal's identity.
\end{proof}

\begin{thm}
    Let $A$ be a Noetherian local ring and $I = (f_1, \ldots, f_r)$ be an ideal of definition. Then
    \[ e_I(M, r) = \sum {(-1)}^i \ell(H_i(K(f_1, \ldots, f_r) \otimes M)). \]
\end{thm}

There is a very long proof of this statement in Stacks using spectral sequences.

\section{Computing intersection multiplicities without derived categories}%
\label{sec:computing_intersection_multiplicities_without_derived_categories}

We give some cases where intersection multiplicities can be computed without using derived categories.

\begin{lem}
    Suppose $\msc{O}_{V,Z}$ and $\msc{O}_{W,Z}$ are Cohen-Macaulay. Then $i(Z, V \cdot W; X) = \ell(\msc{O}_{V \cap W, Z})$.
\end{lem}

\begin{proof}
    Write $A = \msc{O}_{X,Z}, B = \msc{O}_{V,Z}, C = \msc{O}_{W,Z}$. Then by Auslander-Buchsbaum (exercise 4d of the final CA homework), we have a resolution $F_{\bullet} \to B$ of length $\depth A - \depth B = \dim A - \dim B = \dim C$. Then $F_{\bullet} \otimes C$ represents $B \otimes^L C$ and is supported in $\qty{ \mf{m}_A }$, so by Lemma 10.108.2 in Stacks, it has nonzero cohomology only in degree $0$.
\end{proof}

\begin{lem}
    Let $A$ be a Noetherian local ring and $I = (f_1, \ldots, f_r)$ is generated by a regular sequence. If $M$ is a finite $A$-module with $\dim \Supp M/IM = 0$, then
    \[ e_I(M,r) = \sum {(-1)}^i \ell(\Tor_i^A(A/I, M)). \]
\end{lem}

In what follows, we will assume $V$ is cut out in $\msc{O}_{X,Z}$ by a regular sequence $(f_1, \ldots, f_c)$.
\begin{lem}
    In this case, we have $i(Z, V \cdot W; X) = c{!}$. This is the leading coefficient of the ``Hilbert polynomial'' $n \mapsto \ell(\msc{O}_{W,Z} / {(f_1, \ldots, f_c)}^t)$.
\end{lem}

\begin{proof}
    By the previous lemma, $e(Z, V \cdot W; X) = e_{(f_1, \ldots, f_c)}(\msc{O}_{W,Z}(c))$. Now we need to show that $\dim \msc{O}_{W,Z} = c$. But now if $\dim V = r, \dim W = s, \dim X = n$, $\dim Z = r+s-n$, so $k(Z)$ has transcendence degree $r+s-n$. Because $f_1, \ldots, f_c$ is a regular sequence, $r+c=n$, so $\dim \msc{O}_{W,Z} = s-(r+s-n) = s-(n-c+s-n)=c$.
\end{proof}

\begin{lem}
    Assume $c = 1$ (for example, $V$ is an effeective Cartier divisor). Then $i(Z, V \cdot W;X) = \ell(\msc{O}_{W,Z} / (f_1))$.
\end{lem}

\begin{proof}
    Note that $\msc{O}_{W,Z}$ is a Noetherian local domain of dimension $1$. Then it is clear that $\ell(\msc{O}_{W,Z}/(f_1^t)) = t \ell(\msc{O}_{W,Z} / (f_1))$ for all $t \geq 1$.
\end{proof}

\begin{lem}
    Asssume $\msc{O}_{W,Z}$ is Cohen-Macaulay. Then 
    \[ i(Z, V \cdot W; X) = \ell(\msc{O}_{W,Z} / (f_1, \ldots, f_c)). \]
\end{lem}

\begin{proof}
    Because $f_1, \ldots, f_c$ is a regular sequence, it is also quasi-regular by Proposition 3.5.6 of my commutative algebra notes. Then 
    \[ \ell(\msc{O}_{W,Z} / {(f_1, \ldots, f_c)}^t) = \binom{c+t}{c} \ell(\msc{O}_{W,Z} / (f_1, \ldots, f_c)). \]
    Now take the leading coefficient.
\end{proof}



\end{document}
