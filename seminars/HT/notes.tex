\documentclass[leqno, openany]{memoir}
\setulmarginsandblock{3.5cm}{3.5cm}{*}
\setlrmarginsandblock{3cm}{3.5cm}{*}
\checkandfixthelayout

\usepackage{amsmath}
\usepackage{amssymb}
\usepackage{amsthm}
%\usepackage{MnSymbol}
\usepackage{bm}
\usepackage{accents}
\usepackage{mathtools}
\usepackage{tikz}
\usetikzlibrary{calc}
\usetikzlibrary{automata,positioning}
\usepackage{tikz-cd}
\usepackage{forest}
\usepackage{braket} 
\usepackage{listings}
\usepackage{mdframed}
\usepackage{verbatim}
\usepackage{physics}
\usepackage{stmaryrd}
\usepackage{mathrsfs} 
\usepackage{ulem} 
%\usepackage{/home/patrickl/homework/macaulay2}

%font
\usepackage[sc]{mathpazo}
\usepackage{eulervm}
\usepackage[scaled=0.86]{berasans}
\usepackage{inconsolata}
\usepackage{microtype}

%CS packages
\usepackage{algorithmicx}
\usepackage{algpseudocode}
\usepackage{algorithm}

% typeset and bib
\usepackage[english]{babel} 
\usepackage[utf8]{inputenc} 
\usepackage[T1]{fontenc}
\usepackage[backend=biber, style=alphabetic]{biblatex}
\usepackage[bookmarks, colorlinks, breaklinks]{hyperref} 
\hypersetup{linkcolor=black,citecolor=black,filecolor=black,urlcolor=blue}

% other formatting packages
\usepackage{float}
\usepackage{booktabs}
\usepackage[shortlabels]{enumitem}
\usepackage{csquotes}
\usepackage{titlesec}
\usepackage{titling}
% \usepackage{fancyhdr}
% \usepackage{lastpage}
\usepackage{parskip}

\usepackage{lipsum}

% delimiters
\DeclarePairedDelimiter{\gen}{\langle}{\rangle}
\DeclarePairedDelimiter{\floor}{\lfloor}{\rfloor}
\DeclarePairedDelimiter{\ceil}{\lceil}{\rceil}


\newtheorem{thm}{Theorem}[section]
\newtheorem{cor}[thm]{Corollary}
\newtheorem{prop}[thm]{Proposition}
\newtheorem{lem}[thm]{Lemma}
\newtheorem{conj}[thm]{Conjecture}
\newtheorem{quest}[thm]{Question}

\theoremstyle{definition}
\newtheorem{defn}[thm]{Definition}
\newtheorem{defns}[thm]{Definitions}
\newtheorem{con}[thm]{Construction}
\newtheorem{exm}[thm]{Example}
\newtheorem{exms}[thm]{Examples}
\newtheorem{notn}[thm]{Notation}
\newtheorem{notns}[thm]{Notations}
\newtheorem{addm}[thm]{Addendum}
\newtheorem{exer}[thm]{Exercise}

\theoremstyle{remark}
\newtheorem{rmk}[thm]{Remark}
\newtheorem{rmks}[thm]{Remarks}
\newtheorem{warn}[thm]{Warning}
\newtheorem{sch}[thm]{Scholium}


% unnumbered theorems
\theoremstyle{plain}
\newtheorem*{thm*}{Theorem}
\newtheorem*{prop*}{Proposition}
\newtheorem*{lem*}{Lemma}
\newtheorem*{cor*}{Corollary}
\newtheorem*{conj*}{Conjecture}

% unnumbered definitions
\theoremstyle{definition}
\newtheorem*{defn*}{Definition}
\newtheorem*{exer*}{Exercise}
\newtheorem*{defns*}{Definitions}
\newtheorem*{con*}{Construction}
\newtheorem*{exm*}{Example}
\newtheorem*{exms*}{Examples}
\newtheorem*{notn*}{Notation}
\newtheorem*{notns*}{Notations}
\newtheorem*{addm*}{Addendum}


\theoremstyle{remark}
\newtheorem*{rmk*}{Remark}

% shortcuts
\newcommand{\Ima}{\mathrm{Im}}
\newcommand{\A}{\mathbb{A}}
\newcommand{\F}{\mathbb{F}}
\newcommand{\G}{\mathbb{G}}
\newcommand{\N}{\mathbb{N}}
\newcommand{\R}{\mathbb{R}}
\newcommand{\bH}{\mathbb{bH}}
\newcommand{\C}{\mathbb{C}}
\newcommand{\Z}{\mathbb{Z}}
\newcommand{\Q}{\mathbb{Q}}
\renewcommand{\k}{\Bbbk}
\renewcommand{\P}{\mathbb{P}}
\newcommand{\M}{\overline{M}}
\newcommand{\g}{\mathfrak{g}}
\newcommand{\h}{\mathfrak{h}}
\newcommand{\n}{\mathfrak{n}}
\renewcommand{\b}{\mathfrak{b}}
\newcommand{\ep}{\varepsilon}
\newcommand*{\dt}[1]{%
   \accentset{\mbox{\Huge\bfseries .}}{#1}}
\renewcommand{\abstractname}{Official Description}
\newcommand{\mc}[1]{\mathcal{#1}}
\newcommand{\T}{\mathbb{T}}
\newcommand{\mf}[1]{\mathfrak{#1}}
\newcommand{\mr}[1]{\mathrm{#1}}
\newcommand{\ms}[1]{\mathsf{#1}}
\newcommand{\on}[1]{\operatorname{#1}}
\newcommand{\ol}[1]{\overline{#1}}
\newcommand{\ul}[1]{\underline{#1}}
\newcommand{\wt}[1]{\widetilde{#1}}
\newcommand{\wh}[1]{\widehat{#1}}
\renewcommand{\div}{\operatorname{div}}
\newcommand{\bir}{\sim_{\mr{bir}}}

\DeclareMathOperator{\Der}{Der}
\DeclareMathOperator{\Def}{Def}
\DeclareMathOperator{\Bl}{Bl}
\DeclareMathOperator{\NE}{NE}
\DeclareMathOperator{\Tor}{Tor}
\DeclareMathOperator{\Hom}{Hom}
\DeclareMathOperator{\Ext}{Ext}
\DeclareMathOperator{\End}{End}
\DeclareMathOperator{\ad}{ad}
\DeclareMathOperator{\Aut}{Aut}
\DeclareMathOperator{\Rad}{Rad}
\DeclareMathOperator{\Pic}{Pic}
\DeclareMathOperator{\supp}{supp}
\DeclareMathOperator{\Supp}{Supp}
\DeclareMathOperator{\sgn}{sgn}
\DeclareMathOperator{\spec}{Spec}
\DeclareMathOperator{\Spec}{Spec}
\DeclareMathOperator{\proj}{Proj}
\DeclareMathOperator{\Proj}{Proj}
\DeclareMathOperator{\ord}{ord}
\DeclareMathOperator{\Div}{Div}
\DeclareMathOperator{\depth}{depth}
\DeclareMathOperator{\coker}{coker}
\DeclareMathOperator{\res}{res}

% Section formatting
\titleformat{\section}
    {\Large\sffamily\scshape\bfseries}{\thesection}{1em}{}
\titleformat{\subsection}[runin]
    {\large\sffamily\bfseries}{\thesubsection}{1em}{}
\titleformat{\subsubsection}[runin]{\normalfont\itshape}{\thesubsubsection}{1em}{}

\title{COURSE TITLE}
\author{Lectures by INSTRUCTOR, Notes by NOTETAKER}
\date{SEMESTER}

\newcommand*{\titleSW}
    {\begingroup% Story of Writing
    \raggedleft
    \vspace*{\baselineskip}
    {\Huge\itshape Seminar on Mixed Hodge Structures \\ Fall 2021}\\[\baselineskip]
    {\large\itshape Notes by Patrick Lei}\\[0.2\textheight]
    {\Large Lectures by Various}\par
    \vfill
    {\Large \sffamily Columbia University}
    \vspace*{\baselineskip}
\endgroup}
\pagestyle{simple}

\chapterstyle{ell}


%\renewcommand{\cftchapterpagefont}{}
\renewcommand\cftchapterfont{\sffamily}
\renewcommand\cftsectionfont{\scshape}
\renewcommand*{\cftchapterleader}{}
\renewcommand*{\cftsectionleader}{}
\renewcommand*{\cftsubsectionleader}{}
\renewcommand*{\cftchapterformatpnum}[1]{~\textbullet~#1}
\renewcommand*{\cftsectionformatpnum}[1]{~\textbullet~#1}
\renewcommand*{\cftsubsectionformatpnum}[1]{~\textbullet~#1}
\renewcommand{\cftchapterafterpnum}{\cftparfillskip}
\renewcommand{\cftsectionafterpnum}{\cftparfillskip}
\renewcommand{\cftsubsectionafterpnum}{\cftparfillskip}
\setrmarg{3.55em plus 1fil}
\setsecnumdepth{subsection}
\maxsecnumdepth{subsection}
\settocdepth{subsection}

\begin{document}
    
\begin{titlingpage}
\titleSW
\end{titlingpage}

\thispagestyle{empty}
\section*{Disclaimer}%
\label{sec:disclaimer}

These notes were taken during the seminar using the \texttt{vimtex} package of the editor \texttt{neovim}. 
Any errors are mine and not the speakers'. 
In addition, my notes are picture-free (but will include commutative diagrams) and are a mix of my mathematical style and that of the lecturers.
If you find any errors, please contact me at \texttt{plei@math.columbia.edu}.

\vspace*{1cm}

\noindent\textbf{Seminar Website:}  \url{https://www.math.columbia.edu/~kyc2130/f21_mhs.html}
\newpage

\tableofcontents

\chapter{Kevin (Oct 28): Intro to mixed Hodge structures; Hodge theory for smooth varieties}%
\label{cha:kevin_oct_28_intro_to_mixed_hodge_structures_hodge_theory_for_smooth_varieties}

Throughout the seminar, we work over $\C$ like reasonable people.

\section{Review of Hodge theory}%
\label{sec:review_of_hodge_theory}

Let $X$ be a smooth projective variety. Then we have the Hodge decomposition
\[ H^n(X, \C) = \bigoplus_{p+q=n} H^{p,q}(X). \]
Here, $H^{p,q}$ contains forms of type $(p,q)$. Deligne constructs a \textit{mixed Hodge structure} on smooth varieties in \textit{Th\'eorie de Hodge II} and on all varieties in \textit{Th\'eorie de Hodge III}.

\begin{defn}
    Let $A$ be one of $\Z, \Q, \R$. A \textit{Hodge structure} over $A$ can be defined in several ways:
    \begin{enumerate}
        \item A Hodge structure of weight $n$ is a finitely-generated $A$-module $H$ with a decomposition
            \[ H_{\C} = H \otimes \C = \bigoplus_{p+q=n} H^{p,q} \]
            satisfying $H^{p,q} = \ol{H}^{q,p}$.
        \item A Hodge structure of weight $n$ is a finitely-generated $A$-module $H$ with a decreasing filtration $F*{\bullet} H_{\C}$ that is \textit{$n$-opposite} in the sense that if $p+q = n+1$, then $F^p \cap \ol{F}^q = 0$ and $F^p \oplus \ol{F}^q = H_{\C}$.
        \item A Hodge structure of weight $n$ is a finitely-generated $A$-module $H$ with an action of the Deligne torus $\mathbb{S} \coloneqq \operatorname{Res}_{\R}^{\C} \C^{\times}$ on $H_{\R}$ of weight $n$.
    \end{enumerate}
\end{defn}

\begin{rmk}
    In our definition, $p,q,n$ can all be negative.
\end{rmk}

To see that these definitions are equivalent, from a decomposition $\bigoplus_{p+q=n} H^{p,q}$ we take the filtration
\[ F^p H = \bigoplus_{i \geq p} H^{i,n-i}. \]
In the other direction, for a filtration $F$, we take
\[ H^{p,q} = F^p \cap \ol{F}^q. \]
From now on, when we say Hodge structure we will take $A = \Z$.

\begin{defn}
    A \textit{morphism of Hodge structures} of weight $n$ $(H, F) \to (H', F')$ is a morphism of modules that respects the filtration after tensoring with $\C$.
\end{defn}

Tensor products of Hodge structures are Hodge structures after adding the weights, and the dual of a Hodge structure is still a Hodge structure after taking the negative of the weight.

\begin{exms}\leavevmode
    \begin{enumerate}
        \item Let $X$ be a smooth projective variety. Then by the Hodge theorem, $H^n(X, \Z)$ is a Hodge structure of weight $n$.
        \item Let $X$ be a smooth proper variety. Then $H^n(X, \Z)$ is a Hodge structure of weight $n$ by a result of Deligne.
        \item (Tate-Hodge structure) We will define a Hodge structure $\Z(1)$ of weight $-2$ where the underlying $\Z$-module is $2 \pi i \Z$ and $\Z(1) \otimes_{\Z} \C = H^{-1,-1}$. We will denote by $\Z(n)$ the $n$-th tensor power of $\Z(1)$. We may twist Hodge structure by $\Z(n)$ to change the weight.

            If $Z \hookrightarrow X$ is a subvariety of a smooth projective variety of codimension $r$, there are Gysin maps
            \[ i_! \colon H^{n-2r}(Z, \Z) \to H^n(X, \Z). \]
            This is not a map of Hodge structures by weight reasons, but after twisting by $\Z(-r)$ we obtain a morphism
            \[ i_! \colon H^{n-2r}(Z, \Z)(-r) \to H^n(X, \Z) \]
            that is a morphism of Hodge structures.

            Now set $U \coloneqq X \setminus Z$. Then we have a long exact seqeunce
            \[ \cdots \to H^{n-2r}(Z)(-r) \to H^n(X) \to H^n(U) \to H^{n-2r+1}(Z)(-r) \to H^{n+1}(X) \to H^{n+1}(U) \to \cdots. \]

            We will introduce a new increasing filtration $W$ on $H^n(U)$ where
            \[ W_{n-1}(H^n(U)) = 0 \qquad W_n(H^n(U)) = \Im(H^n(X) \to H^n(U)) \qquad W_{n+1}(H^n(U)) = H^n(U). \]
            Taking the associated graded module 
            \[ \mr{Gr}_i^W H^n(U) = \begin{cases}
                \Im(H^n(X) \to H^n(U)) & i = n \\
                \Im(H^n(U) \to H^{n-2r+1}(2)(-r)) & i=n+1 \\
                0 & \text{otherwise}.
            \end{cases} \]
            Now $W$ has allowed us to isolate the weight $n$ part and the weight $n+1$ part, and now we want to consider mixed Hodge structures.
    \end{enumerate}
\end{exms}

\section{Mixed Hodge structures}%
\label{sec:mixed_hodge_structures}

\begin{defn}
    A \textit{mixed Hodge structure} is a finitely-generated $\Z$-module $H$ together with two filtrations:
    \begin{itemize}
        \item The \textit{weight filtraation} $W$, which is an increasing filtration on $H_{\Q}$;
        \item The \textit{Hodge filtration} $F$, which is a decreasing filtration on $H_{\C}$.
    \end{itemize}
    These are required to be compatible in the sense that for all $n$, the associated graded piece
    \[ (\on{Gr}_n^W H_{\Q}, F) \]
    is a Hodge structure over $\Q$.
\end{defn}

\begin{exm}
    Let $H$ be an ordinary Hodge strucure of weight $n$. This is a mixed Hodge structure, where we take $W_{n-1} H_{\Q} = 0$ and $W_n H_{\Q} = H_{\Q}$.
\end{exm}

\begin{defn}
    A \textit{morphism of mixed Hodge structures} $f \colon (H,W,F) \to (H',W',F')$ is a morphism of $\Z$-modules $H \to H'$ that is compatible with both filtrations.
\end{defn}

\begin{thm}
    Any morphism of mixed Hodge structures is automatically \textit{strictly compatible with both filtrations}, which means that if $h' \in \Im(f) \cap W_n'(H')$, there exists $h \in W_n(H)$ such that $f(h) = h'$. The same result is true for $F, F'$.
\end{thm}

\begin{rmk}
    The category of mixed Hodge structures is an abelian category.
\end{rmk}

\begin{thm}[Main theorem of Hodge II]\label{thm:mhs}
    Let $U$ be a smooth variety. Then $H^n(U, \Z)$ is functorially a mixed Hodge structure.
\end{thm}

\section{Hypercohomology spectral sequences}%
\label{sec:hypercohomology_spectral_sequences}

Before we discuss the proof of the main theorem, we will discuss this important tool. Let $X$ be a smooth projective variety. Then recall that 
\[ H^n(X, \C) = \bigoplus_{p+q=n} H^{q}(X,\Omega^p_X). \]
This implies that the Hodge-de Rham spectral sequence degenerates. Recall\footnote{Or look up on } that the Hodge-de Rham spectral sequence is given by
\[ E_1^{p,q} = H^q(\Omega_X^p) \implies H^{p+q}(X, \C). \]
This is a hypercohomoology spectral sequence.

\begin{defn}
    Let $(K, F)$ be a filtered complex of abelian sheaves with $F$ decreasing and finite on each component. Then the \textit{hypercohomology spectral sequence} for $(K, F)$ is given by
    \[ E_1^{p,q} = \bH^{p+q}(X, \on{Gr}_F^p K) \implies \bH^{p+q}(X, K). \]
\end{defn}

In our case, the Hodge-de Rham spectral sequence is the hypercohomology spectral sequence for $(\Omega_X^{\bullet}, F)$, where $F$ is the filtration b\^ete and is defined by
\[ F^p(K)^n = \begin{cases}
    K^n & n \geq p \\
    0 & n < p.
\end{cases}
\]
This gives us $E_1^{p,q} = \bH^{p+q}(X, \Omega_X^p [-p])$. Using the holomorphic Poincar\'e lemma, the map $C \to \Omega_X^{\bullet}$ is a quasi-isomorphism, and this gives us the Hodge-de Rham spectral sequence.

We will now say some things about filtered complexes.
\begin{enumerate}
    \item If $(K, F) \to (K', F')$ is a morphism of filtered complexes, there is an induced map of hypercohomology spectral sequences.
    \item If $(K, F) \to (K', F')$ is a \textit{filtered quasi-isomorphism} (which means it induces quasi-isomorphisms on all associated graded complexes), then the map of hypercohomology spectral sequences is an isomorphism.
    \item $F$ induces a filtration on $H^k(X, K)$ given by
        \[ F^p H^*(X, K) = \Im(\bH^*(X, F^p(K)) \to \bH^*(X, K)). \]
    \item Every complex $K$ carries a \textit{canonical filtration} $\tau$, which is an increasing filtration and is given by
        \[ \tau_p(K)^n = \begin{cases}
            0 & n > p \\
            \ker \dd^p & n = p \\
            K^n & n < p.
        \end{cases}
        \]
        In this case, the associated graded pieces are given by $\on{Gr}_p^{\tau} K = H^p(K)[-p]$. If $K \to K'$ is a quasi-isomorphism, then $(K, \tau_K) \to (K', \tau_{K'})$ is a filtered quasi-isomorphism. Thus we can discuss canonical filtrations of items in the derived category.
\end{enumerate}

\section{Hodge theory for smooth varieties}%
\label{sec:hodge_theory_for_smooth_varieties}

Let $U$ be a smooth variety. We will find a good compactification of $U$:
\begin{enumerate}
    \item First we consider the Nagata compactification $X'$.
    \item Then resolve the Nagata compactification using resolution of singularities, and now we have a good compactification $X$.
\end{enumerate}

\begin{defn}
    A compactification $X$ of $U$ is \textit{good} if $D = X \setminus U$ is a simple normal crossings divisor. This means that $D$ locally analytically looks like a union of hyperplanes.
\end{defn}

Now that we have a good compactification, we know $D$ is a union of smooth irreducible hypersurfaces 
\[ D = \bigcup_{i=1}^S D_i. \]
For all $I \subset [s]$, write $D_I = \bigcap_{i \in I} D_i$. If $I$ is empty, then $D_{\emptyset} = X$. Now write $D^{(i)} \coloneqq \bigsqcup_{\abs{I} = i} D_i$. Also denote the inclusion $j \colon U \hookrightarrow X$.

We want to study $H^*(U, \C_U) = \bH^*(U, \Omega^{\bullet}_U) = \bH^*(X, R_{j*} \Omega^{\bullet}_U)$. But now the higher pushforwards vanish, and so we obtain $\bH^*(X, j_* \Omega_U^{\bullet})$. Our strategy will be to study the \textit{logarithmic de Rham complex}
\[ \Omega^{\bullet}_X(\log D) \subset j_* \Omega^{\bullet}_U, \]
which are defined as follows:
\begin{itemize}
    \item Logarithmic $1$-forms are locally spanned by $\frac{\dd{z_i}}{z_i}$ for $1 \leq i \leq t$ and $\dd{z_i}$ for $t+1 \leq i \leq n$. Here, we choose coordinates such that $D = \qty{z_1 \cdots z_t = 0}$.
    \item Logarithmic $k$-forms are given by $\Omega^k_X(\log D) = \bigwedge^k \Omega^1_X(\log D)$.
\end{itemize}
In fact, the inclusion $\Omega_X^{\bullet}(\log D) \hookrightarrow j_* \Omega_U^{\bullet}$ is a quasi-isomorphism, so we can instead study the hypercohomology $\bH(X, \Omega_X^{\bullet}(\log D))$.

\chapter{Kevin (Nov 04): Hodge theory for smooth varieties II}%
\label{cha:kevin_nov_04_hodge_theory_for_smooth_varieties_ii}

Recall that last time, we considered a smooth compactification $X \supseteq U$ of a smooth variety, and the boundary is a simple normal crossings divisor. We have this chain of quasi-isomorphisms
\[ R j_* \C_U \to R j_* \Omega_U^{\bullet} \simeq j_* \Omega_U^{\bullet} \gets \Omega_X^{\bullet}(\log D). \]
Therefore it suffices to study $\bH^*(X, \Omega_X^{\bullet}(\log D))$. We are still atempting to prove Theorem~\ref{thm:mhs}.

\section{Setting everything up}%
\label{sec:setting_everything_up}

\begin{defn}
    The \textit{weight filtration} $W$ on $\Omega_X^{\bullet}(\log D)$ is defined by
    \[ W_m(\Omega_X^{\bullet}(\log D)) = \ev{ \alpha \wedge \dv{z_{i_1}} \wedge \cdots \wedge \dv{z_{i_m}}}, \]
    where $\alpha$ is a holomorphic form on $X$ and $z_1, \ldots$ are the local equations for $D$.
\end{defn}

Recall the definitions of $D_I, D^{(i)}$ from last time. We will denote the natural maps $a_I \colon D_I \to X$ and $a_i \colon D^{(i)} \to X$.

\begin{defn}
    Let $I \subset [s]$ with $\abs{I} = m$. The \textit{residue map} $\on{res}_I \colon W_m \Omega_X^{\bullet}(\log D) \to a_{I*} \Omega_{D_I}^{\bullet}[-m]$ is defined by
    \[ \alpha \wedge \dv{z_1} \cdots \wedge \dv{z_m} + \alpha' \mapsto \alpha |_{D_I}. \]
    Here, $\alpha$ is holomorphic, $D_I = \qty{z_1 \cdots z_m = 0}$, and $\alpha'$ has fewer poles.
\end{defn}

The important fact is that $\res_I$ vanishes on $W_{m-1}$ and therefore descends to a map
\[ \res_I \colon \on{Gr}_m^W \Omega_X^{\bullet}(\log D) \to a_{I*} \Omega^{\bullet}_{D_I}[-m]. \]

\begin{lem}
    Define
    \[ \res_m = \bigoplus_{\abs{I} = m} \res_m \colon \on{Gr}_m^W \Omega_X^{\bullet}(\log D) \to a_m^* \Omega_{D^{(m)}}^{\bullet} [-m]. \]
    This map is an isomorphism.
\end{lem}

\begin{cor}\leavevmode
    \begin{enumerate}
        \item We have the identity
            \[ \mc{H}^i(\on{Gr}_m^W \Omega_X^{\bullet}(\log D)) \cong \begin{cases}
                a_{m*} \C_{D^{(m)}} & i = m \\
                0 & \text{otherwise}.
            \end{cases}
            \]
        \item The map $(\Omega_X^{\bullet}(\log D), W) \xrightarrow{\mr{id}} (\Omega_X^{\bullet}(\log D), \tau)$ is a filtered quasi-isomorphism.
    \end{enumerate}
\end{cor}

Now we want to upgrade our known quasi-isomorphisms to filtered quasi-isomorphisms. By the previous discussion, we now have filtered quasi-isomorphisms
\[ (R j_* \Z_U, \tau) \otimes_{\Z} \C = (R j_* \C_U, \tau) \to (j_* \Omega_U^{\bullet}, \tau) \gets (\Omega_X^{\bullet}(\log D), \tau) \to (\Omega_X^{\bullet}(\log D), W). \]
This identifies all of the hypercohomology spectral sequences. The final geometric fact that we need is the following:

\begin{lem}
    The identification $R^m j_* \C_U \cong a_{m*} \C_{D^{(m)}}$ is in fact defined over $\Z$. This means that the diagram
    \begin{equation*}
    \begin{tikzcd}
        R^m j_* \C_U \ar{r}{\sim} & a_{m*} \C_{D^{(m)}} \\
        R^m j_* \Z_U \ar{r}{\sim} \ar{u} & a_{m*} \Z_{D^{(m)}}(-m) \ar{u}
    \end{tikzcd}
    \end{equation*}
    commutes.
\end{lem}

\section{An exercise in homological algebra}%
\label{sec:an_exercise_in_homological_algebra}

The first thing we will do is consider the \textit{weight spectral sequence}, which is just the hypercohomology spectral sequence for $(\Omega_X^{\bullet}(\log D), W)$. Here, we have
\[ E_1^{-p,q} = \bH^{-p+q}(X, \on{Gr}_p^W \Omega_X^{\bullet}(\log D)) \implies H^{-p+q}(U, \C). \]
But we can identify the $E_1$-page with
\[ \bH^{-2p+q}(X, a_{p*}(\Omega^{\bullet}_{D^{(p)}})) \cong H^{-2p+q}(D^{(p)}, \C). \]
But now we see that everything is actually defined over $\Q$, so we now have a spectral sequence
\[ E_1^{-p,q} = H^{-2p+q}(D^{(p)}, \Q)(-p) \implies H^{-p+q}(U, \Q). \]
Everything here actually has a Hodge structure. But now the b\^ete filtration $F$ induces filtrations on the weight spectral sequence:
\begin{enumerate}
    \item (first direct filtration) $(F_d)$;
    \item (second direct filtration) $(F_{d^*})$;
    \item (inductive filtration) $(F_r)$.
\end{enumerate}
Some facts about these filtrations are:
\begin{enumerate}
    \item $F_d \subset F_r \subset F_{d^*}$;
    \item The differentials $\dd_r$ are compactible with the direct filtrations;
    \item If $\dd_r$ is strictly compatible with $F_r$ for $r \in \qty{0, \ldots, r_0 - 3}$, then $F_d = F_r = F_{d^*}$ for all $r \in \qty{0, \ldots, r_0}$. Here, $f \colon (H_1, F_1) \to (H_2, F_2)$ is \textit{strict} if for all $a \in F_2^p \cap \Im f$, then $f \in \Im F_1^p$. The important thing is that all morphisms of mixed Hodge structures are strict.
    \item This is basically just the previous fact for $r_0 = \infty$.\footnote{Kevin did not give a precise statement because he ran out of space on the board.}
\end{enumerate}

We are now ready to state the main theorem:
\begin{thm}\leavevmode
    \begin{enumerate}
        \item $F_d = F_r = F_{d^*}$ for all terms of the weight spectral sequence.
        \item The filtration $W$ on $H^k(U, \C)$ comes from a filtration on $H^k(U, \Q)$. Moreover, neither $W$ nor $F$ depends on the choice of compactification $X$.
        \item The filtrations $W[k]$ defined by $W[k]_m = W_{m-k}$ and $F$ make $H^k(U, \Z)$ a functorial mixed Hodge structure.
    \end{enumerate}
\end{thm}

\begin{lem}
    The hypercohomology spectral sequence for $(\on{Gr}_m^W \Omega_X^{\bullet}(\log D), F)$ degenerates at the $E_1$-page. Moreover, the induced filtration on $E_1^{-p,q} \cong H^{-2p+q}(D^{(p)}, \C)$ is $q$-opposite.
\end{lem}

\begin{proof}
    We know that $(\on{Gr}_m^w \Omega_X^{\bullet}(\log D), F) \cong a_{m*} (\Omega^{\bullet}_{D^{(m)}}, F) [-m]$. By the Hodge decomposition for $D^{(m)}$, we obtain degeneration at $E_1$, and by the Hodge structure on $H^{-2p+q}(D^{(p)}, \C)$ we obtain $q$-oppositeness.
\end{proof}

\begin{lem}
    The differential $\dd_1$ is strictly compatible with $F$.
\end{lem}

\begin{proof}
    Because $\dd_1$ is a morphism of Hodge structures, we are done.
\end{proof}

\begin{lem}
    The filtration $F_r$ on $E_2^{-p, q}$ is $q$-opposite.
\end{lem}

\begin{proof}
    We know that $E_2^{-p, q} = \mc{H}(E_1^{-p-1, q} \to E_1^{-p,q} \to E_1^{-p+1, q})$. Because $F_r$ is the induced filtration, we are done by the previous lemma.
\end{proof}

\begin{lem}
    For all $r \geq 0$, $\dd_r$ is strictly compatible with $F_r$. Moreover, for all $r \geq 2$, $\dd_r = 0$.
\end{lem}

\begin{proof}
    The $r=0$ case follows from the first lemma. For $r=1$, this is the second lemma. Finally, to prove degeneration, we can simply proceed by induction. We know that $F_d = F_r = F_{d^*}$ on $E_r$. But now by the second fact, we know that $\dd_r$ is compatible with $F_r$. Thus we have
    \begin{align*}
        \dd_r(E_r^{-p,q}) &= \dd_r \qty(\sum_{a+b=q} F^q (E_r^{-p,q}) \cap \ol{F}^b(E_r^{-p,q})) \\
        &\subset \sum_{a+b=q} F^a(E_r^{-p+r, q-r+1}) \cap \ol{F}^b (E_r^{-p+r, q-r+1}) \\
        &= 0. \qedhere
    \end{align*}
\end{proof}

By the final fact, we see that 
\[ \on{Gr}_p^{W[-p+q]} H^{-p+q}(U, \C) \cong E_2^{-p,q} \] is a subquotient of $H^{-2p+q}(D^{(p)}, \Q)(-r)$. Therefore, $(H^k(U, \Q), W[k], F)$ is a mixed Hodge structure.

Finally, we discuss the functoriality of the mixed Hodge structure. To do this, we want the filtrations
\[ (\Omega_X^{\bullet}(\log D), W, F) \]
to be functorial. For $U \to V$, we want to prove that $H^k(U, \C)$ is functorial. We also compactify the morphism to $X \to Y$. But now we know that $H^k(V, \C) \to H^k(U, \C)$ comes from
\[ \mathbb{H}^k(Y, \Omega_X^{\bullet}(\log D_X)) \to \mathbb{H}^k(Y, \Omega_Y^{\bullet}(\log D_Y)). \]
All we need to do is to show that there is a compactification that works, and this is given by resolving $X$ and taking $Y$ to be the closure of the graph.

To prove independence of the compactification, suppose $X_1, X_2$ are compactifications. Then let $X'$ resolve $X_1, X_2$ simultaneously, and then $X' \to \ol{U} \subset X_1 \times X_2$ is a resolution of singularities. By functoriality, the isomorphism
\[ \mathbb{H}^k(X, \Omega^{\bullet}_X(\log D_X)) \cong \mathbb{H}^k(X', \Omega^{\bullet}_X(\log D_{X'})) \]
is an isomorphism of mixed Hodge structures.

\chapter{Caleb (Nov 11): Cohomological descent}%
\label{cha:caleb_nov_11_cohomological_descent}

\begin{defn}
    A \textit{simplicial object} in a category $C$ is a functor $\Delta^{\mr{op}} \to C$, where $\Delta$ is the category with objects $[n]$ for $n \geq 0$ and morphisms are order-preserving set maps. There are distinguished functions $d^i \colon [n-1] \to [n]$ skipping $i$ and $s^i \colon [n+1] \to [n]$ repeating $i$.
\end{defn}

The maps $d^i, s^i$ induce maps $d_i \colon X_n \to X_{n-1}$ and $S_i \colon X_n \to X_{n+1}$ on a simplicial object. We will refer to simplicial objects in $C$ by $S(C)$ and denote $S_n(C) \coloneqq \Delta^{\mr{op}}_{\leq n} \to C$ and $S_+(C) \coloneqq \Delta_{\geq -1}^{\mr{op}} \to C$ given by adding $[-1] = \emptyset$ to $\Delta$.

For various reasons, we will assume that $C$ has all finite products and fiber products. If this is not sufficient, add assumptions until what is written here makes sense.

\section{Coskeleta}%
\label{sec:coskeleta}

Given a simplicial object $X_{\bullet} \in S(C)$, let $\mr{sk}_n(X) \in S_n(C)$ be the $n$-th truncation of $X_{\bullet}$.

\begin{defn}
    Let $Y_{\bullet} \in S_n(C)$. Then the \textit{coskeleton} functor is defined by
    \[ \Hom_{S_n(C)}(\mr{sk}(X_{\bullet}), Y_{\bullet}) \simeq \Hom_{S(C)}(X_{\bullet}, \mr{cosk}_n(Y_{\bullet})). \]
\end{defn}

\begin{exm}
    Let $n = 0$ and $Y_0 \in C$. Given $X_0 \to Y_0$, consider
    \begin{equation*}
    \begin{tikzcd}
        X_2 \ar[shift left=3]{r} 
        \ar[shift right=1]{r}
        \ar[shift left=1]{r}
        & X_1 \ar{r} \ar[shift left=3]{l} 
        \ar[shift left=2]{r}
        & X_0 \ar{r}{f} 
        \ar[shift left=2]{l}
        & Y_0.
    \end{tikzcd}
    \end{equation*}
    We obtain maps $X_0 \to Y_0 \times Y_0$ and $X_2 \to Y_0 \times Y_0 \times Y_0$, and so on. Then we define
    \[ \mr{cosk}_0(Y_0)_n = Y^n \]
    and give it the natural simplicial structure.
\end{exm}

For an example of a simplicial object, if $Y_{-1}$ is a topological space and $Y_0 = \bigsqcup U_i$, where $U_i$ form an open cover of $Y_{-1}$, then the $Y_i$ are the $i$-intersections.

Now choose $Y_{\bullet} \in S_m(C)$. Set 
\[ \mr{cosk}_m(Y_{\bullet})_n = \varprojlim_{\mr{sk}_m(\Delta[n])} Y_k = \lim_{\substack{[k] \to [n] \\ k \leq m}} Y_k. \]
Given a map $\mr{sk}_n(X_{\bullet}) \to Y_{\bullet}$, we need to define a map $X_n \to \lim_{\substack{[k] \to [n]}} Y_k$. For every $\varphi \colon [k] \to [n]$ and $k \leq m$, we have a morphism
\[ X_n \xrightarrow{\varphi^*} X_k \to Y_k \]
for $k \leq n$.

\begin{prop}
    The map $X_{\bullet} \gets \mr{sk}_m(\mr{cosk}_m(X_{\bullet}))$ is an isomorphism.
\end{prop}
The proof essentially follows from the construction. For $n \geq m$, the single copy of $Y_n$ corresponding to $[n] \to [n]$ determines everything.

\begin{rmk}
    The skeleton $\mr{sk}_m$ also has a left adjoint, which is the inclusion $S_n(C) \to S(C)$.
\end{rmk}

The adjunction $\mr{id} = \mr{cosk}_m \mr{sk}_m$ is not generally an isomorphism, but is an isomorphism for $n$-coskeleta when $0 \leq n \leq m$.

\section{Hypercoverings}%
\label{sec:hypercoverings}

\begin{defn}
    Let $P$ be a class of morphisms containing all isomorphisms and stable under base change and composition. Then a simplicial object $X_{\bullet}$ of $C$ is a \textit{$P$-hypercovering} if for all $n \geq 0$ (or $n \geq -1$), the natural adjunction $X_{\bullet} \to \mr{cosk}_n(\mr{sk}_n(X_{\bullet}))$ induces $X_{n+1} \to \mr{cosk}_n(\mr{sk}_n(X_{\bullet}))_{n+1}$ is in $P$.
\end{defn}

\begin{exm}
    The map $\mr{cosk}_0(S' / S) \to S$ is a $P$-hypercovering if and only if $S' \to S$ is in $P$. 
\end{exm}

For example, if $P$ is a surjective condition for topological spaces, then $S' = \bigsqcup U_i \twoheadrightarrow S$ obtains the \v{C}ech construction.

We may compute cohomology using hypercoverings. Define $\check{H}^i(S, \mc{F}) = H^i(\mr{Moore}(\mc{F}(S)))$, where the Moore complex was defined in a different seminar by Caleb. If we take the limit over all hypercoverings, we recover the derived functor cohomology.

We are most interested in hypercoverings of singular schemes, which we will be used to construct Hodge structures.

\begin{defn}
    A simplicial object $X_{\bullet} \in S(C)$ is \textit{split} if there exist subobjects $NX_j$ with isomorphisms 
    \[ \bigoplus_{[n] \twoheadrightarrow [m]} N X_m \to X_n \]
\end{defn}

\begin{rmk}
    In the case of simplicial abelian groups, we have
    \[ NX_n = \bigcap_{i=0}^{n-1} \ker d_i. \]
\end{rmk}

Under nice conditions, split objects have unique splittings. A nice fact is that given split $n$-truncated objects $\mr{sk}_n(X)$ and $NX_{n+1} \to (\mr{cosk}_n(\mr{sk}_n(X)))_{n+1}$, we can recover $X_{n+1}$. This is because we know that
\[ X_{n+1} = \bigoplus_{[n+1] \to [m]} NX_m \]
and the map to the coskeleton tells us what the maps out of $X_{n+1}$ are.

\begin{thm}
    If $\on{sk}_n X$ is split, then there exists $f \colon X' \to X$ with $\on{sk}_n(f)$ an isomorphism and $X'$ split. Furthermore, if $X$ is an augmented $P$-hypercover, so is $X'$.
\end{thm}

From now on, to be safe, we will assume that we are in one of the following cases:
\begin{enumerate}
    \item $C = \ms{Top}/B$ with $P$ the proper surjective morphisms;
    \item $C$ is the category of schemes \'etale over some base with $P$ the surjective \'etale morphisms;
    \item $C$ is a topos with $P$ the epimorphisms.
\end{enumerate}

\begin{cor}
    For $m \geq 0$ and an augmented $m$-truncated $P$-hypercovering $Z$ in one of the first two situations, the face and degeneracy maps for $Z$ are proper or \'etale. If $X_{\bullet} \to S$ is a proper or \'etale hypercovering, then all structure maps are proper or \'etale.
\end{cor}

\begin{thm}
    Let $S$ be a separated scheme over a field $k$. Then there exists a dense open immersion $S \hookrightarrow \ol{S}$ into a proper $k$-scheme and an augmented proper hypercovering $\ol{X}_{\bullet} \to \ol{S}$ such that $\ol{X}_n$ is a projective regular $k$-scheme and the part of $\ol{X}_n$ lying over $\ol{S} \setminus S$ is a (strict) normal crossings divisor.
\end{thm}

In de Jong's alterations paper, he mentions that this result as an application. Of course, in characteristic $0$, we simply need resolution of singularities, but in positive characteristic, de Jong's alterations theorem is needed.

\begin{proof}
    First, to construct the open immersion, we use Nagata's compactification theorem. Of course, if we are interested in quasiprojective varieties, then this is unnecessary. The upshot is that now we have a dense open immersion $S \hookrightarrow \ol{S}$ into a proper $k$-scheme.

    By Johan, we obtain a regular $\ol{X}_0 \to \ol{S}$ wich is a proper surjection, where the preimage of $\ol{S} \setminus S$ is a strict normal crossings divisor. Now given $X_{\leq m}$, we know that $\on{cosk}_m X_m$ is a proper hypercovering of $S$. We know that each term is $S$-proper. By Johan applied to $(\on{cosk}_n X_{\leq m})_{m+1}$, we obtain $\ol{X}'$ proper over $(\on{cosk}_m X_{\leq m})_{m+1}$ such that the part over $\ol{S} \setminus S$ is a normal crossings divisor.

    Now set $NX_{m+1} = \ol{X}'$ and take the construction
    \[ X_{n+1} = \bigoplus_{[n+1] \twoheadrightarrow [m]} N X_m, \]
    and here we obtain an $(m+1)$-truncated solution. By induction, we are done.
\end{proof}

\section{Cohomological descent}%
\label{sec:cohomological_descent}

Let $C$ be a site with the topology generated by $E$-morphisms for some class of morphisms $E$. Let $X_{\bullet}$ be a simplicial object of $C$. Set $\wt{X}_{\bullet}$ to be the category sheaves on the following site:
\begin{itemize}
    \item The objects are $E$-morphisms $U \to X_n$;
    \item The morphisms are commutative squares
        \begin{equation*}
        \begin{tikzcd}
            U \ar{r}{f} \ar{d} & U' \ar{d} \\
            X_n \ar{r} & X_n'
        \end{tikzcd}
        \end{equation*}
        where $f$ is any morphism in $C$;
    \item A covering of $[U_i \to X_n]$ is given by any covering of $U_i$.
\end{itemize}
This is meant to formalize the notion of sheaves on $X_{\bullet}$. A map $U_{\bullet} \colon X_{\bullet} \to Y_{\bullet}$ induces maps $U_{\bullet *} \colon \wt{X}_{\bullet} \to \wt{Y}_{\bullet}$ and $U_{\bullet}^* \colon \wt{Y}_{\bullet} \to \wt{X}_{\bullet}$.

Now let $a \colon X_{\bullet} \to S$. We have $a^* \colon \wt{S}_{\bullet} \to \wt{X}_{\bullet}$. Here, we have $(a^* \mc{F})_n = a_n^* \mc{F}$. Also, we have $a_* \colon \wt{X}_{\bullet} \to \wt{S}_{\bullet}$, and $a_* \mc{F}^{\bullet}$ is the equalizer of
\[ a_{0*} \mc{F}^0 \rightrightarrows a_{1*} \mc{F}^1. \]
On the level of derived categories, we obtain $a^* \colon D_+(S) \to D_+(X_{\bullet})$ and $Ra_* \colon D_+(X_{\bullet}) \to D_+(S)$.

\begin{defn}
    A morphism $a \colon X_{\bullet} \to S$ is a \textit{morphism of cohomological descent} if the natural transformation $\mr{id} \to Ra_* \circ a^*$ on $D_+(S)$ is an isomorphism.
\end{defn}
Note that the adjunction of $(a^*, Ra_*)$ means that equivalently, $a^* \colon D_+(S) \to D_+(S_{\bullet})$ is fully faithful.

\begin{defn}
    A morphism $a$ is \textit{universally of cohomological descent} if every base change $a_{S'} \colon X_{\bullet} \times_S S' \to S'$ is of cohomological descent.
\end{defn}

\begin{defn}
    A map of spaces $a_0 \colon X_0 \to S$ is a \textit{map of cohomological descent} if $\on{cosk}_0(a_0) \colon \on{cosk}_0(X_{\bullet}/S) \to S$ is of cohomological descent.
\end{defn}

\begin{thm}
    Morphisms universally of cohomological descent form a Grothendieck topology.
\end{thm}

We now discuss some properties and applications. Unfortunately, because of time reasons, we cannot review any descent theory.

First, the condition that $\mr{id} \to Ra_* \circ a^*$ is an isomorphism on $D_+(S)$ is equivalent to the isomorphism
\[ \mc{F} \cong a_* a^* \mc{F} = \ker(a_{0*} a_0^* \mc{F} \to a_{1*} a_1^* \mc{F}) \]
and $R^i a_*(a^* \mc{F}) = 0$. Note that this is a derived version of ordinary descent theory.

\begin{thm}
    Let $X_{\bullet}$ be a simplicial space. For any complex $K'$ in $D_+(X_{\bullet})$, there exists a spectral sequence
    \[ E_1^{p,q} = \mathbb{H}^1(X_p, K'|_{X_p}) \mapsto \mathbb{H}^{p+q}(X_{\bullet} K'). \]
\end{thm}

On the $0$-skeleton, we have
\begin{thm}
    Let $X \to S$ be a proper surjective of either topological spaces or schemes with the \'etale topology. Then $X \to S$ is univerally of cohomological descent.
\end{thm}

\begin{thm}
    Let $X_{\bullet} \to S$ be a proper hypercovering. Then it is universally of cohomological descent.
\end{thm}

All of these theorems have extremely long proofs.

\chapter{Kevin (Nov 18): Applications and examples of mixed Hodge structures; Hodge theory for all varieties}%
\label{cha:kevin_nov_18_}

\section{End of Hodge II}%
\label{sec:end_of_hodge_ii}

The main theorem is as follows:
\begin{thm*}
    The weight spectral sequence
    \[ E_1^{-p,q} = H^{-2p+q}(D^{(p)}, \Q)(-p) \implies H^{-p+q}(U, \Q) \]
    degenerates at the $E_2$ page. The Hodge structures on $E_2$ induce a mixed Hodge structure on $H^{-p+q}(U, \Q)$.
\end{thm*}

\begin{cor}
    If $H^k(U, \C)$ has nonzero weight $(p, q)$, then $0 \leq p,q \leq k$ and $k \leq p + q \leq 2n$.
\end{cor}

\begin{proof}
    We consider the spectral sequence. All terms that do appear must lie above a line of slope $-2$.\footnote{Kevin didn't type this, so I'm not typing it either.} The lines of slope $-1$ are the cohomology, and so by staring at this for long enough, we obtain the desired result.\footnote{Kevin said he was not going to do this, and then he confused me.}
\end{proof}

\begin{cor}\leavevmode
    \begin{enumerate}
        \item If $X$ is any smooth compactification of $U$, then $\Im(H^k(X, \Q) \to H^k(U, \Q))$ is the bottom weight part $W_k(H^k(U, \Q))$.
        \item If $Y \to U \hookrightarrow X$ is any morphism with $Y$ smooth and proper, then 
            \[ \Im(H^k(U, \Q) \to H^k(Y, \Q)) = \Im(H^k(X, \Q) \to H^k(Y, \Q)). \]
    \end{enumerate}
\end{cor}

\begin{proof}\leavevmode
    \begin{enumerate}
        \item We will use the rising sea (Grothendieck) fact that $\Im(H^k(X, \Q) \to H^k(U, \Q))$ does not depend on the choice of $X$. Therefore we may assume that $X \setminus U$ is a simple normal crossings divisor. Now the weight spectral sequence for $X$ maps to the weight spectral sequence for $U$. But recall that the weight spectral sequence for $X$ only consists of $H^k(X)$ in the rightmost column, while the weight spectral sequence for $U$ has nontrivial things to the left. Therefore, 
            \[ \Im(H^k(X, \Q) \to H^k(U, \Q)) = E_2^{0,k} = W_k(H^k(U, \Q)). \]
        \item Recall that maps of mixed Hodge structures are strict. This implies that
            \[ \Im(H^k(U, \Q) \to H^k(Y, \Q)) = \Im(W_k(H^k(U, \Q)) \to H^k(Y, \Q)). \]
            By the previous part of this corollary, we are done. \qedhere
    \end{enumerate}
\end{proof}

\begin{thm}[Global invariant cycles theorem]
    If $f \colon U \to S$ is smooth and proper with $S$ smooth and separated, and $X$ is a smooth compactification of $U$, then $\Im(H^k(X, \Q) \to H^k(U_s, \Q)) = H^k(U_s, \Q)^{\pi_1(S, s)}$.
\end{thm}

\begin{proof}
    We will only prove the projective case. First, we use the fact that $H^k(U, \Q) \twoheadrightarrow H^k(U_s, \Q)^{\pi_1(S, s)}$. This follows from the degeneration of the Leray spectral sequence for smooth projective morphisms, which gives
    \[ H^k(U, \Q) \twoheadrightarrow H^0(S, R^k f_* \Q_U) \cong H^k(U_s, \Q)^{\pi_1(S, s)}. \]
    Using the second corollary for $U_s \hookrightarrow U \hookrightarrow X$, we obtain
    \begin{align*}
        H^k(U_s, \Q)^{\pi_1(S, s)} &= \Im(H^k(U, \Q) \to H^k(U_s, \Q)) \\
        &= \Im(H^k(X, \Q) \to H^k(U_s, \Q))
    \end{align*}
    using the previous corollary. In the general case, we dominate $U$ by something projective.
\end{proof}

\begin{rmk}
    Using the theorem, we can prove the degeneration of the Leray spectral sequence for smooth proper morphisms.
\end{rmk}

Now we will do some computations.
\begin{enumerate}
    \item Consider $U = \C^{\times} \hookrightarrow X = \P^1$. Then $D = \mr{pt} \sqcup \mr{pt}$. The only nonzero differential is
        \[ \Q^2 = H^0(D)(-1) \to H^2(\P^1) = \Q, \]
        and by the cohomology of $\C^{\times}$, we obtain a $\Q$ in the $(0,0)$ position and a $\Q$ in the $(-1,2)$ position. Thus $H^0$ is a pure Hodge structure, but the first cohomology $H^1(\C^{\times}, \Q) = \operatorname{Gr}_2^W H^1(U, \Q)$ actually lives in weight $2$.
    \item Consider a punctured Riemann surface\footnote{Kevin has been posessed by the spirit of D.H. Phong.} $U = R_g \setminus D$, where $D$ is a disjoint union of $p$ points. There is only one nontrivial differential, which is surjective, so
        \[ \on{Gr}_1^W H^1(U, \Q) = \Q^{2g}, \qquad \on{Gr}_2^W H^1(U, \Q) \cong \Q^{p-1}. \]
\end{enumerate}

\section{Hodge III}%
\label{sec:hodge_iii}

Kevin claims that despite the paper being 70 pages, the actual content is only two pages. Today, we will discuss the other 68 pages.

Our goal is to construct a mixed Hodge structure on $H^k(X, \Q)$ for any complex variety $X$. Our strategy is to take a proper hypercovering $Y_{\bullet} \to X$ such that each $Y_n$ is smooth. We will use cohomological descent, which will tell us that
\[ H^k(X, \Q) \cong H^k(Y_{\bullet}, \Q). \]
The right hand side is approximated by $H^k(Y_n, \Q)$ (in the sense that there is some spectral sequence converging to $H^k(Y_{\bullet}, \Q)$), and each of the terms has a mixed Hodge structure, which will induce a mixed Hodge structure on $H^k(Y_{\bullet}, \Q)$.

Now recall the following from Caleb's lecture:
\begin{itemize}
    \item A \textit{simplicial variety} is a sequence of varieties $Y_{\bullet}$ with face and degeneracy maps (a simplicial object in the category of varieties).
    \item An \textit{augmented simplicial variety} $Y_{\bullet} \to X$ is simply a simplicial variety $Y_{\bullet}$ with a map $Y_0 \to X$.
    \item A sheaf $\mc{F}^{\bullet}$ on a simplicial variety is a collection of sheaves $\mc{F}^n$ on each $Y_n$ equipped with maps $\mc{F}^{\bullet}(p) \colon \mc{F}^n \to p_* \mc{F}^m$, where $p \colon [n] \to [m]$ is a map in the simplex category. Of course, these maps are required to be compatible.
\end{itemize}

\begin{rmk}
    If $S_{\bullet}$ is a constant simplicial variety, then a sheaf on $S_{\bullet}$ is a cosimplicial sheaf on $S$.
\end{rmk}

Given a map $f \colon Y_{\bullet}^1 \to Y_{\bullet}^2$, we can define $f_*, f^*$ pointwise. If we have an augmentation $a \colon Y_{\bullet} \to X$, then we define $(a^* \mc{F})^n = a_n^* \mc{F}$, while we define
\[ a_* \mc{G}^{\bullet} = \ker (a_{0*} \mc{G}^0 \rightrightarrows a_{1*} \mc{G}^1). \]
When $X$ is a point, then $a_* = \Gamma$ is the global sections functor. We may then define the right derived functors
\[ R^k \Gamma(-) = H^k(Y_{\bullet}, -). \]

\begin{defn}
    A \textit{proper hypercovering} $Y_{\bullet} \to X$ is a hypercovering such that
    \[ Y_{n+1} \to (\on{cosk}_n \on{sk}_n Y_{\bullet})_{n+1} \]
    is proper and surjective for all $n \geq -1$. For $n = -1$, this says that $Y_0 \to X$ is a proper surjection.
\end{defn}

\begin{thm}[SGA4]
    Proper hypercoverings satisfy cohomological descent. This means that $\mc{F} \to R a_* a^* \mc{F}$ is an isomorphism.
\end{thm}

\begin{cor}
    If $Y_{\bullet} \to X$ is a proper hypercovering, then
    \[ H^k(X, \mc{F}) \cong \bH^k(X, Ra_* a^* \mc{F}) \cong H^k(Y_{\bullet}, a^* \mc{F}). \]
\end{cor}

Thus we can replace $X$ by some simplicial resolution (proper hypercovering) $Y_{\bullet} \to X$ and then do all of our computations on $Y_{\bullet}$. Then whatever structure we have on $Y_{\bullet}$ will descends to $X$.

\section{Mixed Hodge complexes}%
\label{sec:mixed_hodge_complexes}

Apparently all of the homological algebra we did in Hodge II is very general, so we can rip it off.\footnote{Kevin really felt like ripping things off today.} This is exactly what we will do now.

\begin{defn}
    The \textit{filtered derived category} $DF(C)$ of an abelian category $C$ is constructed by taking the category of filtered complexes in $C$, identifying homotopic maps (that respect the filtrations), and inverting filtered quasi-isomorphisms.

    The \textit{bifiltered derived category} $DF_2(C)$ is defined in the same way, except we have two filtrations.
\end{defn}

\begin{defn}
    For $A \in \qty{\Z, \Q, \R}$, an \textit{$A$-Hodge complex of weight $n$} is the following data:
    \begin{enumerate}
        \item A complex $K_A \in D^+(A)$ such that $H^k(K_A)$ is finitely generated;
        \item A filtered complex $(K_{\C}, F) \in D^+ F(\C)$ (where $F$ is a decreasing filtration) and an isomorphism $\alpha \colon K_A \otimes_A \C \to K_{\C}$ in $D^+(\C)$.
    \end{enumerate}
    This data must satisfy the following conditions:
    \begin{enumerate}
        \item The differential $\dd$ on $K_{\C}$ is strict with respect to $F$.
        \item For all $k$, $(H^k(K_A), F)$ is a Hodge structure of weight $n+k$.
    \end{enumerate}
\end{defn}

\begin{defn}
    Let $X$ be a space. An \textit{$A$-Hodge complex of sheaves of weight $n$} is the following data:
    \begin{enumerate}
        \item A complex $K_A \in D^+(X, A)$;
        \item A filtered complex $(K_{\C}, F) \in D^+ F(X, \C)$ and an isomrphism $\alpha \colon K_A \otimes_A \C \to K_{\C}$.
    \end{enumerate}
    We also require that $(R \Gamma(K_A), R\Gamma((K_{\C}, F)), R\Gamma(\alpha))$ is a Hodge complex of weight $n$.
\end{defn}

\begin{exm}
    If $X$ is smooth and proper, then $(\Z_X, (\Omega_X^{\bullet}, F), \alpha)$ is a Hodge complex of sheaves of weight $0$. Here, $\alpha$ is the map $\C_X \simeq \Omega_X^{\bullet}$.
\end{exm}

\begin{defn}
    A \textit{mixed Hodge complex} is the following data:
    \begin{enumerate}
        \item A complex $K \in D^+(\Z)$ such that $H^k(K)$ is finitely generated;
        \item A filtered complex $(K_{\Q}, W) \in D^+ F(\Q)$ (with $W$ an increasing filtration) and an isomorphism $\alpha \colon K \otimes_{\Z} \Q \cong K_{\Q}$;
        \item A bifiltered complex $(K_{\C}, W, F) \in D^+ F_2(\C)$ and an isomorphism of filtered complexes $\beta \colon (K_{\Q}, W) \otimes_{\Q} \C \to (K_{\C}, W)$.
    \end{enumerate}
    We require that for all $n$, $(\on{Gr}_n^W K_{\Q}, (\on{Gr}_n^W K_{\C}, F), \beta)$ is a $\Q$-Hodge structure of weight $n$.
\end{defn}

\begin{defn}
    A \textit{mixed Hodge complex of sheaves} is the following data:
    \begin{enumerate}
        \item A complex $K \in D^+(X, \Z)$ such that $\bH^k(X, K)$ is finitely generated;
        \item A filtered complex $(K_{\Q}, W) \in D^+ F(X, \Q)$ and an isomorphism $K \otimes_{\Z} \Q \simeq K_{\Q}$;
        \item A bifiltered complex $(K_{\C}, W, F) \in D^+ F_2(X, \C)$ and an isomorphism of filtered complexes $\beta \colon (K_{\Q}, W) \otimes_{\Q} \C \to (K_{\C}, W)$.
    \end{enumerate}
    We require that for all $n$, $(\on{Gr}_n^W K_{\Q}, (\on{Gr}_n^W K_{\C}, F), \beta)$ is a $\Q$-Hodge complex of sheaves of weight $n$.
\end{defn}

Here are some facts about these things:
\begin{enumerate}
    \item If $(K, (K_{\Q}, W), \alpha, (K_{\C}, W, F), \beta)$ is a mixed Hodge complex of sheaves, then
        \[ (R \Gamma(K), (R \Gamma(K_{\Q}), W), R \Gamma(\alpha), (R\Gamma(K_{\C}), W, F), R \Gamma(\beta)) \]
        is a mixed Hodge complex.
    \item For a smooth variety $U$, the data $(R j_* \Z_U, (R j_* \Q_U, \tau), \alpha, (\Omega_X^{\bullet}(\log D), W, F), \beta)$ is a mixed Hodge complex of sheaves on $X$. Here, $\beta$ is defined as the zigzag
        \begin{equation*}
        \begin{tikzcd}
            (Rj_* \Q_U, \tau) \otimes_{\Q} \C \ar{d} & (j_* \Omega^{\bullet}_U, \tau) \ar{d} & (\Omega_X^{\bullet}(\log D), \tau) \ar{l} \ar{d} \\
            (R j_* \C_U, \tau) \ar{r} & (R j_* \Omega_U^{\bullet}, \tau)  & (\Omega_X^{\bullet}(\log D), W) 
        \end{tikzcd}
        \end{equation*}
    \item The main theorem of Hodge II says that $(H^k(\text{MHC}), W[k], F)$ is a mixed Hodge structure.
\end{enumerate}

\chapter{Kevin (Dec 02): Hodge theory for all varieties II}%
\label{cha:kevin_dec_02_hodge_theory_for_all_varieties_ii}

Recall our geometric setup, where we have a proper hypercovering $Y_{\bullet} \to X$ of $X$ by smooth varieties. We may assume that $Y_{\bullet} \hookrightarrow \ol{Y}_{\bullet}$, where $\ol{Y}_n = Y_n \sqcup D_n$ and $D_n$ is a simple normal crossings divisor. Because proper hypercoverings satisfy cohomological descent, we can work on $Y_{\bullet}$. Thus we need to do Hodge theory on simplicial varieties.

\section{Existence of mixed Hodge structures}%
\label{sec:existence_of_mixed_hodge_structures}

\begin{defn}
    A \textit{mixed Hodge complex of sheaves on a simplicial space} $Y_{\bullet}$ is a tuple 
    \[ (\mc{K}^{\bullet \bullet}, (\mc{K}_{\Q}^{\bullet \bullet}, W), \alpha, (\mc{K}_{\C}^{\bullet\bullet}, W, F), \beta) \] 
    of sheaves on $Y_{\bullet}$ such that we have a mixed Hodge complex of sheaves on each $Y_n$. Here, the first grading is the degree in the complex and the second is the (co)simplicial degree.
\end{defn}

\begin{exm}
    An example of a mixed Hodge complex of sheaves on $\ol{Y}_{\bullet}$ is given by
    \[ (R j_* \Z_{Y_{\bullet}}, (R j_* \Q_{Y_{\bullet}}, \tau_{\leq}), \alpha, (\Omega^{\bullet}_{\ol{Y}_{\bullet}}(\log D_{\bullet}), W, F), \beta). \]
    Taking the derived functor of global sections on each $\ol{Y}_n$, we obtain a cosimplicial mixed Hodge complex.
\end{exm}

Now we wamt to obtain a complex (in the usual sense) with the same cohomology as a given cosimplicial complex. Suppose we have a sheaf $\mc{F}^{\bullet}$ on $\ol{Y}_{\bullet}$. We find some injective resolution $\mc{F}^{\bullet} \hookrightarrow \mc{K}^{\bullet \bullet}$. Next, we consider the cosimplicial complex
\[ K^{pq} \coloneqq \Gamma(\ol{Y}_q, \mc{K}^{pq}), \]
and this gives us a double complex $K^{\bullet \bullet}$ using the Moore complex. Finally, from a double complex we obtain an ordinary complex
\[ \mathbf{s}K^n = \bigoplus_{p + q = n} \Gamma(X_q, \mc{K}^{pq}) \]
with differential
\[ \dd {x^{pq}} = \dd_{\mc{K}}(x^{pq}) + (-1)^p \sum_i (-1)^i \delta_i(x^{pq}). \]
Also, we have
\[ \Gamma(\ol{Y}_{\bullet}, \mc{K}^{\bullet \bullet}) \simeq \mathbf{s}(K^{\bullet \bullet}) = \mathbf{s} \Gamma^{\bullet}(\ol{Y}_{\bullet}, \mc{K}^{\bullet \bullet}) \]
and also $R \Gamma(\ol{Y}_{\bullet}, \mc{F}^{\bullet}) \simeq \mathbf{s} R \Gamma^{\bullet}(\ol{Y}_{\bullet}, \mc{F}^{\bullet})$.

\begin{rmk}
    There exists a filtration $L$ on $\mathbf{s}K^{\bullet}$ given by $L^r(\mathbf{s}K)^{\bullet} = \bigoplus_{q \geq r} K^{pq}$. This gives us a spectral sequence
    \[ E_1^{st} = H^t(X_s, \mc{F}^s) \implies H^{s+t}(X_{\bullet}, F^{\bullet}). \]
\end{rmk}

Somehow all of the homological algebra we did applies to bounded below complexes of sheaves. The good thing about this is that now $\mathbf{s} R \Gamma^{\bullet} R j_* \Z_{Y_{\bullet}}$ computes the cohomology of $X$. All we need to do is to prove that this is a mixed Hodge complex, and this means that we need to become Deligne.

\begin{thm}[Main theorem]
    Let $K^{\bullet \bullet}$ be a cosimplicial mixed Hodge complex. Then there exists a natural filtration $\delta(W, L)$ on $\mathbf{s} K^{\bullet}$ such that $(s K^{\bullet}, \delta(W, L), F)$ is a mixed Hodge complex.
\end{thm}

\begin{proof}
    Define
    \[ \delta(W, L)_n (\mathbf{s}K^{\bullet}_{\Q}) = \bigoplus_{p,q} W_{n+p}(K_{\Q}^{pq}). \]
    Then we can compute
    \[ \on{Gr}_n^{\delta(W,L)} (\mathbf{s}K^{\bullet}_{\Q}) = \bigoplus_p \on{Gr}_{n+p}^W K_{\Q}^{\bullet p} [-p]. \]
    But now each $K_{\Q}^{\bullet p}$ is a mixed Hodge complex, and thus each of the associated graded pieces is a pure Hodge complex of weight $n$. This implies that $\mathbf{s}K^{\bullet}$ is a mixed Hodge complex.
\end{proof}

Applying the main theorem to $\mathbf{s} R \Gamma^{\bullet} R j_* \Z_{Y_{\bullet}}$, we see that $H^n(X, \Z)$ is a mixed Hodge structure. This is functorial because it is possible to construct compatible proper hyperresolutions, and this implies that it is unique.

If $X$ is smooth, we can choose the proper hyperresolution to simply be $X$ itself, so we have the same mixed Hodge structure as before. From the mixed Hodge complex, we have a weight spectral sequence
\[ E_1^{-a,b} = \bigoplus_{\substack{p+2r = b \\ q-r = -a}} H^p(D_q^{(r)}, \Q)(-r) \implies H^{-a+b}(X, \Q). \]
This spectral sequence degenerates at $E_2$, and the differentials on $E_1$ are sums of Gysin maps.

\begin{prop}
    All of the Hodge numbers $h^{st}$ that appear in $H^n(X, \Q)$ satisfy $0 \leq s,t \leq n$.
\end{prop}

\begin{proof}
    First, all of the $E_1^{-a,b}$ have nonnegative Hodge components. To prove the upper bound, if $H^p(D_q^{(r)}, \Q)(-r)$ contributes to $H^n(X, \Q)$, then $p+q+r = n$. In particular, $p + r \leq n$ and any $h^{s't'}$ appearing in $H^p(D_q^{(r)}, \Q)$ has $0 \leq s', t' \leq p$, so when we twist by $(-r)$, we get $s,t \leq p+r \leq n$.
\end{proof}

\begin{prop}
    If $X$ is proper, then all of the weights of $H^n(X, \Q)$ are at most $n$.
\end{prop}

\begin{proof}
    If $X$ is proper, then each $Y_n$ is already proper, so each $D_n = \emptyset$. Therefore, considering the spectral sequences, we only have nonzero terms for $r = 0$, so the spectral sequence becomes
    \[ E_1^{-a,b} = H^b(Y_{-a}, \Q) \implies H^{-a+b}(X, \Q). \]
    If $H^b(Y_{-a}, \Q)$ contributes to $H^n(X, \Q)$, then we must have $b \leq n$.
\end{proof}

\section{Hodge characteristic}%
\label{sec:hodge_characteristic}

Given a variety $X = U \sqcup Z$, then 
\[ \chi^c(X) = \chi^c(U) + \chi^c(Z). \] For complex varieties, $\chi = \chi^c$, but here we use compactly supported cohomology. For example, if $X$ is compact, consider the long exact sequence
\[ \cdots \to H^n(X, \Z) \to H^n(X) \to H^n(Z) \to \cdots \]
and then use the fact that $H_c^n(U) = H^n(X, Z)$. Also note that 
\[ \chi^c(X \times Y) = \chi^c(X) \chi^c(Y). \]
This tells us that the compactly supported Euler characteristic is motivic and defines a ring homomorphism
\[ \chi^c \colon K_0(\ms{Var}) \to \Z. \]
Here, $K_0(\ms{Var})$ is the \textit{Grothendieck ring of varieties} with relations $[X] = [U] + [Z]$ and $[X \times Y] = [X] [Y]$. This loses a lot of information, however, so we construct something better.

\begin{thm}
    Let $Y \subset X$ be a locally closed subvariety. There is a natural mixed Hodge structure on each $H^n(X, Y, \Z)$ respecting the long exact sequence of the pair $(X, Y)$.
\end{thm}

\begin{thm}
    The (relative) K\"unneth theorem respects mixed Hodge structures.
\end{thm}

\begin{exm}
    We would like to construct a mixed Hodge structure on $H_c^n(\A^1)$. We know this is $\Q$ in degree $2$ and vanishes everywhere else. There is an isomorphism of mixed Hodge structures $H^2_c(\A^1) \cong H^2(\P^1)$, and thus $H^2_c(\A^1)$ is pure of type $(1,1)$. By the K\"unneth formula, $H_c^{2n}(\A^n)$ is pure of type $(n,n)$.
\end{exm}

Recall that any Hodge structure $V$ defines an element $[V_{\Q}] \in K_0(\ms{HS})$ by taking graded pieces. Now define
\[ P_{hn}([V_{\Q}]) = \sum_{p,q} h^{p,q} u^p v^q. \]
\begin{defn}
    The \textit{Hodge characteristic with compact support} is the ring morphism
    \[ \chi_{\mr{Hdg}}^c \colon K_0(\ms{Var}) \to K_0(\ms{hs}) \qquad X \mapsto \sum_i (-1)^i [H_c^i(X, \Q)]. \]
\end{defn}

\begin{defn}
    The \textit{Hodge-Euler polynomial with compact support} is the ring morphism
    \[ e_{\mr{Hdg}}^c \colon K_0(\ms{Var}) \to \Z[u,v] \qquad X \mapsto P_{hn}(\chi_{\mr{Hdg}}^c(X)). \]
\end{defn}

\begin{rmk}
    If $X$ is smooth and proper, then the Hodge-Euler polynomial is a generating function for the Hodge numbers of $X$.
\end{rmk}

\begin{exm}
    Let $X$ be a torix variety and let $s_k$ be the number of torus orbits of dimension $k$. Therefore, we know that
    \[ e_{\mr{Hdg}}^c(X) = \sum_k s_k e_{\mr{Hdg}}^c((\C^{\times})^k). \]
    By the K\"unneth formula and the computation for $\A^1$, we have
    \[ e_{\mr{Hdg}}^c(X) = \sum_k s_k (uv-1)^k. \]
    In the case when $X$ is smooth and proper, the Poincar\'e polynomial is $\sum_k s_k (t^2-1)^k$.
\end{exm}

\begin{exm}
    Let $X$ be proper irreducible nodal curve with geometric genus $g$. Assume that $X$ has $p$ nodes. From topology, we know that $H^0(X, \Q) = \Q$ is pure of type $(0,0)$. Also, $H^2(X, \Q) \cong \Q$ is also pure of type $(1,1)$. It remains to compute the mixed Hodge structure on $H^1(X, \Q)$. This tells us that
    \begin{align*} 
        e_{\mr{Hdg}}^c(X) &= e_{\mr{Hdg}}^c(\Sigma_g) - 2p e_{\mr{Hdg}}^c(\mr{pt}) + p e_{\mr{Hdg}}^c(\mr{pt}) \\
        &= (1-gu-gv + uv) 0 2p + p = (1+uv) - (p + gu + gv),
    \end{align*}
    and thus we have $h^{00} = p, h^{10} = g, h^{01} = g$.
\end{exm}

\chapter{Caleb (Dec 09): The Ax-Schanuel conjecture for period maps}%
\label{cha:the_ax_schanuel_conjecture_for_period_maps}

\textit{Note: this lecture was given over the course of two seminars on two consecutive days.} 

\section{Polarized Hodge structures}%
\label{sec:polarized_hodge_structures}

One motivating question is the one of when a complex torus $\C^g / \Lambda$ is an abelian variety, a Jacobian, etc. When $g = 1$, this is always true because $\C/\Lambda$ is always an elliptic curve by the Weierstrass $\wp$-function (and is its own Jacobian). When $g > 1$, of course complex tori are not always algebraic.

Let $C$ be a curve of genus $g$. Then the Jacobian of $C$ is 
\[ \on{Jac}(C) = H^1(C, \mc{O}_C) / H^1(C, \Z) \simeq H^0(C, \Omega^1_C)^* / H_1(C, \Z). \]
If $\delta_1, \ldots, \delta_{2g}$ is an integral homology basis, we can send each $\delta_i$ to the map $\int_{\delta_i} -$. Now let $\omega_1, \ldots, \omega_g$ be a basis of $H^0(C, \Omega^1_C)$. Then $H^1(C, \Z)$ embeds as a period matrix
\[ \mqty(\int_{\delta_1} \omega_1 & \cdots & \int_{\delta_{2g}} \omega_1 \\
& \vdots & \\
\int_{\delta_1} \omega_g & \cdots & \int_{\delta_{2g}} \omega_g). \]
Now normalize $\omega_i$ so that that the period matrix has the form $\mqty(I_n & \Omega)$. We want conditions on $\Omega$ such such that our complex torus is the Jacobian of a curve. Now on
\[ H^1_{\mr{dR}}(C) \otimes \C = H^0(C, \Omega^1) \oplus \ol{H^0(C, \Omega^1)}, \]
consider the alternating bilinar form
\[ Q(\omega_1, \omega_2) = \int_C \omega_1 \wedge \omega_2. \]
Given two $1$-forms $\omega, \eta$ with period vectors $u = (u_1, \ldots, u_{2g})$ and $v = (v_1, \ldots, v_{2g})$, we can compute
\[ \int_C \omega \wedge \eta = u^T \mqty(0 & I_n \\ -I_n & 0) v. \]
The proof is not completely trivial, but we can take the fundamental polygon of the curve and then use Stokes to write the integral as a sum of products of periods. If $\omega_1, \ldots, \omega_g$ is a basis for $H^0(C, \Omega^1_C)$, then $\omega_i \wedge \omega_j = 0$ because $h^{2,0} = 0$. If we write $P = \mqty(I_n & \Omega)$, we see that
\[ P \mqty(0 & I_n \\ -I_n & 0) P^T = 0. \]
This implies that $\Omega = \Omega^T$. We also see that $i Q(\omega, \ol{\omega}) > 0$. Therefore the imaginary part of $\Omega$ is positive-definite. The two conditions we have obtained are called the \textit{Riemann bilinear relations}. These are the conditions for having a principal polarization, but of course not all principally polarized abelian varieties are Jacobians.

Given a $g \times 2g$ matrix $B$, a \textit{polarization} is a skew-symmetric form $E$ with $BEB^T = 0$ and $BEB^* > 0$.By linear algebra, there exists a diagonal matrix $D$ such that
\[ E = \mqty(0 & D \\ -D & 0). \]
If $D = I_n$, then this is a \textit{principal polarization}.

\begin{rmk}
    A polarization on $H^1(A, \C)$ for $A$ a complex torus gives us an element of $H^2(A, \Z)$. By the Lefschetz $(1,1)$-theorem, this is $c_1(L)$ for some line bundle $L$. By the positive-definite property of the polarization, $L$ is ample.
\end{rmk}

\section{Lefschetz decomposition}%
\label{sec:lefschetz_decomposition}

Recall that the Hodge decomposition is
\[ H^n(X, \Z) \otimes \C = \bigoplus_{p+q = n} H^{p,q}(X). \]
By Dolbeault, we know that $H^{p,q} (X) = H^q(X, \Omega^p)$. Also, $H^{p,q} = \ol{H^{q,p}}$. If we consider a partition of unity (considering $X$ as a real manifold), we obtain a Hermitian metric $h$ on any compact complex manifold $X$. Then we know $h = g-i\omega$, where $\omega$ is a nondegenerate $(1,1)$-form. We know $X$ is K\"ahler if and only if $\omega$ is closed.

We know that $g$ induces an $L^2$-metric on forms, so we obtain the Hodge star operator
\[ * \colon \Omega^k \to \Omega^{2n-k} \qquad \ev{\alpha, \beta}_{L^2} = \int_X \alpha \wedge *\beta. \]
Now assume that $X$ is compact K\"ahler.

\begin{defn}
    Define the operator $L \colon H^k(X, \R) \to H^{k+2}(X, \R)$ by $L(\alpha) = \alpha \wedge [\omega]$.
\end{defn}

\begin{thm}[Hard Lefschetz]
    For $k \leq n$, $L^{n-k} \colon H^k(X, \R) \to H^{2n-k}(X, \R)$ is an isomorphism.
\end{thm}

\begin{defn}
    The \textit{primitive cohomology} of $X$ is defined to be
    \[ H^{n=r}_{\mr{pr}}(X, \R) = \ker(L^{r+1} \colon H^{n-r}(X) \to H^{n+r+2}(X)). \]
\end{defn}

This terminology is justified by the fact that
\[ H^{n-r}(X, \R) = H^{n-r}_{\mr{pr}}(X, \R) \oplus L(H^{n-r-2}(X)). \]
This implies that
\[ H^k(X) = \bigoplus_{k-2r \leq \min(n, 2n-k)} L^r(H^{k-2r}_{\mr{pr}}(X, \R)). \]
\begin{rmk}
    This decomposition works also for complex coefficients because $L$ has degree $(1,1)$ and is thus compatible with the Hodge decomposition.
\end{rmk}

\section{Polarizations}%
\label{sec:polarizations}

Define an integral form on $H^k(X, \R)$ for $k \leq n$ by
\begin{align*}
    Q(\alpha, \beta) &= (-1)^{\frac{k(k-1)}{2}} \ev{L^{n-k} \alpha, \beta} \\
    &= (-1)^{\frac{k(k-1)}{2}} \int_X \omega^{n-k} \wedge \alpha \wedge \beta.
\end{align*}
$Q$ takes real values, and if $[\omega]$ is integral, it takes integral values on $H^k(X, \Z)$. Also, $Q$ is nondegenerate, bilinear, and $(-1)^n$-symmetric. Also, $Q(H^{p,q}, H^{p',q'}) = 0$ unless $(p,q) = (q',p')$. Extending to $H^k(X, \C)$, we have $i^{p-q}Q(\alpha, \alpha) > 0$ for nonzero $\alpha \in H^{p,q}_{\mr{pr}}(X)$. We know $[\omega]$ is integral if and only if $X$ is a smooth projective variety over $\C$ by the Kodaira embedding theorem.

\begin{defn}
    An \textit{integral polarized Hodge structure of weight $k$} is given by a Hodge structure $(V_{\Z}, F^p V_{\C})$ of weight $k$ and an integer-valued $(-1)^k$-symmetric bilinear form $Q$ on $V_{\Z}$ such that
    \begin{enumerate}
        \item $Q(H^{p,q}, H^{p',q'}) = 0$ unless $(p,q) = (q',p')$;
        \item $i^{p-q} Q(\alpha, \ol{\alpha}) > 0$ for all nonzero $\alpha$ of type $(p,q)$.
    \end{enumerate}
\end{defn}
These are called the \textit{Hodge-Riemann} bilinear relations. To see why they are true, consider the following:
\begin{enumerate}
    \item Integrating over $X$ is the same as taking the cap product with $[X]$, and of course $[X]$ has type $(n,n)$. Because $L$ has type $(1,1)$, we need $p + p' = q+q'$.
    \item The Hodge star is given by
        \[ * \omega = (-1)^{k(k+1)}{2} i^{p-q} \frac{L^{n-k} \omega}{(n-k)!}. \]
        This implies that
        \begin{align*}
            i^{p-q} Q(\alpha, \ol{\alpha}) &= i^{p-q} (-1)^{\frac{k(k-1)}{2}} \int_X \alpha \wedge L^{n-k} \ol{\alpha} \\
            &= (n-k)! \int_X \alpha \wedge * \alpha \\
            &> 0.
        \end{align*}
\end{enumerate}

\section{Unpolarized period domains}%
\label{sec:unpolarized_period_domains}

Let $\pi \colon X \to B$ be a smooth proper map or projective varieties (or a proper holomorphic submersion).
\begin{thm}[Ehresmann]
    In this case, $\pi$ is a locally trivial fibration as smooth manifolds. Thus $X \cong X_0 \times B$ locally, and the fibers of $X \to X_0$ are complex submanifolds. Thus the family of complex structures varies holomorphically.
\end{thm}

Consider the sheaf $R^k \pi_* \C$ on $B$. This is a local system because in the diagram,
\begin{equation*}
\begin{tikzcd}
    X_0 \ar{r} \ar{d} & X \times_B U \ar{d} \ar{r} & X \ar{d}{\pi} \\
    0 \ar[hookrightarrow]{r} & U \ar[hookrightarrow]{r} & B
\end{tikzcd}
\end{equation*}
$X \times_B U \cong X_0 \times U$, so $H^k(X_0, \C) \simeq H^k(X_b, \C)$ for all $b \in U$. If $X_0$ is K\"ahler, we may assume $U = B$ and then $H^k(X_0, \C) \simeq H^k(X_b, \C)$ and the Hodge numbers are the same. However, the Hodge filtration is not the same. If $A^k$ is the $k$-differential forms, recall that
\[ F^p A^k(X) = \bigoplus_{r \geq p} A^{r, k-r}. \]
Then the Hodge filtration is given by
\[ F^p H^k(X, \C) = \frac{\ker(\dd \colon F^p A^k(X, \C) \to F^p A^{k+1}(X, \C))}{\Im(\dd \colon F^p A^{k-1}(X, \C) \to F^p A^k(X, \C))}. \]
If we write $b^{p,k} = \dim F^p H^k(X_b, \C)$, the \textit{period map}
\[ \wp^{p,k} \colon B \to \mr{Gr}(b^{p,k}, H^k(X_0, \C)) \]
is defined by $b \mapsto F^p H^k(X_b, \C) \subset H^k(X_b, \C) \simeq H^k(X_0, \C)$. By a theorem of Griffiths, this map is holomorphic. If we take all $p$ for a fixed $k$, the period mappings send $b$ to a complete flag of $H^k(X, \C)$. Because $X_b$ is K\"ahler, Hodge symmetry implies that
\[ F^p H^k(X_b, \C) \oplus \ol{F^{k-p+1} H^k(X_b, \C)} = H^k(X_b, \C). \]
The open set $D \subset \mr{Fl}(H^k(X, \C))$ satisfying this condition is the \textit{period domain}.

\section{Polarized period domains}%
\label{sec:polarized_period_domains}

Consider $X \to B$. Then a \textit{polarized Hodge structure} on $H^k_{\mr{pr}}(X_b)$ is given by a bilinear form $Q$ such that
\begin{enumerate}
    \item $F^p H^k(X_b, \C) = F^{k-p+1} H^k(X_b, \C)^{\perp}$;
    \item $H^k(X_b, \C) = F^p H^k(X_b, \C) \oplus \ol{F^{k-p+1} H^k(X_b, \C)}$;
    \item $(-1)^{\frac{k(k-1)}{2}} i^{p-q} Q(\alpha, \ol{\alpha})$ for nonzero $\alpha \in F^p_{\mr{pr}} H^k(X_b) \cap \ol{F^q H^k_{\mr{pr}}(X_b)}$.
\end{enumerate}
Flags satisfying the first condition are the domain $\check{D} \subset \mr{Fl}(H^k_{\mr{pr}}(X_b))$ called the compact dual. The flags satisfying all three conditions form an open subset $D \subset \check{D}$.

\begin{defn}
    Define $G(R) = \Aut(H_R, Q_R)$, where $R = \Z, \Q, \R, \C$.
\end{defn}

\begin{defn}
    A \textit{variation of Hodge structures} $(V_{\Z}, F^{\bullet})$ of weight $k$ over $B$ consists of a local system $V_{\Z}$ of finite rank and a finite filtration of $V_{\C} \otimes \mc{O}_B$ by holomorphic subbundles $F^p$ such that
    \begin{itemize}
        \item The filtration $F_b^{\bullet}$ of $(V_{\C})_b$ is a Hodge filtration;
        \item (Griffiths transversality) If $\nabla$ is the Gauss-Manin connection of $V_{\C} \otimes \mc{O}_B$, then $\nabla(F^p) \subseteq F^{p-1}$.
    \end{itemize}
\end{defn}

We now have a \textit{global period map} $\varphi \colon B \to G(\Z) \backslash D$. In the case of $X \to B$, we have $b \mapsto [H^k_{\mr{pr}}(X_b)]$. If $\Gamma$ contains the monodromy representation on $H^k_{\mr{pr}}(X_b)$, then $\varphi$ can be lifted to $\Gamma \backslash D$.

\begin{defn}
    Let $H$ be a Hodge structure. Then the \textit{special Mumford-Tate group} $MT_H$ is the algebraic $\Q$-subgroup of $\End(H_{\Q})$ such that for any tensor power $H' = H^{\otimes k} \otimes (H^{\vee})^{\otimes \ell}$, the rational Hodge classes of $H'$ are precisely the rational vectors fixed by $MT_H$.
\end{defn}

\begin{defn}
    The \textit{Deligne torus} $\mathbb{S}$ is the group $\C^{\times}$ considered as a subgroup of $GL_2(\R)$.
\end{defn}

Now let $z$ act on $H^{p,q}$ by $z^p \ol{z}^q$. This is a real representation, so $\R^{\times} \subset \mathbb{S}$ acts on $H_{\R}$. We also have $S^1 \subset \C^{\times}$, and these give different definitions of the Mumford-Tate group.

\begin{defn}
    The Mumford-Tate group is the smallest algebraic group over $\Q$ whose real points contain the image of $S^1$.
\end{defn}

\begin{defn}
    The \textit{big Mumford-Tate group} is the same thing but with $R^{\times}$ in place of $S^1$.
\end{defn}

\begin{exm}
    Consider $\Q(n) = (2 \pi i)^n \subset \C$ of type $(-n, -n)$. Then the big Mumford-Tate group is $\mathbb{G}_m$ if $n \neq 0$ and $\qty{1}$ if $n = 0$, and the special Mumford-Tate group is $\qty{1}$.
\end{exm}

\begin{exm}
    Consider $X = E_{\tau}$ and write $V = H^1(E_{\tau}, \Q)$. Then the special Mumford-Tate group is $\Q(\tau)$ if $E_{\tau}$ has complex multiplication and $SL_2(\Q)$ otherwise.
\end{exm}

\begin{defn}
    Let $D$ be the polarized period domain and suppose $x \in D$.
    \begin{enumerate}
        \item The orbit $MT_x(\R) x$ is called a \textit{Mumford-Tate subdomain}.
        \item If $M$ is a normal subgroup of $MT_x$, then $M(\R) x$ is called a \textit{weak Mumford-Tate subdomain}.
        \item Let $\pi \colon D \to \Gamma \backslash D$ be a quotient map. For $D' \subset D$ a weak Mumford-Tate subdomain, then $\pi(D') \subset \Gamma \backslash D$ is a complex analytic subvariety, called a \textit{weak Mumford-Tate subvariety}. Also, given a period map $\varphi \colon X \to \Gamma \backslash D$, we call $\varphi^{-1} \pi(D')$ also a weak Mumford-Tate subvariety.
    \end{enumerate}
\end{defn}

Now let $X/\C$ be smooth. Write $MT_{H_{\Z}} = MT_X$. Let $\Gamma$ be the image of the monodromy $\pi_1(X) \to MT_{H_{\Z}}(\Q)$. Let $G$ be the identity component of the $\Q$-Zariski closure of $\Gamma$ and $D = D(G)$ be the associated weak Mumford-Tate subdomain, $\varphi \colon X \to \Gamma \backslash D$ is the period map and $\check{D}$ be the compact dual of $D$. Now consider
\begin{equation*}
\begin{tikzcd}
    X_D \ar{r} \ar{d} & D \ar{d} \\
    X \ar{r} & \Gamma \backslash D.
\end{tikzcd}
\end{equation*}
\begin{thm}[Ax-Schanuel]
    Let $V \subset X \times \check{D}$ be an algebraic subvariety and $U$ be an irreducible analytic component of $U \cap X_D$ such that 
    \[ \on{codim}_{X \times \check{D}}(U) < \on{codim}_{X \times \check{D}} (V) + \on{codim}_{X \times \check{D}} (X_D). \] 
    Then the projection of $U \to X$ is contained in a proper weak Mumford-Tate subvariety.
\end{thm}

\section{$o$-minimality}%
\label{sec:_o_minimality}

\begin{defn}
    A \textit{structure} $S$ is a collection of subsets $\qty{S_n}_{n \in \mathbb{N}}$ of $\R^n$ such that
    \begin{enumerate}
        \item $S_n$ is closed under finite intersections, unions, and complements.
        \item The collection $(S_n)$ is closed under finite cartesian products and projection.
        \item For every polynomial $P \in \R[x_1, \ldots, x_n]$, the set of roots of $P$ is an element of $S_n$.
    \end{enumerate}
\end{defn}
Note that if $P(x_1, \ldots, x_n)$ is a polynomial, the vanishing sets of $P(x_1, \ldots, x_n) - x_{n+1}^2$ are in $S_{n+1}$, and now projecting down, we see that $\qty{P(x_1, \ldots, x_n) \geq 0} \in S_n$. We call the elements $U \in S_n$ the \textit{$S$-definable} subsets of $\R^n$. For $U \in S_n, V \in S_m$, a map $f \colon U \to V$ is \textit{definable} if its graph is.

\begin{exm}
    Let $\R_{\mr{alg}}$ be the collection of real semialgebraic subsets of $\R^n$ (the Boolean algebra generated by $\qty{P \geq 0}$).
\end{exm}

\begin{defn}
    A structure $S$ is \textit{$o$-minimal} if $S_1 = (\R_{\mr{alg}})_1$ is the set of finite unions of intervals.
\end{defn}

\begin{exm}
    The $o$-minimal structure $\R_{\mr{exp}}$ is the structure generated by the graph of $\exp \colon \R \to \R$.
\end{exm}

\begin{exm}
    The structure $\R_{\sin}$ is not $o$-minimal.
\end{exm}

\begin{exm}
    The structure $\R_{\mr{an}}$ generated by graphs of restrictions of real analytic functions to open balls $B(R)$ is $o$-minimal.
\end{exm}

\begin{exm}
    The structure $\R_{\mr{an}, \exp}$ generated by $\R_{\mr{an}}, \R_{\exp}$ is $o$-minimal.
\end{exm}

\begin{defn}
    An \textit{$S$-definable topological space} $M$ i a topological space $M$ with a finite covering $\qty{V_i}$ of $M$ and homeomorphisms $\varphi_i \colon V-i \to U_i \subset \R^n$ such that
    \begin{enumerate}
        \item All $U_i$ and all intersections $U_{ij}$ are $S$-definable.
        \item The transition functions $\varphi_{ij} \colon U_{ij} \to U_{ji}$ are $S$-definable.
    \end{enumerate}
\end{defn}

We can also define $S$-definable continuous maps, but it will not be helpful to write down this definition.

\begin{exm}
    Consider $\C^* \subset \C$ and let $e \colon \C \to \C^*$ be the covering map. Then we want to put an $\R_{\mr{alg}}$-definable structure on $\C^*$. We can do it in more than one way:
    \begin{enumerate}
        \item Simply take $\C^* \subset \C = \R^2$ and induce the structure from $\C^*$.
        \item Define $F_a = \qty{z \in \C \mid a \cdot \Im z \subset \Re z \subset (1+\ep) + a \Im z}$. Under the map $e$, this covers $\C^*$. Taking thinner strips, we obtain an open covering of $\C^*$, and we obtain a different structure for each $a$.
    \end{enumerate}
\end{exm}

\begin{exm}
    Let $X$ be a real variety. By covering $X$ with affines, the induced $\R_{\mr{alg}}$-structures on affines give us an $\R_{\mr{alg}}$-structure on $X$.
\end{exm}

\begin{defn}
    Let $X$ be a complex variety. Let $X^{\mr{def}}$ be the $S$-definable topological space with underlying set $X(\C)$ and then use the previous example.
\end{defn}

We will now discuss applications of $o$-minimality. The first is a counting theorem of Pila-Wilkie. We will begin by recalling the elementary theory of heights that we should have learned as undergrads (but didn't).

\begin{defn}
    For $r \in \Q$, define $H(r) = \max\qty{\abs{a}, \abs{b}}$ where $r = \frac{a}{b}$ is reduced. Then for $\alpha \in \Q^n$, write $H(\alpha) = \max H(\alpha_i)$.
\end{defn}

Note that there are only finitely many points of $\Q^n$ with bounded height. Now let $U \subset \R^n$ be a subset. Define $N(U, t) = \# \qty{\alpha \in U \cap \Q^n \mid H(\alpha) \leq t}$. Then let 
\[ U^{\mr{alg}} = \bigcup_{Z \text{connected semialgebraic}} Z \]
and $U^{\mr{tr}} = U \setminus U^{\mr{alg}}$.

\begin{exm}
    Consider $Z = e^x y$. Then $U^{\mr{alg}}$ is given by $y = z = 0$.
\end{exm}

\begin{thm}
    Let $U \subset \R^n$ be definable for some $o$-minimal structure. Then for any $\ep > 0$, $N(U^{\mr{tr}}, t) = O(t^{\ep})$.
\end{thm}

This result shows that most $\Q$-rational points are algebraic. The next application is the Definable Chow theorem of Peterzil-Starchenko. Recall Chow's theorem, which says that if $X$ is a proper complex variety and $Y \subset X^{\mr{an}}$ is a closed complex analytic subvariety, then $Y$ is an algebraic subvariety.

\begin{thm}[Definable Chow]
    Fix an $o$-minimal structure and let $X$ be any complex variety. Let $Y$ be a closed complex analytic subvariety whose underlying set is definable. Then $Y$ is algebraic.
\end{thm}

\end{document}
