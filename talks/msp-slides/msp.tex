% !TEX program = xelatex
% !TEX encoding = UTF-8 Unicode
\documentclass[10pt]{beamer}
% Configuration options specific to Metropolis
\usetheme{metropolis}

% PACKAGES AND THEMES
\usepackage[utf8]{inputenc}
\usepackage{amsmath, amssymb, xcolor}
\usepackage{amsthm}  % Math support and colors
%\usepackage{metropolis} % Modern academic theme

\usepackage{bm}
\usepackage{accents}
\usepackage{mathtools}
\usepackage{tikz}
\usetikzlibrary{calc}
\usetikzlibrary{decorations.pathmorphing,shapes}
\usetikzlibrary{automata,positioning}
\usepackage{tikz-cd}
\usepackage{forest}
\usepackage{braket} 
\usepackage{listings}
\usepackage{mdframed}
\usepackage{verbatim}
\usepackage{physics2}
\usephysicsmodule{ab,ab.legacy,diagmat,xmat,op.legacy}
\usepackage{derivative}
\usepackage{fixdif}
\usepackage{stmaryrd}
% \usepackage{euscript} 
\usepackage{eucal}
\usepackage{stackengine} 
\usepackage{float}
\usepackage{booktabs}
\usepackage{microtype}
\usepackage[mathrm=sym]{unicode-math}
\setmathfont{Fira Math}

\newcommand{\Ima}{\mathrm{Im}}
\newcommand{\A}{\mathbb{A}}
\newcommand{\G}{\mathbb{G}}
\newcommand{\N}{\mathbb{N}}
\newcommand{\R}{\mathbb{R}}
\newcommand{\C}{\mathbb{C}}
\newcommand{\Z}{\mathbb{Z}}
\newcommand{\Q}{\mathbb{Q}}
\newcommand{\F}{\mathbb{F}}
\newcommand{\W}{\mathcal{W}}
\newcommand{\bd}{\mathbf{d}}
\newcommand{\ba}{\mathbf{a}}
\newcommand{\bb}{\mathbf{b}}
\newcommand{\bA}{\mathbf{A}}
\newcommand{\bB}{\mathbf{B}}
\newcommand{\bx}{\mathbf{x}}
\newcommand{\bt}{\mathbf{t}}
\newcommand{\bp}{\mathbf{p}}
\newcommand{\bO}{\mathbf{0}}
\renewcommand{\k}{\Bbbk}
\renewcommand{\L}{\mathbb{L}}
\renewcommand{\P}{\mathbb{P}}
\newcommand{\E}{\mathbb{E}}
\newcommand{\M}{\mathcal{M}}
\newcommand{\Mbar}{\overline{\mathcal{M}}}
\newcommand{\g}{\mathfrak{g}}
\newcommand{\h}{\mathfrak{h}}
\newcommand{\n}{\mathfrak{n}}
\newcommand{\fr}{\mathfrak{r}}
\renewcommand{\b}{\mathfrak{b}}
\newcommand{\ep}{\varepsilon}
\newcommand*{\dt}[1]{%
   \accentset{\mbox{\Huge\bfseries .}}{#1}}
%\renewcommand{\abstractname}{Official Description}
\newcommand{\mc}[1]{\mathcal{#1}}
\newcommand{\msc}[1]{\mathscr{#1}}
\newcommand{\T}{\mathbf{T}}
\newcommand{\mf}[1]{\mathfrak{#1}}
\newcommand{\mbf}[1]{\mathbf{#1}}
\newcommand{\mr}[1]{\mathrm{#1}}
\newcommand{\on}[1]{\operatorname{#1}}
\newcommand{\ms}[1]{\mathsf{#1}}
\newcommand{\mt}[1]{\mathtt{#1}}
\newcommand{\ol}[1]{\overline{#1}}
\newcommand{\ul}[1]{\underline{#1}}
\newcommand{\wt}[1]{\widetilde{#1}}
\newcommand{\wh}[1]{\widehat{#1}}
\renewcommand{\div}{\operatorname{div}}
\newcommand{\1}{\mathbf{1}}
\newcommand{\2}{\mathbf{2}}
\newcommand{\3}{\mathbf{3}}
\newcommand{\I}{\mathrm{I}}
\newcommand{\II}{\mr{I}\hspace{-1.3pt}\mr{I}}
\newcommand{\III}{\mr{I}\hspace{-1.3pt}\mr{I}\hspace{-1.3pt}\mr{I}}
\renewcommand{\v}{\mbf{v}}
\newcommand{\w}{\mbf{w}}
\newcommand{\bmu}{\bm{\mu}}
\newcommand{\pre}{\mr{pre}}
\newcommand{\vir}{\ms{vir}}
\newcommand{\fl}{\mr{fl}}
\newcommand{\pt}{\mr{pt}}
\newcommand{\loc}{\mr{loc}}
\newcommand{\sloop}{\ms{loop}}
\newcommand{\ssmall}{\ms{small}}
\newcommand{\extra}{\ms{extra}}
\newcommand{\tw}{\mr{tw}}
\renewcommand{\top}{\mr{top}}
\newcommand{\ps}[1]{\llbracket #1 \rrbracket}

\DeclareMathOperator{\Der}{Der}
\DeclareMathOperator{\Tor}{Tor}
\DeclareMathOperator{\Hom}{Hom}
\DeclareMathOperator{\End}{End}
\DeclareMathOperator{\Ext}{Ext}
\DeclareMathOperator{\ad}{ad}
\DeclareMathOperator{\Aut}{Aut}
\DeclareMathOperator{\Rad}{Rad}
\DeclareMathOperator{\Pic}{Pic}
\DeclareMathOperator{\NS}{NS}
\DeclareMathOperator{\supp}{supp}
\DeclareMathOperator{\Supp}{Supp}
\DeclareMathOperator{\depth}{depth}
\DeclareMathOperator{\sgn}{sgn}
\DeclareMathOperator{\spec}{Spec}
\DeclareMathOperator{\Spec}{Spec}
\DeclareMathOperator{\proj}{Proj}
\DeclareMathOperator{\Proj}{Proj}
\DeclareMathOperator{\ord}{ord}
\DeclareMathOperator{\Div}{Div}
\DeclareMathOperator{\Bl}{Bl}
\DeclareMathOperator{\coker}{coker}
\DeclareMathOperator{\ev}{ev}
\DeclareMathOperator{\PF}{\msc{PF}}
\DeclareMathOperator{\st}{st}
\DeclareMathOperator{\cl}{cl}
\DeclareMathOperator{\Cont}{Cont}
\newtheorem{thm}{Theorem}[section]
\newtheorem{cor}[thm]{Corollary}
\newtheorem{prop}[thm]{Proposition}
\newtheorem{lem}[thm]{Lemma}
\newtheorem{conj}[thm]{Conjecture}
\newtheorem{quest}[thm]{Question}
\newtheorem{claim}[thm]{Claim}

\theoremstyle{definition}
\newtheorem{defn}[thm]{Definition}
\newtheorem{defns}[thm]{Definitions}
\newtheorem{con}[thm]{Construction}
\newtheorem{exm}[thm]{Example}
\newtheorem{exms}[thm]{Examples}
\newtheorem{notn}[thm]{Notation}
\newtheorem{notns}[thm]{Notations}
\newtheorem{addm}[thm]{Addendum}
\newtheorem{exer}[thm]{Exercise}

\theoremstyle{remark}
\newtheorem{rmk}[thm]{Remark}
\newtheorem{rmks}[thm]{Remarks}
\newtheorem{warn}[thm]{Warning}
\newtheorem{sch}[thm]{Scholium}


% unnumbered theorems
\theoremstyle{plain}
\newtheorem*{thm*}{Theorem}
\newtheorem*{prop*}{Proposition}
\newtheorem*{lem*}{Lemma}
\newtheorem*{cor*}{Corollary}
\newtheorem*{conj*}{Conjecture}

% unnumbered definitions
\theoremstyle{definition}
\newtheorem*{defn*}{Definition}
\newtheorem*{exer*}{Exercise}
\newtheorem*{defns*}{Definitions}
\newtheorem*{con*}{Construction}
\newtheorem*{exm*}{Example}
\newtheorem*{exms*}{Examples}
\newtheorem*{notn*}{Notation}
\newtheorem*{notns*}{Notations}
\newtheorem*{addm*}{Addendum}


\theoremstyle{remark}
\newtheorem*{rmk*}{Remark}



% TITLE SLIDE
\title{Higher-genus GW theory of smooth CY hypersurfaces in weighted $\P^4$}
\author{Patrick Lei}
\institute{Columbia University}
\date{March 30, 2025}

% PRESENTATION STRUCTURE
\begin{document}

\frame[plain]{\titlepage} % Title slide

\begin{frame}{Overview}
  \tableofcontents
\end{frame}

\section{Introduction}
\begin{frame}{Overview}
  \onslide<1->{Our goal is to understand the \textit{Gromov-Witten theory} of a \textbf{Calabi-Yau threefold}, which in this talk will be called $Z$.}
  
  \onslide<2->{\begin{defn}
    The moduli space $\mathcolor{cyan}{\Mbar_{g,0}(Z, d)}$ of genus $g$, degree $d$ unmarked stable maps to $Z$ has a virtual cycle in degree $0$. Define the \textit{GW invariant}
    \[ N_{g,d} \coloneqq \deg [\Mbar_{g,0}(Z,d)]^{\vir} \in \Q. \]
  \end{defn}}
  \onslide<3->{\textbf{Goal:}} \only<3-4>{Calculate all $N_{g,d}$.}\only<4>{ \textcolor{red}{This is too hard!}}\only<5>{Say something about the behavior of the generating series of all $N_{g,d}$ when we fix the genus.}
\end{frame}

\beamerdefaultoverlayspecification{<only@+>}
\AtBeginEnvironment{enumerate}{\onslide<.->}
\begin{frame}{Our results}
  \onslide<1->{We are able to prove the following results for the threefolds $\mathcolor{blue}{Z_6 \subset \P(1,1,1,1,2)}$, $Z_8 \subset \P(1,1,1,1,4)$, and $Z_{10} \subset \P(1,1,1,2,5)$:}
  \onslide<+>
  \begin{enumerate}
    \item If $F_g = \sum_d N_{g,d} Q^d$ is the generating series of genus $g$ invariants, we prove that a normalized version $P_g$ is a polynomial in five generators $\mathcolor{magenta}{A_1}, \mathcolor{purple}{B=B_1, B_2, B_3}, X$ defined using only genus-zero data.
    \item More precisely, Coates-Corti-Lee-Tseng proved that the genus-0 GW theory of $Z$ is controlled by the series
    \begin{align*}
      I(q, z) &\coloneqq z\sum_{d \geq 0} q^d \frac{\prod_{m=1}^{6d}(6H+mz)}{\prod_{m=1}^d (H+mz)^4 \prod_{m=1}^{2d}(2H+mz)} \\
      &= \mathcolor{purple}{I_0 (q)} z + \mathcolor{magenta}{I_1(q)} H + I_2(q) \frac{H^2}{z} + I_3(q) \frac{H^3}{z^2}.
    \end{align*}
    \item Define $\mathcolor{magenta}{I_{11}(q)} \coloneqq D \ab(\frac{I_1(q)}{I_0(q)})$, where $D = q \odv[style-inf=\mathsf{d}]{}{q}$. Then define
    \[ \mathcolor{magenta}{A} \coloneqq \frac{D I_{11}}{I_{11}}, \qquad \mathcolor{purple}{B_k} \coloneqq \frac{D^k I_0}{I_0}, \qquad X \coloneqq 1- \frac{1}{1-\frac{6^6}{2^2}q}. \]
    Also define $Y \coloneqq 1-X$.
    \item The first result, conjectured by Yamaguchi-Yau (2004), is that
    \[ P_{g,n} = \frac{(3Y)^{g-1} I_{11}^n}{I_0^{2g-2}} \ab(Q \odv[style-inf=\mathsf{d}]{}{Q})^n F_g(Q) \in \Q[\mathcolor{magenta}{A}, \mathcolor{purple}{B, B_2, B_3}, X] \]
    after the substitution $Q = q e^{I_1/I_0}$.
    \item The second result is the equality
    \[ P_{1,1} = -\frac{1}{2} \mathcolor{magenta}{A} - \frac{21}{2} \mathcolor{purple}{B} - \frac{1}{12} X - \frac{7}{4}. \]
    \item Consider the following Feynman rule (sum over \textit{stable graphs}) defined by BCOV (1993) defining a power series $f_{g,m,n}$. First, introduce \textit{propagators} $E_{\psi} = B$, $E_{\varphi\varphi} = A+2B$, $E_{\varphi\psi} = -B_2$, and $E_{\psi\psi} = -B_3 + (B-X)B_2 - \frac{13}{36} BX$.
    \item Place $\varphi - E_{\psi} \psi$ at the first $m$ vertices and $\psi$ at the last $n$ vertices, place
    \[ E_{\varphi\varphi}\varphi\otimes \varphi + E_{\varphi\psi}(\varphi \otimes \psi + \psi \otimes \varphi) + E_{\psi\psi} \psi \otimes \psi \]
    at each edge, and place
    \[ \varphi^{\otimes m} \otimes \psi^{\otimes n} \mapsto P_{g,m,n} \coloneqq \frac{(2g-2+m+n-1)!}{(2g-2+m-1)!} P_{g,m} \]
    at each leg. Also, set $P_{1,0,1} = -\frac{19}{2}$.
    \item Our result is that the output $f_{g,m,n}$ is a polynomial of degree at most $3g-3+m$ in $X$. This implies the \textit{modular anomaly equations}
    \footnotesize
    \begin{align*}
      - \partial_{A} P_g = \frac{1}{2} \ab(P_{g-1,2} + \sum_{g_1+g_2 = g} P_{g_1, 1} P_{g_2, 2}),  \\
      \ab(-2 \partial_{A} + \partial_{B} + (A+2B) \partial_{B_2} - \ab((B-X)(A+2B)-B_2-\frac{13}{36} X)\partial_{B_3})P_g = 0.
    \end{align*}
  \end{enumerate}
\end{frame}

\begin{frame}{Past work}
  \begin{itemize}[<+->]
    \item Our first result (polynomiality) was proved for the quintic independently by Chang-Guo-Li and Guo-Janda-Ruan in 2018.
    \item Our genus-one formula was first proved for the quintic by Zinger (2007) and proved for complete intersections in projective space by Popa (2010).
    \item The modular anomaly equations were proved for the quintic independently by Chang-Guo-Li and Guo-Janda-Ruan in 2018.
  \end{itemize}
\end{frame}

\section{Mixed-Spin-P fields}
\begin{frame}{Our approach}
  \onslide<1->{We use the approach of \textbf{Mixed-Spin-P fields}, which were introduced by Chang-Li-Li-Liu (2015, 2016) and Chang-Guo-Li-Li (2018).}

  \onslide<2->{The original geometric intuition was to implement the master space idea of Thaddeus, but in order to perform calculations we need to introduce a parameter $N$ (which is a positive integer) to the theory.}
\end{frame}

\begin{frame}{MSP moduli}
  \onslide<1->{Let $(\C^{\times})^3$ act on $\C^{N+7}$ with the following weights:}
  \onslide<2->{
    \[ \ab[\begin{array}{cccccccccc}
        x_1 & x_2 & x_3 & x_4 & x_5 & p & u_1 & \ldots & u_N & v \\
        \midrule
        1 & 1 & 1 & 1 & 2 & -6 & 1 & \ldots & 1 & 0 \\
        0 & 0 & 0 & 0 & 0 & 0 & 1 & \ldots & 1 & 1 \\
        0 & 0 & 0 & 0 & 0 & 1 & 0 & \ldots & 0 & 0
    \end{array}]. \]
  }
  \onslide<3->{Let $\mc{W} \coloneqq \mc{W}_{g,n,\bd}$ be the stack of commutative diagrams
  \begin{equation*}
  \begin{tikzcd}[ampersand replacement=\&]
    C \arrow[r] \arrow[drr,bend right=10,"\omega^{\log}"] \& {[\C^{N+7}/(\C^{\times})^3] }\arrow[r,""] \& B (\C^{\times})^3 \arrow[d,"\ms{pr}_3"] \\
    \& \& B \C^{\times}
  \end{tikzcd}
\end{equation*}
  subject to a stability condition.}
\end{frame}

\begin{frame}{Virtual localization I}
  \onslide<1->{$\mc{W}$ is not proper, but it has a cosection localized virtual cycle $[\mc{W}]^{\vir}$ supported on a proper closed substack $\mc{W}^-$.} \onslide<2->{This virtual cycle is equivariant with respect to the action of $(\C^{\times})^N$ on $\mc{W}$ by scaling $(u_1, \ldots, u_N)$.}
    \begin{thm}[Chang-Kiem-Li, 2015]<3->
      Indexing the fixed loci by $\Gamma$, we have
      \[ [\mc{W}]^{\vir} = \sum_{\Gamma} \frac{[\mc{W}_{\Gamma}]^{\vir}}{e(N_{\Gamma}^{\vir})}. \]
    \end{thm}
\end{frame}

\begin{frame}{Virtual localization II}
  Fixed loci in the MSP moduli space $\mc{W}$ are indexed by graphs $\Gamma$. There are three kinds (denoted by levels) of vertices:
  \begin{enumerate}[<+->]
    \item Level $0$ vertices contribute GW invariants of $Z$;
    \item Level $1$ vertices contribute Hodge integrals;
    \item Level $\infty$ vertices contribute FJRW invariants.
  \end{enumerate}
  \begin{thm}<+->
    If $\Gamma$ has an edge between a level $0$ and level $\infty$ vertex, then $[\mc{W}_{\Gamma}]^{\vir} = 0$.
  \end{thm}
  \begin{defn}<+->
    Define the MSP $[0,1]$ theory by only considering the contributions of graphs without level $\infty$ vertices.
  \end{defn}
\end{frame}


\section{Calculations}%
\label{sec:Calculations}

\begin{frame}{Genus zero MSP theory}
  \begin{lem}<+->
    Genus zero MSP invariants equal genus zero Gromov-Witten invariants of a degree $6$ hypersurface in $\P(1,1,1,1,2,1,\ldots,1)$.
  \end{lem}
  \onslide<+->{This implies that all genus zero MSP invariants are recovered from the $J$-function
  \[z\sum_{d \geq 0} q^d \frac{\prod_{m=1}^{6d}(6p+mz)}{\prod_{m=1}^d (p+mz)^4 \prod_{m=1}^{2d}(2p+mz) \prod_{m=1}^d ((p+mz)^N - t^N)}. \]
  Here, we specialize the equivariant parameters to $t_{\alpha} = \zeta_N^{\alpha} t$ for $\alpha = 1, \ldots,N$.}
\end{frame}

\begin{frame}{MSP $R$-matrix}
  \onslide<+->{Define the $R$-matrix by the Birkhoff factorization
  \[ S^{\ms{MSP}}(z) \Delta = R(z) S^{\ms{loc}}(z), \]
  where $\Delta$ comes from Quantum Riemann-Roch and $S^{\ms{loc}}$ is simply the direct sum of the $S$-matrix of $Z$ and $N$ copies of the $S$-matrix of a point.}
  \begin{thm}<+->
    The MSP $[0,1]$ theory is a \textit{cohomological field theory} given by the action
    \[ R(z) . \ab(\Omega^Z \oplus \bigoplus_{\alpha=1}^N \Omega^{\ms{pt}_{\alpha}}) \]
    of $R(z)$ on the direct sum of the GW theory of $Z$ and $N$ copies of the GW theory of a point.
  \end{thm}
\end{frame}

\begin{frame}{Proof of polynomiality}
  \begin{lem}<+->
    All entries of $R(z)$ lie in $\Q[\mathcolor{magenta}{A}, \mathcolor{purple}{B, B_2, B_3}, X]$ (possibly after normalization). There are also explicit degree bounds for the entries coming from the $N$ points, which are polynomials in $X$.
  \end{lem}
  \begin{lem}<+->
    The MSP $[0,1]$ correlator $\ab<p^{a_1}, \ldots, p^{a_n}>_{g,n}^{[0,1]}$ is a polynomial in $q$ of degree at most $g-1 + \frac{3g-3+\sum a_i}{N}$.
  \end{lem}
  \begin{cor}<+->
    \[ P_{g,n} = \frac{(3Y)^{g-1} I_{11}^n}{I_0^{2g-2}} \ab(Q \odv[style-inf=\mathsf{d}]{}{Q})^n F_g(Q) \Bigg|_{Q=qe^{I_1/I_0}} \in \Q[\mathcolor{magenta}{A}, \mathcolor{purple}{B, B_2, B_3}, X]. \]
  \end{cor}
\end{frame}

\begin{frame}{Feynman rule I}
  \only<1>{Edge contributions in the action of the MSP $R$-matrix look like the propagators $E_{\varphi\varphi}$, $E_{\varphi\psi}$, and $E_{\psi\psi}$. For example, the edge contribution between two level $0$ vertices starts with
  \[ E_{\psi}(1 \otimes H^2 + H^2 \otimes 1) + \frac{1}{2} \ab(E_{\varphi\varphi} + \frac{13}{36}X) H \otimes H \]
  up to a prefactor.}

  \begin{defn}<only@2>
    Consider the factorization
    \[ R(z) = R^X(z)\pdiagmat[empty={}]{R^{\bA}(z), I_N},\]
    where
    \[ R^{\bA}(z)^{-1} = I - \begin{pmatrix}
      0 & z E_{\psi} & z^2 E_{\varphi\psi} & \cdots \\
      & 0 & z E_{\varphi\varphi} & \cdots \\
      & & 0 & z E_{\psi} \\
      & & & 0
    \end{pmatrix}. \]
  \end{defn}
\end{frame}

\begin{frame}{Feynman rule II}
  \begin{lem}<+->
    The entries of $R^X(z)$ are polynomials in $X$ with an explicit degree bound.
  \end{lem}
  \begin{cor}[MSP Feynman rule]<+->
    Let $f_{g,m,n}^{\bA}$ be the generating functions of the cohomological field theory
    \[ R^{\bA}(z).\Omega^Z. \]
    Then $f_{g,m,n}^{\bA}$ is a polynomial in $X$ of degree at most $3g-3+m$.
  \end{cor}
\end{frame}

\begin{frame}{Feynman rule III}
  \onslide<1->{Recall that we let $f_{g,m,n}$ be the output of the physics Feynman rule.}
  \begin{thm}<2->
    We have
    \[ f_{g,m,n} = f_{g,m,n}^{\bA} - \delta_{g,1} \delta_{m,0} (n-1)!. \]
  \end{thm}
  \onslide<3->{As a corollary, we prove the physics Feynman rule.}
  \begin{cor}<4->
    We have the \textit{modular anomaly equations}
    \footnotesize
    \begin{align*}
      - \partial_{A} P_g = \frac{1}{2} \ab(P_{g-1,2} + \sum_{g_1+g_2 = g} P_{g_1, 1} P_{g_2, 2}),  \\
      \ab(-2 \partial_{A} + \partial_{B} + (A+2B) \partial_{B_2} - \ab((B-X)(A+2B)-B_2-\frac{13}{36} X)\partial_{B_3})P_g = 0.
    \end{align*}
  \end{cor}
  \begin{proof}<5->
    Differentiate the physics Feynman rule.
  \end{proof}
\end{frame}

\section{Future work}%
\label{sec:Speculation}

\begin{frame}{Multi-parameter models}
  \begin{itemize}[<+->]
    \item We have proved results for Calabi-Yau threefolds with $h^{1,1} = 1$. A natural question is to extend these results to the case when $h^{1,1} > 1$.
    \item A natural direction is to study hypersurfaces in toric Fano orbifolds whose corresponding reflexive polytope is a simplex. This includes weighted projective space examples and the mirror quintic.
    \item Stay tuned in the future!
  \end{itemize}
\end{frame}


\end{document}