\documentclass{amsart}
\usepackage{amsmath}
\usepackage{amssymb}
\usepackage{amsthm}
%\usepackage{MnSymbol}
\usepackage{bm}
\usepackage{accents}
\usepackage{mathtools}
\usepackage{tikz}
\usetikzlibrary{calc}
\usetikzlibrary{decorations.pathmorphing,shapes}
\usetikzlibrary{automata,positioning}
\usepackage{tikz-cd}
\usepackage{forest}
\usepackage{braket} 
\usepackage{listings}
\usepackage{mdframed}
\usepackage{verbatim}
\usepackage{physics}
\usepackage{stmaryrd}
\usepackage{mathrsfs} 
\usepackage{stackengine} 
%\usepackage{/home/patrickl/homework/macaulay2}

%font
\usepackage[sc]{mathpazo}
\usepackage{eulervm}
\usepackage[scaled=0.86]{berasans}
\usepackage{inconsolata}
\usepackage{microtype}

%CS packages
\usepackage{algorithmicx}
\usepackage{algpseudocode}
\usepackage{algorithm}

% typeset and bib
\usepackage[english]{babel} 
\usepackage[utf8]{inputenc} 
\usepackage[T1]{fontenc}
% \usepackage[backend=biber, style=alphabetic]{biblatex}
\usepackage[bookmarks, colorlinks, breaklinks]{hyperref} 
\hypersetup{linkcolor=blue,citecolor=magenta,filecolor=black,urlcolor=blue}
\usepackage{graphicx}
\graphicspath{{./}}

% other formatting packages
\usepackage{float}
\usepackage{booktabs}
\usepackage[shortlabels]{enumitem}
\setitemize{noitemsep}
\usepackage{csquotes}
%\usepackage{titlesec}
%\usepackage{titling}
%\usepackage{fancyhdr}
%\usepackage{lastpage}
\usepackage{parskip}

\usepackage{lipsum}

% delimiters
\DeclarePairedDelimiter{\gen}{\langle}{\rangle}
\DeclarePairedDelimiter{\floor}{\lfloor}{\rfloor}
\DeclarePairedDelimiter{\ceil}{\lceil}{\rceil}


\newtheorem{thm}{Theorem}[section]
\newtheorem{cor}[thm]{Corollary}
\newtheorem{prop}[thm]{Proposition}
\newtheorem{lem}[thm]{Lemma}
\newtheorem{conj}[thm]{Conjecture}
\newtheorem{quest}[thm]{Question}

\theoremstyle{definition}
\newtheorem{defn}[thm]{Definition}
\newtheorem{defns}[thm]{Definitions}
\newtheorem{con}[thm]{Construction}
\newtheorem{exm}[thm]{Example}
\newtheorem{exms}[thm]{Examples}
\newtheorem{notn}[thm]{Notation}
\newtheorem{notns}[thm]{Notations}
\newtheorem{addm}[thm]{Addendum}
\newtheorem{exer}[thm]{Exercise}

\theoremstyle{remark}
\newtheorem{rmk}[thm]{Remark}
\newtheorem{rmks}[thm]{Remarks}
\newtheorem{warn}[thm]{Warning}
\newtheorem{sch}[thm]{Scholium}


% unnumbered theorems
\theoremstyle{plain}
\newtheorem*{thm*}{Theorem}
\newtheorem*{prop*}{Proposition}
\newtheorem*{lem*}{Lemma}
\newtheorem*{cor*}{Corollary}
\newtheorem*{conj*}{Conjecture}

% unnumbered definitions
\theoremstyle{definition}
\newtheorem*{defn*}{Definition}
\newtheorem*{exer*}{Exercise}
\newtheorem*{defns*}{Definitions}
\newtheorem*{con*}{Construction}
\newtheorem*{exm*}{Example}
\newtheorem*{exms*}{Examples}
\newtheorem*{notn*}{Notation}
\newtheorem*{notns*}{Notations}
\newtheorem*{addm*}{Addendum}


\theoremstyle{remark}
\newtheorem*{rmk*}{Remark}

% shortcuts
\newcommand{\Ima}{\mathrm{Im}}
\newcommand{\A}{\mathbb{A}}
\newcommand{\G}{\mathbb{G}}
\newcommand{\N}{\mathbb{N}}
\newcommand{\R}{\mathbb{R}}
\newcommand{\C}{\mathbb{C}}
\newcommand{\Z}{\mathbb{Z}}
\newcommand{\Q}{\mathbb{Q}}
\renewcommand{\k}{\Bbbk}
\renewcommand{\L}{\mathbb{L}}
\renewcommand{\P}{\mathbb{P}}
\newcommand{\M}{\mathcal{M}}
\newcommand{\Mbar}{\overline{\mathcal{M}}}
\newcommand{\g}{\mathfrak{g}}
\newcommand{\h}{\mathfrak{h}}
\newcommand{\n}{\mathfrak{n}}
\renewcommand{\b}{\mathfrak{b}}
\newcommand{\ep}{\varepsilon}
\newcommand*{\dt}[1]{%
   \accentset{\mbox{\Huge\bfseries .}}{#1}}
%\renewcommand{\abstractname}{Official Description}
\newcommand{\mc}[1]{\mathcal{#1}}
% \newcommand{\msc}[1]{\mathscr{#1}}
\newcommand{\T}{\mathbb{T}}
\newcommand{\mf}[1]{\mathfrak{#1}}
\newcommand{\mbf}[1]{\mathbf{#1}}
\newcommand{\mr}[1]{\mathrm{#1}}
\newcommand{\on}[1]{\operatorname{#1}}
\newcommand{\ms}[1]{\mathsf{#1}}
\newcommand{\mt}[1]{\mathtt{#1}}
\newcommand{\ol}[1]{\overline{#1}}
\newcommand{\ul}[1]{\underline{#1}}
\newcommand{\wt}[1]{\widetilde{#1}}
\newcommand{\wh}[1]{\widehat{#1}}
\renewcommand{\div}{\operatorname{div}}
\newcommand{\1}{\mathbf{1}}
\newcommand{\2}{\mathbf{2}}
\newcommand{\3}{\mathbf{3}}
\newcommand{\I}{\mathrm{I}}
\newcommand{\II}{\mr{I}\hspace{-1.3pt}\mr{I}}
\newcommand{\III}{\mr{I}\hspace{-1.3pt}\mr{I}\hspace{-1.3pt}\mr{I}}
\renewcommand{\v}{\mbf{v}}
\newcommand{\w}{\mbf{w}}

\DeclareMathOperator{\Der}{Der}
\DeclareMathOperator{\Tor}{Tor}
\DeclareMathOperator{\Hom}{Hom}
\DeclareMathOperator{\End}{End}
\DeclareMathOperator{\Ext}{Ext}
\DeclareMathOperator{\ad}{ad}
\DeclareMathOperator{\Aut}{Aut}
\DeclareMathOperator{\Rad}{Rad}
\DeclareMathOperator{\Pic}{Pic}
\DeclareMathOperator{\NS}{NS}
\DeclareMathOperator{\supp}{supp}
\DeclareMathOperator{\Supp}{Supp}
\DeclareMathOperator{\depth}{depth}
\DeclareMathOperator{\sgn}{sgn}
\DeclareMathOperator{\spec}{Spec}
\DeclareMathOperator{\Spec}{Spec}
\DeclareMathOperator{\proj}{Proj}
\DeclareMathOperator{\Proj}{Proj}
\DeclareMathOperator{\ord}{ord}
\DeclareMathOperator{\Div}{Div}
\DeclareMathOperator{\Bl}{Bl}
\DeclareMathOperator{\coker}{coker}

\title{Algebraic curves and integrable hierarchies}
\author{Patrick Lei}
\date{March 8, 2023}

\begin{document}
    
\maketitle

\begin{abstract}
    In the 1990s, a remarkable correspondence was discovered between the geometry of algebraic curves and infinite-dimensional systems of differential equations. The correspondence has its origin in the study of two-dimensional quantum field theories and is related to many different areas of mathematics. After introducing the relevant objects, I will then state the first result in this story, which was conjectured by Witten and proved by Kontsevich.
\end{abstract}

\section{Algebraic curves and their moduli}

For the purposes of this lecture, we will only define smooth curves. Also, space will really mean algebraic variety/scheme/stack, whichever is needed to make the statements of the results true.

\begin{defn}
    A \textit{smooth curve} is a compact $1$-dimensional complex manifold; i.e. a Riemann surface.
\end{defn}

Note that curves will be $1$-dimensional over $\C$ and $2$-dimensional over $\R$. We will also need our curves to develop singularities, but fortunately we only need one kind of singularity:

\begin{defn}
    A \textit{nodal singularity} is one that locally looks like $\qty{xy = 0} \subseteq \C^2$.
\end{defn}

\subsection{Moduli spaces}

As the simplest algebraic varieties, much has been written on the geometry of curves. For this lecture, we will focus on the study of familes of curves -- a very productive technique in algebraic geometry is to study how properties vary in families. We will begin with a very natural question.

\begin{quest}
    Can we classify all algebraic curves?
\end{quest}

If we first restrict to smooth curves, we know that these are topologically surfaces, and the topological classification of surfaces is very simple: they are controlled by a discrete parameter, the genus. However, for a surface of genus $g \geq 1$, there will be many ways to give it the structure of a complex manifold or algebraic variety.

\begin{exm}
    For any $\lambda \neq 0,1$, there is a genus $1$ smooth curve given by the equation
    \[ y^2=x(x-1)(x-\lambda), \]
    and different $\lambda$ give non-isomorphic curves.
\end{exm}

It is natural to wonder if for a given genus $g$, there is a space $M_g$ paramterizing curves of genus $g$, known as a moduli space. More precisely, for any space $X$, we want the set $\ms{Maps}(X, \mc{M}_g)$ of maps from $X$ to $\mc{M}_g$ to correspond to familes of smooth curves of genus $g$ over $X$. In fact, such a space exists, and the following is true:

\begin{prop}[Riemann, Deligne--Mumford]
    For any $g, n$ such that $2g-2+n > 0$, there is a space $\mc{M}_{g,n}$ of dimension $3g-3+n$ parameterizing smooth curves of genus $g$ with $n$ distinct marked points.
\end{prop}

Unfortunately, $\mc{M}_{g,n}$ is not compact, which geometrically means that there are families of smooth curves over $\C \setminus 0$ which cannot be filled in (meaning adding a fiber over $0$) with a smooth curve. In order to compactify $\mc{M}_{g,n}$, we will add nodal curves.

\begin{prop}[Deligne--Mumford]
    For any $g,n$ such that $2g-2+n > 0$, there exists a compact space $\Mbar_{g,n}$ of dimension $3g-3+n$ parameterizing families of possibly nodal curves $C$ of genus $g$ with $n$ distinct marked points satisfying the following conditions:
    \begin{enumerate}
        \item The marked points are away from the nodes; i.e. they lie on smooth points of $C$.
        \item For any irreducible component $C'$ of $C$ with genus $g'$, if $n'$ denotes the number of nodes and marked points on $C'$, then $2g' - 2 + n' > 0$.
    \end{enumerate}
\end{prop}

Now, the identity map $\Mbar_{g,n} \to \Mbar_{g,n}$ corresponds to some family of curves over $\Mbar_{g,n}$, and this is the \textit{universal family}, which we will denote $\mc{C}_{g,n}$. This means that for any map $f \colon X \to \Mbar_{g,n}$, the corresponding family is given by $f^* \mc{C}_{g,n}$.

\subsection{Integrals on the moduli space of curves}

We will define line bundles $L_i$ for $i=1,\ldots, n$ on $\Mbar_{g,n}$ as follows. For a point $[C, x_1, \ldots, x_n] \in \Mbar_{g,n}$, we will define
\[ (L_i)_{[C,x_1, \ldots, x_n]} = T_{x_i} C. \]
More precisely, if $x_i \colon \Mbar_{g,n} \to \mc{C}_{g,n}$ is the map corresponding to the $i$-th marked point, then $L_i = x_i^* \Omega^1_{\mc{C}_{g,n}/\Mbar_{g,n}}$, where $\Omega^1_{\mc{C}_{g,n}/\Mbar_{g,n}}$ is the relative cotangent bundle. Now, we will define $\psi_i = c_1(L_i) \in H^2(\Mbar_{g,n})$ to be the first Chern class of $L_i$. These $\psi$ classes are known as \textit{gravitational descendents} in the physics literature, where the integrals below are related to 2D topological gravity.

\begin{defn}
    For any integers $a_1, \ldots, a_n \geq 0$, define
    \[ \ev{\psi_1^{a_1} \cdots \psi_n^{a_n}}_{g,n} \coloneqq \int_{\Mbar_{g,n}} \psi_1^{a_1} \psi_2^{a_2} \cdots \psi_n^{a_n}. \]
\end{defn}

\section{The KdV hierarchy}

The behavior of waves in shallow water is described by the \textit{Korteweg-de Vries equation:}
\[ \pdv{u}{t_1} = u \pdv{u}{t_0} + \frac{1}{12} \pdv[3]{u}{t_0}. \]
The coefficients can be scaled to any nonzero real numbers by scaling the variables. The KdV equation has a solution given by
\[ u_1(t_0, t_1) = \frac{c}{2} \sech^2\qty(\frac{\sqrt{c}}{2}(t_0-ct_1+\delta)), \]
where $c$ and $\delta$ are constants and we impose $u(x,t) = f(t_0-ct_1)$ to be a travelling wave. In fact, there are infinitely many exact solutions $u_n(t_0, t_1)$ to the KdV equation, which behave like independent waves for large time (for both $t_1 \ll 0$ and $t_1 \gg 0$) but can collide in small time. Waves which exhibit this type of behavior are called \textit{solitons}. The infinitely many exact solutions make the KdV equation an \textit{integrable system} with infinitely many degrees of freedom.

The KdV equation generates an infinite-dimensional system of differential equations using the following procedure. Define the \textit{Schrodinger operator} by
\[ L \coloneqq (\partial_0^2 + u). \]
Its square root can be written as
\[ (\partial_0^2 + u)^{\frac{1}{2}} = \partial_0 + \frac{1}{2} u \partial_0^{-1} - \frac{1}{4} u_0 \partial_0^{-2} + \qty(\frac{u_{00}}{8} - \frac{u^2}{8}) \partial_0^{-3} + \cdots \]
Now for an expression of the form $M = \sum_{\ell=0}^{\infty} g_{\ell} \partial^{n-\ell}$, define
\[ M_+ \coloneqq \sum_{\ell=0}^n g_{\ell} \partial^{n-\ell}, \qquad M_- \coloneqq M - M_+. \]
We can now define the operators
\[ K_{a}(u) \coloneqq -[L, L^{\frac{2a+1}{2}}_+] \]
for nonnegative integers $a$ and obtain the \textit{KdV hierarchy}
\[ \pdv{u}{t_a} = K_a(u). \]
The first two equations are in fact
\[ \pdv{u}{t_0} = \pdv{u}{t_0}, \qquad \pdv{u}{t_1} = \frac{3}{2} u \pdv{u}{t_0} + \frac{1}{4} \pdv[3]{u}{t_0}, \]
and we see the first is a tautology and the second is the KdV equation up to scaling. What justifies calling this an integrable hierarchy is the commutation relation
\[ \pdv{x_b} K_a(u) = \pdv{x_a} K_b(u) \]
for all $a,b$. This states that the evolution of the system in the $t_b$ direction and the evolution of the system in the $t_a$ direction commute.

\section{The Kontsevich-Witten theorem}

Define the partition function (precisely, this is the all-genus Gromov-Witten potential of a point)
\[ \mc{Z} = \exp\qty(\sum_{\substack{g=0 \\ n=1 \\ 2g-2+n > 1}}^{\infty} \frac{\hslash^{2g-2+n}}{n!} \sum_{a_1, \ldots, a_n} \ev{\psi_1^{a_1} \cdots \psi_n^{a_n}}_{g,n} t_{a_1} \cdots t_{a_n}). \]

\begin{thm}[Kontsevich 1992]
    The function
    \[ u(t_0, t_1, \ldots) = \pdv[2]{t_0} \log \mc{Z} \]
    is the unique solution to the KdV hierarchy with initial condition $u(t_0, 0, 0, \ldots) = t_0$.
\end{thm}

This was proven by Kontsevich using a combinatorial description of $\Mbar_{g,n}$ in terms of thickened graphs and computing various matrix integrals. This theorem has several other proofs using various techniques. We will briefly outline three of them.
\begin{itemize}
    \item Okounkov and Pandharipande (2001) gave a proof which proceeds first by relating the integrals of $\psi$-classes on $\Mbar_{g,n}$ to counts of permutations known as \textit{Hurwitz numbers}, continues by constructing a different matrix model than the one considered by Kontsevich, and concludes by relating their matrix model to that of Kontsevich using techniques from probability theory.
    \item Mirzakhani (2007) gave a proof which proceeds first by defining a metric (the so-called \textit{Weil-Petersson metric}) on $\Mbar_{g,n}$ and computing a recursive formula for the volume of $\Mbar_{g,n}$ and concludes by relating the volumes to the integrals we defined in the first section.
    \item Kazarian and Lando (2007) gave a proof which relates the Hurwitz numbers considered by Okounkov and Pandharipande to a different integrable hierarchy known as the KP hierarchy, and then concludes by reducing the KP hierarchy to the KdV hierarchy.
\end{itemize}




\end{document}
