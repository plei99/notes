\documentclass{amsart}
\usepackage{amsmath}
\usepackage{amssymb}
\usepackage{amsthm}
%\usepackage{MnSymbol}
\usepackage{bm}
\usepackage{accents}
\usepackage{mathtools}
\usepackage{tikz}
\usetikzlibrary{calc}
\usetikzlibrary{decorations.pathmorphing,shapes}
\usetikzlibrary{automata,positioning}
\usepackage{tikz-cd}
\usepackage{forest}
\usepackage{braket} 
\usepackage{listings}
\usepackage{mdframed}
\usepackage{verbatim}
\usepackage{physics2}
\usephysicsmodule{ab,ab.legacy,diagmat,xmat,op.legacy}
\usepackage{derivative}
\usepackage{fixdif}
\usepackage{stmaryrd}
\usepackage{leftindex} 
% \usepackage{euscript} 
\usepackage[mathcal]{eucal}
\usepackage{stackengine} 
%\usepackage{/home/patrickl/homework/macaulay2}

%font
\usepackage[lf]{ebgaramond}
% \usepackage{mathpazo} 
\usepackage{inconsolata}
\usepackage[scaled]{helvet} 
\usepackage{microtype}
% \usepackage{fontspec} 
% \setmainfont{Tex Gyre Pagella}
% \usepackage[OT1,euler-digits]{eulervm}
% \usepackage{euler-math} 
% \usepackage[scaled=0.86]{berasans}
% \let\sffamilyold\sffamily
% \def\sffamily{\fontencoding{T1}\sffamilyold}
% \setmonofont{Inconsolatazi4}

%CS packages
\usepackage{algorithmicx}
\usepackage{algpseudocode}
\usepackage{algorithm}

% typeset and bib
\usepackage[english]{babel} 
% \usepackage[utf8]{inputenc} 
% \usepackage[T1]{fontenc}
% \usepackage[backend=biber,style=alphabetic,maxalphanames=4,maxnames=5,hyperref]{biblatex}
\usepackage[bookmarks, colorlinks, breaklinks]{hyperref} 
\hypersetup{linkcolor=blue,citecolor=magenta,filecolor=black,urlcolor=blue}
\usepackage{cleveref}
\usepackage{graphicx}
\graphicspath{{./}}


% other formatting packages
\usepackage{float}
\usepackage{booktabs}
\usepackage[shortlabels]{enumitem}
\setitemize{noitemsep}
\usepackage{csquotes}
%\usepackage{titlesec}
%\usepackage{titling}
%\usepackage{fancyhdr}
%\usepackage{lastpage}
% \usepackage{parskip}
\newlist{mydescription}{description}{1}
\setlist[mydescription]{style=nextline,
                        font=\bfseries,
                        % Tweak the next 4 options as needed:
                        labelindent=1cm, 
                        leftmargin =2cm,
                        rightmargin=1cm,
                        topsep     =1ex
                       }

\usepackage{lipsum}

% delimiters
\DeclarePairedDelimiter{\gen}{\langle}{\rangle}
\DeclarePairedDelimiter{\floor}{\lfloor}{\rfloor}
\DeclarePairedDelimiter{\ceil}{\lceil}{\rceil}


\newtheorem{thm}{Theorem}[section]
\newtheorem{cor}[thm]{Corollary}
\newtheorem{prop}[thm]{Proposition}
\newtheorem{lem}[thm]{Lemma}
\newtheorem{conj}[thm]{Conjecture}
\newtheorem{quest}[thm]{Question}
\newtheorem{claim}[thm]{Claim}

\theoremstyle{definition}
\newtheorem{defn}[thm]{Definition}
\newtheorem{defns}[thm]{Definitions}
\newtheorem{con}[thm]{Construction}
\newtheorem{exm}[thm]{Example}
\newtheorem{exms}[thm]{Examples}
\newtheorem{notn}[thm]{Notation}
\newtheorem{notns}[thm]{Notations}
\newtheorem{addm}[thm]{Addendum}
\newtheorem{exer}[thm]{Exercise}

\theoremstyle{remark}
\newtheorem{rmk}[thm]{Remark}
\newtheorem{rmks}[thm]{Remarks}
\newtheorem{warn}[thm]{Warning}
\newtheorem{sch}[thm]{Scholium}


% unnumbered theorems
\theoremstyle{plain}
\newtheorem*{thm*}{Theorem}
\newtheorem*{prop*}{Proposition}
\newtheorem*{lem*}{Lemma}
\newtheorem*{cor*}{Corollary}
\newtheorem*{conj*}{Conjecture}

% unnumbered definitions
\theoremstyle{definition}
\newtheorem*{defn*}{Definition}
\newtheorem*{exer*}{Exercise}
\newtheorem*{defns*}{Definitions}
\newtheorem*{con*}{Construction}
\newtheorem*{exm*}{Example}
\newtheorem*{exms*}{Examples}
\newtheorem*{notn*}{Notation}
\newtheorem*{notns*}{Notations}
\newtheorem*{addm*}{Addendum}


\theoremstyle{remark}
\newtheorem*{rmk*}{Remark}

% shortcuts
\newcommand{\Ima}{\mathrm{Im}}
\newcommand{\A}{\mathbb{A}}
\newcommand{\G}{\mathbb{G}}
\newcommand{\N}{\mathbb{N}}
\newcommand{\R}{\mathbb{R}}
\newcommand{\C}{\mathbb{C}}
\newcommand{\Z}{\mathbb{Z}}
\newcommand{\Q}{\mathbb{Q}}
\newcommand{\E}{\mathbb{E}}
\newcommand{\D}{\mathbb{D}}
\renewcommand{\k}{\Bbbk}
\renewcommand{\L}{\mathbb{L}}
\renewcommand{\P}{\mathbb{P}}
\newcommand{\M}{\mathcal{M}}
\newcommand{\sH}{\mathsf{H}}
\newcommand{\pH}{\leftindex^p{\mathsf{H}}}
\newcommand{\Mbar}{\overline{\mathcal{M}}}
\newcommand{\g}{\mathfrak{g}}
\newcommand{\h}{\mathfrak{h}}
\newcommand{\n}{\mathfrak{n}}
\renewcommand{\b}{\mathfrak{b}}
\newcommand{\ep}{\varepsilon}
\newcommand*{\dt}[1]{%
   \accentset{\mbox{\Huge\bfseries .}}{#1}}
%\renewcommand{\abstractname}{Official Description}
\newcommand{\mc}[1]{\mathcal{#1}}
% \newcommand{\msc}[1]{\mathscr{#1}}
\newcommand{\T}{\mathbb{T}}
\newcommand{\mf}[1]{\mathfrak{#1}}
\newcommand{\mbf}[1]{\mathbf{#1}}
\newcommand{\bv}{\mbf{v}}
\newcommand{\bq}{\mbf{q}}
\newcommand{\bp}{\mbf{p}}
\newcommand{\btau}{\bm{\tau}}
\newcommand{\mr}[1]{\mathrm{#1}}
\newcommand{\on}[1]{\operatorname{#1}}
\newcommand{\ms}[1]{\mathsf{#1}}
\newcommand{\mt}[1]{\mathtt{#1}}
\newcommand{\ol}[1]{\overline{#1}}
\newcommand{\ul}[1]{\underline{#1}}
\newcommand{\wt}[1]{\widetilde{#1}}
\newcommand{\wh}[1]{\widehat{#1}}
\renewcommand{\div}{\operatorname{div}}
\newcommand{\1}{\mathbf{1}}
\newcommand{\2}{\mathbf{2}}
\newcommand{\3}{\mathbf{3}}
\newcommand{\I}{\mathrm{I}}
\newcommand{\II}{\mr{I}\hspace{-1.3pt}\mr{I}}
\newcommand{\III}{\mr{I}\hspace{-1.3pt}\mr{I}\hspace{-1.3pt}\mr{I}}
\renewcommand{\v}{\mbf{v}}
\newcommand{\w}{\mbf{w}}
\newcommand{\bmu}{\bm{\mu}}
\newcommand{\pre}{\mr{pre}}
\newcommand{\pt}{\mr{pt}}
\newcommand{\vir}{\mr{vir}}
\newcommand{\fl}{\mr{fl}}
\newcommand{\ps}[1]{\llbracket #1 \rrbracket}
\newcommand{\ls}[1]{\llparenthesis #1 \rrparenthesis}

\DeclareMathOperator{\Der}{Der}
\DeclareMathOperator{\Tor}{Tor}
\DeclareMathOperator{\Hom}{Hom}
\DeclareMathOperator{\iHom}{\ms{Hom}} 
\DeclareMathOperator{\End}{End}
\DeclareMathOperator{\Ext}{Ext}
\DeclareMathOperator{\ad}{ad}
\DeclareMathOperator{\Aut}{Aut}
\DeclareMathOperator{\Rad}{Rad}
\DeclareMathOperator{\Pic}{Pic}
\DeclareMathOperator{\NS}{NS}
\DeclareMathOperator{\supp}{supp}
\DeclareMathOperator{\Supp}{Supp}
\DeclareMathOperator{\depth}{depth}
\DeclareMathOperator{\sgn}{sgn}
\DeclareMathOperator{\spec}{Spec}
\DeclareMathOperator{\Spec}{Spec}
\DeclareMathOperator{\proj}{Proj}
\DeclareMathOperator{\Proj}{Proj}
\DeclareMathOperator{\ord}{ord}
\DeclareMathOperator{\Div}{Div}
\DeclareMathOperator{\Bl}{Bl}
\DeclareMathOperator{\IC}{IC}
\DeclareMathOperator{\coker}{coker}
\DeclareMathOperator{\ev}{ev}
\DeclareMathOperator{\st}{st}
\DeclareMathOperator{\pr}{pr}
\DeclareMathOperator{\ch}{ch}
\DeclareMathOperator{\Cont}{Cont}

% \addbibresource{../../notes/math.bib}

\title{Decomposition theorem and relative Hard Lefschetz theorem}
\author{Patrick Lei}
\date{February 21, 2025}
\allowdisplaybreaks

\begin{document}
    
\begin{abstract}
    I will explain the decomposition theorem and the relative Hard Lefschetz theorem and then discuss some applications.
\end{abstract}

\maketitle

\section{Introduction}%
\label{sec:Introduction}

In this talk, we will work with sheaves with coefficients in a field $\k$ of characteristic $0$. $X$ will be an algebraic variety over $\C$.
Recall that the derived category $D^b_c(X)$ of constructible sheaves has a \textit{perverse $t$-structure} given by 
\begin{align*}
    \leftindex[I]^p D^b_c(X)^{\leq 0} &= \ab\{ \mc{F} \in D_c^b(X) \mid \dim \on{supp} \ms{H}^i(\mc{F}) \leq -i \text{ for all } i \}, \\
    \leftindex[I]^p D^b_c(X)^{\geq 0} &= \ab\{ \mc{F} \in D_c^b(X) \mid \dim \on{supp} \ms{H}^i(\D\mc{F}) \leq -i \text{ for all } i \}, 
\end{align*}
where the Verdier duality functor is given by
\[ \D \mc{F} \coloneqq \iHom(\mc{F}, \omega_X). \]
Here, the \textit{dualizing complex} $\omega_X$ is given by
\[ \omega_X \coloneqq (X \to \pt)^! \ul{\k}. \]
The category $\ms{Perv}(X)$ of perverse sheaves is defined to be the heart of this $t$-structure.

We will now introduce intersection cohomology (IC) complexes. 
\begin{defn}
    Let $j \colon Y \hookrightarrow X$ be a locally closed embedding. Then the \textit{intermediate extension} functor is given by
    \[ j_{!*} \colon \ms{Perv}(Y) \to \ms{Perv}(X), \qquad \mc{F} \mapsto \Im (\leftindex^p \sH^0(j_! \mc{F}) \to \leftindex^p \sH^0(j_* \mc{F})). \]
\end{defn}

\begin{defn}
    Let $j \colon Y \subset X$ be a smooth, locally closed subvariety and $\mc{L}$ be a local system on $Y$. Then the \textit{intersection cohomology complex} of $(Y, \mc{L})$ is the sheaf
    \[ \on{IC}(Y, \mc{L}) \coloneqq j_{!*} \mc{L}[\dim Y]. \]
    If $Y = X^{\ms{sm}}$ is the smooth locus of $X$ and $\mc{L} = \ul{\k}$ is the constant sheaf, then we abbreviate $\IC(X) \coloneqq \IC(X^{\ms{sm}}, \ul{\k})$.
\end{defn}

We are now ready to state the main results. Recall that simple objects in $\ms{Perv}(X)$ are IC complexes.

\begin{thm}[Decomposition theorem]
    Let $f \colon X \to Y$ be proper. Then if $\mc{F} \in \ms{Perv}(X)$ is semisimple, we have the following:
    \begin{enumerate}
        \item The pushforward $f_* \mc{F}$ decomposes as a direct sum
            \[ f_* \mc{F} \cong \bigoplus_{i \in \Z} \leftindex^p \sH^i(f_* \mc{F})[-i]; \]
        \item Each summand decomposes as a direct sum
            \[ \leftindex^p \sH^i(f_* \mc{F}) \cong \bigoplus_{\beta} \IC(X_{\beta}, \mc{L}_{\beta}) \]
            of IC sheaves.
    \end{enumerate}
\end{thm}

This theorem was originally proved by Beilinson-Bernstein-Deligne(-Gabber) using an argument involving reduction to positive characteristic. However, they require that $\mc{F}$ satisfies a technical condition called being \textit{of geometric origin}. There is another proof given by Saito using mixed Hodge modules. Because these proofs are far beyond my mathematical ability, we will not discuss them.

\section{Relative Hard Lefschetz}%
\label{sec:Relative Hard Lefschetz}

Recall that for a compact K\"ahler manifold $X$, the \textit{hard Lefschetz theorem} states that
\[ \omega^k \colon H^{n-k}(X) \to H^{n+k}(X) \]
is an isomorphism, where $\omega$ is the K\"ahler form and $n$ is the complex dimension of $X$. We will generalize this to projective morphisms of algebraic varieties.
Recall that if $f \colon X \to Y$ is projective, it factors as
\[ X \hookrightarrow Y \times \P^N \to Y. \]
Then let $\eta \coloneqq c_1(\mc{O}(1)) \in H^2(Y \times \P^N, \Q)(1)$. By definition, this is a morphism $\eta \colon \ul{\Q}_X \to \ul{\Q}_X[2](1)$, so tensoring with $\mc{F}$ and pushing forward gives us a morphism
\[ f_* (\eta \otimes \on{id}_{\mc{F}}) \colon f_* \mc{F} \to f_* \mc{F}[2](1), \]
which by abuse of notation we will also call $\eta$.

\begin{thm}[Relative Hard Lefschetz]
    For any non-negative $k \geq 0$, the map
    \[ \eta^k \colon \leftindex^p \sH^{-k}(f_* \mc{F}) \to \leftindex^p \sH^k (f_* \mc{F})(k) \]
    is an isomorphism.
\end{thm}

\begin{proof}[Sketch of proof]
    We will outline how to deduce this result from the decomposition theorem. The first simplification we make is that by using the projection formula for the pushforward along $g \colon X \hookrightarrow Y \times \P^N$, we obtain
    \[ g_* (\eta \otimes \on{id}_{\mc{F}}) = \eta \otimes \on{id}_{g_* \mc{F}}. \]
    Then define varieties
    \begin{align*}
        F &\coloneqq \ab\{(L, L') \in \P^N \times (\P^N)^{\vee} \mid L' \text{ annihilates } L \}, \\
        S &\coloneqq (\P^N \times (\P^N)^{\vee}) \setminus F.
    \end{align*}
    Now consider the following diagram:
    \begin{equation*}
    \begin{tikzcd}
        Y \times F \ar{r}{i} & Y \times \P^N \times (\P^N)^{\vee} \ar[swap]{dl}{p'} \ar[swap]{dr}{u'} & Y \times S \ar[crossing over]{dll}{q_2} \ar[swap]{l}{j} \ar{d}{q_1} \\
        Y \times (\P^N)^{\vee} \ar{dr}{u} & & X = Y \times \P^N \ar{dl}{f} \\
        & Y.
    \end{tikzcd}
    \end{equation*}
    \begin{enumerate}
        \item Consider $\eta$ as a class in $H^2(Y \times \P^N \times (\P^N)^{\vee}, \Q)(1)$. Alternatively, consider the divisor class
            \[ \theta = \on{cl}(Y \times F) \in H^2(Y \times \P^N \times (\P^N)^{\vee}, \Q)(1). \]
            Then for any $\mc{G} \in D^b_c(Y \times \P^N \times (\P^N)^{\vee})$, we have $\eta = \theta$ as maps
            \[ \leftindex^p \sH^k(p'_* \mc{G}) \to \leftindex^p \sH^{k+2}(p'_* \mc{G})[2](1). \]
            From now on, we will replace $\mc{F}$ with $(u')^* \mc{F}[N]$, which is fine because
            \[ u^* f_* \mc{F} \cong p'_* (u')^* \mc{F} \]
            by smooth base change.
        \item Consider the diagram
            \begin{equation*}
            \begin{tikzcd}
                \mc{F} \ar{rr} \ar{dd}{\theta} & & i_* i^* \mc{F} \ar{dd}{\theta} \ar{dl}{\sim} \\
                & i_* i^! \mc{F}[2](1) \ar{dl} \\
                \mc{F}[2](1) \ar{rr} & & i_* i^* \mc{F}[2](1),
            \end{tikzcd}
            \end{equation*}
            where all unlabelled arrows come from adjunctions. The fact that the first diagonal morphism exists and is an isomorphism is because the morphism $Y \times F \to X$ is smooth, and so in this case there is a natural isomorphism
            $i^* \mc{F} \to i^! \mc{F}[2](1)$. Because the cycle class $\theta$ is defined by
            \[ \ul{\Q} \to i_* i^*\ul{\Q} \cong i_* i^! \ul{\Q}[2](1) \to \ul{\Q}[2](1), \]
            we see that the top left triangle commutes. Applying adjunction twice to see
            \[ \Hom(i_* i^* \mc{F}, i_* i^* \mc{F}[2](1)) \cong \Hom(i^* \mc{F}, i^* \mc{F}[2](1)) \cong \Hom(\mc{F}, i_* i^* \mc{F}[2](1)), \]
            we see that the bottom right triangle commutes. Then apply $p'_*$ and obtain the diagram
            \begin{equation*}
            \begin{tikzcd}
                p'_* \mc{F} \ar{r}{\alpha} \ar{d}{\theta} & p'_* i_* i^* \mc{F} \ar{d}{\theta} \ar{dl}{\beta} \\
                p'_* \mc{F}[2](1) \ar{r}{\alpha} & p'_* i_* i^* \mc{F}[2](1).
            \end{tikzcd}
            \end{equation*}
        \item Applying $p'_* = p'_!$ (here $p'$ is proper) to the distiguished triangle
            \[ j_! j^* \to \mr{id} \to i_* i^* \xrightarrow{[1]} \]
            yields the distinguished triangle
            \[ (q_2)_! j^* \mc{F} \to p'_* \mc{F} \to p'_* i_* i^* \mc{F} \xrightarrow{[1]}. \]
            Using the fact that $S$ is affine over both $\P^N$ and $(\P^N)^{\vee}$ and the fact that $j^* \mc{F}$ is perverse (compactly supported pushforward along affine morphisms is left exact in the perverse $t$-structure), the long exact sequence
            \[ \cdots \to \leftindex^p \sH^{-1}((q_2)_! j^* \mc{F}) \to \leftindex^p \sH^{-1}(p'_* \mc{F}) \to \leftindex^p \sH^{-1} (p'_*i_* i^* \mc{F}) \to \leftindex^p \sH^{0}((q_2)_! j^* \mc{F}) \to \cdots \]
            tells us that $\leftindex^p \sH^{-k}(\alpha)$ is an isomorphism for $k > 1$ and injective when $k = 1$. In fact, it identifies $\leftindex^p \sH^{-1}(p'_* \mc{F})$ with the largest subobject of $\leftindex^p \sH^{-1}(p'_* i_* i^* \mc{F})$ which lies in the essential image of $u^*[N]$.
        \item By a similar argument, the map
            \[ \leftindex^p \sH^{-1}(\beta) \colon \leftindex^p \sH^{-1}(p'_* i_* i^* \mc{F}) \to \leftindex^p \sH^1(p'_* \mc{F})(1) \]
            is surjective and identifies $\pH^1(p'_* \mc{F})(1)$ with the largest quotient of $\pH^{-1}(p'_* i_* i^* \mc{F})$ which lies in the essential image of $u^*[N]$.
        \item We need to prove that
            \[ \eta^k \colon \leftindex^p \sH^{-k}(f_* \mc{F}) \to \leftindex^p \sH^k (f_* \mc{F})(k) \]
            is an isomorphism. First note that $i^* \mc{F}[-1] = (u' \circ i)^* \mc{F}_X [N-1]$ is a semisimple perverse sheaf, so $i_* i^* \mc{F}[-1] \in \ms{Perv}(Y \times \P^N \times (\P^N)^{\vee})$ is also semisimple. We will now induct on $k$. When $k=0$, $\eta^0$ is the identity, so there is nothing to do.
            When $k = 1$, note that $\pH^{-1}(p'_* i_* i^* \mc{F})$ is semisimple. But this identifies the images of $\pH^{-1}(\alpha)$ and $\pH^{-1}(\beta)$, so the desired result follows from the fact that $\theta = \beta \circ \alpha$. Finally, factorizing
            \begin{align*}
                \theta_{\mc{F}}^k &= \beta[2k-2] \circ \alpha[2k-2] \circ \cdots \circ \beta[2] \circ \alpha[2] \circ \beta \circ \alpha \\
                &= \beta[2k-2] \circ \theta_{i_* i^* \mc{F}} \circ \alpha
            \end{align*}
            and using the inductive hypothesis gives the desired result. \qedhere
    \end{enumerate}
\end{proof}

\section{Example: toric varieties}%
\label{sec:Example: toric varieties}


\begin{defn}
    If $P$ is a simplicial polytope of dimension $n$ with $f_k$ faces of dimension $k$ for all $k < n$, define the \textit{$h$-polynomial} $h(P,t) = \sum h_k(P) t^k$ of $P$ by
    \[ h(P, t) \coloneqq (t-1)^n + f_0 (t-1)^{n-1} + \cdots + f_{n-1}. \]
    Also define the \textit{$g$-polynomial} of $P$ by
    \[ g(P,t) \coloneqq h_0 + \sum_{k=1}^{\floor{n/2}} (h_k - h_{k-1})t^{k-1}. \]
\end{defn}

A well-known result in combinatorics, proved in Fulton's book, is the following.
\begin{thm}
    Let $X_P$ (which is simplicial) denote the toric variety corresponding to $P$. Then $h^{2k}(X_P) = h_k(P)$ for all $0 \leq k \leq n$.
\end{thm}

As a corollary, the $g$-polynomial completely determines the $h$-polynomial and has positive coefficients. Here, use the fact that intersection cohomology and usual cohomology coincide for simplicial toric varieties and then use Poincar\'e duality and hard Lefschetz.

\begin{defn}
    For an arbitrary convex polytope, define $h(P,t)$ inductively by the formula
    \[ h(P,t) \coloneqq \sum_{F < P} g(F,t)(t-1)^{n - 1 - \dim F}, \]
    where we define $g(\emptyset, t) = h(\emptyset,t) = 1$ and the set of proper faces $F < P$ includes the empty set.
\end{defn}

\begin{thm}
    If $P$ is any rational convex polytope, we have
    \[ h_k(P) = \dim \mr{IH}^{2k}(X_P, \Q) \]
    for $0 \leq k \leq n$.
\end{thm}
This result can be proved by using the decomposition theorem for a simplicial subdivision of $P$, but we will just illustrate it with an example.
Let $\Delta$ be the three-dimensional cube and let $X$ be the corresponding toric variety, which has six singular points. We will consider the resolution $\tilde{X}$ of $X$ given by adding a new vertex at the center of each facet of $\Delta$ and taking the stellar subdivision. Call the resulting polytope $\tilde{\Delta}$. First note that
\begin{align*}
    h(\Delta, t) &= g(\emptyset, t) (t-1)^3 + 8 g(\mr{pt}, t)(t-1)^2 + 12 g([0,1], t) (t-1) + g([0,1]^2, t) \\
    &= (t-1)^3 + 8(t-1)^2 + 12(t-1) + 6(t+1) \\
    &= t^3 + 5t^2 + 5t + 1.
\end{align*}
On the other hand, for $\tilde{\Delta}$, we have $f_0 = 14$, $f_1 = 36$, and $f_2 = 24$. This implies that
\[ h(\tilde{\Delta}, t) = (t-1)^3 + 14(t-1)^2 + 36(t-1) + 24 = t^3 + 11t^2 + 11t + 1. \]

Geometrically, $f \colon \tilde{X} \to X$ blows up the six singular points of $X$ and replaces them with a copy of $\P^1 \times \P^1$ each. This implies that
\[ f_* \ul{\Q}[3] = \bigoplus_{k=-1}^1 \pH^k(f_* \ul{\Q}[3]) = \on{IC}(X) \oplus \bigoplus_{i=1}^6 \ul{\Q}_{p_i}[1] \oplus \bigoplus_{i=1}^6 \ul{\Q}_{p_i}[-1]. \]
This implies that $h^{k}(\tilde{\Delta}) = h^k(\Delta) + 6$ for $k = 2,4$, which is exactly what we calculated previously.



\end{document}

%%% Local Variables:
%%% mode: latex
%%% TeX-master: t
%%% End:
