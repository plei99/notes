\documentclass{amsart}
\usepackage{amsmath}
\usepackage{amssymb}
\usepackage{amsthm}
%\usepackage{MnSymbol}
\usepackage{bm}
\usepackage{accents}
\usepackage{mathtools}
\usepackage{tikz}
\usetikzlibrary{calc}
\usetikzlibrary{decorations.pathmorphing,shapes}
\usetikzlibrary{automata,positioning}
\usepackage{tikz-cd}
\usepackage{forest}
\usepackage{braket} 
\usepackage{listings}
\usepackage{mdframed}
\usepackage{verbatim}
\usepackage{physics}
\usepackage{stmaryrd}
\usepackage{mathrsfs} 
\usepackage{stackengine} 
%\usepackage{/home/patrickl/homework/macaulay2}

%font
\usepackage[sc]{mathpazo}
\usepackage{eulervm}
\usepackage[scaled=0.86]{berasans}
\usepackage{inconsolata}
\usepackage{microtype}

%CS packages
\usepackage{algorithmicx}
\usepackage{algpseudocode}
\usepackage{algorithm}

% typeset and bib
\usepackage[english]{babel} 
\usepackage[utf8]{inputenc} 
\usepackage[T1]{fontenc}
%\usepackage[backend=biber, style=alphabetic]{biblatex}
\usepackage[bookmarks, colorlinks, breaklinks]{hyperref} 
\hypersetup{linkcolor=black,citecolor=black,filecolor=black,urlcolor=black}
\usepackage{graphicx}
\graphicspath{{./}}

% other formatting packages
\usepackage{float}
\usepackage{booktabs}
\usepackage[shortlabels]{enumitem}
\usepackage{csquotes}
%\usepackage{titlesec}
%\usepackage{titling}
%\usepackage{fancyhdr}
%\usepackage{lastpage}
\usepackage{parskip}

\usepackage{lipsum}

% delimiters
\DeclarePairedDelimiter{\gen}{\langle}{\rangle}
\DeclarePairedDelimiter{\floor}{\lfloor}{\rfloor}
\DeclarePairedDelimiter{\ceil}{\lceil}{\rceil}


\newtheorem{thm}{Theorem}[section]
\newtheorem{cor}[thm]{Corollary}
\newtheorem{prop}[thm]{Proposition}
\newtheorem{lem}[thm]{Lemma}
\newtheorem{conj}[thm]{Conjecture}
\newtheorem{quest}[thm]{Question}

\theoremstyle{definition}
\newtheorem{defn}[thm]{Definition}
\newtheorem{defns}[thm]{Definitions}
\newtheorem{con}[thm]{Construction}
\newtheorem{exm}[thm]{Example}
\newtheorem{exms}[thm]{Examples}
\newtheorem{notn}[thm]{Notation}
\newtheorem{notns}[thm]{Notations}
\newtheorem{addm}[thm]{Addendum}
\newtheorem{exer}[thm]{Exercise}

\theoremstyle{remark}
\newtheorem{rmk}[thm]{Remark}
\newtheorem{rmks}[thm]{Remarks}
\newtheorem{warn}[thm]{Warning}
\newtheorem{sch}[thm]{Scholium}


% unnumbered theorems
\theoremstyle{plain}
\newtheorem*{thm*}{Theorem}
\newtheorem*{prop*}{Proposition}
\newtheorem*{lem*}{Lemma}
\newtheorem*{cor*}{Corollary}
\newtheorem*{conj*}{Conjecture}

% unnumbered definitions
\theoremstyle{definition}
\newtheorem*{defn*}{Definition}
\newtheorem*{exer*}{Exercise}
\newtheorem*{defns*}{Definitions}
\newtheorem*{con*}{Construction}
\newtheorem*{exm*}{Example}
\newtheorem*{exms*}{Examples}
\newtheorem*{notn*}{Notation}
\newtheorem*{notns*}{Notations}
\newtheorem*{addm*}{Addendum}


\theoremstyle{remark}
\newtheorem*{rmk*}{Remark}

% shortcuts
\newcommand{\Ima}{\mathrm{Im}}
\newcommand{\A}{\mathbb{A}}
\newcommand{\G}{\mathbb{G}}
\newcommand{\N}{\mathbb{N}}
\newcommand{\R}{\mathbb{R}}
\newcommand{\C}{\mathbb{C}}
\newcommand{\Z}{\mathbb{Z}}
\newcommand{\Q}{\mathbb{Q}}
\renewcommand{\k}{\Bbbk}
\renewcommand{\P}{\mathbb{P}}
\newcommand{\M}{\overline{M}}
\newcommand{\g}{\mathfrak{g}}
\newcommand{\h}{\mathfrak{h}}
\newcommand{\n}{\mathfrak{n}}
\renewcommand{\b}{\mathfrak{b}}
\newcommand{\ep}{\varepsilon}
\newcommand*{\dt}[1]{%
   \accentset{\mbox{\Huge\bfseries .}}{#1}}
%\renewcommand{\abstractname}{Official Description}
\newcommand{\mc}[1]{\mathcal{#1}}
% \newcommand{\msc}[1]{\mathscr{#1}}
\newcommand{\T}{\mathbb{T}}
\newcommand{\mf}[1]{\mathfrak{#1}}
\newcommand{\mr}[1]{\mathrm{#1}}
\newcommand{\on}[1]{\operatorname{#1}}
\newcommand{\ms}[1]{\mathsf{#1}}
\newcommand{\ol}[1]{\overline{#1}}
\newcommand{\ul}[1]{\underline{#1}}
\newcommand{\wt}[1]{\widetilde{#1}}
\newcommand{\wh}[1]{\widehat{#1}}
\renewcommand{\div}{\operatorname{div}}
\newcommand{\1}{\mathbf{1}}
\newcommand{\2}{\mathbf{2}}
\newcommand{\3}{\mathbf{3}}
\newcommand{\I}{\mathrm{I}}
\newcommand{\II}{\mr{I}\hspace{-1.3pt}\mr{I}}
\newcommand{\III}{\mr{I}\hspace{-1.3pt}\mr{I}\hspace{-1.3pt}\mr{I}}

\DeclareMathOperator{\Der}{Der}
\DeclareMathOperator{\Tor}{Tor}
\DeclareMathOperator{\Hom}{Hom}
\DeclareMathOperator{\End}{End}
\DeclareMathOperator{\Ext}{Ext}
\DeclareMathOperator{\ad}{ad}
\DeclareMathOperator{\Aut}{Aut}
\DeclareMathOperator{\Rad}{Rad}
\DeclareMathOperator{\Pic}{Pic}
\DeclareMathOperator{\supp}{supp}
\DeclareMathOperator{\Supp}{Supp}
\DeclareMathOperator{\depth}{depth}
\DeclareMathOperator{\sgn}{sgn}
\DeclareMathOperator{\spec}{Spec}
\DeclareMathOperator{\Spec}{Spec}
\DeclareMathOperator{\proj}{Proj}
\DeclareMathOperator{\Proj}{Proj}
\DeclareMathOperator{\ord}{ord}
\DeclareMathOperator{\Div}{Div}
\DeclareMathOperator{\Bl}{Bl}
\DeclareMathOperator{\coker}{coker}

\title{Rational curves on Calabi-Yau hypersurfaces}
\author{Patrick Lei}
\date{March 25, 2022}

\begin{document}
    
\maketitle

\begin{abstract}
    I will discuss sweeping out of Calabi-Yau hypersurfaces by abelian varieties as well as rational curves on such hypersurfaces. We will focus in particular on the case of quantic threefolds and discuss Clemens' conjecture.
\end{abstract}

\section{Introduction}

First, we recall some basic definitions.

\begin{defn}
    Let $X$ be a smooth (or maybe with nice singularities prescribed by the MMP) variety. Then $X$ is \textit{Calabi-Yau} if $\omega_X = \mc{O}_X$.
\end{defn}

\begin{exm}
    By the adjunction formula, a degree $n+2$ hypersurface in $\P^{n+1}$ is Calabi-Yau. For some low dimensional examples, this means that elliptic curves, quartic surfaces, and quintic threefolds are Calabi-Yau.
\end{exm}

\begin{exm}
    Recall that if $f \colon Y \to X$ is a double cover branched along a divisor $D$, we have
    \[ K_Y = f^* \qty(K_X + \frac{1}{2} D). \]
    This implies that the double cover of $\P^2$ branched along a sextic is Calabi-Yau as is the double cover of $\P^3$ branched along an octic surface.
\end{exm}

\section{Rational curves on Calabi-Yau hypersurfaces}

We will now consider rational curves on Calabi-Yau hypersurfaces.

\begin{thm}[Clemens]
    Let $X$ be a general Calabi-Yau hypersurface of dimension $n \geq 2$. Then $X$ contains rational curves of arbitrarily large degree.
\end{thm}

For a quintic threefold, note that if we have $\P^1 \simeq C \subset X$, then the exact sequence
\[ 0 \to T_{C} \to T_{\P^3}|_C \to N_{C/X} \to 0 \]
tells us that $\deg N_{C/X} = -2$. Hirzebruch-Riemann-Roch for a smooth curve $C$ of genus $g$ says that
\[ \chi(C, \mc{E}) = \on{rk} \mc{E} (1-g) + c_1(\mc{E}), \]
and this gives us $\chi(N_{C/X}) = 0$ in our case. Generically, we expect (by upper semicontinuity of cohomology) that $h^0(N_{C/X}) = h^1(N_{C/X}) = 0$, and thus we expect
\[ N_{C/X} = \mc{O}(-1) \oplus \mc{O}(-1). \]
Actually, for any smooth rational curve $\P^1 \hookrightarrow X$ in a general quintic threefold, that is exactly the normal bundle by a result of Bin Wang in 2015. By basic deformation theory, we see that such $C$ are infinitesimally rigid. In fact, we can strengthen the previous theorem:

\begin{thm}
    Let $X$ be a quintic threefold. then $X$ has infinitely many infinitesimally rigid rational curves.
\end{thm}

On the flip side, we expect that there are not too many rational curves on a quintic threefold.

\begin{conj}[Clemens]
    Let $X$ be a general quintic threefold. Recall that by the Lefschetz hyperplane theorem, $H_2(X) = \Z$ is generated by the class of a line. Then for any $d > 0$, $X$ contains only finitely many rational curves of degree $d$.
\end{conj}

This conjecture is true in degrees $d \leq 11$ by a result of Cotterill in 2012. It also inspired the development of enumerative geometry, mirror symmetry, and many other ideas, but we will not discuss them here.

\begin{rmk}
    The conjecture is crucially supported by the fact that deformations of $C \simeq \P^1$ inside $X$ are unobstructed. On the other hand, if $Q$ is a quartic surface and $C \subset Q$ is a rational curve, then $N_{C/Q} = \mc{O}(-2)$, and because $h^1(N_{C/X}) = 1$, we see that deformations of $C$ in $Q$ are obstructed.
\end{rmk}

\begin{rmk}
    The Clemens conjecture is false if we let $X$ be a double cover of $\P^3$ ramified along an octic surface $S$. Consider lines $\ell$ such that $\ell \cap S = 2p_1 + 2 p_2 + 2 p_3 + p_4 + p_5$ as cycles. There is a $1$-dimensional family of such lines. If $\pi \colon X \to \P^3$ is the double cover, then $\pi^{-1}(\ell)$ is ramified over $\ell$ at $2$ points and is therefore rational. But this means we have a $1$-parameter family of rational curves in $X$ all of the same degree.
\end{rmk}

The remark is related to the following result:

\begin{prop}
    Let $X$ be a Calabi-Yau threefold and suppose that $X$ contains a smooth rational curve $C \simeq \P^1$ with normal bundle 
    \[ N_{C/X} \simeq \mc{O}(1) \oplus \mc{O}(1). \]
    Assume there exists a rational curve $C' \subset X$, a neighborhood $U \supset C \cup C'$, and an involution $i \colon U \to U$ such that
    \begin{enumerate}
        \item The fixed locus of $i$ is a smooth hypersurface of $U$ that meets $C$ transversally;
        \item $i(C) = C' \neq C$.
    \end{enumerate}
    Then,
    \begin{enumerate}
        \item If $i$ has at least one fixed point on $C$, then $X$ contains a one-parameter family of rational curves.
        \item If $i$ has two fixed points $p, p' \in C$ and the tangent directions to $C'$ at $p, p'$ are distinct in $\P(N_{C/X}) = C \times \P^1$, then $X$ is swept out by a two-parameter family of elliptic curves.
    \end{enumerate}
\end{prop}

To obtain this result, we need to choose a special octic surface. Let $\ell$ be a line in $\P^3$ and note that the map
\[ H^0(\P^3, \mc{O}_{\P^3}(8)) \to H^0(\P^1, \mc{O}_{\P^1}(8)) \]
is surjective (this is an exercise in Hartshorne). Now if we choose $f \in H^0(\P^1, \mc{O}(8))$ to vanish at three points with degrees $2,2,4$, then by a Bertini-type argument, the general octic surface $S$ living above $f$ is smooth. Now note that $\pi^{-1}(\ell)$ splits into two components because the local equation looks like
\[ y^2 = x^4 (x+1)^2 (x-1)^2, \]
and therefore the two components look like $y = \pm x^2 (x+1)(x-2)$. But now if we take $C, C'$ to be the two components and $i$ to exchange the two sheets, we see that both assumptions of the proposition are satisfied. We have already exhibited a one-parameter family of rational curves above, and later we will produce a two-parameter family  of elliptic curves.

\section{Sweeping out of hypersurfaces}

In this section we will discuss sweeping out of hypersurfaces by abelian varieties. This will roughly mean that a variety is covered by a generically finite morphism from a family of abelian varieties.

\begin{defn}
    Let $S$ be a quasiprojective variety and $f \colon \mc{Y} \to S$ be a smooth projective morphism of relative dimension $r$. Also let $X$ be a variety of dimension $n$. Then $X$ is \textit{rationally swept out} by members of $\mc{Y}$ if there exists a quasiprojective variety $B$ of dimension $n-r$, a morphism $B \to S$, and a dominant rational map $\mc{Y}_B \dashrightarrow X$.
\end{defn}

For example, we can consider when $S$ is a moduli space of varieties and $\mc{Y}$ is the universal family. Also, the definition of sweeping out is equivalent to the property that the union of the images of generically finite rational maps $Y_s \dashrightarrow X$ contains a Zariski open set of $X$.

The main result that Voisin proves about sweeping out of hypersurfaces by varieties is the following:

\begin{thm}[Voisin]
    Let $1 \leq r \leq n$ and $\gamma = \ceil{\frac{r-1}{2}}$. Let $S$ have dimension $C$ and $\mc{Y} \to S$ be a family of $r$-dimensional smooth projective varieties. Finally fix a positive integer $d$. Then, if the two inequalities
    \begin{enumerate}
        \item $(d+1) r \geq 2n + C + 2$;
        \item $(\gamma + 1) d \geq 2n - r + 1 + C$
    \end{enumerate}
    hold, the very general hypersurface of degree $d$ in $\P^{n+1}$ is \textbf{not} swept out by members of the family $\mc{Y} \to S$.
\end{thm}

This result is a consequence of a Hodge-theoretic result. First, let $U \subset \abs{\mc{O}_{\P^{n+1}}(d)}$ be the open set parameterizing smooth hypersurfaces. Let $\rho \colon \mc{M} \to U$ be a morphism with $\mc{M}$ smooth and quasiprojective such that the corank of $\rho$ is constant and equal to $C$ and that the image of $\rho$ is stable under $GL_{n+2}$. Let $\mc{X}_U$ be the universal smooth hypersurface and $j \colon \mc{X}_{\mc{M}} \hookrightarrow \mc{M} \times \P^{n+1}$ be the natural closed immersion.

\begin{thm}[Nori]\leavevmode
    \begin{enumerate}
        \item Assume that $(d+1)r \geq 2n+C+2$. Then the restriction
            \[ j^* \colon F^n H^{2n-r}(\mc{M} \times \P^{n+1}, \C) \to F^n H^{2n-r}(\mc{X}_{\mc{M}}, \C) \]
            is surjective.
        \item If $(\gamma + 1) d \geq 2n+1-r+C$, then the restriction
            \[ j^* \colon H^{2n-r-i}(\mc{M} \times \P^{n+1}, \C) \to H^{2n-r-i}(\mc{X}_{\mc{M}}, \C) \]
            is surjective for any $i \geq 1$.
    \end{enumerate}
\end{thm}

The proof of this is of course Hodge-theoretic and uses spectral sequences, and it is omitted here. Finally, we state a conjecture about the sweeping out of hypersurfaces by abelian varieties.

\begin{proof}[Proof of Voisin assuming Nori]
    Suppose that there is a $B$ of dimension $n-r$, maps $\rho \colon B \to S$ and $m \colon B \to U$, and a dominant $\phi \colon \mc{Y}_B \dashrightarrow \mc{X}_B$. But then if $f \colon \mc{Y}_B \to B$ and $\pi \colon \mc{X}_U \to U$ are the families, we have $\pi \circ \phi = m \circ f$.

    Now by shrinking $B$, we may assume that all $B_s$ for $s \in S$ are smooth and that the corank of $m_s \coloneqq m |_{B_S}$ is constant and $\leq C$. Now if $\mc{Y}_s \coloneqq f^{-1}(\mc{B}_s)$, write $\phi_s \coloneqq \phi |_{\mc{Y}_s}$. Therefore we have a rational map
    \[ \phi_s \colon \mc{Y}_s = Y_s \times \mc{B}_s \dashrightarrow \mc{X}_U \times_U \mc{B}_s \eqqcolon \mc{X}_s. \]
    But then the graph $\Gamma_{\phi_s} \subset Y_s \times \mc{B}_s \times_{\mc{B}_s} \mc{X}_s = Y_s \times \mc{X}_s$. This gives a cohomology class $\gamma_s \in H^{2n}(Y_s \times \mc{X}_s, \Q)$. Now Nori's theorem implies the class $\gamma_{s, r} \in H^r(Y_s, \Q) \otimes H^{2n-r}(\mc{X}_s, \Q)$ vanishes in 
    \[ H^r(Y_s, \Q)_{\mr{tr}} \otimes (H^{2n-r}(\mc{X}_s, \Q) / H^{2n-r}(\P^{n+1} \times \mc{B}_s, \Q)). \]

    Finally, returning to $\Gamma_{\phi} \subset \mc{Y}_B \times_U \mc{X}$, which has codimension $n$. If we fix $s \in S$, the class $[\Gamma_{\phi}] \in H^{2n}(\mc{Y}_B \times_U \mc{X}, \Q)$ restricts to $\gamma_s$. If we fix $u \in U$ and let $B_u \coloneqq m^{-1}(u)$. Then $\phi$ restricts to
    \[ \phi_u \colon \mc{Y}_u \eqqcolon \mc{Y} \times_S B_u \dashrightarrow X_u, \]
    which is dominant and generically finite on fibers. But now, a Hodge-theoretic argument tells us that
    \[ \phi_u^* \colon H^n(X_u, \Q)_{\mr{tr}} \to H^n(\mc{Y}_u, \Q) \]
    is injective and is in fact nonzero in $\Hom(H^n(X_u, \Q)_{\mr{tr}}, H^{n-r}(B_u, R^r f_* \Q_{\mr{tr}}))$. But now $[\Gamma_{\phi}]$ is nontrivial in $H^{2n}(\mc{X}_B, R^r f_* \Q_{\mr{tr}}) / H^{2n}(\mc{Y}_B \times \P^{n+1}, \Q)$, and so by a spectral sequence argument, $\gamma_{s,r}$ is nonzero, which is a contradiction.
\end{proof}

\begin{conj}[Lang]
    Let $X$ be not of general type. Then the union of the images of non-constant rational maps $\phi \colon A \dashrightarrow X$ from an abelian variety $A$ is $X$.
\end{conj}

Equivalently, this becomes:

\begin{conj}
    Let $X$ be a variety of Kodaira dimension $0 \leq \kappa(X) < \dim X$ (in particular, $X$ is not of general type). Then $X$ is rationally swept out by abelian varieties of dimension $r \geq 1$.
\end{conj}

Note that this conjecture is true when $X$ is a double cover of $\P^3$ branched along an octic surface $S$. Consider the two-parameter family of lines $\ell \subset \P^3$ such that
\[ \ell \cap S = 2 p_1 + 2 p_2 + p_3 + p_4 + p_5 + p_6. \]
Then the preimage of $\ell$ in $X$ is branched over $\ell \simeq \P^1$ at $4$ points, and therefore its normalization is an elliptic curve. But this implies that $X$ is swept out by a two-parameter family of elliptic curves.

\section{Sweeping out of Calabi-Yau hypersurfaces by abelian varieties}

\begin{thm}
    Let $X$ be a very general Calabi-Yau hypersurface in $\P^{n+1}$ (that is, of degree $d = n+2$). Then $X$ is \textbf{not} swept out by $r$-dimensional abelian varieties for any $r \geq 2$.
\end{thm}

\begin{proof}
    This boils down to checking some inequalities. First, recall that
    \[ \dim \mc{A}_r = \frac{r(r+1)}{2}. \]
    Then we need to check the two inequalities. The first is
    \[ (n+3)r \geq 2n + \frac{r(r+1)}{2} + 2, \]
    which is clearly true for $r \geq 2$ because
    \begin{align*}
        r\qty(n+3-\frac{r+1}{2}) \geq 2(n+1),
    \end{align*}
    which holds because $n + 3 - \frac{r+1}{2} + r \geq n+3$ and both $r \geq 2$ and $n+3-\frac{r+1}{2} \geq 3$. The second inequality is
    \[ (\gamma + 1)(n+2) \geq 2n - r + 1 + \frac{r(r+1)}{2}, \]
    and this holds for $r \geq 2$ by high school algebra (of the kind that I cannot bother to figure out).
\end{proof}

\begin{cor}
    If Lang's conjecture is true for a very general Calabi-Yau hypersurface $X$, then $X$ is swept out by elliptic curves.
\end{cor}

This will imply that $X$ has a uniruled divisor, but first we need the following lemma:

\begin{lem}
    Let $X$ be a very general Calabi-Yau hypersurface of dimension $\dim X \geq 2$. Then $X$ is not swept out by an isotrivial family of elliptic curves.
\end{lem}

\begin{lem}
    Suppose a variety $X$ is swept out by a non-isotrivial family of elliptic curves. Then $X$ has a uniruled divisor.
\end{lem}

\begin{proof}
    By assumption, we have a diagram
    \begin{equation*}
    \begin{tikzcd}
        \mc{K} \ar[dashrightarrow]{r}{\phi} \ar{d}{\pi} & X \\
        B,
    \end{tikzcd}
    \end{equation*}
    where $\mc{K} \to B$ is a family of elliptic curves. Now this is given by $B \to \mc{M}_{1,1}$, so we can choose a smooth projective model $B' \supset B$ and a smooth projective model $\mc{K}' \supset \mc{K}$. Now we can replace $\phi$ with an honest morphism
    \begin{equation*}
    \begin{tikzcd}
        \mc{K}' \ar{r}{\phi} \ar{d}{\pi} & X \\
        B'.
    \end{tikzcd}
    \end{equation*}
    But now the $j$-invariant map
    \[ j \colon B' \xrightarrow{\mc{K}'} \ol{\mc{M}}_{1,1} \to \P^1 \]
    (composing the defining map of $\mc{K}'$ with the coarse moduli space) is surjective. But now for $t \in \P^1$, consider the divisor
    \[ \mc{K}_t' \coloneqq (j \circ \pi)^{-1}(t). \]
    For a generic $t \in \P^1$, this is sent to a divisor of $X$, and in particular for any $t$, the image $\phi(\mc{K}_t')$ contains a divisor of $X$. But now if we take $t = \infty$, noting that an elliptic curve must degenerate to a union of rational curves, we see that any component of $\mc{K}_{\infty}'$ is uniruled, so we are done.
\end{proof}

\begin{cor}
    Clemens' conjecture and Lang's conjecture contradict each other.
\end{cor}


\end{document}
