\documentclass{amsart}
\usepackage{amsmath}
\usepackage{amssymb}
\usepackage{amsthm}
%\usepackage{MnSymbol}
\usepackage{bm}
\usepackage{accents}
\usepackage{mathtools}
\usepackage{tikz}
\usetikzlibrary{calc}
\usetikzlibrary{decorations.pathmorphing,shapes}
\usetikzlibrary{automata,positioning}
\usepackage{tikz-cd}
\usepackage{forest}
\usepackage{braket} 
\usepackage{listings}
\usepackage{mdframed}
\usepackage{verbatim}
\usepackage{physics}
\usepackage{stmaryrd}
\usepackage{mathrsfs} 
\usepackage{stackengine} 
%\usepackage{/home/patrickl/homework/macaulay2}

%font
\usepackage[sc]{mathpazo}
\usepackage{eulervm}
\usepackage[scaled=0.86]{berasans}
\usepackage{inconsolata}
\usepackage{microtype}

%CS packages
\usepackage{algorithmicx}
\usepackage{algpseudocode}
\usepackage{algorithm}

% typeset and bib
\usepackage[english]{babel} 
\usepackage[utf8]{inputenc} 
\usepackage[T1]{fontenc}
%\usepackage[backend=biber, style=alphabetic]{biblatex}
\usepackage[bookmarks, colorlinks, breaklinks]{hyperref} 
\hypersetup{linkcolor=black,citecolor=black,filecolor=black,urlcolor=black}
\usepackage{graphicx}
\graphicspath{{./}}

% other formatting packages
\usepackage{float}
\usepackage{booktabs}
\usepackage[shortlabels]{enumitem}
\usepackage{csquotes}
%\usepackage{titlesec}
%\usepackage{titling}
%\usepackage{fancyhdr}
%\usepackage{lastpage}
\usepackage{parskip}

\usepackage{lipsum}

% delimiters
\DeclarePairedDelimiter{\gen}{\langle}{\rangle}
\DeclarePairedDelimiter{\floor}{\lfloor}{\rfloor}
\DeclarePairedDelimiter{\ceil}{\lceil}{\rceil}


\newtheorem{thm}{Theorem}[section]
\newtheorem{cor}[thm]{Corollary}
\newtheorem{prop}[thm]{Proposition}
\newtheorem{lem}[thm]{Lemma}
\newtheorem{conj}[thm]{Conjecture}
\newtheorem{quest}[thm]{Question}

\theoremstyle{definition}
\newtheorem{defn}[thm]{Definition}
\newtheorem{defns}[thm]{Definitions}
\newtheorem{con}[thm]{Construction}
\newtheorem{exm}[thm]{Example}
\newtheorem{exms}[thm]{Examples}
\newtheorem{notn}[thm]{Notation}
\newtheorem{notns}[thm]{Notations}
\newtheorem{addm}[thm]{Addendum}
\newtheorem{exer}[thm]{Exercise}

\theoremstyle{remark}
\newtheorem{rmk}[thm]{Remark}
\newtheorem{rmks}[thm]{Remarks}
\newtheorem{warn}[thm]{Warning}
\newtheorem{sch}[thm]{Scholium}


% unnumbered theorems
\theoremstyle{plain}
\newtheorem*{thm*}{Theorem}
\newtheorem*{prop*}{Proposition}
\newtheorem*{lem*}{Lemma}
\newtheorem*{cor*}{Corollary}
\newtheorem*{conj*}{Conjecture}

% unnumbered definitions
\theoremstyle{definition}
\newtheorem*{defn*}{Definition}
\newtheorem*{exer*}{Exercise}
\newtheorem*{defns*}{Definitions}
\newtheorem*{con*}{Construction}
\newtheorem*{exm*}{Example}
\newtheorem*{exms*}{Examples}
\newtheorem*{notn*}{Notation}
\newtheorem*{notns*}{Notations}
\newtheorem*{addm*}{Addendum}


\theoremstyle{remark}
\newtheorem*{rmk*}{Remark}

% shortcuts
\newcommand{\Ima}{\mathrm{Im}}
\newcommand{\A}{\mathbb{A}}
\newcommand{\G}{\mathbb{G}}
\newcommand{\N}{\mathbb{N}}
\newcommand{\R}{\mathbb{R}}
\newcommand{\C}{\mathbb{C}}
\newcommand{\Z}{\mathbb{Z}}
\newcommand{\Q}{\mathbb{Q}}
\renewcommand{\k}{\Bbbk}
\renewcommand{\P}{\mathbb{P}}
\newcommand{\M}{\overline{M}}
\newcommand{\g}{\mathfrak{g}}
\newcommand{\h}{\mathfrak{h}}
\newcommand{\n}{\mathfrak{n}}
\renewcommand{\b}{\mathfrak{b}}
\newcommand{\ep}{\varepsilon}
\newcommand*{\dt}[1]{%
   \accentset{\mbox{\Huge\bfseries .}}{#1}}
%\renewcommand{\abstractname}{Official Description}
\newcommand{\mc}[1]{\mathcal{#1}}
% \newcommand{\msc}[1]{\mathscr{#1}}
\newcommand{\T}{\mathbb{T}}
\newcommand{\mf}[1]{\mathfrak{#1}}
\newcommand{\mr}[1]{\mathrm{#1}}
\newcommand{\on}[1]{\operatorname{#1}}
\newcommand{\ms}[1]{\mathsf{#1}}
\newcommand{\ol}[1]{\overline{#1}}
\newcommand{\ul}[1]{\underline{#1}}
\newcommand{\wt}[1]{\widetilde{#1}}
\newcommand{\wh}[1]{\widehat{#1}}
\renewcommand{\div}{\operatorname{div}}
\newcommand{\1}{\mathbf{1}}
\newcommand{\2}{\mathbf{2}}
\newcommand{\3}{\mathbf{3}}
\newcommand{\I}{\mathrm{I}}
\newcommand{\II}{\mr{I}\hspace{-1.3pt}\mr{I}}
\newcommand{\III}{\mr{I}\hspace{-1.3pt}\mr{I}\hspace{-1.3pt}\mr{I}}

\DeclareMathOperator{\Der}{Der}
\DeclareMathOperator{\Tor}{Tor}
\DeclareMathOperator{\Hom}{Hom}
\DeclareMathOperator{\End}{End}
\DeclareMathOperator{\Ext}{Ext}
\DeclareMathOperator{\ad}{ad}
\DeclareMathOperator{\Aut}{Aut}
\DeclareMathOperator{\Rad}{Rad}
\DeclareMathOperator{\Pic}{Pic}
\DeclareMathOperator{\supp}{supp}
\DeclareMathOperator{\Supp}{Supp}
\DeclareMathOperator{\depth}{depth}
\DeclareMathOperator{\sgn}{sgn}
\DeclareMathOperator{\spec}{Spec}
\DeclareMathOperator{\Spec}{Spec}
\DeclareMathOperator{\proj}{Proj}
\DeclareMathOperator{\Proj}{Proj}
\DeclareMathOperator{\ord}{ord}
\DeclareMathOperator{\Div}{Div}
\DeclareMathOperator{\Bl}{Bl}
\DeclareMathOperator{\coker}{coker}

\title{Lines on a cubic surface}
\author{Patrick Lei}
\date{February 23, 2022}

\begin{document}
    
\maketitle

\begin{abstract}
    I will tell you why there are 27 lines on a smooth cubic surface. Along the way, we will meet useful geometric notions such as Grassmannians, Chern classes, Schubert calculus, and intersection theory.
\end{abstract}

\section{Introduction}

Enumerative geometry, roughly, is a branch of algebraic geometry that counts that counts subvarieties of a given variety satisfying certain incidence conditions. For example, one of the most basic questions is:

\begin{quest}
    How many lines pass through two distinct points in the plane?
\end{quest}

The answer to this question was known probably since the beginning of humanity, but other questions are of course much harder to answer. For example:

\begin{conj}[Clemens]
    Let $X \subset \P^4$ be a smooth quintic threefold. Then for any positive degree $d$, $X$ has finitely many rational curves of degree $d$.
\end{conj}

This problem led to the development of many active areas of modern research coming from theoretical physics. These include mirror symmetry, Gromov-Witten theory, Donaldson-Thomas theory, and many other theories which all appear to be related to each other somehow. This is all beyond the scope of this talk, which will focus on the following problem which inhabits a middle ground of difficulty:

\begin{thm}[Cayley-Salmon]
    Let $X \subset \P^3$ be a smooth cubic surface. Then $X$ contains exactly $27$ lines.
\end{thm}

We will now begin by introducing the basic notions of our story.
\begin{defn}
    \textit{Projective space} $\P^n$ is defined by $\P^n \coloneqq (\C^{n+1} \setminus 0) / \C^*$, where $\C^*$ acts by scaling. Note that projective space has homogeneous coordinates $[x_0:\cdots:x_n]$, where $[x_0:\cdots:x_n] = [\lambda x_0: \cdots : \lambda x_n]$ for all $\lambda \in \C^*$.
\end{defn}

Closed (algebraic) subvarieties $X \subseteq \P^n$ are defined by the common vanishing loci of finitely many homogeneous polynomials $f_1, \ldots, f_n \in \C[x_0, \ldots, x_n]$. Note that if $f$ is homogeneous of degree $d$, then $f(\lambda \ul{x}) = \lambda^d f(\ul{x})$, so there is a well-defined vanishing locus of $f$. If $X$ is cut out by $f_1, \ldots, f_n$, we can check smoothness of $X$ by checking that there is no $x \in X$ where all partial derivatives $\pdv{f_i}{x_j}$ vanish simultaneously.

\begin{defn}
    A \textit{cubic surface} $X \subset \P^3$ is a subvariety cut out by a single homogeneous polynomial $f_3(x_0, x_1, x_2, x_3)$ of degree $3$.
\end{defn}

Using the above Jacobian criterion, we can check whether or not $X$ is smooth. A generic cubic surface is smooth, for example by Bertini's theorem.

\section{Grassmannians}

The Grassmannian $G(k, n)$ is a space that parameterizes $k$-dimensional subspaces of $\C^n$. Near a point $[V]$ corresponding to a subspace $V \subset \C^n$, there is a chart of $G(k,n)$ isomorphic to $\Hom(V, V^{\perp})$ consisting of subspaces $V'$ intersecting $V^{\perp}$ transversely. This is given by taking a morphism $\varphi \colon V \to V^{\perp}$ to its graph $\Gamma_{\varphi} \subset V \oplus V^{\perp} = \C^n$. This shows that $\dim G(k,n) = k(n-k)$. More intrinsically, we have $G(k, n) = U(n) / (U(k) \times U(n-k))$. There is a natural way to realize $G(k, n)$ as a projective variety using Pl\"ucker coordinates, but we will not discuss that here.

We will now define some vector bundles living on $G(k,n)$ which will be useful to us later. First, based on the discussion above, we note that for any $[V] \in G(k, n)$, $T_{[V]} G(k, n) = \Hom(V, \C^n/V)$ (because there is no preferred inner product on $\C^n$, this is preferred to what I wrote above). Next, there is a \textit{tautological bundle} $\mc{S} \subset G(k, n) \times \C^n$ on $G(k,n)$ whose fiber at a point $[V] \in G(k, n)$ is simply $V \subset \C^n$. More precisely, we have
\[ \mc{S} = \qty{([V], v) \in G(k, n) \times \C^n \mid v \in V}. \]

We will now discuss some important subvarieties of $G(k, n)$ which will give a stratification of $G(k, n)$ into affine spaces. Let $e_1, \ldots, e_n$ be the standard basis of $\C^n$ and let $L_i = \ev{e_1, \ldots, e_i}$. Then for a sequence of nonnegative integers $n-k \geq a_1 \geq \cdots \geq a_k \geq 0$, consider the set
\[ \Sigma_{\ul{a}} \coloneqq \qty{[V] \in G(k, n) \mid \dim (V \cap L_{n-k+i-a_i}) \geq i, i = 1, \ldots, k}, \]
called a \textit{Schubert cycle}. If we consider the subspaces
\[ 0 \subset (V \cap L_1) \subset \cdots \subset (V \cap L_{n-1}) \subset V, \]
then $\Sigma_{\ul{a}}$ corresponds to the locus where the $i$-th jump in dimension occurs at least $a_i$ steps before expected (where expectation means as late as possible). The Schubert cycle $\Sigma_{\ul{a}}$ has codimension $\sum a_i$. If we replace all the inequalities in the definition of a Schubert cycle by equalities, then we obtain \textit{Schubert cells}, which give a cell decomposition of $G(k, n)$.

Recall that we are interested in lines in $\P^3$, which are the same as $2$-planes in $\C^4$. To this end, we need to describe the cohomology of $G(2,4)$. There are six Schubert cycles, which are
\begin{align*}
    \Sigma_{0,0} = G(2,4) & & \Sigma_{1,0} = \qty{\dim (V \cap L_2) > 0} & & \Sigma_{2,0} = \qty{V \mid L_1 \subset V} \\
    \Sigma_{1,1} = \qty{V \mid V \subset L_3} & & \Sigma_{2,1} = \qty{V \mid L_1 \subset V \subset L_3} & & \Sigma_{2,2} = \qty{L_2}.
\end{align*}
By Poincar\'e duality, to each Schubert cycle $\Sigma_{a_1, a_2}$, there is a cohomology class $\sigma_{a_1, a_2} \in H^{2(a_1+a_2)}(G(2, 4), \Z)$. Note that $G(k, n)$ has only even-dimensional cells and that the Schubert cells form a complete CW decomposition of $G(k,n)$, so we have found an additive basis of $G(k, n)$. The ring structure is given by

\begin{thm}
    $H^*(G(2,4), \Z)$ is generated as an abelian group by the Schubert classes $\sigma_{a_1, a_2}$ with multiplicative relations being given by 
    \begin{align*}
        \sigma_1^2 = \sigma_{1,1} + \sigma_{2} && \sigma_1 \sigma_{1,1} = \sigma_1 \sigma_2 = \sigma_{2,1} \\
        \sigma_1 \sigma_{2,1} = \sigma_{2,2} && \sigma_{1,1}^2 = \sigma_2^2 = \sigma_{2,2} && \sigma_{1,1} \sigma_2 = 0. 
    \end{align*}
\end{thm}

\section{Chern classes}

We now define Chern classes of vector bundles, which are important invariants. If $X$ is a space and $\mc{E}$ is a (complex) vector bundle on $X$ of rank $r$, then there are classes $c_i(E) \in H^{2i}(X, \Z)$, called \textit{Chern classes}. If we define $c(\mc{E}) = 1 + c_1(\mc{E}) + \cdots + c_r(\mc{E})$, then the Chern classes are completely characterized by the following axioms:
\begin{enumerate}
    \item If $\mc{L}$ is a line bundle on $X$, then $c_1(\mc{L})$ is the Poincar\'e dual of the divisor class $(s_{0}) = (s_{\infty})$, where $s$ is any rational (or meromorphic) section of $\mc{L}$.
    \item If $s_0, \ldots, s_{r-i}$ are global sections of $\mc{E}$ and the zero locus $D$ of $s_1 \wedge \cdots \wedge s_{r-i}$ has codimension $i$, then $c_i(\mc{E})$ is the Poincar\'e dual of $D$.
    \item If $0 \to \mc{E} \to \mc{F} \to \mc{G} \to 0$ is exact, then $c(\mc{F}) = c(\mc{E}) c(\mc{G})$.
    \item If $\varphi \colon Y \to X$ is any map of spaces, then $\varphi^* (c(\mc{E})) = c(\varphi^* (\mc{E}))$.
\end{enumerate}
Chern classes can be constructed from the spaces $BU(n) = G(n, \infty)$ classifying rank $n$ vector bundles. It is a fact that $H^*(BU(n), \Z) = \Z[c_1, \ldots, c_n]$, where $\deg c_i = 2i$. Also, these classes $c_i$ are compatible with pullback under maps $BU(n-1) \to BU(n)$ induced by $U(n-1) \subset U(n)$. For a concrete example, we have $BU(1) = \C\P^{\infty}$ as a classifying space for line bundles and $H^*(\C\P^{\infty}, \Z) = \Z[c_1]$.

A useful result that allows us to do computations with Chern classes is the \textit{splitting principle}, which allows us to pretend that all vector bundles are direct sums of line bundles (this can be realized by pulling back to a flag bundle over $X$). In particular, we can pretend that $c(\mc{E}) = (1 + \alpha_1)(1+\alpha_2) \cdots (1+\alpha_r)$. We call the $\alpha_i$ the \textit{Chern roots} of $\mc{E}$.

For later, we will compute the Chern classes of the tautological bundle $\mc{S}$ on $G(2, 4)$. Recall that for any $G(k,n)$ we have a short exact sequence
\[ 0 \to \mc{S} \to \ul{\C^n} \to \mc{Q} \to 0. \]
Because of the surjection $\C^n \to \mc{Q}$, we can compute its Chern classes using global sections. If $e_1, \ldots, e_m \in \C^n$ are basis vectors, then the corresponding sections of $\mc{Q}$ will fail to be linearly independent at a point $[V] \in G(k,n)$ when the $v_i \in \C^n/V$ are linearly dependent. This is the same as $V \cap \ev{e_1, \ldots, e_m} = V \cap L_m \neq \emptyset$, and so the degeneracy locus is $\Sigma_{n-k-m+1}$. Therefore, we have
\[ c(\mc{Q}) = 1 + \sigma_1 + \cdots + \sigma_{n-k}. \]
Now we can restrict to the case of $G(2,4)$ and use the exact sequence axiom. Note that $G(2,4) \times \C$ is a trivial line bundle and has an everywhere nonzero section $G(2,4) \times \qty{1}$, so $c(\ul{\C}) = c(\ul{\C^n}) = 1$. By the axioms, we have $c(\mc{S}) = 1 + c_1(\mc{S}) + c_2(\mc{S})$, and so by the exact sequence axiom we have
\[ (1 + c_1(\mc{S}) + c_2(\mc{S}))(1 + \sigma_1 + \sigma_2) = 1. \]
This gives us the equations
\[ c_1 + \sigma_1 = 0 \qquad c_1\sigma_2 + c_2 \sigma_1 = 0 \qquad c_2 \sigma_2 = 0 \qquad c_1 \sigma_1 + c_2 + \sigma_2 = 0. \]
But now we see that $c_1(\mc{S}) = -\sigma_1$ and $c_2(\mc{S}) = \sigma_{1,1}$.

\section{Lines on a cubic surface}

We are now ready to count lines on a smooth cubic surface $X$. First, we will construct a closed subscheme $F_1(X)$, called the \textit{Fano scheme of lines}, of $G(2,4)$ parameterizing lines on $S$. Then, we will show that $F_1(X)$ has dimension $0$ and that $F_1(X)$ is the zero section of some rank $4$ vector bundle $\mc{E}$ on $G(2,4)$. Finally, we will compute $c_4(\mc{E})$.

Recall the inclusion $\mc{S} \subset \ul{\C^4}$. This induces a surjection $\ul{(\C^4)^*} \to \mc{S}^*$. Taking symmetric powers, we obtain a map
\[ \on{Sym}^d \ul{(\C^4)}^* \to \on{Sym}^d \mc{S}^* \]
restricting symmetric $d$-linear forms on $\C^4$ to symmetric $d$-linear forms on $\mc{S}$. Recall that $X \subset \P^4$ is cut out by a single homogeneous cubic polynomial $f(x_0, x_1, x_2, x_3)$, so in particular $f$ gives a global section of $\on{Sym}^3 \ul{(\C^4)}^*$. But now if $\sigma_{f}$ is the image of $f$ in $\on{Sym}^3 \mc{S}$, note that $\sigma_f$ vanishes at $V \in G(2,4)$ if and only if $f |_V \equiv 0$. In particular, this is the same the line $L \subset \P^4$ corresponding to $V \in \C^4$ being contained in $X$. Thus we can define $F_1(X)$ to be the vanishing locus of $\sigma_f$.

By a technical argument (that is beyond the scope of this lecture), $F_1(X)$ is a finite set of points whenever $X$ is smooth. Therefore we may compute its number of points as the degree of $c_4(\on{Sym}^4 \mc{S}^*)$. First, note that $c(\mc{S}^*) = 1 + \sigma_1 + \sigma_{1,1}$ (taking the dual of a line bundle takes $c_1$ to $-c_1$, and then use the splitting principle). Let $\alpha, \beta$ be the Chern roots of $\mc{S}^*$. These satisfy $\alpha + \beta = \sigma_1, \alpha \beta = \sigma_{1,1}$. Noting that if $\mc{L}_1, \mc{L}_2$ are line bundles, then
\[ \on{Sym}^d (\mc{L}_1 \oplus \mc{L}_2) = \bigoplus_{i=0}^d \mc{L}_1^{\otimes i} \otimes \mc{L}_2^{d-i}, \]
we can compute
\[ c(\on{Sym}^3 \mc{S}^*) = (1+3\alpha)(1+2\alpha+\beta)(1+\alpha+2\beta)(1+3\beta). \]
This implies that
\begin{align*}
    c_4(\on{Sym}^3 \mc{S}^*) &= 3\alpha(2\alpha+\beta)(\alpha+2\beta)3\beta \\
    &= 9\alpha\beta(2\alpha^2 + 5\alpha\beta + 2\beta^2) \\
    &= 9\alpha\beta (2(\alpha+\beta)^2 + \alpha\beta) \\
    &= 9\sigma_{1,1}(2\sigma_1^2 + \sigma_{1,1}) \\
    &= 9\sigma_{1,1}(3\sigma_{1,1}+2\sigma_2) \\
    &= 27\sigma_{2,2}.
\end{align*}

This has degree $27$, so there must be $27$ lines on $X$.


\end{document}
