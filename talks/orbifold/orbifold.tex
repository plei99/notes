\documentclass{amsart}
\usepackage{amsmath}
\usepackage{amssymb}
\usepackage{amsthm}
%\usepackage{MnSymbol}
\usepackage{bm}
\usepackage{accents}
\usepackage{mathtools}
\usepackage{tikz}
\usetikzlibrary{calc}
\usetikzlibrary{decorations.pathmorphing,shapes}
\usetikzlibrary{automata,positioning}
\usepackage{tikz-cd}
\usepackage{forest}
\usepackage{braket} 
\usepackage{listings}
\usepackage{mdframed}
\usepackage{verbatim}
\usepackage{physics2}
\usephysicsmodule{ab,ab.legacy,diagmat,xmat}
\usepackage{derivative}
\usepackage{fixdif}
\usepackage{stmaryrd}
% \usepackage{euscript} 
% \usepackage[mathcal]{eucal}
\usepackage{stackengine} 
%\usepackage{/home/patrickl/homework/macaulay2}

%font
\usepackage[sc]{mathpazo}
\usepackage{inconsolata}
\usepackage{microtype}
% \usepackage{fontspec} 
% \setmainfont{Tex Gyre Pagella}
\usepackage[OT1,euler-digits]{eulervm}
% \usepackage{euler-math} 
\usepackage[scaled=0.86]{berasans}
% \let\sffamilyold\sffamily
% \def\sffamily{\fontencoding{T1}\sffamilyold}
% \setmonofont{Inconsolatazi4}

%CS packages
\usepackage{algorithmicx}
\usepackage{algpseudocode}
\usepackage{algorithm}

% typeset and bib
\usepackage[english]{babel} 
% \usepackage[utf8]{inputenc} 
% \usepackage[T1]{fontenc}
% \usepackage[backend=biber,style=alphabetic,maxalphanames=4,maxnames=5,hyperref]{biblatex}
\usepackage[bookmarks, colorlinks, breaklinks]{hyperref} 
\hypersetup{linkcolor=blue,citecolor=magenta,filecolor=black,urlcolor=blue}
\usepackage{cleveref}
\usepackage{graphicx}
\graphicspath{{./}}


% other formatting packages
\usepackage{float}
\usepackage{booktabs}
\usepackage[shortlabels]{enumitem}
\setitemize{noitemsep}
\usepackage{csquotes}
%\usepackage{titlesec}
%\usepackage{titling}
%\usepackage{fancyhdr}
%\usepackage{lastpage}
% \usepackage{parskip}
\newlist{mydescription}{description}{1}
\setlist[mydescription]{style=nextline,
                        font=\bfseries,
                        % Tweak the next 4 options as needed:
                        labelindent=1cm, 
                        leftmargin =2cm,
                        rightmargin=1cm,
                        topsep     =1ex
                       }

\usepackage{lipsum}

% delimiters
\DeclarePairedDelimiter{\gen}{\langle}{\rangle}
\DeclarePairedDelimiter{\floor}{\lfloor}{\rfloor}
\DeclarePairedDelimiter{\ceil}{\lceil}{\rceil}


\newtheorem{thm}{Theorem}[section]
\newtheorem{cor}[thm]{Corollary}
\newtheorem{prop}[thm]{Proposition}
\newtheorem{lem}[thm]{Lemma}
\newtheorem{conj}[thm]{Conjecture}
\newtheorem{quest}[thm]{Question}
\newtheorem{claim}[thm]{Claim}

\theoremstyle{definition}
\newtheorem{defn}[thm]{Definition}
\newtheorem{defns}[thm]{Definitions}
\newtheorem{con}[thm]{Construction}
\newtheorem{exm}[thm]{Example}
\newtheorem{exms}[thm]{Examples}
\newtheorem{notn}[thm]{Notation}
\newtheorem{notns}[thm]{Notations}
\newtheorem{addm}[thm]{Addendum}
\newtheorem{exer}[thm]{Exercise}

\theoremstyle{remark}
\newtheorem{rmk}[thm]{Remark}
\newtheorem{rmks}[thm]{Remarks}
\newtheorem{warn}[thm]{Warning}
\newtheorem{sch}[thm]{Scholium}


% unnumbered theorems
\theoremstyle{plain}
\newtheorem*{thm*}{Theorem}
\newtheorem*{prop*}{Proposition}
\newtheorem*{lem*}{Lemma}
\newtheorem*{cor*}{Corollary}
\newtheorem*{conj*}{Conjecture}

% unnumbered definitions
\theoremstyle{definition}
\newtheorem*{defn*}{Definition}
\newtheorem*{exer*}{Exercise}
\newtheorem*{defns*}{Definitions}
\newtheorem*{con*}{Construction}
\newtheorem*{exm*}{Example}
\newtheorem*{exms*}{Examples}
\newtheorem*{notn*}{Notation}
\newtheorem*{notns*}{Notations}
\newtheorem*{addm*}{Addendum}


\theoremstyle{remark}
\newtheorem*{rmk*}{Remark}

% shortcuts
\newcommand{\Ima}{\mathrm{Im}}
\newcommand{\A}{\mathbb{A}}
\newcommand{\G}{\mathbb{G}}
\newcommand{\N}{\mathbb{N}}
\newcommand{\R}{\mathbb{R}}
\newcommand{\C}{\mathbb{C}}
\newcommand{\Z}{\mathbb{Z}}
\newcommand{\Q}{\mathbb{Q}}
\renewcommand{\k}{\Bbbk}
\renewcommand{\L}{\mathbb{L}}
\renewcommand{\P}{\mathbb{P}}
\newcommand{\M}{\mathcal{M}}
\newcommand{\Mbar}{\overline{\mathcal{M}}}
\newcommand{\g}{\mathfrak{g}}
\newcommand{\h}{\mathfrak{h}}
\newcommand{\n}{\mathfrak{n}}
\renewcommand{\b}{\mathfrak{b}}
\newcommand{\ep}{\varepsilon}
\newcommand*{\dt}[1]{%
   \accentset{\mbox{\Huge\bfseries .}}{#1}}
%\renewcommand{\abstractname}{Official Description}
\newcommand{\mc}[1]{\mathcal{#1}}
% \newcommand{\msc}[1]{\mathscr{#1}}
\newcommand{\T}{\mathbb{T}}
\newcommand{\mf}[1]{\mathfrak{#1}}
\newcommand{\mbf}[1]{\mathbf{#1}}
\newcommand{\mr}[1]{\mathrm{#1}}
\newcommand{\on}[1]{\operatorname{#1}}
\newcommand{\ms}[1]{\mathsf{#1}}
\newcommand{\mt}[1]{\mathtt{#1}}
\newcommand{\ol}[1]{\overline{#1}}
\newcommand{\ul}[1]{\underline{#1}}
\newcommand{\wt}[1]{\widetilde{#1}}
\newcommand{\wh}[1]{\widehat{#1}}
\renewcommand{\div}{\operatorname{div}}
\newcommand{\1}{\mathbf{1}}
\newcommand{\2}{\mathbf{2}}
\newcommand{\3}{\mathbf{3}}
\newcommand{\I}{\mathrm{I}}
\newcommand{\II}{\mr{I}\hspace{-1.3pt}\mr{I}}
\newcommand{\III}{\mr{I}\hspace{-1.3pt}\mr{I}\hspace{-1.3pt}\mr{I}}
\renewcommand{\v}{\mbf{v}}
\newcommand{\w}{\mbf{w}}
\newcommand{\bmu}{\bm{\mu}}
\newcommand{\pre}{\mr{pre}}
\newcommand{\vir}{\mr{vir}}
\newcommand{\fl}{\mr{fl}}
\newcommand{\ch}{\on{ch}}

\DeclareMathOperator{\Der}{Der}
\DeclareMathOperator{\Tor}{Tor}
\DeclareMathOperator{\Hom}{Hom}
\DeclareMathOperator{\End}{End}
\DeclareMathOperator{\Ext}{Ext}
\DeclareMathOperator{\ad}{ad}
\DeclareMathOperator{\Aut}{Aut}
\DeclareMathOperator{\Rad}{Rad}
\DeclareMathOperator{\Pic}{Pic}
\DeclareMathOperator{\NS}{NS}
\DeclareMathOperator{\supp}{supp}
\DeclareMathOperator{\Supp}{Supp}
\DeclareMathOperator{\depth}{depth}
\DeclareMathOperator{\sgn}{sgn}
\DeclareMathOperator{\spec}{Spec}
\DeclareMathOperator{\Spec}{Spec}
\DeclareMathOperator{\proj}{Proj}
\DeclareMathOperator{\Proj}{Proj}
\DeclareMathOperator{\ord}{ord}
\DeclareMathOperator{\Div}{Div}
\DeclareMathOperator{\Bl}{Bl}
\DeclareMathOperator{\coker}{coker}
\DeclareMathOperator{\ev}{ev}

% \addbibresource{../../notes/math.bib}

\title{Generalities on orbifold cohomology and toric DM stacks}
\author{Patrick Lei}
\date{April 4, 2024}

\begin{document}
    
\begin{abstract}
    I will explain various technicalities in Gromov-Witten theory for Deligne-Mumford stacks and how to construct toric Deligne-Mumford stacks from (extended) stacky fans.
\end{abstract}

\maketitle

\tableofcontents

\section{Orbifold Gromov-Witten theory}%
\label{sec:Orbifold Gromov-Witten theory}

Let $X$ be a smooth and separated Deligne-Mumford stack of finite type over $\C$. 

\begin{defn}
    The \textit{inertia stack} of $X$ is the fiber product in the diagram
    \begin{equation*}
    \begin{tikzcd}
        IX \ar{r} \ar{d} & X \ar{d}{\Delta} \\
        X \ar{r}{\Delta} & X \times X.
    \end{tikzcd}
    \end{equation*}
\end{defn}
More concretely, we may think about $\ab|X|$ as parameterizing pairs $(x,g)$, where $x\in X$ and $g \in \Aut(x)$. There is another description of $IX$ if $X$ lives over $\C$. In general, $IX$ is disconnected. We will write
\[ IX = \bigsqcup_{i \in I} X_i. \]
It also has an important morphism $\on{inv} \colon IX \to IX$ given by $(x,g) \mapsto (x,g^{-1})$.

\begin{defn}
    A morphism $X \to Y$ of algebraic stacks is \textit{representable} if for all schemes $S$ and morphisms $S \to Y$, the fiber product $X \times_S Y$ is an algebraic space.
\end{defn}

\begin{thm}
    Let
    \[ I_{\mu} X \coloneqq \bigsqcup_{r \geq 0} \Hom_{\mr{rep}}(B\mu_r, X) \]
    denote the stack of representable morphisms from classifying stacks of roots of unity to $X$ (the \textit{cyclotomic inertia stack}). Then $I_{\mu} X \simeq IX$.
\end{thm}

We need to make one more definition, which will appear as a degree shift on cohomology. Let $(x,g) \in X_i$. Because $\ab<g> \subset \Aut(x)$ is cyclic, there is a decomposition
\[ T_x X = \bigoplus_{0 \leq \ell < r_i} V_{\ell}, \]
where $V_{\ell}$ is the eigenspace with eigenvalue $e^{2\pi \sqrt{-1} \frac{\ell}{r_i}}$ and $r_i$ is the order of $g$. Then the function
\[ \on{age} \coloneqq \frac{1}{r_i} \sum_{0 \leq \ell < r_i} \ell \cdot \dim V_{\ell} \]
is constant on $X_i$, so we denote its value by $\on{age}(X_i)$.

Recall that by the Keel-Mori theorem, $X$ (which has finite inertia) has a coarse moduli space $\ab|X|$, which is an algebraic space satisfying two properties:
\begin{itemize}
    \item The morphism $\pi \colon X \to \ab|X|$ is bijective on $k$-points whenever $k$ is an algebraically closed field;
    \item $\ab|X|$ is initial for morphisms from $X$ to any algebraic space.
\end{itemize}
From now on, we will assume that $\ab|X|$ is quasiprojective, and in particular that it is a scheme.

\subsection{Moduli of stable maps}%
\label{sub:Moduli of stable maps}

\begin{defn}
    The moduli space of stable maps $\ol{\mc{M}}_{g,n}(X,\beta)$ parameterizes objects
    \begin{equation*}
    \begin{tikzcd}
        \ab(C, \ab\{\Sigma_i\}) \ar{r}{f} \ar{d} & X \\
        T,
    \end{tikzcd}
    \end{equation*}
    where
    \begin{enumerate}
        \item $C$ is a prestable balanced twisted curve of genus $g$. This means that $C$ has stacky structure only at nodes and marked points, and the nodes are formally locally $[(\C[x,y]/xy)/\mu_r]$, where $\mu_r$ acts by $\zeta(x,y) = (\zeta x, \zeta^{-1} y)$;
        \item $\Sigma_i \subset C$ is an \'etale cyclotomic gerbe over $T$ with a trivialization for all $i$;
        \item $f \colon C \to X$ is representable and the induced morphism between coarse moduli spaces is a stable map of degree $\beta$ with $n$ marked points.
    \end{enumerate}
\end{defn}

We see that $\ol{\mc{M}}_{g,n}(X,\beta)$ has evaluation maps $\on{ev}_i \colon \ol{\mc{M}}_{g,n}(X,\beta) \to IX$. It is also disconnected, with the connected components being indexed by components of $IX$. Let
\[ \ol{\mc{M}}_{g,n}(X,\beta,i_1,\ldots,i_n) \coloneqq \bigcap_{j=1}^n \on{ev}_j^{-1}(X_{i_j}). \]
Then
\[ \ol{\mc{M}}_{g,n}(X,\beta) = \bigsqcup_{i_1, \ldots, i_n} \ol{\mc{M}}_{g,n}(X,\beta,i_1,\ldots,i_n). \]
Each component has a virtual fundamental class
\[ [\ol{\mc{M}}_{g,n}(X,\beta,i_1, \ldots i_n)]^{\vir} \in H_*(\ol{\mc{M}}_{g,n}(X,\beta,i_1,\ldots, i_n), \Q) \]
of virtual dimension
\[ \int_{\beta} c_1(X) + (1-g)(\dim X - 3) + n - \sum_{j=1}^n \on{age}(X_{i_j}). \]
given by the relative perfect obstruction theory $(R \pi_* f^* TX)^{\vee}$, where $\pi \colon C \to \ol{\mc{M}}_{g,n}(X,\beta)$ is the universal curve, over the moduli stack $\mf{M}_{g,n}^{\mr{tw}}$ of prestable twisted curves. Because we chose to work with trivialized gerbe markings, we need to multiply the virtual fundamental class as follows. Note that the $j$-th marked point is 
\[ \Sigma_{j} \cong \ol{\mc{M}}_{g,n}(X,\beta,i_1,\ldots,i_n) \times B\mu_{r_{i_j}}. \] 
Here, if $x = [B\mu_r \to X] \in X_{i_j} \subset IX$, then $r_{i_j} = r$. Then set
\[  [\ol{\mc{M}}_{g,n}(X,\beta,i_1, \ldots i_n)]^{w} \coloneqq \ab(\prod_{j=1}^n r_{i_j}) [\ol{\mc{M}}_{g,n}(X,\beta,i_1, \ldots i_n)]^{\vir}. \]

Now consider the morphism $p \colon \ol{\mc{M}}_{g,n}(X,\beta) \to \ol{\mc{M}}_{g,n}(\ab|X|,\beta)$ given by taking the coarse moduli space. Let $C_{\ab|X|} \to \ol{\mc{M}}_{g,n}(\ab|X|,\beta)$ be the universal curve and $\sigma_{i,\ab|X|}$ be the marked points. Then the descendant classes\footnote{Most people call these $\ol{\psi}$, but I am extremely lazy.} are defined to be
\[ \psi_j \coloneqq p^* c_1(\sigma_j^* \omega_{C_{\ab|X|} / \ol{\mc{M}}_{g,n}(\ab|X|,\beta)}). \]

\subsection{Quantum cohomology}%
\label{sub:Quantum cohomology}

We are now able to define Gromov-Witten invariants. Let $\alpha_j \in H^{p_j}(X_{i_j}, \C)$. Then define
\[ \ab<\alpha_1 \psi^{k_1},\ldots,\alpha_n \psi^{k_n}>_{g,n,\beta}^X \coloneqq \int_{[ \ol{\mc{M}}_{g,n}(X,\beta,i_1,\ldots,i_n) ]^w} \prod_{j=1}^n \on{ev}_j^* \alpha_j \psi_j^{k_j}. \]
We are still able to form generating series $\mc{F}_g, J_X,\ldots$ as before, and the invariants satisfy the string, dilaton, and divisor equations (although we have to be careful that the marked point we delete is a scheme point), so the orbifold Gromov-Witten theory has a Lagrangian cone $\mc{L}_X \subset \mc{H}$.

The \textit{orbifold Poincar\'e pairing} is defined by the formula
\[ (\alpha,\beta) \coloneqq \int_{IX} \alpha \cup \on{inv}^* \beta, \]
where $\cup$ denotes the usual cup product. This is well-defined because of the formula
\[ \on{age}(X_i) + \on{age}(X_{\on{inv}(i)}) = \dim X - \dim X_i \]
when $X$ is proper. When $X$ is not proper, we will assume we are working equivariantly.
Now we may define the \textit{quantum product} by the formula
\[ (a \star_{\tau} b, c) \coloneqq \sum_{n,\beta} \frac{Q^{\beta}}{n!} \ab<a,b,c,\tau,\ldots,\tau>_{0,n+3,\beta}^X \]
for $a,b,c,\tau \in H^*(IX, \C)$. Restricting to the degree $0$ part and setting $\tau = 0$, we obtain the \textit{orbifold cup product}, which is given by
\[ (a \star b, c) = \ab<a,b,c>_{0,3,0}^X. \]
Denote $H^*_{\mr{CR}}(X) \coloneqq (H^*(IX, \C), \cup)$. Note that this product is graded for the grading $\deg(a) = p + 2 \on{age}(X_i)$ for $a \in H^p(X_i)$. Using the quantum product, we may define the quantum connection and its fundamental solution.

\section{Toric Deligne-Mumford stacks}%
\label{sec:Toric Deligne-Mumford stacks}

We will assume the reader is familiar with the fan presentation of a toric variety. If you are not, there are many references.

\begin{defn}
    An \textit{extended stacky fan} is a quadruple $\bm{\Sigma} = (N, \Sigma, \beta, S)$ of
    \begin{enumerate}
        \item A finitely generated abelian group $N$ of rank $n$;
        \item A rational simplicial fan $\Sigma$ in $N_{\R} = N \otimes \R$;
        \item A homomorphism $\beta \colon \Z^m \to N$. We will write $b_i = \beta(e_i) \in N$ for the image of the standard basis vector $e_i \in \Z^m$ and $\ol{b}_i$ for its image in $N_{\R}$;
        \item A subset $S \subset \ab\{1,\ldots,m\}$
    \end{enumerate}
    satisfying the following conditions:
    \begin{enumerate}
        \item The set $\Sigma(1)$ of $1$-dimensional cones is exactly the set $\ab\{\R_{\geq 0} \cdot \ol{b}_i \mid i \notin S \}$;
        \item For all $i \in S$, $\ol{b}_i \in \ab|\Sigma|$.
    \end{enumerate}
\end{defn}

We will now assume that $\ab|\Sigma|$ is convex and full-dimensional and, that there is a strictly convex piecewise linear function $f \colon \ab|\Sigma| \to \R$ which is linear on each cone, and that $\beta$ is surjective. From this data, we will now obtain a GIT presentation. Define $\L$ by the exact sequence
\[ 0 \to \L \xrightarrow{} \Z^m \xrightarrow{\beta} N \to 0. \]
Then define $K \coloneqq \L \otimes \C^{\times}$. Then define $D_i \in \L^{\vee}$ to be the image of the $i$-th standard basis vector in $(\Z^m)^{\vee}$ under the last arrow in the exact sequence
\[ 0 \to N^{\vee} \to (\Z^m)^{\vee} \to \L^{\vee} \]
Finally, set
\[ \mc{A}_{\omega} = \ab\{I \subset \ab\{1,\ldots,m\} \mid S \subset I, \sigma_{\ol{I}} \text{ is a cone of }\Sigma\}. \]
Choose a stability condition
\[ \omega \in C_{\omega} \coloneqq \bigcup_{I \in \mc{A}_{\omega}} \ab\{ \sum_{i \in I} a_i D_i \mid a_i \in \R_{>0}\}. \]
Then we define
\[ X_{\bm{\Sigma}} \coloneqq [(\C^m)^{s} / K]. \]

The ample cone is $C'_{\omega} \subset \L^{\vee}_{\R} / \sum_{i \in S} \R D_i \cong H^2(X_{\bm{\Sigma}}, \R)$, which is defined in the same way as $C_{\omega}$ after deleting $S$ from the extended stacky fan, and the cone of effective curve classes is its dual.

\subsection{Orbifold cohomology}%
\label{sub:Orbifold cohomology}

First, we will describe the equivariant cohomology of $X_{\bm{\Sigma}}$. Let $\mc{Q} = (\C^{\times})^m/K$. Then if $u_i$ is Poincar\'e dual to $(x_i = 0 \subset (\C^m)^s)/K$, we have
\[ H_{\mc{Q}}^*(X_{\bm{\Sigma}}, \C) = H_{\mc{Q}}^*(\mr{pt}, \C) [u_1, \ldots, u_m] / (\mf{I} + \mf{J}), \]
where
\begin{align*}
    \mf{I} &\coloneqq \ab< \chi - \sum_{i=1}^m \ab<\chi, b_i>u_i \mid \chi \in N^{\vee}_{\C}> \\
    \mf{J} &\coloneqq \ab<\prod_{i \notin I} u_i \mid I \notin \mc{A}_{\omega}>.
\end{align*}

There is a combinatorial description of the components of the inertia stack $IX_{\bm{\Sigma}}$. Because $X_{\bm{\Sigma}}$ is a global quotient, the components of the inertia stack correspond to elements $g \in K$ such that $( (\C^m)^s )^g$ is nonempty. Equivalently, if we define
\[ \mathbb{K} \coloneqq \ab\{f \in \L \otimes \Q \mid \ab\{ i \in \ab\{1,\ldots,m\} \mid D_i \cdot f \in \Z\} \in \mc{A}_{\omega}\}, \]
then the components of $IX_{\bm{\Sigma}}$ are in bijection with $\mathbb{K}/\L$. To give a description in terms of the fan, for any $\sigma \in \Sigma(n)$, define
\[ \on{Box}(\sigma) \coloneqq \ab\{v \in N \mid \ol{v} = \sum_{\rho_i \subseteq \sigma} a_i \ol{b}_i \mid 0 \leq a_i < 1\}\]
and then
\[ \on{Box}(\bm{\Sigma}) \coloneqq \bigcup_{\sigma \in \Sigma(n)} \on{Box}(\sigma). \]
Then there is a natural bijection $\mathbb{K}/\L \cong \on{Box}(\bm{\Sigma})$. For any $f \in \mathbb{K}/\L$, $X_f$ is a toric DM stack with $K, \L, \omega$ the same as for $X_{\omega}$ and characters $D_i$ for $i$ such that $D_i \cdot f \in \Z$. At the level of fans, this corresponds to killing the minimal cone of $\Sigma$ containing the corresponding $\ol{v}$.

We will now give the orbifold cohomology of $X_{\bm{\Sigma}}$. Define the \textit{deformed group ring} $\C[N]^{\bm{\Sigma}}$ as the vector space $\C[N]$ with product given by
\[ y^{c_1} \cdot y^{c_2} \coloneqq \begin{cases}
    y^{c_1 + c_2} & \text{ there exists } \sigma \in \Sigma \text{ such that } \ol{c}_1, \ol{c}_2 \in \sigma \\
    0 & \text{otherwise}.
\end{cases}
\]
Then there is an isomorphism of rings
\[ H^*_{\mr{CR}}(X_{\bm{\Sigma}}) \cong \frac{\C[N]^{\bm{\Sigma}}}{\ab<\sum_{i \notin S} \chi(b_i) y^{b_i} \mid \chi \in N^{\vee}>}. \]
\begin{rmk}
    This result also works in families over a base $B$, where $\C^m$ is replaced by a direct sum of $m$ line bundles on $B$. Then we need to add a $c_1(L_{\chi})$ to the relations and obtain
    \[ H^*_{\mr{CR}}(X^B_{\bm{\Sigma}}) \coloneqq \frac{H^*(B)[N]^{\bm{\Sigma}}}{\ab<c_1(L_{\chi}) + \sum_{i \notin S} \chi(b_i) y^{b_i} \mid \chi \in N^{\vee}>}. \]
\end{rmk}

\section{Gamma-integral structure}%
\label{sec:Gamma-integral structure}

Let $IX = \bigsqcup_{v \in \ms{B}} X_v$ and $q_v \colon X_v \to X$ be the restriction of $IX \to X$. Let $E$ be a $T$-equivariant vector bundle on $X$. Recall that $v$ corresponds to some $g_v \in K$, so we obtain an eigenbundle decomposition
\[ q_v^* E = \bigoplus_{0 \leq f < 1} E_{v,f}, \]
where $E_{v,f}$ is the subbundle where $g_v$ acts by $e^{2 \pi i f}$. We now define the orbifold Chern character to be
\[ \wt{\ch}(E) = \bigoplus_{v \in\ms{B}} \sum_{0 \leq f < 1} e^{2\pi i f} \ch(E_{v,f}). \]

Now let $\delta_{v,f,i}$ be the Chern roots of $E_{v,f}$. We define the orbifold Todd class to be
\[ \wt{\mr{Td}}(E) \coloneqq \bigoplus_{v \in \ms{B}} \ab(\prod_{0 < f < 1} \prod_{i} \frac{1}{1-e^{-2\pi i f - \delta_{v,f,i}}}) \prod_i \frac{\delta_{v,0,i}}{1-e^{-\delta_{v,0,i}}}. \]
The $\wh{\Gamma}$-class should be a square root of this and is defined by
\[ \wh{\Gamma}(E) = \bigoplus_{v \in \ms{B}} \prod_{0 \leq f < 1} \prod_i \Gamma(1-f+\delta_{v,f,i}), \]
where we expand $\Gamma$ around $1-f$. The reflection formula for the $\Gamma$-function implies that the $X_v$-component of $\wh{\Gamma}(E^{\vee}) \cup \wh{\Gamma}(E)$ is given by
\[ [\wh{\Gamma}(E^{\vee}) \cup \wh{\Gamma}(E)]_v = (2\pi i)^{\on{rk} (q_v^* E)^{\mr{mov}}} \ab[e^{-\pi i ( \on{age}(q^* E) + c_1(q^* E) )} (2\pi i)^{\frac{\deg_0}{2}} \wt{\mr{Td}}(E)]_{\on{inv}(v)}. \]
Here, $\deg_0$ is the grading operator given by the degree without age shifting.

\begin{defn}
    Define the \textit{$K$-group framing} $\mf{s} \colon K_T(X) \to H^*_{\mr{CR},T}(X) \otimes_{R_T} R_T[\log z] \llparenthesis z^{-\frac{1}{k}} \rrparenthesis \llbracket Q,\tau \rrbracket$ by the formula
    \[ \mf{s}(E)(\tau, z) \coloneqq \frac{1}{(2\pi)^{\frac{\dim X}{2}}} L(\tau, z) z^{-\mu} z^{\rho} \wh{\Gamma}_X \cup (2\pi i)^{\frac{\deg_0}{2}} \on{inv}^* \wt{\ch}(E), \]
    where $L(\tau, z)$ is the fundamental solution to the quantum connection, $\mu$ is the usual grading operator given by $\frac{1}{2} (\deg - \dim X)$ on homogeneous elements, and $\rho = c_1(TX) \in H^2(X)$.
\end{defn}

\begin{prop}
    Define the equivariant Euler pairing by
    \[ \chi(E,F) \coloneqq \sum_{i} (-1)^i \ch^T(\Ext^i(E,F)) \]
    and the modified version $\chi_z(E,F)$ by replacing the equivariant parameters $\lambda_j$ by $\frac{2\pi i \lambda_j}{z}$. Then
    \[ (\mf{s}(E)(\tau, e^{-i\pi}z), \mf{s}(F)(\tau, z)) = \chi_z(E,F). \]
\end{prop}

\begin{rmk}
    Everything we have discussed so far makes sense for toric DM stacks after specializing $Q=1$.
\end{rmk}




\end{document}

%%% Local Variables:
%%% mode: latex
%%% TeX-master: t
%%% End:
