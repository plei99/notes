\documentclass{amsart}
\usepackage{amsmath}
\usepackage{amssymb}
\usepackage{amsthm}
%\usepackage{MnSymbol}
\usepackage{bm}
\usepackage{accents}
\usepackage{mathtools}
\usepackage{tikz}
\usetikzlibrary{calc}
\usetikzlibrary{decorations.pathmorphing,shapes}
\usetikzlibrary{automata,positioning}
\usepackage{tikz-cd}
\usepackage{forest}
\usepackage{braket} 
\usepackage{listings}
\usepackage{mdframed}
\usepackage{verbatim}
\usepackage{physics}
\usepackage{stmaryrd}
\usepackage{mathrsfs} 
\usepackage{stackengine} 
%\usepackage{/home/patrickl/homework/macaulay2}

%font
\usepackage[sc]{mathpazo}
\usepackage{eulervm}
\usepackage[scaled=0.86]{berasans}
\usepackage{inconsolata}
\usepackage{microtype}

%CS packages
\usepackage{algorithmicx}
\usepackage{algpseudocode}
\usepackage{algorithm}

% typeset and bib
\usepackage[english]{babel} 
\usepackage[utf8]{inputenc} 
\usepackage[T1]{fontenc}
% \usepackage[backend=biber, style=alphabetic]{biblatex}
\usepackage[bookmarks, colorlinks, breaklinks]{hyperref} 
\hypersetup{linkcolor=blue,citecolor=magenta,filecolor=black,urlcolor=blue}
\usepackage{graphicx}
\graphicspath{{./}}

% other formatting packages
\usepackage{float}
\usepackage{booktabs}
\usepackage[shortlabels]{enumitem}
\setitemize{noitemsep}
\usepackage{csquotes}
%\usepackage{titlesec}
%\usepackage{titling}
%\usepackage{fancyhdr}
%\usepackage{lastpage}
\usepackage{parskip}

\usepackage{lipsum}

% delimiters
\DeclarePairedDelimiter{\gen}{\langle}{\rangle}
\DeclarePairedDelimiter{\floor}{\lfloor}{\rfloor}
\DeclarePairedDelimiter{\ceil}{\lceil}{\rceil}


\newtheorem{thm}{Theorem}[section]
\newtheorem{cor}[thm]{Corollary}
\newtheorem{prop}[thm]{Proposition}
\newtheorem{lem}[thm]{Lemma}
\newtheorem{conj}[thm]{Conjecture}
\newtheorem{quest}[thm]{Question}

\theoremstyle{definition}
\newtheorem{defn}[thm]{Definition}
\newtheorem{defns}[thm]{Definitions}
\newtheorem{con}[thm]{Construction}
\newtheorem{exm}[thm]{Example}
\newtheorem{exms}[thm]{Examples}
\newtheorem{notn}[thm]{Notation}
\newtheorem{notns}[thm]{Notations}
\newtheorem{addm}[thm]{Addendum}
\newtheorem{exer}[thm]{Exercise}

\theoremstyle{remark}
\newtheorem{rmk}[thm]{Remark}
\newtheorem{rmks}[thm]{Remarks}
\newtheorem{warn}[thm]{Warning}
\newtheorem{sch}[thm]{Scholium}


% unnumbered theorems
\theoremstyle{plain}
\newtheorem*{thm*}{Theorem}
\newtheorem*{prop*}{Proposition}
\newtheorem*{lem*}{Lemma}
\newtheorem*{cor*}{Corollary}
\newtheorem*{conj*}{Conjecture}

% unnumbered definitions
\theoremstyle{definition}
\newtheorem*{defn*}{Definition}
\newtheorem*{exer*}{Exercise}
\newtheorem*{defns*}{Definitions}
\newtheorem*{con*}{Construction}
\newtheorem*{exm*}{Example}
\newtheorem*{exms*}{Examples}
\newtheorem*{notn*}{Notation}
\newtheorem*{notns*}{Notations}
\newtheorem*{addm*}{Addendum}


\theoremstyle{remark}
\newtheorem*{rmk*}{Remark}

% shortcuts
\newcommand{\Ima}{\mathrm{Im}}
\newcommand{\A}{\mathbb{A}}
\newcommand{\G}{\mathbb{G}}
\newcommand{\N}{\mathbb{N}}
\newcommand{\R}{\mathbb{R}}
\newcommand{\C}{\mathbb{C}}
\newcommand{\Z}{\mathbb{Z}}
\newcommand{\Q}{\mathbb{Q}}
\renewcommand{\k}{\Bbbk}
\renewcommand{\L}{\mathbb{L}}
\renewcommand{\P}{\mathbb{P}}
\newcommand{\M}{\overline{M}}
\newcommand{\g}{\mathfrak{g}}
\newcommand{\h}{\mathfrak{h}}
\newcommand{\n}{\mathfrak{n}}
\renewcommand{\b}{\mathfrak{b}}
\newcommand{\ep}{\varepsilon}
\newcommand*{\dt}[1]{%
   \accentset{\mbox{\Huge\bfseries .}}{#1}}
%\renewcommand{\abstractname}{Official Description}
\newcommand{\mc}[1]{\mathcal{#1}}
% \newcommand{\msc}[1]{\mathscr{#1}}
\newcommand{\T}{\mathbb{T}}
\newcommand{\mf}[1]{\mathfrak{#1}}
\newcommand{\mbf}[1]{\mathbf{#1}}
\newcommand{\mr}[1]{\mathrm{#1}}
\newcommand{\on}[1]{\operatorname{#1}}
\newcommand{\ms}[1]{\mathsf{#1}}
\newcommand{\mt}[1]{\mathtt{#1}}
\newcommand{\ol}[1]{\overline{#1}}
\newcommand{\ul}[1]{\underline{#1}}
\newcommand{\wt}[1]{\widetilde{#1}}
\newcommand{\wh}[1]{\widehat{#1}}
\renewcommand{\div}{\operatorname{div}}
\newcommand{\1}{\mathbf{1}}
\newcommand{\2}{\mathbf{2}}
\newcommand{\3}{\mathbf{3}}
\newcommand{\I}{\mathrm{I}}
\newcommand{\II}{\mr{I}\hspace{-1.3pt}\mr{I}}
\newcommand{\III}{\mr{I}\hspace{-1.3pt}\mr{I}\hspace{-1.3pt}\mr{I}}
\renewcommand{\v}{\mbf{v}}
\newcommand{\w}{\mbf{w}}

\DeclareMathOperator{\Der}{Der}
\DeclareMathOperator{\Tor}{Tor}
\DeclareMathOperator{\Hom}{Hom}
\DeclareMathOperator{\End}{End}
\DeclareMathOperator{\Ext}{Ext}
\DeclareMathOperator{\ad}{ad}
\DeclareMathOperator{\Aut}{Aut}
\DeclareMathOperator{\Rad}{Rad}
\DeclareMathOperator{\Pic}{Pic}
\DeclareMathOperator{\NS}{NS}
\DeclareMathOperator{\supp}{supp}
\DeclareMathOperator{\Supp}{Supp}
\DeclareMathOperator{\depth}{depth}
\DeclareMathOperator{\sgn}{sgn}
\DeclareMathOperator{\spec}{Spec}
\DeclareMathOperator{\Spec}{Spec}
\DeclareMathOperator{\proj}{Proj}
\DeclareMathOperator{\Proj}{Proj}
\DeclareMathOperator{\ord}{ord}
\DeclareMathOperator{\Div}{Div}
\DeclareMathOperator{\Bl}{Bl}
\DeclareMathOperator{\coker}{coker}

\title{A Kleiman criterion for stack quotients}
\author{Patrick Lei}
\date{February 1, 2023}

\begin{document}
    
\maketitle

\begin{abstract}
    This is a talk about~\href{https://arxiv.org/pdf/2211.09218.pdf}{\texttt{2211.09218}} in the \href{https://www.math.columbia.edu/~tpad/preprintS2023.html}{preprint seminar} at Columbia University. 
\end{abstract}

\section{Introduction}

\subsection{Classical Kleiman criterion} The Kleiman criterion determines when a divisor on a projective variety $X$ is ample in terms of its intersections with curves in $X$.

\begin{thm}[Kleiman]
    Let $X$ be a projective variety over $\C$. Then let $\mr{NE}(X)$ be the cone spanned by effective curve classes in $N_1(X)_{\R}$. A divisor $D$ on $X$ is ample if and only if
    \[ D \cdot \gamma > 0 \]
    for all $\gamma \in \ol{ \mr{NE}(X) } \setminus 0$.
\end{thm}

This theorem says that the ample cone is the interior of the the nef cone. If $\gamma$ is represented by a smooth curve $C$, then $\mc{O}_X(D)$ is ample on $C$, so its higher cohomology vanishes, and then the Riemann-Roch formula reduces to
\[ h^0(\mc{O}_C(D)) = \deg_C D - g(C) + 1 > 0, \]
and so $D$ has positive degree on $C$.

\subsection{Variation of GIT}

Let $G$ be a reductive group acting on $X$. If we want to construct a ``quotient'' of $X$ by $G$ as a scheme, we must fix a $G$-equivariant line bundle and consider the GIT quotient
\[ X \sslash_L G \coloneqq X^{\mr{ss}}(L) \sslash G. \]
Clearly if $L, L'$ have the same semistable locus, then the GIT quotient (schemes) are isomorphic, but the converse is not necessarily true.

\begin{exm}
    Consider the action of $(\C^{\times})^2$ on $\C^3$ by
    \[ t \cdot x = (t_1^2 x_1, t_1 t_2 x_2, t_2 x_3). \]
    Now let $\theta_1 = (4,2)$ and $\theta_2 = (2,4)$. In the first case, the semistable locus has either $(x,y) \neq (0,0)$ or $(x,z) \neq (0,0)$, so we remove the $y$ and $z$ axes and obtain the GIT quotient stack
    \[ [\C^3 \sslash_{\theta_1} (\C^{\times})^2] = [\P^1 / \mu_2]. \]
    In the second case, the semistable locus has either $(x,z) \neq (0,0)$ or $(y, z) \neq (0,0)$, so we remove the $x$ and $y$ axes and obtain the GIT quotient stack
    \[ [\C^3 \sslash_{\theta_2} (\C^{\times})^2] = \P(2,1). \]
    Both stacks have coarse moduli space $\P^1$, so the GIT quotient varieties $\C^3 \sslash_{\theta_i} (\C^{\times})^2$ are isomorphic for $i = 1,2$.
\end{exm}


We will now establish some notation. Recall that there is an equivariant first Chern class $c_1^G$ which sits inside the exact sequence
\[ 0 \to \Pic_0(X) \to \Pic^G(X) \xrightarrow{c_1^G} H^2_G(X, \Z) \]
and denote $\NS^G(X) \coloneqq \Pic^G(X)/\Pic_0(X)$. Note that if $L$ and $L'$ determine the same class in $\NS^G(X)$, then they have the same semistable locus, which can be seen using localization.

Denote the ample cone of $G$-linearized ample line bundles by $\mr{Amp}^G(X) \subset \NS^G(X)_{\Q}$ and then denote by $C^G(X)$ the cone containing $G$-linearized ample line bundles with $X^{\mr{ss}}(L)$ nonempty. We will call these classes \textit{$G$-ample}. Now, for $L \in C^G(X)$, define
\[ C(L) \coloneqq \qty{L' \in C^G(X) \mid X^{\mr{ss}}(L) \subset X^{\mr{ss}}(L')} \]
and its relative interior (due to Ressayre)
\[ C^{\circ}(L) \coloneqq \qty{L' \in C^G(X) \mid X^{\mr{ss}}(L) = X^{\mr{ss}}(L)}. \]

\section{Quasimaps}

Let $L \in \Pic^G(X)$ be ample.
\begin{defn}
    An \textit{$L$-stable quasimap} from a smooth curve $C$ to $[X/G]$ is a morphism
    \[ f \colon C \to [X/G] \]
    such that $f^{-1}([X^{\mr{ss}}/L])$ is dense.
\end{defn}

Given such a quasimap, for any $G$-equivariant line bundle $N$ on $X$, we can consider the degree
\[ \deg f^*([N/G]), \]
and the map
\[ (N \mapsto \deg f^*([N/G])) \in \Hom(\NS^G(X), \Z) \]
is called the \textit{degree} of $f$. We will call the cone generated by all $\beta \in \Hom(\NS^G(X), \Z)$ that can be realized as degrees of $L$-stable quasimaps
\[ \mr{NE}(L) \subset \Hom(\NS^G(X), \Q). \]

\begin{prop}
    If $f \colon C \to [X/G]$ is an $L$-stable quasimap, then $\deg f^* [L/G] \geq 0$.
\end{prop}

\begin{proof}
    We can assume that $C$ is connected. Then there exists $c \in C$ mapping to $[X^{\mr{ss}}(L)/G]$. We can then lift $f(c)$ to $x \in X^{\mr{ss}}(L)$. But now by definition of semistability, there exists some $m > 0$ and $s \in \Gamma(X, L^m)^G$ such that $s(x) \neq 0$. But now 
    \[ f^* [L^m/G] \]
    has a nonzero section $f^*(s)$, and thus it has non-negative degree on $C$.
\end{proof}

\begin{cor}
    $C(L) \subset \mr{NE}(L)^{\vee}$.
\end{cor}


\section{Main result}

\subsection{For projective $X$}

Now we want to relate the weights $\mu^L(x, \lambda)$ to the degrees of quasimaps. Fix $\lambda \colon \C^{\times} \to G$ and a semistable point $x \in X^{\mr{ss}}(L)$. Because $X$ is proper, there exists a morphism $\ol{\lambda} \colon \C \to X$ extending $t \mapsto \lambda(t) x$. Now define the map
\[ \wt{\phi}_{\lambda, x} \colon \C^2 \setminus \qty{0} \to X \qquad (s,t) \mapsto \ol{\lambda}(t). \]
This is clearly $\C^{\times}$-equivariant (with respect to the scaling action on $\C^2$ and the action of $\lambda$ on $X$), and so we obtain a stable quasimap
\[ \phi_{\lambda, x} \colon \P^1 \to [X/G]. \]

\begin{lem}
    For any $N \in \Pic^G(X)$, 
    \[ \deg \phi^*_{\lambda, x}(N) = \mu^N(x, \lambda). \]
\end{lem}

\begin{proof}
    Note that $\phi_{\lambda, x}$ factors through the projection onto the second factor $\pi_2 \colon \C^2 \setminus 0 \to \C$, and then we simply need to compute the weight of the action of $\C^{\times}$ on $N$ at the origin of $[\C / \C^{\times}]$.
\end{proof}

\begin{prop}
    Suppose $L, N \in \mr{Amp}^G(X)$ such that $N \notin C(L)$. Then there exists an $L$-stable quasimap $f \colon C \to [X/G]$ such that $\deg f^* [N/G] < 0$.
\end{prop}

\begin{proof}
    Consider the morphism $\phi_{\lambda, x}$ for some $x \in X^{\mr{ss}}(L) \setminus X^{\mr{ss}}(N)$.
\end{proof}

Now we may state the main result, which is proven using the preceeding discussion.

\begin{thm}
    Let $L$ be a $G$-ample line bundle on $X$. Then
    \[ C^{\circ}(L) = \mr{relint}(\mr{NE}(L)^{\vee}) \cap \mr{Amp}^G(X). \]
\end{thm}

\subsection{Quotients of vector spaces}

We can extend these results to quotients of vector spaces by embedding $V \subset \P(V \oplus \C)$, where the extra copy of $\C$ carries the trivial representation of $G$. Then for a character $\theta \in \Hom(G, \C^{\times})$, denote its GIT equivalence class by $A(\theta)$. The main result in the context of quotients of vector spaces is stated below.

\begin{prop}
    Suppose that $(\on{Sym}^{\bullet} V^{\vee})^G = \C$. Then for any $\theta \in \Hom(G, \C^{\times})$, 
    \[ A(\theta) = \on{relint}(\on{NE}(\theta)^{\vee}). \]
\end{prop}



\end{document}
