\documentclass{amsart}
\usepackage{amsmath}
\usepackage{amssymb}
\usepackage{amsthm}
%\usepackage{MnSymbol}
\usepackage{bm}
\usepackage{accents}
\usepackage{mathtools}
\usepackage{tikz}
\usetikzlibrary{calc}
\usetikzlibrary{decorations.pathmorphing,shapes}
\usetikzlibrary{automata,positioning}
\usepackage{tikz-cd}
\usepackage{forest}
\usepackage{braket} 
\usepackage{listings}
\usepackage{mdframed}
\usepackage{verbatim}
\usepackage{physics2}
\usephysicsmodule{ab, ab.braket, xmat, diagmat}
\usepackage{derivative}
\usepackage{stmaryrd}
\usepackage{mathrsfs} 
\usepackage{stackengine} 

%\usepackage{/home/patrickl/homework/macaulay2}

%font
\usepackage[sc]{mathpazo}
\usepackage{eulervm}
\usepackage[scaled=0.86]{berasans}
\usepackage{inconsolata}
\usepackage{microtype}

%CS packages
\usepackage{algorithmicx}
\usepackage{algpseudocode}
\usepackage{algorithm}

% typeset and bib
\usepackage[english]{babel} 
\usepackage[utf8]{inputenc} 
\usepackage[T1]{fontenc}
%\usepackage[backend=biber, style=alphabetic]{biblatex}
\usepackage[bookmarks, colorlinks, breaklinks]{hyperref} 
\hypersetup{linkcolor=blue,citecolor=black,filecolor=black,urlcolor=blue}
\usepackage{graphicx}
\graphicspath{{./}}
\usepackage{cleveref}


% other formatting packages
\usepackage{float}
\usepackage{booktabs}
\usepackage[shortlabels]{enumitem}
\usepackage{csquotes}
%\usepackage{titlesec}
%\usepackage{titling}
%\usepackage{fancyhdr}
%\usepackage{lastpage}
\usepackage{parskip}

\usepackage{lipsum}

% delimiters
\DeclarePairedDelimiter{\gen}{\langle}{\rangle}
\DeclarePairedDelimiter{\floor}{\lfloor}{\rfloor}
\DeclarePairedDelimiter{\ceil}{\lceil}{\rceil}


\newtheorem{thm}{Theorem}[section]
\newtheorem{cor}[thm]{Corollary}
\newtheorem{prop}[thm]{Proposition}
\newtheorem{lem}[thm]{Lemma}
\newtheorem{conj}[thm]{Conjecture}
\newtheorem{quest}[thm]{Question}

\theoremstyle{definition}
\newtheorem{defn}[thm]{Definition}
\newtheorem{defns}[thm]{Definitions}
\newtheorem{con}[thm]{Construction}
\newtheorem{exm}[thm]{Example}
\newtheorem{exms}[thm]{Examples}
\newtheorem{notn}[thm]{Notation}
\newtheorem{notns}[thm]{Notations}
\newtheorem{addm}[thm]{Addendum}
\newtheorem{exer}[thm]{Exercise}

\theoremstyle{remark}
\newtheorem{rmk}[thm]{Remark}
\newtheorem{rmks}[thm]{Remarks}
\newtheorem{warn}[thm]{Warning}
\newtheorem{sch}[thm]{Scholium}


% unnumbered theorems
\theoremstyle{plain}
\newtheorem*{thm*}{Theorem}
\newtheorem*{prop*}{Proposition}
\newtheorem*{lem*}{Lemma}
\newtheorem*{cor*}{Corollary}
\newtheorem*{conj*}{Conjecture}

% unnumbered definitions
\theoremstyle{definition}
\newtheorem*{defn*}{Definition}
\newtheorem*{exer*}{Exercise}
\newtheorem*{defns*}{Definitions}
\newtheorem*{con*}{Construction}
\newtheorem*{exm*}{Example}
\newtheorem*{exms*}{Examples}
\newtheorem*{notn*}{Notation}
\newtheorem*{notns*}{Notations}
\newtheorem*{addm*}{Addendum}


\theoremstyle{remark}
\newtheorem*{rmk*}{Remark}

% shortcuts
\newcommand{\Ima}{\mathrm{Im}}
\newcommand{\A}{\mathbb{A}}
\newcommand{\G}{\mathbb{G}}
\newcommand{\N}{\mathbb{N}}
\newcommand{\R}{\mathbb{R}}
\newcommand{\C}{\mathbb{C}}
\newcommand{\Z}{\mathbb{Z}}
\newcommand{\Q}{\mathbb{Q}}
\renewcommand{\k}{\Bbbk}
\renewcommand{\L}{\mathbb{L}}
\renewcommand{\P}{\mathbb{P}}
\newcommand{\M}{\overline{M}}
\newcommand{\g}{\mathfrak{g}}
\newcommand{\h}{\mathfrak{h}}
\newcommand{\n}{\mathfrak{n}}
\renewcommand{\b}{\mathfrak{b}}
\newcommand{\ep}{\varepsilon}
\newcommand*{\dt}[1]{%
   \accentset{\mbox{\Huge\bfseries .}}{#1}}
%\renewcommand{\abstractname}{Official Description}
\newcommand{\mc}[1]{\mathcal{#1}}
% \newcommand{\msc}[1]{\mathscr{#1}}
\newcommand{\T}{\mathbb{T}}
\newcommand{\mf}[1]{\mathfrak{#1}}
\newcommand{\mr}[1]{\mathrm{#1}}
\newcommand{\on}[1]{\operatorname{#1}}
\newcommand{\ms}[1]{\mathsf{#1}}
\newcommand{\mt}[1]{\mathtt{#1}}
\newcommand{\ol}[1]{\overline{#1}}
\newcommand{\ul}[1]{\underline{#1}}
\newcommand{\wt}[1]{\widetilde{#1}}
\newcommand{\wh}[1]{\widehat{#1}}
\renewcommand{\div}{\operatorname{div}}
\newcommand{\1}{\mathbf{1}}
\newcommand{\2}{\mathbf{2}}
\newcommand{\3}{\mathbf{3}}
\newcommand{\I}{\mathrm{I}}
\newcommand{\II}{\mr{I}\hspace{-1.3pt}\mr{I}}
\newcommand{\III}{\mr{I}\hspace{-1.3pt}\mr{I}\hspace{-1.3pt}\mr{I}}

\DeclareMathOperator{\Der}{Der}
\DeclareMathOperator{\Tor}{Tor}
\DeclareMathOperator{\Hom}{Hom}
\DeclareMathOperator{\End}{End}
\DeclareMathOperator{\Ext}{Ext}
\DeclareMathOperator{\ad}{ad}
\DeclareMathOperator{\Aut}{Aut}
\DeclareMathOperator{\Rad}{Rad}
\DeclareMathOperator{\Pic}{Pic}
\DeclareMathOperator{\supp}{supp}
\DeclareMathOperator{\Supp}{Supp}
\DeclareMathOperator{\depth}{depth}
\DeclareMathOperator{\sgn}{sgn}
\DeclareMathOperator{\spec}{Spec}
\DeclareMathOperator{\Spec}{Spec}
\DeclareMathOperator{\proj}{Proj}
\DeclareMathOperator{\Proj}{Proj}
\DeclareMathOperator{\ord}{ord}
\DeclareMathOperator{\Div}{Div}
\DeclareMathOperator{\Bl}{Bl}
\DeclareMathOperator{\coker}{coker}

\title{New techniques in resolution of singularities}
\author{Talk by Michael Temkin, notes by Patrick Lei}
\date{November 21, 2025}

\begin{document}
    
\begin{abstract}
    These are notes taken during a talk by Michael Temkin at AGNES on November 21, 2025. His abstract is reproduced below.

    Since Hironaka's famous resolution of singularities in characteristic zero in 1964, it took about 40 years of intensive work of many mathematicians to simplify the method, describe it using conceptual tools and establish its functoriality. However, one point remained quite mysterious: despite different descriptions of the basic resolution algorithm, it was essentially unique. Was it a necessity or a drawback of the fact that all subsequent methods relied on Hironaka's ideas essentially?

The situation changed in the last decade, when logarithmic, weighted and foliated analogues and generalizations were discovered in works of Abramovich-Temkin-Wlodarzcyk, McQuillan, Quek, Abramovich-Temkin-Wlodarzcyk-Belotto and others. At this stage we can already try to figure out general ideas and principles shared by all these methods and the picture is quite surprising -- it seems that each method is quite determined by its basic setting consisting of the class of geometric objects and basic blowings up one works with. In particular, the classical method is probably the only natural resolution (via principalization) method obtained by blowing up smooth centers in the ambient manifold.

In my talk I'll describe the settings and the methods on a very general level and will add some details about the classical algorithm, the logarithmic one and the simplest dream (or weighted) method, which has no memory and improves the singularity invariant by each weighted blowing up.
\end{abstract}
\maketitle

\section{Introduction}%
\label{sec:Introduction}

Let $X$ be a (singular) variety, analytic space, or scheme. Our goal is to produce a morphism $X_{\mr{res}} \to X$ which is proper, birational, and generically an isomorphism, such that $X_{\mr{res}}$ is smooth. We might also want to do this logarithmically or require some level of functoriality. We will work in characteristic zero. There are some results in positive characteristic, but we are still very far from resolution of singularities for general varieties.

\begin{exm}
    Some examples include the resolution of a nodal cubic by $\P^1$, a cuspidal cubic by $\P^1$, or the cone over a conic by the total space of $\mc{O}_{\P^1}(-2)$. 
\end{exm}

\subsection{History}%
\label{sub:History}

This was first considered for surfaces by the Italian school and was completed rigorously by Zariski. The next major result was by Zariski for threefolds, and then Hironaka in 1964 proved resolution of singularities in characteristic zero in any dimension. Giraux introduced the notion of maximal contact while Hironaka introduced idealistic exponents. Later, Bierstone-Milman and Villamayor made this notion canonical. In 2003, Wlodarzcyk improved canonical to functorial, and in 2006 Kollar wrote a book and gave a graduate course with a simplified proof.

\subsection{Functoriality}%
\label{sub:Functoriality}

We want our resolution to be functorial with respect to smooth morphisms. In particular, if $Y \to X$ is a smooth morphism, then we want the diagram
\begin{equation*}
\begin{tikzcd}
    Y_{\mr{res}} \ar{r} \ar{d} & Y \ar{d} \\
    X_{\mr{res}} \ar{r} & X
\end{tikzcd}
\end{equation*}
to be cartesian. One advantage of this is that we can work locally on $X$. Hironaka's original proof was existential and therefore not functorial, so we want to find a method which is both effective and functorial.

One major advance was due to Abramovich-Temkin-Wlodarcyk, who constructed resolutions of morphisms in 2017. This requires logarithmic geometry and we need toroidal singularities, but it did give a more general notion of functoriality. This also required us to work with stacks, and a method using stacks was developed independently by Abramovich-Temkin-Wlodarzcyk and McQuillan in 2019. These two were unified by Quek in 2020 and extended to filiations by Abramovich-Belotto-Temkin-Wlodarzcyk in 2025.

\section{General principles}%
\label{sec:General principles}

We will consider some general principles which can be seen by study of all of these methods.
\begin{enumerate}
    \item The context (notion of space, smoothness, and basic modifications) determines the method. In fact, it seems that the algorithm is essentially determined by the context.
    \item We should make our method as functorial as possible. This restricts us, but it restricts us in the correct direction.
    \item Our rough idea is to locally embed our singular $Z$ into a smooth ambient space $X$ and modify $X$ as
    \[ X_n \to X_{n-1} \to \cdots \to X_0 =X \]
    such that $Z_n$ is smooth. Formally locally, there is only one embedding of minimal ambient dimension. We will consider blowups of smooth centers in the $X_i$. Note that if $V \subset X$ is smooth, the blowup $\Bl_V(X) = \Bl_{I_V}(X)$ replaces $V$ with the projectivization $\P(N_{V/X})$ of the normal bundle of $V$ inside $X$.
    \item What we really do is \textit{principalize} the ideal $I_Z \subset \mc{O}_X$. What this means is that we find blowups
    \[ X_{n+1} \to X_n \to \cdots \to X_0 = X \]
    such that $X_{i+1} = \Bl_{V_i} X_i$ is the blowup in a smooth center. We also require that $\mc{O}_{X_{n+1}} = I_{n+1} \cdots I_0 = I \subset \mc{O}_X$ satisfies that $I_i \subseteq I_{V_i} \eqqcolon J_i$. In particular, we will see that 
    \[ I_{i+1} = (I_i \mc{O}_{X_{i+1}}) (J_i \mc{O}_{X_{i+1}})^{-1}. \]
\end{enumerate}

\begin{thm}[Main theorem]
    A functorial principalization algorithm exists.
\end{thm}

The following corollary follows by a miracle.
\begin{cor}
    Resolutions of singularities exist in characteristic zero.
\end{cor}

In fact, $Z_n = V(I_n) = V_n$ is smooth, so we have resolved the singularities of $Z$.

\begin{rmk}
    There are two main problems in birational geometry. One is resolution of singularities,\footnote{The other is the minimal model program.} which says that every birational class of varieties has a cofinal class of smooth varieties. We in fact have
    \[ I \mc{O}_{X_{n+1}} = \prod_{i=0}^n J_i \mc{O}_{X_{n+1}}. \]
    In addition, blowups in smooth centers are cofinal among all modifications.
\end{rmk}

\begin{exm}
    Consider the Whitney umbrella 
    \[ Z = V(x^2 + y^2 z) \subset \A^3 = \Spec k[x,y,z] = X, \] 
    which is shown in~\Cref{fig:whitney}.
    Let $C = V(x,y)$ be the singular locus. Taking the blowup $X' = \Bl_0 X$ at the origin, the $z$ chart has coordinates $x' = \frac{x}{z}$, $y' = \frac{y}{z}$, and $z' = z$. The strict transform has equation
    \[ f = (x'z)^2 + (y'z)^2 z = z^2 ((x')^2 + (y')^2 z), \]
    so the singularity does not improve. However, if we blow up $C$, we can resolve the singularity in one step. Therefore, Hironaka's algorithm requires some notion of memory. There is no algorithm which only modifies the origin and makes $X'$ normal crossings. 
    \begin{figure}[htpb]
        \centering
        \includegraphics[width=0.5\textwidth]{WhitneyUmb1.png}
        \caption{Whitney Umbrella}
        \label{fig:whitney}
    \end{figure}
    % TODO insert picture of Whitney umbrella
\end{exm}

\section{Classical method}%
\label{sec:Classical method}

We will choose smooth centers $V$ but set $Y = I_V^d$. The primary invariant of this method is
\[ \on{ord}_X I = \min \ab\{ d \mid D_X^{\leq d} (I) = \mc{O}_X \}, \]
where $D_X$ is the sheaf of derivations on $X$. We also need to consider the \textit{coefficient ideal} 
\[ C(I) = \sum_{i=0}^{d-1} (D_X^{\leq i}(I))^{\frac{d!}{d-i}}. \]
We also need the notional of \textit{maximal contact} at a point $x$ such that $\on{ord}_x I = d$, which is any $t \in D_X^{d-1}(I)$ of order $1$. Then $H = V(t)$ is the maximal contact hypersurface which is the best smooth approximation to our ideal. Lastly, the miracle is that the order reduction of $(I, d)$ is equivalent to the order reduction of $(C(I)|_H, d!)$ on $H$.

\begin{exm}
    If $I = (f)$, and $f = t^d + a_2 t^{d-2} + \cdots + a_d$ is a monic polynomial, then $H = V(t)$. Then $\on{ord}_x I = d$ if and only if $\on{ord}_x a_i \geq i$ for all $i$, so this is equivalent to $\on{ord}_x C(I)|_H \geq d!$.
\end{exm}

One complication is that the inequality $\on{ord}_x C(I)|_H \geq d!$ may be strict. For example, the equation $x^2 + y^3 + z^5$ has order $2$, but if we remove $x$ the order increases. To fix this, Hironaka required us to remember the exceptional divisors, which are simple normal crossings. We will not discuss how he does this, but it makes the algorithm more complicated.

The final invariant of Hironaka is $(e_1 = d_1, e_2, \ldots, e_n)$, where the center is $J = (t_1, \ldots, t_n)^d$ where $d_1$ is the order, $t_1$ is the maximal contact, $e_2$ is the order of $C(I)|_{H_1}$, $t_2$ is the maximal contact, and so on.

\section{Dream algorithm}%
\label{sec:Dream algorithm}

We begin with the observation that if we work with Deligne-Mumford stacks, can define a smooth $[Bl_J(X)] \to X$ with center $J = (t_1^{d_1}, \ldots, t_n^{d_n})$. This is known as a \textit{weighted blowup}.

\begin{exm}
    The blowup $\Bl_{(x, y^2)} \A^2$ is the quotient stack $\A^2 / \mu_2$, which is a cone (remembering the stabilizer at the origin). Of course the coarse moduli space is singular, but the stack is smooth.
\end{exm}

\begin{thm}\leavevmode
    \begin{enumerate}
        \item For any nonzero $I \neq 0$ on a smooth $X$, there exists a unique $J = (t_1^{d_1}, \ldots, t_n^{d_n}) \supseteq I$ such that $\ul{d} = (d_1, \ldots, d_n)$ is maximal in the lexicographic order.
        \item If we define $X' = [\Bl_J X]$ and $I' = (I \mc{O}_{X'})(J \mc{O}_{X'})^{-1}$, then the order improves in the lexigraphic order.
    \end{enumerate}
\end{thm}

\begin{exm}
    For the Whitney umbrella, we consider the weighted blowup which gives $w^2 = z$, $y' = y/w^2$, and $z' = x/w^3$. Then the strict transform is
    \[ f = w^6 ((x')^2 + w^4 (y')^2 w^2) = w^6 ((x')^2 + (y')^2), \]
    which is normal crossings. The order begins as $(2,3,3)$, which is greater than the new order $(2,2)$ in the lexicographic order. This weighted blowup replaces $C$ with $\tilde{C} / \mu_2$, so the ramified cover $\tilde{C} \to C$ becomes unramified with the stacky version of $C$.
\end{exm}


\end{document}
